\RequirePackage{plautopatch}
\documentclass[uplatex,dvipdfmx,a4paper,11pt]{jlreq}
\usepackage[margin=20truemm]{geometry}
\usepackage{bxpapersize}
\usepackage[utf8]{inputenc}
\usepackage{fontenc}
\usepackage{lmodern}
\usepackage{otf}
\usepackage{amsmath}
\usepackage{amssymb}
\usepackage{amsthm}
\usepackage{ascmac}
\usepackage{tcolorbox}
\tcbuselibrary{breakable, skins, theorems}
\usepackage[framemethod=TikZ]{mdframed}
% \usepackage[hyphens]{url}
\usepackage{physics2}
\usephysicsmodule{ab, ab.braket, doubleprod, diagmat, xmat}
\usepackage{diffcoeff}
\usepackage{verbatimbox}
\usepackage{bm}
\usepackage{url}
% \usepackage[dvipdfmx,hiresbb,final]{graphicx}
\usepackage{hyperref}
\usepackage{pxjahyper}
\usepackage{tikz}\usetikzlibrary{cd}
\usepackage{listings}
\usepackage{color}
\usepackage{mathtools}
\usepackage{xspace}
\usepackage{xy}
\usepackage{xypic}
%
\title{量子力学 III \\ 複数の同一粒子からなる量子系:発展編 (第二量子化)}
\author{21B00349 宇佐見大希}
\makeatletter
%
\DeclareMathOperator{\lcm}{lcm}
\DeclareMathOperator{\Kernel}{Ker}
\DeclareMathOperator{\Image}{Im}
\DeclareMathOperator{\ch}{ch}
\DeclareMathOperator{\Aut}{Aut}
\DeclareMathOperator{\Log}{Log}
\DeclareMathOperator{\Arg}{Arg}
\DeclareMathOperator{\sgn}{sgn}
\DeclareMathOperator{\Span}{span}
\DeclareMathOperator{\per}{per}
\DeclareMathOperator{\Det}{det}
%
\newcommand{\CC}{\mathbb{C}}
\newcommand{\RR}{\mathbb{R}}
\newcommand{\QQ}{\mathbb{Q}}
\newcommand{\ZZ}{\mathbb{Z}}
\newcommand{\NN}{\mathbb{N}}
\newcommand{\HH}{\mathcal{H}}
\renewcommand{\SS}{\mathfrak{S}}
\newcommand{\R}{\bm{R}}
\renewcommand{\aa}{\bm{a}}
\newcommand{\bb}{\bm{b}}
\renewcommand{\S}{\mathcal{S}}
\newcommand{\A}{\mathcal{A}}
\newcommand{\rr}{\bm{r}}
\newcommand{\kk}{\bm{k}}
\newcommand{\pp}{\bm{p}}
\renewcommand{\ss}{\bm{s}}
\newcommand{\calB}{\mathcal{B}}
\newcommand{\calF}{\mathcal{F}}
\newcommand{\ignore}[1]{}
\newcommand{\floor}[1]{\left\lfloor #1 \right\rfloor}
% \newcommand{\abs}[1]{\left\lvert #1 \right\rvert}
\newcommand{\lt}{<}
\newcommand{\gt}{>}
\newcommand{\id}{\mathrm{id}}
\newcommand{\rot}{\curl}
\newcommand{\vnabla}{\mathbf{\nabla}}
\newcommand{\laplacian}{\nabla^2}
\renewcommand{\angle}[1]{\left\langle #1 \right\rangle}
\newcommand\mqty[1]{\begin{pmatrix}#1\end{pmatrix}}
\newcommand\vmqty[1]{\begin{vmatrix}#1\end{vmatrix}}
\numberwithin{equation}{section}

\let\oldcite=\cite
\renewcommand\cite[1]{\hyperlink{#1}{\oldcite{#1}}}

\let\oldbibitem=\bibitem
\renewcommand{\bibitem}[2][]{\label{#2}\oldbibitem[#1]{#2}}

% theorem環境の設定
% - 冒頭に改行
% - 末尾にdiamond (amsthm)
\theoremstyle{definition}

\newcommand*{\newqedtheoremx}[2]{
  \newenvironment{#1}[1][]{
    \begin{#2}[##1]
      \leavevmode
      \newline
      \renewcommand{\qedsymbol}{\(\diamond\)}
      \pushQED{\qed}
  }{
      \qedhere
      \popQED
    \end{#2}
  }
}
\newtheorem{theorem*}{定理}[section]
\newqedtheoremx{theorem}{theorem*}
\newcommand*{\newqedtheorem}[2]{
  \newtheorem{#1*}[theorem*]{#2}
  \newqedtheoremx{#1}{#1*}
}
\newqedtheorem{lemma}{補題}
\newqedtheorem{corollary}{系}
\newqedtheorem{example}{例}
\newqedtheorem{proposition}{命題}
\newqedtheorem{remark}{注意}
\newqedtheorem{thesis}{主張}
\newqedtheorem{notation}{記法}
\newqedtheorem{problem}{問題}
\newqedtheorem{algorithm}{アルゴリズム}

\newcommand*{\newscreentheoremx}[2]{
  \newenvironment{#1}[1][]{
    \begin{screen}
    \begin{#2}[##1]
      \leavevmode
      \newline
  }{
    \end{#2}
    \end{screen}
  }
}
\newtheorem{sctheorem*}{定理}[section]
\newscreentheoremx{sctheorem}{sctheorem*}
\newcommand*{\newscreentheorem}[2]{
  \newtheorem{#1*}[theorem*]{#2}
  \newscreentheoremx{#1}{#1*}
}
\newscreentheorem{axiom}{公理}
\newscreentheorem{definition}{定義}
\newscreentheorem{summary}{まとめ}
% \newscreentheorem{theorem}{定義}

\renewenvironment{proof}[1][\proofname]{\par
  \normalfont
  \topsep6\p@\@plus6\p@ \trivlist
  \item[\hskip\labelsep{\bfseries #1}\@addpunct{\bfseries}]\ignorespaces\quad\par
}{
  \qed\endtrivlist\@endpefalse
}
\renewcommand\proofname{証明}

\makeatother
\begin{document}
\maketitle
\tableofcontents
\clearpage

\begin{table}[hbtp]
  \label{table:data_type}
  \centering
  \begin{tabular}{ll}
    \hline
    問題番号    & 正誤                                                                                                                    \\
    \hline \hline
    Q21-1.  & (i) o (ii) o (iii) o (iv) o (v) o (vi) o (vii) o (viii) o (ix) o (x) o (xi) x                                         \\
    Q21-2.  & o                                                                                                                     \\
    Q21-3.  & (i) o (ii) o (iii) o (iv) o (v) o (vi) o (vii) o (viii) o (ix) o (x) o (xi) o (xii) o (xiii) o (xiv) o (xv) o (xvi) o \\
    Q21-4.  & (i) o (ii) o (iii) o (iv) o (v) o (vi) o                                                                              \\
    Q21-5.  & (i) o (ii) o                                                                                                          \\
    Q21-6.  & o                                                                                                                     \\
    Q21-7.  & (i) o (ii) o (iii) o (iv) o                                                                                           \\
    Q21-8.  & (i) o (ii) o (iii) o (iv) o                                                                                           \\
    Q21-9.  & (i) o (ii) o                                                                                                          \\
    Q21-10. & △                                                                                                                     \\
    Q21-11. & o                                                                                                                     \\
    Q21-12. & (i) o (ii) o (iii) o (iv) o (v) o (vi) o (vii) o (viii) o (ix) o                                                      \\
    Q21-13. & o                                                                                                                     \\
    Q21-14. & o                                                                                                                     \\
    Q21-15. & o                                                                                                                     \\
    Q21-16. & o                                                                                                                     \\
    Q21-17. & (i) o (ii) o                                                                                                          \\
    Q21-18. & (i) o (ii) o                                                                                                          \\
    Q21-19. & (i) o (ii) o (iii) o (iv) o (v) o (vi) o                                                                              \\
    Q21-20. & (i) o (ii) o (iii) o (iv) o                                                                                           \\
    Q21-21. & o                                                                                                                     \\
    \hline
  \end{tabular}
  \caption{正誤表}
\end{table}
\clearpage

このレポートでは複数の同一粒子系におけるさまざまな表現を導入することを目的とする.

\section{もし、量子状態の対称化の要請がなかったら?}
量子に関する実験を進めていくと複数の同一粒子はどうしても区別できないことが分かってきた. これを理論へ組み込む為に物理学者は「いかなる粒子状態は粒子交換に関して不変である」という論理の飛躍を用いて説明した.
\begin{align}
             & \textrm{複数の同一粒子は区別できない.}            \\
  \iff       & \textrm{いかなる観測量の期待値は粒子交換に関して不変である.} \\
  \impliedby & \textrm{いかなる粒子状態は粒子交換に関して不変である.}
\end{align}
これを対称化の要請と呼ぶ. ここでは対称化の要請をせずに複数の同一粒子を区別できないという事実だけで導けることを考える.

\begin{definition}[複数の同一粒子系における Hilbert 空間]
  1 粒子状態の Hilbert 空間 $\HH_{single}$ に対して $N$ 個の粒子の粒子状態の Hilbert 空間はテンソル積 $\HH^{(N)}\cong\HH_{single}\otimes\cdots\otimes\HH_{single}$ で表現される.
  そして $\HH^{(N)}$ の粒子状態は $\ket|\psi_1>\cdots\ket|\psi_N> \in\HH^{(N)}$ と書き, $\ket|\psi_1>\cdots\ket|\psi_N>, \ket|\psi_1'>\cdots\ket|\psi_N'>$ の内積は次のように定義する.
  \begin{align}
    (\bra<\psi_1|\cdots\bra<\psi_N|)\cdot(\ket|\psi_1'>\cdots\ket|\psi_N'>) = \braket<\psi_1|\psi_1'>\cdots\braket<\psi_N|\psi_N'>.
  \end{align}
\end{definition}
異なる 1 粒子状態 $\ket|\alpha>,\ket|\beta>\in\HH_{single}$ を持つ粒子による 2 つの粒子系 $\HH^{(2)} \cong \HH_{single}\otimes\HH_{single}$ において次の 2 つを仮定する.

\begin{enumerate}
  \item 2 つの粒子は区別できない.
  \item 粒子の 1 個が $\ket|\alpha>\in\HH_{single}$ となり, もう 1 個は $\ket|\beta>\in\HH_{single}$ となる. (これを仮定 $D$ とおく)
\end{enumerate}
これらの条件は次のように言い換えられる.
\begin{enumerate}
  \item いかなる観測量の期待値は粒子交換に関して不変である.
  \item 任意の粒子状態 $\ket|\Psi>$ は $\ket|\alpha>\ket|\beta>, \ket|\beta>\ket|\alpha>\in\HH^{(2)}$ の重ね合わせにより表現できる.
\end{enumerate}
粒子状態については規格化条件を用いて次のように表現できる.
\begin{align}
  \ket|\Psi> & = c_1\ket|\alpha>\ket|\beta> + c_2\ket|\beta>\ket|\alpha> \qquad (c_1, c_2\in\CC, |c_1|^2 + |c_2|^2 = 1).
\end{align}
今後の為に粒子交換を表す演算子を定義しておく.
\begin{definition}[交換演算子]
  Hilbert 空間 $\HH^{(2)}$ において交換演算子 (exchange operator) $\hat{E}$ を次のように定義する.
  \begin{align}
    \hat{E}\ket|\psi>\ket|\psi'> & = \ket|\psi'>\ket|\psi>.
  \end{align}
\end{definition}


\begin{problem}[Q21-1(i)]
粒子が区別できないならば粒子状態を区別できないとは示せないが, ここでは粒子状態を区別できないと仮定する. このとき粒子状態 $\ket|\Psi>\in\HH^{(2)}$ は粒子を交換しても不変であるから位相を考慮して次の式が成り立つ.
\begin{align}
  \ket|\Psi> & \sim \hat{E}\ket|\Psi> \iff c_1\ket|\alpha>\ket|\beta> + c_2\ket|\beta>\ket|\alpha> \sim c_1\ket|\beta>\ket|\alpha> + c_2\ket|\alpha>\ket|\beta> \iff c_1 = \pm c_2.
\end{align}
よって粒子状態は次のようになる.
\begin{align}
  \ket|\Psi> & = \frac{1}{\sqrt{2}}(\ket|\alpha>\ket|\beta> \pm \ket|\beta>\ket|\alpha>).
\end{align}
これより粒子状態を区別できないならば係数に対して条件を足さなければならないことが分かる.
\label{Q21-1(i)}
\end{problem}

\begin{proposition}[Q21-1(ii)]
  粒子状態 $\ket|\Psi_S>, \ket|\Psi_A>$ を次のように定義する.
  \begin{align}
    \begin{dcases}
      \ket|\Psi_S> = \frac{1}{\sqrt{2}}(\ket|\alpha>\ket|\beta> + \ket|\beta>\ket|\alpha>) \\
      \ket|\Psi_A> = \frac{1}{\sqrt{2}}(\ket|\alpha>\ket|\beta> - \ket|\beta>\ket|\alpha>)
    \end{dcases}.
  \end{align}
  このとき $D$ を満たす任意の粒子状態 $\ket|\Psi>\in\HH^{(2)}$ は次のように表現される.
  \begin{align}
    \ket|\Psi> = c_S\ket|\Psi_S> + c_A\ket|\Psi_A>.
  \end{align}
\end{proposition}
\begin{proof}
  まず十分性について $D$ を満たす任意の粒子状態 $\ket|\Psi>\in\HH^{(2)}$ は $\ket|\alpha>\ket|\beta>, \ket|\beta>\ket|\alpha>$ の重ね合わせにより表現できる. これより
  \begin{align}
    \ket|\Psi> & = c_1\ket|\alpha>\ket|\beta> + c_2\ket|\beta>\ket|\alpha>                                                                                         \\
               & = \frac{c_1 + c_2}{2}(\ket|\alpha>\ket|\beta> + \ket|\beta>\ket|\alpha>) + \frac{c_1 - c_2}{2}(\ket|\alpha>\ket|\beta> - \ket|\beta>\ket|\alpha>) \\
               & = \frac{c_1 + c_2}{\sqrt{2}}\ket|\Psi_S> + \frac{c_1 - c_2}{\sqrt{2}}\ket|\Psi_A>.
  \end{align}
  であり, 次のようにおくことで $\ket|\Psi>$ は $\ket|\Psi_S>, \ket|\Psi_A>$ の重ね合わせとして表現できる.
  \begin{align}
    c_S & = \frac{c_1 + c_2}{\sqrt{2}}, \quad c_A = \frac{c_1 - c_2}{\sqrt{2}}.
  \end{align}
  逆に必要性について任意の係数 $c_S, c_A$ について $\ket|\alpha>\ket|\beta>, \ket|\beta>\ket|\alpha>$ の重ね合わせで表現できることは次のように分かる.
  \begin{align}
    \ket|\Psi> & = c_S\ket|\Psi_S> + c_A\ket|\Psi_A>                                                                                                                 \\
               & = \frac{c_S}{\sqrt{2}}(\ket|\alpha>\ket|\beta> + \ket|\beta>\ket|\alpha>) + \frac{c_A}{\sqrt{2}}(\ket|\alpha>\ket|\beta> - \ket|\beta>\ket|\alpha>) \\
               & = \frac{c_S + c_A}{\sqrt{2}}\ket|\alpha>\ket|\beta> + \frac{c_S - c_A}{\sqrt{2}}\ket|\beta>\ket|\alpha>.
  \end{align}
  よって同値な表現であることが示された.
\end{proof}

\begin{proposition}[Q21-1(iii)(iv)(v)]
  交換演算子について次の性質が認められる.
  \begin{align}
     & \hat{E} = \hat{E}^\dagger = \hat{E}^{-1}, \quad \hat{E}^2 = \hat{1} \\
     & \hat{E}\ket|\Psi> = c_S\ket|\Psi_S> - c_A\ket|\Psi_A>.
  \end{align}
\end{proposition}
\begin{proof}
  まず粒子状態 $\ket|\psi>\ket|\psi'> = \ket|\alpha>\ket|\beta>, \ket|\beta>\ket|\alpha>$ に対して演算子 $\hat{E}^{-1}, \hat{E}^\dagger$ を適用する.
  \begin{align}
    \hat{E}^{-1}\ket|\psi>\ket|\psi'>                                & = \hat{E}^{-1}\hat{E}\ket|\psi'>\ket|\psi> = \ket|\psi'>\ket|\psi>                     \\
    \bra<\psi|\bra<\psi'|\hat{E}^\dagger\hat{E}\ket|\psi>\ket|\psi'> & = \bra<\psi'|\braket<\psi|\psi'>\ket|\psi> = \bra<\psi|\braket<\psi'|\psi>\ket|\psi'>.
  \end{align}
  これより次のことが分かる.
  \begin{align}
    \hat{E} & = \hat{E}^\dagger = \hat{E}^{-1}, \qquad \hat{E}^2 = \hat{E}\hat{E}^{-1} = \hat{1}.
  \end{align}
  次に粒子状態 $\ket|\Psi_S>, \ket|\Psi_A>$ に適用すると
  \begin{align}
    \hat{E}\ket|\Psi_S> & = \frac{1}{\sqrt{2}}\hat{E}(\ket|\alpha>\ket|\beta> + \ket|\beta>\ket|\alpha>) = +\frac{1}{\sqrt{2}}(\ket|\alpha>\ket|\beta> + \ket|\beta>\ket|\alpha>) = +\ket|\Psi_S>  \\
    \hat{E}\ket|\Psi_A> & = \frac{1}{\sqrt{2}}\hat{E}(\ket|\alpha>\ket|\beta> - \ket|\beta>\ket|\alpha>) = -\frac{1}{\sqrt{2}}(\ket|\alpha>\ket|\beta> - \ket|\beta>\ket|\alpha>) = -\ket|\Psi_A>.
  \end{align}
  となるから任意の状態 $\ket|\Psi>$ に適用すると次のようになる.
  \begin{align}
    \hat{E}\ket|\Psi> & = \hat{E}(c_S\ket|\Psi_S> + c_A\ket|\Psi_A>) = c_S\ket|\Psi_S> - c_A\ket|\Psi_A>.
  \end{align}
\end{proof}

\begin{proposition}[Q21-1(vi)(vii)(viii)]
  Hilbert 空間 $\HH^{(2)}$ の任意の観測量 $\hat{O}$ について 2 つの粒子を区別できないことと次の 3 つはそれぞれ同値である.
  \begin{enumerate}
    \item 期待値 $\langle\hat{O}\rangle$ は粒子交換に関して不変である.
    \item 観測量 $\hat{O}$ は粒子交換に関して不変である. つまり $\hat{O} = \hat{E}\hat{O}\hat{E}$ である.
    \item 観測量 $\hat{O}$ と交換演算子 $\hat{E}$ は可換である.
  \end{enumerate}
\end{proposition}
\begin{proof}
  1 から 2 を示す. 期待値について $\ket|\Psi> \to \hat{E}\ket|\Psi>$ と状態を変更しても不変であるから次のようになる.
  \begin{align}
    \langle\hat{O}\rangle & = \bra<\Psi|\hat{O}\ket|\Psi> = \bra<\Psi|\hat{E}^\dagger\hat{O}\hat{E}\ket|\Psi> = \bra<\Psi|\hat{E}\hat{O}\hat{E}\ket|\Psi>.
  \end{align}
  これより $\hat{O} = \hat{E}\hat{O}\hat{E}$ となり, $\hat{O}$ は粒子交換に関して不変であることがわかる.
  念のため $\hat{E}\hat{O}\hat{E}$ について $(\hat{E}\hat{O}\hat{E})^\dagger = \hat{E}^\dagger\hat{O}^\dagger\hat{E}^\dagger = \hat{E}\hat{O}\hat{E}$ と計算できるから $\hat{E}\hat{O}\hat{E}$ は Hermite 演算子となり整合性は保っている.
  次に 2 から 3 を示す.
  \begin{align}
    \hat{E}\hat{O} & = \hat{E}\hat{E}\hat{O}\hat{E} = \hat{O}\hat{E}.
  \end{align}
  つまり $[\hat{O}, \hat{E}] = 0$ であるから $\hat{O}, \hat{E}$ は可換である.
  最後に 3 から 1 は $\hat{E}^\dagger\hat{O}\hat{E} = \hat{E}^\dagger\hat{E}\hat{O} = \hat{O}$ より成り立つ.
  よって全て互いに同値であることは示された.
\end{proof}

\begin{proposition}[Q21-1(ix)]
  観測量 $\hat{O}$ の期待値 $\langle\hat{O}\rangle$ について次のように書ける.
  \begin{align}
    \langle\hat{O}\rangle & = |c_S|^2\bra<\Psi_S|\hat{O}\ket|\Psi_S> + |c_A|^2\bra<\Psi_A|\hat{O}\ket|\Psi_A>.
  \end{align}
\end{proposition}
\begin{proof}
  観測量 $\hat{O}$ の期待値 $\langle\hat{O}\rangle$ は次のように計算できる.
  \begin{align}
    \langle\hat{O}\rangle & = \bra<\Psi|\hat{O}\ket|\Psi>                                                                                                                                         \\
                          & = (c_S^*\bra<\Psi_S| + c_A^*\bra<\Psi_A|)\hat{O}(c_S\ket|\Psi_S> + c_A\ket|\Psi_A>)                                                                                   \\
                          & = |c_S|^2\bra<\Psi_S|\hat{O}\ket|\Psi_S> + |c_A|^2\bra<\Psi_A|\hat{O}\ket|\Psi_A> + c_S^*c_A\bra<\Psi_S|\hat{O}\ket|\Psi_A> + c_A^*c_S\bra<\Psi_A|\hat{O}\ket|\Psi_S> \\
                          & = |c_S|^2\bra<\Psi_S|\hat{O}\ket|\Psi_S> + |c_A|^2\bra<\Psi_A|\hat{O}\ket|\Psi_A>.\label{O expected}
  \end{align}
  ただし式 \eqref{O expected} において次のような計算をした.
  \begin{align}
    \bra<\Psi_S|\hat{O}\ket|\Psi_A> & = \bra<\Psi_S|\hat{E}\hat{O}\hat{E}\ket|\Psi_A> = -\bra<\Psi_S|\hat{O}\ket|\Psi_A> = 0  \\
    \bra<\Psi_A|\hat{O}\ket|\Psi_S> & = \bra<\Psi_A|\hat{E}\hat{O}\hat{E}\ket|\Psi_S> = -\bra<\Psi_A|\hat{O}\ket|\Psi_S> = 0.
  \end{align}
\end{proof}

\begin{problem}[Q21-1(x)]
例えば $\hat{O} = 2\ket|\beta>\ket|\alpha>\bra<\alpha|\bra<\beta|$ とすると
\begin{align}
  \bra<\Psi_S|\hat{O}\ket|\Psi_S> & = \frac{1}{2}(\bra<\alpha|\bra<\beta| + \bra<\beta|\bra<\alpha|)\hat{O}(\ket|\alpha>\ket|\beta> + \ket|\beta>\ket|\alpha>) = +1  \\
  \bra<\Psi_A|\hat{O}\ket|\Psi_A> & = \frac{1}{2}(\bra<\alpha|\bra<\beta| - \bra<\beta|\bra<\alpha|)\hat{O}(\ket|\alpha>\ket|\beta> - \ket|\beta>\ket|\alpha>) = -1.
\end{align}
より $c_S, c_A$ は互いに依存しない.
\end{problem}

\begin{problem}[Q21-1(xi)]
交換演算子が Hilbert 空間の代数構造において既約元であることは直感的に成り立つので, 区別できない情報が観測量の演算子に吸収され, 粒子状態の粒子を区別できないとは示せない. 問題 \ref{Q21-1(i)} のようには係数は決まらず, 理論の予言能力に問題はない.
\end{problem}


\begin{axiom}[対称化の要請]
  いかなる粒子状態は粒子交換に関して不変である.
\end{axiom}

\section{$n$ 次対称群 $\SS_n$}
前章の 2 粒子系で交換演算子を導入したが一般の $N$ 個の粒子系において対応するものが置換演算子である. それを導入する前段階として $n$ 次対称群を整理する.
\begin{definition}[$n$ 次対称群]
  $X$ を集合とするとき $X$ から $X$ への全単射写像 $\sigma: X\to X$ を $X$ の置換という.
  $\sigma,\tau$ を置換とするとき, その積 $\sigma\tau$ を写像としての合成 $\sigma\circ\tau$ と定義する. $X$ の置換全体の集合はこの演算により群となり, これを $X$ の置換群という.
  $X = \{1,2,\ldots,n\}$ のとき $X$ の置換群を $n$ 次対称群といい $\SS_n$ と書く.
\end{definition}
繰り返すが置換の積は写像の合成であり写像は右結合である. (Q21-2(i))
\begin{problem}[Q21-2(ii)]
$X = \{0, 1, 2, 3\}$ の置換群 $G$ に対して $\sigma,\tau\in G$ の積 $\sigma\tau$ を計算せよ.
\begin{align}
  \sigma = \begin{pmatrix}
             0 & 1 & 2 & 3 \\
             3 & 2 & 0 & 1
           \end{pmatrix} \quad , \quad
  \tau = \begin{pmatrix}
           0 & 1 & 2 & 3 \\
           3 & 2 & 1 & 0
         \end{pmatrix}.
\end{align}
\end{problem}
\begin{proof}
  \begin{align}
    \sigma\tau =
    \begin{pmatrix}
      0 & 1 & 2 & 3 \\
      3 & 2 & 0 & 1
    \end{pmatrix}
    \begin{pmatrix}
      0 & 1 & 2 & 3 \\
      3 & 2 & 1 & 0
    \end{pmatrix}
    =
    \begin{pmatrix}
      0 & 1 & 2 & 3 \\
      1 & 0 & 2 & 3
    \end{pmatrix}.
  \end{align}
\end{proof}

\begin{theorem}[Q21-3(i)(ii)(iii)(iv)]
  $n$ 次対称群 $\SS_n$ は群である.
\end{theorem}
\begin{proof}
  $\sigma,\tau\in\SS_n$ に対して $\sigma\tau = \sigma\circ\tau$ が全単射写像であることを示す.
  まず $\sigma\tau$ の全射性について $\sigma$ の全射性より任意の $c\in X$ に対して $\sigma(b) = c$ となる $b\in X$ があり, $\tau(a) = b$ となる $a\in X$ がある.
  これより任意の $c$ に対して次を満たす $a$ がある.
  \begin{align}
    \sigma\tau(a) = \sigma\circ\tau(a) = \sigma(\tau(a)) = c.
  \end{align}
  また $\sigma\tau$ の単射性についてはそれぞれの単射性より次のように満たされる.
  \begin{align}
    \sigma\tau(a) = \sigma\tau(b) \implies \tau(a) = \tau(b) \implies a = b.
  \end{align}
  これより積について閉じていることが分かる.

  単位元は $X$ の恒等写像 $\id_X$ とすることで任意の $\sigma\in\SS_n$ に対して $\sigma\id_X = \id_X\sigma = \sigma$ を満たす.

  また任意の元 $\sigma\in\SS_n$ に対する逆元は逆像 $\sigma^{-1}$ とすることで $\sigma\sigma^{-1} = \id_X$ を満たす.

  そして定義から結合法則 $\sigma_1(\sigma_2\sigma_3) = (\sigma_1\sigma_2)\sigma_3$ も満たすことが分かる.

  よって $n$ 次対称群 $\SS_n$ は群となる.
\end{proof}

\begin{proposition}[Q21-4]
  $n$ 次対称群 $\SS_n$ の位数は $n!$ である.
\end{proposition}
\begin{proof}
  全単射写像は $X$ の順列で被覆できるから位数は $n!$ となる.
\end{proof}

\begin{proposition}[Q21-5(i)(ii), Q21-6(i)(ii)]
  $\sigma_0\in\SS_n$ とすると $\sigma_0\SS_n = \SS_n\sigma_0 = \SS_n^{-1} = \SS_n$ である.
\end{proposition}
\begin{proof}
  $\sigma_0$ を左から掛けることに対して $\sigma_0^{-1}$ を左から掛けることは逆写像となるから, 全単射となる. よって $\sigma_0\SS_n = \SS_n$ となる. 逆も同様なので $\SS_n\sigma_0 = \SS_n$ となる.
  また群の性質より各元の逆元は唯一であるから $\SS_n^{-1} = \SS_n$ となる.
  これより群 $R$ に対して関数 $f:\SS_n\to R$ があるとき次のようになる.
  \begin{align}
    \sum_{\sigma\in\SS_n}f(\sigma) = \sum_{\sigma\in\SS_n}f(\sigma_0\sigma) = \sum_{\sigma\in\SS_n}f(\sigma\sigma_0) = \sum_{\sigma\in\SS_n}f(\sigma^{-1})
  \end{align}
\end{proof}

\begin{definition}[互換, 巡回置換]
  置換 $\sigma\in\SS_n$ に対して $1\leq i < j\leq n$ のとき $k \neq i,j$ なら $\sigma(k) = k$ で $\sigma(i) = j$, $\sigma(j) = i$ であるとき $\sigma$ を互換といい $(i\ j)$ と書く.

  より一般に $i_1\mapsto i_2\mapsto\cdots\mapsto i_m\mapsto i_1$ と移し, 他の元は変えない置換を巡回置換といい $(i_1\ \cdots\ i_m)$ と書く.
\end{definition}

\begin{lemma}
  任意の置換は一意の巡回置換の積で表現できる.
\end{lemma}
\begin{proof}
  置換 $\sigma\in\SS_n$ においてある元 $i_1\in X$ を選び, 移していくと鳩ノ巣原理より必ず $i_1\mapsto i_2\mapsto\cdots\mapsto i_m\mapsto i_1$ と巡回する. これより巡回置換 $(i_1\ \cdots\ i_m)$ と $i_1,\ldots,i_m$ を変えず他の元を $i\mapsto\sigma(i)$ とする置換 $\sigma'$ を用いて $\sigma = (i_1\ \cdots\ i_m)\sigma'$ と表現できる.

  次に $\sigma'$ に対しては $i_1,\ldots,i_m$ ではない元を選び同様の操作を行う. これを帰納的に行うことで巡回置換の積で表せられ, 積の順番を除いて一意に定まることが分かる.
\end{proof}

\begin{theorem}[Q21-7(i)]
  任意の置換は互換の積で表現できる.
\end{theorem}
\begin{proof}
  任意の置換は巡回置換の積で表現できるから, 巡回置換が互換の積で表せられることを示せればよい.
  \begin{align}
    (i_1\ i_2\ \cdots\ i_m) & = (i_1\ i_3\ \cdots\ i_m)(i_1\ i_2)                   \\
                            & = (i_1\ i_4\ \cdots\ i_m)(i_1\ i_3)(i_1\ i_2)         \\
                            & = (i_1\ i_m)(i_1\ i_{m-1})\cdots(i_1\ i_3)(i_1\ i_2).
  \end{align}
  これは上のように変形することにより示される.
\end{proof}

\begin{definition}[符号]
  置換 $\sigma\in\SS_n$ の符号 $\sgn\sigma = (-1)^\sigma$ を次のように定義する.
  \begin{align}
    \sgn\sigma = (-1)^\sigma = \begin{cases}
                                 +1 & (\sigma が偶数個の互換の積で表される) \\
                                 -1 & (\sigma が奇数個の互換の積で表される)
                               \end{cases}.
  \end{align}
\end{definition}

\begin{proposition}[Q21-7(ii)]
  置換の符号は well-defined である.
\end{proposition}
\begin{proof}
  次のように定義される差積 $\Delta(x_1,\ldots,x_n)$ を置換 $\sigma\in\SS_n$ 用いて変数の添字を置換することを考える.
  \begin{align}
    \Delta(x_1,\ldots,x_n) = \prod_{1\leq i<j\leq n}(x_j - x_i).
  \end{align}
  互換 $\sigma = (i\ j)$ で置換するとそれぞれ次のようになるから $\Delta(x_{\sigma(1)},\ldots,x_{\sigma(n)}) = -\Delta(x_1,\ldots,x_n)$ となる.
  \begin{align}
    (x_j - x_i)            & \mapsto -(x_j - x_i)            \\
    (x_a - x_i)(x_a - x_j) & \mapsto (x_a - x_i)(x_a - x_j)  \\
    (x_i - x_a)(x_a - x_j) & \mapsto (x_i - x_a)(x_a - x_j)  \\
    (x_i - x_a)(x_j - x_a) & \mapsto (x_i - x_a)(x_j - x_a).
  \end{align}
  これより置換 $\sigma\in\SS_n$ が異なる互換の積 $\sigma = \sigma_1\cdots\sigma_k = \tau_1\cdots\tau_m$ で表されたとき
  \begin{align}
    \Delta(x_{\sigma(1)},\ldots,x_{\sigma(n)}) = (-1)^k\Delta(x_1,\ldots,x_n) = (-1)^m\Delta(x_1,\ldots,x_n).
  \end{align}
  となる為, 互換の積の個数の偶奇は一致する.
\end{proof}

\begin{proposition}[Q21-8(i)(ii)(iii)]
  置換の符号 $\sgn: \SS_n\to\ZZ^\times$ は準同型写像である.
\end{proposition}
\begin{proof}
  差積を用いることで
  \begin{align}
    \sgn(\sigma\tau)\Delta(x_1,\ldots,x_n) & = \Delta(x_{\sigma\tau(1)},\ldots,x_{\sigma\tau(n)})                                                 \\
                                           & = \sgn(\sigma)\Delta(x_{\tau(1)},\ldots,x_{\tau(n)}) = \sgn(\sigma)\sgn(\tau)\Delta(x_1,\ldots,x_n).
  \end{align}
  より準同型の性質 $\sgn(\sigma\tau) = \sgn(\sigma)\sgn(\tau)$ が成り立つ. 準同型であるから次が成り立つ.
  \begin{align}
    \sgn(\id_X)       & = \sgn(\id_X)\sgn(\id_X) = 1                                      \\
    \sgn(\sigma^{-1}) & = \sgn(\id_X)\sgn(\sigma)^{-1} = \sgn(\sigma)^{-1} = \sgn(\sigma)
  \end{align}
\end{proof}

\section{完全対称な状態と完全反対称な状態の数学的取り扱い}
\begin{definition}[置換演算子]
  $N$ 個の同一の粒子 $X_1,\ldots,X_N$ からなる全体系の Hilbert 空間 $\HH^{(N)}$ において置換 $\sigma\in\SS_N$ を用いた置換演算子 $\hat{P}(\sigma)$ を状態に対して粒子 $X_i$ を粒子 $X_{\sigma(i)}$ に置き換える演算子とする.
\end{definition}

\begin{proposition}[Q21-9, Q21-10(i)(ii)]
  粒子状態に対して置換演算子 $\hat{P}(\sigma)$ は次のように作用する.
  \begin{align}
    \hat{P}(\sigma)\ket|\psi_1>\cdots\ket|\psi_N>         & = \ket|\psi_{\sigma^{-1}(1)}>\cdots\ket|\psi_{\sigma^{-1}(N)}> \\
    \hat{P}^\dagger(\sigma)\ket|\psi_1>\cdots\ket|\psi_N> & = \ket|\psi_{\sigma(1)}>\cdots\ket|\psi_{\sigma(N)}>
  \end{align}
\end{proposition}
\begin{proof}
  置換演算子の行列表示について置換演算子を適用すると粒子 $X_i$ における状態は元々 $X_{\sigma^{-1}(i)}$ であるから次のようになる.
  \begin{align}
    \bra<\xi_1|\cdots\bra<\xi_N|\hat{P}(\sigma)\ket|\psi_1>\cdots\ket|\psi_N> & = \bra<\xi_1|\cdots\bra<\xi_N|\psi_{\sigma^{-1}(1)}\rangle\cdots|\psi_{\sigma^{-1}(N)}\rangle \\
                                                                              & = \langle\xi_{\sigma(1)}|\cdots\langle\xi_{\sigma(N)}|\psi_1\rangle\cdots|\psi_N\rangle
  \end{align}
  これは次のように波動関数表示で書けば粒子 $X_i$ の状態を粒子 $X_{\sigma(i)}$ の状態に置き換えていると解釈できる.
  \begin{align}
    \bra<\xi_1|\bra<\xi_2|\cdots\bra<\xi_N|\hat{P}(\sigma)\ket|\Psi> & = \bra<\xi_{\sigma(1)}|\bra<\xi_{\sigma(2)}|\cdots\braket<\xi_{\sigma(N)}|\Psi> \\
    (\hat{P}(\sigma)\Psi)(\xi_1,\xi_2,\ldots,\xi_N)                  & = \Psi(\xi_{\sigma(1)},\xi_{\sigma(2)},\ldots,\xi_{\sigma(N)}).
  \end{align}
  これより任意の粒子状態 $\ket|\Psi>\in\HH^{(N)}$ に置換演算子を適用すると次のようになる.
  \begin{align}
    \ket|\Psi>                        & = \sum_{i}c^{(i)}\ket|\psi_1^{(i)}>\cdots\ket|\psi_N^{(i)}>                               \\
    \hat{P}(\sigma)\ket|\Psi>         & = \sum_{i}c^{(i)}\ket|\psi_{\sigma^{-1}(1)}^{(i)}>\cdots\ket|\psi_{\sigma^{-1}(N)}^{(i)}> \\
    \hat{P}^\dagger(\sigma)\ket|\Psi> & = \sum_{i}c^{(i)}\ket|\psi_{\sigma(1)}^{(i)}>\cdots\ket|\psi_{\sigma(N)}^{(i)}>.
  \end{align}
\end{proof}

\begin{theorem}[Q21-11(i)(ii)(iii)(iv)]
  $\hat{P}(\sigma)$ は unitary な準同型演算子である.
\end{theorem}
\begin{proof}
  まず unitary 演算子であることは次のようにして成り立つ.
  \begin{align}
    \hat{P}(\sigma)^\dagger\hat{P}(\sigma)\ket|\Psi> & = \ket|\psi_{\sigma\sigma^{-1}(1)}>\cdots\ket|\psi_{\sigma\sigma^{-1}(N)}> = \ket|\Psi>  \\
    \hat{P}(\sigma)\hat{P}(\sigma)^\dagger\ket|\Psi> & = \ket|\psi_{\sigma^{-1}\sigma(1)}>\cdots\ket|\psi_{\sigma^{-1}\sigma(N)}> = \ket|\Psi>.
  \end{align}
  そして準同型であることは次のようにして成り立つ.
  \begin{align}
    \hat{P}(\sigma\tau)\ket|\Psi> & = \ket|\psi_{(\sigma\tau)^{-1}(1)}>\cdots\ket|\psi_{(\sigma\tau)^{-1}(N)}> \\
                                  & = \hat{P}(\sigma)\ket|\psi_{\tau^{-1}(1)}>\cdots\ket|\psi_{\tau^{-1}(N)}>  \\
                                  & = \hat{P}(\sigma)\hat{P}(\tau)\ket|\Psi>.
  \end{align}
  よって $\hat{P}(\sigma)$ は unitary な準同型である. 準同型の性質より
  \begin{align}
    \hat{P}(\id_X)       & = \hat{1}              \\
    \hat{P}(\sigma^{-1}) & = \hat{P}(\sigma)^{-1}
  \end{align}
  となる.
\end{proof}

\begin{definition}[完全対称, 完全反対称]
  Hilbert 空間の状態 $\ket|\Psi>\in\HH^{(N)}$ において任意の置換 $\sigma\in\SS_N$ に対して $\hat{P}(\sigma)\ket|\Psi> = \ket|\Psi>$ となるとき完全対称, $\hat{P}(\sigma)\ket|\Psi> = \sgn(\sigma)\ket|\Psi>$ となるとき完全反対称であると定義する.
  そして完全対称, 完全反対称な状態のなす Hilbert 空間を $\HH_S^{(N)}, \HH_A^{(N)}$ と書き, 全 Hilbert 空間 $\HH^{(N)}$ から $\HH_S^{(N)}, \HH_A^{(N)}$ への射影演算子を $\hat{\S}^{(N)}, \hat{\A}^{(N)}$ とする.
\end{definition}

\begin{lemma}[Q21-12(i)(ii)]
  任意の互換 $\sigma\in\SS_N$ に対して $\hat{P}(\sigma)\ket|\Psi> = \ket|\Psi>$, $\hat{P}(\sigma)\ket|\Psi> = -\ket|\Psi>$ となることは完全対称, 完全反対称であることと同値である.
\end{lemma}
\begin{proof}
  任意の置換 $\sigma\in\SS_N$ は互換の積で表現できるから互換 $\sigma_1,\ldots,\sigma_m\in\SS_N$ を用いて $\sigma = \sigma_1\cdots\sigma_m$ と書け, 次のようになる.
  \begin{align}
    \hat{P}(\sigma)\ket|\Psi> & = \ket|\Psi> = (+1)^m\ket|\Psi>             & (完全対称)  \\
    \hat{P}(\sigma)\ket|\Psi> & = \sgn(\sigma)\ket|\Psi> = (-1)^m\ket|\Psi> & (完全反対称)
  \end{align}
  これより同値であることがわかる.
\end{proof}

\begin{proposition}[Q21-13(i)(ii)]
  $\HH_S^{(N)}$ と $\HH_A^{(N)}$ は直交し, その直和について次のようになる.
  \begin{align}
    \begin{cases}
      \HH_S^{(2)}\oplus\HH_A^{(2)} = \HH^{(2)}                      \\
      \HH_S^{(N)}\oplus\HH_A^{(N)} \subsetneq \HH^{(N)} & (N\geq 3)
    \end{cases}
  \end{align}
\end{proposition}
\begin{proof}
  $\ket|\Psi_S>\in\HH_S^{(N)}$, $\ket|\Psi_A>\in\HH_A^{(N)}$ の内積について互換 $\sigma$ の演算子を挿入することで求まる.
  \begin{align}
    \braket<\Psi_S|\Psi_A> & = \bra<\Psi_S|\hat{P}(\sigma)^\dagger\hat{P}(\sigma)\ket|\Psi_A> \\
                           & = -\braket<\Psi_S|\Psi_A> = 0.
  \end{align}
  これより $\HH_S^{(N)}$ と $\HH_A^{(N)}$ は直交する. 次に $N = 2$ における $\HH_S^{(N)}$, $\HH_A^{(N)}$ は次のように表現できる.
  \begin{align}
    \sum_i c^{(i)}\ab(\ket|\psi_1^{(i)}>\ket|\psi_2^{(i)}> + \ket|\psi_2^{(i)}>\ket|\psi_1^{(i)}>)\in\HH_S^{(2)} \\
    \sum_i c^{(i)}\ab(\ket|\psi_1^{(i)}>\ket|\psi_2^{(i)}> - \ket|\psi_2^{(i)}>\ket|\psi_1^{(i)}>)\in\HH_A^{(2)}.
  \end{align}
  これよりこれらの直和は全空間 $\HH^{(2)}$ を表現できる. $N = 3$ における $\HH_S^{(N)}$, $\HH_A^{(N)}$ の元は例えば次のようになる.
  \begin{align}
    \ket|\psi_1>\ket|\psi_2>\ket|\psi_3> + \ket|\psi_2>\ket|\psi_3>\ket|\psi_1> + \ket|\psi_3>\ket|\psi_1>\ket|\psi_2> + \ket|\psi_1>\ket|\psi_3>\ket|\psi_2> + \ket|\psi_2>\ket|\psi_1>\ket|\psi_3> + \ket|\psi_3>\ket|\psi_2>\ket|\psi_1>\in\HH_S^{(N)} \\
    \ket|\psi_1>\ket|\psi_2>\ket|\psi_3> + \ket|\psi_2>\ket|\psi_3>\ket|\psi_1> + \ket|\psi_3>\ket|\psi_1>\ket|\psi_2> - \ket|\psi_1>\ket|\psi_3>\ket|\psi_2> - \ket|\psi_2>\ket|\psi_1>\ket|\psi_3> - \ket|\psi_3>\ket|\psi_2>\ket|\psi_1>\in\HH_A^{(N)}.
  \end{align}
  これよりこれらの直和でも全空間は表現できない. $N > 3$ も同様である.
\end{proof}

\begin{theorem}[Q21-14(i)(ii)(iii)]
  射影演算子 $\hat{\S}^{(N)}, \hat{\A}^{(N)}$ は次のように表現される.
  \begin{align}
    \hat{\S}^{(N)} & = \frac{1}{N!}\sum_{\sigma\in\SS_N}\hat{P}(\sigma)              \\
    \hat{\A}^{(N)} & = \frac{1}{N!}\sum_{\sigma\in\SS_N}\sgn(\sigma)\hat{P}(\sigma).
  \end{align}
\end{theorem}
\begin{proof}
  演算子 $\hat{\S}^{(N)}, \hat{\A}^{(N)}$ に対して置換演算子 $\hat{P}(\tau)$ を適用すると次のようになる.
  \begin{align}
    \hat{P}(\tau)\hat{\S}^{(N)} & = \frac{1}{N!}\sum_{\sigma\in\SS_N}\hat{P}(\tau\sigma) = \frac{1}{N!}\sum_{\sigma'\in\SS_N}\hat{P}(\sigma') = \hat{\S}^{(N)}                                               \\
    \hat{P}(\tau)\hat{\A}^{(N)} & = \frac{1}{N!}\sum_{\sigma\in\SS_N}\sgn(\sigma)\hat{P}(\tau\sigma) = \sgn(\tau)\frac{1}{N!}\sum_{\sigma'\in\SS_N}\sgn(\sigma')\hat{P}(\sigma') = \sgn(\tau)\hat{\A}^{(N)}.
  \end{align}
  これより演算子 $\hat{\S}^{(N)}: \HH^{(N)}\to\HH_S^{(N)}, \hat{\A}^{(N)}: \HH^{(N)}\to\HH_A^{(N)}$ となる.
  \begin{align}
    (\hat{\S}^{(N)})^2 & = \frac{1}{N!^2}\sum_{\sigma\in\SS_N}\sum_{\tau\in\SS_N}\hat{P}(\sigma\tau) = \frac{1}{N!}\sum_{\sigma'\in\SS_N}\hat{P}(\sigma') = \hat{\S}^{(N)}                               \\
    (\hat{\A}^{(N)})^2 & = \frac{1}{N!^2}\sum_{\sigma\in\SS_N}\sum_{\tau\in\SS_N}\sgn(\sigma\tau)\hat{P}(\sigma\tau) = \frac{1}{N!}\sum_{\sigma'\in\SS_N}\sgn(\sigma')\hat{P}(\sigma') = \hat{\A}^{(N)}.
  \end{align}
  これより $\hat{\S}^{(N)}, \hat{\A}^{(N)}$ で何度射影しても同じ結果となる.
\end{proof}

\begin{proposition}[Q21-14(iii)(iv)(v)]
  射影演算子は Hermite 演算子であり, 積と和について次のようになる.
  \begin{align}
     & \quad \hat{\S}^{(N)}\hat{\A}^{(N)} = \hat{\A}^{(N)}\hat{\S}^{(N)} = 0     \\
     & \begin{cases}
         \hat{\S}^{(2)} + \hat{\A}^{(2)} = \hat{1}_{\HH^{(2)}} \\
         \hat{\S}^{(N)} + \hat{\A}^{(N)} \neq \hat{1}_{\HH^{(N)}} \qquad (N \geq 3)
       \end{cases}.
  \end{align}
\end{proposition}
\begin{proof}
  次に置換演算子の unitary 性より Hermite 演算子となる.
  \begin{align}
    (\hat{\S}^{(N)})^\dagger & = \frac{1}{N!}\sum_{\sigma\in\SS_N}\hat{P}(\sigma)^\dagger = \frac{1}{N!}\sum_{\sigma\in\SS_N}\hat{P}(\sigma^{-1}) = \hat{\S}^{(N)}                          \\
    (\hat{\A}^{(N)})^\dagger & = \frac{1}{N!}\sum_{\sigma\in\SS_N}\sgn(\sigma)\hat{P}(\sigma)^\dagger = \frac{1}{N!}\sum_{\sigma\in\SS_N}\sgn(\sigma)\hat{P}(\sigma^{-1}) = \hat{\A}^{(N)}.
  \end{align}
  演算子 $\hat{\S}^{(N)}, \hat{\A}^{(N)}$ の積について
  \begin{align}
    \hat{\S}^{(N)}\hat{\A}^{(N)} = \hat{\A}^{(N)}\hat{\S}^{(N)} & = \frac{1}{N!^2}\sum_{\sigma\in\SS_N}\sum_{\tau\in\SS_N}\sgn(\tau)\hat{P}(\sigma\tau)                               \\
                                                                & = \frac{1}{N!}\sum_{\sigma\in\SS_N}\sgn(\sigma)\ab(\frac{1}{N!}\sum_{\sigma'\in\SS_N}\sgn(\sigma')\hat{P}(\sigma')) \\
                                                                & = 0.
  \end{align}
  より直交することが分かる. また演算子 $\hat{\S}^{(N)}, \hat{\A}^{(N)}$ の和について
  \begin{align}
    \hat{\S}^{(2)} + \hat{\A}^{(2)} & = \frac{1}{2!}\sum_{\sigma\in\SS_2}\ab(\hat{P}(\sigma) + \sgn(\sigma)\hat{P}(\sigma)) = 1_{\HH^{(2)}}                             \\
    \hat{\S}^{(N)} + \hat{\A}^{(N)} & = \frac{1}{N!}\sum_{\sigma\in\SS_N}\ab(\hat{P}(\sigma) + \sgn(\sigma)\hat{P}(\sigma)) \neq \hat{1}_{\HH^{(N)}} \qquad (N \geq 3).
  \end{align}
  とわかる.
\end{proof}

\begin{theorem}[Q21-15(i)(ii)]
  $1\leq\mu<\nu\leq N$ において $\ket|\psi_\mu>$ と $\ket|\psi_\nu>$ が線形従属であるならば $\hat{\A}^{(N)}\ket|\psi_1>\cdots\ket|\psi_N> = 0$ となる.
\end{theorem}
\begin{proof}
  任意の $\sigma\in\SS_n$ に対して $\tau(\mu) = \sigma(\nu)$, $\tau(\nu) = \sigma(\mu)$ であり, その他の元 $1\leq i\leq N$ で $\tau(i) = \sigma(i)$ となる $\tau$ が一意に取れる. $\tau$ は $\sigma$ に対して符号が反転し, $\hat{P}(\sigma)\ket|\Psi> = \hat{P}(\tau)\ket|\Psi>$ となる. よって $\hat{\A}^{(N)}\ket|\psi_1>\cdots\ket|\psi_N> = 0$ となる.
\end{proof}

\begin{lemma}
  Hilbert 空間に演算子 $\hat{\S}^{(N)}, \hat{\A}^{(N)}$ を作用させるとそれぞれの部分空間となる.
  \begin{align}
    \HH_S^{(N)} & = \hat{\S}^{(N)}\HH^{(N)} \\
    \HH_A^{(N)} & = \hat{\A}^{(N)}\HH^{(N)}
  \end{align}
  \label{hilbert corespondence}
\end{lemma}
\begin{proof}
  $\hat{\S}^{(N)}, \hat{\A}^{(N)}$ は $\HH_S^{(N)}, \HH_A^{(N)}$ への射影演算子であるから $\HH_S^{(N)} \supseteq \hat{\S}^{(N)}\HH^{(N)}, \HH_A^{(N)} \supseteq \hat{\A}^{(N)}\HH^{(N)}$ は成り立つ.
  また $\ket|\Psi_S>\in\HH_S^{(N)}, \ket|\Psi_A>\in\HH_A^{(N)}$ について次が成り立つことが分かる.
  \begin{align}
    \ket|\Psi_S> & = \hat{P}(\sigma)\ket|\Psi_S> = \frac{1}{N!}\sum_{\sigma\in\SS_n}\hat{P}(\sigma)\ket|\Psi_S> = \hat{\S}^{(N)}\ket|\Psi_S>                         \\
    \ket|\Psi_A> & = \sgn(\sigma)\hat{P}(\sigma)\ket|\Psi_A> = \frac{1}{N!}\sum_{\sigma\in\SS_n}\sgn(\sigma)\hat{P}(\sigma)\ket|\Psi_A> = \hat{\A}^{(N)}\ket|\Psi_A>
  \end{align}
  これより $\HH_S^{(N)} \subseteq \hat{\S}^{(N)}\HH^{(N)}, \HH_A^{(N)} \subseteq \hat{\A}^{(N)}\HH^{(N)}$ は成り立つ. よってそれぞれ等しいことが分かる.
\end{proof}

\begin{proposition}[Q21-16(i)(ii), Q21-17(i)(ii), Q21-18(i)(ii)]
  $\HH_{single}$ の完全正規直交系を添字集合 $I$ を用いて $\{\ket|\phi_i>\}_{i\in I}$ とする.
  \begin{align}
    \HH_S^{(N)} & = \Span\ab\{\hat{\S}^{(N)}\ket|\phi_{i_1}>\cdots\ket|\phi_{i_N}> \mid (i_1,\ldots,i_N)\in I_S^{(N)} \} \\
    \HH_A^{(N)} & = \Span\ab\{\hat{\A}^{(N)}\ket|\phi_{i_1}>\cdots\ket|\phi_{i_N}> \mid (i_1,\ldots,i_N)\in I_A^{(N)} \}
  \end{align}
  ただし添字集合 $I_S^{(N)}, I_A^{(N)}$ は次のように定義される.
  \begin{align}
    I_S^{(N)} & = \{(i_1,\ldots,i_N)\mid i_1,\ldots,i_N\in I\land i_1\leq\cdots\leq i_{N}\} \\
    I_A^{(N)} & = \{(i_1,\ldots,i_N)\mid i_1,\ldots,i_N\in I\land i_1<\cdots<i_{N}\}
  \end{align}
\end{proposition}
\begin{proof}
  完全対称化演算子は置換に対して不変であり, 準同型である為に次のように変形できる.
  \begin{align}
    \hat{\S}^{(N)}\HH^{(N)} & = \hat{\S}^{(N)}\Span\ab\{\ket|\psi_1>\cdots\ket|\psi_N> \mid \ket|\psi_1>\cdots\ket|\psi_N>\in\HH^{(N)} \} \\
                            & = \Span\ab\{\hat{\S}^{(N)}\ket|\psi_1>\cdots\ket|\psi_N> \mid \ket|\psi_1>\cdots\ket|\psi_N>\in\HH^{(N)} \} \\
                            & = \hat{\S}^{(N)}\Span\ab\{\ket|\phi_{i_1}>\cdots\ket|\phi_{i_N}> \mid i_1,\ldots,i_N\in I\}                 \\
                            & = \Span\ab\{\hat{\S}^{(N)}\ket|\phi_{i_1}>\cdots\ket|\phi_{i_N}> \mid i_1,\ldots,i_N\in I\}                 \\
                            & = \Span\ab\{\hat{\S}^{(N)}\ket|\phi_{i_1}>\cdots\ket|\phi_{i_N}> \mid (i_1,\ldots,i_N)\in I_S^{(N)} \}
  \end{align}
  同様に完全反対称についても同じ 1 粒子状態があると 0 となるから次のように変形できる.
  \begin{align}
    \hat{\A}^{(N)}\HH^{(N)} & = \hat{\A}^{(N)}\Span\ab\{\ket|\psi_1>\cdots\ket|\psi_N> \mid \ket|\psi_1>\cdots\ket|\psi_N>\in\HH^{(N)} \} \\
                            & = \Span\ab\{\hat{\A}^{(N)}\ket|\psi_1>\cdots\ket|\psi_N> \mid \ket|\psi_1>\cdots\ket|\psi_N>\in\HH^{(N)} \} \\
                            & = \hat{\A}^{(N)}\Span\ab\{\ket|\phi_{i_1}>\cdots\ket|\phi_{i_N}> \mid i_1,\ldots,i_N\in I \}                \\
                            & = \Span\ab\{\hat{\A}^{(N)}\ket|\phi_{i_1}>\cdots\ket|\phi_{i_N}> \mid i_1,\ldots,i_N\in I \}                \\
                            & = \Span\ab\{\hat{\A}^{(N)}\ket|\phi_{i_1}>\cdots\ket|\phi_{i_N}> \mid (i_1,\ldots,i_N)\in I_A^{(N)} \}
  \end{align}
  これらに対して補題 \ref{hilbert corespondence} を適用して示される.
\end{proof}

\begin{definition}[完全対称, 完全反対称な状態の基底とその粒子数]
  Hilbert 空間 $\HH_S^{(N)}$ の基底状態 $\hat{\S}^{(N)}\ket|\phi_{i_1}>\cdots\ket|\phi_{i_N}> \quad (i_1,\ldots,i_N)\in I_S^{(N)}$ を規格化した状態を $\ket|\phi_{i_1}\cdots\phi_{i_N}>_S$ と定義する.
  同様に Hilbert 空間 $\HH_A^{(N)}$ の基底状態 $\hat{\A}^{(N)}\ket|\phi_{i_1}>\cdots\ket|\phi_{i_N}> \quad (i_1,\ldots,i_N)\in I_A^{(N)}$ を規格化した状態を $\ket|\phi_{i_1}\cdots\phi_{i_N}>_A$ と定義する.
  またこれらの状態の粒子数 $n_i\in\ZZ_{\geq 0}$ を $i$ と等しい $i_\mu$ の個数と定義する. これは占有数ともいう.
\end{definition}
\begin{theorem}[Q21-19(i), Q21-20(i)(ii)(iii)]
  完全対称な粒子基底 $\ket|\phi_{i_1}\cdots\phi_{i_N}>_S$ は粒子数 $n_i$ を用いて次のように表現できる.
  \begin{align}
    \ket|\phi_{i_1}\cdots\phi_{i_N}>_S & = \sqrt{\frac{N!}{\prod_{i\in I}n_i!}}\hat{\S}^{(N)}\ket|\phi_{i_1}>\cdots\ket|\phi_{i_N}> = \frac{1}{\sqrt{N!\prod_{i\in I}n_i!}}\per\begin{bmatrix}\ket|\phi_{i_1}> & \cdots & \ket|\phi_{i_N}>\end{bmatrix}
  \end{align}
\end{theorem}
\begin{proof}
  まず $\hat{\S}^{(N)}\ket|\phi_{i_1}>\cdots\ket|\phi_{i_N}>$ のノルムを計算すると次のようになる.
  \begin{align}
    \ab\|\hat{\S}^{(N)}\ket|\phi_{i_1}>\cdots\ket|\phi_{i_N}>\| & = \sqrt{\bra<\phi_{i_1}|\cdots\bra<\phi_{i_N}|\hat{\S}^{(N)\dagger}\hat{\S}^{(N)}\ket|\phi_{i_1}>\cdots\ket|\phi_{i_N}>}                                                                  \\
                                                                & = \frac{1}{N!}\sqrt{\sum_{\sigma\in\SS_N}\sum_{\tau\in\SS_N}\bra<\phi_{i_1}|\cdots\bra<\phi_{i_N}|\hat{P}(\tau)^\dagger\hat{P}(\sigma)\ket|\phi_{i_1}>\cdots\ket|\phi_{i_N}>}             \\
                                                                & = \frac{1}{N!}\sqrt{\sum_{\sigma\in\SS_N}\sum_{\tau\in\SS_N}\bra<\phi_{\tau^{-1}(i_1)}|\cdots\bra<\phi_{\tau^{-1}(i_N)}|\ket|\phi_{\sigma^{-1}(i_1)}>\cdots\ket|\phi_{\sigma^{-1}(i_N)}>} \\
                                                                & = \sqrt{\frac{\prod_{i\in I}n_i!}{N!}}
  \end{align}
  これより基底状態は次のように書ける.
  \begin{align}
    \ket|\phi_{i_1}\cdots\phi_{i_N}>_S & = \sqrt{\frac{N!}{\prod_{i\in I}n_i!}}\hat{\S}^{(N)}\ket|\phi_{i_1}>\cdots\ket|\phi_{i_N}>
  \end{align}
  さらに変形を進めると次のようになる.
  \begin{align}
    \hat{\S}^{(N)}\ket|\phi_{i_1}>\cdots\ket|\phi_{i_N}> & = \frac{1}{N!}\sum_{\sigma\in\SS_N}\hat{P}(\sigma)\ket|\phi_{i_1}>\cdots\ket|\phi_{i_N}>                             \\
                                                         & = \frac{1}{N!}\sum_{\sigma\in\SS_N}\ket|\phi_{\sigma^{-1}(i_1)}>\cdots\ket|\phi_{\sigma^{-1}(i_N)}>                  \\
                                                         & = \frac{1}{N!}\sum_{\sigma\in\SS_N}\ket|\phi_{i_{\sigma(1)}}>\cdots\ket|\phi_{i_{\sigma(N)}}>                        \\
                                                         & = \frac{\prod_{i\in I}n_i!}{N!}\sum_{(i_1,\ldots,i_N)\sim(i_1',\ldots,i_N')}\ket|\phi_{i_1'}>\cdots\ket|\phi_{i_N'}> \\
                                                         & = \frac{1}{N!}\per\begin{bmatrix}
                                                                               \ket|\phi_{i_1}>^{(1)} & \cdots & \ket|\phi_{i_N}>^{(1)} \\
                                                                               \vdots                 & \ddots & \vdots                 \\
                                                                               \ket|\phi_{i_1}>^{(N)} & \cdots & \ket|\phi_{i_N}>^{(N)}
                                                                             \end{bmatrix}                                           \\
                                                         & = \frac{1}{N!}\per\begin{bmatrix}\ket|\phi_{i_1}> & \cdots & \ket|\phi_{i_N}>\end{bmatrix}
  \end{align}
  よって次のようになる.
  \begin{align}
    \ket|\phi_{i_1}\cdots\phi_{i_N}>_S & = \frac{1}{\sqrt{N!\prod_{i\in I}n_i!}}\per\begin{bmatrix}\ket|\phi_{i_1}> & \cdots & \ket|\phi_{i_N}>\end{bmatrix}.
  \end{align}
\end{proof}

\begin{proposition}[Q21-20(iv)(v)(vi)]
  粒子状態 $\ket|\phi_{i_1}\cdots\phi_{i_N}>_S$ は $\HH_S^{(N)}$ の完全正規直交系となる.
\end{proposition}
\begin{proof}
  粒子状態 $\ket|\phi_{i_1}\cdots\phi_{i_N}>_S\in\HH_S^{(N)}$ は次のように展開できる. これを用いて計算する.
  \begin{align}
    \ket|\phi_{i_1}\cdots\phi_{i_N}>_S & = \sqrt{\frac{N!}{\prod_{i\in I}n_i!}}\hat{\S}^{(N)}\ket|\phi_{i_1}>\cdots\ket|\phi_{i_N}> = \frac{1}{\sqrt{N!\prod_{i\in I}n_i!}}\sum_{\sigma\in\SS_N}\hat{P}(\sigma)\ket|\phi_{i_1}>\cdots\ket|\phi_{i_N}>
  \end{align}
  まず正規直交関係については次のように計算できる.
  \begin{align}
    \langle\phi_{i_1}\cdots\phi_{i_N}|\phi_{i_1'}\cdots\phi_{i_N'}\rangle_S & = \frac{N!^{-1}}{\sqrt{\prod_{i\in I}n_i!n_i'!}}\sum_{\sigma\in\SS_N}\sum_{\tau\in\SS_N}\bra<\phi_{i_1}|\cdots\bra<\phi_{i_N}|\hat{P}(\sigma^{-1}\tau)|\phi_{i_1'}\rangle\cdots|\phi_{i_N'}\rangle \\
                                                                            & = \frac{1}{\sqrt{\prod_{i\in I}n_i!n_i'!}}\sum_{\sigma\in\SS_N}\bra<\phi_{i_1}|\cdots\bra<\phi_{i_N}|\hat{P}(\sigma)|\phi_{i_1'}\rangle\cdots|\phi_{i_N'}\rangle                                   \\
                                                                            & = \delta_{i_1i_1'}\cdots\delta_{i_Ni_N'}
  \end{align}
  次に完全性については係数を取り除いて次のように計算できる.
  \begin{align}
    \HH_S^{(N)} & = \Span\ab\{\hat{\S}^{(N)}\ket|\phi_{i_1}>\cdots\ket|\phi_{i_N}> \mid (i_1,\ldots,i_N)\in I_S^{(N)} \} \\
                & = \Span\ab\{\ket|\phi_{i_1'}\cdots\phi_{i_N'}>_S \mid (i_1,\ldots,i_N)\in I_S^{(N)} \}
  \end{align}
  そして完備性については次のように計算できる.
  \begin{align}
     & \sum_{(i_1,\ldots,i_N)\in I_S^{(N)}}\ket|\phi_{i_1}\cdots\phi_{i_N}>_S\bra<\phi_{i_1}\cdots\phi_{i_N}|_S                                                                                                                     \\
     & = \sum_{(i_1,\ldots,i_N)\in I_S^{(N)}}\frac{1}{N!\prod_{i\in I}n_i!}\sum_{\sigma\in\SS_N}\sum_{\tau\in\SS_N}\hat{P}(\sigma)\ket|\phi_{i_1}>\cdots\ket|\phi_{i_N}>\bra<\phi_{i_1}|\cdots\bra<\phi_{i_N}|\hat{P}^\dagger(\tau) \\
     & = \sum_{(i_1,\ldots,i_N)\in I_S^{(N)}}\frac{1}{\prod_{i\in I}n_i!}\sum_{\sigma\in\SS_N}\hat{P}(\sigma)\ket|\phi_{i_1}>\cdots\ket|\phi_{i_N}>\bra<\phi_{i_1}|\cdots\bra<\phi_{i_N}|                                           \\
     & = \sum_{(i_1,\ldots,i_N)\in I_S^{(N)}}\ket|\phi_{i_1}>\cdots\ket|\phi_{i_N}>\bra<\phi_{i_1}|\cdots\bra<\phi_{i_N}|                                                                                                           \\
     & = \hat{1}_{\HH_S^{(N)}}.
  \end{align}
  よって $\ket|\phi_{i_1}\cdots\phi_{i_N}>_S$ は完全正規直交系となる.
\end{proof}

\begin{theorem}[Q21-19(ii), Q21-21(i)(ii)]
  完全反対称な粒子基底 $\ket|\phi_{i_1}\cdots\phi_{i_N}>_A$ は粒子数 $n_i$ を用いて次のように表現できる.
  \begin{align}
    \ket|\phi_{i_1}\cdots\phi_{i_N}>_A & = \sqrt{N!}\hat{\A}^{(N)}\ket|\phi_{i_1}>\cdots\ket|\phi_{i_N}> = \frac{1}{\sqrt{N!}}\Det\begin{bmatrix}\ket|\phi_{i_1}> & \cdots & \ket|\phi_{i_N}>\end{bmatrix}
  \end{align}
\end{theorem}
\begin{proof}
  まず $\hat{\A}^{(N)}\ket|\phi_{i_1}>\cdots\ket|\phi_{i_N}>$ のノルムを計算すると次のようになる.
  \begin{align}
    \ab\|\hat{\A}^{(N)}\ket|\phi_{i_1}>\cdots\ket|\phi_{i_N}>\| & = \sqrt{\bra<\phi_{i_1}|\cdots\bra<\phi_{i_N}|\hat{\A}^{(N)\dagger}\hat{\A}^{(N)}\ket|\phi_{i_1}>\cdots\ket|\phi_{i_N}>}                                                                      \\
                                                                & = \frac{1}{N!}\sqrt{\sum_{\sigma\in\SS_N}\sum_{\tau\in\SS_N}\sgn(\tau\sigma)\bra<\phi_{i_1}|\cdots\bra<\phi_{i_N}|\hat{P}(\tau)^\dagger\hat{P}(\sigma)\ket|\phi_{i_1}>\cdots\ket|\phi_{i_N}>} \\
                                                                & = \frac{1}{N!}\sqrt{\sum_{\sigma\in\SS_N}\sgn(\sigma^2)}                                                                                                                                      \\
                                                                & = \frac{1}{\sqrt{N!}}
  \end{align}
  これより基底状態は次のように書ける.
  \begin{align}
    \ket|\phi_{i_1}\cdots\phi_{i_N}>_A & = \sqrt{N!}\hat{\A}^{(N)}\ket|\phi_{i_1}>\cdots\ket|\phi_{i_N}>
  \end{align}
  さらに変形を進めると次のようになる.
  \begin{align}
    \hat{\A}^{(N)}\ket|\phi_{i_1}>\cdots\ket|\phi_{i_N}> & = \frac{1}{N!}\sum_{\sigma\in\SS_N}\sgn(\sigma)\hat{P}(\sigma)\ket|\phi_{i_1}>\cdots\ket|\phi_{i_N}>            \\
                                                         & = \frac{1}{N!}\sum_{\sigma\in\SS_N}\sgn(\sigma)\ket|\phi_{\sigma^{-1}(i_1)}>\cdots\ket|\phi_{\sigma^{-1}(i_N)}> \\
                                                         & = \frac{1}{N!}\sum_{\sigma\in\SS_N}\sgn(\sigma)\ket|\phi_{i_{\sigma(1)}}>\cdots\ket|\phi_{i_{\sigma(N)}}>       \\
                                                         & = \frac{1}{N!}\Det\begin{bmatrix}
                                                                               \ket|\phi_{i_1}>^{(1)} & \cdots & \ket|\phi_{i_N}>^{(1)} \\
                                                                               \vdots                 & \ddots & \vdots                 \\
                                                                               \ket|\phi_{i_1}>^{(N)} & \cdots & \ket|\phi_{i_N}>^{(N)}
                                                                             \end{bmatrix}                                      \\
                                                         & = \frac{1}{N!}\Det\begin{bmatrix}
                                                                               \ket|\phi_{i_1}> & \cdots & \ket|\phi_{i_N}>
                                                                             \end{bmatrix}
  \end{align}
  よって次のようになる.
  \begin{align}
    \ket|\phi_{i_1}\cdots\phi_{i_N}>_A & = \frac{1}{\sqrt{N!}}\Det\begin{bmatrix}\ket|\phi_{i_1}> & \cdots & \ket|\phi_{i_N}>\end{bmatrix}
  \end{align}
\end{proof}

\begin{proposition}[Q21-21(iii)(iv)(v)]
  粒子状態 $\ket|\phi_{i_1}\cdots\phi_{i_N}>_A$ は $\HH_A^{(N)}$ の完全正規直交系となる.
\end{proposition}
\begin{proof}
  粒子状態 $\ket|\phi_{i_1}\cdots\phi_{i_N}>_A\in\HH_A^{(N)}$ は次のように展開できる. これをそれぞれに適用することで示す.
  \begin{align}
    \ket|\phi_{i_1}\cdots\phi_{i_N}>_A & = \sqrt{N!}\hat{\A}^{(N)}\ket|\phi_{i_1}>\cdots\ket|\phi_{i_N}> = \frac{1}{\sqrt{N!}}\sum_{\sigma\in\SS_N}\sgn(\sigma)\hat{P}(\sigma)\ket|\phi_{i_1}>\cdots\ket|\phi_{i_N}>
  \end{align}
  まず正規直交関係については次のように計算できる.
  \begin{align}
    \langle\phi_{i_1}\cdots\phi_{i_N}|\phi_{i_1'}\cdots\phi_{i_N'}\rangle_A & = \frac{1}{N!}\sum_{\sigma\in\SS_N}\sum_{\tau\in\SS_N}\sgn(\sigma^{-1}\tau)\bra<\phi_{i_1}|\cdots\bra<\phi_{i_N}|\hat{P}(\sigma^{-1}\tau)|\phi_{i_1'}\rangle\cdots|\phi_{i_N'}\rangle \\
                                                                            & = \sum_{\sigma\in\SS_N}\sgn(\sigma)\bra<\phi_{i_1}|\cdots\bra<\phi_{i_N}|\hat{P}(\sigma)|\phi_{i_1'}\rangle\cdots|\phi_{i_N'}\rangle                                                  \\
                                                                            & = \delta_{i_1i_1'}\cdots\delta_{i_Ni_N'}.
  \end{align}
  次に完全性については係数を取り除いて次のように計算できる.
  \begin{align}
    \HH_A^{(N)} & = \Span\ab\{\hat{\A}^{(N)}\ket|\phi_{i_1}>\cdots\ket|\phi_{i_N}> \mid (i_1,\ldots,i_N)\in I_A^{(N)} \} \\
                & = \Span\ab\{\ket|\phi_{i_1'}\cdots\phi_{i_N'}>_A \mid (i_1,\ldots,i_N)\in I_A^{(N)} \}.
  \end{align}
  最後に完備性については次のように計算できる.
  \begin{align}
     & \sum_{(i_1,\ldots,i_N)\in I_A^{(N)}}\ket|\phi_{i_1}\cdots\phi_{i_N}>_A\bra<\phi_{i_1}\cdots\phi_{i_N}|_A                                                                                                                   \\
     & = \sum_{(i_1,\ldots,i_N)\in I_A^{(N)}}\frac{1}{N!}\sum_{\sigma\in\SS_N}\sum_{\tau\in\SS_N}\sgn(\sigma\tau)\hat{P}(\sigma)\ket|\phi_{i_1}>\cdots\ket|\phi_{i_N}>\bra<\phi_{i_1}|\cdots\bra<\phi_{i_N}|\hat{P}^\dagger(\tau) \\
     & = \sum_{(i_1,\ldots,i_N)\in I_A^{(N)}}\sum_{\sigma\in\SS_N}\sgn(\sigma)\hat{P}(\sigma)\ket|\phi_{i_1}>\cdots\ket|\phi_{i_N}>\bra<\phi_{i_1}|\cdots\bra<\phi_{i_N}|                                                         \\
     & = \sum_{(i_1,\ldots,i_N)\in I_A^{(N)}}\ket|\phi_{i_1}>\cdots\ket|\phi_{i_N}>\bra<\phi_{i_1}|\cdots\bra<\phi_{i_N}|                                                                                                         \\
     & = \hat{1}_{\HH_A^{(N)}}.
  \end{align}
  よって $\ket|\phi_{i_1}\cdots\phi_{i_N}>_A$ は $\HH_A^{(N)}$ の完全正規直交系となる.
\end{proof}

\begin{proposition}[Q21-22(i)(ii)(iii)(iv), Q21-23(i)(ii)(iii)(iv)]
  完全正規直交系 $\ket|\phi_{i_1}\cdots\phi_{i_N}>_S, \ket|\phi_{i_1}\cdots\phi_{i_N}>_A$ について完全対称性, 完全反対称性, 線形性が成り立つ.
  \begin{align}
    \ket|\phi_{i_{\sigma(1)}}\cdots\phi_{i_{\sigma(N)}}>_S                                         & = \ket|\phi_{i_1}\cdots\phi_{i_N}>_S                                                                                                    \\
    \ket|\phi_{i_{\sigma(1)}}\cdots\phi_{i_{\sigma(N)}}>_A                                         & = \sgn(\sigma)\ket|\phi_{i_1}\cdots\phi_{i_N}>_A                                                                                        \\
    \ket|\phi_{i_1}\cdots a^{(0)}\phi_{i_\mu}^{(0)} + a^{(1)}\phi_{i_\mu}^{(1)}\cdots\phi_{i_N}>_S & = a^{(0)}\ket|\phi_{i_1}\cdots\phi_{i_\mu}^{(0)}\cdots\phi_{i_N}>_S + a^{(1)}\ket|\phi_{i_1}\cdots\phi_{i_\mu}^{(1)}\cdots\phi_{i_N}>_S \\
    \ket|\phi_{i_1}\cdots a^{(0)}\phi_{i_\mu}^{(0)} + a^{(1)}\phi_{i_\mu}^{(1)}\cdots\phi_{i_N}>_A & = a^{(0)}\ket|\phi_{i_1}\cdots\phi_{i_\mu}^{(0)}\cdots\phi_{i_N}>_A + a^{(1)}\ket|\phi_{i_1}\cdots\phi_{i_\mu}^{(1)}\cdots\phi_{i_N}>_A
  \end{align}
\end{proposition}
\begin{proof}
  まず基底状態について次のように展開できる.
  \begin{align}
    \ket|\phi_{i_1}\cdots\phi_{i_N}>_S & = \frac{1}{\sqrt{N!\prod_{i\in I}n_i!}}\per\begin{bmatrix}\ket|\phi_{i_1}> & \cdots & \ket|\phi_{i_N}>\end{bmatrix} \\
    \ket|\phi_{i_1}\cdots\phi_{i_N}>_A & = \frac{1}{\sqrt{N!}}\Det\begin{bmatrix}\ket|\phi_{i_1}> & \cdots & \ket|\phi_{i_N}>\end{bmatrix}
  \end{align}
  行列に関する性質より次のようになる.
  \begin{align}
    \ket|\phi_{i_{\sigma(1)}}\cdots\phi_{i_{\sigma(N)}}>_S & = \frac{1}{\sqrt{N!\prod_{i\in I}n_i!}}\per\begin{bmatrix}\ket|\phi_{i_{\sigma(1)}}> & \cdots & \ket|\phi_{i_{\sigma(N)}}>\end{bmatrix} \\
                                                           & = \frac{1}{\sqrt{N!\prod_{i\in I}n_i!}}\sum_{\tau\in\SS_N}\ket|\phi_{i_{\sigma\tau(1)}}>\cdots\ket|\phi_{i_{\sigma\tau(N)}}>                                              \\
                                                           & = \frac{1}{\sqrt{N!\prod_{i\in I}n_i!}}\sum_{\tau\in\SS_N}\ket|\phi_{i_{\tau(1)}}>\cdots\ket|\phi_{i_{\tau(N)}}>                                                          \\
                                                           & = \frac{1}{\sqrt{N!\prod_{i\in I}n_i!}}\per\begin{bmatrix}\ket|\phi_{i_1}> & \cdots & \ket|\phi_{i_N}>\end{bmatrix}                                                       \\
                                                           & = \ket|\phi_{i_1}\cdots\phi_{i_N}>_S
  \end{align}
  \begin{align}
    \ket|\phi_{i_{\sigma(1)}}\cdots\phi_{i_{\sigma(N)}}>_A & = \frac{1}{\sqrt{N!}}\Det\begin{bmatrix}\ket|\phi_{i_{\sigma(1)}}> & \cdots & \ket|\phi_{i_{\sigma(N)}}>\end{bmatrix} \\
                                                           & = \frac{1}{\sqrt{N!}}\sum_{\tau\in\SS_N}\sgn(\tau)\ket|\phi_{i_{\sigma\tau(1)}}>\cdots\ket|\phi_{i_{\sigma\tau(N)}}>                                    \\
                                                           & = \frac{1}{\sqrt{N!}}\sum_{\tau\in\SS_N}\sgn(\sigma)\sgn(\tau)\ket|\phi_{i_{\tau(1)}}>\cdots\ket|\phi_{i_{\tau(N)}}>                                    \\
                                                           & = \frac{1}{\sqrt{N!}}\sgn(\sigma)\Det\begin{bmatrix}\ket|\phi_{i_1}> & \cdots & \ket|\phi_{i_N}>\end{bmatrix}                                           \\
                                                           & = \sgn(\sigma)\ket|\phi_{i_1}\cdots\phi_{i_N}>_A
  \end{align}
  次に線形性について順当に計算する.
  \begin{align}
     & \ket|\phi_{i_1}\cdots a^{(0)}\phi_{i_\mu}^{(0)} + a^{(1)}\phi_{i_\mu}^{(1)}\cdots\phi_{i_N}>_S                                                                                                                                                                        \\
     & = \frac{1}{\sqrt{N!\prod_{i\in I}n_i!}}\per\begin{bmatrix}\ket|\phi_{i_1}> & \cdots & a^{(0)}\ket|\phi_{i_\mu}^{(0)}> + a^{(1)}\ket|\phi_{i_\mu}^{(1)}> & \cdots & \ket|\phi_{i_N}>\end{bmatrix}                                                                      \\
     & = \frac{1}{\sqrt{N!\prod_{i\in I}n_i!}}\sum_{\sigma\in\SS_N}\ket|\phi_{i_{\sigma(1)}}>\cdots\ab(a^{(0)}\ket|\phi_{i_\mu}^{(0)}> + a^{(1)}\ket|\phi_{i_\mu}^{(1)}>)\cdots\ket|\phi_{i_{\sigma(N)}}>                                                                    \\
     & = \frac{1}{\sqrt{N!\prod_{i\in I}n_i!}}\sum_{\sigma\in\SS_N}\ab(a^{(0)}\ket|\phi_{i_{\sigma(1)}}>\cdots\ket|\phi_{i_\mu}^{(0)}>\cdots\ket|\phi_{i_{\sigma(N)}}> + a^{(1)}\ket|\phi_{i_{\sigma(1)}}>\cdots\ket|\phi_{i_\mu}^{(1)}>\cdots\ket|\phi_{i_{\sigma(N)}}>)    \\
     & = \frac{1}{\sqrt{N!\prod_{i\in I}n_i!}}\ab(a^{(0)}\per\begin{bmatrix}\ket|\phi_{i_1}>\cdots\ket|\phi_{i_\mu}^{(0)}>\cdots\ket|\phi_{i_N}>\end{bmatrix} + a^{(1)}\per\begin{bmatrix}\ket|\phi_{i_1}>\cdots\ket|\phi_{i_\mu}^{(1)}>\cdots\ket|\phi_{i_N}>\end{bmatrix}) \\
     & = a^{(0)}\ket|\phi_{i_1}\cdots\phi_{i_\mu}^{(0)}\cdots\phi_{i_N}>_S + a^{(1)}\ket|\phi_{i_1}\cdots\phi_{i_\mu}^{(1)}\cdots\phi_{i_N}>_S
  \end{align}
  \begin{align}
     & \ket|\phi_{i_1}\cdots a^{(0)}\phi_{i_\mu}^{(0)} + a^{(1)}\phi_{i_\mu}^{(1)}\cdots\phi_{i_N}>_A                                                                                                                                                               \\
     & = \frac{1}{\sqrt{N!}}\Det\begin{bmatrix}\ket|\phi_{i_1}> & \cdots & a^{(0)}\ket|\phi_{i_\mu}^{(0)}> + a^{(1)}\ket|\phi_{i_\mu}^{(1)}> & \cdots & \ket|\phi_{i_N}>\end{bmatrix}                                                                               \\
     & = \frac{1}{\sqrt{N!}}\sum_{\sigma\in\SS_N}\sgn(\sigma)\ket|\phi_{i_{\sigma(1)}}>\cdots\ab(a^{(0)}\ket|\phi_{i_\mu}^{(0)}> + a^{(1)}\ket|\phi_{i_\mu}^{(1)}>)\cdots\ket|\phi_{i_{\sigma(N)}}>                                                                 \\
     & = \frac{1}{\sqrt{N!}}\sum_{\sigma\in\SS_N}\sgn(\sigma)\ab(a^{(0)}\ket|\phi_{i_{\sigma(1)}}>\cdots\ket|\phi_{i_\mu}^{(0)}>\cdots\ket|\phi_{i_{\sigma(N)}}> + a^{(1)}\ket|\phi_{i_{\sigma(1)}}>\cdots\ket|\phi_{i_\mu}^{(1)}>\cdots\ket|\phi_{i_{\sigma(N)}}>) \\
     & = \frac{1}{\sqrt{N!}}\ab(a^{(0)}\Det\begin{bmatrix}\ket|\phi_{i_1}>\cdots\ket|\phi_{i_\mu}^{(0)}>\cdots\ket|\phi_{i_N}>\end{bmatrix} + a^{(1)}\Det\begin{bmatrix}\ket|\phi_{i_1}>\cdots\ket|\phi_{i_\mu}^{(1)}>\cdots\ket|\phi_{i_N}>\end{bmatrix})          \\
     & = a^{(0)}\ket|\phi_{i_1}\cdots\phi_{i_\mu}^{(0)}\cdots\phi_{i_N}>_A + a^{(1)}\ket|\phi_{i_1}\cdots\phi_{i_\mu}^{(1)}\cdots\phi_{i_N}>_A
  \end{align}
  よって成り立つ.
\end{proof}

\section{複数の同一粒子からなる量子系の状態に対する対称化の要請}
\begin{definition}
  $N$ 個の同一の Bose 粒子による Hilbert 空間は $\HH_S^{(N)}$, また Fermi 粒子による Hilbert 空間は $\HH_A^{(N)}$ となる.
\end{definition}

\section{計算練習}
\begin{example}[Q21-25, Q21-26, Q21-27]
  互いに異なる 1 粒子状態 $\ket|\alpha>\in\HH_{single}$ を持つ Hilbert 空間において 2, 3 個の同一の Bose 粒子, Fermi 粒子の Hilbert 空間は次のようになる.
  \begin{table}
    \centering
    \begin{tabular}{|cccc|}
      \hline
      Bose, Fermi & $\HH_{single}$ の基底          & 全粒子数 $N$ & $\HH_S^{(N)}, \HH_A^{(N)}$ の基底                                                                                                                                       \\
      \hline \hline
      Bose        & $\ket|\alpha>$              & 1        & $\ket|\alpha>$                                                                                                                                                       \\
      Bose        & $\ket|\alpha>, \ket|\beta>$ & 1        & $\ket|\alpha>, \ket|\beta>$                                                                                                                                          \\
      Bose        & $\ket|\alpha>$              & 2        & $\ket|\alpha>\ket|\alpha>$                                                                                                                                           \\
      Bose        & $\ket|\alpha>, \ket|\beta>$ & 2        & $\ket|\alpha>\ket|\alpha>, \ket|\beta>\ket|\beta>, \dfrac{1}{\sqrt{2}}(\ket|\alpha>\ket|\beta> + \ket|\beta>\ket|\alpha>)$                                           \\
      Bose        & $\ket|\alpha>$              & 3        & $\ket|\alpha>\ket|\alpha>\ket|\alpha>$                                                                                                                               \\
      Bose        & $\ket|\alpha>, \ket|\beta>$ & 3        & \begin{tabular}{c}$\ket|\alpha>\ket|\alpha>\ket|\alpha>, \ket|\beta>\ket|\beta>\ket|\beta>$, \\ $\dfrac{1}{\sqrt{3}}(\ket|\alpha>\ket|\alpha>\ket|\beta> + \ket|\alpha>\ket|\beta>\ket|\alpha> + \ket|\beta>\ket|\alpha>\ket|\alpha>)$, \\ $\dfrac{1}{\sqrt{3}}(\ket|\alpha>\ket|\beta>\ket|\beta> + \ket|\beta>\ket|\alpha>\ket|\beta> + \ket|\beta>\ket|\beta>\ket|\alpha>)$\end{tabular} \\
      Fermi       & $\ket|\alpha>$              & 1        & なし                                                                                                                                                                   \\
      Fermi       & $\ket|\alpha>, \ket|\beta>$ & 1        & なし                                                                                                                                                                   \\
      Fermi       & $\ket|\alpha>$              & 2        & なし                                                                                                                                                                   \\
      Fermi       & $\ket|\alpha>, \ket|\beta>$ & 2        & $\dfrac{1}{\sqrt{2}}(\ket|\alpha>\ket|\beta> - \ket|\beta>\ket|\alpha>)$                                                                                             \\
      Fermi       & $\ket|\alpha>$              & 3        & なし                                                                                                                                                                   \\
      Fermi       & $\ket|\alpha>, \ket|\beta>$ & 3        & なし                                                                                                                                                                   \\
      \hline
    \end{tabular}
    \caption{Bose, Fermi 粒子系の基底}
  \end{table}
  互いに異なる 3 つの 1 粒子状態 $\ket|\alpha>, \ket|\beta>, \ket|\gamma>\in\HH_{single}$ を持つ場合においてそれぞれ 1 つずつある全系の状態は次のようになる.
  \begin{align}
    \frac{1}{\sqrt{6}}(\ket|\alpha>\ket|\beta>\ket|\gamma> + \ket|\gamma>\ket|\alpha>\ket|\beta> + \ket|\beta>\ket|\gamma>\ket|\alpha> + \ket|\gamma>\ket|\beta>\ket|\alpha> + \ket|\alpha>\ket|\gamma>\ket|\beta> + \ket|\beta>\ket|\alpha>\ket|\gamma>)\in\HH_S^{(3)} \\
    \frac{1}{\sqrt{6}}(\ket|\alpha>\ket|\beta>\ket|\gamma> + \ket|\gamma>\ket|\alpha>\ket|\beta> + \ket|\beta>\ket|\gamma>\ket|\alpha> - \ket|\gamma>\ket|\beta>\ket|\alpha> - \ket|\alpha>\ket|\gamma>\ket|\beta> - \ket|\beta>\ket|\alpha>\ket|\gamma>)\in\HH_A^{(3)}
  \end{align}
\end{example}

\section{Bose, Fermi 粒子系の量子状態の粒子数表示}
Bose, Fermi 粒子系の完全正規直交系は次のようにラベル付けされていた.
\begin{align}
   & \ket|\phi_{i_1}\cdots\phi_{i_N}>_S \qquad (i_1,\ldots,i_N\in I, i_1\leq\cdots\leq i_N) \\
   & \ket|\phi_{i_1}\cdots\phi_{i_N}>_A \qquad (i_1,\ldots,i_N\in I, i_1<\cdots<i_{N}).
\end{align}
これより粒子状態 $\ket|\phi_{i_1}\cdots\phi_{i_N}>_S, \ket|\phi_{i_1}\cdots\phi_{i_N}>_A$ の粒子数をそれぞれ $n_i^{(s)}, n_i^{(a)}$ とおくと次のような性質を満たす.
\begin{align}
  \begin{alignedat}{3}
    n_i^{(s)} & \in \ZZ_{\geq 0}, \qquad && \sum_{i\in I}n_i^{(s)} = N \\
    n_i^{(a)} & \in\{0, 1\}, && \sum_{i\in I}n_i^{(a)} = N.
  \end{alignedat}
\end{align}
この粒子数を用いて状態を表現することを考える.
\begin{definition}[Bose, Fermi 粒子系の量子状態の粒子数表示]
  Bose, Fermi 粒子系の粒子状態は粒子数 $n_i$ を用いて次のように表現できる.
  \begin{align}
    \ket|(n_i)_{i\in I}>_S & = \ket|n_1,n_2,\ldots,n_i,\ldots>_S = |\underbrace{\phi_1\phi_1\cdots\phi_1}_{n_1}\underbrace{\phi_2\phi_2\cdots\phi_2}_{n_2}\cdots\underbrace{\phi_i\phi_i\cdots\phi_i}_{n_i}\cdots\rangle_S \\
    \ket|(n_i)_{i\in I}>_A & = \ket|n_1,n_2,\ldots,n_i,\ldots>_A = \ket|\phi_{i_1}\cdots\phi_{i_N}>_A.
  \end{align}
  これを粒子数表示または占有数表示という.
\end{definition}
添字 $S, A$ は省略してはいけない.

\begin{proposition}
  Bose, Fermi 粒子系の粒子数表示は well-defined である.
  \label{particles well defined}
\end{proposition}
\begin{proof}
  Bose, Fermi 粒子系の次の粒子数表示があったときに一意に完全正規直交系が存在することを示す.
  \begin{align}
     & \ket|(n_i)_{i\in I}>_S \qquad \ab(n_i\in\ZZ_{\geq 0}, \sum_{i\in I}n_i = N) \\
     & \ket|(n_i)_{i\in I}>_A \qquad \ab(n_i\in\{0, 1\}, \sum_{i\in I}n_i = N)
  \end{align}
  Bose 粒子系の粒子数表示に対して完全正規直交系の表現が存在することは定義から分かり, 昇順にソートされているので一意に $\ket|\phi_{i_1}\cdots\phi_{i_N}>_S$ が定まる. Fermi 粒子系も同様.
\end{proof}

\begin{theorem}
  $N$ 個の Bose, Fermi 粒子系の状態の粒子数表示は完全正規直交系となる.
  \label{Fermi N character}
\end{theorem}
\begin{proof}
  命題 \ref{particles well defined} より Bose, Fermi 粒子系の粒子数表示と完全正規直交系が対応するから成り立つ.
  \begin{align}
     & \braket<(n_i)_{i\in I}|(n_i')_{i\in I}>_S = \prod_{i\in I}\delta_{n_in_i'}                                    \\
     & \braket<(n_i)_{i\in I}|(n_i')_{i\in I}>_A = \prod_{i\in I}\delta_{n_in_i'}                                    \\
     & \Span\ab\{\ket|(n_i)_{i\in I}>_S \mid n_i\in\ZZ_{\geq 0}, \sum_{i\in I}n_i = N \} = \HH_S^{(N)}               \\
     & \Span\ab\{\ket|(n_i)_{i\in I}>_A \mid n_i\in\{0, 1\}, \sum_{i\in I}n_i = N \} = \HH_A^{(N)}                   \\
     & \sum_{n_i\in\ZZ_{\geq 0}, \sum_i n_i = N}\ket|(n_i)_{i\in I}>_S\bra<(n_i)_{i\in I}|_S = \hat{1}_{\HH_S^{(N)}} \\
     & \sum_{n_i\in\{0, 1\}, \sum_i n_i = N} \ket|(n_i)_{i\in I}>_A\bra<(n_i)_{i\in I}|_A = \hat{1}_{\HH_A^{(N)}}
  \end{align}
\end{proof}

\begin{definition}
  また全粒子数を固定しない Bose 粒子系の Hilbert 空間を $\HH_{Bose}$ と書き, 次のように定義する.
  \begin{align}
    \HH_{Bose}  & = \bigoplus_{N=0}^\infty \HH_S^{(N)}     \\
    \HH_{Fermi} & = \bigoplus_{N = 0}^{\infty} \HH_A^{(N)}
  \end{align}
  定義から
  \begin{align}
    N \neq N' \iff \HH_S^{(N)}\perp\HH_S^{(N')} \\
    N\neq N' \iff \HH_A^{(N)}\perp\HH_A^{(N')}
  \end{align}
\end{definition}
\begin{theorem}
  一般の Bose, Fermi 粒子系について完全正規直交系となる.
\end{theorem}
\begin{proof}
  全体粒子数が異なれば異なる粒子数が存在するから正規直交関係を満たし, それぞれの全粒子数の恒等演算子を和を取ることで恒等演算子となり, それぞれの全粒子数で生成する.
  \begin{align}
     & \braket<(n_i)_{i\in I}|(n_i')_{i\in I}>_S = \prod_{i\in I}\delta_{n_in_i'}                   \\
     & \braket<(n_i)_{i\in I}|(n_i')_{i\in I}>_A = \prod_{i\in I}\delta_{n_in_i'}                   \\
     & \sum_{n_i\in\ZZ_{\geq 0}}\ket|(n_i)_{i\in I}>_S\bra<(n_i)_{i\in I}|_S = \hat{1}_{\HH_{Bose}} \\
     & \sum_{n_i\in\{0, 1\}} \ket|(n_i)_{i\in I}>_A\bra<(n_i')_{i\in I}|_A = \hat{1}_{\HH_{Fermi}}  \\
     & \HH_{Bose} = \Span\ab\{\ket|(n_i)_{i\in I}>_S \mid n_i\in\ZZ_{\geq 0} \}                     \\
     & \HH_{Fermi} = \Span\ab\{\ket|(n_i)_{i\in I}>_A \mid n_i\in\{0, 1\} \}
  \end{align}
\end{proof}

\section{Bose 粒子系の消滅演算子 $\hat{a}_i$ と生成演算子 $\hat{a}_i^\dagger$}
\begin{definition}
  Bose 粒子系の消滅演算子 $\hat{a}_i$ と生成演算子 $\hat{a}_i^\dagger$ を次のように定義する.
  \begin{align}
    \begin{dcases}
      \hat{a}_i\frac{1}{\sqrt{N!}}\per\begin{bmatrix}\ket|\phi_{i_1}> \cdots \ket|\phi_{i_N}>\end{bmatrix} = \frac{1}{\sqrt{(N-1)!}}\sum_{\substack{\mu\in X \\ i_\mu = i}}\per\begin{bmatrix}\ket|\phi_{i_1}> \cdots \ket|\phi_{i_{\mu - 1}}> & \ket|\phi_{i_{\mu + 1}}> \cdots \ket|\phi_{i_N}>\end{bmatrix} \\
      \hat{a}_i^\dagger\frac{1}{\sqrt{N!}}\per\begin{bmatrix}\ket|\phi_{i_1}> \cdots \ket|\phi_{i_N}>\end{bmatrix} = \frac{1}{\sqrt{(N+1)!}}\per\begin{bmatrix}\ket|\phi_{i}> & \ket|\phi_{i_1}> \cdots \ket|\phi_{i_N}>\end{bmatrix}
    \end{dcases}
  \end{align}
  その上で個数演算子 $\hat{n}_i = \hat{a}_i^\dagger\hat{a}$ と全粒子数演算子 $\hat{N} = \sum_{i\in I}\hat{n}_i$ と定義する.
\end{definition}
\begin{theorem}
  Bose 粒子系の消滅, 生成演算子の定義と次は同値である.
  \begin{align}
    \begin{dcases}
      \hat{a}_i\ket|\phi_{i_1}\cdots\phi_{i_N}>_S = \sqrt{n_i}\ket|\phi_{i_1}\cdots\phi_{i_{\mu - 1}}\phi_{i_{\mu + 1}}\cdots\phi_{i_N}>_S \\
      \hat{a}_i^\dagger\ket|\phi_{i_1}\cdots\phi_{i_N}>_S = \sqrt{n_i+1}\ket|\phi_{i_1}\cdots\phi_{i}\cdots\phi_{i_N}>_S
    \end{dcases}
  \end{align}
  \label{Bose creation and annihilation 1}
\end{theorem}
\begin{proof}
  Bose 粒子系の粒子数表示は次のように展開できる.
  \begin{align}
    \ket|\phi_{i_1},\ldots,\phi_{i_N}>_S = \frac{1}{\sqrt{N!\prod_{j\in I}n_j!}}\per\begin{bmatrix}\ket|\phi_{i_1}>\cdots\ket|\phi_{i_N}>\end{bmatrix}
  \end{align}
  また permutation は置換に対して不変であるので定義と次は同値である.
  \begin{align}
     & \begin{dcases}
         \hat{a}_i\frac{1}{\sqrt{N!}}\per\begin{bmatrix}\ket|\phi_{i_1}> \cdots \ket|\phi_{i_N}>\end{bmatrix} = \frac{n_i}{\sqrt{(N-1)!}}\per\begin{bmatrix}\ket|\phi_{i_1}> \cdots \ket|\phi_{i_{\mu - 1}}> & \ket|\phi_{i_{\mu + 1}}> \cdots \ket|\phi_{i_N}>\end{bmatrix} \\
         \hat{a}_i^\dagger\frac{1}{\sqrt{N!}}\per\begin{bmatrix}\ket|\phi_{i_1}> \cdots \ket|\phi_{i_N}>\end{bmatrix} = \frac{1}{\sqrt{(N+1)!}}\per\begin{bmatrix}\ket|\phi_{i_1}> \cdots \ket|\phi_{i}> \cdots \ket|\phi_{i_N}>\end{bmatrix}
       \end{dcases}                      \\
    \iff
     & \begin{dcases}
         \hat{a}_i\ket|\phi_{i_1}\cdots\phi_{i_N}>_S = \frac{n_i}{\sqrt{(N-1)!\prod_{j\in I}n_j!}}\per\begin{bmatrix}\ket|\phi_{i_1}> \cdots \ket|\phi_{i_{\mu - 1}}> & \ket|\phi_{i_{\mu + 1}}> \cdots \ket|\phi_{i_N}>\end{bmatrix} \\
         \hat{a}_i^\dagger\ket|\phi_{i_1}\cdots\phi_{i_N}>_S = \frac{1}{\sqrt{(N+1)!\prod_{j\in I}n_j!}}\per\begin{bmatrix}\ket|\phi_{i_1}> \cdots \ket|\phi_i> \cdots \ket|\phi_{i_N}>\end{bmatrix}
       \end{dcases}                                                     \\
    \iff
     & \begin{dcases}
         \hat{a}_i\ket|\phi_{i_1}\cdots\phi_{i_N}>_S = \sqrt{n_i}\ket|\phi_{i_1}\cdots\phi_{i_{\mu - 1}}\phi_{i_{\mu + 1}}\cdots\phi_{i_N}>_S \\
         \hat{a}_i^\dagger\ket|\phi_{i_1}\cdots\phi_{i_N}>_S = \sqrt{n_i+1}\ket|\phi_{i_1}\cdots\phi_{i}\cdots\phi_{i_N}>_S
       \end{dcases}
  \end{align}
  よって $\HH_S^{(N)}$ の完全正規直交系で表現できる.
\end{proof}

\begin{theorem}[Q21-35, Q21-36]
  Bose 粒子系の消滅, 生成演算子の定義と次は同値である.
  \begin{align}
    \begin{dcases}
      \hat{a}_i\ket|\ldots,n_i,\ldots> = \sqrt{n_i}\ket|\ldots,n_i-1,\ldots> \\
      \hat{a}_i^\dagger\ket|\ldots,n_i,\ldots> = \sqrt{n_i+1}\ket|\ldots,n_i+1,\ldots>
    \end{dcases}
  \end{align}
  \label{Bose creation and annihilation 2}
\end{theorem}
\begin{proof}
  定理 \ref{Bose creation and annihilation 1} を吟味することで消滅演算子によって添字 $i$ の 1 粒子状態を消滅させ, 生成演算子によって添字 $i$ の 1 粒子状態を生成していることがわかる. よって粒子数表示に直すことで定義と同値となる.
\end{proof}

\begin{proposition}[Q21-37]
  Bose 粒子系における消滅, 生成演算子の交換関係は次のようになる.
  \begin{align}
    [\hat{a}_i, \hat{a}_j^\dagger] & = \delta_{ij}, \qquad [\hat{a}_i, \hat{a}_j] = [\hat{a}_i^\dagger, \hat{a}_j^\dagger] = 0
  \end{align}
\end{proposition}
\begin{proof}
  消滅演算子 $\hat{a}_i$, 生成演算子 $\hat{a}_i^\dagger$ を状態 $\ket|\ldots,n_i,\ldots>\in\HH_{Bose}$ に適用すると
  \begin{align}
    \hat{a}_i\hat{a}_i^\dagger\ket|\ldots,n_i,\ldots> & = \sqrt{n_i + 1}\hat{a}_i\ket|\ldots,n_i+1,\ldots> = (n_i + 1)\ket|\ldots,n_i,\ldots> \\
    \hat{a}_i^\dagger\hat{a}_i\ket|\ldots,n_i,\ldots> & = \sqrt{n_i}\hat{a}_i^\dagger\ket|\ldots,n_i-1,\ldots> = n_i\ket|\ldots,n_i,\ldots>
  \end{align}
  よりそれぞれの交換関係は次のようになる.
  \begin{align}
    [\hat{a}_i, \hat{a}_i^\dagger] & = \hat{a}_i\hat{a}_i^\dagger - \hat{a}_i^\dagger\hat{a}_i = (n_i + 1) - n_i = 1 \\
    [\hat{a}_i, \hat{a}_i]         & = [\hat{a}_i^\dagger, \hat{a}_i^\dagger] = 0
  \end{align}
  異なる添字 $i, j$ についても状態 $\ket|\ldots,n_i,\ldots,n_j,\ldots>\in\HH_{Bose}$ に適用すると
  \begin{align}
    \hat{a}_i\hat{a}_j\ket|\ldots,n_i,\ldots,n_j,\ldots>                 & = \sqrt{n_in_j}\ket|\ldots,n_i-1,\ldots,n_j-1,\ldots>             \\
    \hat{a}_i\hat{a}_j^\dagger\ket|\ldots,n_i,\ldots,n_j,\ldots>         & = \sqrt{n_i(n_j + 1)}\ket|\ldots,n_i-1,\ldots,n_j+1,\ldots>       \\
    \hat{a}_j^\dagger\hat{a}_i\ket|\ldots,n_i,\ldots,n_j,\ldots>         & = \sqrt{n_i(n_j + 1)}\ket|\ldots,n_i-1,\ldots,n_j+1,\ldots>       \\
    \hat{a}_i^\dagger\hat{a}_j^\dagger\ket|\ldots,n_i,\ldots,n_j,\ldots> & = \sqrt{(n_i + 1)(n_j + 1)}\ket|\ldots,n_i+1,\ldots,n_j+1,\ldots>
  \end{align}
  よりそれぞれの交換関係は次のようになる.
  \begin{align}
    [\hat{a}_i, \hat{a}_j^\dagger] & = [\hat{a}_i, \hat{a}_j] = [\hat{a}_i^\dagger, \hat{a}_j^\dagger] = 0
  \end{align}
  よって示された.
\end{proof}

\begin{proposition}[Q21-38]
  Bose 粒子系における消滅, 生成演算子は互いに Hermite 共役である.
\end{proposition}
\begin{proof}
  次の計算により $\hat{a}_i, \hat{a}_i^\dagger$ は互いに Hermite 共役であることがわかる.
  \begin{align}
    \bra<(n_j)_{j\in I}|\hat{a}_i|(n_j')_{j\in I}\rangle         & = \begin{dcases}
                                                                       \sqrt{n_i'} & (n_i = n_i' - 1) \\
                                                                       0           & (n_i \neq n_i')
                                                                     \end{dcases}    \\
    \langle(n_j')_{j\in I}|\hat{a}_i^\dagger\ket|(n_j)_{j\in I}> & = \begin{dcases}
                                                                       \sqrt{n_i + 1} & (n_i + 1 = n_i') \\
                                                                       0              & (n_i \neq n_i')
                                                                     \end{dcases}
  \end{align}
\end{proof}

\begin{proposition}[Q21-39]
  個数演算子 $\hat{n}_i$ と全粒子数演算子 $\hat{N}$ は Hermite 演算子であり, 固有値は $\hat{n}_i = n_i, \hat{N} = N$ となる.
\end{proposition}
\begin{proof}
  個数演算子 $\hat{n}_i$ は生成消滅演算子に展開することで計算できる.
  \begin{align}
    \hat{n}_i^\dagger                & = (\hat{a}_i^\dagger\hat{a}_i)^\dagger = \hat{a}_i^\dagger\hat{a}_i = \hat{n}_i  \\
    \hat{n}_i\ket|\ldots,n_i,\ldots> & = \hat{a}_i^\dagger\hat{a}_i\ket|\ldots,n_i,\ldots> = n_i\ket|\ldots,n_i,\ldots>
  \end{align}
  全粒子数演算子 $\hat{N}$ は個数演算子に展開することで計算できる.
  \begin{align}
    \hat{N}^\dagger & = \sum_{i\in I}\hat{n}_i^\dagger = \sum_{i\in I}\hat{n}_i = \hat{N} \\
    \hat{N}         & = \sum_{i\in I}\hat{n}_i = \sum_{i\in I}n_i = N
  \end{align}
\end{proof}

\begin{theorem}[Q21-41]
  Bose 粒子系における消滅演算子 $\hat{a}_i$ と生成演算子 $\hat{a}_i^\dagger$ において次の性質は定義と同値である.
  \begin{align}
    (\hat{a}_i)^\dagger = \hat{a}_i^\dagger, \qquad [\hat{a}_i, \hat{a}_j^\dagger] = \delta_{ij}, \qquad [\hat{a}_i, \hat{a}_j] = [\hat{a}_i^\dagger, \hat{a}_j^\dagger] = 0, \qquad \hat{n}_i = \hat{a}_i^\dagger\hat{a}_i = n_i
  \end{align}
\end{theorem}
\begin{proof}
  既に定義から性質を導くことはしているので性質から定義を導く.
  \begin{align}
    \hat{n}_i\hat{a}_i         & = (\hat{a}_i^\dagger\hat{a}_i)\hat{a}_i = (\hat{a}_i\hat{a}_i^\dagger - 1)\hat{a}_i = (n_i - 1)\hat{a}_i                                     \\
    \hat{n}_i\hat{a}_i^\dagger & = \hat{a}_i^\dagger(\hat{a}_i\hat{a}_i^\dagger) = \hat{a}_i^\dagger(\hat{a}_i^\dagger\hat{a}_i + 1) = (n_i + 1)\hat{a}_i^\dagger             \\
    \hat{n}_j\hat{a}_i         & = \hat{a}_j^\dagger\hat{a}_j\hat{a}_i = \hat{a}_i\hat{a}_j^\dagger\hat{a}_j = n_j\hat{a}_i                                       & (i\neq j) \\
    \hat{n}_j\hat{a}_i^\dagger & = \hat{a}_j^\dagger\hat{a}_j\hat{a}_i^\dagger = \hat{a}_i^\dagger\hat{a}_j^\dagger\hat{a}_j = n_j\hat{a}_i^\dagger               & (i\neq j)
  \end{align}
  より $\hat{a}_i, \hat{a}_i^\dagger$ を適用すると状態の粒子数 $n_i$ が 1 だけ上下する. また $(\hat{a}_i)^\dagger = \hat{a}_i^\dagger$ より
  \begin{align}
     & \bra<\ldots,n_i-1,\ldots|\hat{a}_i\ket|\ldots,n_i,\ldots> = \bra<\ldots,n_i,\ldots|\hat{a}_i^\dagger\ket|\ldots,n_i-1,\ldots> \\
     & n_i = \bra<\ldots,n_i,\ldots|\hat{a}_i^\dagger\hat{a}_i\ket|\ldots,n_i,\ldots>
  \end{align}
  であるから次のようになる.
  \begin{align}
    \hat{a}_i\ket|\ldots,n_i,\ldots>         & = \sqrt{n_i}\ket|\ldots,n_i-1,\ldots>     \\
    \hat{a}_i^\dagger\ket|\ldots,n_i,\ldots> & = \sqrt{n_i + 1}\ket|\ldots,n_i+1,\ldots>
  \end{align}
  これらの式から定理 \ref{Bose feature} より定義を導ける.
\end{proof}

\begin{proposition}
  真空状態 $\ket|\mathrm{vac}>$ を次のように定義する.
  \begin{align}
    \ket|\mathrm{vac}> = \ket|0,0,0,\ldots>
  \end{align}
  このとき次のような性質が認められる.
  \begin{align}
    \hat{a}_i\ket|\mathrm{vac}>        & = 0                                                                             \\
    \braket<\mathrm{vac}|\mathrm{vac}> & = 1                                                                             \\
    \ket|(n_i)_{i\in I}>               & = \prod_{i\in I}\frac{(\hat{a}_i^\dagger)^{n_i}}{\sqrt{n_i!}}\ket|\mathrm{vac}>
  \end{align}
\end{proposition}
\begin{proof}
  それぞれ定義を展開することで導かれる.
  \begin{align}
    \hat{a}_i\ket|\mathrm{vac}>        & = \hat{a}_i\ket|0,0,\ldots> = 0                                                                                                                               \\
    \braket<\mathrm{vac}|\mathrm{vac}> & = \braket<0,0,\ldots|0,0,\ldots> = \prod_{i\in I}\delta_{0,0} = 1                                                                                             \\
    \ket|(n_i)_{i\in I}>               & = \prod_{i\in I}\frac{(\hat{a}_i^\dagger)^{n_i}}{\sqrt{n_i!}}\ket|0,0,\ldots> = \prod_{i\in I}\frac{(\hat{a}_i^\dagger)^{n_i}}{\sqrt{n_i!}}\ket|\mathrm{vac}>
  \end{align}
\end{proof}

\section{Fermi 粒子系の消滅演算子 $\hat{c}_i$ と生成演算子 $\hat{c}_i^\dagger$}
\begin{definition}
  \begin{align}
     & \hat{c}_i\frac{1}{\sqrt{N!}}\Det\begin{bmatrix}\ket|\phi_{i_1}> \cdots \ket|\phi_{i_N}>\end{bmatrix} = \frac{(-1)^\mu n_i}{\sqrt{(N-1)!}}\Det\begin{bmatrix}\ket|\phi_{i_1}> \cdots \ket|\phi_{i_{\mu-1}}>\ket|\phi_{i_{\mu+1}}>\cdots\ket|\phi_{i_N}>\end{bmatrix} \\
     & \hat{c}_i^\dagger\frac{1}{\sqrt{N!}}\Det\begin{bmatrix}\ket|\phi_{i_1}> \cdots \ket|\phi_{i_N}>\end{bmatrix} = \frac{1}{\sqrt{(N+1)!}}\Det\begin{bmatrix}\ket|\phi_i> & \ket|\phi_{i_1}> \cdots \ket|\phi_{i_N}>\end{bmatrix}
  \end{align}
  $\hat{n}_i = \hat{c}_i^\dagger\hat{c}_i$
  $\hat{N} = \sum_{i\in I}\hat{n}_i$
\end{definition}
\begin{theorem}
  \begin{align}
    \hat{c}_i\ket|\phi_{i_1}\cdots\phi_{i_N}>_A         & = (-1)^\mu n_i\ket|\phi_{i_1}\cdots\phi_{i_{\mu-1}}\phi_{i_{\mu+1}}\cdots\phi_{i_N}>_A \\
    \hat{c}_i^\dagger\ket|\phi_{i_1}\cdots\phi_{i_N}>_A & = (-1)^\mu\ket|\phi_{i_1}\cdots\phi_i\cdots\phi_{i_N}>_A
  \end{align}
\end{theorem}
\begin{proof}
  Fermi 粒子系の消滅, 生成演算子の定義は $\ket|\phi_{i_1}\cdots\phi_{i_N}>_A$ の粒子数 $n_i$ を用いて次のようになる.
  \begin{align}
    \hat{c}_i\frac{1}{\sqrt{N!}}\Det\begin{bmatrix}\ket|\phi_{i_1}> \cdots \ket|\phi_{i_N}>\end{bmatrix}         & = \frac{(-1)^\mu}{\sqrt{(N-1)!}}n_i\Det\begin{bmatrix}\ket|\phi_{i_1}> \cdots \ket|\phi_{i_{\mu-1}}>\ket|\phi_{i_{\mu+1}}>\cdots\ket|\phi_{i_N}>\end{bmatrix} \\
    \hat{c}_i^\dagger\frac{1}{\sqrt{N!}}\Det\begin{bmatrix}\ket|\phi_{i_1}> \cdots \ket|\phi_{i_N}>\end{bmatrix} & = \frac{1}{\sqrt{(N+1)!}}\Det\begin{bmatrix}\ket|\phi_i> & \ket|\phi_{i_1}> \cdots \ket|\phi_{i_N}>\end{bmatrix}
  \end{align}
  Fermi 粒子系の粒子数表示は次のように展開できる.
  \begin{align}
    \ket|n_1,n_2,\ldots,n_i,\ldots> = \ket|\phi_{i_1}\cdots\phi_{i_N}>_A & = \frac{1}{\sqrt{N!}}\Det\begin{bmatrix}\ket|\phi_{i_1}> & \cdots & \ket|\phi_{i_N}>\end{bmatrix}
  \end{align}
  これより定義と次の式は同値である.
  \begin{align}
    \hat{c}_i\ket|\phi_{i_1}\cdots\phi_{i_N}>_A         & = (-1)^\mu n_i\ket|\phi_{i_1}\cdots\phi_{i_{\mu-1}}\phi_{i_{\mu+1}}\cdots\phi_{i_N}>_A \\
    \hat{c}_i^\dagger\ket|\phi_{i_1}\cdots\phi_{i_N}>_A & = \ket|\phi_i\phi_{i_1}\cdots\phi_{i_N}>_A
  \end{align}
\end{proof}

\begin{theorem}[Q21-50, Q21-51]
  Fermi 粒子系の消滅, 生成演算子を状態に適用すると次のようになる.
  \begin{align}
    \hat{c}_i\ket|\ldots,n_i,\ldots>         & = (-1)^{\sum_{j=1}^{i-1}n_j}n_i\ket|\ldots,1-n_i,\ldots>       \\
    \hat{c}_i^\dagger\ket|\ldots,n_i,\ldots> & = (-1)^{\sum_{j=1}^{i-1}n_j}(1 - n_i)\ket|\ldots,1-n_i,\ldots>
  \end{align}
\end{theorem}
\begin{proof}
  よって次の式は同値である.
  \begin{align}
    \hat{c}_i\ket|\ldots,n_i,\ldots>_A         & = (-1)^{\sum_{j=1}^{i-1}n_j} n_i\ket|\ldots,1-n_i,\ldots>_A    \\
    \hat{c}_i^\dagger\ket|\ldots,n_i,\ldots>_A & = (-1)^{\sum_{j=1}^{i-1}n_j}(1-n_i)\ket|\ldots,1-n_i,\ldots>_A
  \end{align}
\end{proof}

\begin{theorem}[Q21-52]
  Fermi 粒子系における消滅演算子 $\hat{c}_i$ と生成演算子 $\hat{c}_i^\dagger$ の反交換関係は次のようになる.
  \begin{align}
    \{\hat{c}_i, \hat{c}_j^\dagger\} = \delta_{ij}, \qquad \{\hat{c}_i, \hat{c}_j\} = \{\hat{c}_i^\dagger, \hat{c}_j^\dagger\} = 0
  \end{align}
\end{theorem}
\begin{proof}
  \begin{align}
    \hat{c}_i\ket|\ldots,n_i,\ldots>_A         & = (-1)^{\sum_{j=1}^{i-1}n_j}n_i\ket|\ldots,0,\ldots>_A       \\
    \hat{c}_i^\dagger\ket|\ldots,n_i,\ldots>_A & = (-1)^{\sum_{j=1}^{i-1}n_j}(1 - n_i)\ket|\ldots,1,\ldots>_A
  \end{align}
  \begin{align}
    \hat{c}_i\hat{c}_i^\dagger\ket|\ldots,n_i,\ldots>_A         & = (1 - n_i)\ket|\ldots,0,\ldots>_A \\
    \hat{c}_i^\dagger\hat{c}_i\ket|\ldots,n_i,\ldots>_A         & = n_i\ket|\ldots,1,\ldots>_A       \\
    \hat{c}_i\hat{c}_i\ket|\ldots,n_i,\ldots>_A                 & = 0                                \\
    \hat{c}_i^\dagger\hat{c}_i^\dagger\ket|\ldots,n_i,\ldots>_A & = 0
  \end{align}
  \begin{align}
    \{\hat{c}_i, \hat{c}_i^\dagger\} = 1, \qquad \{\hat{c}_i, \hat{c}_i\} = \{\hat{c}_i^\dagger, \hat{c}_i^\dagger\} = 0
  \end{align}
  添字 $i, j$ が $i < j$ の順となっているとき先に $\hat{c}_i$ が適用されると後置の演算子で粒子数が 1 ずれることを考慮して次のようになる.
  \begin{align}
    \hat{c}_i\hat{c}_j^\dagger\ket|\ldots,n_i,\ldots,n_j,\ldots>_A         & = (-1)^{\sum_{k=i}^{j-1}n_k}n_i(1 - n_j)\ket|\ldots,1-n_i,\ldots,1-n_j,\ldots>_A           \\
    \hat{c}_j^\dagger\hat{c}_i\ket|\ldots,n_i,\ldots,n_j,\ldots>_A         & = (-1)^{1 + \sum_{k=i}^{j-1}n_k}n_i(1 - n_j)\ket|\ldots,1-n_i,\ldots,1-n_j,\ldots>_A       \\
    \hat{c}_i^\dagger\hat{c}_j\ket|\ldots,n_i,\ldots,n_j,\ldots>_A         & = (-1)^{\sum_{k=i}^{j-1}n_k}(1 - n_i)n_j\ket|\ldots,1-n_i,\ldots,1-n_j,\ldots>_A           \\
    \hat{c}_j\hat{c}_i^\dagger\ket|\ldots,n_i,\ldots,n_j,\ldots>_A         & = (-1)^{1 + \sum_{k=i}^{j-1}n_k}(1 - n_i)n_j\ket|\ldots,1-n_i,\ldots,1-n_j,\ldots>_A       \\
    \hat{c}_i\hat{c}_j\ket|\ldots,n_i,\ldots,n_j,\ldots>_A                 & = (-1)^{\sum_{k=i}^{j-1}n_k}n_in_j\ket|\ldots,1-n_i,\ldots,1-n_j,\ldots>_A                 \\
    \hat{c}_j\hat{c}_i\ket|\ldots,n_i,\ldots,n_j,\ldots>_A                 & = (-1)^{1 + \sum_{k=i}^{j-1}n_k}n_in_j\ket|\ldots,1-n_i,\ldots,1-n_j,\ldots>_A             \\
    \hat{c}_i^\dagger\hat{c}_j^\dagger\ket|\ldots,n_i,\ldots,n_j,\ldots>_A & = (-1)^{\sum_{k=i}^{j-1}n_k}(1 - n_i)(1 - n_j)\ket|\ldots,1-n_i,\ldots,1-n_j,\ldots>_A     \\
    \hat{c}_j^\dagger\hat{c}_i^\dagger\ket|\ldots,n_i,\ldots,n_j,\ldots>_A & = (-1)^{1 + \sum_{k=i}^{j-1}n_k}(1 - n_i)(1 - n_j)\ket|\ldots,1-n_i,\ldots,1-n_j,\ldots>_A
  \end{align}
  $\{A, B\} = \{B, A\}$ 次の反交換関係が求まる.
  \begin{align}
    \{\hat{c}_i, \hat{c}_j^\dagger\} = \{\hat{c}_i, \hat{c}_j\} = \{\hat{c}_i^\dagger, \hat{c}_j^\dagger\} & = 0 \qquad (i\neq j)
  \end{align}
\end{proof}

\begin{proposition}
  Fermi 粒子系における消滅演算子 $\hat{c}_i$ と生成演算子 $\hat{c}_i^\dagger$ は互いに Hermite 共役である.
\end{proposition}
\begin{proof}
  \begin{align}
    \bra<n_1',\ldots,n_i',\ldots|\hat{c}_i^\dagger\ket|n_1,\ldots,n_i,\ldots>_A & = (-1)^{\sum_{j=1}^{i-1}n_j}\braket<n_1',\ldots,n_i',\ldots|n_1,\ldots,n_i+1,\ldots>_A                                        \\
                                                                                & = (-1)^{\sum_{j=1}^{i-1}n_j}\delta_{n_1n_1'}\cdots\delta_{n_{i-1}n_{i-1}'}\delta_{n_i+1n_{i}'}\delta_{n_{i+1}n_{i+1}'}\cdots  \\
    \bra<n_1,\ldots,n_i,\ldots|\hat{c}_i\ket|n_1',\ldots,n_i',\ldots>_A         & = (-1)^{\sum_{j=1}^{i-1}n_j'}\braket<n_1,\ldots,n_i,\ldots|n_1',\ldots,n_i'-1,\ldots>_A                                       \\
                                                                                & = (-1)^{\sum_{j=1}^{i-1}n_j'}\delta_{n_1n_1'}\cdots\delta_{n_{i-1}n_{i-1}'}\delta_{n_i+1n_{i}'}\delta_{n_{i+1}n_{i+1}'}\cdots
  \end{align}
  より Hermite 共役である.
\end{proof}

\begin{proposition}
  \begin{align}
    \hat{n}_i^\dagger & = \hat{n}_i, \hat{n}_i\ket|\ldots,n_i,\ldots>_A = n_i\ket|\ldots,n_i,\ldots>_A \\
    \hat{N}^\dagger   & = \hat{N}, \hat{N}\ket|(n_i)_{i\in I}>_A = N\ket|(n_i)_{i\in I}>_A
  \end{align}
\end{proposition}
\begin{proof}
  \begin{align}
    \hat{n}_i^\dagger                  & = (\hat{c}_i^\dagger\hat{c}_i)^\dagger = \hat{c}_i^\dagger\hat{c}_i = \hat{n}_i                                                                  \\
    \hat{N}^\dagger                    & = \sum_{i\in I}\hat{n}_i^\dagger = \sum_{i\in I}\hat{n}_i = \hat{N}                                                                              \\
    \hat{n}_i\ket|\ldots,n_i,\ldots>_A & = \hat{c}_i^\dagger\hat{c}_i\ket|\ldots,n_i,\ldots>_A = (-1)^{2\sum_{j=1}^{i-1}n_j}n_i^2\ket|\ldots,n_i,\ldots>_A = n_i\ket|\ldots,n_i,\ldots>_A \\
    \hat{N}\ket|(n_i)_{i\in I}>_A      & = \sum_{i\in I}\hat{n}_i\ket|(n_i)_{i\in I}>_A = \sum_{i\in I}n_i\ket|(n_i)_{i\in I}>_A = N\ket|(n_i)_{i\in I}>_A
  \end{align}
\end{proof}

\begin{proposition}
  \begin{align}
    \ket|\mathrm{vac}> = \ket|0,0,\ldots>
  \end{align}
  \begin{align}
    \hat{c}_i\ket|\mathrm{vac}> = 0 \\
    \braket<\mathrm{vac}|\mathrm{vac}> = 1
  \end{align}
\end{proposition}
\begin{proof}

\end{proof}

\section{Bose, Fermi 粒子系の消滅演算子 $\hat{b}_i$ と生成演算子 $\hat{b}_i^\dagger$}
前 2 章で行った生成, 消滅演算子を統一する.
完全正規直交系
\begin{definition}
  1 粒子状態の Hilbert 空間 $\HH_{single}$ の完全正規直交系 $(\ket|\phi_i>)_{i\in I}$ に対して
\end{definition}

\begin{definition}
  \begin{align}
    \hat{b}_i & = \begin{dcases}
                    \hat{a}_i & (Bose)  \\
                    \hat{c}_i & (Fermi)
                  \end{dcases},\quad
    \Det^{(\pm)} = \begin{dcases}
                     \per & (+) \\
                     \Det & (-)
                   \end{dcases},\quad
    [\hat{A}, \hat{B}]_\mp = \begin{dcases}
                               [\hat{A}, \hat{B}]   & (-) \\
                               \{\hat{A}, \hat{B}\} & (+)
                             \end{dcases} \\
    \begin{dcases}
      \hat{b}_i\ket|\mathrm{vac}> = 0        \\
      \braket<\mathrm{vac}|\mathrm{vac}> = 1 \\
    \end{dcases}
  \end{align}
\end{definition}
これより次の定理が成り立つ. 証明は略.
\begin{theorem}
  \begin{align*}
     & \hat{b}_i\frac{1}{\sqrt{N!}}\Det^{(\pm)}\begin{bmatrix}\ket|\phi_{i_1}> \cdots \ket|\phi_{i_N}>\end{bmatrix} = \frac{1}{\sqrt{(N-1)!}}\sum_{\substack{\mu\in X                                                                                \\ i_\mu = i}}(\pm1)^\mu\Det^{(\pm)}\begin{bmatrix}\ket|\phi_{i_1}> \cdots \ket|\phi_{i_{\mu-1}}> & \ket|\phi_{i_{\mu+1}}>\cdots\ket|\phi_{i_N}>\end{bmatrix} \\
     & \hat{b}_i^\dagger\frac{1}{\sqrt{N!}}\Det^{(\pm)}\begin{bmatrix}\ket|\phi_{i_1}> \cdots \ket|\phi_{i_N}>\end{bmatrix} = \frac{1}{\sqrt{(N+1)!}}\Det^{(\pm)}\begin{bmatrix}\ket|\phi_i> & \ket|\phi_{i_1}> \cdots \ket|\phi_{i_N}>\end{bmatrix}
  \end{align*}
  \begin{align}
    [\hat{b}_i, \hat{b}_j^\dagger]_\mp & = \delta_{ij}, \quad [\hat{b}_i, \hat{b}_j]_\mp = [\hat{b}_i^\dagger, \hat{b}_j^\dagger]_\mp = 0 \\
    \hat{n}_i                          & = \hat{b}_i^\dagger\hat{b}_i = n_i
  \end{align}
  \begin{align}
    \Det^{(\pm)}\begin{bmatrix}\ket|\phi_{i_{\sigma(1)}}> \cdots \ket|\phi_{i_{\sigma(N)}}>\end{bmatrix} = (\pm 1)^\sigma\Det^{(\pm)}\begin{bmatrix}\ket|\phi_{i_1}> \cdots \ket|\phi_{i_N}>\end{bmatrix}
  \end{align}
  \begin{align}
    \hat{b}_i\ket|\mathrm{vac}>
  \end{align}
\end{theorem}

基底状態
\begin{align}
  \frac{1}{\sqrt{N!}}\Det^{(\pm)}\begin{bmatrix}\ket|\phi_{i_1}>\cdots\ket|\phi_{i_N}>\end{bmatrix}
\end{align}


\section{演算子の粒子数表示: 1 粒子演算子とその和、2 粒子演算子とその和の導入}
現実の粒子系における観測量はある 1 つの相互作用に関して関与する粒子数は 1 個か 2 個しかない. これを 1 粒子演算子, 2 粒子演算子と呼ぶ.

\begin{definition}[$n$ 粒子演算子]
  Hilbert 空間 $\HH^{(n)}$ において粒子交換に関して対称な演算子を $n$ 粒子演算子と呼ぶ.
  このとき $n$ 粒子演算子 $\hat{f}$ を Hilbert 空間 $\HH^{(N)}$ の粒子 $\mu_1,\ldots,\mu_n$ に対して埋め込んだ演算子を $\hat{f}_{\mu_1\cdots\mu_n}$ と書く.
  そして $n$ 粒子演算子の粒子対に関する和 $\hat{f}^{\mathrm{tot}}$ を次のように定義する.
  \begin{align}
    \hat{f}^{\mathrm{tot}} = \sum_{\substack{\mu_1,\ldots,\mu_n\in X \\ \mu_1 < \cdots < \mu_n}}\hat{f}_{\mu_1\cdots\mu_n}
  \end{align}
  特に量子力学では今のところ 3 粒子以上が相互に作用することはない為に 1 粒子演算子と 2 粒子演算子のみが扱われる.
  そして $\HH^{(N)}$ において明らかに状態が $\{\ket|\phi_i>\}_{i\in I}$ を用いて表現されているならば添字を用いて表示すると定義する.
  \begin{align}
    \ket|i_1\cdots i_N> = \ket|\phi_{i_1}>\cdots\ket|\phi_{i_N}>
  \end{align}
\end{definition}
\begin{example}
  例えば Hamiltonian 演算子 $\hat{H}$ は 1 粒子演算子の粒子に関する和 $\hat{h}^{\mathrm{tot}}$ と2 粒子演算子の粒子対に関する和 $\hat{v}^{\mathrm{tot}}$ で表現できる.
  外部から磁場 $B$ をかけた多電子原子を考える。原子番号 $Z$ の多電子原子を考えることにします。原点に電荷 $+Ze$ を持ち無限に重い原子核が位置しているとします。その回りに、$N$ 個のそれぞれが電荷 $-e$ と質量 $me$ を持つ電子が運動しているとします。この原子が中性原子の状態にあるならば $N = Z$ であり、また、自然数 $n = 1, 2,\ldots$ に関して $n$ 価の陽イオンの状態にあるならば $N = Z - n$ であります。この $N$ 個の電子という同種粒子からなる物理系を記述する Hamiltonian 演算子 $\hat{H}$ は次のように与えられます.
  $\hat{H} = \hat{h}^{\mathrm{tot}} + \hat{v}^{\mathrm{tot}}$
  \begin{align}
    \hat{H}          & = \frac{1}{2m_e}\sum_{\mu=1}^{N}\hat{\pp}_\mu^2 - Ze^2\sum_{\mu=1}^{N}\frac{1}{|\hat{\rr}_\mu|} + e^2\sum_{1\leq\mu<\nu\leq N}\frac{1}{|\hat{\rr}_\mu - \hat{\rr}_\nu|} + \frac{e}{2m_ec}(\hat{\bm{L}} + 2\hat{\bm{S}})\cdot\bm{B} + \frac{e^2}{8m_ec^2}\sum_{\mu=1}^{N}(\bm{B}\times\hat{\rr}_\mu)^2 \\
    \hat{h}_\mu      & = \frac{1}{2m_e}\hat{\pp}_\mu^2 - \frac{Ze^2}{|\hat{\rr}_\mu|} + \frac{e}{2m_ec}(\hat{\bm{l}_\mu} + 2\hat{\bm{s}_\mu})\cdot\bm{B} + \frac{e^2}{8m_ec^2}(\bm{B}\times\hat{\rr}_\mu)^2                                                                                                                  \\
    \hat{v}_{\mu\nu} & = \frac{e^2}{|\hat{\rr}_\mu - \hat{\rr}_\nu|}
  \end{align}
\end{example}

\section{$n$ 粒子演算子の和の粒子数表示}
Bose, Fermi 粒子系や 1, 2 粒子演算子を分ける理由がよく分からなかったので 1 つにまとめました. これらの章の採点については難しければ 0 点でいいです.
2 粒子演算子において $\alpha, \beta$ の定義がよろしくないです.

\begin{theorem}
  $n$ 粒子演算子について次のような性質が認められる.
  \begin{align}
    \hat{f}^{\mathrm{tot}} & = \hat{P}(\sigma)\hat{f}^{\mathrm{tot}}\hat{P}^\dagger(\sigma) \\
    \hat{f}^{\mathrm{tot}} & = \frac{1}{n!}\sum_{\substack{\mu_1,\ldots,\mu_n\in X          \\ \mu_\nu \neq \mu_\xi}}\hat{f}_{\mu_1\cdots\mu_n}
  \end{align}
\end{theorem}
\begin{proof}
  置換に関して対称な演算子であるから $\mu_1,\ldots,\mu_n$ 番目の状態の置換に対して不変であり, その他の添字については置換しても両側で対応を取れているのでこちらも置換に対して不変である.
  \begin{align}
    \bra<i_1\cdots i_N|\hat{P}(\sigma)\hat{f}^{\mathrm{tot}}\hat{P}^\dagger(\sigma)\ket|j_1\cdots j_N> & = \sum_{\substack{\mu_1,\ldots,\mu_n\in X                      \\ \mu_1 < \cdots < \mu_n }}\bra<i_{\sigma(1)}\cdots i_{\sigma(N)}|\hat{f}_{\mu_1\cdots\mu_n}\ket|j_{\sigma(1)}\cdots j_{\sigma(N)}> \\
                                                                                                       & = \sum_{\substack{\mu_1,\ldots,\mu_n\in X                      \\ \mu_1 < \cdots < \mu_n }}\bra<i_1\cdots i_N|\hat{f}_{\mu_1\cdots\mu_n}\ket|j_1\cdots j_N> \\
                                                                                                       & = \bra<i_1\cdots i_N|\hat{f}^{\mathrm{tot}}\ket|j_1\cdots j_N>
  \end{align}
  また置換に対して対称であるから次のようにも変形できる.
  \begin{align}
    \hat{f}^{\mathrm{tot}} & = \sum_{\substack{\mu_1,\ldots,\mu_n\in X \\ \mu_1 < \cdots < \mu_n}}\hat{f}_{\mu_1\cdots\mu_n} = \frac{1}{n!}\sum_{\substack{\mu_1,\ldots,\mu_n\in X \\ \mu_n \neq \mu_{n'}}}\hat{f}_{\mu_1\cdots\mu_n}
  \end{align}
\end{proof}
\begin{example}
  1 粒子演算子 $\hat{h}$, 2 粒子演算子 $\hat{v}$ についても上の定理が成り立つ.
  \begin{align}
    \hat{h} & = \hat{P}(\sigma)\hat{h}\hat{P}^\dagger(\sigma)  \\
    \hat{v} & = \hat{P}(\sigma)\hat{v}\hat{P}^\dagger(\sigma).
  \end{align}
  例えば 2 粒子演算子 $\hat{v}$ について交換演算子で置換すると
  \begin{align}
    \bra<ji|v\ket|lk> & = \bra<ji|\hat{E}\hat{v}\hat{E}^\dagger\ket|lk> = \bra<ij|v\ket|kl>
  \end{align}
  となる.
\end{example}

\begin{theorem}[Q21-61(i)(ii)(iii)(iv)(v)(vi)(vii)(viii), Q21-62, Q21-63(i)(ii)(iii), Q21-64(i)(ii)(iii), Q21-65(i)(ii)(iii), Q21-66(i)(ii), Q21-67(i)(ii)(iii)(iv)(v)(vi)(vii)(viii)(ix)(x)(xi)(xii)(xiii)(xiv)(xv)(xvi), Q21-68, Q21-69(i)(ii)(iii)(iv), Q21-70(i)(ii)(iii)(iv)(v)(vi)(vii)(viii), Q21-71(i)(ii)(iii), Q21-72(i)(ii)(iii), Q21-73(i)(ii)(iii), Q21-74(i)(ii), \\ Q21-75(i)(ii)(iii)(iv)(v)(vi)(vii)(viii)(ix)(x)(xi)(xii)(xiii)(xiv)(xv)(xvi), Q21-76, Q21-77(i)(ii)(iii)(iv)]
  Bose, Fermi 粒子系の Hilbert 空間 $\HH_{\mathrm{M.P.}}$ において $n$ 粒子演算子 $\hat{f}$ の和 $\hat{f}^{\mathrm{tot}}$ は消滅, 生成演算子 $\hat{b}_i, \hat{b}_i^\dagger$ を用いて次のように表現できる.
  \begin{align}
    \hat{f}^{\mathrm{tot}} & = \sum_{\substack{j_1,\ldots,j_n\in I \\ k_1,\ldots,k_n\in I}}\bra<j_1\cdots j_n|f\ket|k_1\cdots k_n>\hat{b}_{j_1}^\dagger\cdots\hat{b}_{j_n}^\dagger\hat{b}_{k_1}\cdots\hat{b}_{k_n}
  \end{align}
\end{theorem}
\begin{proof}
  $n$ 粒子演算子を適用する
  \begin{align}
    \hat{f}^{\mathrm{tot}}\frac{1}{\sqrt{N!}}\Det^{(\pm)}\begin{bmatrix}\ket|i_1>\cdots\ket|i_N>\end{bmatrix}
     & = \frac{1}{\sqrt{N!}}\sum_{\sigma\in\SS_N}(\pm1)^\sigma\hat{f}^{\mathrm{tot}}\hat{P}^\dagger(\sigma)\ket|i_1\cdots i_N>
  \end{align}
  ここで完全性を用いて次のように単位演算子の分解ができる.
  \begin{align}
    \sum_{j_1,\ldots,j_n\in I}\ket|j_1\cdots j_n>\bra<j_1\cdots j_n| = \hat{1}
  \end{align}
  これより $\hat{f}^{\mathrm{tot}}\hat{P}(\sigma)$ は次のように変形できる.
  \begin{align}
     & \hat{f}^{\mathrm{tot}}\hat{P}(\sigma)\ket|i_1\cdots i_N> \\
     & = \frac{1}{n!}\sum_{\substack{\mu_1,\ldots,\mu_n\in X    \\ \mu_\nu\neq\mu_\xi}}\hat{f}_{\mu_1\cdots\mu_n}\ket|i_{\sigma(1)}\cdots i_{\sigma(N)}> \\
     & = \frac{1}{n!}\sum_{\substack{\mu_1,\ldots,\mu_n\in X    \\ \mu_\nu\neq\mu_\xi}}\sum_{j_1,\ldots,j_n\in I}\ket|j_1\cdots j_n>\bra<j_1\cdots j_n|\hat{f}_{\mu_1\cdots\mu_n}\ket|i_{\sigma(1)}\cdots i_{\sigma(N)}>                                                           \\
     & = \frac{1}{n!}\sum_{\substack{\mu_1,\ldots,\mu_n\in X    \\ \mu_\nu\neq\mu_\xi}}\sum_{j_1,\ldots,j_n\in I}\ket|i_{\sigma(1)}\cdots j_1\cdots j_n\cdots i_{\sigma(N)}>\bra<j_1\cdots j_n|f\ket|i_{\sigma(\mu_1)}\cdots i_{\sigma(\mu_n)}>                                                           \\
     & = \frac{1}{n!}\sum_{\substack{j_1,\ldots,j_n\in I        \\ k_1,\ldots,k_n\in I}}\bra<j_1\cdots j_n|f\ket|k_1\cdots k_n>\sum_{\substack{\mu_1,\ldots,\mu_n\in X \\ \mu_\nu\neq\mu_\xi \\ i_{\sigma(\mu_\nu)} = k_\nu}}|i_{\sigma(1)}\cdots\underbrace{j_1}_{\mu_1}\cdots\underbrace{j_n}_{\mu_n}\cdots i_{\sigma(N)}\rangle \\
     & = \frac{1}{n!}\sum_{\substack{j_1,\ldots,j_n\in I        \\ k_1,\ldots,k_n\in I}}\bra<j_1\cdots j_n|f\ket|k_1\cdots k_n>\sum_{\substack{\mu_1,\ldots,\mu_n\in X \\ \mu_\nu\neq\mu_\xi \\ i_{\sigma(\mu_\nu)} = k_\nu}}\hat{P}(\sigma)|i_1\cdots\underbrace{j_1}_{\sigma(\mu_1)}\cdots\underbrace{j_n}_{\sigma(\mu_n)}\cdots i_N\rangle
  \end{align}
  そして総和の変数を $\mu_\nu \to \sigma^{-1}(\alpha_\nu)$ と書き換えて総和の順序を交換することで permutation に変形できる.
  \begin{align}
     & \frac{1}{n!}\sum_{\sigma\in\SS_N}\sum_{\substack{\mu_1,\ldots,\mu_n\in X \\ \mu_\nu\neq\mu_\xi \\ i_{\sigma(\mu_\nu)} = k_\nu}}(\pm 1)^\sigma\hat{P}^\dagger(\sigma)|i_1\cdots\underbrace{j_\nu}_{\sigma(\mu_\nu)}\cdots i_N\rangle \\
     & = \frac{1}{n!}\sum_{\substack{\alpha_1,\ldots,\alpha_n\in X              \\ \alpha_\nu\neq\alpha_\xi \\ i_{\alpha_\nu} = k_\nu}}\sum_{\sigma\in\SS_N}(\pm 1)^\sigma\hat{P}^\dagger(\sigma)|i_1\cdots\underbrace{j_\nu}_{\alpha_\nu}\cdots i_N\rangle \\
     & = \sum_{\substack{\alpha_1,\ldots,\alpha_n\in X                          \\ \alpha_1 < \cdots < \alpha_n \\ i_{\alpha_\nu} = k_\nu}}\Det^{(\pm)}[\ket|i_1>\cdots\underbrace{\ket|j_\nu>}_{\alpha_\nu}\cdots\ket|i_N>]
  \end{align}
  次に消滅, 生成演算子 $\hat{b}_i, \hat{b}_i^\dagger$ の定義を用いてそれぞれ後ろから, 前からの順番で適用していくことで次のように変形できる.
  \begin{align}
     & \sum_{\substack{\alpha_1 < \cdots < \alpha_n                                                                                                                                                                                                      \\ i_{\alpha_\nu} = k_\nu}}\frac{1}{\sqrt{N!}}\Det^{(\pm)}\begin{bmatrix}\ket|i_1>\cdots\ket|i_{\alpha_\nu-1}> & \ket|j_\nu> & \ket|i_{\alpha_\nu+1}>\cdots\ket|i_N>\end{bmatrix} \\
     & = \hat{b}_{k_n}^\dagger\cdots\hat{b}_{k_1}^\dagger\sum_{\substack{\alpha_1 < \cdots < \alpha_n                                                                                                                                                    \\ i_{\alpha_\nu} = k_\nu}}\frac{(\pm 1)^{\sum_{\nu}\alpha_\nu}}{\sqrt{(N - n)!}}\Det^{(\pm)}\begin{bmatrix}\ket|i_1>\cdots\ket|i_{\alpha_\nu-1}> & \ket|i_{\alpha_\nu+1}>\cdots\ket|i_N>\end{bmatrix}                        \\
     & = \hat{b}_{k_n}^\dagger\cdots\hat{b}_{k_1}^\dagger\hat{b}_{k_1}\cdots\hat{b}_{k_n}\frac{1}{\sqrt{N!}}\Det^{(\pm)}\begin{bmatrix}\ket|i_1>\cdots\ket|i_{\alpha_\nu-1}> & \ket|i_{\alpha_\nu}> & \ket|i_{\alpha_\nu+1}>\cdots\ket|i_N>\end{bmatrix}
  \end{align}
  結局, 次のように変形できることがわかる.
  \begin{align}
     & \hat{f}^{\mathrm{tot}}\frac{1}{\sqrt{N!}}\Det^{(\pm)}\begin{bmatrix}\ket|i_1>\cdots\ket|i_N>\end{bmatrix} \\
     & = \sum_{\substack{j_1,\ldots,j_n\in I                                                                     \\ k_1,\ldots,k_n\in I}}\bra<j_1\cdots j_n|f\ket|k_1\cdots k_n>\hat{b}_{j_1}^\dagger\cdots\hat{b}_{j_n}^\dagger\hat{b}_{k_1}\cdots\hat{b}_{k_n}\ket|\phi_{i_1}\cdots\phi_{i_N}>_S\frac{1}{\sqrt{N!}}\Det^{(\pm)}\begin{bmatrix}\ket|i_1>\cdots\ket|i_N>\end{bmatrix}
  \end{align}
  さらに全粒子数 $N$ の Hilbert 空間における $n$ 粒子演算子を一般の Bose 粒子系に埋め込むことで $\HH_{Bose}$ 上では次のように表現できる.
  \begin{align}
    \hat{f}^{\mathrm{tot}} & = \frac{1}{n!}\sum_{\substack{j_1,\ldots,j_n\in I \\ k_1,\ldots,k_n\in I}}\bra<j_1\cdots j_n|f\ket|k_1\cdots k_n>\hat{b}_{j_1}^\dagger\cdots\hat{b}_{j_n}^\dagger\hat{b}_{k_n}\cdots\hat{b}_{k_1}
  \end{align}
\end{proof}

\section{1 粒子状態の完全正規直交系の取り替え}
\begin{theorem}
  2 つの完全正規直交系 $(\ket|\phi_i>)_{i\in I}, (\ket|\phi_i'>)_{i\in I}$ に対してそれぞれ消滅演算子 $\hat{b}_i, \hat{b}_i'$
  \begin{align}
    \hat{b}_i' = \sum_{j\in I}\braket<\phi_i'|\phi_j>\hat{b}_j
  \end{align}
\end{theorem}
\begin{proof}
  \begin{align}
    \hat{b}_i^{\prime\dagger}\frac{1}{\sqrt{N!}}\Det^{(\pm)}\begin{bmatrix}\ket|\phi_{i_1}'>\cdots\ket|\phi_{i_N}'>\end{bmatrix}
     & = \frac{1}{\sqrt{(N+1)!}}\Det^{(\pm)}\begin{bmatrix}\ket|\phi_i'> & \ket|\phi_{i_1}'>\cdots\ket|\phi_{i_N}'>\end{bmatrix}                                    \\
     & = \frac{1}{\sqrt{(N+1)!}}\Det^{(\pm)}\begin{bmatrix}\sum_{j\in I}\ket|\phi_j>\braket<\phi_j|\phi_i'> & \ket|\phi_{i_1}'>\cdots\ket|\phi_{i_N}'>\end{bmatrix} \\
     & = \sum_{j\in I}\braket<\phi_j|\phi_i'>\frac{1}{\sqrt{(N+1)!}}\Det^{(\pm)}\begin{bmatrix}\ket|\phi_j> & \ket|\phi_{i_1}'>\cdots\ket|\phi_{i_N}'>\end{bmatrix} \\
     & = \sum_{j\in I}\braket<\phi_j|\phi_i'>\hat{b}_j^\dagger\frac{1}{\sqrt{N!}}\Det^{(\pm)}\begin{bmatrix}\ket|\phi_{i_1}'>\cdots\ket|\phi_{i_N}'>\end{bmatrix}
  \end{align}
  これを
  \begin{align}
    \hat{b}_i^{\prime\dagger} & = \sum_{j\in I}\braket<\phi_j|\phi_i'>\hat{b}_j^\dagger
  \end{align}
  Hermite 共役を取ると示される.
  \begin{align}
    \hat{b}_i^{\prime} & = \sum_{j\in I}\braket<\phi_i'|\phi_j>\hat{b}_j
  \end{align}
\end{proof}
さらに完全正規直交系を入れ替えると次のような式が成り立つ. (Q21-79(iv))
\begin{align}
  \hat{b}_i         & = \sum_{j\in I}\braket<\phi_i|\phi_j'>\hat{b}_j'                \\
  \hat{b}_i^\dagger & = \sum_{j\in I}\braket<\phi_j'|\phi_i>\hat{b}_j^{\prime\dagger}
\end{align}

\begin{proposition}[Q21-80]
  ある完全正規直交系の生成消滅演算子について交換・反交換関係が成り立つことは他の完全正規直交系でも成り立つことと同値である.
\end{proposition}
\begin{proof}
  \begin{align}
    [\hat{b}_i, \hat{b}_j^\dagger]_\mp = \delta_{ij}, \quad [\hat{b}_i, \hat{b}_j]_\mp = [\hat{b}_i^\dagger, \hat{b}_j^\dagger]_\mp = 0
  \end{align}
  \begin{align}
    [\hat{b}_i, \hat{b}_j^\dagger]_\mp         & = \ab[\sum_{k\in I}\braket<\phi_i|\phi_k'>\hat{b}_k', \sum_{l\in I}\braket<\phi_l'|\phi_j>\hat{b}_l^{\prime\dagger}]_\mp = \sum_{k\in I}\sum_{l\in I}\braket<\phi_i|\phi_k'>\braket<\phi_l'|\phi_j>\ab[\hat{b}_k', \hat{b}_l^{\prime\dagger}]_\mp                                   \\
                                               & = \sum_{k\in I}\sum_{l\in I}\braket<\phi_i|\phi_k'>\braket<\phi_l'|\phi_j>\delta_{kl} = \sum_{k\in I}\braket<\phi_i|\phi_k'>\braket<\phi_k'|\phi_j> = \braket<\phi_i|\phi_j> = \delta_{ij}                                                                                          \\
    [\hat{b}_i, \hat{b}_j]_\mp                 & = \ab[\sum_{k\in I}\braket<\phi_i|\phi_k'>\hat{b}_k', \sum_{l\in I}\braket<\phi_j|\phi_l'>\hat{b}_l']_\mp = \sum_{k\in I}\sum_{l\in I}\braket<\phi_i|\phi_k'>\braket<\phi_j|\phi_l'>\ab[\hat{b}_k', \hat{b}_l']_\mp = 0                                                             \\
    [\hat{b}_i^\dagger, \hat{b}_j^\dagger]_\mp & = \ab[\sum_{k\in I}\braket<\phi_k'|\phi_i>\hat{b}_k^{\prime\dagger}, \sum_{l\in I}\braket<\phi_l'|\phi_j>\hat{b}_l^{\prime\dagger}]_\mp = \sum_{k\in I}\sum_{l\in I}\braket<\phi_k'|\phi_i>\braket<\phi_j|\phi_l'>\ab[\hat{b}_k^{\prime\dagger}, \hat{b}_l^{\prime\dagger}]_\mp = 0
  \end{align}
  完全正規直交系を入れ替えれば逆も示せることがわかる.
\end{proof}

\section{場の演算子の導入}
\begin{definition}
  位置座標・スピンの $z$ 成分に固有状態 $(\ket|\rr, s_z>)_{\rr\in\RR^3, s_z = -s,-s+1,\ldots,s-1,s}$ は完全正規直交系 $(\ket|\phi_i'>)_{i\in I'}$ を用いて表現できる.
  \begin{align}
    (\ket|\phi_i'>)_{i\in I'} & := (\ket|\rr, s_z>)_{\rr\in\RR^3, s_z = -s,-s+1,\ldots,s-1,s}
  \end{align}
  これに対して場の演算子 $\hat{\phi}(\rr, s_z)$ や 1 粒子状態 $\ket|\phi_i>$ $(i\in I)$ に関する波動関数 $\phi_i(\rr, s_z)$ を次のように定義する.
  \begin{align}
    \hat{\phi}(\rr, s_z) & := \hat{b}_{\rr,s_z}'       \\
    \phi_i(\rr, s_z)     & := \braket<\rr, s_z|\phi_i>
  \end{align}
\end{definition}
\begin{theorem}[Q21-81(i)(ii), Q21-82]
  場の演算子とその Hermite 共役 $\hat{\phi}, \hat{\phi}^\dagger$ と消滅, 生成演算子 $\hat{b}_i, \hat{b}_i^\dagger$ は互いに表現できる.
  \begin{align}
    \hat{\phi}(\rr, s_z)         & = \sum_{i\in I}\phi_i(\rr, s_z)\hat{b}_i           & \hat{b}_i         & = \sum_{s_z = -s}^s\int\dl{\rr}\phi_i^*(\rr, s_z)\hat{\phi}(\rr, s_z)       \\
    \hat{\phi}^\dagger(\rr, s_z) & = \sum_{i\in I}\phi_i^*(\rr, s_z)\hat{b}_i^\dagger & \hat{b}_i^\dagger & = \sum_{s_z = -s}^s\int\dl{\rr}\phi_i(\rr, s_z)\hat{\phi}^\dagger(\rr, s_z)
  \end{align}
\end{theorem}
\begin{proof}
  \begin{align}
    \hat{\phi}(\rr, s_z)         & = \hat{b}_{\rr,s_z}' = \sum_{i\in I}\braket<\rr,s_z|\phi_i>\hat{b}_i = \sum_{i\in I}\phi_i(\rr, s_z)\hat{b}_i                                  \\
    \hat{\phi}^\dagger(\rr, s_z) & = \hat{b}_{\rr,s_z}^{\prime\dagger} = \sum_{i\in I}\braket<\phi_i|\rr,s_z>\hat{b}_i^\dagger = \sum_{i\in I}\phi_i^*(\rr, s_z)\hat{b}_i^\dagger
  \end{align}
  \begin{align}
    \hat{b}_i         & = \sum_{s_z = -s}^s\int\dl{\rr}\braket<\phi_i|\rr, s_z>\hat{b}_{\rr,s_z}' = \sum_{s_z = -s}^s\int\dl{\rr}\phi_i^*(\rr, s_z)\hat{\phi}(\rr, s_z)                      \\
    \hat{b}_i^\dagger & = \sum_{s_z = -s}^s\int\dl{\rr}\braket<\rr, s_z|\phi_i>\hat{b}_{\rr,s_z}^{\prime\dagger} = \sum_{s_z = -s}^s\int\dl{\rr}\phi_i(\rr, s_z)\hat{\phi}^\dagger(\rr, s_z)
  \end{align}
\end{proof}
\begin{theorem}
  場の演算子とその Hermite 共役の交換・反交換関係
  \begin{align}
    [\hat{\phi}(\rr, s_z), \hat{\phi}^\dagger(\rr', s_z')]_\mp & = \delta(\rr - \rr')\delta_{s_zs_z'}                                     \\
    [\hat{\phi}(\rr, s_z), \hat{\phi}(\rr', s_z')]_\mp         & = [\hat{\phi}^\dagger(\rr, s_z), \hat{\phi}^\dagger(\rr', s_z')]_\mp = 0
  \end{align}
\end{theorem}
\begin{theorem}
  粒子数密度演算子
\end{theorem}
\begin{proof}
  \begin{align}
    \hat{N} & = \sum_{i\in I}\hat{b}_i^\dagger\hat{b}_i = \sum_{s_z = -s}^{s}\int\dl{\rr}\hat{\rho}(\rr, s_z)
  \end{align}
\end{proof}
\begin{definition}
  \begin{align}
    \hat{\phi}(\rr, s_z)\ket|\mathrm{vac}> = 0 \\
    \braket<\mathrm{vac}|\mathrm{vac}> = 1
  \end{align}
  \begin{align}
    |(\rr_\mu, s_{z\mu})_{\mu\in X}) & := \frac{1}{\sqrt{N!}}\hat{\phi}^\dagger(\rr_1, s_{z,1})\cdots\hat{\phi}^\dagger(\rr_N, s_{z,N})\ket|\mathrm{vac}>
  \end{align}
\end{definition}
\begin{theorem}
  \begin{align}
    |(\rr_{\sigma(\mu)}, s_{z\sigma(\mu)})_{\mu\in X}) & = (\pm 1)^\sigma|(\rr_{\mu}, s_{z\mu})_{\mu\in X})                                                  \\
    |(\rr_{\mu}, s_{z\mu})_{\mu\in X})                 & = \frac{1}{N!}\sum_{\sigma\in\SS_N}(\pm 1)^\sigma|(\rr_{\sigma(\mu)}, s_{z\sigma(\mu)})_{\mu\in X})
  \end{align}
\end{theorem}
\begin{proof}
  \begin{align}
    |(\rr_{\sigma(\mu)}, s_{z\sigma(\mu)})_{\mu\in X}) & = \frac{1}{\sqrt{N!}}\hat{\phi}^\dagger(\rr_{\sigma(1)}, s_{z\sigma(1)})\cdots\hat{\phi}^\dagger(\rr_{\sigma(N)}, s_{z\sigma(N)})\ket|\mathrm{vac}>             \\
                                                       & = \frac{1}{\sqrt{N!}}\hat{b}^{\prime\dagger}_{\rr_{\sigma(1)}, s_{z\sigma(1)}}\cdots\hat{b}^{\prime\dagger}_{\rr_{\sigma(N)}, s_{z\sigma(N)}}\ket|\mathrm{vac}> \\
                                                       & = (\pm 1)^\sigma\frac{1}{\sqrt{N!}}\hat{b}^{\prime\dagger}_{\rr_1, s_{z1}}\cdots\hat{b}^{\prime\dagger}_{\rr_{N}, s_{zN}}\ket|\mathrm{vac}>                     \\
                                                       & = (\pm 1)^\sigma\frac{1}{\sqrt{N!}}\hat{\phi}^\dagger(\rr_1, s_{z1})\cdots\hat{\phi}^\dagger(\rr_N, s_{zN})\ket|\mathrm{vac}>                                   \\
                                                       & = (\pm 1)^\sigma|(\rr_{\mu}, s_{z\mu})_{\mu\in X})
  \end{align}
\end{proof}
\begin{theorem}
  \begin{align}
     & ((\rr_{\mu}, s_{z\mu})_{\mu\in X}|(\rr_{\mu}', s_{z\mu}')_{\mu\in X'})                                                                                                                          \\
     & = \delta_{NN'}\frac{1}{N!}\sum_{\sigma\in\SS_N}(\pm 1)^\sigma\delta(\rr_1 - \rr_{\sigma(1)}')\delta_{s_{z1}s_{z\sigma(1)}'}\cdots\delta(\rr_N - \rr_{\sigma(N)}')\delta_{s_{zN}s_{z\sigma(N)}'}
  \end{align}
\end{theorem}
\begin{proof}
  まずは $N = N'$ の場合を考える.
  \begin{align}
     & ((\rr_{\mu}, s_{z\mu})_{\mu\in X}|(\rr_{\mu}', s_{z\mu}')_{\mu\in X}) = \frac{1}{N!}\sum_{\sigma\in\SS_N}(\pm 1)^\sigma((\rr_{\mu}, s_{z\mu})_{\mu\in X}|(\rr_{\sigma(\mu)}', s_{z\sigma(\mu)}')_{\mu\in X})                                                    \\
     & = \frac{1}{N!^2}\sum_{\sigma\in\SS_N}(\pm 1)^\sigma\bra<\mathrm{vac}|\hat{\phi}(\rr_N, s_{zN})\cdots\hat{\phi}(\rr_1, s_{z1})\hat{\phi}^\dagger(\rr_{\sigma(1)}', s_{z\sigma(1)}')\cdots\hat{\phi}^\dagger(\rr_{\sigma(N)}', s_{z\sigma(N)}')\ket|\mathrm{vac}>
  \end{align}
  TODO:
  \begin{align}
     & \hat{\phi}(\rr_N, s_{zN})\cdots\hat{\phi}(\rr_1, s_{z1})\hat{\phi}^\dagger(\rr_{\sigma(1)}', s_{z\sigma(1)}')\cdots\hat{\phi}^\dagger(\rr_{\sigma(N)}', s_{z\sigma(N)}') \\
     & = \hat{\phi}(\rr_1, s_{z1})\hat{\phi}^\dagger(\rr_{\sigma(1)}', s_{z\sigma(1)}')\cdots\hat{\phi}(\rr_N, s_{zN})\hat{\phi}^\dagger(\rr_{\sigma(N)}', s_{z\sigma(N)}')
  \end{align}
  真空状態について交換関係・反交換関係を用いて次のように計算できる.
  \begin{align}
    \hat{\phi}(\rr, s_z)\hat{\phi}^\dagger(\rr', s_z')\ket|\mathrm{vac}> & = \pm\hat{\phi}^\dagger(\rr', s_z')\hat{\phi}(\rr, s_z)\ket|\mathrm{vac}> + \delta(\rr - \rr')\delta_{s_zs_z'}\ket|\mathrm{vac}> \\
                                                                         & = \delta(\rr - \rr')\delta_{s_zs_z'}\ket|\mathrm{vac}>
  \end{align}
  これを帰納的に適用することで次のように計算できる.
  \begin{align}
     & \bra<\mathrm{vac}|\hat{\phi}(\rr_1, s_{z1})\hat{\phi}^\dagger(\rr_{\sigma(1)}', s_{z\sigma(1)}')\cdots\hat{\phi}(\rr_N, s_{zN})\hat{\phi}^\dagger(\rr_{\sigma(N)}', s_{z\sigma(N)}')\ket|\mathrm{vac}> \\
     & = \delta(\rr_1 - \rr_{\sigma(1)}')\delta_{s_{z1}s_{z\sigma(1)}'}\cdots\delta(\rr_N - \rr_{\sigma(N)}')\delta_{s_{zN}s_{z\sigma(N)}'}\braket<\mathrm{vac}|\mathrm{vac}>                                 \\
     & = \delta(\rr_1 - \rr_{\sigma(1)}')\delta_{s_{z1}s_{z\sigma(1)}'}\cdots\delta(\rr_N - \rr_{\sigma(N)}')\delta_{s_{zN}s_{z\sigma(N)}'}
  \end{align}
  よって
  \begin{align}
     & ((\rr_{\mu}, s_{z\mu})_{\mu\in X}|(\rr_{\mu}', s_{z\mu}')_{\mu\in X})                                                                                                               \\
     & = \frac{1}{N!}\sum_{\sigma\in\SS_N}(\pm 1)^\sigma\delta(\rr_1 - \rr_{\sigma(1)}')\delta_{s_{z1}s_{z\sigma(1)}'}\cdots\delta(\rr_N - \rr_{\sigma(N)}')\delta_{s_{zN}s_{z\sigma(N)}'}
  \end{align}
  となる. $N \neq N'$ の場合, Bose, Fermi 粒子系どちらも粒子数が異なる状態の内積は 0 より固有状態の内積も 0 となる.
\end{proof}

\begin{theorem}
  \begin{align}
    \hat{1} & = \sum_{N = 0}^{\infty}\sum_{s_{z1} = -s}^{s}\int\dl{\rr_1}\cdots\sum_{s_{zN} = -s}^{s}\int\dl{\rr_N}|(\rr_\mu, s_{z\mu})_{\mu\in X})((\rr_\mu, s_{z\mu})_{\mu\in X}|
  \end{align}
\end{theorem}
\begin{proof}
  \begin{align}
     & \sum_{N = 0}^{\infty}\sum_{s_{z1} = -s}^{s}\int\dl{\rr_1}\cdots\sum_{s_{zN} = -s}^{s}\int\dl{\rr_N}|(\rr_\mu, s_{z\mu})_{\mu\in X})((\rr_\mu, s_{z\mu})_{\mu\in X}|(\rr_\mu', s_{z\mu}')_{\mu\in X'}) \\
     & = \sum_{N = 0}^{\infty}\sum_{s_{z1} = -s}^{s}\int\dl{\rr_1}\cdots\sum_{s_{zN} = -s}^{s}\int\dl{\rr_N}|(\rr_\mu, s_{z\mu})_{\mu\in X})                                                                 \\
     & \quad\ \delta_{NN'}\frac{1}{N!}\sum_{\sigma\in\SS_N}(\pm 1)^\sigma\delta(\rr_1 - \rr_{\sigma(1)}')\delta_{s_{z1}s_{z\sigma(1)}'}\cdots\delta(\rr_N - \rr_{\sigma(N)}')\delta_{s_{zN}s_{z\sigma(N)}'}  \\
     & = \frac{1}{N'!}\sum_{\sigma\in\SS_{N'}}(\pm 1)^\sigma\sum_{s_{z1} = -s}^{s}\int\dl{\rr_1}\cdots\sum_{s_{zN'} = -s}^{s}\int\dl{\rr_{N'}}|(\rr_\mu, s_{z\mu})_{\mu\in X})                               \\
     & \quad\ \delta(\rr_1 - \rr_{\sigma(1)}')\delta_{s_{z1}s_{z\sigma(1)}'}\cdots\delta(\rr_{N'} - \rr_{\sigma(N')}')\delta_{s_{zN'}s_{z\sigma(N')}'}                                                       \\
     & = \frac{1}{N'!}\sum_{\sigma\in\SS_{N'}}(\pm 1)^\sigma|(\rr_{\sigma(\mu)}', s_{z\sigma(\mu)}')_{\mu\in X'})                                                                                            \\
     & = |(\rr_{\mu}', s_{z\mu}')_{\mu\in X'})
  \end{align}
\end{proof}

\begin{theorem}
  \begin{align}
    \hat{\rho}(\rr, s_z)|(\rr_\mu, s_{z\mu})_{\mu\in\ZZ_N}) = \ab(\sum_{\mu\in\ZZ_N}\delta(\rr - \rr_\mu)\delta_{s_zs_{z\mu}})|(\rr_\mu, s_{z\mu})_{\mu\in\ZZ_N})
  \end{align}
\end{theorem}
\begin{proof}
  \begin{align}
    \hat{\rho}(\rr, s_z)\hat{\phi}^\dagger(\rr', s_z') & = \hat{\phi}^\dagger(\rr, s_z)\hat{\phi}(\rr, s_z)\hat{\phi}^\dagger(\rr', s_z')                                                                       \\
                                                       & = \pm\hat{\phi}^\dagger(\rr, s_z)\hat{\phi}^\dagger(\rr', s_z')\hat{\phi}(\rr, s_z) + \delta(\rr - \rr')\delta_{s_zs_{z'}}\hat{\phi}^\dagger(\rr, s_z) \\
                                                       & = \hat{\phi}^\dagger(\rr', s_z')\hat{\phi}^\dagger(\rr, s_z)\hat{\phi}(\rr, s_z) + \delta(\rr - \rr')\delta_{s_zs_{z'}}\hat{\phi}^\dagger(\rr', s_z')  \\
                                                       & = \hat{\phi}^\dagger(\rr', s_z')\hat{\rho}(\rr, s_z) + \delta(\rr - \rr')\delta_{s_zs_{z'}}\hat{\phi}^\dagger(\rr', s_z')
  \end{align}
  \begin{align}
    \hat{\rho}(\rr, s_z)|(\rr_\mu, s_{z\mu})_{\mu\in\ZZ_N}) & = \frac{1}{\sqrt{N!}}\hat{\rho}(\rr, s_z)\hat{\phi}^\dagger(\rr_1, s_{z,1})\cdots\hat{\phi}^\dagger(\rr_N, s_{z,N})\ket|\mathrm{vac}>                                             \\
                                                            & = \frac{1}{\sqrt{N!}}\hat{\phi}^\dagger(\rr_1, s_{z,1})\cdots\hat{\phi}^\dagger(\rr_N, s_{z,N})\hat{\rho}(\rr, s_z)\ket|\mathrm{vac}>                                             \\
                                                            & + \ab(\sum_{\mu\in\ZZ_N}\delta(\rr - \rr_\mu)\delta_{s_zs_{z\mu}})\frac{1}{\sqrt{N!}}\hat{\phi}^\dagger(\rr_1, s_{z,1})\cdots\hat{\phi}^\dagger(\rr_N, s_{z,N})\ket|\mathrm{vac}> \\
                                                            & = \ab(\sum_{\mu\in\ZZ_N}\delta(\rr - \rr_\mu)\delta_{s_zs_{z\mu}})|(\rr_\mu, s_{z\mu})_{\mu\in\ZZ_N})
  \end{align}
\end{proof}

\begin{theorem}
  \begin{align}
    \hat{N}|(\rr_\mu, s_{z\mu})_{\mu\in\ZZ_N}) = N|(\rr_\mu, s_{z\mu})_{\mu\in\ZZ_N})
  \end{align}
\end{theorem}
\begin{proof}
\end{proof}

\begin{theorem}
  \begin{align}
    \hat{f}^{\mathrm{tot}} =
  \end{align}
\end{theorem}
\begin{proof}
  \begin{align}
    \hat{b}_i         & = \sum_{s_z = -s}^s\int\dl{\rr}\phi_i^*(\rr, s_z)\hat{\phi}(\rr, s_z)       \\
    \hat{b}_i^\dagger & = \sum_{s_z = -s}^s\int\dl{\rr}\phi_i(\rr, s_z)\hat{\phi}^\dagger(\rr, s_z)
  \end{align}
  \begin{align}
    \hat{f}^{\mathrm{tot}} & = \frac{1}{n!}\sum_{\substack{i_1,\ldots,i_n\in I                                                                                                                                                                     \\ j_1,\ldots,j_n\in I}}\bra<\phi_{i_1}\cdots\phi_{i_n}|f\ket|\phi_{j_1}\cdots\phi_{j_n}>\hat{b}_{i_1}^\dagger\cdots\hat{b}_{i_n}^\dagger\hat{b}_{j_n}\cdots\hat{b}_{j_1} \\
                           & = \frac{1}{n!}\sum_{s_{z1} = -s}^s\int\dl{\rr_1}\cdots\sum_{s_{zn} = -s}^s\int\dl{\rr_n}\sum_{s_{z1}' = -s}^s\int\dl{\rr_1'}\cdots\sum_{s_{zn}' = -s}^s\int\dl{\rr_n'}                                                \\
                           & \sum_{\substack{i_1,\ldots,i_n\in I                                                                                                                                                                                   \\ j_1,\ldots,j_n\in I}}
    \braket<\phi_{i_1}|\rr_1,s_{z1}>\cdots\braket<\phi_{i_n}|\rr_n,s_{zn}>\braket<\rr_1',s_{z1}'|\phi_{j_1}>\cdots\braket<\rr_n',s_{zn}'|\phi_{j_n}>                                                                                               \\
                           & \bra<\rr_1,s_{z1},\ldots,\rr_n,s_{zn}|f\ket|\rr_1',s_{z1}',\ldots,\rr_n',s_{zn}'>                                                                                                                                     \\
                           & \phi_{i_1}(\rr_1, s_{z1})\hat{\phi}^\dagger(\rr_1, s_{z1})\cdots\phi_{i_n}(\rr_n, s_{zn})\hat{\phi}^\dagger(\rr_n, s_{zn})                                                                                            \\
                           & \phi_{j_n}^*(\rr_n', s_{zn}')\hat{\phi}(\rr_n', s_{zn}')\cdots\phi_{j_1}^*(\rr_1', s_{z1}')\hat{\phi}(\rr_1', s_{z1}')                                                                                                \\
                           & = \frac{1}{n!}\sum_{s_{z1} = -s}^s\int\dl{\rr_1}\cdots\sum_{s_{zn} = -s}^s\int\dl{\rr_n}\sum_{s_{z1}' = -s}^s\int\dl{\rr_1'}\cdots\sum_{s_{zn}' = -s}^s\int\dl{\rr_n'}                                                \\
                           & \hat{\phi}^\dagger(\rr_1, s_{z1})\cdots\hat{\phi}^\dagger(\rr_n, s_{zn})\bra<\rr_1,s_{z1},\ldots,\rr_n,s_{zn}|f\ket|\rr_1',s_{z1}',\ldots,\rr_n',s_{zn}'>\hat{\phi}(\rr_n', s_{zn}')\cdots\hat{\phi}(\rr_1', s_{z1}') \\
                           & = \frac{1}{n!}\sum_{1,\ldots,2n}\hat{\phi}^\dagger(1)\cdots\hat{\phi}^\dagger(n)\bra<1,\ldots,n|f\ket|n+1,\ldots,2n>\hat{\phi}(2n)\cdots\hat{\phi}(n+1)
  \end{align}
\end{proof}

\begin{theorem}

\end{theorem}
\begin{proof}
  \begin{align}
    \hat{h}^{\mathrm{tot}} & = \sum_{s_z = -s}^{s}\sum_{s_z' = -s}^{s}\int\dl{\rr}\hat{\phi}^\dagger(\rr, s_z)\ab[-\frac{\hbar^2}{2m}\delta_{s_zs_z'}\Delta_{\rr} + V_{s_zs_z'}(\rr)]\hat{\phi}(\rr, s_z')                     \\
                           & = \sum_{s_z = -s}^{s}\sum_{s_z' = -s}^{s}\int\dl{\rr}\hat{\phi}^\dagger(\rr, s_z)\ab[-\frac{\hbar^2}{2m}\delta_{s_zs_z'}\Delta_{\rr} + \bra<s_z|V(\rr, \hat{\ss})\ket|s_z'>]\hat{\phi}(\rr, s_z')
  \end{align}
\end{proof}

\begin{align}
  V_{s_zs_z'}(\rr) = \bra<s_z|V(\rr, \hat{\ss})\ket|s_z'> = \bra<s_z'|V(\rr, \hat{\ss})\ket|s_z>^* = V_{s_z's_z}(\rr)^*
\end{align}
\begin{align}
  V_{s_{z1}s_{z2}s_{z1}'s_{z2}'}(\rr_1, \rr_2) & = \bra<s_{z1}s_{z2}|V(\rr_1, \hat{\ss}_1, \rr_2, \hat{\ss}_2)\ket|s_{z1}'s_{z2}'>   \\
                                               & = \bra<s_{z1}'s_{z2}'|V(\rr_1, \hat{\ss}_1, \rr_2, \hat{\ss}_2)\ket|s_{z1}s_{z2}>^* \\
                                               & = V_{s_{z1}'s_{z2}'s_{z1}s_{z2}}(\rr_1, \rr_2)^*
\end{align}

\begin{example}
  \begin{align}
    \hat{h} & = \frac{\hat{\pp}^2}{2m} + V(\hat{\rr}, \hat{\ss}) = -\frac{\hbar^2}{2m}\nabla_{\rr}^2 + V(\hat{\rr}, \hat{\ss})
  \end{align}
  \begin{align}
    \hat{H} & = \sum_{s_z = -s}^{s}\int\dl{\rr}\hat{\phi}^\dagger(\rr, s_z)
  \end{align}
\end{example}

\section{量子化された場の理論は粒子数を固定しない多体系の量子力学に等しい。}
\begin{definition}
  \begin{align}
    \hat{H}                & = \hat{H}_{\mathrm{one}} + \hat{H}_{\mathrm{two}}                                                                                                                                                                                                                                     \\
    \hat{H}_{\mathrm{one}} & = \sum_{s_z = -s}^{s}\sum_{s_z' = -s}^{s}\int\dl{\rr}\hat{\phi}^\dagger(\rr, s_z)\ab[-\frac{\hbar^2}{2m}\delta_{s_zs_z'}\Delta_{\rr} + V_{s_zs_z'}(\rr)]\hat{\phi}(\rr, s_z')                                                                                                         \\
    \hat{H}_{\mathrm{two}} & = \frac{1}{2}\sum_{s_{z1} = -s}^{s}\sum_{s_{z1}' = -s}^{s}\int\dl{\rr_1}\sum_{s_{z2} = -s}^{s}\sum_{s_{z2}' = -s}^{s}\int\dl{\rr_2}\hat{\phi}^\dagger(\rr_1, s_{z1})\hat{\phi}^\dagger(\rr_2, s_{z2})V_{s_{z1}s_{z2}s_{z1}'s_{z2}'}\hat{\phi}(\rr_1, s_{z1})\hat{\phi}(\rr_2, s_{z2})
  \end{align}
  \begin{align}
    i\hbar\diff{}{t}\ket|\Psi(t)> = \hat{H}\ket|\Psi(t)>
  \end{align}
  \begin{align}
    \ket|\Psi(t)> & = \sum_{N=0}^{\infty}\sum_{s_{z1} = -s}^{s}\int\dl{\rr_1}\cdots\sum_{s_{zN} = -s}^{s}\int\dl{\rr_N}\Psi(\rr_1,s_{z1},\ldots,\rr_N,s_{zN};t)|\rr_1,s_{z1},\ldots,\rr_N,s_{zN})
  \end{align}
\end{definition}
\begin{theorem}
  \begin{align}
    \Psi(\rr_{\sigma(1)},s_{z\sigma(1)},\ldots,\rr_{\sigma(N)},s_{z\sigma(N)};t) = (\pm 1)^\sigma\Psi(\rr_1, s_{z1},\ldots,\rr_N,s_{zN}; t)
  \end{align}
\end{theorem}
\begin{proof}

\end{proof}

\end{document}