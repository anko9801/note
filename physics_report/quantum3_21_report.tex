\RequirePackage{plautopatch}
\documentclass[uplatex,dvipdfmx,a4paper,11pt]{jlreq}
\usepackage{bxpapersize}
\usepackage[utf8]{inputenc}
\usepackage{fontenc}
\usepackage{lmodern}
\usepackage{otf}
\usepackage{amsmath}
\usepackage{amssymb}
\usepackage{amsthm}
\usepackage{ascmac}
% \usepackage[hyphens]{url}
\usepackage{physics2}
\usephysicsmodule{ab, ab.braket, doubleprod, diagmat, xmat}
\usepackage{diffcoeff}
% \usepackage{braket}
\usepackage{verbatimbox}
\usepackage{bm}
\usepackage{url}
% \usepackage[dvipdfmx,hiresbb,final]{graphicx}
\usepackage{hyperref}
\usepackage{pxjahyper}
\usepackage{tikz}\usetikzlibrary{cd}
\usepackage{listings}
\usepackage{color}
\usepackage{mathtools}
\usepackage{xspace}
\usepackage{xy}
\usepackage{xypic}
%
\title{量子力学 III \\ 複数の同一粒子からなる量子系:発展編 (第二量子化)}
\author{21B00349 宇佐見大希}
\makeatletter
%
\DeclareMathOperator{\lcm}{lcm}
\DeclareMathOperator{\Kernel}{Ker}
\DeclareMathOperator{\Image}{Im}
\DeclareMathOperator{\ch}{ch}
\DeclareMathOperator{\Aut}{Aut}
\DeclareMathOperator{\Log}{Log}
\DeclareMathOperator{\Arg}{Arg}
\DeclareMathOperator{\sgn}{sgn}
\DeclareMathOperator{\Span}{span}
\DeclareMathOperator{\per}{per}
%
\newcommand{\CC}{\mathbb{C}}
\newcommand{\RR}{\mathbb{R}}
\newcommand{\QQ}{\mathbb{Q}}
\newcommand{\ZZ}{\mathbb{Z}}
\newcommand{\NN}{\mathbb{N}}
\newcommand{\HH}{\mathcal{H}}
\renewcommand{\SS}{\mathfrak{S}}
\newcommand{\R}{\bm{R}}
\renewcommand{\aa}{\bm{a}}
\newcommand{\bb}{\bm{b}}
\renewcommand{\S}{\mathcal{S}}
\newcommand{\A}{\mathcal{A}}
\newcommand{\rr}{\bm{r}}
\newcommand{\kk}{\bm{k}}
\newcommand{\pp}{\bm{p}}
\newcommand{\calB}{\mathcal{B}}
\newcommand{\calF}{\mathcal{F}}
\newcommand{\ignore}[1]{}
\newcommand{\floor}[1]{\left\lfloor #1 \right\rfloor}
% \newcommand{\abs}[1]{\left\lvert #1 \right\rvert}
\newcommand{\lt}{<}
\newcommand{\gt}{>}
\newcommand{\id}{\mathrm{id}}
\newcommand{\rot}{\curl}
\newcommand{\vnabla}{\mathbf{\nabla}}
\newcommand{\laplacian}{\nabla^2}
\renewcommand{\angle}[1]{\left\langle #1 \right\rangle}
\newcommand\mqty[1]{\begin{pmatrix}#1\end{pmatrix}}
\newcommand\vmqty[1]{\begin{vmatrix}#1\end{vmatrix}}
\numberwithin{equation}{section}

\let\oldcite=\cite
\renewcommand\cite[1]{\hyperlink{#1}{\oldcite{#1}}}

\let\oldbibitem=\bibitem
\renewcommand{\bibitem}[2][]{\label{#2}\oldbibitem[#1]{#2}}

% theorem環境の設定
% - 冒頭に改行
% - 末尾にdiamond (amsthm)
\theoremstyle{definition}
\newcommand*{\newscreentheoremx}[2]{
  \newenvironment{#1}[1][]{
    \begin{screen}
    \begin{#2}[##1]
      \leavevmode
      \newline
  }{
    \end{#2}
    \end{screen}
  }
}
\newcommand*{\newqedtheoremx}[2]{
  \newenvironment{#1}[1][]{
    \begin{#2}[##1]
      \leavevmode
      \newline
      \renewcommand{\qedsymbol}{\(\diamond\)}
      \pushQED{\qed}
  }{
      \qedhere
      \popQED
    \end{#2}
  }
}
\newtheorem{theorem*}{定理}[section]

\newqedtheoremx{theorem}{theorem*}
\newcommand*\newqedtheorem@unstarred[2]{%
  \newtheorem{#1*}[theorem*]{#2}
  \newqedtheoremx{#1}{#1*}
}
\newcommand*\newqedtheorem@starred[2]{%
  \newtheorem*{#1*}{#2}
  \newqedtheoremx{#1}{#1*}
}
\newcommand*{\newqedtheorem}{\@ifstar{\newqedtheorem@starred}{\newqedtheorem@unstarred}}

\newtheorem{sctheorem*}{定理}[section]
\newscreentheoremx{sctheorem}{sctheorem*}
\newcommand*\newscreentheorem@unstarred[2]{%
  \newtheorem{#1*}[theorem*]{#2}
  \newscreentheoremx{#1}{#1*}
}
\newcommand*\newscreentheorem@starred[2]{%
  \newtheorem*{#1*}{#2}
  \newscreentheoremx{#1}{#1*}
}
\newcommand*{\newscreentheorem}{\@ifstar{\newscreentheorem@starred}{\newscreentheorem@unstarred}}

%\newtheorem*{definition}{定義}
%\newtheorem{theorem}{定理}
%\newtheorem{proposition}[theorem]{命題}
%\newtheorem{lemma}[theorem]{補題}
%\newtheorem{corollary}[theorem]{系}

\newqedtheorem{lemma}{補題}
\newqedtheorem{corollary}{系}
\newqedtheorem{example}{例}
\newqedtheorem{proposition}{命題}
\newqedtheorem{remark}{注意}
\newqedtheorem{thesis}{主張}
\newqedtheorem{notation}{記法}
\newqedtheorem{problem}{問題}
\newqedtheorem{algorithm}{アルゴリズム}

\newscreentheorem*{axiom}{公理}
\newscreentheorem*{definition}{定義}

\renewenvironment{proof}[1][\proofname]{\par
  \normalfont
  \topsep6\p@\@plus6\p@ \trivlist
  \item[\hskip\labelsep{\bfseries #1}\@addpunct{\bfseries}]\ignorespaces\quad\par
}{%
  \qed\endtrivlist\@endpefalse
}
\renewcommand\proofname{証明}

\makeatother

\begin{document}
\maketitle
\tableofcontents
\clearpage

すみません, 自分にとって分かりやすいように一部文節構造を崩してしまいました. 分かりにくいとは思いますが, よろしくお願いします.

\section{もし、量子状態の対称化の要請がなかったら?}
\begin{definition}
  1 粒子状態の Hilbert 空間 $\HH_{single}$ に対して $N$ 個の粒子の粒子状態の Hilbert 空間はテンソル積 $\HH^{(N)}\cong\HH_{single}\otimes\cdots\otimes\HH_{single}$ で表現される.
\end{definition}

\begin{theorem}[Q21-1(i)(viii)]
  異なる 1 粒子状態 $\ket|\alpha>,\ket|\beta>\in\HH_{single}$ を持つ粒子による 2 つの粒子系 $\HH^{(2)} \cong \HH_{single}\otimes\HH_{single}$ において次の 4 つを仮定する.
  \begin{enumerate}
    \item 2 つの粒子は区別できない.
    \item Hilbert 空間 $\HH^{(2)}$ において交換演算子 $\hat{E}$ は既約元である.
    \item 粒子の順序に依存する観測量が存在する.
    \item 粒子の 1 個が $\ket|\alpha>\in\HH_{single}$ となり, もう 1 個は $\ket|\beta>\in\HH_{single}$ となる. (これを仮定 $D$ とおく)
  \end{enumerate}
  このとき粒子状態 $\ket|\Psi>\in\HH^{(2)}$ は次のようになる.
  \begin{align}
    \ket|\Psi> & = \frac{1}{\sqrt{2}}\ket|\alpha>\ket|\beta> \pm \frac{1}{\sqrt{2}}\ket|\beta>\ket|\alpha>
  \end{align}
\end{theorem}
\begin{proof}
  2 つの粒子が区別できないことはいかなる観測量の期待値は粒子交換に関して不変であると言える. 粒子交換は既約であるから分解できず, 演算子と状態のどちらかが交換に関して不変量となるが, 観測量の演算子に粒子の順序に依存するものがある為, 状態について粒子交換が成り立つと考えられる.

  また $\HH^{(2)}$ の粒子状態について $\ket|\alpha>\ket|\beta>$ と $\ket|\beta>\ket|\alpha>$ の重ね合わせにより表現でき, 規格化条件から次のように書ける.
  \begin{align}
    \ket|\Psi> & = c_1\ket|\alpha>\ket|\beta> + c_2\ket|\beta>\ket|\alpha> \qquad (c_1, c_2\in\CC) \\
    |c_1|^2    & + |c_2|^2 = 1
  \end{align}
  粒子を交換しても状態が不変であるから次のようになる.
  \begin{align}
    \ket|\Psi> & = c_1\ket|\alpha>\ket|\beta> + c_2\ket|\beta>\ket|\alpha> = c_2\ket|\alpha>\ket|\beta> + c_1\ket|\beta>\ket|\alpha>
  \end{align}
  これより位相を考慮して係数について次のような関係が成り立つ.
  \begin{align}
    c_1 = \pm c_2
  \end{align}
  よって粒子状態は次のようになる.
  \begin{align}
    \ket|\Psi> & = \frac{1}{\sqrt{2}}(\ket|\alpha>\ket|\beta> \pm \ket|\beta>\ket|\alpha>)
  \end{align}
  また逆は自明である.
\end{proof}

\begin{proposition}[Q21-1(ii)]
  粒子状態の基底を次のように定義する.
  \begin{align}
    \begin{dcases}
      \ket|\Psi_S> = \frac{1}{\sqrt{2}}(\ket|\alpha>\ket|\beta> + \ket|\beta>\ket|\alpha>) \\
      \ket|\Psi_A> = \frac{1}{\sqrt{2}}(\ket|\alpha>\ket|\beta> - \ket|\beta>\ket|\alpha>)
    \end{dcases}
  \end{align}
  このとき $D$ を満たす任意の粒子状態 $\ket|\Psi>\in\HH^{(2)}$ は次のように表現される.
  \begin{align}
    \ket|\Psi> = c_S\ket|\Psi_S> + c_A\ket|\Psi_A>
  \end{align}
\end{proposition}
\begin{proof}
  まず十分性について $D$ を満たす任意の粒子状態 $\ket|\Psi>\in\HH^{(2)}$ は $\ket|\alpha>\ket|\beta>, \ket|\beta>\ket|\alpha>$ の重ね合わせにより表現できる. これより
  \begin{align}
    \ket|\Psi> & = c_1\ket|\alpha>\ket|\beta> + c_2\ket|\beta>\ket|\alpha>                                                                                         \\
               & = \frac{c_1 + c_2}{2}(\ket|\alpha>\ket|\beta> + \ket|\beta>\ket|\alpha>) + \frac{c_1 - c_2}{2}(\ket|\alpha>\ket|\beta> - \ket|\beta>\ket|\alpha>) \\
               & = \frac{c_1 + c_2}{\sqrt{2}}\ket|\Psi_S> + \frac{c_1 - c_2}{\sqrt{2}}\ket|\Psi_A>.
  \end{align}
  であり, 次のようにおくことで $\ket|\Psi>$ は $\ket|\Psi_S>, \ket|\Psi_A>$ の重ね合わせとして表現できる.
  \begin{align}
    c_S & = \frac{c_1 + c_2}{\sqrt{2}}, \quad c_A = \frac{c_1 - c_2}{\sqrt{2}}
  \end{align}
  逆に必要性について任意の係数 $c_S, c_A$ について $\ket|\alpha>\ket|\beta>, \ket|\beta>\ket|\alpha>$ の重ね合わせで表現できることは次のように分かる.
  \begin{align}
    \ket|\Psi> & = c_S\ket|\Psi_S> + c_A\ket|\Psi_A>                                                                                                                 \\
               & = \frac{c_S}{\sqrt{2}}(\ket|\alpha>\ket|\beta> + \ket|\beta>\ket|\alpha>) + \frac{c_A}{\sqrt{2}}(\ket|\alpha>\ket|\beta> - \ket|\beta>\ket|\alpha>) \\
               & = \frac{c_S + c_A}{\sqrt{2}}\ket|\alpha>\ket|\beta> + \frac{c_S - c_A}{\sqrt{2}}\ket|\beta>\ket|\alpha>
  \end{align}
  よって同値な表現であることが示された.
\end{proof}

\begin{proposition}[Q21-1(iii)(iv)(v)]
  交換演算子 (exchange operator) $\hat{E}$ を次のように定義する.
  \begin{align}
    \hat{E}\ket|\psi>\ket|\psi'> & = \ket|\psi'>\ket|\psi>
  \end{align}
  このとき次の性質が認められる.
  \begin{align}
    \hat{E}           & = \hat{E}^\dagger = \hat{E}^{-1}, \quad \hat{E}^2 = \hat{1} \\
    \hat{E}\ket|\Psi> & = c_S\ket|\Psi_S> - c_A\ket|\Psi_A>
  \end{align}
\end{proposition}
\begin{proof}
  まず粒子状態 $\ket|\psi>\ket|\psi'> = \ket|\alpha>\ket|\beta>, \ket|\beta>\ket|\alpha>$ に対して演算子 $\hat{E}^{-1}, \hat{E}^\dagger$ を適用する.
  \begin{align}
    \hat{E}^{-1}\ket|\psi>\ket|\psi'>                                & = \hat{E}^{-1}\hat{E}\ket|\psi'>\ket|\psi> = \ket|\psi'>\ket|\psi>                    \\
    \bra<\psi|\bra<\psi'|\hat{E}^\dagger\hat{E}\ket|\psi>\ket|\psi'> & = \bra<\psi'|\braket<\psi|\psi'>\ket|\psi> = \bra<\psi|\braket<\psi'|\psi>\ket|\psi'>
  \end{align}
  これより次のことが分かる.
  \begin{align}
    \hat{E} & = \hat{E}^\dagger = \hat{E}^{-1}
  \end{align}
  また 2 回適用すると恒等演算子となる.
  \begin{align}
    \hat{E}^2 & = \hat{E}\hat{E}^{-1} = \hat{1}
  \end{align}
  次に基底 $\ket|\Psi_S>, \ket|\Psi_A>$ に適用する.
  \begin{align}
    \hat{E}\ket|\Psi_S> & = \frac{1}{\sqrt{2}}\hat{E}(\ket|\alpha>\ket|\beta> + \ket|\beta>\ket|\alpha>) = +\frac{1}{\sqrt{2}}(\ket|\alpha>\ket|\beta> + \ket|\beta>\ket|\alpha>) = +\ket|\Psi_S> \\
    \hat{E}\ket|\Psi_A> & = \frac{1}{\sqrt{2}}\hat{E}(\ket|\alpha>\ket|\beta> - \ket|\beta>\ket|\alpha>) = -\frac{1}{\sqrt{2}}(\ket|\alpha>\ket|\beta> - \ket|\beta>\ket|\alpha>) = -\ket|\Psi_A>
  \end{align}
  よって任意の状態 $\ket|\Psi>$ に適用すると次のようになる.
  \begin{align}
    \hat{E}\ket|\Psi> & = \hat{E}(c_S\ket|\Psi_S> + c_A\ket|\Psi_A>) = c_S\ket|\Psi_S> - c_A\ket|\Psi_A>
  \end{align}
\end{proof}

\begin{proposition}[Q21-1(vi)(vii)]
  2 つの粒子は区別できない, つまり Hilbert 空間 $\HH^{(2)}$ の任意の観測量 $\hat{O}$ は粒子交換に関して不変であることは $\hat{O}, \hat{E}$ が可換であることと同値である.
\end{proposition}
\begin{proof}
  観測量 $\hat{O}$ に対して演算子 $\hat{E}\hat{O}\hat{E}$ は粒子交換した後に観測量を適用して戻したものであり, 次のように Hermite であるから観測量を粒子交換した結果に等しい.
  \begin{align}
    (\hat{E}\hat{O}\hat{E})^\dagger = \hat{E}^\dagger\hat{O}^\dagger\hat{E}^\dagger = \hat{E}\hat{O}\hat{E}.
  \end{align}
  この観測量 $\hat{O}$ が粒子交換した観測量 $\hat{E}\hat{O}\hat{E}$ でも不変であるから $\hat{E}\hat{O}\hat{E} = \hat{O}$ である. これより次のように計算できる.
  \begin{align}
    \hat{O}\hat{E}     & = \hat{E}^{-1}\hat{E}\hat{O}\hat{E} = \hat{E}^{-1}\hat{O} = \hat{E}\hat{O} \\
    [\hat{O}, \hat{E}] & = \hat{O}\hat{E} - \hat{E}\hat{O} = 0.
  \end{align}
  よって $\hat{O}, \hat{E}$ は可換である.
\end{proof}

\begin{proposition}[Q21-1(ix)]
  観測量 $\hat{O}$ の期待値 $\langle\hat{O}\rangle$ について次のように書ける.
  \begin{align}
    \langle\hat{O}\rangle & = |c_S|^2\bra<\Psi_S|\hat{O}\ket|\Psi_S> + |c_A|^2\bra<\Psi_A|\hat{O}\ket|\Psi_A>
  \end{align}
\end{proposition}
\begin{proof}
  観測量 $\hat{O}$ の期待値 $\langle\hat{O}\rangle$ は次のように計算できる.
  \begin{align}
    \langle\hat{O}\rangle & = \bra<\Psi|\hat{O}\ket|\Psi>                                                                                                                                         \\
                          & = (c_S^*\bra<\Psi_S| + c_A^*\bra<\Psi_A|)\hat{O}(c_S\ket|\Psi_S> + c_A\ket|\Psi_A>)                                                                                   \\
                          & = |c_S|^2\bra<\Psi_S|\hat{O}\ket|\Psi_S> + |c_A|^2\bra<\Psi_A|\hat{O}\ket|\Psi_A> + c_S^*c_A\bra<\Psi_S|\hat{O}\ket|\Psi_A> + c_A^*c_S\bra<\Psi_A|\hat{O}\ket|\Psi_S> \\
                          & = |c_S|^2\bra<\Psi_S|\hat{O}\ket|\Psi_S> + |c_A|^2\bra<\Psi_A|\hat{O}\ket|\Psi_A>.\label{O expected}
  \end{align}
  ただし式 \eqref{O expected} において次のような計算をした.
  \begin{align}
    \bra<\Psi_S|\hat{O}\ket|\Psi_A> & = \bra<\Psi_S|\hat{E}\hat{O}\hat{E}\ket|\Psi_A> = -\bra<\Psi_S|\hat{O}\ket|\Psi_A> = 0  \\
    \bra<\Psi_A|\hat{O}\ket|\Psi_S> & = \bra<\Psi_A|\hat{E}\hat{O}\hat{E}\ket|\Psi_S> = -\bra<\Psi_A|\hat{O}\ket|\Psi_S> = 0.
  \end{align}
\end{proof}
例えば $\hat{O} = 2\ket|\beta>\ket|\alpha>\bra<\alpha|\bra<\beta|$ とすると
\begin{align}
  \bra<\Psi_S|\hat{O}\ket|\Psi_S> & = \frac{1}{2}(\bra<\alpha|\bra<\beta| + \bra<\beta|\bra<\alpha|)\hat{O}(\ket|\alpha>\ket|\beta> + \ket|\beta>\ket|\alpha>) = +1 \\
  \bra<\Psi_A|\hat{O}\ket|\Psi_A> & = \frac{1}{2}(\bra<\alpha|\bra<\beta| - \bra<\beta|\bra<\alpha|)\hat{O}(\ket|\alpha>\ket|\beta> - \ket|\beta>\ket|\alpha>) = -1
\end{align}
より期待値は $c_S, c_A$ に依存する. (Q21-1(x))

Q21-1(xi) についてはよく分からなかった. 観測量に対応する演算子がすべて粒子交換に関して不変ならば理論の整合性は保っており, 問題ないと思われる. 逆に粒子交換によって変化する観測量が存在することを実験によって確かめられれば, 状態に関して条件を足すべきである為, 予言能力が不足していることになる.

\section{$n$ 次対称群 $\SS_n$}
\begin{definition}[$n$ 次対称群]
  $X$ を集合とするとき $X$ から $X$ への全単射写像 $\sigma: X\to X$ を $X$ の置換という.
  $\sigma,\tau$ を置換とするとき, その積 $\sigma\tau$ を写像としての合成 $\sigma\circ\tau$ と定義する. $X$ の置換全体の集合はこの演算により群となり, これを $X$ の置換群という.
  $X = \{1,2,\ldots,n\}$ のとき $X$ の置換群を $n$ 次対称群といい $\SS_n$ と書く.
\end{definition}
繰り返すが置換の積は写像の合成であり写像は右結合である. (Q21-2(i))
\begin{problem}[Q21-2(ii)]
$X = \{0, 1, 2, 3\}$ の置換群 $G$ に対して $\sigma,\tau\in G$ の積 $\sigma\tau$ を計算せよ.
\begin{align}
  \sigma = \begin{pmatrix}
             0 & 1 & 2 & 3 \\
             3 & 2 & 0 & 1
           \end{pmatrix} \quad , \quad
  \tau = \begin{pmatrix}
           0 & 1 & 2 & 3 \\
           3 & 2 & 1 & 0
         \end{pmatrix}.
\end{align}
\end{problem}
\begin{proof}
  \begin{align}
    \sigma\tau =
    \begin{pmatrix}
      0 & 1 & 2 & 3 \\
      3 & 2 & 0 & 1
    \end{pmatrix}
    \begin{pmatrix}
      0 & 1 & 2 & 3 \\
      3 & 2 & 1 & 0
    \end{pmatrix}
    =
    \begin{pmatrix}
      0 & 1 & 2 & 3 \\
      1 & 0 & 2 & 3
    \end{pmatrix}
  \end{align}
\end{proof}

\begin{theorem}[Q21-3]
  $n$ 次対称群 $\SS_n$ は群である.
\end{theorem}
\begin{proof}
  $\sigma,\tau\in\SS_n$ に対して $\sigma\tau = \sigma\circ\tau$ が全単射写像であることを示す.
  まず $\sigma\tau$ の全射性について $\sigma$ の全射性より任意の $c\in X$ に対して $\sigma(b) = c$ となる $b\in X$ があり, $\tau(a) = b$ となる $a\in X$ がある.
  これより任意の $c$ に対して次を満たす $a$ がある.
  \begin{align}
    \sigma\tau(a) = \sigma\circ\tau(a) = \sigma(\tau(a)) = c.
  \end{align}
  また $\sigma\tau$ の単射性についてはそれぞれの単射性より次のように満たされる.
  \begin{align}
    \sigma\tau(a) = \sigma\tau(b) \implies \tau(a) = \tau(b) \implies a = b.
  \end{align}
  これより積について閉じていることが分かる.

  単位元は $X$ の恒等写像 $\id_X$ とすることで任意の $\sigma\in\SS_n$ に対して $\sigma\id_X = \id_X\sigma = \sigma$ を満たす.

  また任意の元 $\sigma\in\SS_n$ に対する逆元は逆像 $\sigma^{-1}$ とすることで $\sigma\sigma^{-1} = \id_X$ を満たす.

  そして定義から結合法則 $\sigma_1(\sigma_2\sigma_3) = (\sigma_1\sigma_2)\sigma_3$ も満たすことが分かる.

  よって $n$ 次対称群 $\SS_n$ は群となる.
\end{proof}

\begin{proposition}[Q21-4]
  $n$ 次対称群 $\SS_n$ の位数は $n!$ である.
\end{proposition}
\begin{proof}
  全単射写像は $X$ の順列で被覆できるから位数は $n!$ となる.
\end{proof}

\begin{proposition}[Q21-5, Q21-6]
  $\sigma_0\in\SS_n$ とすると $\sigma_0\SS_n = \SS_n\sigma_0 = \SS_n^{-1} = \SS_n$ である.
\end{proposition}
\begin{proof}
  $\sigma_0$ を左から掛けることに対して $\sigma_0^{-1}$ を左から掛けることは逆写像となるから, 全単射となる. よって $\sigma_0\SS_n = \SS_n$ となる. 逆も同様なので $\SS_n\sigma_0 = \SS_n$ となる.
  また群の性質より各元の逆元は唯一であるから $\SS_n^{-1} = \SS_n$ となる.
\end{proof}

\begin{definition}[互換, 巡回置換]
  置換 $\sigma\in\SS_n$ に対して $1\leq i < j\leq n$ のとき $k \neq i,j$ なら $\sigma(k) = k$ で $\sigma(i) = j$, $\sigma(j) = i$ であるとき $\sigma$ を互換といい $(i\ j)$ と書く.

  より一般に $i_1\mapsto i_2\mapsto\cdots\mapsto i_m\mapsto i_1$ と移し, 他の元は変えない置換を巡回置換といい $(i_1\ \cdots\ i_m)$ と書く.
\end{definition}

\begin{lemma}
  任意の置換は一意の巡回置換の積で表現できる.
\end{lemma}
\begin{proof}
  置換 $\sigma\in\SS_n$ においてある元 $i_1\in X$ を選び, 移していくと鳩ノ巣原理より必ず $i_1\mapsto i_2\mapsto\cdots\mapsto i_m\mapsto i_1$ と巡回する. これより巡回置換 $(i_1\ \cdots\ i_m)$ と $i_1,\ldots,i_m$ を変えず他の元を $i\mapsto\sigma(i)$ とする置換 $\sigma'$ を用いて $\sigma = (i_1\ \cdots\ i_m)\sigma'$ と表現できる.

  次に $\sigma'$ に対しては $i_1,\ldots,i_m$ ではない元を選び同様の操作を行う. これを帰納的に行うことで巡回置換の積で表せられ, 積の順番を除いて一意に定まることが分かる.
\end{proof}

\begin{theorem}[Q21-7(i)]
  任意の置換は互換の積で表現できる.
\end{theorem}
\begin{proof}
  任意の置換は巡回置換の積で表現できるから, 巡回置換が互換の積で表せられることを示せればよい.
  \begin{align}
    (i_1\ i_2\ \cdots\ i_m) & = (i_1\ i_3\ \cdots\ i_m)(i_1\ i_2)                   \\
                            & = (i_1\ i_4\ \cdots\ i_m)(i_1\ i_3)(i_1\ i_2)         \\
                            & = (i_1\ i_m)(i_1\ i_{m-1})\cdots(i_1\ i_3)(i_1\ i_2).
  \end{align}
  これは上のように変形することにより示される.
\end{proof}

\begin{definition}[符号]
  置換 $\sigma\in\SS_n$ の符号 $\sgn\sigma = (-1)^\sigma$ を次のように定義する.
  \begin{align}
    \sgn\sigma = (-1)^\sigma = \begin{cases}
                                 +1 & (\sigma が偶数個の互換の積で表される) \\
                                 -1 & (\sigma が奇数個の互換の積で表される)
                               \end{cases}.
  \end{align}
\end{definition}

\begin{proposition}[Q21-7(ii)]
  置換の符号は well-defined である.
\end{proposition}
\begin{proof}
  次のように定義される差積 $\Delta(x_1,\ldots,x_n)$ を置換 $\sigma\in\SS_n$ 用いて変数の添字を置換することを考える.
  \begin{align}
    \Delta(x_1,\ldots,x_n) = \prod_{1\leq i<j\leq n}(x_j - x_i).
  \end{align}
  互換 $\sigma = (i\ j)$ で置換するとそれぞれ次のようになるから $\Delta(x_{\sigma(1)},\ldots,x_{\sigma(n)}) = -\Delta(x_1,\ldots,x_n)$ となる.
  \begin{align}
    (x_j - x_i)            & \mapsto -(x_j - x_i)            \\
    (x_a - x_i)(x_a - x_j) & \mapsto (x_a - x_i)(x_a - x_j)  \\
    (x_i - x_a)(x_a - x_j) & \mapsto (x_i - x_a)(x_a - x_j)  \\
    (x_i - x_a)(x_j - x_a) & \mapsto (x_i - x_a)(x_j - x_a).
  \end{align}
  これより置換 $\sigma\in\SS_n$ が異なる互換の積 $\sigma = \sigma_1\cdots\sigma_k = \tau_1\cdots\tau_m$ で表されたとき
  \begin{align}
    \Delta(x_{\sigma(1)},\ldots,x_{\sigma(n)}) = (-1)^k\Delta(x_1,\ldots,x_n) = (-1)^m\Delta(x_1,\ldots,x_n)
  \end{align}
  となる為, 互換の積の個数の偶奇は一致する.
\end{proof}

\begin{proposition}
  置換の符号の性質として次を満たす.
  \begin{align}
    \sgn(\sigma\tau)  & = \sgn(\sigma)\sgn(\tau) \\
    \sgn(\id_X)       & = +1                     \\
    \sgn(\sigma^{-1}) & = \sgn(\sigma).
  \end{align}
\end{proposition}
\begin{proof}
  上 2 つは差積に適用することで確かめる.
  \begin{align}
    \Delta(x_{\sigma\tau(1)},\ldots,x_{\sigma\tau(n)}) & = \sgn(\sigma)\Delta(x_{\tau(1)},\ldots,x_{\tau(n)}) = \sgn(\sigma)\sgn(\tau)\Delta(x_1,\ldots,x_n) \\
                                                       & = \sgn(\sigma\tau)\Delta(x_1,\ldots,x_n)                                                            \\
    \Delta(x_{\id_X(1)},\ldots,x_{\id_X(n)})           & = \Delta(x_1,\ldots,x_n) = \sgn(\id_X)\Delta(x_1,\ldots,x_n).
  \end{align}
  より $\sgn(\sigma\tau) = \sgn(\sigma)\sgn(\tau), \sgn(\id_X) = 1$ となる. また $\sgn(\sigma\tau) = \sgn(\sigma)\sgn(\tau)$ より
  \begin{align}
    \sgn(\sigma)\sgn(\sigma^{-1}) & = \sgn(\id_X) = 1.
  \end{align}
  であるから $\sgn(\sigma^{-1}) = \sgn(\sigma)$ となる.
\end{proof}

\section{完全対称な状態と完全反対称な状態の数学的取り扱い}
\begin{definition}
  $N$ 個の同一の粒子 $X_1,\ldots,X_N$ からなる全体系の Hilbert 空間 $\HH^{(N)}\cong\HH_{single}\otimes\cdots\otimes\HH_{single}$ において置換 $\sigma\in\SS_N$ として演算子 $\hat{P}(\sigma): \HH^{(N)}\to\HH^{(N)}$ を状態に対して粒子 $X_i$ を粒子 $X_{\sigma(i)}$ に置き換えて得られる状態とする.
\end{definition}

具体的に状態 $\ket|\Psi>\in\HH^{(N)}$ に適用すると粒子 $X_i$ における状態は元々 $X_{\sigma^{-1}(i)}$ であるから次のようになる. (Q21-9, Q21-10(i))
\begin{align}
  \ket|\Psi>                                                       & = \sum_{i}c^{(i)}\ket|\psi_1^{(i)}>\ket|\psi_2^{(i)}>\cdots\ket|\psi_N^{(i)}>                                                                                     \\
  \hat{P}(\sigma)\ket|\Psi>                                        & = \sum_{i}c^{(i)}\ket|\psi_{\sigma^{-1}(1)}^{(i)}>\ket|\psi_{\sigma^{-1}(2)}^{(i)}>\cdots\ket|\psi_{\sigma^{-1}(N)}^{(i)}>                                        \\
  \bra<\xi_1|\bra<\xi_2|\cdots\bra<\xi_N|\hat{P}(\sigma)\ket|\Psi> & = \sum_{i}c^{(i)}\bra<\xi_1|\bra<\xi_2|\cdots\bra<\xi_N|\ket|\psi_{\sigma^{-1}(1)}^{(i)}>\ket|\psi_{\sigma^{-1}(2)}^{(i)}>\cdots\ket|\psi_{\sigma^{-1}(N)}^{(i)}> \\
                                                                   & = \bra<\xi_{\sigma(1)}|\bra<\xi_{\sigma(2)}|\cdots\braket<\xi_{\sigma(N)}|\Psi>.
\end{align}
これは次のように波動関数表示で書けば粒子 $X_i$ の状態を粒子 $X_{\sigma(i)}$ の状態に置き換えていると解釈できる. (Q21-10(ii))
\begin{align}
  (\hat{P}(\sigma)\Psi)(\xi_1,\xi_2,\ldots,\xi_N) = \Psi(\xi_{\sigma(1)},\xi_{\sigma(2)},\ldots,\xi_{\sigma(N)})
\end{align}

\begin{theorem}[Q21-11]
  $\hat{P}(\sigma)$ は unitary な準同型である.
\end{theorem}
\begin{proof}
  まず unitary 演算子であることは次のようにして成り立つ.
  \begin{align}
    \hat{P}(\sigma)^\dagger\hat{P}(\sigma)\ket|\Psi> & = \ket|\psi_{\sigma\sigma^{-1}(1)}>\cdots\ket|\psi_{\sigma\sigma^{-1}(N)}> = \ket|\Psi>  \\
    \hat{P}(\sigma)\hat{P}(\sigma)^\dagger\ket|\Psi> & = \ket|\psi_{\sigma^{-1}\sigma(1)}>\cdots\ket|\psi_{\sigma^{-1}\sigma(N)}> = \ket|\Psi>.
  \end{align}
  そして準同型であることは次のようにして成り立つ.
  \begin{align}
    \hat{P}(\sigma\tau)\ket|\Psi> & = \ket|\psi_{(\sigma\tau)^{-1}(1)}>\cdots\ket|\psi_{(\sigma\tau)^{-1}(N)}> \\
                                  & = \hat{P}(\sigma)\ket|\psi_{\tau^{-1}(1)}>\cdots\ket|\psi_{\tau^{-1}(N)}>  \\
                                  & = \hat{P}(\sigma)\hat{P}(\tau)\ket|\Psi>.
  \end{align}
  よって $\hat{P}(\sigma)$ は unitary な準同型である. 準同型の性質より
  \begin{align}
    \hat{P}(\id_X)       & = \hat{1}              \\
    \hat{P}(\sigma^{-1}) & = \hat{P}(\sigma)^{-1}
  \end{align}
  となる.
\end{proof}

\begin{definition}[完全対称, 非完全対称]
  Hilbert 空間の状態 $\ket|\Psi>\in\HH^{(N)}$ において任意の置換 $\sigma\in\SS_N$ に対して $\hat{P}(\sigma)\ket|\Psi> = \ket|\Psi>$ となるとき完全対称, $\hat{P}(\sigma)\ket|\Psi> = \sgn(\sigma)\ket|\Psi>$ となるとき反完全対称であると定義する.
  そして完全対称, 反完全対称な状態のなす Hilbert 空間を $\HH_S^{(N)}, \HH_A^{(N)}$ と書き, 全 Hilbert 空間 $\HH^{(N)}$ から $\HH_S^{(N)}, \HH_A^{(N)}$ への射影演算子を $\hat{\S}^{(N)}, \hat{\A}^{(N)}$ とする.
\end{definition}

\begin{lemma}[Q21-12]
  任意の互換 $\sigma\in\SS_N$ に対して $\hat{P}(\sigma)\ket|\Psi> = \ket|\Psi>$, $\hat{P}(\sigma)\ket|\Psi> = -\ket|\Psi>$ となることは完全対称, 反完全対称であることと同値である.
\end{lemma}
\begin{proof}
  任意の置換 $\sigma\in\SS_N$ は互換の積で表現できるから互換 $\sigma_1,\ldots,\sigma_m\in\SS_N$ を用いて $\sigma = \sigma_1\cdots\sigma_m$ と書け, 次のようになる.
  \begin{align}
    \hat{P}(\sigma)\ket|\Psi> & = \ket|\Psi> = (+1)^m\ket|\Psi>             & (完全対称)  \\
    \hat{P}(\sigma)\ket|\Psi> & = \sgn(\sigma)\ket|\Psi> = (-1)^m\ket|\Psi> & (反完全対称)
  \end{align}
  これより同値であることがわかる.
\end{proof}

\begin{proposition}[Q21-13]
  $\HH_S^{(N)}$ と $\HH_A^{(N)}$ は直交し, その直和について次のようになる.
  \begin{align}
    \begin{cases}
      \HH_S^{(2)}\oplus\HH_A^{(2)} = \HH^{(2)}                      \\
      \HH_S^{(N)}\oplus\HH_A^{(N)} \subsetneq \HH^{(N)} & (N\geq 3)
    \end{cases}
  \end{align}
\end{proposition}
\begin{proof}
  $\ket|\Psi_S>\in\HH_S^{(N)}$, $\ket|\Psi_A>\in\HH_A^{(N)}$ の内積について互換 $\sigma$ の演算子を挿入することで求まる.
  \begin{align}
    \braket<\Psi_S|\Psi_A> & = \bra<\Psi_S|\hat{P}(\sigma)^\dagger\hat{P}(\sigma)\ket|\Psi_A> \\
                           & = -\braket<\Psi_S|\Psi_A> = 0
  \end{align}
  これより $\HH_S^{(N)}$ と $\HH_A^{(N)}$ は直交する. 次に $N = 2$ における $\HH_S^{(N)}$, $\HH_A^{(N)}$ は次のように表現できる.
  \begin{align}
    \sum_i c^{(i)}\ab(\ket|\psi_1^{(i)}>\ket|\psi_2^{(i)}> + \ket|\psi_2^{(i)}>\ket|\psi_1^{(i)}>)\in\HH_S^{(2)} \\
    \sum_i c^{(i)}\ab(\ket|\psi_1^{(i)}>\ket|\psi_2^{(i)}> - \ket|\psi_2^{(i)}>\ket|\psi_1^{(i)}>)\in\HH_A^{(2)}
  \end{align}
  これよりこれらの直和は全空間 $\HH^{(2)}$ を表現できる. $N = 3$ における $\HH_S^{(N)}$, $\HH_A^{(N)}$ の元は例えば次のようになる.
  \begin{align}
    \ket|\psi_1>\ket|\psi_2>\ket|\psi_3> + \ket|\psi_2>\ket|\psi_3>\ket|\psi_1> + \ket|\psi_3>\ket|\psi_1>\ket|\psi_2> + \ket|\psi_1>\ket|\psi_3>\ket|\psi_2> + \ket|\psi_2>\ket|\psi_1>\ket|\psi_3> + \ket|\psi_3>\ket|\psi_2>\ket|\psi_1>\in\HH_S^{(N)} \\
    \ket|\psi_1>\ket|\psi_2>\ket|\psi_3> + \ket|\psi_2>\ket|\psi_3>\ket|\psi_1> + \ket|\psi_3>\ket|\psi_1>\ket|\psi_2> - \ket|\psi_1>\ket|\psi_3>\ket|\psi_2> - \ket|\psi_2>\ket|\psi_1>\ket|\psi_3> - \ket|\psi_3>\ket|\psi_2>\ket|\psi_1>\in\HH_A^{(N)}
  \end{align}
  これよりこれらの直和でも全空間は表現できない. $N > 3$ も同様である.
\end{proof}

\begin{theorem}[Q21-14(i)(ii)(iii)]
  射影演算子 $\hat{\S}^{(N)}, \hat{\A}^{(N)}$ は次のように表現される.
  \begin{align}
    \hat{\S}^{(N)} & = \frac{1}{N!}\sum_{\sigma\in\SS_N}\hat{P}(\sigma)             \\
    \hat{\A}^{(N)} & = \frac{1}{N!}\sum_{\sigma\in\SS_N}\sgn(\sigma)\hat{P}(\sigma)
  \end{align}
\end{theorem}
\begin{proof}
  演算子 $\hat{\S}^{(N)}, \hat{\A}^{(N)}$ と置換演算子 $\hat{P}(\tau)$ を適用すると次のようになる.
  \begin{align}
    \hat{P}(\tau)\hat{\S}^{(N)} & = \frac{1}{N!}\sum_{\sigma\in\SS_N}\hat{P}(\tau\sigma) = \frac{1}{N!}\sum_{\sigma'\in\SS_N}\hat{P}(\sigma') = \hat{\S}^{(N)}                                              \\
    \hat{P}(\tau)\hat{\A}^{(N)} & = \frac{1}{N!}\sum_{\sigma\in\SS_N}\sgn(\sigma)\hat{P}(\tau\sigma) = \sgn(\tau)\frac{1}{N!}\sum_{\sigma'\in\SS_N}\sgn(\sigma')\hat{P}(\sigma') = \sgn(\tau)\hat{\A}^{(N)}
  \end{align}
  これより演算子 $\hat{\S}^{(N)}, \hat{\A}^{(N)}$ によって $\HH_S^{(N)}, \HH_A^{(N)}$ へ移すことが分かる.
  次に置換演算子の unitary 性より
  \begin{align}
    (\hat{\S}^{(N)})^\dagger & = \frac{1}{N!}\sum_{\sigma\in\SS_N}\hat{P}(\sigma)^\dagger = \frac{1}{N!}\sum_{\sigma\in\SS_N}\hat{P}(\sigma^{-1}) = \hat{\S}^{(N)}                         \\
    (\hat{\A}^{(N)})^\dagger & = \frac{1}{N!}\sum_{\sigma\in\SS_N}\sgn(\sigma)\hat{P}(\sigma)^\dagger = \frac{1}{N!}\sum_{\sigma\in\SS_N}\sgn(\sigma)\hat{P}(\sigma^{-1}) = \hat{\A}^{(N)}
  \end{align}
  これより $\hat{\S}^{(N)}, \hat{\A}^{(N)}$ は Hermite 演算子である. また演算子 $\hat{\S}^{(N)}, \hat{\A}^{(N)}$ を 2 回適用してみる.
  \begin{align}
    (\hat{\S}^{(N)})^2 & = \frac{1}{N!^2}\sum_{\sigma\in\SS_N}\sum_{\tau\in\SS_N}\hat{P}(\sigma\tau) = \frac{1}{N!}\sum_{\sigma'\in\SS_N}\hat{P}(\sigma') = \hat{\S}^{(N)}                              \\
    (\hat{\A}^{(N)})^2 & = \frac{1}{N!^2}\sum_{\sigma\in\SS_N}\sum_{\tau\in\SS_N}\sgn(\sigma\tau)\hat{P}(\sigma\tau) = \frac{1}{N!}\sum_{\sigma'\in\SS_N}\sgn(\sigma')\hat{P}(\sigma') = \hat{\A}^{(N)}
  \end{align}
  これより $\hat{\S}^{(N)}, \hat{\A}^{(N)}$ は射影演算子となる.
\end{proof}

\begin{proposition}[Q21-14(iv)(v)]
  射影演算子の積と和について次のようになる.
  \begin{align}
     & \quad \hat{\S}^{(N)}\hat{\A}^{(N)} = \hat{\A}^{(N)}\hat{\S}^{(N)} = 0     \\
     & \begin{cases}
         \hat{\S}^{(2)} + \hat{\A}^{(2)} = \hat{1}_{\HH^{(2)}} \\
         \hat{\S}^{(N)} + \hat{\A}^{(N)} \neq \hat{1}_{\HH^{(N)}} \qquad (N \geq 3)
       \end{cases}
  \end{align}
\end{proposition}
\begin{proof}
  演算子 $\hat{\S}^{(N)}, \hat{\A}^{(N)}$ の積について
  \begin{align}
    \hat{\S}^{(N)}\hat{\A}^{(N)} = \hat{\A}^{(N)}\hat{\S}^{(N)} & = \frac{1}{N!^2}\sum_{\sigma\in\SS_N}\sum_{\tau\in\SS_N}\sgn(\tau)\hat{P}(\sigma\tau)                               \\
                                                                & = \frac{1}{N!}\sum_{\sigma\in\SS_N}\sgn(\sigma)\ab(\frac{1}{N!}\sum_{\sigma'\in\SS_N}\sgn(\sigma')\hat{P}(\sigma')) \\
                                                                & = 0
  \end{align}
  より直交することが分かる. また演算子 $\hat{\S}^{(N)}, \hat{\A}^{(N)}$ の和について
  \begin{align}
    \hat{\S}^{(2)} + \hat{\A}^{(2)} & = \frac{1}{2!}\sum_{\sigma\in\SS_2}\ab(\hat{P}(\sigma) + \sgn(\sigma)\hat{P}(\sigma)) = 1_{\HH^{(2)}}                            \\
    \hat{\S}^{(N)} + \hat{\A}^{(N)} & = \frac{1}{N!}\sum_{\sigma\in\SS_N}\ab(\hat{P}(\sigma) + \sgn(\sigma)\hat{P}(\sigma)) \neq \hat{1}_{\HH^{(N)}} \qquad (N \geq 3)
  \end{align}
  とわかる.
\end{proof}

\begin{theorem}[Q21-15]
  $1\leq\mu<\nu\leq N$ において $\ket|\psi_\mu>$ と $\ket|\psi_\nu>$ が線形従属であるならば $\hat{\A}^{(N)}\ket|\psi_1>\cdots\ket|\psi_N> = 0$ となる.
\end{theorem}
\begin{proof}
  任意の $\sigma\in\SS_n$ に対して $\tau(\mu) = \sigma(\nu)$, $\tau(\nu) = \sigma(\mu)$ であり, その他の元 $1\leq i\leq N$ で $\tau(i) = \sigma(i)$ となる $\tau$ が一意に取れる. $\tau$ は $\sigma$ に対して符号が反転し, $\hat{P}(\sigma)\ket|\Psi> = \hat{P}(\tau)\ket|\Psi>$ となる. よって $\hat{\A}^{(N)}\ket|\psi_1>\cdots\ket|\psi_N> = 0$ となる.
\end{proof}

\begin{lemma}
  Hilbert 空間に演算子 $\hat{\S}^{(N)}, \hat{\A}^{(N)}$ を作用させるとそれぞれの部分空間となる.
  \begin{align}
    \HH_S^{(N)} & = \hat{\S}^{(N)}\HH^{(N)} \\
    \HH_A^{(N)} & = \hat{\A}^{(N)}\HH^{(N)}
  \end{align}
  \label{hilbert corespondence}
\end{lemma}
\begin{proof}
  $\hat{\S}^{(N)}, \hat{\A}^{(N)}$ は $\HH_S^{(N)}, \HH_A^{(N)}$ への射影演算子であるから $\HH_S^{(N)} \supseteq \hat{\S}^{(N)}\HH^{(N)}, \HH_A^{(N)} \supseteq \hat{\A}^{(N)}\HH^{(N)}$ は成り立つ.
  また $\ket|\Psi_S>\in\HH_S^{(N)}, \ket|\Psi_A>\in\HH_A^{(N)}$ について次が成り立つことが分かる.
  \begin{align}
    \ket|\Psi_S> & = \hat{P}(\sigma)\ket|\Psi_S> = \frac{1}{N!}\sum_{\sigma\in\SS_n}\hat{P}(\sigma)\ket|\Psi_S> = \hat{\S}^{(N)}\ket|\Psi_S>                         \\
    \ket|\Psi_A> & = \sgn(\sigma)\hat{P}(\sigma)\ket|\Psi_A> = \frac{1}{N!}\sum_{\sigma\in\SS_n}\sgn(\sigma)\hat{P}(\sigma)\ket|\Psi_A> = \hat{\A}^{(N)}\ket|\Psi_A>
  \end{align}
  これより $\HH_S^{(N)} \subseteq \hat{\S}^{(N)}\HH^{(N)}, \HH_A^{(N)} \subseteq \hat{\A}^{(N)}\HH^{(N)}$ は成り立つ. よってそれぞれ等しいことが分かる.
\end{proof}

\begin{proposition}[Q21-16, Q21-17, Q21-18]
  $\HH_{single}$ の完全正規直交基底を添字集合 $I$ を用いて $\{\ket|\phi_i>\}_{i\in I}$ とする.
  \begin{align}
    \HH_S^{(N)} & = \Span\ab\{\hat{\S}^{(N)}\ket|\phi_{i_1}>\cdots\ket|\phi_{i_N}> \mid (i_1,\ldots,i_N)\in I_S^{(N)} \} \\
    \HH_A^{(N)} & = \Span\ab\{\hat{\A}^{(N)}\ket|\phi_{i_1}>\cdots\ket|\phi_{i_N}> \mid (i_1,\ldots,i_N)\in I_A^{(N)} \}
  \end{align}
  ただし添字集合 $I_S^{(N)}, I_A^{(N)}$ は次のように定義される.
  \begin{align}
    I_S^{(N)} & = \{(i_1,\ldots,i_N)\mid i_1,\ldots,i_N\in I\land i_1\leq\cdots\leq i_{N}\} \\
    I_A^{(N)} & = \{(i_1,\ldots,i_N)\mid i_1,\ldots,i_N\in I\land i_1<\cdots<i_{N}\}
  \end{align}
\end{proposition}
\begin{proof}
  完全対称化演算子は置換に対して不変であり, 準同型である為に次のように変形できる.
  \begin{align}
    \hat{\S}^{(N)}\HH^{(N)} & = \hat{\S}^{(N)}\Span\ab\{\ket|\psi_1>\cdots\ket|\psi_N> \mid \ket|\psi_1>\cdots\ket|\psi_N>\in\HH^{(N)} \} \\
                            & = \Span\ab\{\hat{\S}^{(N)}\ket|\psi_1>\cdots\ket|\psi_N> \mid \ket|\psi_1>\cdots\ket|\psi_N>\in\HH^{(N)} \} \\
                            & = \hat{\S}^{(N)}\Span\ab\{\ket|\phi_{i_1}>\cdots\ket|\phi_{i_N}> \mid i_1,\ldots,i_N\in I\}                 \\
                            & = \Span\ab\{\hat{\S}^{(N)}\ket|\phi_{i_1}>\cdots\ket|\phi_{i_N}> \mid i_1,\ldots,i_N\in I\}                 \\
                            & = \Span\ab\{\hat{\S}^{(N)}\ket|\phi_{i_1}>\cdots\ket|\phi_{i_N}> \mid (i_1,\ldots,i_N)\in I_S^{(N)} \}
  \end{align}
  同様に完全反対称についても同じ 1 粒子状態があると 0 となるから次のように変形できる.
  \begin{align}
    \hat{\A}^{(N)}\HH^{(N)} & = \hat{\A}^{(N)}\Span\ab\{\ket|\psi_1>\cdots\ket|\psi_N> \mid \ket|\psi_1>\cdots\ket|\psi_N>\in\HH^{(N)} \} \\
                            & = \Span\ab\{\hat{\A}^{(N)}\ket|\psi_1>\cdots\ket|\psi_N> \mid \ket|\psi_1>\cdots\ket|\psi_N>\in\HH^{(N)} \} \\
                            & = \hat{\A}^{(N)}\Span\ab\{\ket|\phi_{i_1}>\cdots\ket|\phi_{i_N}> \mid i_1,\ldots,i_N\in I \}                \\
                            & = \Span\ab\{\hat{\A}^{(N)}\ket|\phi_{i_1}>\cdots\ket|\phi_{i_N}> \mid i_1,\ldots,i_N\in I \}                \\
                            & = \Span\ab\{\hat{\A}^{(N)}\ket|\phi_{i_1}>\cdots\ket|\phi_{i_N}> \mid (i_1,\ldots,i_N)\in I_A^{(N)} \}
  \end{align}
  これらに対して補題 \ref{hilbert corespondence} を適用して示される.
\end{proof}

\begin{definition}[完全対称, 完全反対称な状態の基底とその粒子数]
  Hilbert 空間 $\HH_S^{(N)}$ の基底状態 $\hat{\S}^{(N)}\ket|\phi_{i_1}>\cdots\ket|\phi_{i_N}> \quad (i_1,\ldots,i_N)\in I_S^{(N)}$ を規格化した状態を $\ket|\phi_{i_1}\cdots\phi_{i_N}>_S$ と定義する.
  同様に Hilbert 空間 $\HH_A^{(N)}$ の基底状態 $\hat{\A}^{(N)}\ket|\phi_{i_1}>\cdots\ket|\phi_{i_N}> \quad (i_1,\ldots,i_N)\in I_A^{(N)}$ を規格化した状態を $\ket|\phi_{i_1}\cdots\phi_{i_N}>_A$ と定義する.
  またこれらの状態の粒子数 $n_i\in\ZZ_{\geq 0}$ を $i$ と等しい $i_\mu$ の個数と定義する. これは占有数ともいう.
\end{definition}
\begin{theorem}[Q21-19, Q21-20, Q21-21]
  粒子状態 $\ket|\phi_{i_1}\cdots\phi_{i_N}>_S, \ket|\phi_{i_1}\cdots\phi_{i_N}>_A$ は粒子数 $n_i$ を用いて次のように表現できる.
  \begin{align}
    \ket|\phi_{i_1}\cdots\phi_{i_N}>_S & = \sqrt{\frac{N!}{\prod_{i\in I}n_i!}}\hat{\S}^{(N)}\ket|\phi_{i_1}>\cdots\ket|\phi_{i_N}> = \frac{1}{\sqrt{N!\prod_{i\in I}n_i!}}\per\begin{bmatrix}\ket|\phi_{i_1}> & \cdots & \ket|\phi_{i_N}>\end{bmatrix} \\
    \ket|\phi_{i_1}\cdots\phi_{i_N}>_A & = \sqrt{N!}\hat{\A}^{(N)}\ket|\phi_{i_1}>\cdots\ket|\phi_{i_N}> = \frac{1}{\sqrt{N!}}\det\begin{bmatrix}\ket|\phi_{i_1}> & \cdots & \ket|\phi_{i_N}>\end{bmatrix}
  \end{align}
\end{theorem}
\begin{proof}
  まず $\hat{\S}^{(N)}\ket|\phi_{i_1}>\cdots\ket|\phi_{i_N}>, \hat{\A}^{(N)}\ket|\phi_{i_1}>\cdots\ket|\phi_{i_N}>$ のノルムを計算すると次のようになる.
  \begin{align}
    \ab\|\hat{\S}^{(N)}\ket|\phi_{i_1}>\cdots\ket|\phi_{i_N}>\| & = \sqrt{\bra<\phi_{i_1}|\cdots\bra<\phi_{i_N}|\hat{\S}^{(N)\dagger}\hat{\S}^{(N)}\ket|\phi_{i_1}>\cdots\ket|\phi_{i_N}>}                                                                  \\
                                                                & = \frac{1}{N!}\sqrt{\sum_{\sigma\in\SS_N}\sum_{\tau\in\SS_N}\bra<\phi_{i_1}|\cdots\bra<\phi_{i_N}|\hat{P}(\tau)^\dagger\hat{P}(\sigma)\ket|\phi_{i_1}>\cdots\ket|\phi_{i_N}>}             \\
                                                                & = \frac{1}{N!}\sqrt{\sum_{\sigma\in\SS_N}\sum_{\tau\in\SS_N}\bra<\phi_{\tau^{-1}(i_1)}|\cdots\bra<\phi_{\tau^{-1}(i_N)}|\ket|\phi_{\sigma^{-1}(i_1)}>\cdots\ket|\phi_{\sigma^{-1}(i_N)}>} \\
                                                                & = \sqrt{\frac{\prod_{i\in I}n_i!}{N!}}
  \end{align}
  \begin{align}
    \ab\|\hat{\A}^{(N)}\ket|\phi_{i_1}>\cdots\ket|\phi_{i_N}>\| & = \sqrt{\bra<\phi_{i_1}|\cdots\bra<\phi_{i_N}|\hat{\A}^{(N)\dagger}\hat{\A}^{(N)}\ket|\phi_{i_1}>\cdots\ket|\phi_{i_N}>}                                                                      \\
                                                                & = \frac{1}{N!}\sqrt{\sum_{\sigma\in\SS_N}\sum_{\tau\in\SS_N}\sgn(\tau\sigma)\bra<\phi_{i_1}|\cdots\bra<\phi_{i_N}|\hat{P}(\tau)^\dagger\hat{P}(\sigma)\ket|\phi_{i_1}>\cdots\ket|\phi_{i_N}>} \\
                                                                & = \frac{1}{N!}\sqrt{\sum_{\sigma\in\SS_N}\sgn(\sigma^2)}                                                                                                                                      \\
                                                                & = \frac{1}{\sqrt{N!}}
  \end{align}
  これより基底状態は次のように書ける.
  \begin{align}
    \ket|\phi_{i_1}\cdots\phi_{i_N}>_S & = \sqrt{\frac{N!}{\prod_{i\in I}n_i!}}\hat{\S}^{(N)}\ket|\phi_{i_1}>\cdots\ket|\phi_{i_N}> \\
    \ket|\phi_{i_1}\cdots\phi_{i_N}>_A & = \sqrt{N!}\hat{\A}^{(N)}\ket|\phi_{i_1}>\cdots\ket|\phi_{i_N}>
  \end{align}
  さらに変形を進めると次のようになる.
  \begin{align}
    \hat{\S}^{(N)}\ket|\phi_{i_1}>\cdots\ket|\phi_{i_N}> & = \frac{1}{N!}\sum_{\sigma\in\SS_N}\hat{P}(\sigma)\ket|\phi_{i_1}>\cdots\ket|\phi_{i_N}>            \\
                                                         & = \frac{1}{N!}\sum_{\sigma\in\SS_N}\ket|\phi_{\sigma^{-1}(i_1)}>\cdots\ket|\phi_{\sigma^{-1}(i_N)}> \\
                                                         & = \frac{1}{N!}\sum_{\sigma\in\SS_N}\ket|\phi_{i_{\sigma(1)}}>\cdots\ket|\phi_{i_{\sigma(N)}}>       \\
                                                         & = \frac{1}{N!}\per\begin{bmatrix}
                                                                               \ket|\phi_{i_1}>^{(1)} & \cdots & \ket|\phi_{i_N}>^{(1)} \\
                                                                               \vdots                 & \ddots & \vdots                 \\
                                                                               \ket|\phi_{i_1}>^{(N)} & \cdots & \ket|\phi_{i_N}>^{(N)}
                                                                             \end{bmatrix}                          \\
                                                         & = \frac{1}{N!}\per\begin{bmatrix}\ket|\phi_{i_1}> & \cdots & \ket|\phi_{i_N}>\end{bmatrix}
  \end{align}
  \begin{align}
    \hat{\A}^{(N)}\ket|\phi_{i_1}>\cdots\ket|\phi_{i_N}> & = \frac{1}{N!}\sum_{\sigma\in\SS_N}\sgn(\sigma)\hat{P}(\sigma)\ket|\phi_{i_1}>\cdots\ket|\phi_{i_N}>            \\
                                                         & = \frac{1}{N!}\sum_{\sigma\in\SS_N}\sgn(\sigma)\ket|\phi_{\sigma^{-1}(i_1)}>\cdots\ket|\phi_{\sigma^{-1}(i_N)}> \\
                                                         & = \frac{1}{N!}\sum_{\sigma\in\SS_N}\sgn(\sigma)\ket|\phi_{i_{\sigma(1)}}>\cdots\ket|\phi_{i_{\sigma(N)}}>       \\
                                                         & = \frac{1}{N!}\det\begin{bmatrix}
                                                                               \ket|\phi_{i_1}>^{(1)} & \cdots & \ket|\phi_{i_N}>^{(1)} \\
                                                                               \vdots                 & \ddots & \vdots                 \\
                                                                               \ket|\phi_{i_1}>^{(N)} & \cdots & \ket|\phi_{i_N}>^{(N)}
                                                                             \end{bmatrix}                                      \\
                                                         & = \frac{1}{N!}\det\begin{bmatrix}
                                                                               \ket|\phi_{i_1}> & \cdots & \ket|\phi_{i_N}>
                                                                             \end{bmatrix}
  \end{align}
  これより示せた.
\end{proof}
\begin{proposition}
  \begin{align}
     & = \sqrt{\frac{\prod_{i\in I}n_i!}{N!}}\sum_{(i_1,\ldots,i_N)\sim(i_1',\ldots,i_N')}\ket|\phi_{i_1'}>\cdots\ket|\phi_{i_N'}>
  \end{align}
\end{proposition}


\begin{proposition}[Q21-20, Q21-21]
  \begin{align}
     & \braket<\phi_{i_1}\cdots\phi_{i_N}|\phi_{i_1'}\cdots\phi_{i_N'}>_S = \delta_{i_1i_1'}\cdots\delta_{i_Ni_N'}                        \\
     & \braket<\phi_{i_1}\cdots\phi_{i_N}|\phi_{i_1'}\cdots\phi_{i_N'}>_A = \delta_{i_1i_1'}\cdots\delta_{i_Ni_N'}                        \\
     & \sum_{(i_1,\ldots,i_N)\in I_S^{(N)}}\ket|\phi_{i_1'}\cdots\phi_{i_N'}>_S\bra<\phi_{i_1}\cdots\phi_{i_N}|_S = \hat{1}_{\HH_S^{(N)}} \\
     & \sum_{(i_1,\ldots,i_N)\in I_A^{(N)}}\ket|\phi_{i_1'}\cdots\phi_{i_N'}>_A\bra<\phi_{i_1}\cdots\phi_{i_N}|_A = \hat{1}_{\HH_A^{(N)}} \\
     & \HH_S^{(N)} = \Span\ab\{\ket|\phi_{i_1'}\cdots\phi_{i_N'}>_S \mid (i_1,\ldots,i_N)\in I_S^{(N)} \}                                 \\
     & \HH_A^{(N)} = \Span\ab\{\ket|\phi_{i_1'}\cdots\phi_{i_N'}>_A \mid (i_1,\ldots,i_N)\in I_A^{(N)} \}
  \end{align}
\end{proposition}
\begin{proof}
\end{proof}

\section{複数の同一粒子からなる量子系の状態に対する対称化の要請}
\begin{definition}
  $N$ 個の同一の Bose 粒子による Hilbert 空間は $\HH_S^{(N)}$, また Fermi 粒子による Hilbert 空間は $\HH_A^{(N)}$ となる.
\end{definition}

\section{計算練習}
\begin{example}[Q21-25, Q21-26, Q21-27]
  1 粒子状態 $\ket|\alpha>\in\HH_{single}$ を持つ Hilbert 空間において 2, 3 個の同一の Bose 粒子, Fermi 粒子の Hilbert 空間は次のようになる.
  \begin{align}
    \HH_S^{(2)} & = \Span\ab\{\ket|\alpha>\ket|\alpha>\}             \\
    \HH_A^{(2)} & = \emptyset                                        \\
    \HH_S^{(3)} & = \Span\ab\{\ket|\alpha>\ket|\alpha>\ket|\alpha>\} \\
    \HH_A^{(3)} & = \emptyset
  \end{align}
  互いに異なる 2 つの 1 粒子状態 $\ket|\alpha>, \ket|\beta>\in\HH_{single}$ を持つ場合は次のようになる.
  \begin{align}
    \HH_S^{(2)} & = \Span\ab\{\ket|\alpha>\ket|\alpha>, \ket|\beta>\ket|\beta>, \frac{1}{\sqrt{2}}(\ket|\alpha>\ket|\beta> + \ket|\beta>\ket|\alpha>)\}               \\
    \HH_A^{(2)} & = \Span\ab\{\frac{1}{\sqrt{2}}(\ket|\alpha>\ket|\beta> - \ket|\beta>\ket|\alpha>)\}                                                                 \\
    \HH_S^{(3)} & = \Span\Big\{\ket|\alpha>\ket|\alpha>\ket|\alpha>, \ket|\beta>\ket|\beta>\ket|\beta>,                                                               \\
                & \qquad\qquad \frac{1}{\sqrt{3}}(\ket|\alpha>\ket|\alpha>\ket|\beta> + \ket|\alpha>\ket|\beta>\ket|\alpha> + \ket|\beta>\ket|\alpha>\ket|\alpha>),   \\
                & \qquad\qquad \frac{1}{\sqrt{3}}(\ket|\alpha>\ket|\beta>\ket|\beta> + \ket|\beta>\ket|\alpha>\ket|\beta> + \ket|\beta>\ket|\beta>\ket|\alpha>)\Big\} \\
    \HH_A^{(3)} & = \emptyset
  \end{align}
  互いに異なる 3 つの 1 粒子状態 $\ket|\alpha>, \ket|\beta>, \ket|\gamma>\in\HH_{single}$ を持つ場合においてそれぞれ 1 つずつある全系の状態は次のようになる.
  \begin{align}
    \frac{1}{\sqrt{6}}(\ket|\alpha>\ket|\beta>\ket|\gamma> + \ket|\gamma>\ket|\alpha>\ket|\beta> + \ket|\beta>\ket|\gamma>\ket|\alpha> + \ket|\gamma>\ket|\beta>\ket|\alpha> + \ket|\alpha>\ket|\gamma>\ket|\beta> + \ket|\beta>\ket|\alpha>\ket|\gamma>)\in\HH_S^{(3)} \\
    \frac{1}{\sqrt{6}}(\ket|\alpha>\ket|\beta>\ket|\gamma> + \ket|\gamma>\ket|\alpha>\ket|\beta> + \ket|\beta>\ket|\gamma>\ket|\alpha> - \ket|\gamma>\ket|\beta>\ket|\alpha> - \ket|\alpha>\ket|\gamma>\ket|\beta> - \ket|\beta>\ket|\alpha>\ket|\gamma>)\in\HH_A^{(3)}
  \end{align}
\end{example}

\section{Bose 粒子系の量子状態の粒子数表示}
\begin{theorem}
  Bose, Fermi 粒子系の粒子数 $n_i, m_i$ について次のような性質を満たす.
  \begin{align}
    \begin{alignedat}{2}
      n_i & \in \ZZ_{\geq 0}, \qquad & \sum_{i\in I}n_i = N \\
      m_i & \in\{0, 1\}, & \sum_{i\in I}m_i = N
    \end{alignedat}
  \end{align}
\end{theorem}
\begin{proof}
  Bose, Fermi 粒子系の粒子状態は次のようにラベル付けされていた.
  \begin{align}
     & \ket|\phi_{i_1}\cdots\phi_{i_N}>_S \qquad (i_1,\ldots,i_N\in I, i_1\leq\cdots\leq i_N) \\
     & \ket|\phi_{i_1}\cdots\phi_{i_N}>_A \qquad (i_1,\ldots,i_N\in I, i_1<\cdots<i_{N})
  \end{align}
  これより
  これより粒子数は非負整数であり, 粒子数の総和は $N$ となる.
  これよりそれぞれの状態は多くとも 1 個であり, $n_i$ を全て合わせて $N$ 個となる.
\end{proof}

\begin{definition}[Bose 粒子系の状態の粒子数表示]
  Bose 粒子系の粒子状態は粒子数を用いて次のように表現できる.
  \begin{align}
    \ket|(n_i)_{i\in I}> = \ket|n_1,n_2,\ldots,n_i,\ldots> = |\underbrace{\phi_1\phi_1\cdots\phi_1}_{n_1}\underbrace{\phi_2\phi_2\cdots\phi_2}_{n_2}\cdots\underbrace{\phi_i\phi_i\cdots\phi_i}_{n_i}\cdots\rangle_S
  \end{align}
  これを粒子数表示または占有数表示という.
\end{definition}
\begin{definition}[Fermi 粒子系の状態の粒子数表示]
  Fermi 粒子系における粒子数表示または占有数表示は次のように書ける.
  \begin{align}
    \ket|(n_i)_{i\in I}> = \ket|n_1,n_2,\ldots,n_i,\ldots> = \ket|\phi_{i_1}\cdots\phi_{i_N}>_A
  \end{align}
\end{definition}
\begin{theorem}
  $N$ 個の Bose 粒子系について正規直交関係, 完備性, 状態の完全性が成り立つ.
  \begin{align}
     & \braket<(n_i)_{i\in I}|(n_i')_{i\in I}> = \prod_{i\in I}\delta_{n_in_i'}                      \\
     & \sum_{\sum_i n_i = N}\ket|(n_i)_{i\in I}>\bra<(n_i)_{i\in I}| = \hat{1}_{\HH_S^{(N)}}         \\
     & \HH_S^{(N)} = \Span\ab\{\ket|(n_i)_{i\in I}> \mid n_i\in\ZZ_{\geq 0}, \sum_{i\in I}n_i = N \}
  \end{align}
  \label{Bose N character}
\end{theorem}
\begin{proof}
  それぞれ元の粒子状態に展開することで示せる.
  \begin{align}
     & \braket<(n_i)_{i\in I}|(n_i')_{i\in I}> = \braket<\phi_{i_1}\cdots\phi_{i_N}|\phi_{i_1'}\cdots\phi_{i_N'}>_S = \prod_{i_\mu\in I}\delta_{i_\mu i_\mu'} = \prod_{i\in I}\delta_{n_in_i'} \\
     & \Span\ab\{\ket|(n_i)_{i\in I}> \mid n_i\in\ZZ_{\geq 0}, \sum_{i\in I}n_i = N \} = \Span\ab\{\ket|\phi_{i_1}\cdots\phi_{i_N}>_S \mid i_\mu\in I \} = \HH_S^{(N)}                         \\
     & \sum_{\sum_i n_i = N}\ket|(n_i)_{i\in I}>\bra<(n_i)_{i\in I}| = \sum_{i_\mu\in I}\ket|\phi_{i_1}\cdots\phi_{i_N}>_S\bra<\phi_{i_1}\cdots\phi_{i_N}|_S = \hat{1}_{\HH_S^{(N)}}
  \end{align}
\end{proof}

\begin{definition}
  また全粒子数を固定しない Bose 粒子系の Hilbert 空間を $\HH_{Bose}$ と書き, 次のように定義する.
  \begin{align}
    \HH_{Bose} & = \bigoplus_{N=0}^\infty \HH_S^{(N)}
  \end{align}
  定義から
  \begin{align}
    N \neq N' \iff \HH_S^{(N)}\perp\HH_S^{(N')}
  \end{align}
\end{definition}
\begin{theorem}
  一般の Bose 粒子系について正規直交関係, 完備性, 状態の完全性が成り立つ.
  \begin{align}
     & \braket<(n_i)_{i\in I}|(n_i')_{i\in I}> = \prod_{i\in I}\delta_{n_in_i'}  \\
     & \sum_{n_i}\ket|(n_i)_{i\in I}>\bra<(n_i)_{i\in I}| = \hat{1}_{\HH_{Bose}} \\
     & \HH_{Bose} = \Span\ab\{\ket|(n_i)_{i\in I}> \mid n_i\in\ZZ_{\geq 0} \}
  \end{align}
\end{theorem}
\begin{proof}
  定理 \ref{Bose N character} について $N$ が異なる状態も考えれば成り立つことが分かる.
\end{proof}

\section{Fermi 粒子系の量子状態の粒子数表示}
\begin{theorem}
  $N$ 個の Fermi 粒子系について正規直交関係, 完備性, 状態の完全性が成り立つ.
  \begin{align}
     & \braket<(n_i)_{i\in I}|(n_i')_{i\in I}> = \prod_{i\in I}\delta_{n_in_i'}                                      \\
     & \sum_{n_i\in\{0, 1\}, \sum_{i\in I}n_i = N} \ket|(n_i)_{i\in I}>\bra<(n_i')_{i\in I}| = \hat{1}_{\HH_A^{(N)}} \\
     & \HH_A^{(N)} = \Span\ab\{\ket|(n_i)_{i\in I}> \mid n_i\in\{0, 1\}, \sum_{i\in I}n_i = N \}
  \end{align}
  \label{Fermi N character}
\end{theorem}
\begin{proof}
  それぞれ元の粒子状態に展開することで示せる.
  \begin{align}
    \braket<(n_i)_{i\in I}|(n_i')_{i\in I}>                                              & = \braket<\phi_{i_1}\cdots\phi_{i_N}|\phi_{i_1'}\cdots\phi_{i_N'}>_A = \prod_{i\in I}\delta_{n_in_i'}   \\
    \sum_{n_i\in\{0, 1\}, \sum_{i\in I}n_i = N} \ket|(n_i)_{i\in I}>\bra<(n_i)_{i\in I}| & = \sum_{i_\mu} \ket|\phi_{i_1}\cdots\phi_{i_N}>\bra<\phi_{i_1}\cdots\phi_{i_N}| = \hat{1}_{\HH_A^{(N)}} \\
    \HH_A^{(N)}                                                                          & = \Span\ab\{\ket|\phi_{i_1}\cdots\phi_{i_N}>_A \mid (i_1,\ldots,i_N)\in I_A^{(N)} \}                    \\
                                                                                         & = \Span\ab\{\ket|(n_i)_{i\in I}> \mid n_i\in\{0, 1\}, \sum_{i\in I}n_i = N \}
  \end{align}
\end{proof}

\begin{definition}
  \begin{align}
    \HH_{Fermi} & = \bigoplus_{N = 0}^{\infty} \HH_A^{(N)}
  \end{align}

  \begin{align}
    N\neq N' \iff \HH_A^{(N)}\perp\HH_A^{(N')}
  \end{align}
\end{definition}
\begin{theorem}
  一般の Fermi 粒子系について正規直交関係, 完備性, 状態の完全性が成り立つ.
  \begin{align}
    \braket<(n_i)_{i\in I}|(n_i')_{i\in I}>                                               & = \prod_{i\in I}\delta_{n_in_i'}                                              \\
    \sum_{n_i\in\{0, 1\}, \sum_{i\in I}n_i = N} \ket|(n_i)_{i\in I}>\bra<(n_i')_{i\in I}| & = \hat{1}_{\HH_{Fermi}}                                                       \\
    \HH_{Fermi}                                                                           & = \Span\ab\{\ket|(n_i)_{i\in I}> \mid n_i\in\{0, 1\}, \sum_{i\in I}n_i = N \}
  \end{align}
\end{theorem}
\begin{proof}
  定理 \ref{Fermi N character} について $N$ が異なる状態も考えれば成り立つことが分かる.
\end{proof}

\section{Bose 粒子系の消滅演算子 $\hat{a}_i$ と生成演算子 $\hat{a}_i^\dagger$}
\begin{definition}
  Bose 粒子系の消滅演算子 $\hat{a}_i$ と生成演算子 $\hat{a}_i^\dagger$ を次のように定義する.
  \begin{align}
     & \hat{a}_i\frac{1}{\sqrt{N!}}\per\begin{bmatrix}\ket|\phi_{i_1}> \cdots \ket|\phi_{i_N}>\end{bmatrix} = \frac{1}{\sqrt{(N-1)!}}\sum_{i_\mu = i}\per\begin{bmatrix}\ket|\phi_{i_1}> \cdots \ket|\phi_{i_{\mu - 1}}> & \ket|\phi_{i_{\mu + 1}}> \cdots \ket|\phi_{i_N}>\end{bmatrix} \\
     & \hat{a}_i^\dagger\frac{1}{\sqrt{N!}}\per\begin{bmatrix}\ket|\phi_{i_1}> \cdots \ket|\phi_{i_N}>\end{bmatrix} = \frac{1}{\sqrt{(N+1)!}}\per\begin{bmatrix}\ket|\phi_{i}> & \ket|\phi_{i_1}> \cdots \ket|\phi_{i_N}>\end{bmatrix}
  \end{align}
  その上で個数演算子 $\hat{n}_i$ と全粒子数演算子 $\hat{N}$ を次のように定義する.
  \begin{align}
    \hat{n}_i & = \hat{a}_i^\dagger\hat{a} \\
    \hat{N}   & = \sum_{i\in I}\hat{n}_i
  \end{align}
\end{definition}
\begin{theorem}[Q21-35, Q21-36]
  Bose 粒子系の消滅演算子 $\hat{a}_i$ と生成演算子 $\hat{a}_i^\dagger$ について次の 2 式と定義は同値である.
  \begin{align}
    \hat{a}_i\ket|\ldots,n_i,\ldots>         & = \sqrt{n_i}\ket|\ldots,n_i-1,\ldots>   \\
    \hat{a}_i^\dagger\ket|\ldots,n_i,\ldots> & = \sqrt{n_i+1}\ket|\ldots,n_i+1,\ldots>
  \end{align}
  \label{Bose feature}
\end{theorem}
\begin{proof}
  定義は $\ket|\phi_{i_1}\cdots\phi_{i_N}>_S$ の粒子数 $n_i$ を用いて次のようになる.
  \begin{align}
     & \hat{a}_i\frac{1}{\sqrt{N!\prod_{j\in I}n_j!}}\per\begin{bmatrix}\ket|\phi_{i_1}> \cdots \ket|\phi_{i_N}>\end{bmatrix} = \frac{n_i}{\sqrt{(N-1)!\prod_{j\in I}n_j!}}\per\begin{bmatrix}\ket|\phi_{i_1}> \cdots \ket|\phi_{i_{\mu - 1}}> & \ket|\phi_{i_{\mu + 1}}> \cdots \ket|\phi_{i_N}>\end{bmatrix} \\
     & \hat{a}_i^\dagger\frac{1}{\sqrt{N!\prod_{j\in I}n_j!}}\per\begin{bmatrix}\ket|\phi_{i_1}> \cdots \ket|\phi_{i_N}>\end{bmatrix} = \frac{1}{\sqrt{(N+1)!\prod_{j\in I}n_j!}}\per\begin{bmatrix}\ket|\phi_{i}> & \ket|\phi_{i_1}> \cdots \ket|\phi_{i_N}>\end{bmatrix}
  \end{align}
  Bose 粒子系の粒子数表示は次のように展開できる.
  \begin{align}
    \ket|\ldots,n_i,\ldots> & = \ket|\phi_{i_1},\ldots,\phi_{i_N}>_S = \frac{1}{\sqrt{N!\prod_{j\in I}n_j!}}\per\begin{bmatrix}\ket|\phi_{i_1}>\cdots\ket|\phi_{i_N}>\end{bmatrix}
  \end{align}
  これより定義と次の式は同値である.
  \begin{align}
    \hat{a}_i\ket|\phi_{i_1}\cdots\phi_{i_N}>_S         & = \sqrt{n_i}\ket|\phi_{i_1}\cdots\phi_{i_{\mu - 1}}\phi_{i_{\mu + 1}}\cdots\phi_{i_N}>_S \\
    \hat{a}_i^\dagger\ket|\phi_{i_1}\cdots\phi_{i_N}>_S & = \sqrt{n_i+1}\ket|\phi_{i} \phi_{i_1}\cdots\phi_{i_N}>_S
  \end{align}
  よって次の式は同値である.
  \begin{align}
    \hat{a}_i\ket|\ldots,n_i,\ldots>         & = \sqrt{n_i}\ket|\ldots,n_i-1,\ldots>     \\
    \hat{a}_i^\dagger\ket|\ldots,n_i,\ldots> & = \sqrt{n_i + 1}\ket|\ldots,n_i+1,\ldots>
  \end{align}
\end{proof}

\begin{proposition}[Q21-37]
  Bose 粒子系における消滅演算子 $\hat{a}_i$ と生成演算子 $\hat{a}_i^\dagger$ の交換関係は次のようになる.
  \begin{align}
    [\hat{a}_i, \hat{a}_j^\dagger] & = \delta_{ij}, \qquad [\hat{a}_i, \hat{a}_j] = [\hat{a}_i^\dagger, \hat{a}_j^\dagger] = 0
  \end{align}
\end{proposition}
\begin{proof}
  消滅演算子 $\hat{a}_i$, 生成演算子 $\hat{a}_i^\dagger$ を状態 $\ket|\ldots,n_i,\ldots>\in\HH_{Bose}$ に適用すると
  \begin{align}
    \hat{a}_i\hat{a}_i^\dagger\ket|\ldots,n_i,\ldots> & = \sqrt{n_i + 1}\hat{a}_i\ket|\ldots,n_i+1,\ldots> = (n_i + 1)\ket|\ldots,n_i,\ldots> \\
    \hat{a}_i^\dagger\hat{a}_i\ket|\ldots,n_i,\ldots> & = \sqrt{n_i}\hat{a}_i^\dagger\ket|\ldots,n_i-1,\ldots> = n_i\ket|\ldots,n_i,\ldots>
  \end{align}
  よりそれぞれの交換関係は次のようになる.
  \begin{align}
    [\hat{a}_i, \hat{a}_i^\dagger] & = \hat{a}_i\hat{a}_i^\dagger - \hat{a}_i^\dagger\hat{a}_i = (n_i + 1) - n_i = 1 \\
    [\hat{a}_i, \hat{a}_i]         & = [\hat{a}_i^\dagger, \hat{a}_i^\dagger] = 0
  \end{align}
  異なる添字 $i, j$ についても状態 $\ket|\ldots,n_i,\ldots,n_j,\ldots>\in\HH_{Bose}$ に適用すると
  \begin{align}
    \hat{a}_i\hat{a}_j\ket|\ldots,n_i,\ldots,n_j,\ldots>                 & = \sqrt{n_in_j}\ket|\ldots,n_i-1,\ldots,n_j-1,\ldots>             \\
    \hat{a}_i\hat{a}_j^\dagger\ket|\ldots,n_i,\ldots,n_j,\ldots>         & = \sqrt{n_i(n_j + 1)}\ket|\ldots,n_i-1,\ldots,n_j+1,\ldots>       \\
    \hat{a}_i^\dagger\hat{a}_j\ket|\ldots,n_i,\ldots,n_j,\ldots>         & = \sqrt{(n_i + 1)n_j}\ket|\ldots,n_i+1,\ldots,n_j-1,\ldots>       \\
    \hat{a}_i^\dagger\hat{a}_j^\dagger\ket|\ldots,n_i,\ldots,n_j,\ldots> & = \sqrt{(n_i + 1)(n_j + 1)}\ket|\ldots,n_i+1,\ldots,n_j+1,\ldots>
  \end{align}
  よりそれぞれの交換関係は次のようになる.
  \begin{align}
    [\hat{a}_i, \hat{a}_j^\dagger]         & = \hat{a}_i\hat{a}_j^\dagger - \hat{a}_j^\dagger\hat{a}_i = 0                 \\
    [\hat{a}_i, \hat{a}_j]                 & = \hat{a}_i\hat{a}_j - \hat{a}_j\hat{a}_i = 0                                 \\
    [\hat{a}_i^\dagger, \hat{a}_j^\dagger] & = \hat{a}_i^\dagger\hat{a}_j^\dagger - \hat{a}_j^\dagger\hat{a}_i^\dagger = 0
  \end{align}
  よって示された.
\end{proof}

\begin{proposition}[Q21-38]
  Bose 粒子系における消滅演算子 $\hat{a}_i$ と生成演算子 $\hat{a}_i^\dagger$ は互いに Hermite 共役である.
\end{proposition}
\begin{proof}
  計算することで次式が成り立つ.
  \begin{align}
    \bra<\ldots,n_i-1,\ldots|\hat{a}_i\ket|\ldots,n_i,\ldots> & = \bra<\ldots,n_i,\ldots|\hat{a}_i^\dagger\ket|\ldots,n_i-1,\ldots> = \sqrt{n_i}               \\
    \bra<(n_i)_{i\in I}|\hat{a}_i\ket|(n_i')_{i\in I}>        & = \bra<(n_i')_{i\in I}|\hat{a}_i^\dagger\ket|(n_i)_{i\in I}> = 0                 & (otherwise)
  \end{align}
  よって $\hat{a}_i, \hat{a}_i^\dagger$ は互いに Hermite 共役である.
\end{proof}

\begin{proposition}[Q21-39]
  個数演算子 $\hat{n}_i$ と全粒子数演算子 $\hat{N}$ は Hermite 演算子であり, 固有値は $\hat{n}_i = n_i, \hat{N} = N$ となる.
\end{proposition}
\begin{proof}
  個数演算子 $\hat{n}_i$ は生成消滅演算子に展開することで計算できる.
  \begin{align}
    \hat{n}_i^\dagger                & = (\hat{a}_i^\dagger\hat{a}_i)^\dagger = \hat{a}_i^\dagger\hat{a}_i = \hat{n}_i  \\
    \hat{n}_i\ket|\ldots,n_i,\ldots> & = \hat{a}_i^\dagger\hat{a}_i\ket|\ldots,n_i,\ldots> = n_i\ket|\ldots,n_i,\ldots>
  \end{align}
  全粒子数演算子 $\hat{N}$ は個数演算子に展開することで計算できる.
  \begin{align}
    \hat{N}^\dagger             & = \sum_{i\in I}\hat{n}_i^\dagger = \sum_{i\in I}\hat{n}_i = \hat{N}                                         \\
    \hat{N}\ket|(n_i)_{i\in I}> & = \sum_{i\in I}\hat{n}_i\ket|(n_i)_{i\in I}> = \sum_{i\in I}n_i\ket|(n_i)_{i\in I}> = N\ket|(n_i)_{i\in I}>
  \end{align}
\end{proof}

\begin{theorem}[Q21-41]
  Bose 粒子系における消滅演算子 $\hat{a}_i$ と生成演算子 $\hat{a}_i^\dagger$ において次の性質は定義と同値である.
  \begin{align}
    (\hat{a}_i)^\dagger = \hat{a}_i^\dagger, \qquad [\hat{a}_i, \hat{a}_j^\dagger] = \delta_{ij}, \qquad [\hat{a}_i, \hat{a}_j] = [\hat{a}_i^\dagger, \hat{a}_j^\dagger] = 0, \qquad \hat{n}_i = \hat{a}_i^\dagger\hat{a}_i = n_i
  \end{align}
\end{theorem}
\begin{proof}
  既に定義から性質を導くことはしているので性質から定義を導く.
  \begin{align}
    \hat{n}_i\hat{a}_i         & = (\hat{a}_i^\dagger\hat{a}_i)\hat{a}_i = (\hat{a}_i\hat{a}_i^\dagger - 1)\hat{a}_i = (n_i - 1)\hat{a}_i                                     \\
    \hat{n}_i\hat{a}_i^\dagger & = \hat{a}_i^\dagger(\hat{a}_i\hat{a}_i^\dagger) = \hat{a}_i^\dagger(\hat{a}_i^\dagger\hat{a}_i + 1) = (n_i + 1)\hat{a}_i^\dagger             \\
    \hat{n}_j\hat{a}_i         & = \hat{a}_j^\dagger\hat{a}_j\hat{a}_i = \hat{a}_i\hat{a}_j^\dagger\hat{a}_j = n_j\hat{a}_i                                       & (i\neq j) \\
    \hat{n}_j\hat{a}_i^\dagger & = \hat{a}_j^\dagger\hat{a}_j\hat{a}_i^\dagger = \hat{a}_i^\dagger\hat{a}_j^\dagger\hat{a}_j = n_j\hat{a}_i^\dagger               & (i\neq j)
  \end{align}
  より $\hat{a}_i, \hat{a}_i^\dagger$ を適用すると状態の粒子数 $n_i$ が 1 だけ上下する. また $(\hat{a}_i)^\dagger = \hat{a}_i^\dagger$ より
  \begin{align}
     & \bra<\ldots,n_i-1,\ldots|\hat{a}_i\ket|\ldots,n_i,\ldots> = \bra<\ldots,n_i,\ldots|\hat{a}_i^\dagger\ket|\ldots,n_i-1,\ldots> \\
     & n_i = \bra<\ldots,n_i,\ldots|\hat{a}_i^\dagger\hat{a}_i\ket|\ldots,n_i,\ldots>
  \end{align}
  であるから次のようになる.
  \begin{align}
    \hat{a}_i\ket|\ldots,n_i,\ldots>         & = \sqrt{n_i}\ket|\ldots,n_i-1,\ldots>     \\
    \hat{a}_i^\dagger\ket|\ldots,n_i,\ldots> & = \sqrt{n_i + 1}\ket|\ldots,n_i+1,\ldots>
  \end{align}
  これらの式から定理 \ref{Bose feature} より定義を導ける.
\end{proof}

\begin{proposition}
  真空状態 $\ket|\mathrm{vac}>$ を次のように定義する.
  \begin{align}
    \ket|\mathrm{vac}> = \ket|0,0,0,\ldots>
  \end{align}
  このとき次のような性質が認められる.
  \begin{align}
    \hat{a}_i\ket|\mathrm{vac}>        & = 0                                                                             \\
    \braket<\mathrm{vac}|\mathrm{vac}> & = 1                                                                             \\
    \ket|(n_i)_{i\in I}>               & = \prod_{i\in I}\frac{(\hat{a}_i^\dagger)^{n_i}}{\sqrt{n_i!}}\ket|\mathrm{vac}>
  \end{align}
\end{proposition}
\begin{proof}
  それぞれ定義を展開することで導かれる.
  \begin{align}
    \hat{a}_i\ket|\mathrm{vac}>        & = \hat{a}_i\ket|0,0,\ldots> = 0                                                                                                                               \\
    \braket<\mathrm{vac}|\mathrm{vac}> & = \braket<0,0,\ldots|0,0,\ldots> = \prod_{i\in I}\delta_{0,0} = 1                                                                                             \\
    \ket|(n_i)_{i\in I}>               & = \prod_{i\in I}\frac{(\hat{a}_i^\dagger)^{n_i}}{\sqrt{n_i!}}\ket|0,0,\ldots> = \prod_{i\in I}\frac{(\hat{a}_i^\dagger)^{n_i}}{\sqrt{n_i!}}\ket|\mathrm{vac}>
  \end{align}
\end{proof}

\section{Fermi 粒子系の消滅演算子 $\hat{c}_i$ と生成演算子 $\hat{c}_i^\dagger$}
\begin{definition}
  \begin{align}
     & \hat{c}_i\frac{1}{\sqrt{N!}}\det\begin{bmatrix}\ket|\phi_{i_1}> \cdots \ket|\phi_{i_N}>\end{bmatrix} = \begin{dcases}
                                                                                                                \frac{(-1)^\mu}{\sqrt{(N-1)!}}\det\begin{bmatrix}\ket|\phi_{i_1}> \cdots \ket|\phi_{i_{\mu-1}}>\ket|\phi_{i_{\mu+1}}>\cdots\ket|\phi_{i_N}>\end{bmatrix} & (n_i = 1) \\
                                                                                                                0                                                                                                                                                                                                  & (n_i = 0)
                                                                                                              \end{dcases} \\
     & \hat{c}_i^\dagger\frac{1}{\sqrt{N!}}\det\begin{bmatrix}\ket|\phi_{i_1}> \cdots \ket|\phi_{i_N}>\end{bmatrix} = \frac{1}{\sqrt{(N+1)!}}\det\begin{bmatrix}\ket|\phi_i> & \ket|\phi_{i_1}> \cdots \ket|\phi_{i_N}>\end{bmatrix}
  \end{align}
  \begin{align}
    \hat{n}_i & = \hat{c}_i^\dagger\hat{c}_i \\
    \hat{N}   & = \sum_{i\in I}\hat{n}_i
  \end{align}
\end{definition}
\begin{proposition}
  差分
\end{proposition}

\begin{theorem}[Q21-50, Q21-51]
  \begin{align}
    \hat{c}_i\ket|\ldots,n_i,\ldots>         & = (-1)^{\sum_{j=1}^{i-1}n_j}n_i\ket|\ldots,0,\ldots>       \\
    \hat{c}_i^\dagger\ket|\ldots,n_i,\ldots> & = (-1)^{\sum_{j=1}^{i-1}n_j}(1 - n_i)\ket|\ldots,1,\ldots>
  \end{align}
\end{theorem}
\begin{proof}
  定義は $\ket|\phi_{i_1}\cdots\phi_{i_N}>_A$ の粒子数 $n_i$ を用いて次のようになる.
  \begin{align}
    \hat{c}_i\frac{1}{\sqrt{N!}}\det\begin{bmatrix}\ket|\phi_{i_1}> \cdots \ket|\phi_{i_N}>\end{bmatrix}         & = \frac{(-1)^\mu}{\sqrt{(N-1)!}}n_i\det\begin{bmatrix}\ket|\phi_{i_1}> \cdots \ket|\phi_{i_{\mu-1}}>\ket|\phi_{i_{\mu+1}}>\cdots\ket|\phi_{i_N}>\end{bmatrix} \\
    \hat{c}_i^\dagger\frac{1}{\sqrt{N!}}\det\begin{bmatrix}\ket|\phi_{i_1}> \cdots \ket|\phi_{i_N}>\end{bmatrix} & = \frac{1}{\sqrt{(N+1)!}}\det\begin{bmatrix}\ket|\phi_i> & \ket|\phi_{i_1}> \cdots \ket|\phi_{i_N}>\end{bmatrix}
  \end{align}
  Fermi 粒子系の粒子数表示は次のように展開できる.
  \begin{align}
    \ket|n_1,n_2,\ldots,n_i,\ldots> = \ket|\phi_{i_1}\cdots\phi_{i_N}>_A & = \frac{1}{\sqrt{N!}}\det\begin{bmatrix}\ket|\phi_{i_1}> & \cdots & \ket|\phi_{i_N}>\end{bmatrix}
  \end{align}
  これより定義と次の式は同値である.
  \begin{align}
    \hat{c}_i\ket|\phi_{i_1}\cdots\phi_{i_N}>         & = (-1)^\mu n_i\ket|\phi_{i_1}\cdots\phi_{i_{\mu-1}}\phi_{i_{\mu+1}}\cdots\phi_{i_N}>_A \\
    \hat{c}_i^\dagger\ket|\phi_{i_1}\cdots\phi_{i_N}> & = \ket|\phi_i\phi_{i_1}\cdots\phi_{i_N}>_A
  \end{align}
  よって次の式は同値である.
  \begin{align}
    \hat{c}_i\ket|\ldots,n_i,\ldots>         & = (-1)^{\sum_{j=1}^{i-1}n_j} n_i\ket|\ldots,1-n_i,\ldots>    \\
    \hat{c}_i^\dagger\ket|\ldots,n_i,\ldots> & = (-1)^{\sum_{j=1}^{i-1}n_j}(1-n_i)\ket|\ldots,1-n_i,\ldots>
  \end{align}
\end{proof}

\begin{theorem}[Q21-52]
  Fermi 粒子系における消滅演算子 $\hat{c}_i$ と生成演算子 $\hat{c}_i^\dagger$ の反交換関係は次のようになる.
  \begin{align}
    \{\hat{c}_i, \hat{c}_j^\dagger\} = \delta_{ij}, \qquad \{\hat{c}_i, \hat{c}_j\} = \{\hat{c}_i^\dagger, \hat{c}_j^\dagger\} = 0
  \end{align}
\end{theorem}
\begin{proof}
  \begin{align}
    \hat{c}_i\ket|\ldots,n_i,\ldots>         & = (-1)^{\sum_{j=1}^{i-1}n_j}n_i\ket|\ldots,0,\ldots>       \\
    \hat{c}_i^\dagger\ket|\ldots,n_i,\ldots> & = (-1)^{\sum_{j=1}^{i-1}n_j}(1 - n_i)\ket|\ldots,1,\ldots>
  \end{align}
  \begin{align}
    \hat{c}_i\hat{c}_i^\dagger\ket|\ldots,n_i,\ldots>         & = (1 - n_i)\ket|\ldots,0,\ldots> \\
    \hat{c}_i^\dagger\hat{c}_i\ket|\ldots,n_i,\ldots>         & = n_i\ket|\ldots,1,\ldots>       \\
    \hat{c}_i\hat{c}_i\ket|\ldots,n_i,\ldots>                 & = 0                              \\
    \hat{c}_i^\dagger\hat{c}_i^\dagger\ket|\ldots,n_i,\ldots> & = 0
  \end{align}
  \begin{align}
    \{\hat{c}_i, \hat{c}_i^\dagger\} = 1, \qquad \{\hat{c}_i, \hat{c}_i\} = \{\hat{c}_i^\dagger, \hat{c}_i^\dagger\} = 0
  \end{align}
  添字 $i, j$ が $i < j$ の順となっているとき先に $\hat{c}_i$ が適用されると後置の演算子で粒子数が 1 ずれることを考慮して次のようになる.
  \begin{align}
    \hat{c}_i\hat{c}_j^\dagger\ket|\ldots,n_i,\ldots,n_j,\ldots>         & = (-1)^{\sum_{k=i}^{j-1}n_k}n_i(1 - n_j)\ket|\ldots,1-n_i,\ldots,1-n_j,\ldots>           \\
    \hat{c}_j^\dagger\hat{c}_i\ket|\ldots,n_i,\ldots,n_j,\ldots>         & = (-1)^{1 + \sum_{k=i}^{j-1}n_k}n_i(1 - n_j)\ket|\ldots,1-n_i,\ldots,1-n_j,\ldots>       \\
    \hat{c}_i^\dagger\hat{c}_j\ket|\ldots,n_i,\ldots,n_j,\ldots>         & = (-1)^{\sum_{k=i}^{j-1}n_k}(1 - n_i)n_j\ket|\ldots,1-n_i,\ldots,1-n_j,\ldots>           \\
    \hat{c}_j\hat{c}_i^\dagger\ket|\ldots,n_i,\ldots,n_j,\ldots>         & = (-1)^{1 + \sum_{k=i}^{j-1}n_k}(1 - n_i)n_j\ket|\ldots,1-n_i,\ldots,1-n_j,\ldots>       \\
    \hat{c}_i\hat{c}_j\ket|\ldots,n_i,\ldots,n_j,\ldots>                 & = (-1)^{\sum_{k=i}^{j-1}n_k}n_in_j\ket|\ldots,1-n_i,\ldots,1-n_j,\ldots>                 \\
    \hat{c}_j\hat{c}_i\ket|\ldots,n_i,\ldots,n_j,\ldots>                 & = (-1)^{1 + \sum_{k=i}^{j-1}n_k}n_in_j\ket|\ldots,1-n_i,\ldots,1-n_j,\ldots>             \\
    \hat{c}_i^\dagger\hat{c}_j^\dagger\ket|\ldots,n_i,\ldots,n_j,\ldots> & = (-1)^{\sum_{k=i}^{j-1}n_k}(1 - n_i)(1 - n_j)\ket|\ldots,1-n_i,\ldots,1-n_j,\ldots>     \\
    \hat{c}_j^\dagger\hat{c}_i^\dagger\ket|\ldots,n_i,\ldots,n_j,\ldots> & = (-1)^{1 + \sum_{k=i}^{j-1}n_k}(1 - n_i)(1 - n_j)\ket|\ldots,1-n_i,\ldots,1-n_j,\ldots>
  \end{align}
  $\{A, B\} = \{B, A\}$ 次の反交換関係が求まる.
  \begin{align}
    \{\hat{c}_i, \hat{c}_j^\dagger\} = \{\hat{c}_i, \hat{c}_j\} = \{\hat{c}_i^\dagger, \hat{c}_j^\dagger\} & = 0 \qquad (i\neq j)
  \end{align}
\end{proof}

\begin{proposition}
  Fermi 粒子系における消滅演算子 $\hat{c}_i$ と生成演算子 $\hat{c}_i^\dagger$ は互いに Hermite 共役である.
\end{proposition}
\begin{proof}
  \begin{align}
    \bra<n_1',\ldots,n_i',\ldots|\hat{c}_i^\dagger\ket|n_1,\ldots,n_i,\ldots> & = (-1)^{\sum_{j=1}^{i-1}n_j}\braket<n_1',\ldots,n_i',\ldots|n_1,\ldots,n_i+1,\ldots>                                          \\
                                                                              & = (-1)^{\sum_{j=1}^{i-1}n_j}\delta_{n_1n_1'}\cdots\delta_{n_{i-1}n_{i-1}'}\delta_{n_i+1n_{i}'}\delta_{n_{i+1}n_{i+1}'}\cdots  \\
    \bra<n_1,\ldots,n_i,\ldots|\hat{c}_i\ket|n_1',\ldots,n_i',\ldots>         & = (-1)^{\sum_{j=1}^{i-1}n_j'}\braket<n_1,\ldots,n_i,\ldots|n_1',\ldots,n_i'-1,\ldots>                                         \\
                                                                              & = (-1)^{\sum_{j=1}^{i-1}n_j'}\delta_{n_1n_1'}\cdots\delta_{n_{i-1}n_{i-1}'}\delta_{n_i+1n_{i}'}\delta_{n_{i+1}n_{i+1}'}\cdots
  \end{align}
  より Hermite 共役である.
\end{proof}

\begin{proposition}
  \begin{align}
    \hat{n}_i^\dagger & = \hat{n}_i, \hat{n}_i\ket|\ldots,n_i,\ldots> = n_i\ket|\ldots,n_i,\ldots> \\
    \hat{N}^\dagger   & = \hat{N}, \hat{N}\ket|(n_i)_{i\in I}> = N\ket|(n_i)_{i\in I}>
  \end{align}
\end{proposition}
\begin{proof}
  \begin{align}
    \hat{n}_i^\dagger                & = (\hat{c}_i^\dagger\hat{c}_i)^\dagger = \hat{c}_i^\dagger\hat{c}_i = \hat{n}_i                                                            \\
    \hat{N}^\dagger                  & = \sum_{i\in I}\hat{n}_i^\dagger = \sum_{i\in I}\hat{n}_i = \hat{N}                                                                        \\
    \hat{n}_i\ket|\ldots,n_i,\ldots> & = \hat{c}_i^\dagger\hat{c}_i\ket|\ldots,n_i,\ldots> = (-1)^{2\sum_{j=1}^{i-1}n_j}n_i^2\ket|\ldots,n_i,\ldots> = n_i\ket|\ldots,n_i,\ldots> \\
    \hat{N}\ket|(n_i)_{i\in I}>      & = \sum_{i\in I}\hat{n}_i\ket|(n_i)_{i\in I}> = \sum_{i\in I}n_i\ket|(n_i)_{i\in I}> = N\ket|(n_i)_{i\in I}>
  \end{align}
\end{proof}

\begin{proposition}
  \begin{align}
    \ket|\mathrm{vac}> = \ket|0,0,\ldots>
  \end{align}
  \begin{align}
    \hat{c}_i\ket|\mathrm{vac}> = 0 \\
    \braket<\mathrm{vac}|\mathrm{vac}> = 1
  \end{align}
\end{proposition}
\begin{proof}

\end{proof}

\section{演算子の粒子数表示: 1 粒子演算子とその和、2 粒子演算子とその和の導入}
\begin{align}
  \hat{H} & = \frac{1}{2m_e}\sum_{\mu=1}^{N}\hat{\pp}_\mu^2 - Ze^2\sum_{\mu=1}^{N}\frac{1}{|\hat{\rr}_\mu|} + e^2\sum_{1\leq\mu<\nu\leq N}\frac{1}{|\hat{\rr}_\mu - \hat{\rr}_\nu|} + \frac{e}{2m_ec}(\hat{\bm{L}} + 2\hat{\bm{S}})\cdot\bm{B} + \frac{e^2}{8m_ec^2}\sum_{\mu=1}^{N}(\bm{B}\times\hat{\rr}_\mu)^2
\end{align}

\end{document}