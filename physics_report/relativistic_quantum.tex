\RequirePackage{plautopatch}
\documentclass[uplatex,dvipdfmx,a4paper,11pt]{jlreq}
\usepackage{bxpapersize}
\usepackage[utf8]{inputenc}
\usepackage{fontenc}
\usepackage{lmodern}
\usepackage{otf}
\usepackage{amsmath}
\usepackage{amssymb}
\usepackage{amsthm}
\usepackage{ascmac}
% \usepackage[hyphens]{url}
\usepackage{physics2}
\usephysicsmodule{ab, ab.braket, doubleprod, diagmat, xmat}
\usepackage{diffcoeff}
% \usepackage{braket}
\usepackage{verbatimbox}
\usepackage{bm}
\usepackage{url}
% \usepackage[dvipdfmx,hiresbb,final]{graphicx}
\usepackage{hyperref}
\usepackage{pxjahyper}
\usepackage{tikz}\usetikzlibrary{cd}
\usepackage{listings}
\usepackage{color}
\usepackage{mathtools}
\usepackage{xspace}
\usepackage{xy}
\usepackage{xypic}
%
\title{相対論的量子力学 期末レポート}
\author{}
\makeatletter
%
\DeclareMathOperator{\lcm}{lcm}
\DeclareMathOperator{\Kernel}{Ker}
\DeclareMathOperator{\Image}{Im}
\DeclareMathOperator{\ch}{ch}
\DeclareMathOperator{\Aut}{Aut}
\DeclareMathOperator{\Log}{Log}
\DeclareMathOperator{\Arg}{Arg}
\DeclareMathOperator{\sgn}{sgn}
%
\newcommand{\CC}{\mathbb{C}}
\newcommand{\RR}{\mathbb{R}}
\newcommand{\QQ}{\mathbb{Q}}
\newcommand{\ZZ}{\mathbb{Z}}
\newcommand{\NN}{\mathbb{N}}
\newcommand{\FF}{\mathbb{F}}
\newcommand{\PP}{\mathbb{P}}
\newcommand{\GG}{\mathbb{G}}
\newcommand{\TT}{\mathbb{T}}
\newcommand{\EE}{\bm{E}}
\newcommand{\rr}{\bm{r}}
\newcommand{\kk}{\bm{k}}
\newcommand{\pp}{\bm{p}}
\newcommand{\calB}{\mathcal{B}}
\newcommand{\calF}{\mathcal{F}}
\newcommand{\ignore}[1]{}
\newcommand{\floor}[1]{\left\lfloor #1 \right\rfloor}
% \newcommand{\abs}[1]{\left\lvert #1 \right\rvert}
\newcommand{\lt}{<}
\newcommand{\gt}{>}
\newcommand{\id}{\mathrm{id}}
\newcommand{\rot}{\curl}
\renewcommand{\angle}[1]{\left\langle #1 \right\rangle}
\newcommand\mqty[1]{\begin{pmatrix}#1\end{pmatrix}}
\newcommand\vmqty[1]{\begin{vmatrix}#1\end{vmatrix}}
\numberwithin{equation}{section}

\let\oldcite=\cite
\renewcommand\cite[1]{\hyperlink{#1}{\oldcite{#1}}}

\let\oldbibitem=\bibitem
\renewcommand{\bibitem}[2][]{\label{#2}\oldbibitem[#1]{#2}}

% theorem環境の設定
% - 冒頭に改行
% - 末尾にdiamond (amsthm)
\theoremstyle{definition}
\newcommand*{\newscreentheoremx}[2]{
  \newenvironment{#1}[1][]{
    \begin{screen}
    \begin{#2}[##1]
      \leavevmode
      \newline
  }{
    \end{#2}
    \end{screen}
  }
}
\newcommand*{\newqedtheoremx}[2]{
  \newenvironment{#1}[1][]{
    \begin{#2}[##1]
      \leavevmode
      \newline
      \renewcommand{\qedsymbol}{\(\diamond\)}
      \pushQED{\qed}
  }{
      \qedhere
      \popQED
    \end{#2}
  }
}
\newtheorem{theorem*}{定理}[section]

\newqedtheoremx{theorem}{theorem*}
\newcommand*\newqedtheorem@unstarred[2]{%
  \newtheorem{#1*}[theorem*]{#2}
  \newqedtheoremx{#1}{#1*}
}
\newcommand*\newqedtheorem@starred[2]{%
  \newtheorem*{#1*}{#2}
  \newqedtheoremx{#1}{#1*}
}
\newcommand*{\newqedtheorem}{\@ifstar{\newqedtheorem@starred}{\newqedtheorem@unstarred}}

\newtheorem{sctheorem*}{定理}[section]
\newscreentheoremx{sctheorem}{sctheorem*}
\newcommand*\newscreentheorem@unstarred[2]{%
  \newtheorem{#1*}[theorem*]{#2}
  \newscreentheoremx{#1}{#1*}
}
\newcommand*\newscreentheorem@starred[2]{%
  \newtheorem*{#1*}{#2}
  \newscreentheoremx{#1}{#1*}
}
\newcommand*{\newscreentheorem}{\@ifstar{\newscreentheorem@starred}{\newscreentheorem@unstarred}}

%\newtheorem*{definition}{定義}
%\newtheorem{theorem}{定理}
%\newtheorem{proposition}[theorem]{命題}
%\newtheorem{lemma}[theorem]{補題}
%\newtheorem{corollary}[theorem]{系}

\newqedtheorem{lemma}{補題}
\newqedtheorem{corollary}{系}
\newqedtheorem{example}{例}
\newqedtheorem{proposition}{命題}
\newqedtheorem{remark}{注意}
\newqedtheorem{thesis}{主張}
\newqedtheorem{notation}{記法}
\newqedtheorem{problem}{問題}
\newqedtheorem{algorithm}{アルゴリズム}

\newscreentheorem*{axiom}{公理}
\newscreentheorem*{definition}{定義}

\renewenvironment{proof}[1][\proofname]{\par
  \normalfont
  \topsep6\p@\@plus6\p@ \trivlist
  \item[\hskip\labelsep{\bfseries #1}\@addpunct{\bfseries}]\ignorespaces\quad\par
}{%
  \qed\endtrivlist\@endpefalse
}
\renewcommand\proofname{証明}

\makeatother

\begin{document}
\maketitle
\tableofcontents
\clearpage

\section{2次元時空におけるディラック方程式}
\begin{problem}
2次元時空におけるディラック方程式は次のように考えられる。
\begin{align}
  (i\hbar\gamma^\mu\partial_\mu - mc)\psi(x) = 0
\end{align}
このときガンマ行列 $\gamma^0, \gamma^1$ は次を満たす。
\begin{align}
  \lbrace\gamma^\mu, \gamma^\nu\rbrace = 2g^{\mu\nu}, \qquad (\gamma^0)^\dagger = \gamma^0, \qquad (\gamma^1)^\dagger = - \gamma^1
\end{align}
またカイラリティ $\gamma^5$ は次を満たす。
\begin{align}
  (\gamma^5)^\dagger = \gamma^5, \qquad (\gamma^5)^2 = 1, \qquad \lbrace\gamma^\mu, \gamma^5\rbrace = 0
\end{align}
$\gamma^5$ を $\gamma^0, \gamma^1$ を用いて表わせ。
\end{problem}
\begin{proof}
  カイラリティがガンマ行列の複素数係数多項式で表されるとするとガンマ行列の性質より次のように書ける。
  \begin{align}
    \gamma^5 & = \sum_{e_0, e_1}a_{e_0,e_1}(\gamma^0)^{e_0}(\gamma^1)^{e_1} = \alpha_0 + \alpha_1\gamma^0 + \alpha_2\gamma^1 + \alpha_3\gamma^0\gamma^1 & \ab(a_{e_0, e_1}, \alpha_i\in\CC)
  \end{align}
  これを代入するとガンマ行列の直交性より
  \begin{align}
    (\gamma^5)^\dagger = \gamma^5          & \iff \alpha_0\in\RR, \alpha_1\in\RR, \alpha_2\in i\RR, \alpha_3\in\RR \\
    \lbrace\gamma^\mu, \gamma^5\rbrace = 0 & \implies \alpha_0 = \alpha_1 = \alpha_2 = 0                           \\
    (\gamma^5)^2 = 1                       & \implies \alpha_4 = \pm 1
  \end{align}
  となる。よって $\gamma^5 = \pm \gamma^0\gamma^1$ となる。ここでは特に $\gamma^5 = \gamma^0\gamma^1$ とする。
\end{proof}

\begin{problem}
$\gamma_\pm = \dfrac{1 \pm \gamma^5}{2}$ とするとき $(\gamma_+)^a$, $(\gamma_-)^b$, $(\gamma_+)^a(\gamma_-)^b$, $(\gamma_-)^b(\gamma_+)^a$ ($a, b\in\ZZ_{>0}$) を $\gamma_\pm$ を用いて表わせ。
\end{problem}
\begin{proof}
  \begin{align}
    (\gamma_\pm)^2   & = \frac{1 \pm 2\gamma^5 + (\gamma^5)^2}{2^2} = \frac{1 \pm \gamma^5}{2} = \gamma_\pm \\
    \gamma_+\gamma_- & = \gamma_-\gamma_+ = \frac{1 + \gamma^5 - \gamma^5 - (\gamma^5)^2}{2^2} = 0
  \end{align}
  より帰納法から次が示せる。
  \begin{align}
    (\gamma_+)^a = \gamma_+, \qquad (\gamma_-)^b = \gamma_-, \qquad (\gamma_+)^a(\gamma_-)^b = 0, \qquad (\gamma_-)^b(\gamma_+)^a = 0
  \end{align}
\end{proof}

\begin{problem}
$\psi_{\pm}(x) = \gamma_{\pm}\psi(x)$ は $\gamma^5$ の固有関数である。それぞれの固有値を求めよ。
\end{problem}
\begin{proof}
  カイラリティを作用させることで固有関数 $\psi_\pm(x)$ の固有値は $\pm 1$ となる。
  \begin{align}
    \gamma^5\psi_\pm(x) & = \gamma^5\gamma_\pm\psi(x) = \frac{\gamma^5 \pm 1}{2}\psi(x) = \pm\gamma^5\psi(x)
  \end{align}
\end{proof}

\begin{problem}
$\psi_\pm(x)$ が満たす連立微分方程式をディラック方程式から求めよ。
\end{problem}
\begin{proof}
  $\lbrace\gamma^\mu, \gamma^5\rbrace = 0$ より $\gamma^\mu\gamma_\pm = \gamma_\mp\gamma^\mu$ となる。よって
  \begin{align}
         & \begin{dcases}
             \gamma_+(i\hbar\gamma^\mu\partial_\mu - mc)\psi(x) = 0 \\
             \gamma_-(i\hbar\gamma^\mu\partial_\mu - mc)\psi(x) = 0 \\
           \end{dcases} \\
    \iff & \begin{dcases}
             i\hbar\gamma^\mu\partial_\mu\psi_-(x) = mc\psi_+(x) \\
             i\hbar\gamma^\mu\partial_\mu\psi_+(x) = mc\psi_-(x) \\
           \end{dcases}
  \end{align}
  となる。
\end{proof}

\begin{problem}
$m = 0$ の場合に $\psi_+(x) \propto e^{-iEt/\hbar+ipx/\hbar}$ が解となるとき、$E, p$ が満たす関係式を求めよ。
\end{problem}
\begin{proof}
  $m = 0$ のとき $i\hbar\gamma^\mu\partial_\mu\psi_+(x) = 0$ となる。 $\gamma^1 = \gamma^0(\gamma_+ - \gamma_-)$ より
  \begin{align}
    i\hbar\gamma^\mu\partial_\mu\psi_+(x) & = i\hbar(\gamma^0c\partial_t + \gamma^1\partial_x)\psi_+(x) \\
                                          & = (\gamma^0Ec - \gamma^1p)\psi_+(x)                         \\
                                          & = \gamma^0(Ec - (\gamma_+ - \gamma_-)p)\psi_+(x)            \\
                                          & = \gamma^0(Ec - p)\psi_+(x) = 0
  \end{align}
  となる。よって $Ec = p$ を満たす。
\end{proof}

\begin{problem}
$m = 0$ の場合に $\psi_-(x) \propto e^{-iEt/\hbar+ipx/\hbar}$ が解となるとき、$E, p$ が満たす関係式を求めよ。
\end{problem}
\begin{proof}
  $m = 0$ のとき $i\hbar\gamma^\mu\partial_\mu\psi_-(x) = 0$ となる。 $\gamma^1 = \gamma^0(\gamma_+ - \gamma_-)$ より
  \begin{align}
    i\hbar\gamma^\mu\partial_\mu\psi_-(x) & = i\hbar(\gamma^0c\partial_t + \gamma^1\partial_x)\psi_-(x) \\
                                          & = (\gamma^0Ec - \gamma^1p)\psi_-(x)                         \\
                                          & = \gamma^0(Ec - (\gamma_+ - \gamma_-)p)\psi_-(x)            \\
                                          & = \gamma^0(Ec + p)\psi_-(x) = 0
  \end{align}
  となる。よって $Ec = -p$ を満たす。
\end{proof}

\begin{problem}
$\gamma^0, \gamma^1, \gamma^5$ を 2 行 2 列の行列とするとき、それらの具体形をパウリ行列を用いて表せ。
\end{problem}
\begin{proof}
  パウリ行列を次のように定義する。
  \begin{align}
    \sigma_1 = \begin{pmatrix}
                 0 & 1 \\
                 1 & 0 \\
               \end{pmatrix}, \qquad
    \sigma_2 = \begin{pmatrix}
                 0 & -i \\
                 i & 0  \\
               \end{pmatrix}, \qquad
    \sigma_3 = \begin{pmatrix}
                 1 & 0  \\
                 0 & -1 \\
               \end{pmatrix}
  \end{align}
  これより次のようにおくとそれぞれの性質を満たす。
  \begin{align}
    \gamma^0 = \sigma_1, \qquad \gamma^1 = i\sigma_2, \qquad \gamma^5 = -\sigma_3
  \end{align}
\end{proof}

\section{指数関数}
\renewcommand{\AA}{\hat{A}}
\newcommand{\BB}{\hat{B}}
\begin{problem}
次の式を示せ。
\begin{align}
  e^{i\BB}\AA e^{-i\BB} = \sum_{n=0}^{\infty}\frac{i^n}{n!}\underbrace{[\BB, \ldots [\BB, [\BB}_{n}, \AA]] \ldots]
\end{align}
\end{problem}
\begin{proof}
  $e^{i\lambda\BB}\AA e^{-i\lambda\BB}$ について考える。これを $\lambda$ について展開すると
  \begin{align}
    e^{i\lambda\BB}\AA e^{-i\lambda\BB} & = \sum_{n=0}^{\infty}\frac{\lambda^n}{n!}\ab[\diff[n]{}{\lambda}e^{i\lambda\BB}\AA e^{i\lambda\BB}]_{\lambda = 0}                                         \\
                                        & = \sum_{n=0}^{\infty}\frac{\lambda^n}{n!}\ab[\diff[n-1]{}{\lambda}e^{i\lambda\BB}i[\BB, \AA]e^{i\lambda\BB}]_{\lambda = 0}                                \\
                                        & = \sum_{n=0}^{\infty}\frac{\lambda^n}{n!}\Big[e^{i\lambda\BB}i^n\underbrace{[\BB, \ldots [\BB, [\BB}_{n}, \AA]] \ldots]e^{i\lambda\BB}\Big]_{\lambda = 0} \\
                                        & = \sum_{n=0}^{\infty}\frac{(i\lambda)^n}{n!}\underbrace{[\BB, \ldots [\BB, [\BB}_{n}, \AA]] \ldots]
  \end{align}
  よって $\lambda = 1$ を代入することで示せる。
  \begin{align}
    e^{i\BB}\AA e^{-i\BB} & = \sum_{n=0}^{\infty}\frac{i^n}{n!}\underbrace{[\BB, \ldots [\BB, [\BB}_{n}, \AA]] \ldots]
  \end{align}
\end{proof}

\begin{problem}
$\partial$ を微分演算子とするとき、$(\partial e^{i\BB})e^{-i\BB} = -e^{i\BB}(\partial e^{-i\BB})$ を示せ。
\end{problem}
\begin{proof}
  $1 = e^{i\BB}e^{-i\BB}$ に微分演算子を作用させることで示せる。
  \begin{align}
    0 = \partial 1 = \partial(e^{i\BB}e^{-i\BB}) & = (\partial e^{i\BB})e^{-i\BB} + e^{i\BB}(\partial e^{-i\BB}) \\
    (\partial e^{i\BB})e^{-i\BB}                 & = -e^{i\BB}(\partial e^{-i\BB})
  \end{align}
\end{proof}

\begin{problem}
次の式を示せ。
\begin{align}
  e^{i\BB}(\partial e^{-i\BB}) = -\sum_{n=1}^{\infty}\frac{i^n}{n!}\underbrace{[\BB, \ldots [\BB, [\BB}_{n-1}, \partial\BB]] \ldots]
\end{align}
\end{problem}
\begin{proof}
  (1) において $\AA = \partial$ を代入して示せる。
  \begin{align}
    e^{i\BB}(\partial e^{-i\BB}) & = \sum_{n=0}^{\infty}\frac{i^n}{n!}\underbrace{[\BB, \ldots [\BB, [\BB}_{n}, \partial]] \ldots]            \\
                                 & = \partial + \sum_{n=1}^{\infty}\frac{i^n}{n!}\underbrace{[\BB, \ldots [\BB, [\BB}_{n}, \partial]] \ldots] \\
                                 & = -\sum_{n=1}^{\infty}\frac{i^n}{n!}\underbrace{[\BB, \ldots [\BB, [\BB}_{n-1}, \partial\BB]] \ldots]      \\
  \end{align}
\end{proof}

\section{スピノル球関数}
\begin{problem}
スピノル球関数を球面調和関数を用いて次のように定義する。
\begin{align}
  \mathcal{Y}_{j, m}^\pm(\theta, \phi) & = \frac{1}{\sqrt{2l + 1}}\begin{pmatrix}
                                                                    \sqrt{l + \frac{1}{2} \pm m}Y_l^{m - 1/2}(\theta, \phi)    \\
                                                                    \pm\sqrt{l + \frac{1}{2} \mp m}Y_l^{m + 1/2}(\theta, \phi) \\
                                                                  \end{pmatrix}_{j = l\pm 1/2}
\end{align}
軌道角運動量 $\hat{\bm{L}}$, スピン角運動量 $\hat{\bm{S}}$, 全角運動量 $\hat{\bm{J}}$ とする。

$\mathcal{Y}_{j, m}^\pm(\theta, \phi)$ が持つパリティを求めよ。
\end{problem}
\begin{proof}
  球面調和関数におけるパリティは $(-1)^l$ となるからスピノル球関数のパリティは $(-1)^l$ となる。
  \begin{align}
    Y_l^m(\pi - \theta, \pi + \phi)                  & = (-1)^lY_l^m(\theta, \phi)                  \\
    \mathcal{Y}_{j, m}^\pm(\pi - \theta, \pi + \phi) & = (-1)^l\mathcal{Y}_{j, m}^\pm(\theta, \phi)
  \end{align}
\end{proof}

\begin{problem}
$\mathcal{Y}_{j, m}^\pm(\theta, \phi)$ が持つ $\hat{J}_z$ の固有値を求めよ。
\end{problem}
\begin{proof}
  球面調和関数における固有値からスピノル球関数の $\hat{J}_z$ の固有値は $m\hbar$ となる。
  \begin{align}
    \hat{J}_zY_l^m(\theta, \phi)                  & = m\hbar Y_l^m(\theta, \phi)                  \\
    \hat{J}_z\mathcal{Y}_{j, m}^\pm(\theta, \phi) & = m\hbar \mathcal{Y}_{j, m}^\pm(\theta, \phi)
  \end{align}
\end{proof}

\begin{problem}
$\mathcal{Y}_{j, m}^\pm(\theta, \phi)$ が持つ $\hat{\bm{L}}^2$ の固有値を求めよ。
\end{problem}
\begin{proof}
  球面調和関数における固有値からスピノル球関数の $\hat{\bm{L}}^2$ の固有値は $\hbar^2l(l+1)$ となる。
  \begin{align}
    \hat{\bm{L}}^2Y_l^m(\theta, \phi)                  & = \hbar^2l(l+1) Y_l^m(\theta, \phi)                  \\
    \hat{\bm{L}}^2\mathcal{Y}_{j, m}^\pm(\theta, \phi) & = \hbar^2l(l+1) \mathcal{Y}_{j, m}^\pm(\theta, \phi)
  \end{align}
\end{proof}

\begin{problem}
$\mathcal{Y}_{j, m}^\pm(\theta, \phi)$ が持つ $\hat{\bm{L}}\cdot\hat{\bm{S}}$ の固有値を求めよ。
\end{problem}
\begin{proof}
  $\mathcal{Y}_{j, m}^\pm(\theta, \phi)$ に $\hat{\bm{L}}\cdot\hat{\bm{S}}$ を作用させると
  \begin{align}
    \hat{\bm{L}}\cdot\hat{\bm{S}}\mathcal{Y}_{j, m}^\pm(\theta, \phi) & = \frac{1}{2}(\hat{\bm{J}}^2 - \hat{\bm{L}}^2 - \hat{\bm{S}}^2)\mathcal{Y}_{j, m}^\pm(\theta, \phi)                             \\
                                                                      & = \frac{\hbar^2}{2}\ab(j(j+1) - l(l+1) - \frac{3}{4})\mathcal{Y}_{j, m}^\pm(\theta, \phi)                                       \\
                                                                      & = \frac{\hbar^2}{2}\ab(\ab(l\pm \frac{1}{2})\ab(l\pm \frac{1}{2}+1) - l(l+1) - \frac{3}{4})\mathcal{Y}_{j, m}^\pm(\theta, \phi) \\
                                                                      & = \frac{\hbar^2}{2}\ab(\pm\ab(l + \frac{1}{2}) - \frac{1}{2})\mathcal{Y}_{j, m}^\pm(\theta, \phi)                               \\
  \end{align}
  より固有値は $\dfrac{\hbar^2l}{2}, -\dfrac{\hbar^2(l+1)}{2}$ となる。
\end{proof}

\begin{problem}
$\mathcal{Y}_{j, m}^\pm(\theta, \phi)$ が持つ $\hat{\bm{J}}^2$ の固有値を求めよ。
\end{problem}
\begin{proof}
  $\mathcal{Y}_{j, m}^\pm(\theta, \phi)$ に $\hat{\bm{J}}^2$ を作用させると
  \begin{align}
    \hat{\bm{J}}^2\mathcal{Y}_{j, m}^\pm(\theta, \phi) & = \hbar^2j(j+1) \mathcal{Y}_{j, m}^\pm(\theta, \phi)                                                   \\
                                                       & = \hbar^2\ab(l \pm \frac{1}{2})\ab(\ab(l \pm \frac{1}{2}) + 1)\mathcal{Y}_{j, m}^\pm(\theta, \phi)     \\
                                                       & = \hbar^2\ab(\ab(l \pm \frac{1}{2} + \frac{1}{2})^2 - \frac{1}{4})\mathcal{Y}_{j, m}^\pm(\theta, \phi)
  \end{align}
  より固有値は $l^2 - \dfrac{1}{4}, (l + 1)^2 - \dfrac{1}{4}$ となる。
\end{proof}

\begin{problem}
パウリ行列 $\bm{\sigma}$ と位置ベクトル $\rr = r(\sin\theta\cos\phi, \sin\theta\sin\phi, \cos\theta)$ に対して $\bm{\sigma}\cdot\dfrac{\rr}{r}\mathcal{Y}_{j, m}^\pm(\theta, \phi)$ を計算し、スピノル球関数のみを用いて表せ。
\end{problem}
\begin{proof}
  まず演算子を計算すると
  \begin{align}
    \bm{\sigma}\cdot\dfrac{\rr}{r} & = \bm{\sigma}\cdot (\sin\theta\cos\phi, \sin\theta\sin\phi, \cos\theta) =
    \begin{pmatrix}
      \cos\theta           & \sin\theta e^{-i\phi} \\
      \sin\theta e^{i\phi} & -\cos\theta           \\
    \end{pmatrix}
  \end{align}
  となる。三角関数を球面調和関数に作用させたときの固有値は次のようになるから
  \begin{align}
    \cos\theta Y_{l}^m(\theta, \phi)            & = \sqrt{\frac{(l + m + 1)(l - m + 1)}{(2l + 1)(2l + 3)}}Y_{l+1}^m(\theta, \phi) + \sqrt{\frac{(l + m)(l - m)}{(2l + 1)(2l - 1)}}Y_{l-1}^m(\theta, \phi)              \\
    \sin\theta e^{i\phi} Y_{l}^m(\theta, \phi)  & = -\sqrt{\frac{(l + m + 1)(l + m + 2)}{(2l + 1)(2l + 3)}}Y_{l+1}^{m+1}(\theta, \phi) + \sqrt{\frac{(l - m)(l - m - 1)}{(2l + 1)(2l - 1)}}Y_{l-1}^{m+1}(\theta, \phi) \\
    \sin\theta e^{-i\phi} Y_{l}^m(\theta, \phi) & = \sqrt{\frac{(l - m + 1)(l - m + 2)}{(2l + 1)(2l + 3)}}Y_{l+1}^{m-1}(\theta, \phi) + \sqrt{\frac{(l + m)(l + m - 1)}{(2l + 1)(2l - 1)}}Y_{l-1}^{m-1}(\theta, \phi)
  \end{align}
  次のように計算できる。
  \begin{alignat}{3}
    \bm{\sigma}\cdot\dfrac{\rr}{r}\mathcal{Y}_{j, m}^\pm(\theta, \phi) & = \frac{1}{\sqrt{2l + 1}}       &  &
    \begin{pmatrix}
      \cos\theta           & \sin\theta e^{-i\phi} \\
      \sin\theta e^{i\phi} & -\cos\theta           \\
    \end{pmatrix}
    \begin{pmatrix}
      \sqrt{l + \frac{1}{2} \pm m}Y_l^{m - 1/2}(\theta, \phi)    \\
      \pm\sqrt{l + \frac{1}{2} \mp m}Y_l^{m + 1/2}(\theta, \phi) \\
    \end{pmatrix}_{j = l\pm 1/2}                                                                                                                                                                                                             \\
                                                                       & = \frac{1}{\sqrt{2l + 1}}       &  & \Bigg(\sqrt{l + \frac{1}{2} \pm m}\cos\theta Y_l^{m - 1/2}(\theta, \phi) \pm\sqrt{l + \frac{1}{2} \mp m}\sin\theta e^{-i\phi}Y_l^{m + 1/2}(\theta, \phi)                 \\
                                                                       &                                 &  & , \sqrt{l + \frac{1}{2} \pm m}\sin\theta e^{i\phi}Y_l^{m - 1/2}(\theta, \phi) \mp\sqrt{l + \frac{1}{2} \mp m}\cos\theta Y_l^{m + 1/2}(\theta, \phi)\Bigg)_{j = l\pm 1/2} \\
                                                                       & = \frac{1}{\sqrt{2l + 1}}       &  & \Bigg(\sqrt{l + \frac{1}{2} \pm m}\sqrt{\frac{(l + m + \frac{1}{2})(l - m + \frac{3}{2})}{(2l + 1)(2l + 3)}}Y_{l+1}^{m-1/2}(\theta, \phi)                                \\
                                                                       &                                 &  & + \sqrt{l + \frac{1}{2} \pm m}\sqrt{\frac{(l + m - \frac{1}{2})(l - m + \frac{1}{2})}{(2l + 1)(2l - 1)}}Y_{l-1}^{m-1/2}(\theta, \phi)                                    \\
                                                                       &                                 &  & \pm\sqrt{l + \frac{1}{2} \mp m}\sqrt{\frac{(l - m + \frac{1}{2})(l - m + \frac{3}{2})}{(2l + 1)(2l + 3)}}Y_{l+1}^{m-1/2}(\theta, \phi)                                   \\
                                                                       &                                 &  & \pm\sqrt{l + \frac{1}{2} \mp m}\sqrt{\frac{(l + m + \frac{1}{2})(l + m - \frac{1}{2})}{(2l + 1)(2l - 1)}}Y_{l-1}^{m-1/2}(\theta, \phi)                                   \\
                                                                       &                                 &  & , -\sqrt{l + \frac{1}{2} \pm m}\sqrt{\frac{(l + m + \frac{1}{2})(l + m + \frac{3}{2})}{(2l + 1)(2l + 3)}}Y_{l+1}^{m+1/2}(\theta, \phi)                                   \\
                                                                       &                                 &  & + \sqrt{l + \frac{1}{2} \pm m}\sqrt{\frac{(l - m + \frac{1}{2})(l - m - \frac{1}{2})}{(2l + 1)(2l - 1)}}Y_{l-1}^{m+1/2}(\theta, \phi)                                    \\
                                                                       &                                 &  & \mp\sqrt{l + \frac{1}{2} \mp m}\sqrt{\frac{(l + m + \frac{3}{2})(l - m + \frac{1}{2})}{(2l + 1)(2l + 3)}}Y_{l+1}^{m+1/2}(\theta, \phi)                                   \\
                                                                       &                                 &  & \mp\sqrt{l + \frac{1}{2} \mp m}\sqrt{\frac{(l + m + \frac{1}{2})(l - m - \frac{1}{2})}{(2l + 1)(2l - 1)}}Y_{l-1}^{m+1/2}(\theta, \phi)\Bigg)_{j = l\pm 1/2}              \\
                                                                       & = \frac{1}{\sqrt{2l + 1}}       &  & \Bigg((2l + 1)\sqrt{\frac{l \mp m + \frac{1}{2} \pm 1}{(2l + 1)(2l + 1 \pm 2)}}Y_{l\pm 1}^{m-1/2}(\theta, \phi)                                                          \\
                                                                       &                                 &  & ,\mp(2l + 1)\sqrt{\frac{l \pm m + \frac{1}{2} \pm 1}{(2l + 1)(2l + 1 \pm 2)}}Y_{l\pm 1}^{m+1/2}(\theta, \phi)\Bigg)_{j = l\pm 1/2}                                       \\
                                                                       & = \frac{1}{\sqrt{2j + 1 \pm 1}} &  & \Bigg(\sqrt{j + \frac{1}{2} \mp \ab(m - \frac{1}{2})}Y_{j\pm 1/2}^{m-1/2}(\theta, \phi)                                                                                  \\
                                                                       &                                 &  & ,\mp\sqrt{j + \frac{1}{2} \pm \ab(m + \frac{1}{2})}Y_{j\pm 1/2}^{m+1/2}(\theta, \phi)\Bigg) = \mathcal{Y}_{j, m}^\mp(\theta, \phi)
  \end{alignat}
  よって次の式となる。
  \begin{align}
    \bm{\sigma}\cdot\dfrac{\rr}{r}\mathcal{Y}_{j, m}^\pm(\theta, \phi) & = \mathcal{Y}_{j, m}^\mp(\theta, \phi)
  \end{align}
\end{proof}

\section{水素原子における電子のエネルギー準位}
\begin{problem}
中心力ポテンシャル $V(r) = -\dfrac{\alpha\hbar c}{r}$ のもとでディラック方程式を解くことにより得られる水素原子中の電子のエネルギー準位を考える。

主量子数 $n$ が与えられたとき、全角運動量 $j$ が取り得る値を答えよ。
\end{problem}
\begin{proof}
  $n$ と $j$ に関して $n = j + n' + 1/2$ という関係があるから $j = 1/2,\ldots,(2n-1)/2$ を取る。
\end{proof}

\begin{problem}
主量子数 $n$, 全角運動量 $j$ を持つ状態のエネルギー固有値の表式を書き下せ。また、そのエネルギー固有値の縮重度を答えよ。
\end{problem}
\begin{proof}
  主量子数 $n$, 全角運動量 $j$ を持つ状態のエネルギー固有値の表式は次のようになる。
  \begin{align}
    E & = \frac{mc^2}{\sqrt{1 + \frac{(Z\alpha)^2}{\ab(n - \ab(j + \frac{1}{2}) + \sqrt{(j + \frac{1}{2})^2 - (Z\alpha)^2})^2}}}
  \end{align}
  また縮重度は $j = l \pm 1/2$ より $n' = 0$ において $2j + 1$、$n' > 0$ において $2(2j + 1)$ となる。
\end{proof}

\begin{problem}
主量子数 $n$ を持つ状態の総数を求めよ。
\end{problem}
\begin{proof}
  $n = j + n' + 1/2$ と $j = l \pm 1/2$ より状態の総数は $2n^2$ となる。
  \begin{align}
    2n + 2\times\sum_{n'=1}^{n-1}2(n-n') & = 2n^2
  \end{align}
\end{proof}

\begin{problem}
電子の静止エネルギーから測った束縛エネルギーの大きさが縮退を除いて 7 番目と 10 番目に大きい準位の主量子数 $n$ と全角運動量 $j$ をそれぞれ答えよ。また、それらの準位の束縛エネルギーの大きさを有効数字 6 桁で求めよ。
\end{problem}
\begin{proof}
  7 番目に大きい準位は $n = 4, j = 1/2$ で 10 番目に大きい準位は $n = 4, j = 7/2$ である。
  またそれぞれの束縛エネルギーは
  \begin{align}
    \mathcal{E}         & \approx -\frac{\alpha^2 mc^2}{2n^2} - \ab(\frac{1}{j + 1/2} - \frac{3}{4n})\frac{\alpha^4mc^2}{2n^3}   \\
                        & \approx -\frac{13.60569}{n^2} - \ab(\frac{1}{j + 1/2} - \frac{3}{4n})\frac{7.249022\times10^{-4}}{n^3} \\
    \mathcal{E}_{4,1/2} & \approx 0.850365                                                                                       \\
    \mathcal{E}_{4,7/2} & \approx 0.850356
  \end{align}
  となる。
\end{proof}

\begin{problem}
同じ主量子数 $n$ を持つ状態でも全角運動量 $j$ に依存してエネルギー準位が分裂する. この現象を表す名称を答えよ.
\end{problem}
\begin{proof}
  微細構造
\end{proof}

\end{document}