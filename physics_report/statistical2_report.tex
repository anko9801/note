\RequirePackage{plautopatch}
\documentclass[uplatex,dvipdfmx,a4paper,11pt]{jlreq}
\usepackage{bxpapersize}
\usepackage[utf8]{inputenc}
\usepackage{fontenc}
\usepackage{lmodern}
\usepackage{otf}
\usepackage{amsmath}
\usepackage{amssymb}
\usepackage{amsthm}
\usepackage{ascmac}
% \usepackage[hyphens]{url}
\usepackage{physics2}
\usephysicsmodule{ab, ab.braket, doubleprod, diagmat, xmat}
\usepackage{diffcoeff}
% \usepackage{braket}
\usepackage{verbatimbox}
\usepackage{bm}
\usepackage{url}
% \usepackage[dvipdfmx,hiresbb,final]{graphicx}
\usepackage{hyperref}
\usepackage{pxjahyper}
\usepackage{tikz}\usetikzlibrary{cd}
\usepackage{listings}
\usepackage{color}
\usepackage{mathtools}
\usepackage{xspace}
\usepackage{xy}
\usepackage{xypic}
%
\title{}
\author{anko9801}
\makeatletter
%
\DeclareMathOperator{\lcm}{lcm}
\DeclareMathOperator{\Kernel}{Ker}
\DeclareMathOperator{\Image}{Im}
\DeclareMathOperator{\ch}{ch}
\DeclareMathOperator{\Aut}{Aut}
\DeclareMathOperator{\Log}{Log}
\DeclareMathOperator{\Arg}{Arg}
\DeclareMathOperator{\sgn}{sgn}
%
\newcommand{\CC}{\mathbb{C}}
\newcommand{\RR}{\mathbb{R}}
\newcommand{\QQ}{\mathbb{Q}}
\newcommand{\ZZ}{\mathbb{Z}}
\newcommand{\NN}{\mathbb{N}}
\newcommand{\FF}{\mathbb{F}}
\newcommand{\PP}{\mathbb{P}}
\newcommand{\GG}{\mathbb{G}}
\newcommand{\TT}{\mathbb{T}}
\newcommand{\EE}{\bm{E}}
\newcommand{\BB}{\bm{B}}
\renewcommand{\AA}{\bm{A}}
\newcommand{\rr}{\bm{r}}
\newcommand{\kk}{\bm{k}}
\newcommand{\pp}{\bm{p}}
\newcommand{\calB}{\mathcal{B}}
\newcommand{\calF}{\mathcal{F}}
\newcommand{\ignore}[1]{}
\newcommand{\floor}[1]{\left\lfloor #1 \right\rfloor}
% \newcommand{\abs}[1]{\left\lvert #1 \right\rvert}
\newcommand{\lt}{<}
\newcommand{\gt}{>}
\newcommand{\id}{\mathrm{id}}
\newcommand{\rot}{\curl}
\newcommand{\vnabla}{\mathbf{\nabla}}
\newcommand{\laplacian}{\nabla^2}
\renewcommand{\angle}[1]{\left\langle #1 \right\rangle}
\newcommand\mqty[1]{\begin{pmatrix}#1\end{pmatrix}}
\newcommand\vmqty[1]{\begin{vmatrix}#1\end{vmatrix}}
\numberwithin{equation}{section}

\let\oldcite=\cite
\renewcommand\cite[1]{\hyperlink{#1}{\oldcite{#1}}}

\let\oldbibitem=\bibitem
\renewcommand{\bibitem}[2][]{\label{#2}\oldbibitem[#1]{#2}}

% theorem環境の設定
% - 冒頭に改行
% - 末尾にdiamond (amsthm)
\theoremstyle{definition}
\newcommand*{\newscreentheoremx}[2]{
  \newenvironment{#1}[1][]{
    \begin{screen}
    \begin{#2}[##1]
      \leavevmode
      \newline
  }{
    \end{#2}
    \end{screen}
  }
}
\newcommand*{\newqedtheoremx}[2]{
  \newenvironment{#1}[1][]{
    \begin{#2}[##1]
      \leavevmode
      \newline
      \renewcommand{\qedsymbol}{\(\diamond\)}
      \pushQED{\qed}
  }{
      \qedhere
      \popQED
    \end{#2}
  }
}
\newtheorem{theorem*}{定理}[subsection]

\newqedtheoremx{theorem}{theorem*}
\newcommand*\newqedtheorem@unstarred[2]{%
  \newtheorem{#1*}[theorem*]{#2}
  \newqedtheoremx{#1}{#1*}
}
\newcommand*\newqedtheorem@starred[2]{%
  \newtheorem*{#1*}{#2}
  \newqedtheoremx{#1}{#1*}
}
\newcommand*{\newqedtheorem}{\@ifstar{\newqedtheorem@starred}{\newqedtheorem@unstarred}}

\newtheorem{sctheorem*}{定理}[section]
\newscreentheoremx{sctheorem}{sctheorem*}
\newcommand*\newscreentheorem@unstarred[2]{%
  \newtheorem{#1*}[theorem*]{#2}
  \newscreentheoremx{#1}{#1*}
}
\newcommand*\newscreentheorem@starred[2]{%
  \newtheorem*{#1*}{#2}
  \newscreentheoremx{#1}{#1*}
}
\newcommand*{\newscreentheorem}{\@ifstar{\newscreentheorem@starred}{\newscreentheorem@unstarred}}

%\newtheorem*{definition}{定義}
%\newtheorem{theorem}{定理}
%\newtheorem{proposition}[theorem]{命題}
%\newtheorem{lemma}[theorem]{補題}
%\newtheorem{corollary}[theorem]{系}

\newqedtheorem{lemma}{補題}
\newqedtheorem{corollary}{系}
\newqedtheorem{example}{例}
\newqedtheorem{proposition}{命題}
\newqedtheorem{remark}{注意}
\newqedtheorem{thesis}{主張}
\newqedtheorem{notation}{記法}
\newqedtheorem{problem}{問題}
\newqedtheorem{algorithm}{アルゴリズム}

\newscreentheorem*{axiom}{公理}
\newscreentheorem*{definition}{定義}

\renewenvironment{proof}[1][\proofname]{\par
  \normalfont
  \topsep6\p@\@plus6\p@ \trivlist
  \item[\hskip\labelsep{\bfseries #1}\@addpunct{\bfseries}]\ignorespaces\quad\par
}{%
  \qed\endtrivlist\@endpefalse
}
\renewcommand\proofname{証明}

\makeatother

\begin{document}
\maketitle
\tableofcontents
\clearpage

\section{熱力学の復習、古典・量子統計力学の復習、グランドカノニカル分布の基礎}
\setcounter{subsection}{3}
\subsection{物理数学の復習}
\begin{problem}
ゼータ関数について $\zeta(2), \zeta(4)$ を求めよ。
\begin{align}
  \zeta(x) := \sum_{n=1}^{\infty}\frac{1}{n^x} \qquad (x > 1)
\end{align}
\end{problem}
\begin{proof}
  $f_m(x) = x^m$ をフーリエ展開する。
  \begin{align}
    f_m(x) = x^m = \sum_{n=-\infty}^{\infty}c_{n,m}e^{inx}
  \end{align}
  このときの係数は次のようになる。
  \begin{align}
    c_{n,m} & = \frac{1}{2\pi}\int_{-\pi}^{\pi}x^me^{-inx}\dl{x}                                                                                              \\
            & = \frac{1}{2\pi}\sum_{i=0}^{4}\ab[\frac{m!(-1)^i}{(m-i)!(-in)^{i+1}}x^{m-i}e^{-inx}]_{-\pi}^{\pi}                                               \\
            & = \frac{1}{2\pi}\ab(-\frac{m}{(-in)^{2}}(-1)^n\ab(\pi^{m-1} - (-\pi^{m-1})) - \frac{m(m-1)(m-2)}{(-in)^{4}}(-1)^n\ab(\pi^{m-3} - (-\pi)^{m-3})) \\
            & = (-1)^n\ab(\frac{m}{n^2}\pi^{m-2} - \frac{m(m-1)(m-2)}{n^4}\pi^{m-4})                                                                          \\
    c_{0,m} & = \frac{1}{2\pi}\int_{-\pi}^\pi x^m\dl{x} = \frac{\pi^m}{m+1}
  \end{align}
  これより $f_m(x)$ が求まり、$x = \pi$ を代入することでゼータ関数の値が分かる。
  \begin{align}
    f_m(x)           & = \frac{\pi^m}{m+1} + \sum_{n\neq 0}\ab(\frac{m}{n^2}\pi^{m-2} - \frac{m(m-1)(m-2)}{n^4}\pi^{m-4})(-1)^ne^{inx}                                     \\
    f_2(\pi) = \pi^2 & = \frac{\pi^2}{3} + \sum_{n\neq 0}\frac{2}{n^2}, \qquad f_4(\pi) = \pi^4 = \frac{\pi^4}{5} + \sum_{n\neq 0}\ab(\frac{4}{n^2}\pi^2 - \frac{24}{n^4}) \\
    \zeta(2)         & = \sum_{n=1}^\infty\frac{1}{n^2} = \frac{\pi^2}{6}, \qquad \zeta(4) = \sum_{n=1}^\infty\frac{1}{n^4} = \frac{\pi^4}{90}
  \end{align}
\end{proof}

\begin{problem}
次の関数 $I_\pm(\alpha)$ が収束する実数 $\alpha$ の範囲とその収束値を求めよ。
\begin{align}
  I_\pm(\alpha) & = \int_0^\infty\frac{z^{\alpha - 1}}{e^z \pm 1}\dl{z}
\end{align}
\end{problem}
\begin{proof}
  \begin{align}
    I_\pm(\alpha) & = \int_0^\infty\frac{z^{\alpha - 1}}{e^z \pm 1}\dl{z}                                                                                                          \\
                  & = \int_0^\infty z^{\alpha - 1}(\pm 1 - e^{-z} \pm e^{-2z} - \cdots)\dl{z}                                                                                      \\
                  & = \pm\ab[\frac{z^\alpha}{\alpha}]_0^\infty - \int_0^\infty z^{\alpha - 1}e^{-z}\dl{z} \pm \int_0^\infty z^{\alpha - 1}e^{-2z}\dl{z} - \cdots                   \\
                  & = \pm\ab[\frac{z^\alpha}{\alpha}]_0^\infty - \ab(1 \mp \frac{1}{2^\alpha} + \frac{1}{3^\alpha} \mp \cdots)\int_0^\infty z^{\alpha - 1}e^{-z}\dl{z} & (z' = kz) \\
                  & = \begin{dcases}
                        +\ab[\frac{z^\alpha}{\alpha}]_0^\infty - (1 - 2^{1 - \alpha})\zeta(\alpha)\Gamma(\alpha) \\
                        -\ab[\frac{z^\alpha}{\alpha}]_0^\infty - \zeta(\alpha)\Gamma(\alpha)                     \\
                      \end{dcases}
  \end{align}
  これより $|\alpha| \geq 1$ のとき $I_{\pm}(\alpha)$ は発散し、$|\alpha| < 1$ のとき次のようになる。
  \begin{align}
    I_+(\alpha) & = - (1 - 2^{1 - \alpha})\zeta(\alpha)\Gamma(\alpha) \\
    I_-(\alpha) & = - \zeta(\alpha)\Gamma(\alpha)
  \end{align}
\end{proof}

\setcounter{subsection}{5}
\subsection{3 次元調和振動子}
\begin{align}
  \hat{H} & = -\frac{\hbar^2}{2m}\laplacian + V(\rr), \qquad V(\rr) = \frac{k}{2}|\rr|^2
\end{align}
\begin{problem}
固有関数を $\psi(\rr) = X(x)Y(y)Z(z)$ と変数分離できるとすると固有エネルギーを求めよ。
\end{problem}
\begin{proof}
  \begin{align}
    \hat{H}\psi & = -\frac{\hbar^2}{2m}\laplacian\psi + V(\rr)\psi                                                               \\
                & = -\frac{\hbar^2}{2m}(X''YZ + XY''Z + XYZ'') + V(\rr)XYZ                                                       \\
                & = \ab(-\frac{\hbar^2}{2m}\ab(\frac{X''}{X} + \frac{Y''}{Y} + \frac{Z''}{Z}) + \frac{1}{2}(x^2 + y^2 + z^2))XYZ \\
                & = \sum_{i}\ab(-\frac{\hbar^2}{2m}\frac{X_i''(x_i)}{X_i(x_i)} + \frac{k}{2}x_i^2)\psi = E\psi
  \end{align}
  総和の各項はそれぞれ変数が独立しているから定数となり、それぞれ $E_i$ とおく。
  \begin{align}
    -\frac{\hbar^2}{2m}\frac{X_i''(x_i)}{X_i(x_i)} + \frac{k}{2}x_i^2 & = E_i    \\
    -\frac{\hbar^2}{2m}X_i'' + \frac{k}{2}x_i^2X_i                    & = E_iX_i
  \end{align}
  これは 1 次元調和振動子のポテンシャルであるので固有エネルギーは次のようになる。
  \begin{align}
    E_{i,n}             & = \ab(n + \frac{1}{2})\hbar\omega               \\
    E_{(n_x, n_y, n_z)} & = \ab(n_x + n_y + n_z + \frac{3}{2})\hbar\omega
  \end{align}
\end{proof}

\begin{problem}
固有エネルギー $\varepsilon$ が $E_0 = 100\hbar\omega \leq \varepsilon < E_0 + \delta E = 110\hbar\omega$ を満たす独立な固有状態は何個あるか? まず ($\hbar\omega\ll \delta E \ll E_0$ として) 概数を評価する方法を考えて評価し、次に具体的に求めてみよう。
\end{problem}
\begin{proof}

\end{proof}

\begin{problem}
極座標において固有関数が $\psi(\rr) = R(r)Y(\theta, \phi) = R(r)\Theta(\theta)\Phi(\phi)$ と変数分離できるとき固有関数と固有エネルギーはどのように求められるか。
\begin{align}
  x          & = r\sin\theta\cos\phi, y = r\sin\theta\sin\phi, z = r\cos\theta                                                                                                        \\
  \laplacian & = \diffp[2]{}{r} + \frac{2}{r}\diffp{}{r} + \frac{1}{r^2}\ab(\frac{1}{\sin\theta}\diffp{}{\theta}\sin\theta\diffp{}{\theta} + \frac{1}{\sin^2\theta}\diffp[2]{}{\phi})
\end{align}
\end{problem}
\begin{proof}
  \begin{align}
    \hat{H} & = -\frac{\hbar^2}{2m}\laplacian + V(r)                                                                                                                                                                        \\
            & = -\frac{\hbar^2}{2m}\ab(\frac{1}{r^2}\diffp{}{r}\ab(r^2\diffp{}{r}) + \frac{1}{r^2\sin\theta}\diffp{}{\theta}\ab(\sin\theta\diffp{}{\theta}) + \frac{1}{r^2\sin^2\theta}\diffp[2]{}{\phi}) + V(r)            \\
    0       & = \ab(\diffp{}{r}\ab(r^2\diffp{}{r}) + \frac{1}{\sin\theta}\diffp{}{\theta}\ab(\sin\theta\diffp{}{\theta}) + \frac{1}{\sin^2\theta}\diffp[2]{}{\phi} + \frac{2m r^2(E - V(r))}{\hbar^2})\psi(r, \theta, \phi)
  \end{align}
  と書ける。$k = m\omega^2$ とおくと独立な変数であるから定数 $\lambda, m$ を用いて
  \begin{align}
     & \ab(\diffp{}{r}\ab(r^2\diffp{}{r}) + \frac{2m r^2}{\hbar^2}E - \frac{m^2\omega^2r^4}{\hbar^2})R(r) = \lambda R(r)                                             \\
     & \ab(\frac{1}{\sin\theta}\diffp{}{\theta}\ab(\sin\theta\diffp{}{\theta}) + \frac{1}{\sin^2\theta}\diffp[2]{}{\phi}) Y(\theta, \phi) = -\lambda Y(\theta, \phi) \\
     & \ab(\sin\theta\diffp{}{\theta}\ab(\sin\theta\diffp{}{\theta}) + \lambda \sin^2\theta)\Theta(\theta) = m^2\Theta(\theta)                                       \\
     & \diff[2]{\Phi(\phi)}{\phi} = -m^2\Phi(\phi)
  \end{align}
  となる。まず $\Phi(\phi)$ の一般解は次のようになる。
  \begin{align}
     & \diff[2]{\Phi(\phi)}{\phi} + m^2\Phi(\phi) = 0             \\
     & \Phi(\phi) = \begin{cases}
                      Ae^{i|m|\phi} + Be^{-i|m|\phi} & (m^2 \neq 0) \\
                      C\phi + D                      & (m^2 = 0)    \\
                    \end{cases}
  \end{align}
  波動関数は連続であるから $\Phi(0) = \Phi(2\pi)$ であり、規格化条件を満たす。$C = D = 0$ となる解は意味を成さず、$m\in\ZZ$ となる。$L_z$ の固有関数となることから
  \begin{align}
    \Phi(\phi) & = \frac{1}{\sqrt{2\pi}}e^{im\phi} \qquad (m\in\ZZ)
  \end{align}
  となる。次に $\Theta(\theta)$ について解く。$z = \cos\theta$ とおくと,
  \begin{align}
    \ab(\sin\theta\diff{}{\theta}\ab(\sin\theta\diff{}{\theta}) + \lambda \sin^2\theta)\Theta(\theta) & = m^2\Theta(\theta) \\
    \diff{}{z}\ab((1 - z^2)\diff{\Theta}{z}) + \ab(\lambda - \frac{m^2}{1 - z^2})\Theta(z)            & = 0
  \end{align}
  となる。$m = 0$ において $\Theta(z)$ はルジャンドルの微分方程式を満たす。$\Theta(z)$ をべき展開することで
  \begin{align}
     & (1 - z^2)\Theta'' - 2z\Theta' + \lambda\Theta = 0, \qquad \Theta(z) = \sum_{k = 0}^\infty a_kz^k                          \\
     & (1 - z^2)\sum_{k = 2}^\infty k(k-1)a_kz^{k-2} - 2z\sum_{k = 1}^\infty ka_kz^{k-1} + \lambda\sum_{k = 0}^\infty a_kz^k = 0 \\
     & \sum_{k = 0}^\infty \ab((k+1)(k+2)a_{k+2} + \ab(\lambda - k(k+1))a_k)z^k + \mathcal{O}(z) = 0                             \\
     & a_{k+2} = \frac{k(k+1) - \lambda}{(k+2)(k+1)}a_k
  \end{align}
  となる。よって $z$ について一般に発散しない為には $\lambda = l(l+1)\ (l\in\ZZ_{>0})$ とならければならない。すると $m\neq 0$ のときはルジャンドルの陪微分方程式となる。
  \begin{align}
    \diff{}{z}\ab((1 - z^2)\diff{\Theta}{z}) + \ab(l(l+1) - \frac{m^2}{1 - z^2})\Theta(z) & = 0
  \end{align}
  これよりルジャンドルの陪関数 $P_l^m(z)$ と規格化条件から $\Theta_{lm}(\theta)$ は
  \begin{align}
    \Theta_{lm}(\theta) & = (-1)^{\frac{m + |m|}{2}}\sqrt{\ab(l + \frac{1}{2})\frac{(l - |m|)!}{(l + |m|)!}}P_l^{|m|}(\cos\theta)
  \end{align}
  と書ける。また $R_l(r)$ については $\rho = \sqrt{\dfrac{m\omega}{\hbar}}r$ と無次元化すると
  \begin{align}
     & \diff[2]{}{r}R_l(r) + \frac{2}{r}\diff{}{r}R_l(r) + \frac{2m}{\hbar^2}\ab(E - \frac{1}{2}m\omega^2r^2 - \frac{l(l+1)\hbar^2}{2m r^2})R_l(r) = 0                                         \\
     & \diff[2]{}{\rho}R_l(\rho) + \frac{2}{\rho}\diff{}{\rho}R_l(\rho) + \ab(\lambda + \rho^2 - \frac{l(l+1)}{\rho^2})R_l(\rho) = 0                   & \ab(\lambda = \frac{2E}{\hbar\omega})
  \end{align}
  となる。$x = \rho^2$ と変数変換すると
  \begin{align}
     & x\diff[2]{}{x}R_l(x) + \frac{3}{2}\diff{}{x}R_l(x) + \frac{1}{4}\ab(\lambda + x - \frac{l(l+1)}{x})R_l(x) = 0
  \end{align}
  となり, 級数展開法より $\rho\to\infty$ で発散しない為には $n$ を非負整数として $\lambda = 4n + 2l + 3$ となる。
  $\rho\to\infty$, $\rho\to 0$ のときの漸近解はそれぞれ $e^{-x/2}$, $x^{l/2}$ となるので $R_l(x) = x^{l/2}e^{-x/2}S_n^\alpha(x)$ と分離すると
  \begin{align}
    x\diff[2]{}{x}S_n^\alpha + (\alpha + 1 - x)\diff{}{x}S_n^\alpha + nS_n^\alpha = 0
  \end{align}
  これはソニンの多項式となるので解はラゲールの陪関数を用いて $S_n^\alpha = L_{n + \alpha}^\alpha$ と書ける。
  よって固有関数は次のように書ける。
  \begin{align}
    \psi(r, \theta, \phi) & = R_l(\rho)\Theta_{lm}(\theta)\Phi_m(\phi)                                                                                                           \\
    \Phi_m(\phi)          & = \frac{1}{\sqrt{2\pi}}e^{im\phi}                                                                                                                    \\
    \Theta_{lm}(\theta)   & = (-1)^{\frac{m + |m|}{2}}\sqrt{\ab(l + \frac{1}{2})\frac{(l - |m|)!}{(l + |m|)!}}P_l^{|m|}(\cos\theta)                                              \\
    R_{nl}(\rho)          & = \rho^{l}e^{-\rho^2/2}L_{n + \alpha}^\alpha(\rho^2)                                                    & \ab(\rho = \sqrt{\dfrac{m\omega}{\hbar}}r)
  \end{align}
  固有エネルギーについては次のようになる。
  \begin{align}
    E & = \frac{\lambda}{2}\hbar\omega = \ab(2n + l + \frac{3}{2})\hbar\omega
  \end{align}
\end{proof}

\subsection{2 準位系, 3 準位系}

\begin{problem}
エネルギー準位が $0$ と $\varepsilon$ からなり、それぞれ $m, n$ 重に縮重する互いに独立な $N$ 個の系が温度 $T$ の熱平衡状態にあるとする。このときの分配関数、エネルギーの期待値、比熱を求めよ。$a = n/m, \beta = \dfrac{1}{k_BT}$ とおく。
\end{problem}
\begin{proof}
  \begin{align}
    Z_N(\beta)        & = \ab(m + ne^{-\beta \varepsilon})^N = m^N\ab(1 + ae^{-\beta \varepsilon})^N                                                                                                                         \\
    E(\beta)          & = -\diffp{}{\beta}\ln Z_N(\beta) = N\frac{a\varepsilon e^{-\beta \varepsilon}}{1 + ae^{-\beta \varepsilon}}                                                                                          \\
    C(T)              & = \diff{E}{T} = -\frac{1}{k_BT^2}\diff{E}{\beta}                                                                                                                                                     \\
                      & = -Nk_B\beta^2\frac{-a\varepsilon^2 e^{-\beta\varepsilon}(1 + ae^{-\beta \varepsilon}) + a\varepsilon e^{-\beta\varepsilon}\cdot a\varepsilon e^{-\beta\varepsilon}}{(1 + ae^{-\beta\varepsilon})^2} \\
                      & = Nk_B\beta^2\frac{a\varepsilon^2 e^{-\beta\varepsilon}}{(1 + ae^{-\beta\varepsilon})^2}                                                                                                             \\
    \frac{C(T)}{Nk_B} & = \ab(\frac{\varepsilon}{k_BT})^2\frac{ae^{-\varepsilon/k_BT}}{(1 + ae^{-\varepsilon/k_BT})^2}                                                                                                       \\
                      & = \begin{dcases}
                            \ab(\frac{\varepsilon}{k_BT})^2 ae^{-\varepsilon/k_BT}          & (a \ll 1) \\
                            \ab(\frac{\varepsilon}{k_BT})^2\frac{1}{ae^{-\varepsilon/k_BT}} & (a \gg 1)
                          \end{dcases}
  \end{align}
\end{proof}

\begin{problem}
エネルギー準位が $0, \varepsilon, b\varepsilon$ からなる独立な $N$ 個の系が温度 $T$ の熱平衡状態にあるとする。このとき分配関数、エネルギーの期待値、比熱を求めよ。
\end{problem}
\begin{proof}
  \begin{align}
    Z_N(\beta)        & = \ab(1 + e^{-\beta\varepsilon} + e^{-\beta b\varepsilon})^N                                                                                                                                                                                                                                      \\
    E(\beta)          & = -\diffp{}{\beta}\ln Z_N(\beta)                                                                                                                                                                                                                                                                  \\
                      & = N\frac{\varepsilon e^{-\beta\varepsilon} + b\varepsilon e^{-\beta b\varepsilon}}{1 + e^{-\beta\varepsilon} + e^{-\beta b\varepsilon}}                                                                                                                                                           \\
    C(T)              & = \diff{E}{T} = -\frac{1}{k_BT^2}\diff{E}{\beta}                                                                                                                                                                                                                                                  \\
                      & = Nk_B\beta^2\frac{(\varepsilon^2 e^{-\beta\varepsilon} + b^2\varepsilon^2 e^{-\beta b\varepsilon})(1 + e^{-\beta\varepsilon} + e^{-\beta b\varepsilon}) - (\varepsilon e^{-\beta\varepsilon} + b\varepsilon e^{-\beta b\varepsilon})^2}{(1 + e^{-\beta\varepsilon} + e^{-\beta b\varepsilon})^2} \\
                      & = Nk_B(\beta\varepsilon)^2\frac{e^{-\beta\varepsilon} + (b - 1)^2e^{-\beta(1 + b)\varepsilon} + b^2e^{-\beta b\varepsilon}}{(1 + e^{-\beta\varepsilon} + e^{-\beta b\varepsilon})^2}                                                                                                              \\
    \frac{C(T)}{Nk_B} & = \ab(\frac{\varepsilon}{k_BT})^2\frac{e^{-\varepsilon/k_BT} + (b - 1)^2e^{-(1 + b)\varepsilon/k_BT} + b^2e^{-b\varepsilon/k_BT}}{(1 + e^{-\varepsilon/k_BT} + e^{-b\varepsilon/k_BT})^2}
  \end{align}

\end{proof}


\section{理想量子気体とグランドカノニカル分布}
\setcounter{subsection}{5}
\subsection{状態を占める粒子数の揺らぎ}
\begin{problem}
Fermi 粒子系、Bose 粒子系における粒子数の揺らぎを調べよ。
\end{problem}
\begin{proof}
  グランドカノニカル分布の分配関数 $\Xi$ が与えられたときに粒子数の揺らぎは次のように書ける。
  \begin{align}
    N                         & = \frac{1}{\beta}\ab(\diffp{\ln\Xi}{\mu})_{T,V}                              \\
    \ab(\diffp{N}{\mu})_{T,V} & = \beta\ab(\langle N^2\rangle - N^2) = \beta\langle(\hat{\Delta N})^2\rangle
  \end{align}
  これより Fermi 粒子系の粒子数の揺らぎは次のように書ける。
  \begin{align}
    \Xi                              & = \prod_{j=1}^{\infty}(1 + e^{-\beta(\varepsilon_j - \mu)})                                        \\
    N                                & = \sum_{j=1}^{\infty}\frac{1}{e^{\beta(\varepsilon_j - \mu)} + 1}                                  \\
    \langle(\hat{\Delta N})^2\rangle & = \sum_{j=1}^{\infty}\frac{e^{\beta(\varepsilon_j - \mu)}}{(e^{\beta(\varepsilon_j - \mu)} + 1)^2}
  \end{align}
  同様に Bose 粒子系の粒子数の揺らぎは次のようになる。
  \begin{align}
    \Xi                              & = \prod_{j=1}^{\infty}\frac{1}{1 - e^{-\beta(\varepsilon_j - \mu)}}                                \\
    N                                & = \sum_{j=1}^{\infty}\frac{1}{e^{\beta(\varepsilon_j - \mu)} - 1}                                  \\
    \langle(\hat{\Delta N})^2\rangle & = \sum_{j=1}^{\infty}\frac{e^{\beta(\varepsilon_j - \mu)}}{(e^{\beta(\varepsilon_j - \mu)} - 1)^2}
  \end{align}
  \begin{align}
    \frac{\sqrt{\langle(\hat{\Delta N})^2\rangle}}{N} & \to 1
  \end{align}

\end{proof}

\section{理想ボーズ気体、ボーズ凝縮}
\subsection{格子比熱 (Debye 模型)}
縦波と 2 つの独立な横波のモードが可能であり、それらの分散関係は $\omega = v_l|\kk|$, $\omega = v_t|\kk|$ と表される。
\begin{problem}
固体の体積を $V$、全原子数を $N$($\gg 1$) として、振動数が $\omega$ と $\omega + \dl{\omega}$ の間にある状態の数 $D(ω)dω$ を求めよ。固体を各辺の長さが $L$ ($L^3 = V$) の立方体と考え、周期境界条件をとってよい。
\end{problem}
\begin{proof}
  周期境界条件と媒体が奇妙な振動をしない条件として次のように書ける。分散関係$\omega = v_l|\kk|$
  \begin{align}
    \kk                   & = \frac{\pi}{L}(n_x, n_y, n_z) \qquad \ab(0 \leq n_i \leq \sqrt[3]{N}) \\
    \frac{\omega L}{v\pi} & = \sqrt{n_x^2 + n_y^2 + n_z^2}
  \end{align}
  $v$ に対して $\dfrac{\omega L}{v\pi}$ を半径とする第一象限の表面積と近似できる。
  \begin{align}
    D(\omega)            & = \frac{1}{8}\frac{4\pi}{3}\ab(\frac{\omega L}{v_l\pi})^3 + \frac{2}{8}\frac{4\pi}{3}\ab(\frac{\omega L}{v_t\pi})^3 \\
                         & = \frac{\omega^3 L^3}{6\pi^2}\ab(\frac{1}{v_l^3} + \frac{2}{v_t^3})                                                 \\
                         & = \frac{\omega^3 L^3}{6\pi^2}\frac{v_t^3 + 2v_l^3}{v_l^3v_t^3}                                                      \\
    D(\omega)\dl{\omega} & = \frac{\omega^2 L^3}{2\pi^2}\frac{v_t^3 + 2v_l^3}{v_l^3v_t^3}\dl{\omega}                                           \\
  \end{align}
\end{proof}

\begin{problem}
$D(\omega)$ を Debye 振動数 $\omega_D$ を用いて表せ。
\end{problem}
\begin{proof}
  \begin{align}
    \int_0^{\omega_D}D(\omega)\dl{\omega} & = \int_0^{\omega_D}\frac{\omega^2 L^3}{2\pi^2}\frac{v_t^3 + 2v_l^3}{v_l^3v_t^3}\dl{\omega} = 3N \\
    \omega_D                              & = \frac{v_lv_t}{L}\ab(\frac{18N\pi^2}{v_t^3 + 2v_l^3})^{1/3}
  \end{align}
  これより $\omega_D$ を用いて $D(\omega)$ は次のように求まる。
  \begin{align}
    D(\omega) & = \begin{dcases}
                    \frac{9N\omega^2}{\omega_D^3} & (\omega < \omega_D) \\
                    0                             & (\omega > \omega_D)
                  \end{dcases}
  \end{align}
\end{proof}

\begin{problem}
この模型における固体の定積比熱 $C$ を求め、高温、低温での振る舞いを調べよ。また、Einstein 模型 ($3N$ 個の独立な調和振動子が、いずれも等しい振動数 $\omega$ を持つ) と比較せよ。
\end{problem}
\begin{proof}
  $\omega$ に対する調和振動子における比熱 $c(\omega)$ を用いて比熱 $C$ は次のように求まる。
  \begin{align}
    C & = \int_0^\infty D(\omega)c(\omega)\dl{\omega}                                                                                                                                                            \\
      & = \int_0^{\omega_D}\frac{9N\omega^2}{\omega_D^3}k_B\ab(\frac{\beta\hbar\omega e^{\beta\hbar\omega/2}}{e^{\beta\hbar\omega} - 1})^2\dl{\omega}                                                            \\
      & = 9Nk_Bb^2\int_0^1\frac{x^4e^{bx}}{(e^{bx} - 1)^2}\dl{x}                                                                                      & \ab(x = \frac{\omega}{\omega_D}, b = \beta\hbar\omega_D) \\
      & = -9Nk_Bb^2\diff{}{b}\int_0^1\frac{x^3}{e^{bx} - 1}\dl{x}
  \end{align}
  高温極限 ($b \ll 1$) のとき Bernoulli 数 $B_n$ の定義を用いて次のように計算できる。
  \begin{align}
    \int_0^1\frac{x^3}{e^{bx} - 1}\dl{x} & = \int_0^1\sum_{n=0}^{\infty}\frac{B_n b^{n-1}}{n!}x^{n+2}\dl{x}                                 \\
                                         & = \sum_{n=0}^{\infty}\frac{B_n}{(n + 3)n!}b^{n-1}                                                \\
                                         & = \frac{1}{3b} - \frac{1}{8} + \frac{1}{60}b - \frac{1}{5040}b^3 + \frac{1}{272160}b^5 - \cdots.
  \end{align}
  低温極限 ($b \gg 1$) のとき分母を展開することで次のように計算できる。
  \begin{align}
    \int_0^1\frac{x^3}{e^{bx} - 1}\dl{x} & = \int_0^1x^3\sum_{n=1}^{\infty}e^{-nbx}\dl{x} = \sum_{n=1}^{\infty}\int_0^1x^3e^{-nbx}\dl{x} \\
                                         & = \sum_{n=1}^{\infty}\frac{1}{(nb)^4}\int_0^{nb}t^3e^{-t}\dl{t} \qquad (t = nbx)              \\
                                         & \approx \frac{1}{b^4}\zeta(4)\Gamma(4)                                                        \\
                                         & = \frac{1}{b^4}\frac{\pi^4}{15}
  \end{align}
  よって比熱は高温、低温について次のような値となる。
  \begin{align}
    C & = 9Nk_Bb^2\int_0^1\frac{x^4e^{bx}}{(e^{bx} - 1)^2}\dl{x} \\
      & \approx \begin{dcases}
                  3Nk_B                    & (b \ll 1) \\
                  3Nk_B\frac{4\pi^4}{5b^3} & (b \gg 1)
                \end{dcases}
  \end{align}
  Einstein 模型における比熱は次のようになる。
  \begin{align}
    C & = 3Nk_B\ab(\frac{\beta\hbar\omega}{2\sinh\frac{1}{2}\beta\hbar\omega})^2 \\
      & \approx 3Nk_Bb^2e^{-b}
  \end{align}
  よって温度 $T$ に対する依存性が異なることが分かる。
\end{proof}

\begin{problem}
一般化して、$d$ 次元における格子比熱の低温 ($T \ll \omega_D$) での温度依存性を調べよ。
\end{problem}
\begin{proof}
  一般の $d$ 次元において状態密度 $D(\omega)$ と比熱 $C$ は次のようになる。
  \begin{align}
    D(\omega) & = \begin{dcases}
                    3N\frac{d\omega^{d-1}}{\omega_D^d} & (\omega < \omega_D) \\
                    0                                  & (\omega > \omega_D)
                  \end{dcases} \\
    C         & = \int_0^\infty D(\omega)c(\omega)\dl{\omega}                         \\
              & = 3dNk_Bb^2\int_0^1\frac{x^{d+1}e^{bx}}{(e^{bx} - 1)^2}\dl{x}         \\
  \end{align}
  低温極限 ($b \gg 1$) において
  \begin{align}
    \int_0^1\frac{x^{d}}{e^{bx} - 1}\dl{x} & = \int_0^1x^d\sum_{n=1}^{\infty}e^{-nbx}\dl{x} = \sum_{n=1}^{\infty}\int_0^1x^de^{-nbx}\dl{x} \\
                                           & = \sum_{n=1}^{\infty}\frac{1}{(nb)^{d+1}}\int_0^{nb}t^de^{-t}\dl{t} \qquad (t = nbx)          \\
                                           & \approx \frac{1}{b^{d+1}}\zeta(d+1)\Gamma(d+1)
  \end{align}
  これより比熱は次のようになる。
  \begin{align}
    C & \approx 3Nk_B\frac{1}{b^d}d(d+1)\zeta(d+1)\Gamma(d+1)
  \end{align}
  よって比熱は温度 $T$ に対して $T^d$ に比例する。
\end{proof}

\setcounter{subsection}{6}
\subsection{3 次元調和トラップ中での Bose-Einstein 凝縮での比熱の変化}
\begin{problem}
3 次元等方調和ポテンシャル $V(r) = (m\omega^2/2)|r|^2$ 中での質量 $m$ の単原子分子の気体の Bose-Einstein 凝縮 (BEC) について、凝縮温度 $T_c$ のすぐ上およびすぐ下での比熱を計算し、比較せよ。原始相互作用が無視できるとして原子の質量 $1.44 \times 10^{-25}$ kg, 調和振動子閉じ込めの角周波数 $\omega =2π \times 60$ Hz, 原子の個数 $2.0 \times 10^3$ 個のとき、転移温度を求めよ。
\end{problem}
\begin{proof}

\end{proof}

\section{理想フェルミ気体、低温展開}
\subsection{Pauli 常磁性}
\begin{problem}
磁場による効果を無視した自由電子におけるスピン磁化率 $\chi_s$ を求める。ただし磁場 $H$ における 1 電子のエネルギーは次のようになる。
\begin{align}
  \varepsilon = \frac{p^2}{2m} \pm \mu_BH
\end{align}
このとき磁場下での磁化 $M$ と絶対零度のスピン磁化率を求めよ。
\end{problem}
\begin{proof}
  \begin{align}
    M    & = \mu_B(N_+ - N_-) \approx \mu_B(2\mu_B\mu_0HD(\varepsilon_F)) \\
    \chi & := \lim_{H\to 0}\frac{M}{H} = 2\mu_B^2\mu_0D(\varepsilon_F)
  \end{align}
\end{proof}

\begin{problem}
$\chi_s$ について、低温での、$T$ について最低次の補正を求めよ。
\end{problem}
\begin{proof}

\end{proof}

\begin{problem}
Fermi 縮退温度 $T_F$ に比べ充分高温での表式を求めよ。低温の場合と比べてどうなるか?
\end{problem}
\begin{proof}

\end{proof}

\subsection{ブロッホの定理}

\subsection{1 次元周期的井戸型ポテンシャル}

\subsection{グラフェンにおける分散関係}

\end{document}