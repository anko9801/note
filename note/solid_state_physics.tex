\RequirePackage{plautopatch}
\documentclass[uplatex,dvipdfmx,a4paper,11pt]{jlreq}

% \usepackage{indentfirst} % 最初の段落にインデント
% \usepackage{wrapfig} % 表や画像の周りに文字を回り込ませる
% \usepackage{comment} % コメント環境
% \usepackage{docmute} % ファイル分割
\usepackage{listings} % ソースコードの挿入
\lstset{
  language=C++,
  breaklines=true,
  keywordstyle = {\color[rgb]{0,0,1}},
  stringstyle = {\color[rgb]{1,0,0}},
  commentstyle = { \color[rgb]{0,1,0}},
  numbers=left,
  frame=lines
}
\usepackage{bxpapersize} % A4判サイズを指定する
\usepackage[utf8]{inputenc}
\usepackage{fontenc} % フォントエンコーディング指定
\usepackage{lmodern} % Latin Modern フォント
\usepackage{otf}
\usepackage{amsmath}
\usepackage{amssymb}
\usepackage{amsthm}
\usepackage{ascmac}
% \usepackage[hyphens]{url}
\usepackage{mhchem}
\usepackage{siunitx}
\usepackage{physics2}
\usephysicsmodule{ab, ab.braket, doubleprod, diagmat, xmat}
\usepackage[DIF = {
      op-symbol = d,
      op-order-nudge = 1 mu,
      outer-Ldelim = \left . ,
      outer-Rdelim = \right |,
      sub-nudge = 0 mu
    }]{diffcoeff}
% \usepackage{braket}
\usepackage{verbatimbox}
\usepackage{bm} % 太字斜体
\usepackage{url}
% \usepackage[dvipdfmx,hiresbb,final]{graphicx}
\usepackage{hyperref} % リンク埋め込み
\usepackage{pxjahyper}
\usepackage{tikz} % グラフや図形を描く
\usetikzlibrary{cd}
% \usetikzlibrary{cd, intersections, calc, arrows, positioning, arrows.meta, automata}
\usepackage{tikz-feynhand}
\usepackage{listings}
\usepackage{color}
\usepackage{mathtools}
\usepackage{xspace}
\usepackage{xy}
\usepackage{xypic}

\makeatletter
%
\DeclareMathOperator{\lcm}{lcm}
\DeclareMathOperator{\Kernel}{Ker}
\DeclareMathOperator{\Image}{Im}
\DeclareMathOperator{\ch}{ch}
\DeclareMathOperator{\Aut}{Aut}
\DeclareMathOperator{\Log}{Log}
\DeclareMathOperator{\Arg}{Arg}
\DeclareMathOperator{\sgn}{sgn}
\DeclareMathOperator{\Res}{Res}
%
\newcommand{\CC}{\mathbb{C}}
\newcommand{\RR}{\mathbb{R}}
\newcommand{\QQ}{\mathbb{Q}}
\newcommand{\ZZ}{\mathbb{Z}}
\newcommand{\NN}{\mathbb{N}}
\newcommand{\FF}{\mathbb{F}}
\newcommand{\PP}{\mathbb{P}}
\newcommand{\GG}{\mathbb{G}}
\newcommand{\TT}{\mathbb{T}}
\newcommand{\EE}{\bm{E}}
\newcommand{\BB}{\bm{B}}
\renewcommand{\AA}{\bm{A}}
\newcommand{\rr}{\bm{r}}
\newcommand{\kk}{\bm{k}}
\newcommand{\pp}{\bm{p}}
\newcommand{\calB}{\mathcal{B}}
\newcommand{\calF}{\mathcal{F}}
\newcommand{\ignore}[1]{}
\newcommand{\floor}[1]{\left\lfloor #1 \right\rfloor}
% \newcommand{\abs}[1]{\left\lvert #1 \right\rvert}
\newcommand{\lt}{<}
\newcommand{\gt}{>}
\newcommand{\id}{\mathrm{id}}
\newcommand{\rot}{\curl}
\renewcommand{\angle}[1]{\left\langle #1 \right\rangle}
\newcommand\mqty[1]{\begin{pmatrix}#1\end{pmatrix}}
\newcommand\vmqty[1]{\begin{vmatrix}#1\end{vmatrix}}
\numberwithin{equation}{section}

\let\oldcite=\cite
\renewcommand\cite[1]{\hyperlink{#1}{\oldcite{#1}}}

\let\oldbibitem=\bibitem
\renewcommand{\bibitem}[2][]{\label{#2}\oldbibitem[#1]{#2}}

% theorem環境の設定
% - 冒頭に改行
% - 末尾にdiamond (amsthm)
\theoremstyle{definition}
\newcommand*{\newscreentheoremx}[2]{
  \newenvironment{#1}[1][]{
    \begin{screen}
    \begin{#2}[##1]
    \leavevmode
    \newline
  }{
    \end{#2}
    \end{screen}
  }
}
\newcommand*{\newqedtheoremx}[2]{
  \newenvironment{#1}[1][]{
    \begin{#2}[##1]
    \leavevmode
    \newline
    \renewcommand{\qedsymbol}{\(\diamond\)}
    \pushQED{\qed}
  }{
    \qedhere
    \popQED
    \end{#2}
  }
}
\newtheorem{theorem*}{定理}[section]

\newqedtheoremx{theorem}{theorem*}
\newcommand*\newqedtheorem@unstarred[2]{%
  \newtheorem{#1*}[theorem*]{#2}
  \newqedtheoremx{#1}{#1*}
}
\newcommand*\newqedtheorem@starred[2]{%
  \newtheorem*{#1*}{#2}
  \newqedtheoremx{#1}{#1*}
}
\newcommand*{\newqedtheorem}{\@ifstar{\newqedtheorem@starred}{\newqedtheorem@unstarred}}

\newtheorem{sctheorem*}{定理}[section]
\newscreentheoremx{sctheorem}{sctheorem*}
\newcommand*\newscreentheorem@unstarred[2]{%
  \newtheorem{#1*}[theorem*]{#2}
  \newscreentheoremx{#1}{#1*}
}
\newcommand*\newscreentheorem@starred[2]{%
  \newtheorem*{#1*}{#2}
  \newscreentheoremx{#1}{#1*}
}
\newcommand*{\newscreentheorem}{\@ifstar{\newscreentheorem@starred}{\newscreentheorem@unstarred}}

%\newtheorem*{definition}{定義}
%\newtheorem{theorem}{定理}
%\newtheorem{proposition}[theorem]{命題}
%\newtheorem{lemma}[theorem]{補題}
%\newtheorem{corollary}[theorem]{系}

\newqedtheorem{lemma}{補題}
\newqedtheorem{corollary}{系}
\newqedtheorem{example}{例}
\newqedtheorem{proposition}{命題}
\newqedtheorem{remark}{注意}
\newqedtheorem{thesis}{主張}
\newqedtheorem{notation}{記法}
\newqedtheorem{problem}{問題}
\newqedtheorem{algorithm}{アルゴリズム}

\newscreentheorem*{axiom}{公理}
\newscreentheorem*{definition}{定義}

\renewenvironment{proof}[1][\proofname]{\par
  \normalfont
  \topsep6\p@\@plus6\p@ \trivlist
  \item[\hskip\labelsep{\bfseries #1}\@addpunct{\bfseries}]\ignorespaces\quad\par
}{%
  \qed\endtrivlist\@endpefalse
}
\renewcommand\proofname{証明}

\makeatother

\newcommand{\R}{\bm{R}}
\newcommand{\E}{\mathcal{E}}
\renewcommand{\aa}{\bm{a}}
\newcommand{\bb}{\bm{b}}
\newcommand{\KK}{\bm{K}}
\title{固体物理学}
\author{anko9801}
\begin{document}
\maketitle
\tableofcontents
\clearpage

\section{格子}

\subsection{格子空間と逆格子空間}
結晶中での原子・分子は周期的に配列された格子 (lattice) という構造を持っている。
\begin{definition}[格子]
  ユークリッド空間 $\RR^n$ において基底となる $\{\aa_i\}_{1\leq i\leq n}$ を選び、$\R_n$ をその整数倍の線形結合で表現できるとする。
  \begin{align}
    \R_n & = \sum_{i = 1}^n n_i\aa_i \qquad (n_i\in\ZZ)
  \end{align}
  このとき $\aa_i$ を基本並進ベクトル (primitive translation vector) と呼び、$\R_n$ を格子点 (lattice point) または格子ベクトル、$\R_n$ の集合全体を格子 (lattice) と呼ぶ。
\end{definition}
\begin{theorem}
  同じ格子でも基本並進ベクトルの取り方は無数にある。\\
\end{theorem}

3 次元格子の格子点を原子・分子と対応付けたものが結晶構造となる。
そして格子を細かく区切っていくとある形の繰り返しの構造となっていて、その内面積最小となるものを基本単位胞 (primitive unit cell) と呼ぶ。
基本単位胞の形状も無数にあり、点のない場所に枠を作ってもよい。
格子点同士を結ぶ線分の垂直二等分面で囲まれた領域を基本単位胞と選んだものをウィグナーザイツ胞 (Wigner-Seitz cell) と呼ぶ。 \\

1850 年に Bravie は 3 次元空間における格子は 14 種類に限られることを示した。
その格子群は Bravie 格子と呼ばれる。
対称性によって格子を分類する。
\begin{itemize}
  \item 並進対称性: すべての格子がもつ
  \item 回転対称性: $C_1$ (恒等), $C_2$ (180\textdegree), $C_3$ (120\textdegree), $C_4$ (90\textdegree), $C_6$ (60\textdegree) のみ
  \item 鏡映 $m$: $(x, y, z)\mapsto(-x, y, z)$
  \item 反転 $\overline{1}$: $(x, y, z)\mapsto(-x, -y, -z)$
  \item 回反 $\overline{4}$: 90\textdegree 回転し上下反転して一致する。
\end{itemize}
\begin{table}[hbtp]
  \centering
  \begin{tabular}{|c|c|c|c|c|}
    \hline
    結晶  & 格子条件                                  & 単純   & 面心     & 対称性      \\
    \hline \hline
    正方晶 & $a = b, \theta = 90$\textdegree       & 正方格子 &        & $C_4, m$ \\
    直方晶 & $a \neq b, \theta = 90$\textdegree    & 直方格子 & 面心直方格子 & $C_2, m$ \\
    斜方晶 & $a \neq b, \theta \neq 90$\textdegree & 斜方格子 &        & $C_2$    \\
    六方晶 & $\theta = 60$\textdegree              & 六方格子 &        & $C_6$    \\
    \hline
  \end{tabular}
  \caption{2 次元 Bravie 格子}
  \label{table:2D Bravie}
\end{table}
\begin{table}[hbtp]
  \centering
  \begin{tabular}{|c|c|c|c|c|c|}
    \hline
    結晶系 & 格子条件                                                                             & 単純     & 体心     & 面心     & 底心     \\
    \hline \hline
    立方晶 & $a = b = c$, $\alpha = \beta = \gamma = 90$\textdegree                           & 単純立方格子 & 体心立方格子 & 面心立方格子 &        \\
    正方晶 & $a = b \neq c$, $\alpha = \beta = \gamma = 90$\textdegree                        & 単純正方格子 & 体心正方格子 &        &        \\
    直方晶 & $a \neq b \neq c$, $\alpha = \beta = \gamma = 90$\textdegree                     & 単純直方格子 & 体心直方格子 & 面心直方格子 & 底心直方格子 \\
    単斜晶 & $a \neq b \neq c$, $\alpha = \gamma = 90$\textdegree, $\beta \neq 90$\textdegree & 単純単斜格子 &        &        & 底心単斜格子 \\
    三方晶 & $a = b = c$, $\alpha = \beta = \gamma \neq 90$\textdegree                        & 単純三方晶  &        &        &        \\
    六方晶 & $a = b \neq c$, $\alpha = \beta = 90$\textdegree, $\gamma = 120$\textdegree      & 単純六方晶  &        &        &        \\
    三斜晶 & $a \neq b \neq c$, $\alpha \neq \beta \neq \gamma$                               & 単純三斜晶  &        &        &        \\
    \hline
  \end{tabular}
  \caption{3 次元 Bravie 格子}
  \label{table:3D Bravie}
\end{table}

周期条件 $f(\rr + \bm{R}_n) = f(\rr)$ を満たす関数 $f(\rr)$ を Fourier 変換すると次のようになる。
\begin{align}
  f(\rr) & = \sum_{m}A_m\exp(i\bm{G}_m\cdot\rr) \qquad \ab(\exp(i\bm{G}_m\cdot\bm{R}_n) = 1)
\end{align}
これより $\bm{G}_m\cdot\bm{R}_n = 2\pi N$ となるから $\bm{G}_m$ は次のように表現できる。
\begin{align}
  \bm{G}_m & = m_1\bb_1 + m_2\bb_2 + m_3\bb_3 \qquad \ab(\aa_i\cdot\bb_j = 2\pi\delta_{ij})
\end{align}
$\bm{G}_m$ を 3 次元の逆格子ベクトル (reciprocal lattice vector) といい、$\bb_1, \bb_2, \bb_3$ を逆格子の基本ベクトルと呼ぶ。
逆格子ベクトルの集合を逆格子空間と呼ぶ。
Wigner-Seitz 胞の逆格子空間を Brillouin ゾーンという。

\begin{example}[単純立方格子 (simple cubic lattice)]
  ($3C_4, 4C_3, 7m$)
  例えば単純立方格子の逆格子空間は単純立方格子
  \begin{align}
    \begin{alignedat}{3}
      \aa_1 & = a(1, 0, 0) & \bb_1 & = \frac{2\pi}{a}(1, 0, 0) \\
      \aa_2 & = a(0, 1, 0) \quad\implies\quad & \bb_2 & = \frac{2\pi}{a}(0, 1, 0) \\
      \aa_3 & = a(0, 0, 1) & \bb_3 & = \frac{2\pi}{a}(0, 0, 1)
    \end{alignedat}
  \end{align}
\end{example}
\begin{example}[面心立方格子]
  逆格子空間は体心立方格子
  \begin{align}
    \begin{alignedat}{3}
      \aa_1 & = \frac{a}{2}(0, 1, 1) & \bb_1 & = \frac{2\pi}{a}(-1, 1, 1) \\
      \aa_2 & = \frac{a}{2}(1, 0, 1) \quad\implies\quad & \bb_2 & = \frac{2\pi}{a}(1, -1, 1) \\
      \aa_3 & = \frac{a}{2}(1, 1, 0) & \bb_3 & = \frac{2\pi}{a}(1, 1, -1)
    \end{alignedat}
  \end{align}
\end{example}
\begin{example}[体心立方格子]
  逆格子空間は面心立方格子
  \begin{align}
    \begin{alignedat}{3}
      \aa_1 & = \frac{a}{2}(0, 1, 1) & \bb_1 & = \frac{2\pi}{a}(-1, 1, 1) \\
      \aa_2 & = \frac{a}{2}(1, 0, 1) \quad\implies\quad & \bb_2 & = \frac{2\pi}{a}(1, -1, 1) \\
      \aa_3 & = \frac{a}{2}(1, 1, 0) & \bb_3 & = \frac{2\pi}{a}(1, 1, -1)
    \end{alignedat}
  \end{align}
\end{example}

\subsection{回折}
Laue 条件
\begin{align}
  \kk = \kk_0 + \bm{G}_m
\end{align}
$\kk$ と $\kk_0$ のなす角 $2\theta$
\begin{align}
  2|\kk|\sin\theta & = |\bm{G}_m|                                                                    \\
  2d\sin\theta     & = \lambda \qquad \ab(|\kk| = \frac{2\pi}{\lambda}, |\bm{G}_m| = \frac{2\pi}{d})
\end{align}
これを Bragg の条件という。
\begin{align}
  A(\KK) & = \int_a
\end{align}

\begin{definition}
  Miller 指数 (Miller indices)
  $[h\ k\ l]$
  \begin{align}
    \bm{A} & = h\aa_1 + k\aa_2 + l\aa_3
  \end{align}
\end{definition}



\section{固体における結合}
\subsection{共有結合 (covalent bond)}
電子に対して陽子は質量が大きい為、陽子は固定されていると考えて陽子の運動エネルギーを除く。
これを断熱近似 (adiabatic approximation) という。
\begin{align}
  \hat{H} & = -\frac{\hbar^2}{2m_e}\nabla^2 - \frac{e^2}{4\pi\varepsilon_0r_1} - \frac{e^2}{4\pi\varepsilon_0r_2} + \frac{e^2}{4\pi\varepsilon_0R}
\end{align}
LCAO 法 (linear combination of atomic orbitals method)
\begin{align}
  \varphi & = c_1\varphi_1 + c_2\varphi_2 \qquad (H\varphi_i = \E\varphi_i)
\end{align}

\begin{align}
  \int\varphi_i\hat{H}\varphi\dl{\rr} & = \E\int\varphi_i\varphi\dl{\rr} \\
  c_jH_{ij}                           & = \E c_jS_{ij}
\end{align}
ただし $j$ で縮約を取り、次のように定義した。
\begin{align}
  H_{ij} & := \int\varphi_i^*\hat{H}\varphi_j\dl{\rr} & (H_{11} = H_{22}, H_{12} = H_{21})     \\
  S_{ij} & := \int\varphi_i^*\varphi_j\dl{\rr}        & (S_{11} = S_{22} = 1, S_{12} = S_{21})
\end{align}
このとき次が成り立つ。
\begin{align}
  \begin{pmatrix}
    H_{11} - S_{11}\mathcal{E} & H_{12} - S_{12}\mathcal{E} \\
    H_{21} - S_{21}\mathcal{E} & H_{22} - S_{22}\mathcal{E} \\
  \end{pmatrix}
  \begin{pmatrix}
    c_1 \\ c_2
  \end{pmatrix}
  =
  \begin{pmatrix}
    H_{11} - \mathcal{E}  & H_{12} - S\mathcal{E} \\
    H_{12} - S\mathcal{E} & H_{11} - \mathcal{E}  \\
  \end{pmatrix}
  \begin{pmatrix}
    c_1 \\ c_2
  \end{pmatrix}
  = 0
\end{align}
このとき行列式を考えることで次の式が成り立つ。
\begin{align}
  (H_{11} - \mathcal{E})^2 & - (H_{12} - S\mathcal{E})^2 = 0     \\
  H_{11} - \mathcal{E}     & = \pm (H_{12} - S\mathcal{E})       \\
  \mathcal{E}_\pm          & = \frac{H_{11} \pm H_{12}}{1 \pm S}
\end{align}
これを代入すると
\begin{align}
  \frac{1}{1 \pm S}
  \begin{pmatrix}
    \pm H_{11}S \mp H_{12} & H_{12} - SH_{11}       \\
    H_{12} - SH_{11}       & \pm H_{11}S \mp H_{12} \\
  \end{pmatrix}
  \begin{pmatrix}
    c_1 \\ c_2
  \end{pmatrix}
  = 0
\end{align}
より $c_2 = \pm c_1$ となる。
\begin{align}
  \int|\varphi|^2\dl{\rr} & = \int|c_1\varphi_1 \pm c_1\varphi_2|^2\dl{\rr} = |c_1|^2\int(|\varphi_1|^2 \pm \varphi_1^*\varphi_2 \pm\varphi_2^*\varphi_1 + |\varphi_2|^2)\dl{\rr} \\
                          & = |c_1|^2(2 \pm 2S) = 1                                                                                                                               \\
  c_1                     & = \frac{1}{\sqrt{2(1 \pm S)}}
\end{align}
よって波動関数は次のように表示できる。
\begin{align}
  \varphi_\pm & = \frac{1}{\sqrt{2(1 \pm S)}}(\varphi_1 \pm \varphi_2)
\end{align}


\section{格子振動}
\subsection{1 種類の原子からなる 1 次元格子振動}
\begin{align}
  M\diff[2]{u_j}{t} & = -K(u_j - u_{j-1}) + K(u_{j+1} - u_j)
\end{align}
\begin{align}
  u_j & = Ae^{ijka}e^{-i\omega t}
\end{align}
\begin{align}
  -M\omega^2 & = -K(2 - e^{-ika} - e^{ika}) = -2K(1 - \cos ka) = -4K\sin^2\frac{ka}{2} \\
  \omega     & = 2\sqrt{\frac{K}{M}}\ab|\sin\frac{ka}{2}|
\end{align}
$\omega$ と $k$ の関係は分散関係 (dispersion relation) と呼ぶ。
このとき $\omega$ は周期 $\frac{2\pi}{a}$ で振動する。
\begin{align}
  k & = 0 \qquad \omega = 0                               \\
  k & = \frac{\pi}{a} \qquad \omega = 2\sqrt{\frac{K}{M}} \\
  k & = \frac{2\pi}{a} \qquad \omega = 0
\end{align}


\subsection{2 種類の原子からなる 1 次元格子振動}
結晶には 2 種類以上の原子からなるものも多い。
\ce{GaAs} など
\begin{align}
  M_A\diff[2]{u_j^A}{t} & = c(u_j^B - u_j^A) - c(u_j^A - u_{j-1}^B) \\
  M_B\diff[2]{u_j^B}{t} & = c(u_{j+1}^A - u_j^B) - c(u_j^B - u_j^A)
\end{align}
\begin{align}
  u_j^A & = Ae^{ik(n-1)a - i\omega t}           \\
  u_j^B & = Be^{ik(n-\frac{1}{2})a - i\omega t}
\end{align}
\begin{align}
  -M_A\omega^2A & = -c(2A - B - Be^{-ika}) \\
  -M_B\omega^2B & = -c(2B - A - Ae^{ika})
\end{align}
\begin{align}
  \begin{pmatrix}
    M_A\omega^2 - 2c & c(1 + e^{-ika})  \\
    c(1 + e^{ika})   & M_B\omega^2 - 2c
  \end{pmatrix}
  \begin{pmatrix}
    A \\ B
  \end{pmatrix}
  = 0
\end{align}
行列式が $0$ であることから
\begin{align}
  \omega^2 & = c\ab(\ab(\frac{1}{M_A} + \frac{1}{M_B})\pm\sqrt{\ab(\frac{1}{M_A} + \frac{1}{M_B})^2 - 4\frac{\sin^2\frac{ka}{2}}{M_AM_B}})
\end{align}
$k \approx 0$ において
\begin{align}
  \omega^2 & = c\ab(\ab(\frac{1}{M_A} + \frac{1}{M_B})\pm\sqrt{\ab(\frac{1}{M_A} + \frac{1}{M_B})^2 - 4\frac{\sin^2\frac{ka}{2}}{M_AM_B}}) \\
           & \approx c\ab(\ab(\frac{1}{M_A} + \frac{1}{M_B})\pm\ab(\ab(\frac{1}{M_A} + \frac{1}{M_B}) - \frac{(ka)^2}{2(M_A + M_B)}))      \\
           & \approx 2c\ab(\frac{1}{M_A} + \frac{1}{M_B}), \quad \frac{c(ka)^2}{2(M_A + M_B)}                                              \\
  \omega   & \approx \sqrt{2c\ab(\frac{1}{M_A} + \frac{1}{M_B})}, \quad a\sqrt{\frac{c}{2(M_A + M_B)}}k
\end{align}
これより $k \approx 0$ において線形的な音響モードと定数的な光学モードに分けられる。
音響モードは同じ向き
光学モードは逆向き
音速が速いと硬い

1次元の格子振動を量子化する。
より一般の 3 次元の格子振動はモードの種類 ($s = 1, 2, 3$) による。

\subsection{フォノン}
\begin{align}
  u_j                            & = \frac{1}{\sqrt{N}}\sum_{k}u_ke^{ijka} \\
  \sum_{j = 1}^N e^{ij(k + k')a} & = N\delta_{k,-k'}
\end{align}
\begin{align}
  H & = \sum_{j = 1}^{N}\ab(\frac{1}{2}M\ab(\diffp{u_j}{t})^2 + \frac{1}{2}K(u_{j+1} - u_j)^2)                                                                                    \\
    & = \sum_{j = 1}^{N}\ab(\frac{1}{2}M\ab(\frac{1}{\sqrt{N}}\sum_{k}\diffp{u_k}{t}e^{ijka})^2 + \frac{1}{2}K\ab(\frac{1}{\sqrt{N}}\sum_{k}u_ke^{ijka}(e^{ika} - 1))^2)          \\
    & = \sum_{j = 1}^{N}\ab(\frac{1}{2N}M\sum_{k,k'}\diffp{u_k}{t}\diffp{u_{k'}}{t}e^{ij(k + k')a} + \frac{1}{2N}K\sum_{k,k'}u_ku_{k'}(e^{ika} - 1)(e^{ik'a} - 1)e^{ij(k + k')a}) \\
    & = \frac{1}{2}M\sum_{k}\diffp{u_k}{t}\diffp{u_{-k}}{t} + \frac{1}{2N}K\sum_{k}u_ku_{-k}(e^{ika} - 1)(e^{-ika} - 1)                                                           \\
    & = \sum_{k}\ab(\frac{1}{2}M\diffp{u_k}{t}\diffp{u_{-k}}{t} + 2K\sin^2\frac{ka}{2}u_ku_{-k})                                                                                  \\
    & = \sum_{k}\ab(\frac{p_{k}p_{-k}}{2M} + \frac{1}{2}M\omega^2u_ku_{-k})
\end{align}
\begin{align}
  \hat{a}_k         & = \sqrt{\frac{M\omega}{2\hbar}}u_k + \frac{i}{\sqrt{2M\hbar\omega}}p_{-k}   \\
  \hat{a}_k^\dagger & = \sqrt{\frac{M\omega}{2\hbar}}u_{-k} - \frac{i}{\sqrt{2M\hbar\omega}}p_{k}
\end{align}
\begin{align}
  \hat{H} = \sum_{k}\ab(\hat{a}_k^\dagger\hat{a}_k + \frac{1}{2})\hbar\omega
\end{align}
\begin{align}
  [a_k, a_{k'}^\dagger] = \delta_{kk'}
\end{align}
波数 $\kk$ とモードの種類 $s$ のフォノンのエネルギーは $E_{\kk,s}$
\begin{align}
  E_{\kk,s} = \ab(n_{\kk,s} + \frac{1}{2})\hbar\omega_{\kk,s}
\end{align}

\section{固体の熱的性質}
\subsection{}

\section{自由電子}
\subsection{Nearly free electron model}
$V(\rr) \ll 1$
\begin{align}
  V(\rr) & = \sum_{\kk}V_{\kk}e^{i\kk\cdot\rr}
\end{align}


\end{document}