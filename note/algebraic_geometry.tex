\RequirePackage{plautopatch}
\documentclass[uplatex,dvipdfmx,a4paper,11pt]{jlreq}
\usepackage{bxpapersize}
\usepackage[utf8]{inputenc}
\usepackage{fontenc}
\usepackage{lmodern}
\usepackage{otf}
\usepackage{amsmath}
\usepackage{amssymb}
\usepackage{amsthm}
\usepackage{ascmac}
% \usepackage[hyphens]{url}
\usepackage{physics2}
\usephysicsmodule{ab, ab.braket, doubleprod, diagmat, xmat}
\usepackage{diffcoeff}
% \usepackage{braket}
\usepackage{verbatimbox}
\usepackage{bm}
\usepackage{url}
% \usepackage[dvipdfmx,hiresbb,final]{graphicx}
\usepackage{hyperref}
\usepackage{pxjahyper}
\usepackage{tikz}\usetikzlibrary{cd}
\usepackage{listings}
\usepackage{color}
\usepackage{mathtools}
\usepackage{xspace}
\usepackage{xy}
\usepackage{xypic}
%
\title{}
\author{anko9801}
\makeatletter
%
\DeclareMathOperator{\lcm}{lcm}
\DeclareMathOperator{\Kernel}{Ker}
\DeclareMathOperator{\Image}{Im}
\DeclareMathOperator{\ch}{ch}
\DeclareMathOperator{\Aut}{Aut}
\DeclareMathOperator{\Log}{Log}
\DeclareMathOperator{\Arg}{Arg}
\DeclareMathOperator{\sgn}{sgn}
%
\newcommand{\CC}{\mathbb{C}}
\newcommand{\RR}{\mathbb{R}}
\newcommand{\QQ}{\mathbb{Q}}
\newcommand{\ZZ}{\mathbb{Z}}
\newcommand{\NN}{\mathbb{N}}
\newcommand{\FF}{\mathbb{F}}
\newcommand{\PP}{\mathbb{P}}
\newcommand{\GG}{\mathbb{G}}
\newcommand{\TT}{\mathbb{T}}
\newcommand{\EE}{\bm{E}}
\newcommand{\BB}{\bm{B}}
\renewcommand{\AA}{\mathbb{A}}
\newcommand{\rr}{\bm{r}}
\newcommand{\kk}{\bm{k}}
\newcommand{\pp}{\mathfrak{p}}
\newcommand{\mm}{\mathfrak{m}}
\newcommand{\calB}{\mathcal{B}}
\newcommand{\calF}{\mathcal{F}}
\newcommand{\ignore}[1]{}
\newcommand{\floor}[1]{\left\lfloor #1 \right\rfloor}
% \newcommand{\abs}[1]{\left\lvert #1 \right\rvert}
\newcommand{\lt}{<}
\newcommand{\gt}{>}
\newcommand{\id}{\mathrm{id}}
\newcommand{\rot}{\curl}
\renewcommand{\angle}[1]{\left\langle #1 \right\rangle}
\newcommand\mqty[1]{\begin{pmatrix}#1\end{pmatrix}}
\newcommand\vmqty[1]{\begin{vmatrix}#1\end{vmatrix}}
\numberwithin{equation}{section}

\let\oldcite=\cite
\renewcommand\cite[1]{\hyperlink{#1}{\oldcite{#1}}}

\let\oldbibitem=\bibitem
\renewcommand{\bibitem}[2][]{\label{#2}\oldbibitem[#1]{#2}}

% theorem環境の設定
% - 冒頭に改行
% - 末尾にdiamond (amsthm)
\theoremstyle{definition}
\newcommand*{\newscreentheoremx}[2]{
  \newenvironment{#1}[1][]{
    \begin{screen}
    \begin{#2}[##1]
      \leavevmode
      \newline
  }{
    \end{#2}
    \end{screen}
  }
}
\newcommand*{\newqedtheoremx}[2]{
  \newenvironment{#1}[1][]{
    \begin{#2}[##1]
      \leavevmode
      \newline
      \renewcommand{\qedsymbol}{\(\diamond\)}
      \pushQED{\qed}
  }{
      \qedhere
      \popQED
    \end{#2}
  }
}
\newtheorem{theorem*}{定理}[section]

\newqedtheoremx{theorem}{theorem*}
\newcommand*\newqedtheorem@unstarred[2]{%
  \newtheorem{#1*}[theorem*]{#2}
  \newqedtheoremx{#1}{#1*}
}
\newcommand*\newqedtheorem@starred[2]{%
  \newtheorem*{#1*}{#2}
  \newqedtheoremx{#1}{#1*}
}
\newcommand*{\newqedtheorem}{\@ifstar{\newqedtheorem@starred}{\newqedtheorem@unstarred}}

\newtheorem{sctheorem*}{定理}[section]
\newscreentheoremx{sctheorem}{sctheorem*}
\newcommand*\newscreentheorem@unstarred[2]{%
  \newtheorem{#1*}[theorem*]{#2}
  \newscreentheoremx{#1}{#1*}
}
\newcommand*\newscreentheorem@starred[2]{%
  \newtheorem*{#1*}{#2}
  \newscreentheoremx{#1}{#1*}
}
\newcommand*{\newscreentheorem}{\@ifstar{\newscreentheorem@starred}{\newscreentheorem@unstarred}}

%\newtheorem*{definition}{定義}
%\newtheorem{theorem}{定理}
%\newtheorem{proposition}[theorem]{命題}
%\newtheorem{lemma}[theorem]{補題}
%\newtheorem{corollary}[theorem]{系}

\newqedtheorem{lemma}{補題}
\newqedtheorem{corollary}{系}
\newqedtheorem{example}{例}
\newqedtheorem{proposition}{命題}
\newqedtheorem{remark}{注意}
\newqedtheorem{thesis}{主張}
\newqedtheorem{notation}{記法}
\newqedtheorem{problem}{問題}
\newqedtheorem{algorithm}{アルゴリズム}

\newscreentheorem*{axiom}{公理}
\newscreentheorem*{definition}{定義}

\renewenvironment{proof}[1][\proofname]{\par
  \normalfont
  \topsep6\p@\@plus6\p@ \trivlist
  \item[\hskip\labelsep{\bfseries #1}\@addpunct{\bfseries}]\ignorespaces\quad\par
}{%
  \qed\endtrivlist\@endpefalse
}
\renewcommand\proofname{証明}

\makeatother

\begin{document}
\maketitle
\tableofcontents
\clearpage

\section{代数多様体}
代数幾何学は代数方程式で定められる図形の幾何学である。
\begin{theorem}[Hilbert の基底定理]
  Noether 環のイデアルは有限生成である。

  特に多項式環 $k[x_1,\ldots,x_n]$ は Noether 環であるからイデアルは $I = (f_1,\ldots,f_n)$ と書ける。
\end{theorem}

\subsection{アフィン空間}
\begin{definition}[アフィン空間]
  代数的閉体 $k$ に対して解の全体 $k^n$ を $n$ 次元アフィン空間 (affine space) と呼び、$\AA_k^n$ または $\AA^n$ と書く。
\end{definition}
代数的閉体 $k$ 上の $n$ 変数多項式環 $k[x_1,\ldots,x_n]$ について連立方程式の解の 1 つはアフィン空間の元 $(a_1,\ldots,a_n)\in k^n$ として書ける。このとき連立方程式の解の全体を \textbf{代数的集合 (algebraic set)} と呼ぶ。多項式 $f_1,\ldots,f_l$ やイデアル $I = (f_1,\ldots,f_l)$ からなる代数的集合をそれぞれ $V(f_1,\ldots,f_l)$, $V(I)$ と書く。
\begin{align}
  V(f_1, \ldots, f_l) & = \lbrace(a_1, \ldots, a_n)\in k^n\mid f_\alpha(a_1,\ldots,a_n) = 0,\ \alpha = 1,\ldots,l\rbrace \\
  V(I)                & = \lbrace(a_1, \ldots, a_n)\in k^n\mid \forall f\in I \ f(a_1,\ldots,a_n) = 0\rbrace
\end{align}

\begin{itembox}[l]{点、直線}
  \begin{equation*}
    \Downarrow
  \end{equation*}
  代数的閉体 $k$ 上の多項式環 $k[x_1,\ldots,x_n]$ の極大イデアルの代数的集合を
  \begin{align}
    \mm    & = (x_1 - a_1,\ldots,x_n - a_n)   \\
    V(\mm) & = \lbrace(a_1,\ldots,a_n)\rbrace
  \end{align}
\end{itembox}
\begin{proposition}
  \begin{enumerate}
    \item $V((f_1,\ldots,f_l)) = V(f_1,\ldots,f_l)$
    \item $V(I)\cup V(J) = V(I\cap J)$
    \item $\bigcup V(I_\lambda) \subsetneq V(\bigcap I_\lambda)$
    \item $\bigcap V(I_\lambda) = V(\sum_{\lambda}I_\lambda)$
    \item $\sqrt{I}\supset\sqrt{J}$ であれば $V(I)\supset V(J)$
    \item $I = (1) \iff V(I) = \emptyset$
    \item $I = (0) \iff V(I) = \AA_k^n$
    \item $I(V(J)) = \sqrt{J}$ Hilbert の零点定理
  \end{enumerate}
\end{proposition}
\begin{proof}
  \begin{enumerate}
    \item イデアル $I = (f_1,\ldots,f_l)$ の任意の元 $f(x_1,\ldots,x_n)\in I$ について
          \begin{align}
            f(x_1,\ldots,x_n) & = \sum_{\alpha=1}^{l}g_\alpha(x_1,\ldots,x_n)f_\alpha(x_1,\ldots,x_n)
          \end{align}
          と書ける。これより
          \begin{alignat}{3}
             & (a_1,\ldots,a_n)\in V(f_1,\ldots,f_l) & f(a_1,\ldots,a_n) = 0        & \implies V(f_1,\ldots,f_l)\subseteq V(I) \\
             & (b_1,\ldots,b_n)\in V(I)              & f_\alpha(b_1,\ldots,b_n) = 0 & \implies V(f_1,\ldots,f_l)\supseteq V(I)
          \end{alignat}
          よって $V(I) = V(f_1,\ldots,f_l)$ となる。
    \item 方程式の包含関係を考えることで
          \begin{align}
            V(I) \subseteq V(I\cap J), V(J) \subseteq V(I\cap J) \implies V(I) \cup V(J) \subseteq V(I\cap J)
          \end{align}
          となる。逆に $(a_1,\ldots,a_n)\in V(I\cap J)$ について $I\cap J\supseteq IJ$ より $f\in I, g\in J$ とすると $fg\in I\cap J$ であるから
          \begin{align}
            f(a_1,\ldots,a_n)g(a_1,\ldots,a_n) = 0 \implies f(a_1,\ldots,a_n) = 0 または g(a_1,\ldots,a_n) = 0
          \end{align}
          より $(a_1,\ldots,a_n)\in V(I)\cup V(J)$ となる。よって $V(I) \cup V(J) = V(I\cap J)$ となる。
    \item 可算無限個の相違なる元 $c_1,\ldots,c_n,\ldots$ を取り出し $k[x]$ のイデアル $I_j = (x - c_j)$ とすると $\bigcup_j V(I_j) = \lbrace c_1, c_2, \ldots \rbrace$ である。一方、可算無限次数の多項式は存在しないので
          \begin{align}
            V(I_{j_1}\cap \cdots\cap I_{j_n}) & = V\ab(\ab(\prod_{i=1}^{n}(x - c_{j_i}))) = \lbrace c_{j_1},\ldots,c_{j_n} \rbrace \\
            V\ab(\bigcap_j I_j)               & = V\ab((0)) = \AA_k^1                                                              \\
            \bigcup_j V(I_j)                  & \subsetneq V\ab(\bigcap_j I_j)
          \end{align}
          となる。
    \item $\mu$ に対して
          \begin{align}
            V(I_\mu) \supseteq V\ab(\sum_\lambda I_\lambda)
          \end{align}
          $(a_1,\ldots,a_n)\in\bigcup_\lambda V(I_\lambda)$ について $I_\lambda = (h_{\lambda 1},\ldots,h_{\lambda m_\lambda})$ とすると
          \begin{align}
             & h_{\lambda i}(a_1,\ldots,a_n) = 0                                    \\
             & \sum_{\lambda, i} f(a_1,\ldots,a_n)h_{\lambda i}(a_1,\ldots,a_n) = 0
          \end{align}
          より
          \begin{align}
            \bigcap_\lambda V(I_\lambda) \subseteq V\ab(\sum_\lambda I_\lambda)
          \end{align}
    \item
    \item ($\implies$) それぞれの解が独立かつアフィン空間を被覆する方程式を作れるから成り立つ。 \\
          ($\impliedby$) この対偶について任意のイデアル $I \neq (1)$ はそれを含む極大イデアル $\mm\supseteq I$ があり $V(\mm) \neq \emptyset$ となるから $V(I) \neq \emptyset$ である。
    \item ($\implies$) $0 = 0$ は任意のアフィン空間の元が成り立つ。 \\
          ($\impliedby$) $V(I) = \AA_k^n$ のとき Hilbert の零点定理より $I = \sqrt{(0)} = (0)$ となる。
  \end{enumerate}
\end{proof}

\begin{proposition}
  1 次元アフィン空間 $\AA^1$ 内の代数的集合は $\AA^1$ 以外有限個の点である。
\end{proposition}
\begin{proof}
  体 $k$ の $1$ 変数多項式環は単項イデアル整域であるから自明でないイデアルは $f(x)$ を用いて $I = (f(x)) \neq (0)$ と表される。これより
  \begin{align}
    V(I) & = \lbrace a\in k\mid f(a) = 0\rbrace
  \end{align}
  となるから有限個の解しかない。
\end{proof}

\begin{proposition}
  実数体 $\RR$ 上の 1 変数多項式環の極大イデアルは
  \begin{align}
     & (x - a), a\in\RR                        \\
     & (x^2 + ax + b), a,b\in\RR, a^2 - 4b < 0
  \end{align}
  の形となる。
\end{proposition}
\begin{proof}
  単項イデアル整域より既約元のイデアルと極大イデアルは同値である。
\end{proof}

\begin{itembox}[l]{交点}
  平面曲線 $C_f: f(x, y) = 0$ と $C_g: g(x, y) = 0$ の交点について
\end{itembox}

\begin{definition}[座標環]
  \begin{align}
    k[V] := k[x_1,x_2,\ldots,x_n]/I(V)
  \end{align}
  変数変換を一般化した

  代数的集合 $V, W$ において写像 $\varphi: V\to W$ を $V$ の点 $P = (a_1,\ldots,a_m)$ に対して射 (morphism) を定義する。
  \begin{alignat}{3}
    \varphi:\  & V \quad & \to\quad     & W                      \\
               & P \quad & \mapsto\quad & (f_1(P),\ldots,f_n(P))
  \end{alignat}
  $k[V] \subseteq \AA_k^m, k[W] \subseteq \AA_k^n$
\end{definition}

\begin{example}
  3 次曲線 $C = V(y^2 - x^3) \subseteq \AA_k^2$ を考える。アフィン直線 $\AA^1$ とアフィン平面 $\AA^2$ の座標環はそれぞれ $k[\AA^1] = k[t]$, $k[\AA^2] = k[x, y]$ と与えられる。
  \begin{align}
    x = t^2, y = t^3
  \end{align}
  \begin{alignat}{3}
    \varphi: \        & \AA^1 \quad      & \to\quad     & C          \\
                      & a     \quad      & \mapsto\quad & (a^2, a^3) \\
    \tilde\varphi: \  & \AA^1 \quad      & \to\quad     & \AA^2      \\
                      & a     \quad      & \mapsto\quad & (a^2, a^3) \\
    \iota: \          & \AA^2 \quad      & \to\quad     & C          \\
                      & (a^2, a^3) \quad & \mapsto\quad & (a^2, a^3)
  \end{alignat}
  \begin{alignat}{3}
    \varphi^\#: \        & k[C] = k[x,y]/(y^2 - x^3) \quad                    & \to\quad     & k[\AA^1] = k[t]           \\
                         & \overline{f(x,y)} = f(x,y)\bmod{(y^2 - x^3)} \quad & \mapsto\quad & f(t^2, t^3)               \\
    \tilde\varphi^\#: \  & k[\AA^2] = k[x,y] \quad                            & \to\quad     & k[\AA^1] = k[t]           \\
                         & f(x,y) \quad                                       & \mapsto\quad & f(t^2, t^3)               \\
    \iota^\#: \          & k[\AA^2] = k[x,y] \quad                            & \to\quad     & k[C] = k[x,y]/(y^2 - x^3) \\
                         & f(x,y) \quad                                       & \mapsto\quad & \overline{f(x,y)}
  \end{alignat}
  \[
    \begin{tikzcd}
      \AA^1 \arrow[r,"\tilde\varphi"]\arrow[dr,"\varphi"] & C \arrow[d,"\iota"] \\
      & \AA^2
    \end{tikzcd}
    \qquad
    \begin{tikzcd}
      k[\AA^1] & k[C] \arrow[l,"\tilde\varphi^\#"] \\
      & k[\AA^2] \arrow[u,"\iota^\#"]\arrow[ul,"\varphi^\#"]
    \end{tikzcd}
  \]
  ここで $\varphi$ は全単射であるが $\varphi^\#$ は $t$ の一次式は得られないことから全射ではない。
\end{example}

\begin{example}
  \begin{align}
    E & = V(y^2 - x^3 + 1) \subseteq \AA^2                    \\
    D & = V((x_2^2 - x_1^3 + 1, x_3 - x_1^2)) \subseteq \AA^3
  \end{align}
  \begin{alignat}{3}
    \psi: \        & E \quad                     & \to\quad     & D           \\
                   & x_1 = x, x_2 = y, x_3 = x^2                              \\
    \tilde\psi: \  & \AA^2 \quad                 & \to\quad     & \AA^3       \\
                   & (a, b) \quad                & \mapsto\quad & (a, b, a^2)
  \end{alignat}
  \begin{alignat}{3}
    \psi^\#: \        & k[D] = k[x_1,x_2,x_3]/I \quad   & \to\quad     & k[E] = k[x,y]/J       \\
                      & \overline{g(x_1,x_2,x_3)} \quad & \mapsto\quad & \overline{g(x,y,x^2)} \\
    \tilde\psi^\#: \  & k[\AA^3] = k[x_1,x_2,x_3] \quad & \to\quad     & k[\AA^2] = k[x,y]     \\
                      & g(x_1,x_2,x_3) \quad            & \mapsto\quad & g(x,y,x^2)
  \end{alignat}
\end{example}

\begin{proposition}
  代数的集合の射 $\varphi: V\to W$ が与えられると、座標環の $k$ 準同型写像 $\varphi^\#: k[W]\to k[V]$ が定まり、かつ点 $(a_1,\ldots,a_m)\in V$ から定まる
\end{proposition}

\begin{itembox}[l]{重複度・局所交点数}
  平面曲線 $C_f: f(x, y) = 0$ と $C_g: g(x, y) = 0$ の交点が重根によって表現されることがある。
  \begin{equation*}
    \Downarrow
  \end{equation*}
  $n$ 変数多項式環 $R = k[x_1,\ldots,x_n]$ の点 $P$ での局所化 $R_P$ について連立方程式のイデアル $I = (f_1,\ldots,f_l)$ で割った環の $k$ に関する次元を局所交点数と呼び、$I_P(C_1,\ldots,C_l)$ と書く。
  \begin{align}
    I_P(C_1,\ldots,C_l) = \dim_k R_P/I
  \end{align}
\end{itembox}
\begin{proof}
  $R = k[x_1,\ldots,x_n]$ の $I = (f_1,\ldots,f_l)$ に関する局所化 $R_P$
  \begin{align}
    f(x) & = a_0\prod_{j=1}^{m}(x - \alpha_j)^{n_j}
  \end{align}
\end{proof}
\begin{example}

\end{example}

\subsection{射影空間}
無限遠点を含む空間を用いて議論したい。
\begin{definition}[射影空間]
  代数的閉体 $k$ 上の $n + 1$ 次元アフィン空間 $k^{n+1}$ から原点を除いたものを $W$ とおく。
  \begin{align}
    W = k^{n+1}/\lbrace(0,\ldots,0)\rbrace
  \end{align}
  $W$ に同値関係 $\sim$ を次のように定義する。
  \begin{align}
    (a_0,\ldots,a_n)\sim(b_0,\ldots,b_n) \iff \exists\alpha\in k^\times (a_0,\ldots,a_n) = (\alpha b_0,\ldots,\alpha b_n)
  \end{align}
  この同値関係 $\sim$ による $W$ の商空間 $W/\sim$ を $\PP_k^n$ と記し、$n$ 次元射影空間 ($n$-dimensional projective space) という。
  $\PP_k^n$ の元は $(a_0, a_1, \ldots, a_n)$ の定める同値類を $(a_0 : a_1 : \ldots : a_n)$ と記し $\PP_k^n$ の点という。
  \begin{align}
    (a_0 : a_1 : \ldots : a_n)\in\PP_k^n
  \end{align}
\end{definition}
斉次式


\end{document}