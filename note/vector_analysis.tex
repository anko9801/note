\RequirePackage{plautopatch}
\documentclass[uplatex,dvipdfmx,a4paper,11pt]{jlreq}
\usepackage{bxpapersize}
\usepackage[utf8]{inputenc}
\usepackage{fontenc}
\usepackage{lmodern}
\usepackage{otf}
\usepackage{amsmath}
\usepackage{amssymb}
\usepackage{amsthm}
\usepackage{ascmac}
% \usepackage[hyphens]{url}
\usepackage{physics}
\usepackage{braket}
\usepackage{verbatimbox}
\usepackage{bm}
\usepackage{url}
% \usepackage[dvipdfmx,hiresbb,final]{graphicx}
\usepackage{hyperref}
\usepackage{pxjahyper}
\usepackage{tikz}\usetikzlibrary{cd}
\usepackage{listings}
\usepackage{color}
\usepackage{mathtools}
\usepackage{xspace}
\usepackage{xy}
\usepackage{xypic}
%
\title{ベクトル解析}
\author{Anko}
\makeatletter
%
\DeclareMathOperator{\lcm}{lcm}
\DeclareMathOperator{\Kernel}{Ker}
\DeclareMathOperator{\Image}{Im}
\DeclareMathOperator{\ch}{ch}
\DeclareMathOperator{\Aut}{Aut}
\DeclareMathOperator{\Log}{Log}
\DeclareMathOperator{\Arg}{Arg}
\DeclareMathOperator{\sgn}{sgn}
%
\newcommand{\CC}{\mathbb{C}}
\newcommand{\RR}{\mathbb{R}}
\newcommand{\QQ}{\mathbb{Q}}
\newcommand{\ZZ}{\mathbb{Z}}
\newcommand{\NN}{\mathbb{N}}
\newcommand{\FF}{\mathbb{F}}
\newcommand{\PP}{\mathbb{P}}
\newcommand{\GG}{\mathbb{G}}
\newcommand{\TT}{\mathbb{T}}
\newcommand{\calB}{\mathcal{B}}
\newcommand{\calF}{\mathcal{F}}
\newcommand{\ignore}[1]{}
\newcommand{\floor}[1]{\left\lfloor #1 \right\rfloor}
% \newcommand{\abs}[1]{\left\lvert #1 \right\rvert}
\newcommand{\lt}{<}
\newcommand{\gt}{>}
\newcommand{\id}{\mathrm{id}}
\newcommand{\rot}{\curl}
\renewcommand{\angle}[1]{\left\langle #1 \right\rangle}
\newcommand{\EE}{\bm{E}}
\newcommand{\BB}{\bm{B}}
\renewcommand{\AA}{\bm{A}}
\newcommand{\rr}{\bm{r}}
\newcommand{\kk}{\bm{k}}
\newcommand{\pp}{\bm{p}}

\let\oldcite=\cite
\renewcommand\cite[1]{\hyperlink{#1}{\oldcite{#1}}}

\let\oldbibitem=\bibitem
\renewcommand{\bibitem}[2][]{\label{#2}\oldbibitem[#1]{#2}}

% theorem環境の設定
% - 冒頭に改行
% - 末尾にdiamond (amsthm)
\theoremstyle{definition}
\newcommand*{\newscreentheoremx}[2]{
  \newenvironment{#1}[1][]{
    \begin{screen}
    \begin{#2}[##1]
      \leavevmode
      \newline
  }{
    \end{#2}
    \end{screen}
  }
}
\newcommand*{\newqedtheoremx}[2]{
  \newenvironment{#1}[1][]{
    \begin{#2}[##1]
      \leavevmode
      \newline
      \renewcommand{\qedsymbol}{\(\diamond\)}
      \pushQED{\qed}
  }{
      \qedhere
      \popQED
    \end{#2}
  }
}
\newtheorem{theorem*}{定理}

\newqedtheoremx{theorem}{theorem*}
\newcommand*\newqedtheorem@unstarred[2]{%
  \newtheorem{#1*}[theorem*]{#2}
  \newqedtheoremx{#1}{#1*}
}
\newcommand*\newqedtheorem@starred[2]{%
  \newtheorem*{#1*}{#2}
  \newqedtheoremx{#1}{#1*}
}
\newcommand*{\newqedtheorem}{\@ifstar{\newqedtheorem@starred}{\newqedtheorem@unstarred}}

\newtheorem{sctheorem*}{定理}
\newscreentheoremx{sctheorem}{sctheorem*}
\newcommand*\newscreentheorem@unstarred[2]{%
  \newtheorem{#1*}[theorem*]{#2}
  \newscreentheoremx{#1}{#1*}
}
\newcommand*\newscreentheorem@starred[2]{%
  \newtheorem*{#1*}{#2}
  \newscreentheoremx{#1}{#1*}
}
\newcommand*{\newscreentheorem}{\@ifstar{\newscreentheorem@starred}{\newscreentheorem@unstarred}}

%\newtheorem*{definition}{定義}
%\newtheorem{theorem}{定理}
%\newtheorem{proposition}[theorem]{命題}
%\newtheorem{lemma}[theorem]{補題}
%\newtheorem{corollary}[theorem]{系}

\newqedtheorem{lemma}{補題}
\newqedtheorem{corollary}{系}
\newqedtheorem{example}{例}
\newqedtheorem{proposition}{命題}
\newqedtheorem{remark}{注意}
\newqedtheorem{thesis}{主張}
\newqedtheorem{notation}{記法}
\newqedtheorem{problem}{問題}
\newqedtheorem{algorithm}{アルゴリズム}

\newscreentheorem*{definition}{定義}

\renewenvironment{proof}[1][\proofname]{\par
  \normalfont
  \topsep6\p@\@plus6\p@ \trivlist
  \item[\hskip\labelsep{\bfseries #1}\@addpunct{\bfseries}]\ignorespaces\quad\par
}{%
  \qed\endtrivlist\@endpefalse
}
\renewcommand\proofname{証明}

\makeatother

\begin{document}
\maketitle
\tableofcontents
\clearpage

\section{記法・記号}
\begin{definition}[Kronecker のデルタ]
  \begin{align}
    \delta_{ij} = \begin{cases}
                    1 & (i = j)    \\
                    0 & (i \neq j)
                  \end{cases}
  \end{align}
\end{definition}
\begin{definition}[レビ・チビタの完全反対称テンソル (Levi-Civita antisymmetric tensor)]
  $\epsilon_{\mu_1\cdots\mu_k}$ は $\mu_1\cdots\mu_k$ が順列のとき $1\cdots k$ の偶置換なら $1$、奇置換なら $-1$ とする。順列ではないときは $0$ とする。
  \begin{align}
    \epsilon_{\mu_1\cdots\mu_k} & :=
    \begin{dcases}
      \sgn\begin{pmatrix}
            1     & \cdots & k     \\
            \mu_1 & \cdots & \mu_k
          \end{pmatrix} & (\mu_1\cdots\mu_k が順列のとき) \\
      0                         & (else)
    \end{dcases} \\
                                & =
    \begin{dcases}
      1  & (\mu_1\cdots\mu_k が偶置換のとき) \\
      -1 & (\mu_1\cdots\mu_k が奇置換のとき) \\
      0  & (else)
    \end{dcases}
  \end{align}
\end{definition}

\begin{definition}[Einstein の縮約記法]
  同じ項で添字が重なる場合はその添字について和を取る.
  \begin{align}
    A_iB_i = \sum_{i=x,y,z}A_iB_i
  \end{align}
\end{definition}


\section{ベクトル空間}

\begin{definition}[ベクトル空間]
  体 $K$ における加群をベクトル空間である。
\end{definition}

\begin{definition}
  ベクトルについて内積と外積を定義する。
  \begin{align}
    \bm{A}\vdot\bm{B}  & = A_xB_x + A_yB_y + A_zB_z = A_iB_i                                                  \\
    \bm{A}\cross\bm{B} & = (A_yB_z - A_zB_y, A_zB_x - A_xB_z, A_xB_y - A_yB_x) = \bm{e}_i\epsilon_{ijk}A_jB_k
  \end{align}
\end{definition}

\section{ベクトル解析}
\begin{definition}
  ベクトル $\bm{A} = (A_x, A_y, A_z)$ について勾配 $\grad{\bm{A}}$ と発散 $\div{\bm{A}}$、回転 $\rot{\bm{A}}$ を次のように定義する。
  \begin{align}
    \grad{\bm{A}} & = \qty(\pdv{A_x}{x}, \pdv{A_y}{y}, \pdv{A_z}{z})                                                                                    \\
    \div{\bm{A}}  & = \pdv{A_x}{x} + \pdv{A_y}{y} + \pdv{A_z}{z} = \partial_iA_i                                                                        \\
    \rot{\bm{A}}  & = \qty(\pdv{A_z}{y} - \pdv{A_y}{z}, \pdv{A_x}{z} - \pdv{A_z}{x}, \pdv{A_y}{x} - \pdv{A_x}{y}) = \bm{e}_i\epsilon_{ijk}\partial_jA_k
  \end{align}
\end{definition}

\begin{theorem}[勾配・発散・回転の線形性]
  \begin{align}
    \grad{(f+g)}          & = \grad{f} + \grad{g}         \\
    \div{(\bm{A}+\bm{B})} & = \div{\bm{A}} + \div{\bm{B}} \\
    \rot{(\bm{A}+\bm{B})} & = \rot{\bm{A}} + \rot{\bm{B}}
  \end{align}
\end{theorem}
\begin{proof}
  \begin{align}
    (\grad{(f+g)})_i            & = \partial_i(f+g)                                           \\
                                & = \partial_i f + \partial_i g                               \\
                                & = (\grad{f} + \grad{g})_i                                   \\
    \div{(\bm{A}+\bm{B})}       & = \partial_i(A_i + B_i)                                     \\
                                & = \partial_i A_i + \partial_i B_i                           \\
                                & = \div{\bm{A}}+\div{\bm{B}}                                 \\
    (\rot{(\bm{A} + \bm{B})})_i & = \epsilon_{ijk}\partial_j(A_k + B_k)                       \\
                                & = \epsilon_{ijk}\partial_jA_k + \epsilon_{ijk}\partial_jB_k \\
                                & = \rot{\bm{A}} + \rot{\bm{B}}
  \end{align}
\end{proof}

\begin{theorem}[勾配・発散・回転のスカラー倍]
  \begin{align}
    \grad{(fg)}     & = f\grad{g} + g\grad{f}                    \\
    \div{(f\bm{A})} & = f(\div{\bm{A}}) + \bm{A}\vdot(\grad{f})  \\
    \rot{(f\bm{A})} & = f(\rot{\bm{A}}) - \bm{A}\cross(\grad{f})
  \end{align}
\end{theorem}
\begin{proof}
  \begin{align}
    (\grad{(fg)})_i     & = \partial_i(fg)                                              \\
                        & = f\partial_i g + g\partial_i f                               \\
                        & = (f\grad{g} + g\grad{f})_i                                   \\
    \div{(f\bm{A})}     & = \partial_i(fA_i)                                            \\
                        & = f\partial_iA_i + A_i\partial_if                             \\
                        & = f(\div{\bm{A}}) + \bm{A}\vdot(\grad{f})                     \\
    (\rot{(f\bm{A})})_i & = \epsilon_{ijk}\partial_j(fA_k)                              \\
                        & = f\epsilon_{ijk}\partial_jA_k + \epsilon_{ijk}A_k\partial_jf \\
                        & = f\epsilon_{ijk}\partial_jA_k - \epsilon_{ikj}A_k\partial_jf \\
                        & = (f(\rot{\bm{A}}) - \bm{A}\cross(\grad{f}))_i
  \end{align}
\end{proof}

\begin{theorem}
  \begin{align}
    \grad{(\bm{A}\vdot\bm{B})} & = \bm{A}\cross(\rot{\bm{B}}) + \bm{B}\cross(\rot{\bm{A}}) + (\bm{A}\vdot\vnabla)\bm{B} + (\bm{B}\vdot\vnabla)\bm{A} \\
    \div{(\bm{A}\cross\bm{B})} & = \bm{B}\vdot(\rot{\bm{A}}) - \bm{A}\vdot(\rot{\bm{B}})                                                             \\
    \rot{(\bm{A}\cross\bm{B})} & = \bm{A}(\div{\bm{B}}) - \bm{B}(\div{\bm{A}}) + (\bm{B}\vdot\vnabla)\bm{A} - (\bm{A}\vdot\vnabla)\bm{B}
  \end{align}
\end{theorem}
\begin{proof}
  \begin{align}
    (\grad{(\bm{A}\vdot\bm{B})})_i & = \partial_i(A_jB_j)                                                                                                                \\
                                   & = (A_j\partial_iB_j + B_j\partial_iA_j) - (A_j\partial_jB_i + B_j\partial_jA_i) + (A_j\partial_jB_i + B_j\partial_jA_i)             \\
                                   & = (\delta_{il}\delta_{jm} - \delta_{im}\delta_{jl})(A_j\partial_lB_m + B_j\partial_lA_m) + (A_j\partial_jB_i + B_j\partial_jA_i)    \\
                                   & = \epsilon_{kij}\epsilon_{klm}(A_j\partial_lB_m + B_j\partial_lA_m) + (A_j\partial_jB_i + B_j\partial_jA_i)                         \\
                                   & = \epsilon_{ijk}A_j\epsilon_{klm}\partial_lB_m + \epsilon_{ijk}B_j\epsilon_{klm}\partial_lA_m + A_j\partial_jB_i + B_j\partial_jA_i \\
                                   & = (\bm{A}\cross(\rot{\bm{B}}) + \bm{B}\cross(\rot{\bm{A}}) + (\bm{A}\vdot\vnabla)\bm{B} + (\bm{B}\vdot\vnabla)\bm{A})_i             \\
    \div{(\bm{A}\cross\bm{B})}     & = \partial_i(\epsilon_{ijk}A_jB_k)                                                                                                  \\
                                   & = \epsilon_{ijk}(B_k\partial_iA_j + A_j\partial_iB_k)                                                                               \\
                                   & = B_k\epsilon_{kij}\partial_iA_j - A_j\epsilon_{jik}\partial_iB_k                                                                   \\
                                   & = \bm{B}\vdot(\rot{\bm{A}}) - \bm{A}\vdot(\rot{\bm{B}})                                                                             \\
    (\rot{(\bm{A}\cross\bm{B})})_i & = \epsilon_{ijk}\partial_j\epsilon_{klm}A_lB_m                                                                                      \\
                                   & = \epsilon_{kij}\epsilon_{klm}(B_m\partial_jA_l + A_l\partial_jB_m)                                                                 \\
                                   & = (\delta_{il}\delta_{jm} - \delta_{im}\delta_{jl})(B_m\partial_jA_l + A_l\partial_jB_m)                                            \\
                                   & = (B_j\partial_jA_i + A_i\partial_jB_j) - (B_i\partial_jA_j + A_j\partial_jB_i)                                                     \\
                                   & = A_i\partial_jB_j - B_i\partial_jA_j + B_j\partial_jA_i - A_j\partial_jB_i                                                         \\
                                   & = \bm{A}(\div{\bm{B}}) + \bm{B}(\div{\bm{A}}) + (\bm{B}\vdot\vnabla)\bm{A} - (\bm{A}\vdot\vnabla)\bm{B}
  \end{align}
\end{proof}

\begin{theorem}
  \begin{align}
    \div{(\rot{\bm{A}})} & = 0                                          \\
    \rot{(\grad{f})}     & = \bm{0}                                     \\
    \rot{(\rot{\bm{A}})} & = \div{(\grad{\bm{A}})} - \laplacian{\bm{A}}
  \end{align}
\end{theorem}
\begin{proof}
  \begin{align}
    \div{(\rot{\bm{A}})} & = \partial_i(\epsilon_{ijk}\partial_jA_k) \\
                         & = \epsilon_{ijk}\partial_i\partial_jA_k
  \end{align}
  このとき対称性から $i$, $j$ を交換しても等しい。
  \begin{align}
         & \epsilon_{ijk}\partial_i\partial_jA_k = \epsilon_{jik}\partial_j\partial_iA_k  \\
    \iff & \epsilon_{ijk}\partial_i\partial_jA_k = -\epsilon_{ijk}\partial_i\partial_jA_k \\
    \iff & \epsilon_{ijk}\partial_i\partial_jA_k = 0
  \end{align}
  よって
  \begin{align}
    \div{(\rot{\bm{A}})} & = 0
  \end{align}
  他も同様にして
  \begin{align}
    (\rot{(\grad{f})})_i     & = \epsilon_{ijk}\partial_j\partial_kf = 0                                  \\
    (\rot{(\rot{\bm{A}})})_i & = \epsilon_{ijk}\partial_j(\epsilon_{klm}\partial_lA_m)                    \\
                             & = \epsilon_{kij}\epsilon_{klm}\partial_j\partial_lA_m                      \\
                             & = (\delta_{il}\delta_{jm} - \delta_{im}\delta_{jl})\partial_j\partial_lA_m \\
                             & = \partial_j\partial_iA_j - \partial_j^2A_i                                \\
                             & = (\div{(\grad{\bm{A}})} - \laplacian{\bm{A}})_i
  \end{align}
\end{proof}

\end{document}