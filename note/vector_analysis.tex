\RequirePackage{plautopatch}
\documentclass[uplatex,dvipdfmx,a4paper,11pt]{jlreq}
\usepackage{bxpapersize}
\usepackage[utf8]{inputenc}
\usepackage{fontenc}
\usepackage{lmodern}
\usepackage{otf}
\usepackage{amsmath}
\usepackage{amssymb}
\usepackage{amsthm}
\usepackage{ascmac}
% \usepackage[hyphens]{url}
\usepackage{physics2}
\usephysicsmodule{ab, ab.braket, doubleprod, diagmat, xmat}
\usepackage{diffcoeff}
\usepackage{verbatimbox}
\usepackage{bm}
\usepackage{url}
% \usepackage[dvipdfmx,hiresbb,final]{graphicx}
\usepackage{hyperref}
\usepackage{pxjahyper}
\usepackage{tikz}\usetikzlibrary{cd}
\usepackage{listings}
\usepackage{color}
\usepackage{mathtools}
\usepackage{xspace}
\usepackage{xy}
\usepackage{xypic}
%
\title{ベクトル解析}
\author{Anko}
\makeatletter
%
\DeclareMathOperator{\lcm}{lcm}
\DeclareMathOperator{\Kernel}{Ker}
\DeclareMathOperator{\Image}{Im}
\DeclareMathOperator{\ch}{ch}
\DeclareMathOperator{\Aut}{Aut}
\DeclareMathOperator{\Log}{Log}
\DeclareMathOperator{\Arg}{Arg}
\DeclareMathOperator{\sgn}{sgn}
%
\newcommand{\CC}{\mathbb{C}}
\newcommand{\RR}{\mathbb{R}}
\newcommand{\QQ}{\mathbb{Q}}
\newcommand{\ZZ}{\mathbb{Z}}
\newcommand{\NN}{\mathbb{N}}
\newcommand{\FF}{\mathbb{F}}
\newcommand{\PP}{\mathbb{P}}
\newcommand{\GG}{\mathbb{G}}
\newcommand{\TT}{\mathbb{T}}
\newcommand{\calB}{\mathcal{B}}
\newcommand{\calF}{\mathcal{F}}
\newcommand{\ignore}[1]{}
\newcommand{\floor}[1]{\left\lfloor #1 \right\rfloor}
% \newcommand{\abs}[1]{\left\lvert #1 \right\rvert}
\newcommand{\lt}{<}
\newcommand{\gt}{>}
\newcommand{\id}{\mathrm{id}}
% \newcommand{\rot}{\curl}
\renewcommand{\angle}[1]{\left\langle #1 \right\rangle}
\renewcommand{\AA}{\bm{A}}
\newcommand{\BB}{\bm{B}}
\newcommand{\ee}{\bm{e}}
\newcommand{\kk}{\bm{k}}
\newcommand{\pp}{\bm{p}}
\newcommand{\grad}{\nabla}
\renewcommand{\div}{\nabla\cdot}
\newcommand{\rot}{\nabla\times}
\newcommand{\laplacian}{\nabla^2}

\let\oldcite=\cite
\renewcommand\cite[1]{\hyperlink{#1}{\oldcite{#1}}}

\let\oldbibitem=\bibitem
\renewcommand{\bibitem}[2][]{\label{#2}\oldbibitem[#1]{#2}}

% theorem環境の設定
% - 冒頭に改行
% - 末尾にdiamond (amsthm)
\theoremstyle{definition}
\newcommand*{\newscreentheoremx}[2]{
  \newenvironment{#1}[1][]{
    \begin{screen}
    \begin{#2}[##1]
      \leavevmode
      \newline
  }{
    \end{#2}
    \end{screen}
  }
}
\newcommand*{\newqedtheoremx}[2]{
  \newenvironment{#1}[1][]{
    \begin{#2}[##1]
      \leavevmode
      \newline
      \renewcommand{\qedsymbol}{\(\diamond\)}
      \pushQED{\qed}
  }{
      \qedhere
      \popQED
    \end{#2}
  }
}
\newtheorem{theorem*}{定理}

\newqedtheoremx{theorem}{theorem*}
\newcommand*\newqedtheorem@unstarred[2]{%
  \newtheorem{#1*}[theorem*]{#2}
  \newqedtheoremx{#1}{#1*}
}
\newcommand*\newqedtheorem@starred[2]{%
  \newtheorem*{#1*}{#2}
  \newqedtheoremx{#1}{#1*}
}
\newcommand*{\newqedtheorem}{\@ifstar{\newqedtheorem@starred}{\newqedtheorem@unstarred}}

\newtheorem{sctheorem*}{定理}
\newscreentheoremx{sctheorem}{sctheorem*}
\newcommand*\newscreentheorem@unstarred[2]{%
  \newtheorem{#1*}[theorem*]{#2}
  \newscreentheoremx{#1}{#1*}
}
\newcommand*\newscreentheorem@starred[2]{%
  \newtheorem*{#1*}{#2}
  \newscreentheoremx{#1}{#1*}
}
\newcommand*{\newscreentheorem}{\@ifstar{\newscreentheorem@starred}{\newscreentheorem@unstarred}}

%\newtheorem*{definition}{定義}
%\newtheorem{theorem}{定理}
%\newtheorem{proposition}[theorem]{命題}
%\newtheorem{lemma}[theorem]{補題}
%\newtheorem{corollary}[theorem]{系}

\newqedtheorem{lemma}{補題}
\newqedtheorem{corollary}{系}
\newqedtheorem{example}{例}
\newqedtheorem{proposition}{命題}
\newqedtheorem{remark}{注意}
\newqedtheorem{thesis}{主張}
\newqedtheorem{notation}{記法}
\newqedtheorem{problem}{問題}
\newqedtheorem{algorithm}{アルゴリズム}

\newscreentheorem*{definition}{定義}

\renewenvironment{proof}[1][\proofname]{\par
  \normalfont
  \topsep6\p@\@plus6\p@ \trivlist
  \item[\hskip\labelsep{\bfseries #1}\@addpunct{\bfseries}]\ignorespaces\quad\par
}{%
  \qed\endtrivlist\@endpefalse
}
\renewcommand\proofname{証明}

\makeatother

\begin{document}
\maketitle
\tableofcontents
\clearpage

\section{ベクトル空間}
\subsection{ベクトルの定義}
\begin{definition}[Einstein の縮約記法]
  同じ項で添字が重なる場合にはその添字について和を取る。
\end{definition}

\begin{definition}[ベクトル空間]
  体 $K$ 上の加群を $K$ 上のベクトル空間といい、ベクトル空間の元をベクトルという。
  $\AA = A_i\ee_i = A_1\ee_1 + A_2\ee_2 + A_3\ee_3$
\end{definition}

\begin{definition}[ベクトル空間における内積と外積]
  ベクトル $\AA = A_i\ee_i, \BB = B_j\ee_j$ における内積と外積を定義する。
  \begin{align}
    \AA\cdot\BB  & = A_i\ee_i\cdot B_j\ee_j = g_{ij}A_iB_j                                    \\
    \AA\times\BB & = (A_2B_3 - A_3B_2)\ee_1 + (A_3B_1 - A_1B_3)\ee_2 + (A_1B_2 - A_2B_1)\ee_3
  \end{align}
\end{definition}

\begin{theorem}
  内積と外積について Einstein の縮約記法を用いて次のように書ける。
  \begin{align}
    \AA\cdot\BB  & = A_i\ee_i\cdot B_j\ee_j = A_iB_i \\
    \AA\times\BB & = \bm{e}_i\varepsilon_{ijk}A_jB_k
  \end{align}
\end{theorem}
\begin{proof}
  内積については自明。外積について次のように求められる。
  \begin{align}
    (\AA\times\BB)_1 & = \varepsilon_{1jk}A_jB_k = A_2B_3 - A_3B_2 \\
    (\AA\times\BB)_2 & = \varepsilon_{2jk}A_jB_k = A_3B_1 - A_1B_3 \\
    (\AA\times\BB)_3 & = \varepsilon_{3jk}A_jB_k = A_1B_2 - A_2B_1
  \end{align}
\end{proof}



\section{ベクトル解析}
\begin{definition}[Kronecker のデルタ]
  \begin{align}
    \delta_{ij} = \begin{cases}
                    1 & (i = j)    \\
                    0 & (i \neq j)
                  \end{cases}
  \end{align}
\end{definition}

\begin{definition}[レビ・チビタの完全反対称テンソル (Levi-Civita antisymmetric tensor)]
  $\varepsilon_{\mu_1\cdots\mu_k}$ は $\mu_1\cdots\mu_k$ が順列のとき $1\cdots k$ の偶置換なら $1$、奇置換なら $-1$ とする。順列ではないときは $0$ とする。
  \begin{align}
    \varepsilon_{\mu_1\cdots\mu_k} & :=
    \begin{dcases}
      \sgn\begin{pmatrix}
            1     & \cdots & k     \\
            \mu_1 & \cdots & \mu_k
          \end{pmatrix} & (\mu_1\cdots\mu_k が順列のとき) \\
      0                         & (else)
    \end{dcases} \\
                                   & =
    \begin{dcases}
      1  & (\mu_1\cdots\mu_k が偶置換のとき) \\
      -1 & (\mu_1\cdots\mu_k が奇置換のとき) \\
      0  & (else)
    \end{dcases}
  \end{align}
\end{definition}
\begin{theorem}
  $f_{ij} = f_{ji}$ と対称性があるとき $\varepsilon_{ijk}f_{ij} = 0$ となる。
\end{theorem}
\begin{proof}
  $i, j$ を交換しても等しいことから
  \begin{align}
    \varepsilon_{ijk}f_{ij} = \varepsilon_{jik}f_{ji} = -\varepsilon_{ijk}f_{ij} = 0
  \end{align}
  となる。
\end{proof}

\begin{definition}
  ベクトル $\AA = A_i\ee_i$ について勾配 $\grad{\AA}$ と発散 $\div{\AA}$、回転 $\rot{\AA}$ を次のように定義する。
  \begin{align}
    \grad{\bm{A}} & = \ee_i\partial_iA_i                     \\
    \div{\bm{A}}  & = \partial_iA_i                          \\
    \rot{\bm{A}}  & = \bm{e}_i\varepsilon_{ijk}\partial_jA_k
  \end{align}
  ただし
  \begin{align}
    \partial_i = \diffp{}{x^i}
  \end{align}
\end{definition}

\begin{theorem}[勾配・発散・回転の線形性]
  それぞれ線形性が成り立つ。
  \begin{align}
    \grad{(f+g)}          & = \grad{f} + \grad{g}         \\
    \div{(\bm{A}+\bm{B})} & = \div{\bm{A}} + \div{\bm{B}} \\
    \rot{(\bm{A}+\bm{B})} & = \rot{\bm{A}} + \rot{\bm{B}}
  \end{align}
\end{theorem}
\begin{proof}
  \begin{align}
    \grad{(f+g)}            & = \ee_i\partial_i(f+g)                                                      \\
                            & = \ee_i\partial_i f + \ee_i\partial_i g                                     \\
                            & = \grad{f} + \grad{g}                                                       \\
    \div{(\bm{A}+\bm{B})}   & = \partial_i(A_i + B_i)                                                     \\
                            & = \partial_i A_i + \partial_i B_i                                           \\
                            & = \div{\bm{A}}+\div{\bm{B}}                                                 \\
    \rot{(\bm{A} + \bm{B})} & = \ee_i\varepsilon_{ijk}\partial_j(A_k + B_k)                               \\
                            & = \ee_i\varepsilon_{ijk}\partial_jA_k + \ee_i\varepsilon_{ijk}\partial_jB_k \\
                            & = \rot{\bm{A}} + \rot{\bm{B}}
  \end{align}
\end{proof}

\begin{theorem}[スカラー倍の勾配・発散・回転]
  スカラー倍はそれぞれ次のようになる。
  \begin{align}
    \grad{(fg)}     & = f\grad{g} + g\grad{f}                    \\
    \div{(f\bm{A})} & = f(\div{\bm{A}}) + \bm{A}\cdot(\grad{f})  \\
    \rot{(f\bm{A})} & = f(\rot{\bm{A}}) - \bm{A}\times(\grad{f})
  \end{align}
\end{theorem}
\begin{proof}
  \begin{align}
    \grad{(fg)}     & = \ee_i\partial_i(fg)                                                         \\
                    & = f\ee_i\partial_i g + g\ee_i\partial_i f                                     \\
                    & = f\grad{g} + g\grad{f}                                                       \\
    \div{(f\bm{A})} & = \partial_i(fA_i)                                                            \\
                    & = f\partial_iA_i + A_i\partial_if                                             \\
                    & = f(\div{\bm{A}}) + \bm{A}\cdot(\grad{f})                                     \\
    \rot{(f\bm{A})} & = \ee_i\varepsilon_{ijk}\partial_j(fA_k)                                      \\
                    & = f\ee_i\varepsilon_{ijk}\partial_jA_k + \ee_i\varepsilon_{ijk}A_k\partial_jf \\
                    & = f\ee_i\varepsilon_{ijk}\partial_jA_k - \ee_i\varepsilon_{ikj}A_k\partial_jf \\
                    & = f(\rot{\bm{A}}) - \bm{A}\times(\grad{f})
  \end{align}
\end{proof}

\begin{theorem}[ベクトルの内積・外積の勾配・発散・回転]
  \begin{align}
    \grad{(\bm{A}\cdot\bm{B})} & = \bm{A}\times(\rot{\bm{B}}) + \bm{B}\times(\rot{\bm{A}}) + (\bm{A}\cdot\nabla)\bm{B} + (\bm{B}\cdot\nabla)\bm{A} \\
    \div{(\bm{A}\times\bm{B})} & = \bm{B}\cdot(\rot{\bm{A}}) - \bm{A}\cdot(\rot{\bm{B}})                                                           \\
    \rot{(\bm{A}\times\bm{B})} & = \bm{A}(\div{\bm{B}}) - \bm{B}(\div{\bm{A}}) + (\bm{B}\cdot\nabla)\bm{A} - (\bm{A}\cdot\nabla)\bm{B}
  \end{align}
\end{theorem}
\begin{proof}
  \begin{align}
    \grad{(\bm{A}\cdot\bm{B})} & = \ee_i\partial_i(A_jB_j)                                                                                                                                           \\
                               & = \ee_i((A_j\partial_iB_j + B_j\partial_iA_j) - (A_j\partial_jB_i + B_j\partial_jA_i) + (A_j\partial_jB_i + B_j\partial_jA_i))                                      \\
                               & = \ee_i(\delta_{il}\delta_{jm} - \delta_{im}\delta_{jl})(A_j\partial_lB_m + B_j\partial_lA_m) + \ee_i(A_j\partial_jB_i + B_j\partial_jA_i)                          \\
                               & = \ee_i\varepsilon_{kij}\varepsilon_{klm}(A_j\partial_lB_m + B_j\partial_lA_m) + \ee_i(A_j\partial_jB_i + B_j\partial_jA_i)                                         \\
                               & = \ee_i\varepsilon_{ijk}A_j\varepsilon_{klm}\partial_lB_m + \ee_i\varepsilon_{ijk}B_j\varepsilon_{klm}\partial_lA_m + \ee_iA_j\partial_jB_i + \ee_iB_j\partial_jA_i \\
                               & = \bm{A}\times(\rot{\bm{B}}) + \bm{B}\times(\rot{\bm{A}}) + (\bm{A}\cdot\nabla)\bm{B} + (\bm{B}\cdot\nabla)\bm{A}                                                   \\
    \div{(\bm{A}\times\bm{B})} & = \partial_i(\varepsilon_{ijk}A_jB_k)                                                                                                                               \\
                               & = \varepsilon_{ijk}(B_k\partial_iA_j + A_j\partial_iB_k)                                                                                                            \\
                               & = B_k\varepsilon_{kij}\partial_iA_j - A_j\varepsilon_{jik}\partial_iB_k                                                                                             \\
                               & = \bm{B}\cdot(\rot{\bm{A}}) - \bm{A}\cdot(\rot{\bm{B}})                                                                                                             \\
    \rot{(\bm{A}\times\bm{B})} & = \ee_i\varepsilon_{ijk}\partial_j\varepsilon_{klm}A_lB_m                                                                                                           \\
                               & = \ee_i\varepsilon_{kij}\varepsilon_{klm}(B_m\partial_jA_l + A_l\partial_jB_m)                                                                                      \\
                               & = \ee_i(\delta_{il}\delta_{jm} - \delta_{im}\delta_{jl})(B_m\partial_jA_l + A_l\partial_jB_m)                                                                       \\
                               & = \ee_i(B_j\partial_jA_i + A_i\partial_jB_j) - \ee_i(B_i\partial_jA_j + A_j\partial_jB_i)                                                                           \\
                               & = \ee_iA_i\partial_jB_j - \ee_iB_i\partial_jA_j + \ee_iB_j\partial_jA_i - \ee_iA_j\partial_jB_i                                                                     \\
                               & = \bm{A}(\div{\bm{B}}) + \bm{B}(\div{\bm{A}}) + (\bm{B}\cdot\nabla)\bm{A} - (\bm{A}\cdot\nabla)\bm{B}
  \end{align}
\end{proof}

\begin{theorem}[有名定理]
  \begin{align}
    \div{(\rot{\bm{A}})} & = 0                                          \\
    \rot{(\grad{f})}     & = \bm{0}                                     \\
    \rot{(\rot{\bm{A}})} & = \div{(\grad{\bm{A}})} - \laplacian{\bm{A}}
  \end{align}
\end{theorem}
\begin{proof}
  \begin{align}
    \div{(\rot{\bm{A}})} & = \partial_i(\varepsilon_{ijk}\partial_jA_k)                                    \\
                         & = \varepsilon_{ijk}\partial_i\partial_jA_k = 0                                  \\
    \rot{(\grad{f})}     & = \ee_i\varepsilon_{ijk}\partial_j\partial_kf = 0                               \\
    \rot{(\rot{\bm{A}})} & = \ee_i\varepsilon_{ijk}\partial_j(\varepsilon_{klm}\partial_lA_m)              \\
                         & = \ee_i\varepsilon_{kij}\varepsilon_{klm}\partial_j\partial_lA_m                \\
                         & = \ee_i(\delta_{il}\delta_{jm} - \delta_{im}\delta_{jl})\partial_j\partial_lA_m \\
                         & = \ee_i\partial_j\partial_iA_j - \ee_i\partial_j^2A_i                           \\
                         & = \grad{(\div{\bm{A}})} - \laplacian{\bm{A}}
  \end{align}
\end{proof}

\begin{theorem}[Gauss の定理]
  \begin{align}
    \int_V\div{\AA}\dl{V} & = \oint_{\partial V}\AA\cdot\dl{\bm{S}}
  \end{align}
\end{theorem}

\end{document}