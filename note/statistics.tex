\RequirePackage{plautopatch}
\documentclass[uplatex,diffipdfmx,a4paper,11pt]{jlreq}
\usepackage{bxpapersize}
\usepackage[utf8]{inputenc}
\usepackage{fontenc}
\usepackage{lmodern}
\usepackage{otf}
\usepackage{amsmath}
\usepackage{amssymb}
\usepackage{amsthm}
\usepackage{ascmac}
% \usepackage[hyphens]{url}
\usepackage{physics2}
\usephysicsmodule{ab, ab.braket, doubleprod, diagmat, xmat}
\usepackage{diffcoeff}
\usepackage{verbatimbox}
\usepackage{bm}
\usepackage{url}
\usepackage{siunitx}
% \usepackage[diffipdfmx,hiresbb,final]{graphicx}
\usepackage{hyperref}
\usepackage{pxjahyper}
\usepackage{tikz}\usetikzlibrary{cd}
\usepackage{listings}
\usepackage{color}
\usepackage{mathtools}
\usepackage{xspace}
\usepackage{xy}
\usepackage{xypic}
%
\title{統計力学}
\author{anko9801}
\makeatletter
%
\DeclareMathOperator{\lcm}{lcm}
\DeclareMathOperator{\Kernel}{Ker}
\DeclareMathOperator{\Image}{Im}
\DeclareMathOperator{\ch}{ch}
\DeclareMathOperator{\Aut}{Aut}
\DeclareMathOperator{\Log}{Log}
\DeclareMathOperator{\Arg}{Arg}
\DeclareMathOperator{\sgn}{sgn}
%
\newcommand{\CC}{\mathbb{C}}
\newcommand{\RR}{\mathbb{R}}
\newcommand{\QQ}{\mathbb{Q}}
\newcommand{\ZZ}{\mathbb{Z}}
\newcommand{\NN}{\mathbb{N}}
\newcommand{\FF}{\mathbb{F}}
\newcommand{\PP}{\mathbb{P}}
\newcommand{\GG}{\mathbb{G}}
\newcommand{\TT}{\mathbb{T}}
\newcommand{\calB}{\mathcal{B}}
\newcommand{\calF}{\mathcal{F}}
\newcommand{\ignore}[1]{}
\newcommand{\floor}[1]{\left\lfloor #1 \right\rfloor}
% \newcommand{\abs}[1]{\left\lvert #1 \right\rvert}
\newcommand{\lt}{<}
\newcommand{\gt}{>}
\newcommand{\id}{\mathrm{id}}
\newcommand{\rot}{\curl}
\renewcommand{\angle}[1]{\left\langle #1 \right\rangle}
\newcommand{\EE}{\bm{E}}
\newcommand{\BB}{\bm{B}}
\renewcommand{\AA}{\bm{A}}
\newcommand{\rr}{\bm{r}}
\newcommand{\kk}{\bm{k}}
\newcommand{\pp}{\bm{p}}
\newcommand\mqty[1]{\begin{pmatrix}#1\end{pmatrix}}
\newcommand\vmqty[1]{\begin{vmatrix}#1\end{vmatrix}}
\numberwithin{equation}{section}

\let\oldcite=\cite
\renewcommand\cite[1]{\hyperlink{#1}{\oldcite{#1}}}

\let\oldbibitem=\bibitem
\renewcommand{\bibitem}[2][]{\label{#2}\oldbibitem[#1]{#2}}

% theorem環境の設定
% - 冒頭に改行
% - 末尾にdiamond (amsthm)
\theoremstyle{definition}
\newcommand*{\newscreentheoremx}[2]{
  \newenvironment{#1}[1][]{
    \begin{screen}
    \begin{#2}[##1]
      \leavevmode
      \newline
  }{
    \end{#2}
    \end{screen}
  }
}
\newcommand*{\newqedtheoremx}[2]{
  \newenvironment{#1}[1][]{
    \begin{#2}[##1]
      \leavevmode
      \newline
      \renewcommand{\qedsymbol}{\(\diamond\)}
      \pushQED{\qed}
  }{
      \qedhere
      \popQED
    \end{#2}
  }
}
\newtheorem{theorem*}{定理}[section]

\newqedtheoremx{theorem}{theorem*}
\newcommand*\newqedtheorem@unstarred[2]{%
  \newtheorem{#1*}[theorem*]{#2}
  \newqedtheoremx{#1}{#1*}
}
\newcommand*\newqedtheorem@starred[2]{%
  \newtheorem*{#1*}{#2}
  \newqedtheoremx{#1}{#1*}
}
\newcommand*{\newqedtheorem}{\@ifstar{\newqedtheorem@starred}{\newqedtheorem@unstarred}}

\newtheorem{sctheorem*}{定理}
\newscreentheoremx{sctheorem}{sctheorem*}
\newcommand*\newscreentheorem@unstarred[2]{%
  \newtheorem{#1*}[theorem*]{#2}
  \newscreentheoremx{#1}{#1*}
}
\newcommand*\newscreentheorem@starred[2]{%
  \newtheorem*{#1*}{#2}
  \newscreentheoremx{#1}{#1*}
}
\newcommand*{\newscreentheorem}{\@ifstar{\newscreentheorem@starred}{\newscreentheorem@unstarred}}

%\newtheorem*{definition}{定義}
%\newtheorem{theorem}{定理}
%\newtheorem{proposition}[theorem]{命題}
%\newtheorem{lemma}[theorem]{補題}
%\newtheorem{corollary}[theorem]{系}

\newqedtheorem{lemma}{補題}
\newqedtheorem{corollary}{系}
\newqedtheorem{example}{例}
\newqedtheorem{proposition}{命題}
\newqedtheorem{remark}{注意}
\newqedtheorem{thesis}{主張}
\newqedtheorem{notation}{記法}
\newqedtheorem{problem}{問題}
\newqedtheorem{algorithm}{アルゴリズム}

\newscreentheorem*{axiom}{公理}
\newscreentheorem*{definition}{定義}

\renewenvironment{proof}[1][\proofname]{\par
  \normalfont
  \topsep6\p@\@plus6\p@ \trivlist
  \item[\hskip\labelsep{\bfseries #1}\@addpunct{\bfseries}]\ignorespaces\quad\par
}{%
  \qed\endtrivlist\@endpefalse
}
\renewcommand\proofname{証明}

\makeatother

\begin{document}
\maketitle
\tableofcontents
\clearpage

\section{統計力学の基礎}
エルゴード理論により次の原理が成り立つこととする。
\begin{axiom}[等確率の原理]
  孤立系を十分に長時間放置しておくと物体の実現可能な量子状態はエネルギーのゆらぎを除いてすべて等確率で実現する。
\end{axiom}

2 つの系 $A, B$ があるとする。系 $A$ のエネルギー $E_A$ と系 $B$ のエネルギー $E_B$ の和が一定で $A, B$ の間にエネルギーのやり取りができるとする。
\begin{align}
  E_A + E_B = const.
\end{align}

例えると子どもたちが 12 人居て $A$ と $B$ のグループにそれぞれ 4 人、8 人で分ける。そして 6 個あるリンゴを 1 人複数個もらっても良いとして等確率に配ったとき、それぞれのグループに配られるリンゴで最も確率の高いものは何か。
\begin{table}[hbtp]
  \label{table:micro}
  \centering
  \begin{tabular}{|c|c|l|}
    \hline
    A   & B   & 組合せ                                                         \\
    \hline
    0 個 & 6 個 & ${}_4H_0\times {}_{8}H_6 = {}_3C_0\times {}_{13}C_6 = 1716$ \\
    1 個 & 5 個 & ${}_4H_1\times {}_{8}H_5 = {}_4C_1\times {}_{12}C_5 = 3168$ \\
    2 個 & 4 個 & ${}_4H_2\times {}_{8}H_4 = {}_5C_2\times {}_{11}C_4 = 3300$ \\
    3 個 & 3 個 & ${}_4H_3\times {}_{8}H_3 = {}_6C_3\times {}_{10}C_3 = 2406$ \\
    4 個 & 2 個 & ${}_4H_4\times {}_{8}H_2 = {}_7C_4\times {}_{9}C_2 = 1260$  \\
    5 個 & 1 個 & ${}_4H_5\times {}_{8}H_1 = {}_8C_5\times {}_{8}C_1 = 448$   \\
    6 個 & 0 個 & ${}_4H_6\times {}_{8}H_0 = {}_9C_6\times {}_{7}C_0 = 84$    \\
    \hline
  \end{tabular}
  \caption{組合せ}
\end{table}

より $A$, $B$ のグループにそれぞれ 2 個、4 個で分ける確率が最も高い。
この分布を二項分布という。

\begin{proposition}
  二項分布の極限が正規分布である。
\end{proposition}
\begin{proof}
\end{proof}

\begin{definition}
  あるエネルギー $E$ のときに実現可能な量子状態数を $W(E)$ とおき、その対数を取ったものをエントロピー $S(E)$ という。
  \begin{align}
    S(E) & = k_B\log W(E)                     \\
    k_B  & = 1.380658\times 10^{-23} \si{J/K}
  \end{align}
  ただし $k_B$ をボルツマン定数 (Boltzmann constant) という。ある系 $X$ のエネルギーを $E_X$、状態数を $W_X(E_X)$、エントロピーを $S_X(E_X)$ と書くことにする。
\end{definition}
状態数で計算すると指数が出がちなのでエントロピーで計算すると簡単になる。

\begin{theorem}
  $N$ 次元の調和振動子で $E = M\hbar\omega$ とおくと状態数とエントロピーは次のように書ける。
  \begin{align}
    W(E) & = \mqty{M + N - 1                                                                       \\ N - 1} \\
    S(E) & \approx k_BN\ab(\ab(1 + \frac{M}{N})\log(1 + \frac{M}{N}) - \frac{M}{N}\log\frac{M}{N})
  \end{align}
\end{theorem}
\begin{proof}
  $N$ 次元の調和振動子系では $(n_1,\ldots,n_N)$ が全体の量子状態を決める量子数となる。このときのエネルギーは次のように表される。
  \begin{align}
    E_{(n_1,\ldots,n_N)} & = n_1\hbar\omega + \cdots + n_N\hbar\omega
  \end{align}
  等しいエネルギーの状態の条件は $M = n_1 + \cdots + n_N$ と書ける。これより状態数の組合せは次のように書ける。
  \begin{align}
    W(E) & = \mqty{M + N - 1 \\ N - 1} = \frac{(M + N - 1)!}{(N - 1)!M!}
  \end{align}
  またエントロピーは Stirling の公式 $\log n! \approx n(\log n - 1)$ を用いて
  \begin{align}
    S(E) & = k_B\log W(E)                                                                    \\
         & = k_B\log\frac{(M + N - 1)!}{(N - 1)!M!}                                          \\
         & \approx k_B\ab((N + M)(\log(N + M) - 1) - N(\log N - 1) - M(\log M - 1))          \\
         & = k_BN\ab(\ab(1 + \frac{M}{N})\log(1 + \frac{M}{N}) - \frac{M}{N}\log\frac{M}{N})
  \end{align}
\end{proof}

\begin{theorem}
  熱平衡の条件は系 $A$ の温度 $T_A$ と系 $B$ の温度 $T_B$ が一致すること。
\end{theorem}
\begin{proof}
  各系の状態数の積が全体系の状態数となるので各系と全体系のエントロピーの関係は
  \begin{align}
    S(E_A, E_B) & = k_B\log W(E_A, E_B)                 \\
                & = k_B\log W_A(E_A)W_B(E_B)            \\
                & = k_B\log W_A(E_A) + k_B\log W_B(E_B) \\
                & = S_A(E_A) + S_B(E_B)
  \end{align}
  となる。このとき熱平衡状態とはエントロピーが最大の状態であるから $\diff*{S}{E_A} = 0$ となるエネルギー $E_A, E_B$ を考えると
  \begin{align}
    \diff{S(E_A, E_B)}{E_A}   & = \diff{S_A(E_A)}{E_A} + \diff{S_A(E_B)}{E_A} = \diff{S_A(E_A)}{E_A} - \diff{S_A(E_B)}{E_B} = 0 \\
    \iff \diff{S_A(E_A)}{E_A} & = \diff{S_A(E_B)}{E_B}
  \end{align}
  よりエントロピーのエネルギー微分を温度の逆数 $1/T$ と定義すると温度が一致するときに熱平衡状態となる。
\end{proof}

\begin{definition}
  絶対温度 (absolute temperature) $T$ を次のように定義する。
  \begin{align}
    \frac{1}{T} & := \diff{S}{E}
  \end{align}
\end{definition}

この温度の定義は理想気体で正当化される。

\begin{theorem}[理想気体]
  理想気体、つまり 3 次元箱型ポテンシャル中の独立な区別できない $N$ 個の粒子について
  \begin{align}
    S & = Nk_B\ab(\frac{3}{2}\ln\frac{E}{V} + \frac{5}{2}\ln\frac{V}{N} + \ln\alpha + \mathcal{O}(N^{-1}\ln N)) & \ab(\alpha = \ab(\frac{me}{3\pi\hbar^2})^{\frac{3}{2}}) \\
    E & = \frac{3}{2}Nk_BT
  \end{align}
\end{theorem}
\begin{proof}
  部分系の固有状態と固有エネルギーが分かれば全体系のも分かる。
  \begin{align}
    E_{(n_{i, a})_{i=1,\ldots,N,a=x,y,z}}    & = E_0\sum_{i=1}^{N}\sum_{a=x,y,z}n_{i,a}^2                                               \\
    \psi_{(n_{i, a})_{i=1,\ldots,N,a=x,y,z}} & = \ab(\frac{2}{L})^{3N/2}\prod_{i=1}^{N}\prod_{a=x,y,z}\sin(\frac{n_{i,a}\pi}{L}x_{i,a})
  \end{align}
  これよりあるエネルギー $E > 0$ 以下である区別できる固有状態数 $\Omega(E)$ について
  \begin{align}
    \Omega(E) & = \ab(半径 \sqrt{\frac{E}{E_0}} の 3N 次元超球の第一象限に含まれる格子点の個数)                                                       \\
              & \approx \frac{1}{2^{3N}}\sqrt{\frac{E}{E_0}}^{3N}\frac{\pi^{3N/2}}{(3N/2)!}                                    \\
              & = \frac{1}{(3N/2)!}\frac{\pi^{3N/2}}{2^{3N}}\ab(\frac{2mL^2}{\pi^2\hbar^2})^{3N/2}E^{3N/2}                     \\
              & = \frac{1}{(3N/2)!}\ab(\frac{m}{2\pi\hbar^2})^{3N/2}E^{3N/2}V^N                                                \\
              & = \frac{1}{\sqrt{3\pi N}(3N/2)^{3N/2}e^{-3N/2}}\ab(\frac{m}{2\pi\hbar^2})^{3N/2}E^{3N/2}V^N                    \\
              & = \frac{1}{\sqrt{3\pi N}}N^{N}\ab(\frac{me}{3\pi\hbar^2})^{3N/2}\ab(\frac{E}{V})^{3N/2}\ab(\frac{V}{N})^{5N/2} \\
  \end{align}
  これを区別しないから
  \begin{align}
    \Omega^{区別できない}(E) & =\frac{1}{N!}\Omega(E)                                                                                                                         \\
                       & = \frac{1}{N!}\frac{1}{\sqrt{3\pi N}}N^{N}\ab(\frac{me}{3\pi\hbar^2})^{3N/2}\ab(\frac{E}{V})^{3N/2}\ab(\frac{V}{N})^{5N/2}                     \\
                       & = \frac{1}{\sqrt{2\pi N}N^Ne^{-N}}\frac{1}{\sqrt{3\pi N}}N^{N}\ab(\frac{me}{3\pi\hbar^2})^{3N/2}\ab(\frac{E}{V})^{3N/2}\ab(\frac{V}{N})^{5N/2} \\
                       & = \frac{e^N}{\sqrt{6}\pi N}\ab(\frac{me}{3\pi\hbar^2})^{3N/2}\ab(\frac{E}{V})^{3N/2}\ab(\frac{V}{N})^{5N/2}
  \end{align}
  これよりエントロピーは
  \begin{align}
    S(E) & = k_B\ln\Omega^{区別できない}(E)                                                                                                                    \\
         & = Nk_B\ab(\frac{3}{2}\ln\frac{E}{V} + \frac{5}{2}\ln\frac{V}{N} + \frac{3}{2}\ln(\frac{me}{3\pi\hbar^2}) - \frac{1}{N}\ln(\sqrt{6}\pi N) + 1)
  \end{align}
  よって温度を計算すると式が示せる。
  \begin{align}
    \frac{1}{T} = \diff{S}{E} & = \frac{3}{2}Nk_B\frac{1}{E} \\
    E                         & = \frac{3}{2}Nk_BT
  \end{align}
\end{proof}

\section{ミクロカノニカル分布}
\subsection{ミクロカノニカルアンサンブル}
\begin{axiom}[等重率の原理]
  孤立した物理系 $X$ において、外部から指定されたある狭いエネルギー範囲 $[U - ∆U, U]$ に固有エネルギー $E_i$ が属するような微視的なエネルギー固有状態 $\ket{\phi_i}$ のひとつひとつが実現される等しい確からしさを持っている。
\end{axiom}
エネルギーの低い順にエネルギーシェル $E$ から $E + \Delta E$ までの中の状態を 1 つのグループでまとめてラベル付けする。
\begin{align}
  N = \sum_{l}N_l, \qquad E = \sum_{l}E_lN_l, \qquad W = \prod_{l}\frac{M_l^{N_l}}{N_l!}, \qquad S = k_B\sum_{l}N_l\ab(\log\frac{M_l}{N_l} + 1)
\end{align}

\begin{align}
  \tilde{S}              & = k_B\sum_{l}N_l\ab(\log\frac{M_l}{N_l} + 1) - k_B\alpha\sum_{l}N_l - k_B\beta\sum_{l}E_lN_l \\
  \diffp{\tilde{S}}{N_l} & = 0 \iff \frac{M_l}{N_l} = e^{\alpha + \beta E_l}
\end{align}

\begin{align}
  N & = \sum_{l}M_le^{-\alpha-\beta E_l}    \\
  E & = \sum_{l}M_lE_le^{-\alpha-\beta E_l} \\
  S & = k_B\ab((1 + \alpha)N + \beta E)
\end{align}
エネルギーで微分すると
\begin{align}
  0           & = \sum_{l}M_l\ab(\diff{\alpha}{E} + \diff{\beta}{E}E_l)e^{-\alpha-\beta E_l} = \diff{\alpha}{E}N + \diff{\beta}{E}E \\
  \diff{S}{E} & = k_B\ab(\diff{\alpha}{E}N + \diff{\beta}{E}E + \beta) = k_B\beta
\end{align}
より $\alpha, \beta$ は次のように表される。
\begin{align}
  \beta       & = \frac{1}{k_BT}                            \\
  e^{-\alpha} & = \frac{N}{\sum_{i}e^{-\varepsilon_i/k_BT}}
\end{align}


\subsection{熱と仕事}
\begin{definition}
  内部エネルギー $E(S, V)$ とその束縛変数を変更させたエンタルピー $H(S, p)$ と Helmholtz 自由エネルギー $F(T, V)$ と Gibbs 自由エネルギー $G(T, p)$ を次のように定義する。
  \begin{align}
               & \qquad \dl{E} = T\dl{S} - p\dl{V}  \\
    H = E + pV & \qquad \dl{H} = T\dl{S} + V\dl{p}  \\
    F = E - TS & \qquad \dl{F} = -S\dl{T} - p\dl{V} \\
    G = F + pV & \qquad \dl{G} = -S\dl{T} + V\dl{p}
  \end{align}
\end{definition}
特に扱いやすい変数 $T$, $V$ を持つ Helmholtz 自由エネルギー $F(T, V)$ は重宝される。
\begin{theorem}
  定義より次の関係式を満たす。
  \begin{alignat}{3}
    T  & = \ab(\diffp{E}{S})_V & \qquad -p & = \ab(\diffp{E}{V})_S \\
    T  & = \ab(\diffp{H}{S})_p & \qquad V  & = \ab(\diffp{H}{p})_S \\
    -S & = \ab(\diffp{F}{T})_V & \qquad -p & = \ab(\diffp{F}{V})_T \\
    -S & = \ab(\diffp{G}{T})_p & \qquad V  & = \ab(\diffp{G}{p})_T
  \end{alignat}
\end{theorem}

\begin{theorem}[Maxwell の関係式]
  $C^2$ 級の関数において偏微分は交換できるから次の関係式を満たす。
  \begin{align}
    \diffp{U}{S,V} & = \ab(\diffp{T}{V})_S = -\ab(\diffp{p}{S})_V  \\
    \diffp{H}{S,p} & = \ab(\diffp{T}{p})_S = \ab(\diffp{V}{S})_p   \\
    \diffp{F}{T,V} & = -\ab(\diffp{S}{V})_T = -\ab(\diffp{p}{T})_V \\
    \diffp{G}{T,p} & = -\ab(\diffp{S}{p})_T = \ab(\diffp{V}{T})_p
  \end{align}
\end{theorem}

\begin{theorem}[理想気体の状態方程式]
  \begin{align}
    pV & = Nk_BT
  \end{align}
\end{theorem}
\begin{align}
  S(E, V) & = Nk_B\ab(\frac{3}{2}\ln\frac{E}{V} + \frac{5}{2}\ln\frac{V}{N} + \frac{3}{2}\ln(\frac{me}{3\pi\hbar^2}) - \frac{1}{N}\ln(\sqrt{6}\pi N) + 1) \\
  0       & = Nk_B\ab(\frac{3}{2}\frac{1}{E}\ab(\diffp{E}{V})_S + \frac{1}{V})                                                                            \\
  p       & = -\ab(\diffp{E}{V})_S = \frac{2}{3}\frac{E}{V} = \frac{Nk_BT}{V}                                                                             \\
  pV      & = Nk_BT
\end{align}

\begin{align}
  C_V = \frac{3}{2}R
\end{align}

\begin{definition}[比熱]
  \begin{align}
    C   & = T\diff{S}{T}         \\
    C_X & = \ab(T\diffp{S}{T})_X
  \end{align}
\end{definition}
これ以降の話は熱力学の方で書きたい。

\section{カノニカル分布}
ある温度の環境の中で理想気体や
\subsection{ミクロカノニカル分布からカノニカル分布へ}

\begin{definition}
  $\beta = \dfrac{1}{k_BT}$
  \begin{align}
    Z(\beta) & = \sum_{i}e^{-\beta E_i}
  \end{align}
\end{definition}

\begin{theorem}
  \begin{align}
    p & = \frac{e^{-\beta E_i}}{Z(\beta)}                                                           \\
    F & = -k_BT\ln Z                                                                                \\
    S & = -\diffp{F}{T} = k_B\beta^2\diffp{F}{\beta}                                                \\
    U & = -T^2\diffp{}{T}\ab(\frac{F}{T}) = -\diffp{}{\beta}\ln Z(\beta) = \diffp{}{\beta}(\beta F) \\
    C & = -\beta\diffp{S}{\beta} = k_B\beta^2\diffp[2]{}{\beta}\ln Z(\beta) = F
  \end{align}
\end{theorem}

\begin{theorem}
  $N$ 個の独立な部分系からなる全体系の熱力学量は次のようになる。
  \begin{align}
    Z(\beta) = z(\beta)^N, \qquad F = Nf, \qquad S = Ns, \qquad U = Nu, \qquad C = c
  \end{align}
\end{theorem}

\subsection{二準位系}
絶対温度 $T$ の熱浴に系 $X$ が浸けられている状態として、系 $X$ の Hamilton 演算子 $\hat{h}_X$ の固有状態は $\ket{\varphi_1}$ と $\ket{\varphi_2}$ の 2 つだけであり、$\ket{\varphi_1}$ の固有エネルギーは $E_1$ であり、$\ket{\varphi_2}$ の固有エネルギーは $E_2$ であるとする:
\begin{align}
  \hat{h}_X\ket{\varphi_i} & = E_i\ket{\varphi_i} \qquad (i = 1, 2)。
\end{align}
ただし $0 < E_1 < E_2$ $\beta = 1/k_BT$ とする。

\begin{theorem}[1個の二準位系]
  二準位系における熱力学的量を考える。低温の漸近領域 ($\beta(E_2 - E_1) \gg 1$, $\beta E_1 \gg 1$) と高温の漸近領域 ($\beta(E_2 - E_1) \ll 1$, $\beta E_1 \ll 1$) は次のようになる。
  \begin{align}
    Z(\beta) & = e^{-\beta E_1} + e^{-\beta E_2}                                                                                                          \\
             & = \begin{dcases}
                   e^{-\frac{E_1}{k_BT}} \to 0      & (低温) \\
                   2 - \frac{E_1 + E_2}{k_BT} \to 2 & (高温)
                 \end{dcases}                                                                                                  \\
    F        & = -\frac{1}{\beta}\ln(e^{-\beta E_1} + e^{-\beta E_2})                                                                                     \\
             & = \begin{dcases}
                   E_1 - k_BTe^{-\frac{E_2 - E_1}{k_BT}} \to E_1  & (低温) \\
                   \frac{1}{2}(E_1 + E_2) - k_BT\ln 2 \to -\infty & (高温)
                 \end{dcases}                                                                                    \\
    S        & = k_B\ab(\ln(e^{-\beta E_1} + e^{-\beta E_2}) + \frac{\beta E_1e^{-\beta E_1} + \beta E_2e^{-\beta E_2}}{e^{-\beta E_1} + e^{-\beta E_2}}) \\
             & = \begin{dcases}
                   k_B \frac{E_2 - E_1}{k_BT}e^{- \frac{E_2 - E_1}{k_BT}} \to 0          & (低温) \\
                   k_B\ab(\ln 2 - \frac{1}{4}\ab(\frac{E_2 - E_1}{k_BT})^2) \to k_B\ln 2 & (高温)
                 \end{dcases}                                                     \\
    U        & = \frac{E_1e^{-\beta E_1} + E_2e^{-\beta E_2}}{e^{-\beta E_1} + e^{-\beta E_2}}                                                            \\
             & = \begin{dcases}
                   E_1 + (E_2 - E_1)e^{- \frac{E_2 - E_1}{k_BT}} \to E_1                                     & (低温) \\
                   \frac{1}{2}(E_1 + E_2) - \frac{1}{4}\frac{(E_2 - E_1)^2}{k_BT} \to \frac{1}{2}(E_1 + E_2) & (高温)
                 \end{dcases}                                        \\
    C        & = k_B\ab(\frac{\frac{1}{2}\beta(E_2 - E_1)}{\cosh\frac{1}{2}\beta(E_2 - E_1)})^2                                                           \\
             & = \begin{dcases}
                   k_B\ab(\frac{E_2 - E_1}{k_BT})^2e^{-\frac{E_2 - E_1}{k_BT}} \to 0 & (低温) \\
                   \frac{k_B}{4}\ab(\frac{E_2 - E_1}{k_BT})^2 \to 0                  & (高温)
                 \end{dcases}
  \end{align}
  TODO: グラフ
\end{theorem}
\begin{proof}
  $x \to 0$ において $(1 + x)^{-1} \approx 1 - x$, $e^x \approx 1 + x$ と近似できる。
  \begin{align}
    Z(\beta) & = \sum_{i} e^{-\beta E_i} = e^{-\beta E_1} + e^{-\beta E_2} \\
             & = e^{-\beta E_1}(1 + e^{-\beta (E_2 - E_1)})                \\
             & \approx \begin{dcases}
                         e^{-\frac{E_1}{k_BT}}                 \\
                         (1 - \beta E_1)(2 -\beta (E_2 - E_1)) \\
                       \end{dcases}               \\
             & = \begin{dcases}
                   e^{-\frac{E_1}{k_BT}} \to 0      & (低温) \\
                   2 - \frac{E_1 + E_2}{k_BT} \to 2 & (高温)
                 \end{dcases}
  \end{align}
  \begin{align}
    F & = -k_BT\ln Z(\beta)                                                                                                                                                                \\
      & = -\frac{1}{\beta}\ln(e^{-\beta E_1} + e^{-\beta E_2})                                                                                                                             \\
      & = \begin{dcases}
            -\frac{1}{\beta}\ln e^{-\beta E_1}(1 + e^{-\beta (E_2 - E_1)}) \\
            -\frac{1}{\beta}\ln e^{-\frac{1}{2}\beta (E_1 + E_2)}(e^{\frac{1}{2}\beta (E_2 - E_1)} + e^{-\frac{1}{2}\beta (E_2 - E_1)})
          \end{dcases} \\
      & \approx \begin{dcases}
                  E_1 - \frac{1}{\beta}e^{-\beta (E_2 - E_1)} \approx E_1 - k_BTe^{-\frac{E_2 - E_1}{k_BT}} \to E_1 \\
                  \frac{1}{2}(E_1 + E_2) - k_BT\ln 2 \to -\infty                                                    \\
                \end{dcases}
  \end{align}
  \begin{align}
    S & = - \ab(\diffp{F}{T})_{V,N} = - \ab(\diffp{F}{\beta}\diffp{\beta}{T})_{V,N} = k_B\beta^2\ab(\diffp{F}{\beta})_{V,N}                                                                                                                                                                                                                                                                   \\
      & = k_B\beta^2\ab(\frac{1}{\beta^2}\ln(e^{-\beta E_1} + e^{-\beta E_2}) - \frac{1}{\beta}\frac{-E_1e^{-\beta E_1} - E_2e^{-\beta E_2}}{e^{-\beta E_1} + e^{-\beta E_2}})                                                                                                                                                                                                                \\
      & = k_B\ab(\ln(e^{-\beta E_1} + e^{-\beta E_2}) + \frac{\beta E_1e^{-\beta E_1} + \beta E_2e^{-\beta E_2}}{e^{-\beta E_1} + e^{-\beta E_2}})                                                                                                                                                                                                                                            \\
      & = \begin{dcases}
            k_B\ab(\ln e^{-\beta E_1}(1 + e^{-\beta (E_2 - E_1)}) + \frac{\beta E_1 + \beta E_2e^{-\beta (E_2 - E_1)}}{1 + e^{-\beta (E_2 - E_1)}})                                                                                                                                                   \\
            k_B\ab(\ln e^{-\frac{1}{2}\beta (E_1 + E_2)}(e^{\frac{1}{2}\beta (E_2 - E_1)} + e^{-\frac{1}{2}\beta (E_2 - E_1)}) + \frac{\beta E_1e^{\frac{1}{2}\beta (E_2 - E_1)} + \beta E_2e^{-\frac{1}{2}\beta (E_2 - E_1)}}{e^{\frac{1}{2}\beta (E_2 - E_1)} + e^{-\frac{1}{2}\beta (E_2 - E_1)}}) \\
          \end{dcases} \\
      & \approx
    \begin{dcases}
      k_B\ab(-\beta E_1 + e^{-\beta (E_2 - E_1)} + (\beta E_1 + \beta E_2e^{-\beta (E_2 - E_1)})(1 - e^{-\beta (E_2 - E_1)}))                                \\
      k_B\ab(\ln 2 - \frac{1}{2}\beta (E_1 + E_2) + \frac{\beta}{2}\ab(E_1\ab(1 + \frac{1}{2}\beta (E_2 - E_1)) + E_2\ab(1 - \frac{1}{2}\beta (E_2 - E_1)))) \\
    \end{dcases}                                                                                                                                                                                                                        \\
      & \approx
    \begin{dcases}
      k_B \frac{E_2 - E_1}{k_BT}e^{- \frac{E_2 - E_1}{k_BT}} \to 0          \\
      k_B\ab(\ln 2 - \frac{1}{4}\ab(\frac{E_2 - E_1}{k_BT})^2) \to k_B\ln 2 \\
    \end{dcases}
  \end{align}
  \begin{align}
    U & = F + TS                                                                                                                                                                                      \\
      & = \frac{E_1e^{-\beta E_1} + E_2e^{-\beta E_2}}{e^{-\beta E_1} + e^{-\beta E_2}}                                                                                                               \\
      & = \begin{dcases}
            \frac{E_1 + E_2e^{-\beta (E_2 - E_1)}}{1 + e^{-\beta (E_2 - E_1)}}                                                                                      \\
            \frac{E_1e^{\frac{1}{2}\beta (E_2 - E_1)} + E_2e^{-\frac{1}{2}\beta (E_2 - E_1)}}{e^{\frac{1}{2}\beta (E_2 - E_1)} + e^{-\frac{1}{2}\beta (E_2 - E_1)}} \\
          \end{dcases} \\
      & \approx \begin{dcases}
                  (E_1 + E_2e^{-\beta (E_2 - E_1)})(1 - e^{-\beta (E_2 - E_1)})                                       \\
                  \frac{1}{2}\ab(E_1\ab(1 + \frac{1}{2}\beta (E_2 - E_1)) + E_2\ab(1 - \frac{1}{2}\beta (E_2 - E_1))) \\
                \end{dcases}                                                                             \\
      & \approx \begin{dcases}
                  E_1 + (E_2 - E_1)e^{- \frac{E_2 - E_1}{k_BT}} \to E_1                                     \\
                  \frac{1}{2}(E_1 + E_2) - \frac{1}{4}\frac{(E_2 - E_1)^2}{k_BT} \to \frac{1}{2}(E_1 + E_2) \\
                \end{dcases}
  \end{align}
  \begin{align}
    C & = \diffp{U}{T} = \diffp{U}{\beta}\diffp{\beta}{T} = -k_B\beta^2\diffp{U}{\beta}                                                                                                 \\
      & = -k_B\beta^2\diffp{\beta}\ab(\frac{E_1 + E_2e^{\beta(E_1 - E_2)}}{1 + e^{\beta(E_1 - E_2)}})                                                                                   \\
      & = -k_B\beta^2\frac{E_2(E_1 - E_2)e^{\beta(E_1 - E_2)}(1 + e^{\beta(E_1 - E_2)}) - (E_1 + E_2e^{\beta(E_1 - E_2)})(E_1 - E_2)e^{\beta(E_1 - E_2)}}{(1 + e^{\beta(E_1 - E_2)})^2} \\
      & = k_B\beta^2\frac{(E_2 - E_1)^2e^{\beta(E_1 - E_2)}}{(1 + e^{\beta(E_1 - E_2)})^2} = k_B\ab(\frac{\frac{1}{2}\beta(E_2 - E_1)}{\cosh\frac{1}{2}\beta(E_2 - E_1)})^2             \\
      & = \begin{dcases}
            k_B\ab(\frac{\beta(E_2 - E_1)}{1 + e^{-\beta(E_2 - E_1)}})^2e^{-\beta(E_2 - E_1)} \\
            k_B\ab(\frac{\beta(E_2 - E_1)}{e^{\frac{1}{2}\beta(E_2 - E_1)} + e^{-\frac{1}{2}\beta(E_2 - E_1)}})^2
          \end{dcases}                                                          \\
      & \approx \begin{dcases}
                  k_B\ab(\frac{E_2 - E_1}{k_BT})^2e^{-\frac{E_2 - E_1}{k_BT}} \to 0 \\
                  \frac{k_B}{4}\ab(\frac{E_2 - E_1}{k_BT})^2 \to 0
                \end{dcases}
  \end{align}
  各固有状態の実現確率について高温極限 ($\beta(E_2 - E_1) \ll 1$) のときそれぞれの固有状態は同じ確率で実現し、低温極限 ($\beta(E_2 - E_1) \gg 1$) のとき固有エネルギーの低い固有状態にほぼ確実に実現する。
  \begin{align}
    \quad p_\beta(i) & = \frac{e^{-\beta (E_i - E_1)}}{1 + e^{-\beta(E_2 - E_1)}} \approx \begin{dcases}
                                                                                            e^{-\beta (E_i - E_1)} & (\beta(E_2 - E_1) \gg 1) \\
                                                                                            \frac{1}{2}            & (\beta(E_2 - E_1) \ll 1)
                                                                                          \end{dcases}
  \end{align}
  $F = E - TS$ の最小化を考える。低温極限でエントロピーを上げるよりエネルギーが低いものを選んだ方がエネルギーが得となる為に固有エネルギーの低い状態に集まる。高温極限でエントロピーを増大させるとエネルギーが得となる為に半々となる。
\end{proof}

\begin{itembox}[l]{Q 15-2.}
  Q 15-1.では Helmholtz 自由エネルギーを計算して、後は熱力学の公式を用いて計算しましたが、今回は正準集団の理論における固有状態の実現確率を与える確率関数 $p_\beta^{正準}(i)\ (i = 1, 2)$ を計算して、内部エネルギー $u$ とエントロピー $s$ を求める。
\end{itembox}
まず確率関数 $p_\beta(i)$ は定義より次のようになる。
\begin{align}
  p_\beta(i) & = \frac{e^{-\beta E_i}}{z(\beta)}.
\end{align}
内部エネルギー $u$ はエネルギーの平均を取ることで分かる。
\begin{align}
  u & = \sum_i E_ip_\beta^{正準}(i) = \frac{E_1e^{-\beta E_1} + E_2e^{-\beta E_2}}{e^{-\beta E_1} + e^{-\beta E_2}}.
\end{align}
比熱も Q15-1.と同様に求まる。

エントロピー $s$ は Shannon のエントロピーの公式に代入することで求まる。
\begin{align}
  s & = -k_B\sum_{i = 1,2}p_\beta^{正準}(i)\ln p_\beta^{正準}(i)                                                                                      \\
    & = -k_B\sum_{i = 1,2}\frac{e^{-\beta E_i}}{z(\beta)}(- \ln z(\beta) - \beta E_i)                                                             \\
    & = k_B\ab(\ln(e^{-\beta E_1} + e^{-\beta E_2}) + \frac{\beta E_1e^{-\beta E_1} + \beta E_2e^{-\beta E_2}}{e^{-\beta E_1} + e^{-\beta E_2}}).
\end{align}

\begin{itembox}[l]{Q 15-7.}
  比熱について解析せよ。
\end{itembox}

まず比熱について次のように定義した関数 $\phi(x)$ を用いて表される。
\begin{align}
  \phi(x) & := \frac{x}{\cosh x}                                                             \\
  c       & = k_B\ab(\frac{\frac{1}{2}\beta(E_2 - E_1)}{\cosh\frac{1}{2}\beta(E_2 - E_1)})^2 \\
          & = k_B\ab(\phi\ab(\frac{1}{2}\beta(E_2 - E_1)))^2
\end{align}
ここで $x\geq 0$ の範囲において $\phi(x)$ が極大となる $x = x_0$ の値を考える。
\begin{align}
       & \left.\diff{\phi}{x}\right|_{x = x_0} = 0        \\
  \iff & \frac{\cosh x_0 - x_0\sinh x_0}{\cosh^2 x_0} = 0 \\
  \iff & x_0\tanh x_0 = 1                                 \\
  \iff & x_0 = 1.199678640257734\ldots
\end{align}

ただしプログラム \ref{newton} を用いて $x\geq 0$ の範囲で $x_0\tanh x_0 = 1$ は $x_0 = 1.199678640257734\ldots$ のとき満たすことが分かる。これより比熱 $c$ は次のように定義される $T_0$ のときに極大を取る。
\begin{align}
   & x_0 = \frac{1}{2}\beta_0(E_2 - E_1) = \frac{1}{2}\frac{E_2 - E_1}{k_BT_0} \\
   & \frac{k_BT_0}{E_2 - E_1} = \frac{1}{2x_0} =  0.41677827980048\ldots
\end{align}

低温、高温で比熱が 0 となる理由は比熱が $C = \diff{E}{T}$ であることより Q15-3, Q15-4 よりエネルギーの確率が極限的に定数となることから比熱は 0 となることが分かる。

\subsection{調和振動子系の統計力学}

\begin{theorem}
  \begin{align}
    z(\beta) & = \frac{1}{2\sinh \frac{1}{2}\beta\hbar\omega}                                                                     \\
    f        & = \frac{1}{\beta}\ln\ab(2\sinh \frac{1}{2}\beta\hbar\omega)                                                        \\
    s        & = k_B\ab(-\ln\ab(2\sinh\frac{1}{2}\beta\hbar\omega) + \frac{1}{2}\beta\hbar\omega\coth\frac{1}{2}\beta\hbar\omega) \\
    u        & = \frac{1}{2}\hbar\omega\coth\frac{1}{2}\beta\hbar\omega                                                           \\
    c        & = k_B\ab(\frac{\frac{1}{2}\beta\hbar\omega}{\sinh\frac{1}{2}\beta\hbar\omega})^2
  \end{align}
\end{theorem}
\begin{proof}
  低温の漸近領域において Q 16-1 の結果は次のように近似できる。ただし、$x\to 0$ のとき $e^x \approx 1 + x$, $(1 + x)^{-1} \approx 1 - x$ と近似できることを用いる。
  高温の漸近領域において Q 16-1 の結果は次のように近似できる。ただし、$x\to 0$ のとき $\ln(1 + x) \approx x$ と近似できることとテイラー展開を用いる。
  \begin{align}
    z(\beta) & = \sum_{i = 0}^{\infty}e^{-\beta E_i} = \sum_{i = 0}^{\infty}e^{-\beta\ab(n + \frac{1}{2})\hbar\omega} = \frac{e^{-\frac{1}{2}\beta\hbar\omega}}{1 - e^{-\beta\hbar\omega}} = \frac{1}{2\sinh \frac{1}{2}\beta\hbar\omega} \\
             & = \begin{dcases}
                   \frac{e^{-\frac{1}{2}\beta\hbar\omega}}{1 - e^{-\beta\hbar\omega}} \\
                   \frac{e^{-\frac{1}{2}\beta\hbar\omega}}{1 - e^{-\beta\hbar\omega}} \\
                 \end{dcases}                                                                                                                                                \\
             & \approx \begin{dcases}
                         e^{-\frac{\hbar\omega}{2k_BT}} \to 0                                                                                \\
                         \frac{1 - \frac{1}{2}\beta\hbar\omega}{\beta\hbar\omega} \approx \frac{k_BT}{\hbar\omega} - \frac{1}{2} \to +\infty \\
                       \end{dcases}
  \end{align}
  \begin{align}
    f & = -k_BT\ln z(\beta)                                                                                                                                               \\
      & = -\frac{1}{\beta}\ln \frac{e^{-\frac{1}{2}\beta\hbar\omega}}{1 - e^{-\beta\hbar\omega}} = \frac{1}{\beta}\ln(1 - e^{-\beta\hbar\omega}) + \frac{1}{2}\hbar\omega \\
      & = -\frac{1}{\beta}\ln \frac{1}{2\sinh \frac{1}{2}\beta\hbar\omega} = \frac{1}{\beta}\ln(2\sinh \frac{1}{2}\beta\hbar\omega)                                       \\
      & \approx \begin{dcases}
                  \frac{1}{\beta}(-e^{-\beta\hbar\omega}) + \frac{1}{2}\hbar\omega       \\
                  \frac{1}{\beta}\ab(\ln \beta\hbar\omega + \frac{1}{2}\beta\hbar\omega) \\
                \end{dcases}                                                                                    \\
      & \approx \begin{dcases}
                  \frac{1}{2}\hbar\omega - k_BTe^{-\frac{\hbar\omega}{k_BT}} \to \frac{1}{2}\hbar\omega \\
                  -k_BT\ln\frac{k_BT}{\hbar\omega} \to - \infty                                         \\
                \end{dcases}
  \end{align}
  \begin{align}
    s & = - \ab(\diffp{f}{T})_{V,N} = k_B\beta^2\ab(\diffp{f}{\beta})_{V,N}                                                                                                                                              \\
      & = k_B\beta^2\ab(-\frac{1}{\beta^2}\ln(1 - e^{-\beta\hbar\omega}) + \frac{\hbar\omega}{\beta(1 - e^{-\beta\hbar\omega})})                                                                                         \\
      & = k_B\beta^2\ab(-\frac{1}{\beta^2}\ln\ab(2\sinh \frac{1}{2}\beta\hbar\omega) + \frac{\frac{1}{2}\hbar\omega\cosh \frac{1}{2}\beta\hbar\omega}{\beta\sinh \frac{1}{2}\beta\hbar\omega})                           \\
      & = k_B\ab(-\ln(1 - e^{-\beta\hbar\omega}) + \frac{\beta\hbar\omega}{e^{\beta\hbar\omega} - 1}) = k_B\ab(-\ln\ab(2\sinh\frac{1}{2}\beta\hbar\omega) + \frac{1}{2}\beta\hbar\omega\coth\frac{1}{2}\beta\hbar\omega) \\
      & \approx \begin{dcases}
                  k_B\ab(e^{-\beta\hbar\omega} + \beta\hbar\omega e^{-\beta\hbar\omega}(1 + e^{-\beta\hbar\omega})) \\
                  k_B\ab(-\ln\beta\hbar\omega + 1)                                                                  \\
                \end{dcases}                                                                                                        \\
      & \approx \begin{dcases}
                  k_B\frac{\hbar\omega}{k_BT}e^{- \frac{\hbar\omega}{k_BT}} \to 0 \\
                  k_B\ln\frac{k_BT}{\hbar\omega} \to +\infty                      \\
                \end{dcases}
  \end{align}
  \begin{align}
    u & = f + Ts                                                                                                                                                                                   \\
      & = \frac{1}{\beta}\ln(1 - e^{-\beta\hbar\omega}) + \frac{1}{2}\hbar\omega + \frac{1}{\beta}\ab(-\ln(1 - e^{-\beta\hbar\omega}) + \frac{\beta\hbar\omega}{e^{\beta\hbar\omega} - 1})         \\
      & = \frac{1}{\beta}\ln\ab(2\sinh \frac{1}{2}\beta\hbar\omega) + \frac{1}{\beta}\ab(-\ln\ab(2\sinh\frac{1}{2}\beta\hbar\omega) + \frac{1}{2}\beta\hbar\omega\coth\frac{1}{2}\beta\hbar\omega) \\
      & = \ab(\frac{1}{2} + \frac{1}{e^{\beta\hbar\omega} - 1})\hbar\omega = \frac{1}{2}\hbar\omega\coth\frac{1}{2}\beta\hbar\omega                                                                \\
      & \approx \begin{dcases}
                  \ab(\frac{1}{2} + e^{-\beta\hbar\omega}(1 + e^{-\beta\hbar\omega}))\hbar\omega                                        \\
                  \frac{1}{2}\hbar\omega\ab(\ab(\frac{\beta\hbar\omega}{2})^{-1} + \frac{1}{3}\ab(\frac{\beta\hbar\omega}{2}) + \cdots) \\
                \end{dcases}                                                              \\
      & \approx \begin{dcases}
                  \frac{1}{2}\hbar\omega + e^{-\beta\hbar\omega}\hbar\omega \to \frac{1}{2}\hbar\omega \\
                  k_BT\ab(1 + \frac{1}{12}\ab(\frac{\hbar\omega}{k_BT})^{2} + \cdots) \to +\infty      \\
                \end{dcases}
  \end{align}
  \begin{align}
    c & = \diffp{u}{T} = -k_B\beta^2\diffp{u}{\beta} = -k_B\beta^2\ab(-\frac{\hbar\omega e^{\beta\hbar\omega}}{(e^{\beta\hbar\omega} - 1)^2})\hbar\omega                              \\
      & = k_B\ab(\beta\hbar\omega\frac{e^{\frac{1}{2}\beta\hbar\omega}}{e^{\beta\hbar\omega} - 1})^2 = k_B\ab(\frac{\frac{1}{2}\beta\hbar\omega}{\sinh\frac{1}{2}\beta\hbar\omega})^2 \\
      & \approx \begin{dcases}
                  k_B\ab(\beta\hbar\omega(e^{-\frac{1}{2}\beta\hbar\omega})(1 + e^{-\beta\hbar\omega}))^2                                             \\
                  k_B\ab(\frac{\beta\hbar\omega}{2}\ab(\ab(\frac{\beta\hbar\omega}{2})^{-1} - \frac{1}{6}\ab(\frac{\beta\hbar\omega}{2}) + \cdots))^2 \\
                \end{dcases}                                                                               \\
      & \approx \begin{dcases}
                  k_B\ab(\frac{\hbar\omega}{k_BT})^2e^{-\frac{\hbar\omega}{k_BT}} \to 0 \\
                  k_B\ab(1 - \frac{1}{24}\ab(\frac{\hbar\omega}{k_BT}) + \cdots)^2 = k_B\ab(1 - \frac{1}{12}\ab(\frac{\hbar\omega}{k_BT}) + \cdots) \to k_B
                \end{dcases}
  \end{align}
\end{proof}


\subsection{固体の比熱の Einstein 模型}
ある元素の原子 $n$ [\si{mol}] からなる個体を考える。Einstein 模型では、結晶を構成するそれぞれの原子は平衡位置の回りに独立に同一の角振動数 $\omega_E$ を持って調和振動すると考える。
\begin{align}
  \hat{H} & = \sum_{j=1}^{N}\ab(\frac{\hat{p}_j^2}{2m} + \frac{1}{2}m\omega_E^2\hat{x}_j^2)
\end{align}

独立な調和振動子の集まりの系として記述される系 $X$ において角振動数が $\omega$ から $\omega + \dl{\omega}$ の範囲にある調和振動子の個数を $g(\omega)\dl{\omega}$ と定義する。つまり $g(\omega)$ は調和振動子の角振動数に対する個数分布関数である。

調和振動子の角振動数の個数について、各原子の自由度が $3$ であるから Avogadro 数 $N_A = 6.02\ldots\times 10^{23}$ [\si{1/mol}] を用いて全体の個数は $3N = 3nN_A$ であることが分かる。
これより Einstein 模型における調和振動子の角振動数の個数分布関数 $g(\omega)$ は次のように表される。
\begin{align}
  g(\omega) = 3N\delta(\omega - \omega_E)
\end{align}
このとき角運動量が $\omega$ である調和振動子 1 個の Helmholtz 自由エネルギー, エントロピー, 内部エネルギー, 比熱をそれぞれ $f(\omega), s(\omega), u(\omega), c(\omega)$ と書くこととすると Einstein 模型は次のように書ける
\begin{align}
  F & = \int_0^\infty\dl{\omega}g(\omega)f(\omega) = 3N\frac{1}{\beta}\ln\ab(2\sinh \frac{1}{2}\beta\hbar\omega_E)                                                            \\
  S & = \int_0^\infty\dl{\omega}g(\omega)s(\omega) = 3Nk_B\ab(-\ln\ab(2\sinh\frac{1}{2}\beta\hbar\omega_E) + \frac{1}{2}\beta\hbar\omega_E\coth\frac{1}{2}\beta\hbar\omega_E) \\
  U & = \int_0^\infty\dl{\omega}g(\omega)u(\omega) = 3N\frac{1}{2}\hbar\omega_E\coth\frac{1}{2}\beta\hbar\omega_E                                                             \\
  C & = \int_0^\infty\dl{\omega}g(\omega)c(\omega) = 3Nk_B\ab(\frac{\frac{1}{2}\beta\hbar\omega_E}{\sinh\frac{1}{2}\beta\hbar\omega_E})^2
\end{align}

\begin{itembox}[l]{実験事実}
  \begin{enumerate}
    \item (高温での固体の比熱の振る舞い : Dulong-Petit の法則) 十分に高温では、$n$ [\si{mol}] の固体の比熱 $C$ は、固体を構成する物質によらずに、$3nR$ の一定値を取る。ここで、$R = 8.314\ldots$ [\si{J/(mol\cdot K)}] は気体定数である。
    \item (低温での固体の比熱の大雑把な振る舞い) 温度 $T$ が $0$ に近付くとき、固体の比熱 $C$ は小さくなっていく。温度 $T$ が $0$ に近付く極限では、比熱 $C$ はゼロになるようだ。
    \item (低温での固体の比熱の精密な振る舞い) 温度 $T$ が $0$ に近付くとき、固体の比熱 $C$ は $C \propto T^3$ であり、$\lim_{T\to 0} C = 0$ となる。
  \end{enumerate}
\end{itembox}
高温の漸近領域において比熱 $C$ は次のようになる。
\begin{align}
  C & = 3Nk_B\ab(\frac{\frac{1}{2}\beta\hbar\omega_E}{\sinh\frac{1}{2}\beta\hbar\omega_E})^2 \\
    & \approx 3Nk_B\ab(1 - \frac{1}{12}\ab(\frac{\hbar\omega}{k_BT})^2 + \cdots)             \\
    & \to 3nR
\end{align}
低温の漸近領域において比熱 $C$ は次のようになる。
\begin{align}
  C & = 3Nk_B\ab(\frac{\frac{1}{2}\beta\hbar\omega_E}{\sinh\frac{1}{2}\beta\hbar\omega_E})^2 \\
    & \approx 3Nk_B\ab(\frac{\hbar\omega}{k_BT})^2e^{-\frac{\hbar\omega}{k_BT}}              \\
    & \approx 3nR\ab(\frac{\hbar\omega}{k_BT})^2e^{-\frac{\hbar\omega}{k_BT}}                \\
    & \propto \frac{1}{T^{2}e^{\frac{1}{T}}} \to 0
\end{align}
これより $C \propto T^3$ とはならない為、固体の比熱の Einstein 模型は実験事実と合致しない。

\subsection{固体の比熱の Debye 模型}
ここでは固体の比熱 $C$ の Debye 模型を学ぶ. Debye 模型は高温における $C\approx 3nR$ と低温における $C\propto T^3$ の両方を正しく説明する.

3 次元結晶を $N^3$ 個の原子があり、固体を構成する各原子は隣り合った原子間力によるバネ定数 $\kappa$ のバネにより結びついているとする。
\begin{align}
  \hat{H} & = \frac{1}{2m}\sum_{i=1}^{N}p_i^2 + \frac{1}{2}\kappa\sum_{i=0}^{N}(q_i - q_{i+1})^2
\end{align}
このとき Fourier 展開して解析力学の結果により $3N^3$ 個の調和振動子の系と同等となる。
\begin{align}
  \hat{H}      & = \sum_{j=1}^{N}\ab(\frac{1}{2m}P_j^2 + \frac{1}{2}m\omega_j^2Q_j^2)                                   \\
  \omega_{\kk} & = 2\sqrt{\frac{\kappa}{m}}\sqrt{\sin^2\frac{1}{2}k_xa + \sin^2\frac{1}{2}k_ya + \sin^2\frac{1}{2}k_za} \\
  \kk          & = \frac{\pi}{a(N+1)}(j_x, j_y, j_z)
\end{align}
ただし 1 辺の長さ $L$ の立方体の固定端境界条件を持つとする。
フォノンの各振動数がデカすぎると奇妙な波となるので制約を設ける。
\begin{align}
  \sqrt{\frac{m}{\kappa}}\frac{N+1}{\pi}\omega \leq N
\end{align}

分散関係がこのままだと解析的に解けない。この困難を打開する為に分散関係を修正して解析計算ができる要請と高温極限と低温極限において Einstein 模型と同じ結果を導くという要請をした。これを Debye 模型という。
\begin{enumerate}
  \item 十分に高温において前節の模型と新しい模型が同じ比熱の極限値を持つには独立な調和振動子の総数について一致することが必要十分である.
        十分に高温ではエントロピーが高くなる為, すべての独立な調和振動子のエネルギー状態について実現確率は等分配される. このとき比熱は独立な調和振動子の総数のみに依存するから前節の模型と等しい総数となることが必要十分である.
  \item 十分に低温において前節の模型と新しい模型が同じ比熱の漸近的な振る舞いを示すためには分散関係の関数 $\omega(\bm{k})$ が長波長の漸近領域 $a|\bm{k}| \ll 1$ において一致することが十分である.
        十分に低温ではエントロピーが低くなり, エネルギーが低い状態, つまり長波長に関する状態に実現確率が集まるので, 前節の模型と新しい模型について長波長の漸近領域において分散関係が一致するなら同じ比熱の漸近的な振る舞いとなることが言える.
\end{enumerate}
\begin{align}
  \omega_{\kk} & \approx \sqrt{\frac{\kappa}{m}}a|\bm{k}| + \mathcal{O}(|\bm{k}|^3) \qquad (a|\bm{k}| \ll 1)               \\
  \omega_{\kk} & = \sqrt{\frac{\kappa}{m}}a|\bm{k}| = \sqrt{\frac{\kappa}{m}}\frac{\pi}{N + 1}\sqrt{j_x^2 + j_y^2 + j_z^2}
\end{align}
また打ち切る角振動数 $\omega_D$ を次のように定める.
\begin{align}
  \int_0^\infty\dl{\omega}g(\omega) & = \int_0^{\omega_D}\dl{\omega}g(\omega) = 3N^3.
\end{align}
この $\omega_D$ を Debye の角振動数という.

\begin{theorem}
  独立な調和振動子の角振動数に関する個数分布関数 $g(\omega)$ について幾何学的解釈で近似できる.
  \begin{align}
    g(\omega) & = \begin{dcases}
                    \frac{9N^3}{\omega_D}\ab(\frac{\omega}{\omega_D})^2 & (\omega\leq\omega_D) \\
                    0                                                   & (\omega > \omega_D)
                  \end{dcases}.
  \end{align}
\end{theorem}
\begin{proof}
  \begin{align}
    g(\omega) & \approx 3\sqrt{\frac{m}{\kappa}}\frac{N+1}{\pi}\times\ab(半径 \sqrt{\frac{m}{\kappa}}\frac{N+1}{\pi}\omega の 2 次元球面 S_2 を第 1 象限で切り取った曲面の表面積) \\
              & = 3\sqrt{\frac{m}{\kappa}}\frac{N+1}{\pi}\times\frac{4\pi}{8}\ab(\sqrt{\frac{m}{\kappa}}\frac{N+1}{\pi}\omega)^2                           \\
              & = \frac{3\pi}{2}\ab(\sqrt{\frac{m}{\kappa}}\frac{N+1}{\pi})^3\omega^2.
  \end{align}
  正方形から球へ近似
  ただし $N\gg 1$ であることから $N+1$ を $N$ と近似し,
  \begin{align}
    \int_0^{\infty}\dl{\omega}g(\omega) & = \int_0^{\omega_D}\dl{\omega}\frac{3\pi}{2}\ab(\sqrt{\frac{m}{\kappa}}\frac{N}{\pi})^3\omega^2 = \frac{\pi}{2}\ab(\sqrt{\frac{m}{\kappa}}\frac{N}{\pi})^3\omega_D^3 = 3N^3 \\
    \omega_D                            & = (6\pi^2)^{1/3}\sqrt{\frac{\kappa}{m}}
  \end{align}
  \begin{align}
    g(\omega) & = \begin{dcases}
                    \frac{3\pi}{2}\ab(\sqrt{\frac{m}{\kappa}}\frac{N}{\pi})^3\omega^2 & (\omega\leq\omega_D) \\
                    0                                                                 & (\omega > \omega_D)
                  \end{dcases} \\
              & = \begin{dcases}
                    \frac{9N^3}{\omega_D}\ab(\frac{\omega}{\omega_D})^2 & (\omega\leq\omega_D) \\
                    0                                                   & (\omega > \omega_D)
                  \end{dcases}.
  \end{align}
\end{proof}
現実の物質に Debye 模型を当てはめるときには, それぞれの物質は固有の Debye 角振動数 $\omega_D$ を持つことになる.

\begin{theorem}
  Debye 模型における内部エネルギーの表式は次のようになる.
  \begin{align}
    U & = U_0 + 9N^3\hbar\omega_DI(\beta\hbar\omega_D) \qquad \ab(U_0 = \frac{3}{8}(3N^3)\hbar\omega_D) \\
    C & = 3nR\cdot(-3)b^2\diff{I(b)}{b}
  \end{align}
  ただし $I(b)$ について次のように定められる.
  \begin{align}
    I(b) & = \int_0^1\dl{x}\frac{x^3}{e^{bx} - 1}.
  \end{align}
\end{theorem}
\begin{proof}
  以下からは $b = \beta\hbar\omega_D = \hbar\omega_D/(k_BT)$ という関係を用いる。
  \begin{align}
    U & = \int_0^\infty\dl{\omega}g(\omega)u(\omega)                                                                                                      \\
      & = \int_0^{\omega_D}\dl{\omega}\frac{9N^3}{\omega_D}\ab(\frac{\omega}{\omega_D})^2\ab(\frac{1}{2} + \frac{1}{e^{\beta\hbar\omega} - 1})\hbar\omega \\
      & = 9N^3\hbar\int_0^{\omega_D}\dl{\omega}\ab(\frac{\omega}{\omega_D})^3\ab(\frac{1}{2} + \frac{1}{e^{\beta\hbar\omega} - 1})                        \\
      & = 9N^3\hbar\omega_D\int_0^1\dl{x}\ab(\frac{1}{2} + \frac{1}{e^{\beta\hbar\omega_Dx} - 1})x^3                                                      \\
      & = \frac{3}{8}(3N^3)\hbar\omega_D + 9N^3\hbar\omega_DI(\beta\hbar\omega_D)
  \end{align}
  比熱の定義式に代入することで次のようになる。
  \begin{align}
    C & = \int_0^\infty\dl{\omega}g(\omega)c(\omega)                                                                                                                         \\
      & = \int_0^{\omega_D}\dl{\omega}\frac{9N^3}{\omega_D}\ab(\frac{\omega}{\omega_D})^2k_B\ab(\frac{\beta\hbar\omega e^{\beta\hbar\omega/2}}{e^{\beta\hbar\omega} - 1})^2  \\
      & = 9k_BN^3(\beta\hbar\omega_D)^2\int_0^{\omega_D}\frac{\dl{\omega}}{\omega_D}\ab(\frac{\omega}{\omega_D})^4\frac{ e^{\beta\hbar\omega}}{(e^{\beta\hbar\omega} - 1)^2} \\
      & = 3nR\cdot 3b^2\int_0^1\dl{x}\frac{x^4e^{bx}}{(e^{bx} - 1)^2}                                                                                                        \\
      & = 3nR\cdot (-3)b^2\diff{I(b)}{b}.
  \end{align}
\end{proof}

\begin{theorem}
  高温と低温の漸近領域における積分 $I(b)$ は次のように評価できる.
  \begin{align}
    I(b) =
    \begin{dcases}
      \frac{1}{3b} - \frac{1}{8} + \frac{1}{60}b - \frac{1}{5040}b^3 + \frac{1}{272160}b^5 - \cdots & (b\ll 1) \\
      \frac{\pi^4}{15}\frac{1}{b^4} + \mathcal{O}(b^{-1}e^{-b})                                     & (b\gg 1)
    \end{dcases}
  \end{align}
\end{theorem}
\begin{proof}
  Bernoulli 数 $B_n$ の定義を用いて次のように計算できる。
  \begin{align}
    I(b) & = \int_0^1\dl{x}\frac{x^3}{e^{bx} - 1}                                                           \\
         & = \int_0^1\dl{x}\sum_{n=0}^{\infty}\frac{B_n b^{n-1}}{n!}x^{n+2}                                 \\
         & = \sum_{n=0}^{\infty}\frac{B_n}{(n + 3)n!}b^{n-1}                                                \\
         & = \frac{1}{3b} - \frac{1}{8} + \frac{1}{60}b - \frac{1}{5040}b^3 + \frac{1}{272160}b^5 - \cdots.
  \end{align}

  初項 $e^{-bx}$ 公比 $e^{-bx}$ の無限等比数列の和は $1/(e^{bx} + 1)$ である. これより $I(b)$ は次のように表される.
  \begin{align}
    I(b) & = \int_0^1\dl{x}\frac{x^3}{e^{bx} - 1} = \int_0^1\dl{x}x^3\sum_{n=1}^{\infty}e^{-nbx} = \sum_{n=1}^{\infty}\int_0^1\dl{x}x^3e^{-nbx} \\
         & = \sum_{n=1}^{\infty}\frac{1}{(nb)^4}\int_0^{nb}\dl{t}t^3e^{-t} \qquad (t = nbx)                                                     \\
         & = \sum_{n=1}^{\infty}\frac{1}{(nb)^4}\gamma(4, nb)                                                                                   \\
         & = \sum_{n=1}^{\infty}\frac{1}{(nb)^4}(\Gamma(4) - \Gamma(4, nb))                                                                     \\
         & = \frac{1}{b^4}\ab(\Gamma(4)\zeta(4) - \sum_{n=1}^{\infty}\frac{1}{n^4}\Gamma(4, nb))
  \end{align}
  ただし, 第一種不完全ガンマ関数 $\gamma(z, p)$ は次の式で定義される.第 2 種不完全ガンマ関数 $\Gamma(z, p)$, ガンマ関数 $\Gamma(z)$, ゼータ関数 $\zeta(z)$ は次のように定義される.
  \begin{align}
    \gamma(z, p) & := \int_0^p\dl{t}t^{z-1}e^{-t}, \qquad \Gamma(z, p) := \int_p^\infty\dl{t}t^{z-1}e^{-t} \\
    \Gamma(z)    & := \int_0^\infty\dl{t}t^{z-1}e^{-t} = \gamma(z, p) + \Gamma(z, p)                       \\
    \zeta(s)     & := \sum_{n=1}^{\infty}\frac{1}{n^s}.
  \end{align}
  ここでゼータ関数 $\zeta(4)$ の値は次の通りとなる。第二種不完全ガンマ関数 $\Gamma(z,p)$ の $p$ の極限について積分範囲が小さくなっていき, 被積分関数は発散しないので次のようになる.
  \begin{align}
    \Gamma(4) & = 6                \\
    \zeta(4)  & = \frac{\pi^4}{90}
  \end{align}
  $\Gamma(z, p)$ について部分積分することで次のように書ける.
  \begin{align}
    \Gamma(z, p) & = \int_p^\infty\dl{t}t^{z-1}e^{-t}                                                                                      \\
                 & = \sum_{m=0}^n\ab(-\ab[(z-1)\cdots(z-m)t^{z-m-1}e^{-t}]_p^\infty) + \int_p^\infty\dl{t} (z-1)\cdots(z-n)t^{z-n-1}e^{-t} \\
                 & = \sum_{m=0}^n\ab((z-1)\cdots(z-m)p^{z-m-1}e^{-p}) + \int_p^\infty\dl{t} (z-1)\cdots(z-n)t^{z-n-1}e^{-t}                \\
                 & = p^{z-1}e^{-p}\ab(1 + \sum_{m=1}^{\infty}\frac{1}{p^m}(z-1)(z-2)\cdots(z-m)) \qquad (\because n\to\infty).
  \end{align}
  $z = 4$ を代入すると次のようになる.
  \begin{align}
    \Gamma(4, p) & = p^{3}e^{-p}\ab(1 + \frac{3}{p} + \frac{6}{p^2} + \frac{6}{p^3}) \\
                 & = e^{-p}\ab(p^3 + 3p^2 + 6p + 6).
  \end{align}
  これより積分 $I(b)$ の第二種不完全ガンマ関数を展開することで次のようになる.
  \begin{align}
    I(b) & = \frac{1}{b^4}\ab(\frac{\pi^4}{15} - \sum_{n=1}^{\infty}\frac{1}{n^4}e^{-nb}\ab((nb)^3 + 3(nb)^2 + 6nb + 6))                           \\
         & = \frac{1}{b^4}\ab(\frac{\pi^4}{15} - \sum_{n=1}^{\infty}\ab(\frac{b^3}{n} + \frac{3b^2}{n^2} + \frac{6b}{n^3} + \frac{6}{n^4})e^{-nb}) \\
         & < \frac{1}{b^4}\ab(\frac{\pi^4}{15} - \ab(b^3 + 3b^2 + 6b + 6)\sum_{n=1}^{\infty}e^{-nb})                                               \\
         & = \frac{1}{b^4}\ab(\frac{\pi^4}{15} - \ab(b^3 + 3b^2 + 6b + 6)\frac{e^{-b}}{1 - e^{-b}})                                                \\
         & \sim \frac{1}{b^4}\ab(\frac{\pi^4}{15} - b^3e^{-b})
  \end{align}
  これより上界が指数関数的に小さくなることから $b\gg 1$ のとき $I(b)$ の最低次の漸近評価は十分正確である.
\end{proof}

\begin{theorem}
  比熱 $C$ は次のように評価できる.
  \begin{align}
    C =
    \begin{dcases}
      3nR\ab(1 - \frac{1}{20}\ab(\frac{\hbar\omega_D}{k_BT})^2 + \frac{1}{560}\ab(\frac{\hbar\omega_D}{k_BT})^4 - \frac{1}{18144}\ab(\frac{\hbar\omega_D}{k_BT})^6 + \cdots) & (b\ll 1) \\
      3nR\ab(\frac{4\pi^4}{5}\ab(\frac{k_BT}{\hbar\omega_D})^3 + \mathcal{O}(e^{-k_BT/\hbar\omega_D}))                                                                       & (b\gg 1)
    \end{dcases}
  \end{align}
\end{theorem}
\begin{proof}
  \begin{align}
    C & = 3nR\cdot (-3)b^2\diff{I(b)}{b}
  \end{align}
  \begin{align}
    C & \approx 3nR\cdot (-3)b^2\diff{}{b}\ab(\frac{1}{3b} - \frac{1}{8} + \frac{1}{60}b - \frac{1}{5040}b^3 + \frac{1}{272160}b^5 - \cdots)                                      \\
      & = 3nR\cdot (-3)b^2\ab(-\frac{1}{3b^2} + \frac{1}{60} - \frac{1}{1680}b^2 + \frac{1}{54432}b^4 - \cdots)                                                                   \\
      & = 3nR\ab(1 - \frac{1}{20}b^2 + \frac{1}{560}b^4 - \frac{1}{18144}b^6 + \cdots)                                                                                            \\
      & = 3nR\ab(1 - \frac{1}{20}\ab(\frac{\hbar\omega_D}{k_BT})^2 + \frac{1}{560}\ab(\frac{\hbar\omega_D}{k_BT})^4 - \frac{1}{18144}\ab(\frac{\hbar\omega_D}{k_BT})^6 + \cdots).
  \end{align}

  \begin{align}
    C & \approx 3nR\cdot(-3)b^2\ab(-\frac{\pi^4}{15}\frac{4}{b^5})    \\
      & = 3nR\times\frac{4\pi^4}{5}\ab(\frac{1}{b})^3                 \\
      & = 3nR\times\frac{4\pi^4}{5}\ab(\frac{k_BT}{\hbar\omega_D})^3.
  \end{align}
\end{proof}

よって Debye 模型の比熱は次のようにまとめられる.
\begin{itembox}[l]{Debye 模型の比熱}
  \begin{align}
    C & \approx 3nR\times\begin{dcases}
                           1                                                 & (k_BT\gg \hbar\omega_D) \\
                           \frac{4\pi^4}{5}\ab(\frac{k_BT}{\hbar\omega_D})^3 & (k_BT\ll \hbar\omega_D)
                         \end{dcases}.
  \end{align}
\end{itembox}

\subsection{黒体輻射}

\section{古典統計力学 (classical statistical mechanics) 近似}

\begin{theorem}
  \begin{align}
    Z & = \frac{1}{(2\pi\hbar)^f}\int e^{-H(p, q)/k_BT}\prod_{i=1}^{f}\dl{p_i}\dl{q_i}
  \end{align}
\end{theorem}

\subsection{振動子系の古典近似}
\subsection{理想気体の古典近似}
\subsection{非調和振動子系の古典近似}


\section{グランドカノニカル分布}
\begin{definition}
  内部エネルギー $U(S, V)$ とその束縛変数を変更させたエンタルピー $H(S, p)$ と Helmholtz 自由エネルギー $F(T, V)$ と Gibbs 自由エネルギー $G(T, p)$ を次のように定義する。
  グランドポテンシャル (grand potential) または熱力学ポテンシャル (thermodynamic potential) $J(T, V, \mu)$
  \begin{align}
                 & \qquad \dl{U} = T\dl{S} - p\dl{V} + \mu\dl{N}  \\
    H = U + pV   & \qquad \dl{H} = T\dl{S} + V\dl{p} + \mu\dl{N}  \\
    F = U - TS   & \qquad \dl{F} = -S\dl{T} - p\dl{V} + \mu\dl{N} \\
    G = F + pV   & \qquad \dl{G} = -S\dl{T} + V\dl{p} + \mu\dl{N} \\
    J = F - N\mu & \qquad \dl{J} = -S\dl{T} - p\dl{V} - N\dl{\mu}
  \end{align}
\end{definition}

\begin{theorem}
  \begin{alignat}{5}
    T  & = \ab(\diffp{U}{S})_{V,\mu} & \qquad -p & = \ab(\diffp{U}{V})_{S,\mu} & \qquad \mu & = \ab(\diffp{U}{\mu})_{S,V} \\
    T  & = \ab(\diffp{H}{S})_{p,\mu} & \qquad V  & = \ab(\diffp{H}{p})_{S,\mu} & \qquad \mu & = \ab(\diffp{H}{\mu})_{S,p} \\
    -S & = \ab(\diffp{F}{T})_{V,\mu} & \qquad -p & = \ab(\diffp{F}{V})_{T,\mu} & \qquad \mu & = \ab(\diffp{F}{\mu})_{T,V} \\
    -S & = \ab(\diffp{G}{T})_{p,\mu} & \qquad V  & = \ab(\diffp{G}{p})_{T,\mu} & \qquad \mu & = \ab(\diffp{G}{\mu})_{T,p} \\
    -S & = \ab(\diffp{J}{T})_{V,\mu} & \qquad -p & = \ab(\diffp{J}{V})_{T,\mu} & \qquad -N  & = \ab(\diffp{J}{\mu})_{T,V}
  \end{alignat}
\end{theorem}

\begin{proposition}[Maxwell の関係式]
  \begin{align*}
    \diffp{U}{S,V} & = \ab(\diffp{T}{V})_{S,N} = -\ab(\diffp{p}{S})_{V,N}      & \diffp{U}{S,N}   & = \ab(\diffp{T}{N})_{S,V} = \ab(\diffp{\mu}{S})_{V,N}     & \diffp{U}{V,N}   & = -\ab(\diffp{p}{N})_{S,V} = \ab(\diffp{\mu}{V})_{S,N}    \\
    \diffp{H}{S,p} & = \ab(\diffp{T}{p})_{S,N} = \ab(\diffp{V}{S})_{p,N}       & \diffp{H}{S,N}   & = \ab(\diffp{T}{N})_{S,p} = \ab(\diffp{\mu}{S})_{p,N}     & \diffp{H}{p,N}   & = \ab(\diffp{V}{N})_{S,p} = \ab(\diffp{\mu}{p})_{S,N}     \\
    \diffp{F}{T,V} & = -\ab(\diffp{S}{V})_{T,N} = -\ab(\diffp{p}{T})_{V,N}     & \diffp{F}{T,N}   & = -\ab(\diffp{S}{N})_{T,V} = \ab(\diffp{\mu}{T})_{V,N}    & \diffp{F}{V,N}   & = -\ab(\diffp{p}{N})_{T,V} = \ab(\diffp{\mu}{V})_{T,N}    \\
    \diffp{G}{T,p} & = -\ab(\diffp{S}{p})_{T,N} = \ab(\diffp{V}{T})_{p,N}      & \diffp{G}{T,N}   & = -\ab(\diffp{S}{N})_{T,p} = \ab(\diffp{\mu}{T})_{p,N}    & \diffp{G}{p,N}   & = \ab(\diffp{V}{N})_{T,p} = \ab(\diffp{\mu}{p})_{T,N}     \\
    \diffp{J}{T,V} & = -\ab(\diffp{S}{V})_{T,\mu} = -\ab(\diffp{p}{T})_{V,\mu} & \diffp{J}{T,\mu} & = -\ab(\diffp{S}{\mu})_{T,V} = -\ab(\diffp{N}{T})_{V,\mu} & \diffp{J}{V,\mu} & = -\ab(\diffp{p}{\mu})_{T,V} = -\ab(\diffp{N}{V})_{T,\mu}
  \end{align*}
\end{proposition}

\begin{theorem}[Gibbs-Duhem の関係]
  グランドポテンシャルについて次の関係式が成り立つ。
  \begin{align}
     & J = -pV                           \\
     & V\dl{p} - S\dl{T} - N\dl{\mu} = 0
  \end{align}
  第二式を Gibbs-Duhem の関係という。
\end{theorem}
\begin{proof}
  グランドポテンシャル $J(T, V, \mu)$ について $V$ は示量変数、$T, \mu$ は示強変数であるから系の大きさを $\lambda$ 倍すると
  \begin{align}
    J(T, \lambda V, \mu) & = \lambda J(T, V, \mu)
  \end{align}
  となる。これに両辺 $\lambda$ で微分して $\lambda = 1$ を代入する。
  \begin{align}
    \left.\diff{}{\lambda}J(T, \lambda V, \mu)\right|_{\lambda = 1} & = \left.\ab(\diffp{(\lambda V)}{\lambda}\diffp{}{(\lambda V)}J(T, \lambda V, \mu))_{T,\mu}\right|_{\lambda = 1} = V\ab(\diffp{J}{V})_{T,\mu} = -pV
  \end{align}
  よって $J = -pV$ となる。また定義式より Gibbs-Duhem の関係が求まる。
  \begin{align}
    \dl{J} = \dl{(-pV)} = - p\dl{V} - V\dl{p} & = -S\dl{T} - p\dl{V} - N\dl{\mu} \\
    V\dl{p} - S\dl{T} - N\dl{\mu}             & = 0
  \end{align}
\end{proof}

\begin{definition}[グランドカノニカル分布]
  グランドカノニカル分布において分配関数 $\Xi(T, V, \mu)$ を次のように定義する。
  \begin{align}
    \Xi = \sum_{n}e^{-\beta(E_n - \mu N_n)}
  \end{align}
\end{definition}

\begin{theorem}
  このときグランドポテンシャル $J$ や粒子数 $N$ など
  \begin{align}
    J & = -k_BT\ln\Xi                                   \\
    N & = \frac{1}{\beta}\ab(\diffp{\ln\Xi}{\mu})_{T,V}
  \end{align}
\end{theorem}
\begin{proof}
  整理して両辺を微分すると
  \begin{align}
    \dl{(\log\Xi)}           & = \frac{\dl{\Xi}}{\Xi}                                                                                                                                                                                         \\
                             & = -\beta\frac{\sum_{n}\dl{(E_n - \mu N_n)}e^{-\beta(E_n - \mu N_n)}}{\sum_{n}e^{-\beta(E_n - \mu N_n)}} - \dl{\beta}\frac{\sum_{n}(E_n - \mu N_n)e^{-\beta(E_n - \mu N_n)}}{\sum_{n}e^{-\beta(E_n - \mu N_n)}} \\
                             & = -\beta(\langle\dl{E}\rangle - \langle N\rangle\dl{\mu}) - \dl{\beta}(\langle E\rangle - \langle N\rangle\mu)                                                                                                 \\
                             & = -\frac{1}{k_BT}(-p\dl{V} - N\dl{\mu}) + \frac{\dl{T}}{k_BT^2}(U - N\mu)                                                                                                                                      \\
    \dl{\ab(\frac{J}{k_BT})} & = \ab(\frac{\dl{J}}{k_BT}) - \ab(\frac{J}{k_BT^2})\dl{T}                                                                                                                                                       \\
                             & = \ab(\frac{-S\dl{T} - p\dl{V} - N\dl{\mu}}{k_BT}) - \ab(\frac{U - TS - N\mu}{k_BT^2})\dl{T}                                                                                                                   \\
                             & = \frac{1}{k_BT}(-p\dl{V} - N\dl{\mu}) - \frac{\dl{T}}{k_BT^2}(U - N\mu)
  \end{align}

\end{proof}

\begin{theorem}[粒子数の揺らぎ]
  \begin{align}
    \ab(\diffp{N}{\mu})_{T,V} = \beta\langle\Delta N^2\rangle
  \end{align}
\end{theorem}

\section{Bose 統計と Fermi 統計}
\begin{theorem}
  2 粒子の波動関数は $\varphi(\rr_1, \rr_2)$ と書かれる。
  \begin{align}
    \varphi(\rr_1, \rr_2) = \pm \varphi(\rr_2, \rr_1)
  \end{align}
  対称な粒子、反対称な粒子
\end{theorem}
\begin{proof}
  添字を交換しても物理的な状態としては同一なので定数 $\alpha$ を用いて $\varphi(\rr_1, \rr_2) = \alpha\varphi(\rr_2, \rr_1)$ と書ける。
  \begin{align}
    \alpha^2 = 1 \iff \alpha = \pm 1
  \end{align}
\end{proof}

\begin{definition}
  上の定理において $\alpha = 1$ となる粒子をボース粒子またはボゾン (boson) といい、$\alpha = -1$ となる粒子をフェルミ粒子またはフェルミオン (fermion) という。

  \begin{itemize}
    \item Fermi 粒子: 電子・陽子
    \item Bose 粒子: 光子
  \end{itemize}
\end{definition}

\begin{itemize}
  \item Fermi 統計: $e^{-\beta(\varepsilon_1 + \varepsilon_2)} + e^{-\beta(\varepsilon_2 + \varepsilon_3)} + e^{-\beta(\varepsilon_3 + \varepsilon_1)}$
  \item Bose 統計: $e^{-2\beta\varepsilon_1} + e^{-2\beta\varepsilon_2} + e^{-2\beta\varepsilon_3} + e^{-\beta(\varepsilon_1 + \varepsilon_2)} + e^{-\beta(\varepsilon_2 + \varepsilon_3)} + e^{-\beta(\varepsilon_3 + \varepsilon_1)}$
  \item ボルツマン統計: $\dfrac{1}{2!}(e^{-2\beta\varepsilon_1} + e^{-2\beta\varepsilon_2} + e^{-2\beta\varepsilon_3})$ \\
        $\dfrac{1}{2}e^{-2\beta\varepsilon_1} + \dfrac{1}{2}e^{-2\beta\varepsilon_2} + \dfrac{1}{2}e^{-2\beta\varepsilon_3} + e^{-\beta(\varepsilon_1 + \varepsilon_2)} + e^{-\beta(\varepsilon_2 + \varepsilon_3)} + e^{-\beta(\varepsilon_3 + \varepsilon_1)}$
\end{itemize}

\begin{align}
  N & = -\ab(\diffp{J}{\mu})_T = \frac{1}{\beta}\ab(\diffp{\ln\Xi(\beta,\mu)}{\mu})
\end{align}

\begin{theorem}[分配関数と分布関数]
  Fermi 統計と Bose 統計における分配関数 $\Xi(\beta,\mu)$、分布関数 $f(\varepsilon)$ は次のようになる。
  \begin{alignat}{3}
    \Xi_B(\beta, \mu) & = \prod_{j=1}^{\infty}\frac{1}{1 - e^{-\beta(\varepsilon_j - \mu)}} & \qquad f_B(\varepsilon) & = \frac{1}{e^{\beta(\varepsilon_j - \mu)} - 1} \\
    \Xi_F(\beta, \mu) & = \prod_{j=1}^{\infty}\ab(1 + e^{-\beta(\varepsilon_j - \mu)})      & \qquad f_F(\varepsilon) & = \frac{1}{e^{\beta(\varepsilon_j - \mu)} + 1}
  \end{alignat}
\end{theorem}
\begin{proof}
  Bose 統計
  \begin{align}
    \Xi_B^{(j)}(\beta, \mu)                  & = \sum_{n=0}^{\infty}e^{-\beta(\varepsilon_j - \mu)n} = \frac{1}{1 - e^{-\beta(\varepsilon_j - \mu)}}                                                                                     \\
    \Xi_B(\beta, \mu)                        & = \prod_{j=1}^{\infty}\Xi_B^{(j)}(\beta, \mu) = \prod_{j=1}^{\infty}\frac{1}{1 - e^{-\beta(\varepsilon_j - \mu)}}                                                                         \\
    f_B(\varepsilon_j) := \langle n_j\rangle & = \frac{1}{\beta}\ab(\diffp{}{\mu}\ln\Xi_B^{(j)}(\beta,\mu)) = \frac{e^{-\beta(\varepsilon_j - \mu)}}{1 - e^{-\beta(\varepsilon_j - \mu)}} = \frac{1}{e^{\beta(\varepsilon_j - \mu)} - 1} \\
    N                                        & = \sum_{j=1}^{\infty}\frac{1}{e^{\beta(\varepsilon_j - \mu)} - 1}
  \end{align}
  Fermi 統計において
  \begin{align}
    \Xi_F^{(j)}(\beta, \mu)                  & = \sum_{n=0}^{1}e^{-\beta(\varepsilon_j - \mu)n} = 1 + e^{-\beta(\varepsilon_j - \mu)}                                                                                                    \\
    \Xi_F(\beta, \mu)                        & = \prod_{j=1}^{\infty}\Xi_F^{(j)}(\beta, \mu) = \prod_{j=1}^{\infty}\ab(1 - e^{-\beta(\varepsilon_j - \mu)})                                                                              \\
    f_F(\varepsilon_j) := \langle n_j\rangle & = \frac{1}{\beta}\ab(\diffp{}{\mu}\ln\Xi_F^{(j)}(\beta,\mu)) = \frac{e^{-\beta(\varepsilon_j - \mu)}}{1 + e^{-\beta(\varepsilon_j - \mu)}} = \frac{1}{e^{\beta(\varepsilon_j - \mu)} + 1} \\
    N                                        & = \sum_{j=1}^{\infty}\frac{1}{e^{\beta(\varepsilon_j - \mu)} + 1}
  \end{align}
\end{proof}

\subsection{Fermi-Dirac 統計力学}
Fermi 粒子において分布関数は次のようだった。
\begin{align}
  f_F(\varepsilon) & = \frac{1}{e^{\beta(\varepsilon - \mu)} + 1}
\end{align}
これはエネルギーに対して次のような関数となる。
低温と高温の極限において考えることで
\begin{align}
  \lim_{T\to 0}f_F(\varepsilon)      & = \begin{dcases}
                                           1   & (\varepsilon < \mu) \\
                                           1/2 & (\varepsilon = \mu) \\
                                           0   & (\varepsilon > \mu)
                                         \end{dcases}           \\
  \lim_{T\to \infty}f_F(\varepsilon) & \approx e^{-\beta(\varepsilon - \mu)}
\end{align}

\begin{theorem}[ゾンマーフェルト展開]
  次の積分を次のように展開できる。
  \begin{align}
    I(\beta, \mu) & := \int_{-\infty}^{\infty}\dl{\varepsilon}g(\varepsilon)f_F(\varepsilon) = \int_{-\infty}^\mu g(\varepsilon)\dl{\varepsilon} + \frac{\pi^2}{6}g'(\mu)(k_BT)^2 + O((k_BT)^{4})
  \end{align}
\end{theorem}
\begin{proof}
  \begin{align}
    G(\varepsilon) & = \int_{-\infty}^\varepsilon g(\varepsilon)\dl{\varepsilon}
  \end{align}

  \begin{align}
    I(\beta, \mu) & = \int_{-\infty}^{\infty}\dl{\varepsilon}g(\varepsilon)f_F(\varepsilon)                                                                                  \\
                  & = [G(\varepsilon)f_F(\varepsilon)]_{-\infty}^{\infty} - \int_{-\infty}^{\infty}\dl{\varepsilon}G(\varepsilon)f_F'(\varepsilon)                           \\
                  & = - \int_{-\infty}^{\infty}\dl{\varepsilon}G(\varepsilon)f_F'(\varepsilon)                                                                               \\
                  & = - \int_{-\infty}^{\infty}\dl{\varepsilon}\ab[G(\mu) + G'(\mu)(\varepsilon - \mu) + \frac{1}{2}G''(\mu)(\varepsilon - \mu)^2 + \cdots]f_F'(\varepsilon)
  \end{align}
  $x = \beta(\varepsilon - \mu)$ と変数変換すると奇関数性より
  \begin{align}
    \int_{-\infty}^{\infty}\dl{\varepsilon}f_F'(\varepsilon)                      & = [f_F(\varepsilon)]_{-\infty}^{\infty} = -1                                                                                                                        \\
    \int_{-\infty}^{\infty}\dl{\varepsilon}f_F'(\varepsilon)(\varepsilon - \mu)   & = \int_{-\infty}^{\infty}\dl{\varepsilon}\frac{x}{\beta}\frac{e^x}{(e^x + 1)^2} = \int_{-\infty}^{\infty}\dl{\varepsilon}\frac{x}{\beta}\frac{x}{4\cosh^2(x/2)} = 0 \\
    \int_{-\infty}^{\infty}\dl{\varepsilon}f_F'(\varepsilon)(\varepsilon - \mu)^2 & = \frac{1}{\beta^2}\int_{-\infty}^{\infty}\dl{\varepsilon}x^2\diff{}{x}\ab(\frac{1}{e^x + 1})                                                                       \\
                                                                                  & = \frac{2}{\beta^2}\ab[\frac{x^2}{e^x + 1}]_{0}^{\infty} - \frac{4}{\beta^2}\int_{0}^{\infty}\dl{\varepsilon}\frac{x}{e^x + 1}                                      \\
                                                                                  & = - \frac{4}{\beta^2}\int_{0}^{\infty}\dl{\varepsilon}x\sum_{n=1}^{\infty}(-1)^{n-1}e^{-nx}                                                                         \\
                                                                                  & = - \frac{4}{\beta^2}\sum_{n=1}^{\infty}\frac{(-1)^{n-1}}{2n^2}                                                                                                     \\
                                                                                  & = - \frac{4}{\beta^2}\frac{\pi^2}{12} = -\frac{\pi^2}{3\beta^2}
  \end{align}
  \begin{align}
    I(\beta, \mu) & = G(\mu) + \frac{\pi^2}{6\beta^2}G''(\mu) + O(\beta^{-4})                                            \\
                  & = \int_{-\infty}^\mu g(\varepsilon)\dl{\varepsilon} + \frac{\pi^2}{6}g'(\mu)(k_BT)^2 + O((k_BT)^{4})
  \end{align}
\end{proof}

\begin{theorem}
  \begin{align}
    \mu        & \approx \varepsilon_F - \frac{\pi^2}{6}\frac{\nu'(\varepsilon_F)}{\nu(\varepsilon_F)}(k_BT)^2 \\
    c(T, \rho) & = \frac{\pi^2}{3}\nu(\varepsilon_F)k_B^2T
  \end{align}
\end{theorem}
\begin{proof}
  \begin{align}
    N & = \int_{-\infty}^{\infty}\dl{\varepsilon}\nu(\varepsilon)f_F(\varepsilon)                                                                                        \\
      & \approx \int_{-\infty}^{\mu}\dl{\varepsilon}\nu(\varepsilon) + \frac{\pi^2}{6}\nu'(\mu)(k_BT)^2                                                                  \\
      & = \int_{-\infty}^{\varepsilon_F}\dl{\varepsilon}\nu(\varepsilon) + \int_{\varepsilon_F}^{\mu}\dl{\varepsilon}\nu(\varepsilon) + \frac{\pi^2}{6}\nu'(\mu)(k_BT)^2 \\
      & \approx N + (\mu - \varepsilon_F)\nu(\varepsilon_F) + \frac{\pi^2}{6}\nu'(\mu)(k_BT)^2
  \end{align}
  \begin{align}
    \mu \approx \varepsilon_F - \frac{\pi^2}{6}\frac{\nu'(\varepsilon_F)}{\nu(\varepsilon_F)}(k_BT)^2
  \end{align}


  \begin{align}
    U & = \int_{-\infty}^{\infty}\dl{\varepsilon}\varepsilon\nu(\varepsilon)f_F(\varepsilon)                                                                                                                                                     \\
      & \approx \int_{-\infty}^\mu\dl{\varepsilon}\varepsilon\nu(\varepsilon) + \frac{\pi^2}{6}(\varepsilon\nu(\varepsilon))'|_{\varepsilon = \mu}(k_BT)^2                                                                                       \\
      & = \int_{-\infty}^{\varepsilon_F}\dl{\varepsilon}\varepsilon\nu(\varepsilon) + \int_{\varepsilon_F}^\mu\dl{\varepsilon}\varepsilon\nu(\varepsilon) + \frac{\pi^2}{6}(\varepsilon\nu(\varepsilon))'|_{\varepsilon = \mu}(k_BT)^2           \\
      & = \int_{-\infty}^{\varepsilon_F}\dl{\varepsilon}\varepsilon\nu(\varepsilon) + (\mu - \varepsilon_F)\varepsilon_F\nu(\varepsilon_F) + \frac{\pi^2}{6}\nu(\varepsilon_F)(k_BT)^2 + \frac{\pi^2}{6}\varepsilon_F\nu'(\varepsilon_F)(k_BT)^2 \\
      & \approx \int_{-\infty}^{\varepsilon_F}\dl{\varepsilon}\varepsilon\nu(\varepsilon) + \frac{\pi^2}{6}\nu(\varepsilon_F)(k_BT)^2
  \end{align}

  \begin{align}
    c(T, \rho) & = \diffp{U}{T} = \frac{\pi^2}{3}\nu(\varepsilon_F)k_B^2T
  \end{align}
\end{proof}

\subsection{Bose-Einstein 統計力学}
Bose 粒子において分布関数は次のようだった。
\begin{align}
  f_B(\varepsilon) & = \frac{1}{e^{\beta(\varepsilon - \mu)} - 1}
\end{align}
$\varepsilon = \mu$ において発散する関数となる。通常は $\mu < 0$ であるため、積分区間では被積分関数の発散が起こらないが $\mu \approx 0$ となると $\varepsilon = 0$ において離散的な値となる。

TODO: 解釈

\begin{theorem}
  3 次元空間の自由なボゾンを考えてエネルギー $\varepsilon$ とエネルギーに対する状態密度 $\nu(\varepsilon)$ が次のように与えられるとする。
  \begin{align}
    \varepsilon = \frac{\hbar^2k^2}{2m}, \qquad \nu(\varepsilon) = c\sqrt{\varepsilon}, \qquad c = \frac{V}{4\pi^2}\ab(\frac{2m}{\hbar^2})^{3/2}
  \end{align}
  ただしエネルギーゼロの準位の状態は 1 つとする。このとき転移温度 $T_c$ 以下においてエネルギーゼロの準位に入る粒子数 $N_0$ がマクロな個数となる。これをボーズ凝縮 (Bose condensation) という。
  \begin{align}
    T_c = \frac{2\pi\hbar^2}{mk_B}\ab(\frac{1}{\zeta\ab(\frac{3}{2})}\frac{N}{V})^{2/3} \approx 0.5273\frac{2\pi\hbar^2}{mk_B}\rho^{2/3}
  \end{align}
  \begin{align}
    N_0 & = \frac{1}{e^{-\beta\mu} - 1} \approx N\ab(1 - \ab(\frac{T}{T_c})^{3/2})        \\
    U_0 & = \frac{3V}{2}\ab(\frac{m}{2\pi\hbar^2})^{3/2}(k_BT)^{5/2}\zeta\ab(\frac{5}{2}) \\
    C_0 & = \frac{15V}{4}\ab(\frac{mT}{2\pi\hbar^2})^{3/2}k_B^{5/2}\zeta\ab(\frac{5}{2})
  \end{align}
\end{theorem}
\begin{proof}
  系の粒子数 $N$ に対してエネルギー準位がゼロとそれ以外の粒子数をそれぞれ $N_0(\beta, \mu)$, $N'(\beta, \mu)$ とする。それぞれ次のように計算できる。
  \begin{align}
    N_0(\beta, \mu) & = \frac{1}{e^{-\beta\mu} - 1}                                                                                                                                                \\
    N'(\beta, \mu)  & = c\int_{0}^{\infty}\dl{\varepsilon}\frac{\sqrt{\varepsilon}}{e^{\beta(\varepsilon - \mu)} - 1} = c\beta^{-3/2}\int_{0}^{\infty}\dl{u}\frac{u^{1/2}}{e^{-\beta\mu}e^{u} - 1}
  \end{align}
  粒子数 $N'(\beta, \mu)$ は $\mu = 0$ において最大となる。これを $N'_{\max}$ とおく。
  \begin{align}
    N'(\beta, 0) & = c\beta^{-3/2}\int_{0}^{\infty}\dl{u}\frac{u^{1/2}}{e^{u} - 1}           \\
                 & = c\beta^{-3/2}\int_{0}^{\infty}\dl{u}\sum_{n=1}^{\infty}u^{1/2}e^{-nu}   \\
                 & = c\beta^{-3/2}\sum_{n=1}^{\infty}\frac{1}{n^{3/2}}\Gamma\ab(\frac{3}{2}) \\
                 & = c\beta^{-3/2}\zeta\ab(\frac{3}{2})\Gamma\ab(\frac{3}{2})                \\
                 & = \frac{V}{8}\ab(\frac{2mk_BT}{\pi\hbar^2})^{3/2}\zeta\ab(\frac{3}{2})
  \end{align}
  このとき系の粒子数 $N$ と $N'_{\max}$ の大小関係に着目する。このとき温度 $T$ と転移温度 $T_c$ の大小関係と対応できる。
  \begin{align}
    \begin{dcases}
      N   < N'_{\max} \propto T_c^{3/2} < T^{3/2} \iff T > T_c \\
      N   > N'_{\max} \propto T_c^{3/2} > T^{3/2} \iff T < T_c
    \end{dcases} \qquad
    \ab(T_c = \frac{2\pi\hbar^2}{mk_B}\ab(\frac{1}{\zeta\ab(\frac{3}{2})}\frac{N}{V})^{2/3})
  \end{align}
  このように
  \begin{enumerate}
    \item $N < N'_{\max}$ つまり転移温度より高温のとき $\mu$ は非ゼロの負の値となり、$N_0 \approx 0$ となる。
    \item $N > N'_{\max}$ つまり転移温度より低温のとき $\mu$ はゼロに近い負の値となり $N_0 \approx N - N'_{\max}$ となる。
  \end{enumerate}
  \begin{align}
    N_0 \approx N - N'_{\max} = N - \frac{V}{8}\ab(\frac{2mk_BT}{\pi\hbar^2})^{3/2}\zeta\ab(\frac{3}{2}) = N\ab(1 - \ab(\frac{T}{T_c})^{3/2})
  \end{align}

  さらにエネルギーと比熱について
  \begin{align}
    U   & = \int_{0}^{\infty}\dl{\varepsilon}\varepsilon\nu(\varepsilon)\frac{1}{e^{\beta(\varepsilon - \mu)} - 1} = c\int_{0}^{\infty}\dl{\varepsilon}\frac{\varepsilon^{3/2}}{e^{\beta(\varepsilon - \mu)} - 1} \\
        & = \frac{V}{4\pi^2}\ab(\frac{2m}{\hbar^2})^{3/2}(k_BT)^{5/2}\int_{0}^{\infty}\dl{u}\frac{u^{3/2}}{e^ue^{-\beta\mu} - 1}                                                                                  \\
    U_0 & = \frac{V}{4\pi^2}\ab(\frac{2m}{\hbar^2})^{3/2}(k_BT)^{5/2}\int_{0}^{\infty}\dl{u}\frac{u^{3/2}}{e^u - 1}                                                                                               \\
        & = \frac{V}{4\pi^2}\ab(\frac{2m}{\hbar^2})^{3/2}(k_BT)^{5/2}\Gamma\ab(\frac{5}{2})\zeta\ab(\frac{5}{2})                                                                                                  \\
        & = \frac{3V}{2}\ab(\frac{m}{2\pi\hbar^2})^{3/2}(k_BT)^{5/2}\zeta\ab(\frac{5}{2})
  \end{align}

  \begin{align}
    C_0 & = \diff{U_0}{T} = \frac{15V}{4}\ab(\frac{mT}{2\pi\hbar^2})^{3/2}k_B^{5/2}\zeta\ab(\frac{5}{2})
  \end{align}
\end{proof}

\section{相転移と臨界現象}
\subsection{相と相平衡}
\subsection{Landau 理論}

\section{イジング模型}

\begin{align}
  \langle S\rangle
\end{align}


\end{document}