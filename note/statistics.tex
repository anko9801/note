\RequirePackage{plautopatch}
\documentclass[uplatex,dvipdfmx,a4paper,11pt]{jlreq}
\usepackage{bxpapersize}
\usepackage[utf8]{inputenc}
\usepackage{fontenc}
\usepackage{lmodern}
\usepackage{otf}
\usepackage{amsmath}
\usepackage{amssymb}
\usepackage{amsthm}
\usepackage{ascmac}
% \usepackage[hyphens]{url}
\usepackage{physics}
\usepackage{braket}
\usepackage{verbatimbox}
\usepackage{bm}
\usepackage{url}
\usepackage{siunitx}
% \usepackage[dvipdfmx,hiresbb,final]{graphicx}
\usepackage{hyperref}
\usepackage{pxjahyper}
\usepackage{tikz}\usetikzlibrary{cd}
\usepackage{listings}
\usepackage{color}
\usepackage{mathtools}
\usepackage{xspace}
\usepackage{xy}
\usepackage{xypic}
%
\title{統計力学}
\author{Anko}
\makeatletter
%
\DeclareMathOperator{\lcm}{lcm}
\DeclareMathOperator{\Kernel}{Ker}
\DeclareMathOperator{\Image}{Im}
\DeclareMathOperator{\ch}{ch}
\DeclareMathOperator{\Aut}{Aut}
\DeclareMathOperator{\Log}{Log}
\DeclareMathOperator{\Arg}{Arg}
\DeclareMathOperator{\sgn}{sgn}
%
\newcommand{\CC}{\mathbb{C}}
\newcommand{\RR}{\mathbb{R}}
\newcommand{\QQ}{\mathbb{Q}}
\newcommand{\ZZ}{\mathbb{Z}}
\newcommand{\NN}{\mathbb{N}}
\newcommand{\FF}{\mathbb{F}}
\newcommand{\PP}{\mathbb{P}}
\newcommand{\GG}{\mathbb{G}}
\newcommand{\TT}{\mathbb{T}}
\newcommand{\calB}{\mathcal{B}}
\newcommand{\calF}{\mathcal{F}}
\newcommand{\ignore}[1]{}
\newcommand{\floor}[1]{\left\lfloor #1 \right\rfloor}
% \newcommand{\abs}[1]{\left\lvert #1 \right\rvert}
\newcommand{\lt}{<}
\newcommand{\gt}{>}
\newcommand{\id}{\mathrm{id}}
\newcommand{\rot}{\curl}
\renewcommand{\angle}[1]{\left\langle #1 \right\rangle}
\newcommand{\EE}{\bm{E}}
\newcommand{\BB}{\bm{B}}
\renewcommand{\AA}{\bm{A}}
\newcommand{\rr}{\bm{r}}
\newcommand{\kk}{\bm{k}}
\newcommand{\pp}{\bm{p}}

\let\oldcite=\cite
\renewcommand\cite[1]{\hyperlink{#1}{\oldcite{#1}}}

\let\oldbibitem=\bibitem
\renewcommand{\bibitem}[2][]{\label{#2}\oldbibitem[#1]{#2}}

% theorem環境の設定
% - 冒頭に改行
% - 末尾にdiamond (amsthm)
\theoremstyle{definition}
\newcommand*{\newscreentheoremx}[2]{
  \newenvironment{#1}[1][]{
    \begin{screen}
    \begin{#2}[##1]
      \leavevmode
      \newline
  }{
    \end{#2}
    \end{screen}
  }
}
\newcommand*{\newqedtheoremx}[2]{
  \newenvironment{#1}[1][]{
    \begin{#2}[##1]
      \leavevmode
      \newline
      \renewcommand{\qedsymbol}{\(\diamond\)}
      \pushQED{\qed}
  }{
      \qedhere
      \popQED
    \end{#2}
  }
}
\newtheorem{theorem*}{定理}

\newqedtheoremx{theorem}{theorem*}
\newcommand*\newqedtheorem@unstarred[2]{%
  \newtheorem{#1*}[theorem*]{#2}
  \newqedtheoremx{#1}{#1*}
}
\newcommand*\newqedtheorem@starred[2]{%
  \newtheorem*{#1*}{#2}
  \newqedtheoremx{#1}{#1*}
}
\newcommand*{\newqedtheorem}{\@ifstar{\newqedtheorem@starred}{\newqedtheorem@unstarred}}

\newtheorem{sctheorem*}{定理}
\newscreentheoremx{sctheorem}{sctheorem*}
\newcommand*\newscreentheorem@unstarred[2]{%
  \newtheorem{#1*}[theorem*]{#2}
  \newscreentheoremx{#1}{#1*}
}
\newcommand*\newscreentheorem@starred[2]{%
  \newtheorem*{#1*}{#2}
  \newscreentheoremx{#1}{#1*}
}
\newcommand*{\newscreentheorem}{\@ifstar{\newscreentheorem@starred}{\newscreentheorem@unstarred}}

%\newtheorem*{definition}{定義}
%\newtheorem{theorem}{定理}
%\newtheorem{proposition}[theorem]{命題}
%\newtheorem{lemma}[theorem]{補題}
%\newtheorem{corollary}[theorem]{系}

\newqedtheorem{lemma}{補題}
\newqedtheorem{corollary}{系}
\newqedtheorem{example}{例}
\newqedtheorem{proposition}{命題}
\newqedtheorem{remark}{注意}
\newqedtheorem{thesis}{主張}
\newqedtheorem{notation}{記法}
\newqedtheorem{problem}{問題}
\newqedtheorem{algorithm}{アルゴリズム}

\newscreentheorem*{axiom}{公理}
\newscreentheorem*{definition}{定義}

\renewenvironment{proof}[1][\proofname]{\par
  \normalfont
  \topsep6\p@\@plus6\p@ \trivlist
  \item[\hskip\labelsep{\bfseries #1}\@addpunct{\bfseries}]\ignorespaces\quad\par
}{%
  \qed\endtrivlist\@endpefalse
}
\renewcommand\proofname{証明}

\makeatother

\begin{document}
\maketitle
\tableofcontents
\clearpage

\section{統計力学の基礎}
エルゴード理論により次の原理が成り立つこととする。
\begin{axiom}[等確率の原理]
  孤立系を十分に長時間放置しておくと物体の実現可能な量子状態はエネルギーのゆらぎを除いてすべて等確率で実現する。
\end{axiom}

2 つの系 $A, B$ があるとする。系 $A$ のエネルギー $E_A$ と系 $B$ のエネルギー $E_B$ の和が一定で $A, B$ の間にエネルギーのやり取りができるとする。
\begin{align}
  E_A + E_B = const.
\end{align}

例えると子どもたちが 12 人居て $A$ と $B$ のグループにそれぞれ 4 人、8 人で分ける。そして 6 個あるリンゴを 1 人複数個もらっても良いとして等確率に配ったとき、それぞれのグループに配られるリンゴで最も確率の高いものは何か。
\begin{table}[hbtp]
  \label{table:micro}
  \centering
  \begin{tabular}{|c|c|l|}
    \hline
    A   & B   & 組合せ                                                         \\
    \hline
    0 個 & 6 個 & ${}_4H_0\times {}_{8}H_6 = {}_3C_0\times {}_{13}C_6 = 1716$ \\
    1 個 & 5 個 & ${}_4H_1\times {}_{8}H_5 = {}_4C_1\times {}_{12}C_5 = 3168$ \\
    2 個 & 4 個 & ${}_4H_2\times {}_{8}H_4 = {}_5C_2\times {}_{11}C_4 = 3300$ \\
    3 個 & 3 個 & ${}_4H_3\times {}_{8}H_3 = {}_6C_3\times {}_{10}C_3 = 2406$ \\
    4 個 & 2 個 & ${}_4H_4\times {}_{8}H_2 = {}_7C_4\times {}_{9}C_2 = 1260$  \\
    5 個 & 1 個 & ${}_4H_5\times {}_{8}H_1 = {}_8C_5\times {}_{8}C_1 = 448$   \\
    6 個 & 0 個 & ${}_4H_6\times {}_{8}H_0 = {}_9C_6\times {}_{7}C_0 = 84$    \\
    \hline
  \end{tabular}
  \caption{組合せ}
\end{table}

より $A$, $B$ のグループにそれぞれ 2 個、4 個で分ける確率が最も高い。
この分布を二項分布という。

\begin{proposition}
  二項分布の極限が正規分布である。
\end{proposition}
\begin{proof}
\end{proof}

\begin{definition}
  あるエネルギー $E$ のときに実現可能な量子状態数を $W(E)$ とおき、その対数を取ったものをエントロピー $S(E)$ という。
  \begin{align}
    S(E) & = k_B\log W(E)                     \\
    k_B  & = 1.380658\times 10^{-23} \si{J/K}
  \end{align}
  ただし $k_B$ をボルツマン定数 (Boltzmann constant) という。ある系 $X$ のエネルギーを $E_X$、状態数を $W_X(E_X)$、エントロピーを $S_X(E_X)$ と書くことにする。
\end{definition}
状態数で計算すると指数が出がちなのでエントロピーで計算すると簡単になる。

\begin{theorem}
  $N$ 次元の調和振動子で $E = M\hbar\omega$ とおくと状態数とエントロピーは次のように書ける。
  \begin{align}
    W(E) & = \mqty(M + N - 1                                                                         \\ N - 1) \\
    S(E) & \approx k_BN\qty(\qty(1 + \frac{M}{N})\log(1 + \frac{M}{N}) - \frac{M}{N}\log\frac{M}{N})
  \end{align}
\end{theorem}
\begin{proof}
  $N$ 次元の調和振動子系では $(n_1,\ldots,n_N)$ が全体の量子状態を決める量子数となる。このときのエネルギーは次のように表される。
  \begin{align}
    E_{(n_1,\ldots,n_N)} & = n_1\hbar\omega + \cdots + n_N\hbar\omega
  \end{align}
  等しいエネルギーの状態の条件は $M = n_1 + \cdots + n_N$ と書ける。これより状態数の組合せは次のように書ける。
  \begin{align}
    W(E) & = \mqty(M + N - 1 \\ N - 1) = \frac{(M + N - 1)!}{(N - 1)!M!}
  \end{align}
  またエントロピーは Stirling の公式 $\log n! \approx n(\log n - 1)$ を用いて
  \begin{align}
    S(E) & = k_B\log W(E)                                                                      \\
         & = k_B\log\frac{(M + N - 1)!}{(N - 1)!M!}                                            \\
         & \approx k_B\qty((N + M)(\log(N + M) - 1) - N(\log N - 1) - M(\log M - 1))           \\
         & = k_BN\qty(\qty(1 + \frac{M}{N})\log(1 + \frac{M}{N}) - \frac{M}{N}\log\frac{M}{N})
  \end{align}
\end{proof}

\begin{theorem}
  熱平衡の条件は系 $A$ の温度 $T_A$ と系 $B$ の温度 $T_B$ が一致すること。
\end{theorem}
\begin{proof}
  各系の状態数の積が全体系の状態数となるので各系と全体系のエントロピーの関係は
  \begin{align}
    S(E_A, E_B) & = k_B\log W(E_A, E_B)                 \\
                & = k_B\log W_A(E_A)W_B(E_B)            \\
                & = k_B\log W_A(E_A) + k_B\log W_B(E_B) \\
                & = S_A(E_A) + S_B(E_B)
  \end{align}
  となる。このとき熱平衡状態とはエントロピーが最大の状態であるから $\dv*{S}{E_A} = 0$ となるエネルギー $E_A, E_B$ を考えると
  \begin{align}
    \dv{S(E_A, E_B)}{E_A}   & = \dv{S_A(E_A)}{E_A} + \dv{S_A(E_B)}{E_A} = \dv{S_A(E_A)}{E_A} - \dv{S_A(E_B)}{E_B} = 0 \\
    \iff \dv{S_A(E_A)}{E_A} & = \dv{S_A(E_B)}{E_B}
  \end{align}
  よりエントロピーのエネルギー微分を温度の逆数 $1/T$ と定義すると温度が一致するときに熱平衡状態となる。
\end{proof}

\begin{definition}
  絶対温度 (absolute temperature) $T$ を次のように定義する。
  \begin{align}
    \frac{1}{T} & := \dv{S}{E}
  \end{align}
\end{definition}

この温度の定義は理想気体で正当化される。

\begin{theorem}[理想気体]
  理想気体、つまり 3 次元箱型ポテンシャル中の独立な区別できない $N$ 個の粒子について
  \begin{align}
    S & = Nk_B\qty(\frac{3}{2}\ln\frac{E}{V} + \frac{5}{2}\ln\frac{V}{N} + \ln\alpha + \mathcal{O}(N^{-1}\ln N)) & \qty(\alpha = \qty(\frac{me}{3\pi\hbar^2})^{\frac{3}{2}}) \\
    E & = \frac{3}{2}Nk_BT
  \end{align}
\end{theorem}
\begin{proof}
  部分系の固有状態と固有エネルギーが分かれば全体系のも分かる。
  \begin{align}
    E_{(n_{i, a})_{i=1,\ldots,N,a=x,y,z}}    & = E_0\sum_{i=1}^{N}\sum_{a=x,y,z}n_{i,a}^2                                                \\
    \psi_{(n_{i, a})_{i=1,\ldots,N,a=x,y,z}} & = \qty(\frac{2}{L})^{3N/2}\prod_{i=1}^{N}\prod_{a=x,y,z}\sin(\frac{n_{i,a}\pi}{L}x_{i,a})
  \end{align}
  これよりあるエネルギー $E > 0$ 以下である区別できる固有状態数 $\Omega(E)$ について
  \begin{align}
    \Omega(E) & = \qty(半径 \sqrt{\frac{E}{E_0}} の 3N 次元超球の第一象限に含まれる格子点の個数)                                                         \\
              & \approx \frac{1}{2^{3N}}\sqrt{\frac{E}{E_0}}^{3N}\frac{\pi^{3N/2}}{(3N/2)!}                                       \\
              & = \frac{1}{(3N/2)!}\frac{\pi^{3N/2}}{2^{3N}}\qty(\frac{2mL^2}{\pi^2\hbar^2})^{3N/2}E^{3N/2}                       \\
              & = \frac{1}{(3N/2)!}\qty(\frac{m}{2\pi\hbar^2})^{3N/2}E^{3N/2}V^N                                                  \\
              & = \frac{1}{\sqrt{3\pi N}(3N/2)^{3N/2}e^{-3N/2}}\qty(\frac{m}{2\pi\hbar^2})^{3N/2}E^{3N/2}V^N                      \\
              & = \frac{1}{\sqrt{3\pi N}}N^{N}\qty(\frac{me}{3\pi\hbar^2})^{3N/2}\qty(\frac{E}{V})^{3N/2}\qty(\frac{V}{N})^{5N/2} \\
  \end{align}
  これを区別しないから
  \begin{align}
    \Omega^{区別できない}(E) & =\frac{1}{N!}\Omega(E)                                                                                                                            \\
                       & = \frac{1}{N!}\frac{1}{\sqrt{3\pi N}}N^{N}\qty(\frac{me}{3\pi\hbar^2})^{3N/2}\qty(\frac{E}{V})^{3N/2}\qty(\frac{V}{N})^{5N/2}                     \\
                       & = \frac{1}{\sqrt{2\pi N}N^Ne^{-N}}\frac{1}{\sqrt{3\pi N}}N^{N}\qty(\frac{me}{3\pi\hbar^2})^{3N/2}\qty(\frac{E}{V})^{3N/2}\qty(\frac{V}{N})^{5N/2} \\
                       & = \frac{e^N}{\sqrt{6}\pi N}\qty(\frac{me}{3\pi\hbar^2})^{3N/2}\qty(\frac{E}{V})^{3N/2}\qty(\frac{V}{N})^{5N/2}
  \end{align}
  これよりエントロピーは
  \begin{align}
    S(E) & = k_B\ln\Omega^{区別できない}(E)                                                                                                                     \\
         & = Nk_B\qty(\frac{3}{2}\ln\frac{E}{V} + \frac{5}{2}\ln\frac{V}{N} + \frac{3}{2}\ln(\frac{me}{3\pi\hbar^2}) - \frac{1}{N}\ln(\sqrt{6}\pi N) + 1)
  \end{align}
  よって温度を計算すると式が示せる。
  \begin{align}
    \frac{1}{T} = \dv{S}{E} & = \frac{3}{2}Nk_B\frac{1}{E} \\
    E                       & = \frac{3}{2}Nk_BT
  \end{align}
\end{proof}

\section{ミクロカノニカル分布}
\subsection{エネルギーシェル}
\begin{axiom}[等重率の原理]
  孤立した物理系 $X$ において、外部から指定されたある狭いエネルギー範囲 $[U - ∆U, U]$ に固有エネルギー $E_i$ が属するような微視的なエネルギー固有状態 $\ket{\phi_i}$ のひとつひとつが実現される等しい確からしさを持っている。
\end{axiom}
エネルギーの低い順にエネルギーシェル $E$ から $E + \Delta E$ までの中の状態を 1 つのグループでまとめてラベル付けする。
\begin{align}
  N = \sum_{l}N_l, \qquad E = \sum_{l}E_lN_l, \qquad W = \prod_{l}\frac{M_l^{N_l}}{N_l!}, \qquad S = k_B\sum_{l}N_l\qty(\log\frac{M_l}{N_l} + 1)
\end{align}

\begin{align}
  \tilde{S}            & = k_B\sum_{l}N_l\qty(\log\frac{M_l}{N_l} + 1) - k_B\alpha\sum_{l}N_l - k_B\beta\sum_{l}E_lN_l \\
  \pdv{\tilde{S}}{N_l} & = 0 \iff \frac{M_l}{N_l} = e^{\alpha + \beta E_l}
\end{align}

\begin{align}
  N & = \sum_{l}M_le^{-\alpha-\beta E_l}    \\
  E & = \sum_{l}M_lE_le^{-\alpha-\beta E_l} \\
  S & = k_B\qty((1 + \alpha)N + \beta E)
\end{align}
エネルギーで微分すると
\begin{align}
  0         & = \sum_{l}M_l\qty(\dv{\alpha}{E} + \dv{\beta}{E}E_l)e^{-\alpha-\beta E_l} = \dv{\alpha}{E}N + \dv{\beta}{E}E \\
  \dv{S}{E} & = k_B\qty(\dv{\alpha}{E}N + \dv{\beta}{E}E + \beta) = k_B\beta
\end{align}
より $\alpha, \beta$ は次のように表される。
\begin{align}
  \beta       & = \frac{1}{k_BT}                            \\
  e^{-\alpha} & = \frac{N}{\sum_{i}e^{-\varepsilon_i/k_BT}}
\end{align}


\subsection{熱と仕事}
\begin{definition}
  内部エネルギー $E(S, V)$ とその束縛変数を変更させたエンタルピー $H(S, p)$ と Helmholtz 自由エネルギー $F(T, V)$ と Gibbs 自由エネルギー $G(T, p)$ を次のように定義する。
  \begin{align}
               & \qquad \dd{E} = T\dd{S} - p\dd{V}  \\
    H = E + pV & \qquad \dd{H} = T\dd{S} + V\dd{p}  \\
    F = E - TS & \qquad \dd{F} = -S\dd{T} - p\dd{V} \\
    G = F + pV & \qquad \dd{G} = -S\dd{T} + V\dd{p}
  \end{align}
\end{definition}
特に扱いやすい変数 $T$, $V$ を持つ Helmholtz 自由エネルギー $F(T, V)$ は重宝される。
\begin{proposition}
  定義より次の関係式を満たす。
  \begin{align}
    T = \qty(\pdv{E}{S})_V,  & \qquad p = -\qty(\pdv{E}{V})_S \\
    T = \qty(\pdv{H}{S})_p,  & \qquad V = \qty(\pdv{H}{p})_S  \\
    S = -\qty(\pdv{F}{T})_V, & \qquad p = -\qty(\pdv{F}{V})_T \\
    S = -\qty(\pdv{G}{T})_p, & \qquad V = \qty(\pdv{G}{p})_T
  \end{align}
\end{proposition}

\begin{proposition}[Maxwell の関係式]
  \begin{align}
    \qty(\pdv{T}{V})_S  & = -\qty(\pdv{p}{S})_V \\
    \qty(\pdv{T}{p})_S  & = \qty(\pdv{V}{S})_p  \\
    -\qty(\pdv{S}{V})_T & = -\qty(\pdv{p}{T})_V \\
    -\qty(\pdv{S}{p})_T & = \qty(\pdv{V}{T})_p
  \end{align}
\end{proposition}
\begin{proof}

\end{proof}

\begin{theorem}[理想気体の状態方程式]
  \begin{align}
    pV & = Nk_BT
  \end{align}
\end{theorem}
\begin{align}
  S(E, V) & = Nk_B\qty(\frac{3}{2}\ln\frac{E}{V} + \frac{5}{2}\ln\frac{V}{N} + \frac{3}{2}\ln(\frac{me}{3\pi\hbar^2}) - \frac{1}{N}\ln(\sqrt{6}\pi N) + 1) \\
  0       & = Nk_B\qty(\frac{3}{2}\frac{1}{E}\qty(\pdv{E}{V})_S + \frac{1}{V})                                                                             \\
  p       & = -\qty(\pdv{E}{V})_S = \frac{2}{3}\frac{E}{V} = \frac{Nk_BT}{V}                                                                               \\
  pV      & = Nk_BT
\end{align}

\begin{align}
  C_V = \frac{3}{2}R
\end{align}

\begin{definition}[比熱]
  \begin{align}
    C   & = T\dv{S}{T}          \\
    C_X & = \qty(T\pdv{S}{T})_X
  \end{align}
\end{definition}
これ以降の話は熱力学の方で書きたい。

\section{カノニカル分布}
ある温度の環境の中で理想気体や
\subsection{ミクロカノニカル分布からカノニカル分布へ}

\begin{definition}
  \begin{align}
    \beta    & = \frac{1}{k_BT}                  \\
    Z(\beta) & = \sum_{i}e^{-\beta E_i}          \\
    p        & = \frac{e^{-\beta E_i}}{Z(\beta)} \\
    F        & = -k_BT\ln Z                      \\
    S        & =
  \end{align}
\end{definition}

\begin{theorem}
  $N$ 個の独立な部分系からなる全体系の熱力学量は次のようになる。
  \begin{align}
    Z(\beta) = z(\beta)^N, \qquad F = Nf, \qquad S = Ns, \qquad U = Nu, \qquad C = c
  \end{align}
\end{theorem}

\subsection{二準位系}
絶対温度 $T$ の熱浴に系 $X$ が浸けられている状態として、系 $X$ の Hamilton 演算子 $\hat{h}_X$ の固有状態は $\ket{\varphi_1}$ と $\ket{\varphi_2}$ の 2 つだけであり、$\ket{\varphi_1}$ の固有エネルギーは $E_1$ であり、$\ket{\varphi_2}$ の固有エネルギーは $E_2$ であるとする:
\begin{align}
  \hat{h}_X\ket{\varphi_i} & = E_i\ket{\varphi_i} \qquad (i = 1, 2)。
\end{align}
ただし $0 < E_1 < E_2$ $\beta = 1/k_BT$ とする。

\begin{theorem}[1個の二準位系]
  二準位系における熱力学的量を考える。低温の漸近領域 ($\beta(E_2 - E_1) \gg 1$, $\beta E_1 \gg 1$) と高温の漸近領域 ($\beta(E_2 - E_1) \ll 1$, $\beta E_1 \ll 1$) は次のようになる。
  \begin{align}
    Z(\beta) & = e^{-\beta E_1} + e^{-\beta E_2}                                                                                                           \\
             & = \begin{dcases}
                   e^{-\frac{E_1}{k_BT}} \to 0      & (低温) \\
                   2 - \frac{E_1 + E_2}{k_BT} \to 2 & (高温)
                 \end{dcases}                                                                                                   \\
    F        & = -\frac{1}{\beta}\ln(e^{-\beta E_1} + e^{-\beta E_2})                                                                                      \\
             & = \begin{dcases}
                   E_1 - k_BTe^{-\frac{E_2 - E_1}{k_BT}} \to E_1  & (低温) \\
                   \frac{1}{2}(E_1 + E_2) - k_BT\ln 2 \to -\infty & (高温)
                 \end{dcases}                                                                                     \\
    S        & = k_B\qty(\ln(e^{-\beta E_1} + e^{-\beta E_2}) + \frac{\beta E_1e^{-\beta E_1} + \beta E_2e^{-\beta E_2}}{e^{-\beta E_1} + e^{-\beta E_2}}) \\
             & = \begin{dcases}
                   k_B \frac{E_2 - E_1}{k_BT}e^{- \frac{E_2 - E_1}{k_BT}} \to 0            & (低温) \\
                   k_B\qty(\ln 2 - \frac{1}{4}\qty(\frac{E_2 - E_1}{k_BT})^2) \to k_B\ln 2 & (高温)
                 \end{dcases}                                                     \\
    U        & = \frac{E_1e^{-\beta E_1} + E_2e^{-\beta E_2}}{e^{-\beta E_1} + e^{-\beta E_2}}                                                             \\
             & = \begin{dcases}
                   E_1 + (E_2 - E_1)e^{- \frac{E_2 - E_1}{k_BT}} \to E_1                                     & (低温) \\
                   \frac{1}{2}(E_1 + E_2) - \frac{1}{4}\frac{(E_2 - E_1)^2}{k_BT} \to \frac{1}{2}(E_1 + E_2) & (高温)
                 \end{dcases}                                         \\
    C        & = k_B\qty(\frac{\frac{1}{2}\beta(E_2 - E_1)}{\cosh\frac{1}{2}\beta(E_2 - E_1)})^2                                                           \\
             & = \begin{dcases}
                   k_B\qty(\frac{E_2 - E_1}{k_BT})^2e^{-\frac{E_2 - E_1}{k_BT}} \to 0 & (低温) \\
                   \frac{k_B}{4}\qty(\frac{E_2 - E_1}{k_BT})^2 \to 0                  & (高温)
                 \end{dcases}
  \end{align}
  TODO: グラフ
\end{theorem}
\begin{proof}
  $x \to 0$ において $(1 + x)^{-1} \approx 1 - x$, $e^x \approx 1 + x$ と近似できる。
  \begin{align}
    Z(\beta) & = \sum_{i} e^{-\beta E_i} = e^{-\beta E_1} + e^{-\beta E_2} \\
             & = e^{-\beta E_1}(1 + e^{-\beta (E_2 - E_1)})                \\
             & \approx \begin{dcases}
                         e^{-\frac{E_1}{k_BT}}                 \\
                         (1 - \beta E_1)(2 -\beta (E_2 - E_1)) \\
                       \end{dcases}               \\
             & = \begin{dcases}
                   e^{-\frac{E_1}{k_BT}} \to 0      & (低温) \\
                   2 - \frac{E_1 + E_2}{k_BT} \to 2 & (高温)
                 \end{dcases}
  \end{align}
  \begin{align}
    F & = -k_BT\ln Z(\beta)                                                                                                                                                                \\
      & = -\frac{1}{\beta}\ln(e^{-\beta E_1} + e^{-\beta E_2})                                                                                                                             \\
      & = \begin{dcases}
            -\frac{1}{\beta}\ln e^{-\beta E_1}(1 + e^{-\beta (E_2 - E_1)}) \\
            -\frac{1}{\beta}\ln e^{-\frac{1}{2}\beta (E_1 + E_2)}(e^{\frac{1}{2}\beta (E_2 - E_1)} + e^{-\frac{1}{2}\beta (E_2 - E_1)})
          \end{dcases} \\
      & \approx \begin{dcases}
                  E_1 - \frac{1}{\beta}e^{-\beta (E_2 - E_1)} \approx E_1 - k_BTe^{-\frac{E_2 - E_1}{k_BT}} \to E_1 \\
                  \frac{1}{2}(E_1 + E_2) - k_BT\ln 2 \to -\infty                                                    \\
                \end{dcases}
  \end{align}
  \begin{align}
    S & = - \qty(\pdv{F}{T})_{V,N} = - \qty(\pdv{F}{\beta}\pdv{\beta}{T})_{V,N} = k_B\beta^2\qty(\pdv{F}{\beta})_{V,N}                                                                                                                                                                                                                                                                         \\
      & = k_B\beta^2\qty(\frac{1}{\beta^2}\ln(e^{-\beta E_1} + e^{-\beta E_2}) - \frac{1}{\beta}\frac{-E_1e^{-\beta E_1} - E_2e^{-\beta E_2}}{e^{-\beta E_1} + e^{-\beta E_2}})                                                                                                                                                                                                                \\
      & = k_B\qty(\ln(e^{-\beta E_1} + e^{-\beta E_2}) + \frac{\beta E_1e^{-\beta E_1} + \beta E_2e^{-\beta E_2}}{e^{-\beta E_1} + e^{-\beta E_2}})                                                                                                                                                                                                                                            \\
      & = \begin{dcases}
            k_B\qty(\ln e^{-\beta E_1}(1 + e^{-\beta (E_2 - E_1)}) + \frac{\beta E_1 + \beta E_2e^{-\beta (E_2 - E_1)}}{1 + e^{-\beta (E_2 - E_1)}})                                                                                                                                                   \\
            k_B\qty(\ln e^{-\frac{1}{2}\beta (E_1 + E_2)}(e^{\frac{1}{2}\beta (E_2 - E_1)} + e^{-\frac{1}{2}\beta (E_2 - E_1)}) + \frac{\beta E_1e^{\frac{1}{2}\beta (E_2 - E_1)} + \beta E_2e^{-\frac{1}{2}\beta (E_2 - E_1)}}{e^{\frac{1}{2}\beta (E_2 - E_1)} + e^{-\frac{1}{2}\beta (E_2 - E_1)}}) \\
          \end{dcases} \\
      & \approx
    \begin{dcases}
      k_B\qty(-\beta E_1 + e^{-\beta (E_2 - E_1)} + (\beta E_1 + \beta E_2e^{-\beta (E_2 - E_1)})(1 - e^{-\beta (E_2 - E_1)}))                                   \\
      k_B\qty(\ln 2 - \frac{1}{2}\beta (E_1 + E_2) + \frac{\beta}{2}\qty(E_1\qty(1 + \frac{1}{2}\beta (E_2 - E_1)) + E_2\qty(1 - \frac{1}{2}\beta (E_2 - E_1)))) \\
    \end{dcases}                                                                                                                                                                                                                     \\
      & \approx
    \begin{dcases}
      k_B \frac{E_2 - E_1}{k_BT}e^{- \frac{E_2 - E_1}{k_BT}} \to 0            \\
      k_B\qty(\ln 2 - \frac{1}{4}\qty(\frac{E_2 - E_1}{k_BT})^2) \to k_B\ln 2 \\
    \end{dcases}
  \end{align}
  \begin{align}
    U & = F + TS                                                                                                                                                                                      \\
      & = \frac{E_1e^{-\beta E_1} + E_2e^{-\beta E_2}}{e^{-\beta E_1} + e^{-\beta E_2}}                                                                                                               \\
      & = \begin{dcases}
            \frac{E_1 + E_2e^{-\beta (E_2 - E_1)}}{1 + e^{-\beta (E_2 - E_1)}}                                                                                      \\
            \frac{E_1e^{\frac{1}{2}\beta (E_2 - E_1)} + E_2e^{-\frac{1}{2}\beta (E_2 - E_1)}}{e^{\frac{1}{2}\beta (E_2 - E_1)} + e^{-\frac{1}{2}\beta (E_2 - E_1)}} \\
          \end{dcases} \\
      & \approx \begin{dcases}
                  (E_1 + E_2e^{-\beta (E_2 - E_1)})(1 - e^{-\beta (E_2 - E_1)})                                          \\
                  \frac{1}{2}\qty(E_1\qty(1 + \frac{1}{2}\beta (E_2 - E_1)) + E_2\qty(1 - \frac{1}{2}\beta (E_2 - E_1))) \\
                \end{dcases}                                                                          \\
      & \approx \begin{dcases}
                  E_1 + (E_2 - E_1)e^{- \frac{E_2 - E_1}{k_BT}} \to E_1                                     \\
                  \frac{1}{2}(E_1 + E_2) - \frac{1}{4}\frac{(E_2 - E_1)^2}{k_BT} \to \frac{1}{2}(E_1 + E_2) \\
                \end{dcases}
  \end{align}
  \begin{align}
    C & = \pdv{U}{T} = \pdv{U}{\beta}\pdv{\beta}{T} = -k_B\beta^2\pdv{U}{\beta}                                                                                                         \\
      & = -k_B\beta^2\pdv{\beta}\qty(\frac{E_1 + E_2e^{\beta(E_1 - E_2)}}{1 + e^{\beta(E_1 - E_2)}})                                                                                    \\
      & = -k_B\beta^2\frac{E_2(E_1 - E_2)e^{\beta(E_1 - E_2)}(1 + e^{\beta(E_1 - E_2)}) - (E_1 + E_2e^{\beta(E_1 - E_2)})(E_1 - E_2)e^{\beta(E_1 - E_2)}}{(1 + e^{\beta(E_1 - E_2)})^2} \\
      & = k_B\beta^2\frac{(E_2 - E_1)^2e^{\beta(E_1 - E_2)}}{(1 + e^{\beta(E_1 - E_2)})^2} = k_B\qty(\frac{\frac{1}{2}\beta(E_2 - E_1)}{\cosh\frac{1}{2}\beta(E_2 - E_1)})^2            \\
      & = \begin{dcases}
            k_B\qty(\frac{\beta(E_2 - E_1)}{1 + e^{-\beta(E_2 - E_1)}})^2e^{-\beta(E_2 - E_1)} \\
            k_B\qty(\frac{\beta(E_2 - E_1)}{e^{\frac{1}{2}\beta(E_2 - E_1)} + e^{-\frac{1}{2}\beta(E_2 - E_1)}})^2
          \end{dcases}                                                         \\
      & \approx \begin{dcases}
                  k_B\qty(\frac{E_2 - E_1}{k_BT})^2e^{-\frac{E_2 - E_1}{k_BT}} \to 0 \\
                  \frac{k_B}{4}\qty(\frac{E_2 - E_1}{k_BT})^2 \to 0
                \end{dcases}
  \end{align}
  各固有状態の実現確率について高温極限 ($\beta(E_2 - E_1) \ll 1$) のときそれぞれの固有状態は同じ確率で実現し、低温極限 ($\beta(E_2 - E_1) \gg 1$) のとき固有エネルギーの低い固有状態にほぼ確実に実現する。
  \begin{align}
    \quad p_\beta(i) & = \frac{e^{-\beta (E_i - E_1)}}{1 + e^{-\beta(E_2 - E_1)}} \approx \begin{dcases}
                                                                                            e^{-\beta (E_i - E_1)} & (\beta(E_2 - E_1) \gg 1) \\
                                                                                            \frac{1}{2}            & (\beta(E_2 - E_1) \ll 1)
                                                                                          \end{dcases}
  \end{align}
  $F = E - TS$ の最小化を考える。低温極限でエントロピーを上げるよりエネルギーが低いものを選んだ方がエネルギーが得となる為に固有エネルギーの低い状態に集まる。高温極限でエントロピーを増大させるとエネルギーが得となる為に半々となる。
\end{proof}

\begin{itembox}[l]{Q 15-2.}
  Q 15-1.では Helmholtz 自由エネルギーを計算して、後は熱力学の公式を用いて計算しましたが、今回は正準集団の理論における固有状態の実現確率を与える確率関数 $p_\beta^{正準}(i)\ (i = 1, 2)$ を計算して、内部エネルギー $u$ とエントロピー $s$ を求める。
\end{itembox}
まず確率関数 $p_\beta(i)$ は定義より次のようになる。
\begin{align}
  p_\beta(i) & = \frac{e^{-\beta E_i}}{z(\beta)}.
\end{align}
内部エネルギー $u$ はエネルギーの平均を取ることで分かる。
\begin{align}
  u & = \sum_i E_ip_\beta^{正準}(i) = \frac{E_1e^{-\beta E_1} + E_2e^{-\beta E_2}}{e^{-\beta E_1} + e^{-\beta E_2}}.
\end{align}
比熱も Q15-1.と同様に求まる。

エントロピー $s$ は Shannon のエントロピーの公式に代入することで求まる。
\begin{align}
  s & = -k_B\sum_{i = 1,2}p_\beta^{正準}(i)\ln p_\beta^{正準}(i)                                                                                       \\
    & = -k_B\sum_{i = 1,2}\frac{e^{-\beta E_i}}{z(\beta)}(- \ln z(\beta) - \beta E_i)                                                              \\
    & = k_B\qty(\ln(e^{-\beta E_1} + e^{-\beta E_2}) + \frac{\beta E_1e^{-\beta E_1} + \beta E_2e^{-\beta E_2}}{e^{-\beta E_1} + e^{-\beta E_2}}).
\end{align}

\begin{itembox}[l]{Q 15-7.}
  比熱について解析せよ。
\end{itembox}

まず比熱について次のように定義した関数 $\phi(x)$ を用いて表される。
\begin{align}
  \phi(x) & := \frac{x}{\cosh x}                                                              \\
  c       & = k_B\qty(\frac{\frac{1}{2}\beta(E_2 - E_1)}{\cosh\frac{1}{2}\beta(E_2 - E_1)})^2 \\
          & = k_B\qty(\phi\qty(\frac{1}{2}\beta(E_2 - E_1)))^2
\end{align}
ここで $x\geq 0$ の範囲において $\phi(x)$ が極大となる $x = x_0$ の値を考える。
\begin{align}
       & \left.\dv{\phi}{x}\right|_{x = x_0} = 0          \\
  \iff & \frac{\cosh x_0 - x_0\sinh x_0}{\cosh^2 x_0} = 0 \\
  \iff & x_0\tanh x_0 = 1                                 \\
  \iff & x_0 = 1.199678640257734\ldots
\end{align}

ただしプログラム \ref{newton} を用いて $x\geq 0$ の範囲で $x_0\tanh x_0 = 1$ は $x_0 = 1.199678640257734\ldots$ のとき満たすことが分かる。これより比熱 $c$ は次のように定義される $T_0$ のときに極大を取る。
\begin{align}
   & x_0 = \frac{1}{2}\beta_0(E_2 - E_1) = \frac{1}{2}\frac{E_2 - E_1}{k_BT_0} \\
   & \frac{k_BT_0}{E_2 - E_1} = \frac{1}{2x_0} =  0.41677827980048\ldots
\end{align}

低温、高温で比熱が 0 となる理由は比熱が $C = \dv{E}{T}$ であることより Q15-3, Q15-4 よりエネルギーの確率が極限的に定数となることから比熱は 0 となることが分かる。

\subsection{調和振動子系の統計力学}

\begin{theorem}
  \begin{align}
    z(\beta) & = \frac{1}{2\sinh \frac{1}{2}\beta\hbar\omega}                                                                       \\
             & \approx \begin{dcases}
                         e^{-\frac{\hbar\omega}{2k_BT}} \to 0 \\
                       \end{dcases}                                                                         \\
    f        & = \frac{1}{\beta}\ln\qty(2\sinh \frac{1}{2}\beta\hbar\omega)                                                         \\
    s        & = k_B\qty(-\ln\qty(2\sinh\frac{1}{2}\beta\hbar\omega) + \frac{1}{2}\beta\hbar\omega\coth\frac{1}{2}\beta\hbar\omega) \\
    u        & = \frac{1}{2}\hbar\omega\coth\frac{1}{2}\beta\hbar\omega                                                             \\
    c        & = k_B\qty(\frac{\frac{1}{2}\beta\hbar\omega}{\sinh\frac{1}{2}\beta\hbar\omega})^2
  \end{align}
\end{theorem}
\begin{proof}
  低温の漸近領域において Q 16-1 の結果は次のように近似できる。ただし、$x\to 0$ のとき $e^x \approx 1 + x$, $(1 + x)^{-1} \approx 1 - x$ と近似できることを用いる。
  高温の漸近領域において Q 16-1 の結果は次のように近似できる。ただし、$x\to 0$ のとき $\ln(1 + x) \approx x$ と近似できることとテイラー展開を用いる。
  \begin{align}
    z(\beta) & = \sum_{i = 0}^{\infty}e^{-\beta E_i} = \sum_{i = 0}^{\infty}e^{-\beta\qty(n + \frac{1}{2})\hbar\omega} = \frac{e^{-\frac{1}{2}\beta\hbar\omega}}{1 - e^{-\beta\hbar\omega}} = \frac{1}{2\sinh \frac{1}{2}\beta\hbar\omega} \\
             & = \begin{dcases}
                   \frac{e^{-\frac{1}{2}\beta\hbar\omega}}{1 - e^{-\beta\hbar\omega}} \\
                   \frac{e^{-\frac{1}{2}\beta\hbar\omega}}{1 - e^{-\beta\hbar\omega}} \\
                 \end{dcases}                                                                                                                                                 \\
             & \approx \begin{dcases}
                         e^{-\frac{\hbar\omega}{2k_BT}} \to 0                                                                                \\
                         \frac{1 - \frac{1}{2}\beta\hbar\omega}{\beta\hbar\omega} \approx \frac{k_BT}{\hbar\omega} - \frac{1}{2} \to +\infty \\
                       \end{dcases}
  \end{align}
  \begin{align}
    f & = -k_BT\ln z(\beta)                                                                                                                                               \\
      & = -\frac{1}{\beta}\ln \frac{e^{-\frac{1}{2}\beta\hbar\omega}}{1 - e^{-\beta\hbar\omega}} = \frac{1}{\beta}\ln(1 - e^{-\beta\hbar\omega}) + \frac{1}{2}\hbar\omega \\
      & = -\frac{1}{\beta}\ln \frac{1}{2\sinh \frac{1}{2}\beta\hbar\omega} = \frac{1}{\beta}\ln(2\sinh \frac{1}{2}\beta\hbar\omega)                                       \\
      & \approx \begin{dcases}
                  \frac{1}{\beta}(-e^{-\beta\hbar\omega}) + \frac{1}{2}\hbar\omega        \\
                  \frac{1}{\beta}\qty(\ln \beta\hbar\omega + \frac{1}{2}\beta\hbar\omega) \\
                \end{dcases}                                                                                   \\
      & \approx \begin{dcases}
                  \frac{1}{2}\hbar\omega - k_BTe^{-\frac{\hbar\omega}{k_BT}} \to \frac{1}{2}\hbar\omega \\
                  -k_BT\ln\frac{k_BT}{\hbar\omega} \to - \infty                                         \\
                \end{dcases}
  \end{align}
  \begin{align}
    s & = - \qty(\pdv{f}{T})_{V,N} = k_B\beta^2\qty(\pdv{f}{\beta})_{V,N}                                                                                                                                                   \\
      & = k_B\beta^2\qty(-\frac{1}{\beta^2}\ln(1 - e^{-\beta\hbar\omega}) + \frac{\hbar\omega}{\beta(1 - e^{-\beta\hbar\omega})})                                                                                           \\
      & = k_B\beta^2\qty(-\frac{1}{\beta^2}\ln\qty(2\sinh \frac{1}{2}\beta\hbar\omega) + \frac{\frac{1}{2}\hbar\omega\cosh \frac{1}{2}\beta\hbar\omega}{\beta\sinh \frac{1}{2}\beta\hbar\omega})                            \\
      & = k_B\qty(-\ln(1 - e^{-\beta\hbar\omega}) + \frac{\beta\hbar\omega}{e^{\beta\hbar\omega} - 1}) = k_B\qty(-\ln\qty(2\sinh\frac{1}{2}\beta\hbar\omega) + \frac{1}{2}\beta\hbar\omega\coth\frac{1}{2}\beta\hbar\omega) \\
      & \approx \begin{dcases}
                  k_B\qty(e^{-\beta\hbar\omega} + \beta\hbar\omega e^{-\beta\hbar\omega}(1 + e^{-\beta\hbar\omega})) \\
                  k_B\qty(-\ln\beta\hbar\omega + 1)                                                                  \\
                \end{dcases}                                                                                                          \\
      & \approx \begin{dcases}
                  k_B\frac{\hbar\omega}{k_BT}e^{- \frac{\hbar\omega}{k_BT}} \to 0 \\
                  k_B\ln\frac{k_BT}{\hbar\omega} \to +\infty                      \\
                \end{dcases}
  \end{align}
  \begin{align}
    u & = f + Ts                                                                                                                                                                                      \\
      & = \frac{1}{\beta}\ln(1 - e^{-\beta\hbar\omega}) + \frac{1}{2}\hbar\omega + \frac{1}{\beta}\qty(-\ln(1 - e^{-\beta\hbar\omega}) + \frac{\beta\hbar\omega}{e^{\beta\hbar\omega} - 1})           \\
      & = \frac{1}{\beta}\ln\qty(2\sinh \frac{1}{2}\beta\hbar\omega) + \frac{1}{\beta}\qty(-\ln\qty(2\sinh\frac{1}{2}\beta\hbar\omega) + \frac{1}{2}\beta\hbar\omega\coth\frac{1}{2}\beta\hbar\omega) \\
      & = \qty(\frac{1}{2} + \frac{1}{e^{\beta\hbar\omega} - 1})\hbar\omega = \frac{1}{2}\hbar\omega\coth\frac{1}{2}\beta\hbar\omega                                                                  \\
      & \approx \begin{dcases}
                  \qty(\frac{1}{2} + e^{-\beta\hbar\omega}(1 + e^{-\beta\hbar\omega}))\hbar\omega                                          \\
                  \frac{1}{2}\hbar\omega\qty(\qty(\frac{\beta\hbar\omega}{2})^{-1} + \frac{1}{3}\qty(\frac{\beta\hbar\omega}{2}) + \cdots) \\
                \end{dcases}                                                              \\
      & \approx \begin{dcases}
                  \frac{1}{2}\hbar\omega + e^{-\beta\hbar\omega}\hbar\omega \to \frac{1}{2}\hbar\omega \\
                  k_BT\qty(1 + \frac{1}{12}\qty(\frac{\hbar\omega}{k_BT})^{2} + \cdots) \to +\infty    \\
                \end{dcases}
  \end{align}
  \begin{align}
    c & = \pdv{u}{T} = -k_B\beta^2\pdv{u}{\beta} = -k_B\beta^2\qty(-\frac{\hbar\omega e^{\beta\hbar\omega}}{(e^{\beta\hbar\omega} - 1)^2})\hbar\omega                                   \\
      & = k_B\qty(\beta\hbar\omega\frac{e^{\frac{1}{2}\beta\hbar\omega}}{e^{\beta\hbar\omega} - 1})^2 = k_B\qty(\frac{\frac{1}{2}\beta\hbar\omega}{\sinh\frac{1}{2}\beta\hbar\omega})^2 \\
      & \approx \begin{dcases}
                  k_B\qty(\beta\hbar\omega(e^{-\frac{1}{2}\beta\hbar\omega})(1 + e^{-\beta\hbar\omega}))^2                                                \\
                  k_B\qty(\frac{\beta\hbar\omega}{2}\qty(\qty(\frac{\beta\hbar\omega}{2})^{-1} - \frac{1}{6}\qty(\frac{\beta\hbar\omega}{2}) + \cdots))^2 \\
                \end{dcases}                                                                             \\
      & \approx \begin{dcases}
                  k_B\qty(\frac{\hbar\omega}{k_BT})^2e^{-\frac{\hbar\omega}{k_BT}} \to 0 \\
                  k_B\qty(1 - \frac{1}{24}\qty(\frac{\hbar\omega}{k_BT}) + \cdots)^2 = k_B\qty(1 - \frac{1}{12}\qty(\frac{\hbar\omega}{k_BT}) + \cdots) \to k_B
                \end{dcases}
  \end{align}
\end{proof}


\subsection{固体の比熱の Einstein 模型}
\begin{itembox}[l]{Q 16-5.}
  独立な調和振動子の集まりの系として記述される系 $X$ において$\dd{\omega}$ が十分小さいとして、角振動数が $\omega$ から $\omega + \dd{\omega}$ の範囲にある調和振動子の個数を $g(\omega)\dd{\omega}$ と定義する。つまり $g(\omega)$ は調和振動子の角振動数の個数分布関数である。
\end{itembox}
このとき角運動量が $\omega$ である調和振動子 1 個の Helmholtz 自由エネルギー, エントロピー, 内部エネルギー, 比熱をそれぞれ $f(\omega), s(\omega), u(\omega), c(\omega)$ と書くこととすると、$\dd{\omega}$ が十分小さいことから近い角運動量の変数を個数倍して積分することで元の変数と一致する。これより次のような式が成り立つ。
\begin{align}
  F & = \int_0^\infty\dd{\omega}g(\omega)f(\omega) \\
  S & = \int_0^\infty\dd{\omega}g(\omega)s(\omega) \\
  U & = \int_0^\infty\dd{\omega}g(\omega)u(\omega) \\
  C & = \int_0^\infty\dd{\omega}g(\omega)c(\omega)
\end{align}

\begin{itembox}[l]{Q 16-6.}
  ある元素の原子 $n$ [\si{mol}] からなる個体を考える。Einstein 模型では、結晶を構成するそれぞれの原子は平衡位置の回りに独立に同一の角振動数 $\omega_E$ を持って調和振動すると考える。ここで次の観測結果に対して Einstein 模型は妥当性があることを説明せよ。
  \begin{enumerate}
    \item (高温での固体の比熱の振る舞い : Dulong-Petit の法則) 十分に高温では、$n$ [\si{mol}] の固体の比熱 $C$ は、固体を構成する物質によらずに、$3nR$ の一定値を取る。ここで、$R = 8.314\ldots$ [\si{J/(mol\cdot K)}] は気体定数である。
    \item (低温での固体の比熱の大雑把な振る舞い) 温度 $T$ が $0$ に近付くとき、固体の比熱 $C$ は小さくなっていく。温度 $T$ が $0$ に近付く極限では、比熱 $C$ はゼロになるようだ。
  \end{enumerate}
\end{itembox}
調和振動子の角振動数の個数について、各原子の自由度が $3$ であるから Avogadro 数 $N_A = 6.02\ldots\times 10^{23}$ [\si{1/mol}] を用いて全体の個数は $3N = 3nN_A$ であることが分かる。
これより Einstein 模型における調和振動子の角振動数の個数分布関数 $g(\omega)$ は次のように表される。
\begin{align}
  g(\omega) = 3N\delta(\omega - \omega_E).
\end{align}

これより比熱は次のように表される。
\begin{align}
  C & = \int_0^\infty\dd{\omega}g(\omega)c(\omega)                                            \\
    & = \int_0^\infty\dd{\omega}3N\delta(\omega - \omega_E)c(\omega)                          \\
    & = 3Nc(\omega_E)                                                                         \\
    & = 3Nk_B\qty(\frac{\frac{1}{2}\beta\hbar\omega_E}{\sinh\frac{1}{2}\beta\hbar\omega_E})^2
\end{align}
高温の漸近領域において比熱 $C$ は次のようになる。
\begin{align}
  C & = 3Nk_B\qty(\frac{\frac{1}{2}\beta\hbar\omega_E}{\sinh\frac{1}{2}\beta\hbar\omega_E})^2 \\
    & \approx 3Nk_B\qty(1 - \frac{1}{12}\qty(\frac{\hbar\omega}{k_BT})^2 + \cdots)            \\
    & \approx 3nR
\end{align}
低温の漸近領域において比熱 $C$ は次のようになる。
\begin{align}
  C & = 3Nk_B\qty(\frac{\frac{1}{2}\beta\hbar\omega_E}{\sinh\frac{1}{2}\beta\hbar\omega_E})^2 \\
    & \approx 3Nk_B\qty(\frac{\hbar\omega}{k_BT})^2e^{-\frac{\hbar\omega}{k_BT}}              \\
    & \approx 3nR\qty(\frac{\hbar\omega}{k_BT})^2e^{-\frac{\hbar\omega}{k_BT}}
\end{align}
よって低温領域で温度 $T$ が小さくなっていくとき、比熱 $C$ が小さくなる。
\begin{align}
  \lim_{T\to 0} C & = \lim_{T\to 0}3nR\qty(\frac{\hbar\omega}{k_BT})^2e^{-\frac{\hbar\omega}{k_BT}} = 0.
\end{align}
これらの結果は観測結果と一致している為、妥当性がある。

\begin{itembox}[l]{Q 16-7.}
  固体の比熱の Einstein 模型は次の実験事実と合致しないことを確認せよ。
  \begin{enumerate}
    \item (低温での固体の比熱の精密な振る舞い) 温度 $T$ が $0$ に近付くとき、固体の比熱 $C$ は $C \propto T^3$ であり、$\lim_{T\to 0} C = 0$ となる。
  \end{enumerate}
\end{itembox}
低温領域で温度 $T$ が小さくなっていくとき、比熱 $C$ は次のように小さくなる。
\begin{align}
  C & \approx 3nR\qty(\frac{\hbar\omega}{k_BT})^2e^{-\frac{\hbar\omega}{k_BT}} \\
    & \propto \frac{1}{T^{2}e^{\frac{1}{T}}}
\end{align}
これより $C \propto T^3$ とはならない為、固体の比熱の Einstein 模型は実験事実と合致しない。

\section{古典統計力学 (classical statistical mechanics) 近似}

\begin{theorem}
  \begin{align}
    Z & = \frac{1}{(2\pi\hbar)^f}\int e^{-H(p, q)/k_BT}\prod_{i=1}^{f}\dd{p_i}\dd{q_i}
  \end{align}
\end{theorem}

\subsection{振動子系の古典近似}
\subsection{理想気体の古典近似}
\subsection{非調和振動子系の古典近似}

\section{固体の比熱の Debye 模型}
ここでは固体の比熱 $C$ の Debye 模型を学ぶ. Debye 模型は高温における $C\approx 3nR$ と低温における $C\propto T^3$ の両方を正しく説明する.
\subsection{Debye 模型の基本的な考え方}
Debye 模型は Einstein 模型と同様に固体の比熱を独立な調和振動子の集まりの比熱として捉える. ただ Debye 模型は Einstein 模型に加え, 固体を構成する各原子は原子同士の原子間力によるバネにより結びついていると考える.

\subsection{解析力学の復習:点正準変換}
ある $N$ 自由度の系の一般化座標を $q_1, \ldots, q_N$ として Lagrange 形式では一般化座標 $q_i$ と一般化速度 $\dot{q}_i$ を用いて表現される. このとき一般化運動量 $p_i$ は次のように定められる.
\begin{align}
  L   & = L(q_1,\ldots,q_N,\dot{q}_1,\ldots,\dot{q}_N),                                                                                    \\
  p_i & = \qty(\pdv{L}{\dot{q}_i})_{q_1,\ldots,q_N,\dot{q}_1,\ldots,\dot{q}_{i-1},\dot{q}_{i+1},\ldots,\dot{q}_N} \qquad (i = 1,\ldots,N).
\end{align}
一方 Hamilton 形式では一般化座標 $q_i$ と一般化運動量 $p_i$ を用いて表現される.
\begin{align}
  H           & = H(q_1,\ldots,q_N,p_1,\ldots,p_N) = \sum_{i=1}^{N}p_i\dot{q}_i - L,        \\
  \dv{q_i}{t} & = \pdv{H}{p_i}, \qquad \dv{p_i}{t} = -\pdv{H}{q_i} \qquad (i = 1,\ldots,N).
\end{align}

\begin{itembox}[l]{Q 17-1.}
  Lagrange 形式での一般座標変換 $(q_1,\ldots,q_N)\to(Q_1,\ldots,Q_N)$ に対応する Hamilton 形式で正準変換を点正準変換といい, $(q_1,\ldots,q_N,p_1,\ldots,p_N)\to(Q_1,\ldots,Q_N,P_1,\ldots,P_N)$ を求める.
  \begin{align}
    q_i = f_i(Q_1,\ldots,Q_N).
  \end{align}
\end{itembox}

(i) 新しい運動量 $P_j$ は Lagrange 形式を用いて次のように求められる.
\begin{align}
  P_j & = \qty(\pdv{L}{\dot{Q}_j})_{Q_1,\ldots,Q_N,\dot{Q}_1,\ldots,\dot{Q}_{j-1},\dot{Q}_{j+1},\ldots,\dot{Q}_N} \qquad (j=1,2,\ldots,N) \\
      & = \sum_{i=1}^{N}\pdv{L}{\dot{q}_i}\pdv{\dot{q}_i}{\dot{Q}_j}                                                                      \\
      & = \sum_{i=1}^{N}p_i\pdv{q_i}{Q_j}                                                                                                 \\
      & = \sum_{i=1}^{N}\pdv{f_i(Q_1,\ldots,Q_N)}{Q_j}p_i.
\end{align}

(ii) また新しい Hamilton 関数は定義式から古い Hamilton 関数と一致する.
\begin{align}
  H' = H'(Q_1,\ldots,Q_N,P_1,\ldots,P_N) = \sum_{j=1}^{N}P_j\dot{Q}_j - L = \sum_{j=1}^{N}\sum_{i=1}^{N}\pdv{f_i(Q_1,\ldots,Q_N)}{Q_j}p_i\dot{Q}_j - L = \sum_{i=1}^{N}p_i\dot{q}_i - L = H
\end{align}

\subsection{1 次元結晶における平衡位置の回りの調和振動を記述する Hamilton 関数}
直線上に等間隔の平衡位置を持って並んだ $N$ 個の原子からなる 1 次元結晶を物理系として記述して古典力学により考察する. $i$ 番目の原子の位置座標の平衡位置からのずれを $q_i$ として, その運動量を $p_i$ とする.
\begin{itembox}[l]{Q 17-2.}
  1 次元結晶の Hamilton 関数は次のように表される.
  \begin{align}
    H^{1次元結晶}(q_1,\ldots,q_N, p_1,\ldots,p_N) & := \frac{1}{2m}\sum_{i=1}^{N}p_i^2 + \frac{1}{2}\kappa\sum_{i=0}^{N}(q_i - q_{i+1})^2
  \end{align}
  ただし $\kappa$ は隣り合った原子の間の原子間力のバネ定数とし, 両端の原子は固定されている $q_0 = q_{N+1} = 0$ と仮定する.
\end{itembox}

$i$ 番目の原子の運動エネルギーは運動量 $p_i$ を用いて次のように表される.
\begin{align}
  \frac{p_i^2}{2m}.
\end{align}
また隣り合う $i, i+1$ 番目の原子の原子間力のポテンシャルエネルギーはバネ定数 $\kappa$ を用いて次のように表される.
\begin{align}
  \frac{1}{2}\kappa(q_i - q_{i+1})^2.
\end{align}
これより Hamilton 関数は次のように表される.
\begin{align}
  H^{1次元結晶}(q_1,\ldots,q_N, p_1,\ldots,p_N) & := \frac{1}{2m}\sum_{i=1}^{N}p_i^2 + \frac{1}{2}\kappa\sum_{i=0}^{N}(q_i - q_{i+1})^2.
\end{align}

\subsection{1 次元結晶における平衡位置の回りの調和振動の基準モードの計算}

\begin{itembox}[l]{Q 17-3.}
  固定端境界条件の 1 次元結晶の系を考えているので Fourier 展開した基底が基準振動となる.
  \begin{align}
    H^{1次元結晶}(Q_1,\ldots,Q_N, P_1,\ldots,P_N) & = \sum_{j=1}^{N}\qty(\frac{1}{2m}P_j^2 + \frac{1}{2}m\omega_j^2Q_j^2).
  \end{align}
  ただし, $\omega_j$ を次のように定める.
  \begin{align}
    \omega_j = 2\sqrt{\frac{\kappa}{m}}\sin\qty(\frac{\pi}{2(N+1)}j).
  \end{align}
\end{itembox}

固定端境界条件の 1 次元結晶の系を考えているので Fourier Sine 展開の基底が基準振動になっているとする.
\begin{align}
  q_i^{(j)} & = \sqrt{\frac{2}{N+1}}\sin\qty(\frac{\pi}{N+1}ji).
\end{align}
まず計算に必要な関数を定義する. \\

(i) $\alpha \neq 0 \pmod{2\pi}$ に対して $F(\alpha), G(\alpha)$ を次のように定義する.
\begin{align}
  F(\alpha) & := \sum_{i=1}^{N}\cos(\alpha i), \\
  G(\alpha) & := \sum_{i=1}^{N}\sin(\alpha i).
\end{align}
このとき $F(\alpha), G(\alpha)\in\RR$ より $F(\alpha) + \sqrt{-1}G(\alpha)\in\CC$ の実部と虚部はそれぞれ $F(\alpha), G(\alpha)$ と対応した値となる. Euler の公式を用いて次のように計算できる.
\begin{align}
  F(\alpha) + \sqrt{-1}G(\alpha) & = \sum_{i=1}^{N}e^{\sqrt{-1}\alpha i}                                                                                                                                                     \\
                                 & = \frac{e^{\sqrt{-1}\alpha} - e^{\sqrt{-1}\alpha (N+1)}}{1 - e^{\sqrt{-1}\alpha}}                                                                                                         \\
                                 & = \frac{2e^{\sqrt{-1}\alpha}e^{\sqrt{-1}\alpha \frac{N}{2}}\sin{\alpha \frac{N}{2}}}{2e^{\sqrt{-1}\alpha\frac{1}{2}}\sin{\alpha\frac{1}{2}}}                                              \\
                                 & = \frac{e^{\sqrt{-1}\frac{\alpha}{2}(N+1)}\sin{\frac{\alpha}{2}N}}{\sin{\frac{\alpha}{2}}}                                                                                                \\
                                 & = \frac{\cos\qty(\frac{\alpha}{2}(N+1))\sin{\frac{\alpha}{2}N}}{\sin{\frac{\alpha}{2}}} + \sqrt{-1}\frac{\sin\qty(\frac{\alpha}{2}(N+1))\sin{\frac{\alpha}{2}N}}{\sin{\frac{\alpha}{2}}}.
\end{align}
これより実部虚部の対応から $F(\alpha), G(\alpha)$ が求まる.
\begin{align}
  F(\alpha) & := \sum_{i=1}^{N}\cos(\alpha i) = \frac{\cos\qty(\frac{\alpha}{2}(N+1))\sin\qty(\frac{\alpha}{2}N)}{\sin{\frac{\alpha}{2}}}, \\
  G(\alpha) & := \sum_{i=1}^{N}\sin(\alpha i) = \frac{\sin\qty(\frac{\alpha}{2}(N+1))\sin\qty(\frac{\alpha}{2}N)}{\sin{\frac{\alpha}{2}}}.
\end{align}

(ii) $j, j' = 1,\ldots,N$ とすると $j - j' = -(N - 1),\ldots,N - 1$ かつ $j + j' = 2,\ldots,2N$ である. これより $j - j' = 0$ である場合に限り $j - j' = 0 \pmod{2(N+1)}$ が成り立ち, $j + j' = 0 \pmod{2(N+1)}$ が成り立つ場合は存在せず, 逆に主結合子の前件が恒偽ならばその論理式は真である. よって次の同値関係が成り立つ.
\begin{align}
   & \frac{\pi}{N+1}(j - j') = 0 \pmod{2\pi} \iff j - j' = 0 \pmod{2(N+1)} \iff j = j', \label{Q17-3. ii-1} \\
   & \frac{\pi}{N+1}(j + j') = 0 \pmod{2\pi} \iff j + j' = 0 \pmod{2(N+1)} \iff false. \label{Q17-3. ii-2}
\end{align}

(iii) $j, j' = 1,\ldots,N$ に対して次のように内積を定義する. このときこの内積の正規直交関係を示す.
\begin{align}
  (q^{(j)}, q^{(j')}) & := \sum_{i = 1}^{N}q_i^{(j)}q_i^{(j')}.
\end{align}
まず (i), (ii) を用いることで次のように式変形できる.
\begin{align}
  (q^{(j)}, q^{(j')}) & := \sum_{i = 1}^{N}q_i^{(j)}q_i^{(j')}                                                                                                                                                                                                                                                 \\
                      & = \frac{2}{N+1}\sum_{i = 1}^{N}\sin\qty(\frac{\pi}{N+1}ji)\sin\qty(\frac{\pi}{N+1}j'i)                                                                                                                                                                                                 \\
                      & = \frac{1}{N+1}\sum_{i = 1}^{N}\qty(\cos\qty(\frac{\pi}{N+1}(j - j')i) - \cos\qty(\frac{\pi}{N+1}(j + j')i))                                                                                                                                                                           \\
                      & = \begin{dcases}
                            \frac{1}{N+1}\qty(\frac{\cos\qty(\frac{\pi}{2}(j - j'))\sin\qty(\frac{N\pi}{2(N+1)}(j - j'))}{\sin\qty(\frac{\pi}{2(N+1)}(j - j'))} - \frac{\cos\qty(\frac{\pi}{2}(j + j'))\sin\qty(\frac{N\pi}{2(N+1)}(j + j'))}{\sin\qty(\frac{\pi}{2(N+1)}(j + j'))}) & (j \neq j') \\
                            \frac{1}{N+1}\qty(N - \frac{\cos\qty(j\pi)\sin\qty(\frac{jN}{N+1}\pi)}{\sin\qty(\frac{j}{N+1}\pi)})                                                                                                                                                      & (j = j')
                          \end{dcases}.
\end{align}
先に $j \neq j'$ の場合を考える. 括弧内を通分した分子の第一項と第二項についてそれぞれ計算する. 第一項について
\begin{align}
    & \cos\qty(\frac{\pi}{2}(j - j'))\sin\qty(\frac{N\pi}{2(N+1)}(j - j'))\sin\qty(\frac{\pi}{2(N+1)}(j + j'))                                                       \\
  = & \cos\qty(\frac{j - j'}{2}\pi)\qty(\cos\qty(\frac{(N-1)j - (N+1)j'}{2(N+1)}\pi) - \cos\qty(\frac{(N+1)j - (N-1)j'}{2(N+1)}\pi))                                 \\
  = & \cos\qty(\frac{j - j'}{2}\pi)\cos\qty(\frac{(N-1)j - (N+1)j'}{2(N+1)}\pi) - \cos\qty(\frac{j - j'}{2}\pi)\cos\qty(\frac{(N+1)j - (N-1)j'}{2(N+1)}\pi)          \\
  = & \cos\qty(\frac{j}{N+1}\pi) + \cos\qty(\frac{Nj - (N+1)j'}{N+1}\pi) - \cos\qty(\frac{j'}{N+1}\pi) - \cos\qty(\frac{(N+1)j - Nj'}{N+1}\pi). \label{Q17-3. iii 1}
\end{align}
第二項について
\begin{align}
    & \cos\qty(\frac{\pi}{2}(j + j'))\sin\qty(\frac{N\pi}{2(N+1)}(j + j'))\sin\qty(\frac{\pi}{2(N+1)}(j - j'))                                                       \\
  = & \cos\qty(\frac{j + j'}{2}\pi)\qty(\cos\qty(\frac{(N-1)j + (N+1)j'}{2(N+1)}\pi) - \cos\qty(\frac{(N+1)j + (N-1)j'}{2(N+1)}\pi))                                 \\
  = & \cos\qty(\frac{j + j'}{2}\pi)\cos\qty(\frac{(N-1)j + (N+1)j'}{2(N+1)}\pi) - \cos\qty(\frac{j + j'}{2}\pi)\cos\qty(\frac{(N+1)j + (N-1)j'}{2(N+1)}\pi)          \\
  = & \cos\qty(\frac{Nj + (N+1)j'}{N+1}\pi) + \cos\qty(\frac{j}{N+1}\pi) - \cos\qty(\frac{(N+1)j + Nj'}{N+1}\pi) - \cos\qty(\frac{j'}{N+1}\pi). \label{Q17-3. iii 2}
\end{align}
これより分子は次のようになる.
\begin{align}
  \eqref{Q17-3. iii 1} - \eqref{Q17-3. iii 2} & = \qty(\cos\frac{j}{N+1}\pi + \cos\qty(\frac{Nj}{N+1} - j')\pi - \cos\frac{j'}{N+1}\pi - \cos\qty(j - \frac{Nj'}{N+1})\pi)                  \\
                                              & - \qty(\cos\qty(\frac{Nj}{N+1} + j')\pi + \cos\frac{j}{N+1}\pi - \cos\qty(j + \frac{Nj'}{N+1})\pi - \cos\frac{j'}{N+1}\pi)                  \\
                                              & = \cos\qty(\frac{Nj}{N+1} - j')\pi - \cos\qty(\frac{Nj}{N+1} + j')\pi + \cos\qty(j + \frac{Nj'}{N+1})\pi - \cos\qty(j - \frac{Nj'}{N+1})\pi \\
                                              & = 2\sin\qty(j'\pi)\sin\qty(\frac{Nj}{N+1}\pi) - 2\sin\qty(j\pi)\sin\qty(\frac{Nj'}{N+1}\pi)                                                 \\
                                              & = 0 \qquad (\because j, j'\in\ZZ).
\end{align}
よって $j \neq j'$ のときは $(q^{(j)}, q^{(j')}) = 0$ となる.

次に $j = j'$ の場合を考える. これは $j$ が奇数か偶数かで場合分けして考える.
\begin{align}
  \frac{\cos\qty(j\pi)\sin\qty(\frac{jN}{N+1}\pi)}{\sin\qty(\frac{j}{N+1}\pi)} & =
  \begin{dcases}
    \frac{\cos\qty(2k\pi)\sin\qty(\frac{2kN}{N+1}\pi)}{\sin\qty(\frac{2k}{N+1}\pi)}           & (j = 2k, k\in\ZZ)   \\
    \frac{\cos\qty((2k-1)\pi)\sin\qty(\frac{(2k-1)N}{N+1}\pi)}{\sin\qty(\frac{2k-1}{N+1}\pi)} & (j = 2k-1, k\in\ZZ)
  \end{dcases} \\ & =
  \begin{dcases}
    \frac{1\cdot\sin\qty(2k\pi\frac{N}{N+1} - 2k\pi)}{\sin\qty(2k\pi\frac{1}{N+1})} \\
    \frac{-1 \cdot -\sin\qty((2k-1)\pi\frac{N}{N+1} - (2k-1)\pi)}{\sin\qty((2k-1)\pi\frac{1}{N+1})}
  \end{dcases}         \\
                                                                               & = -1.
\end{align}
よって $j = j'$ のときは $(q^{(j)}, q^{(j')}) = 1$ となる. これより, まとめると次の式が成り立つ.
\begin{align}
  (q^{(j)}, q^{(j')}) = \delta_{j,j'}.
\end{align}


(iv) ここで行列 $A_{ij} := q_i^{(j)}$ を定義する. このとき次の計算から $A_{ij}$ は直交行列であるとわかる.
\begin{align}
  (A^{\top}A)_{ij} & = \sum_{k=1}^{N}A_{ik}^\top A_{kj} = \sum_{k=1}^{N}A_{ki}A_{kj} = \sum_{k=1}^{N}q_k^{(i)}q_k^{(j)} = (q^{(i)}, q^{(j)}) = \delta_{i,j}.
\end{align}

(v) また $A_{ij}$ が直交行列であるから次のような正規直交関係もある.
\begin{align}
  (AA^{\top})_{ij} & = \sum_{k=1}^{N}A_{ik}A_{kj}^{\top} = \sum_{k=1}^{N}A_{ik}A_{jk} = \sum_{k=1}^{N}q_i^{(k)}q_j^{(k)} = \delta_{i,j}.
\end{align}

(vi) ここで原子の変位を表す古い座標系 $q_1, \ldots, q_N$ を $q^{(1)}, \ldots, q^{(N)}$ で離散 Fourier Sine 展開した振幅を新しい座標系 $Q_1, \ldots, Q_N$ と定義する.
\begin{align}
  q_i = \sum_{j=1}^{N}Q_jq_i^{(j)}.
\end{align}
これは点正準変換を用いて新しい運動量を古い運動量を表せられる.
\begin{align}
  P_j = \sum_{i=1}^{N}\pdv{q_i}{Q_j}p_i = \sum_{i=1}^{N}q_i^{(j)}p_i.
\end{align}

(vii) Hamilton 関数の運動エネルギーの表式の核の部分について次のように表される.
\begin{align}
  \sum_{j=1}^{N}P_j^2 = \sum_{j=1}^{N}\qty(\sum_{i=1}^{N}q_i^{(j)}p_i)^2 = \sum_{j=1}^{N}\sum_{i=1}^{N}\sum_{i'=1}^{N}(q_i^{(j)}p_i)(q_{i'}^{(j)}p_{i'}) = \sum_{i=1}^{N}p_i^2.
\end{align}

(viii) Hamilton 関数のポテンシャルエネルギーの核の部分について次のような表される.
\begin{align}
  \sum_{i=0}^{N}(q_i - q_{i+1})^2 & = \sum_{i=0}^{N}\qty(\sum_{j=1}^{N}\qty(Q_jq_i^{(j)} - Q_jq_{i+1}^{(j)}))^2                                                     \\
                                  & = \sum_{i=0}^{N}\sum_{j=1}^{N}\sum_{j'=1}^{N}\qty(Q_jq_i^{(j)} - Q_jq_{i+1}^{(j)})\qty(Q_{j'}q_i^{(j')} - Q_{j'}q_{i+1}^{(j')}) \\
                                  & = \sum_{j=1}^{N}\sum_{j'=1}^{N}\sum_{i=0}^{N}(q_i^{(j)} - q_{i+1}^{(j)})(q_i^{(j')} - q_{i+1}^{(j')})Q_jQ_{j'}                  \\
                                  & = \sum_{j=1}^{N}\sum_{j'=1}^{N}B_{j,j'}Q_jQ_{j'}.
\end{align}
ただし, $B_{j,j'}$ を次のように定める.
\begin{align}
  B_{j,j'} := \sum_{i=0}^{N}(q_i^{(j)} - q_{i+1}^{(j)})(q_i^{(j')} - q_{i+1}^{(j')}).
\end{align}

(ix) 次に $B_{j,j'}$ を求める. まず $q_i^{(j)} - q_{i+1}^{(j)}$ は次のように求められる.
\begin{align}
  q_i^{(j)} - q_{i+1}^{(j)} & = \sqrt{\frac{2}{N+1}}\sin\qty(\frac{\pi}{N+1}ji) - \sqrt{\frac{2}{N+1}}\sin\qty(\frac{\pi}{N+1}j(i+1)) \\
                            & = \sqrt{\frac{2}{N+1}}\qty(\sin\qty(\frac{\pi}{N+1}ji) - \sin\qty(\frac{\pi}{N+1}j(i+1)))               \\
                            & = -2\sqrt{\frac{2}{N+1}}\cos\qty(\frac{\pi}{2}\frac{(2i+1)j}{N+1})\sin\qty(\frac{\pi}{2}\frac{j}{N+1}).
\end{align}

(x) これより $B_{j,j'}$ は次のように計算できる.
\begin{align}
  B_{j,j'} & = \sum_{i=0}^{N}(q_i^{(j)} - q_{i+1}^{(j)})(q_i^{(j')} - q_{i+1}^{(j')})                                                                                                                                                               \\
           & = \sum_{i=0}^{N}\qty(-2\sqrt{\frac{2}{N+1}}\cos\qty(\frac{\pi}{2}\frac{(2i+1)j}{N+1})\sin\qty(\frac{\pi}{2}\frac{j}{N+1}))\qty(-2\sqrt{\frac{2}{N+1}}\cos\qty(\frac{\pi}{2}\frac{(2i+1)j'}{N+1})\sin\qty(\frac{\pi}{2}\frac{j'}{N+1})) \\
           & = 4\sin\qty(\frac{\pi}{2}\frac{j}{N+1})\sin\qty(\frac{\pi}{2}\frac{j'}{N+1})\frac{2}{N+1}\sum_{i=0}^{N}\cos\qty(\frac{\pi}{N+1}j\qty(i + \frac{1}{2}))\cos\qty(\frac{\pi}{N+1}j'\qty(i + \frac{1}{2}))                                 \\
           & = 4\sin\qty(\frac{\pi}{2}\frac{j}{N+1})\sin\qty(\frac{\pi}{2}\frac{j'}{N+1})\frac{1}{N+1}\sum_{i=0}^{N}\qty(\cos\qty(\frac{\pi}{N+1}(j + j')\qty(i + \frac{1}{2})) + \cos\qty(\frac{\pi}{N+1}(j - j')\qty(i + \frac{1}{2})))           \\
           & = 4\sin\qty(\frac{\pi}{2}\frac{j}{N+1})\sin\qty(\frac{\pi}{2}\frac{j'}{N+1})\tilde{B}_{j,j'}.
\end{align}
ただし, $\tilde{B}_{j,j'}$ を次のように定める.
\begin{align}
  \tilde{B}_{j,j'} & := \frac{1}{N+1}\sum_{i=0}^{N}\qty(\cos\qty(\frac{\pi}{N+1}(j + j')\qty(i + \frac{1}{2})) + \cos\qty(\frac{\pi}{N+1}(j - j')\qty(i + \frac{1}{2}))).
\end{align}

(xi) さらに $\tilde{B}_{j,j'}$ は次のように計算できる.
\begin{align}
  \tilde{B}_{j,j'} & = \frac{1}{N+1}\sum_{i=0}^{N}\qty(\cos\qty(\pi\frac{j + j'}{N+1}\qty(i + \frac{1}{2})) + \cos\qty(\pi\frac{j - j'}{N+1}\qty(i + \frac{1}{2}))) \\
                   & \ \begin{aligned}
                         = \frac{1}{N+1}\sum_{i=0}^{N}\bigg[ & \quad\cos\qty(\frac{\pi}{2}\frac{j + j'}{N+1})\cos\qty(\pi\frac{j + j'}{N+1}i)    \\
                                                             & - \sin\qty(\frac{\pi}{2}\frac{j + j'}{N+1})\sin\qty(\pi\frac{j + j'}{N+1}i)       \\
                                                             & + \cos\qty(\frac{\pi}{2}\frac{j - j'}{N+1})\cos\qty(\pi\frac{j - j'}{N+1}i)       \\
                                                             & - \sin\qty(\frac{\pi}{2}\frac{j - j'}{N+1})\sin\qty(\pi\frac{j - j'}{N+1}i)\bigg]
                       \end{aligned}                                         \\
                   & \ \begin{aligned}
                         = \frac{1}{N+1}\bigg[ & \quad\cos\qty(\frac{\pi}{2}\frac{j + j'}{N+1})\qty(1 + F\qty(\pi\frac{j + j'}{N+1})) \\
                                               & - \sin\qty(\frac{\pi}{2}\frac{j + j'}{N+1})G\qty(\pi\frac{j + j'}{N+1})              \\
                                               & + \cos\qty(\frac{\pi}{2}\frac{j - j'}{N+1})\qty(1 + F\qty(\pi\frac{j - j'}{N+1}))    \\
                                               & - \sin\qty(\frac{\pi}{2}\frac{j - j'}{N+1})G\qty(\pi\frac{j - j'}{N+1})\bigg]
                       \end{aligned}.
\end{align}

(xii) まず $\tilde{B}_{j,j'}$ について $j = j'$ の場合を考える.
\begin{align}
  \tilde{B}_{j,j'} & = \tilde{B}_{j,j}                                                                                                                                                                                                                                      \\
                   & = \frac{1}{N+1}\qty[\cos\qty(\frac{1}{N+1}j\pi)\qty(1 + F\qty(\frac{2}{N+1}j\pi)) - \sin\qty(\frac{1}{N+1}j\pi)G\qty(\frac{2}{N+1}j\pi) + \qty(1 + N) - 0]                                                                                             \\
                   & = 1 + \frac{1}{N+1}\qty(\cos\qty(\frac{1}{N+1}j\pi)\qty(1 + \frac{\cos\qty(j\pi)\sin\qty(\frac{N}{N+1}j\pi)}{\sin\qty(\frac{1}{N+1}j\pi)}) - \sin\qty(\frac{1}{N+1}j\pi)\frac{\sin\qty(j\pi)\sin\qty(\frac{N}{N+1}j\pi)}{\sin\qty(\frac{1}{N+1}j\pi)}) \\
                   & = 1 + \frac{1}{N+1}\qty(\cos\qty(\frac{1}{N+1}j\pi) + \qty(\cos\qty(\frac{1}{N+1}j\pi)\cos\qty(j\pi) - \sin\qty(\frac{1}{N+1}j\pi)\sin\qty(j\pi))\frac{\sin\qty(\frac{N}{N+1}j\pi)}{\sin\qty(\frac{1}{N+1}j\pi)})                                      \\
                   & = 1 + \frac{1}{N+1}\qty(\cos\qty(\frac{1}{N+1}j\pi) + \cos\qty(\frac{N+2}{N+1}j\pi)\frac{\sin\qty(\frac{N}{N+1}j\pi)}{\sin\qty(\frac{1}{N+1}j\pi)})                                                                                                    \\
                   & = 1 + \frac{1}{N+1}\qty(\cos\qty(\frac{1}{N+1}j\pi)\sin\qty(\frac{1}{N+1}j\pi) + \cos\qty(\frac{N+2}{N+1}j\pi)\sin\qty(\frac{N}{N+1}j\pi))\bigg/\sin\qty(\frac{1}{N+1}j\pi)                                                                            \\
                   & = 1 + \frac{1}{N+1}\qty(\frac{1}{2}\sin\qty(\frac{2}{N+1}j\pi) + \frac{1}{2}\sin\qty(-\frac{2}{N+1}j\pi))\bigg/\sin\qty(\frac{1}{N+1}j\pi)                                                                                                             \\
                   & = 1.
\end{align}

(xiii) 次に $\tilde{B}_{j,j'}$ について $j \neq j'$ の場合を考える.
\begin{align}
  \tilde{B}_{j,j'} & = \tilde{B}_{j,j'}                                                                                                                                                                         \\
                   & \ \begin{aligned}
                         = \frac{1}{N+1}\bigg[ & \quad\cos\qty(\frac{\pi}{2}\frac{j + j'}{N+1})\qty(1 + F\qty(\pi\frac{j + j'}{N+1})) \\
                                               & - \sin\qty(\frac{\pi}{2}\frac{j + j'}{N+1})G\qty(\pi\frac{j + j'}{N+1})              \\
                                               & + \cos\qty(\frac{\pi}{2}\frac{j - j'}{N+1})\qty(1 + F\qty(\pi\frac{j - j'}{N+1}))    \\
                                               & - \sin\qty(\frac{\pi}{2}\frac{j - j'}{N+1})G\qty(\pi\frac{j - j'}{N+1})\bigg]
                       \end{aligned}                                                                                               \\
                   & \ \begin{aligned}
                         = \frac{1}{N+1}\Bigg[ & \quad\cos\qty(\frac{\pi}{2}\frac{j + j'}{N+1})\qty(1 + \frac{\cos\qty(\frac{1}{2}(j+j')\pi)\sin\qty(\frac{N}{2(N+1)}(j+j')\pi)}{\sin\qty(\frac{1}{2(N+1)}(j+j')\pi)}) \\
                                               & - \sin\qty(\frac{\pi}{2}\frac{j + j'}{N+1})\frac{\sin\qty(\frac{1}{2}(j+j')\pi)\sin\qty(\frac{N}{2(N+1)}(j+j')\pi)}{\sin\qty(\frac{1}{2(N+1)}(j+j')\pi)}              \\
                                               & + \cos\qty(\frac{\pi}{2}\frac{j - j'}{N+1})\qty(1 + \frac{\cos\qty(\frac{1}{2}(j-j')\pi)\sin\qty(\frac{N}{2(N+1)}(j-j')\pi)}{\sin\qty(\frac{1}{2(N+1)}(j-j')\pi)})    \\
                                               & - \sin\qty(\frac{\pi}{2}\frac{j - j'}{N+1})\frac{\sin\qty(\frac{1}{2}(j-j')\pi)\sin\qty(\frac{N}{2(N+1)}(j-j')\pi)}{\sin\qty(\frac{1}{2(N+1)}(j-j')\pi)}\Bigg]
                       \end{aligned}                                                                                                \\
                   & \ \begin{aligned}
                         = \frac{1}{N+1}\Bigg[ & \quad\cos\qty(\frac{\pi}{2}\frac{j + j'}{N+1}) + \qty(\cos\qty(\frac{\pi}{2}\frac{j + j'}{N+1})\cos\qty(\frac{j+j'}{2}\pi) - \sin\qty(\frac{\pi}{2}\frac{j + j'}{N+1})\sin\qty(\frac{j+j'}{2}\pi))\frac{\sin\qty(\frac{N(j+j')}{2(N+1)}\pi)}{\sin\qty(\frac{j+j'}{2(N+1)}\pi)}    \\
                                               & + \cos\qty(\frac{\pi}{2}\frac{j - j'}{N+1}) + \qty(\cos\qty(\frac{\pi}{2}\frac{j - j'}{N+1})\cos\qty(\frac{j-j'}{2}\pi) - \sin\qty(\frac{\pi}{2}\frac{j - j'}{N+1})\sin\qty(\frac{j-j'}{2}\pi))\frac{\sin\qty(\frac{N(j-j')}{2(N+1)}\pi)}{\sin\qty(\frac{j-j'}{2(N+1)}\pi)}\Bigg]
                       \end{aligned}                                                                             \\
                   & \ \begin{aligned}
                         = \frac{1}{N+1}\Bigg[ & \quad\cos\qty(\frac{\pi}{2}\frac{j + j'}{N+1}) + \cos\qty(\frac{N+2}{2(N+1)}(j + j')\pi)\frac{\sin\qty(\frac{N}{2(N+1)}(j+j')\pi)}{\sin\qty(\frac{1}{2(N+1)}(j+j')\pi)}    \\
                                               & + \cos\qty(\frac{\pi}{2}\frac{j - j'}{N+1}) + \cos\qty(\frac{N+2}{2(N+1)}(j - j')\pi)\frac{\sin\qty(\frac{N}{2(N+1)}(j-j')\pi)}{\sin\qty(\frac{1}{2(N+1)}(j-j')\pi)}\Bigg]
                       \end{aligned}                                                                         \\
                   & \ \begin{aligned}
                         = \frac{1}{N+1}\Bigg[ & \quad\frac{1}{2}\qty(\sin\qty(\frac{j + j'}{N+1}\pi) + \sin\qty((j + j')\pi) + \sin\qty(-\frac{j+j'}{N+1}\pi))\bigg/\sin\qty(\frac{1}{2(N+1)}(j+j')\pi)    \\
                                               & + \frac{1}{2}\qty(\sin\qty(\frac{j - j'}{N+1}\pi) + \sin\qty((j - j')\pi) + \sin\qty(-\frac{j-j'}{N+1}\pi))\bigg/\sin\qty(\frac{1}{2(N+1)}(j-j')\pi)\bigg]
                       \end{aligned} \\
                   & = 0.
\end{align}
よって (xii), (xiii) の考察から次の式が成り立つ.
\begin{align}
  \tilde{B}_{j,j'} = \delta_{j,j'}.
\end{align}

(xiv) これより $B_{j,j'}$ は (x) の考察から次のようになる.
\begin{align}
  B_{j,j'} & = 4\sin\qty(\frac{\pi}{2}\frac{j}{N+1})\sin\qty(\frac{\pi}{2}\frac{j'}{N+1})\tilde{B}_{j,j'} \\
           & = \delta_{j,j'}4\sin^2\qty(\frac{\pi}{2(N+1)}j).
\end{align}

(xv) ポテンシャルエネルギーの表式 (vii) に代入して次のようになる.
\begin{align}
  \sum_{i=0}^{N}(q_i - q_{i+1})^2 & = \sum_{j=1}^{N}\sum_{j'=1}^{N}B_{j,j'}Q_jQ_{j'}                                      \\
                                  & = \sum_{j=1}^{N}\sum_{j'=1}^{N}\delta_{j,j'}4\sin^2\qty(\frac{\pi}{2(N+1)}j)Q_jQ_{j'} \\
                                  & = 4\sum_{j=1}^{N}\sin^2\qty(\frac{\pi}{2(N+1)}j)Q_j^2.
\end{align}

(xvi) よって Hamilton 関数は (vii) (xv) から次のように表される.
\begin{align}
  H^{1次元結晶}(q_1,\ldots,q_N, p_1,\ldots,p_N) & = \frac{1}{2m}\sum_{i=1}^{N}p_i^2 + \frac{1}{2}\kappa\sum_{i=0}^{N}(q_i - q_{i+1})^2          \\
                                            & = \frac{1}{2m}\sum_{j=1}^{N}P_j^2 + 2\kappa\sum_{j=1}^{N}\sin^2\qty(\frac{\pi}{2(N+1)}j)Q_j^2 \\
  H^{1次元結晶}(Q_1,\ldots,Q_N, P_1,\ldots,P_N) & = \sum_{j=1}^{N}\qty(\frac{1}{2m}P_j^2 + \frac{1}{2}m\omega_j^2Q_j^2).
\end{align}
ただし, $\omega_j$ を次のように定めた.
\begin{align}
  \omega_j = 2\sqrt{\frac{\kappa}{m}}\sin\qty(\frac{\pi}{2(N+1)}j) \qquad (j = 1,\ldots,N).
\end{align}

\begin{itembox}[l]{Q 17-4.}
  1 次元結晶中の波数 $k$ に対する分散関係 $\omega(k)$ は次のようになる.
  \begin{align}
    \omega(k) & = 2\sqrt{\frac{\kappa}{m}}\sin\qty(\frac{1}{2}ka) \approx \sqrt{\frac{\kappa}{m}}ka + \mathcal{O}((ka)^3) \qquad (ka\ll 1).
  \end{align}
\end{itembox}

(i) $j = 1,\ldots,N$ に対して $j$ 番目の基準振動 $q_i^{(j)}$ は次のように計算される.
\begin{align}
  q_i^{(j)} & = \sqrt{\frac{2}{N+1}}\sin\qty(\frac{\pi}{N+1}ji)             \\
            & = \sqrt{\frac{2}{N+1}}\sin\qty(\frac{\pi}{a}\frac{j}{N+1}x_i) \\
            & = \sqrt{\frac{2}{N+1}}\sin\qty(k_jx_i).
\end{align}
ただし, $i$ 番目の原子の平衡位置の座標を $x_i = ai$ とし, $j$ 番目の基準振動の波数 $k_j$ を次のように定める.
\begin{align}
  k_j := \frac{\pi}{a}\frac{j}{N+1} \qquad (j = 1,\ldots,N).
\end{align}

(ii) 基準振動 $q_i^{(j)}$ の角振動数 $\omega_j$ を波数 $k_j$ の関数として次のように表される.
\begin{align}
  \omega(k_j) & = 2\sqrt{\frac{\kappa}{m}}\sin\qty(\frac{\pi}{2(N+1)}j) \\
              & = 2\sqrt{\frac{\kappa}{m}}\sin\qty(\frac{1}{2}k_ja).
\end{align}
よって分散関係 $\omega = \omega(k)$ は次のように与えられる.
\begin{align}
  \omega(k) & = 2\sqrt{\frac{\kappa}{m}}\sin\qty(\frac{1}{2}ka).
\end{align}

(iii) この 1 次元結晶を伝わる線形波動 (弾性波, 音波) が波数ごとに異なる速さを持って伝播するということから, 1次元結晶中にこれらを重ね合わせて波束が作られたとすると次第に波束の形が変化していき最終的に崩壊する.

(iv) 十分に長波長 $ka\ll 1$ のとき次のように近似することで分散関係 $\omega(k)$ は線形関係となる.
\begin{align}
  \omega(k) & = 2\sqrt{\frac{\kappa}{m}}\sin\qty(\frac{1}{2}ka)                         \\
            & \approx 2\sqrt{\frac{\kappa}{m}}\qty(\frac{1}{2}ka + \mathcal{O}((ka)^3)) \\
            & = \sqrt{\frac{\kappa}{m}}ka + \mathcal{O}((ka)^3) \qquad (ka\ll 1).
\end{align}

(v) 長波長の極限での弾性波の速さを音速という. 固体の音速 $v$ は次のようになる.
\begin{align}
  v & = \lim_{ka\to 0}\frac{\omega(k)}{k} = \sqrt{\frac{\kappa}{m}}a.
\end{align}

(vi) (iv), (v) の考察より十分に長波長のとき分散関係が線形関係となるので 1 次元結晶中では線形波動は音速 $v$ と等しい速さを持って伝搬する.

\begin{itembox}[l]{Q 17-5.}
  1 次元結晶における基準振動の角振動数 $\omega_j$ の分布を明らかにする.
\end{itembox}

(i)(ii) $\omega_j$ は次のように表されることから $j=1,\ldots,N$ に対して単調増加となる.
\begin{align}
  \omega_j & = 2\sqrt{\frac{\kappa}{m}}\sin\qty(\frac{\pi}{2(N+1)}j).
\end{align}
これより $\omega_j$ の最大値と最小値は次のようになる.
\begin{align}
  \omega_{\max} & := \max_{1\leq j\leq N}\omega_j = \omega_N = 2\sqrt{\frac{\kappa}{m}}\sin\qty(\frac{\pi N}{2(N+1)}) \approx 2\sqrt{\frac{\kappa}{m}},                                                          \\
  \omega_{\min} & := \min_{1\leq j\leq N}\omega_j = \omega_1 = 2\sqrt{\frac{\kappa}{m}}\sin\qty(\frac{\pi}{2(N+1)}) \approx 2\sqrt{\frac{\kappa}{m}}\frac{\pi}{2(N+1)} = \sqrt{\frac{\kappa}{m}}\frac{\pi}{N+1}.
\end{align}
\subsection{3 次元結晶における平衡位置の回りの調和振動を記述する Hamilton 関数}
立方格子の各点に平衡位置を持つ $N^3$ 個の原子が全体として立方体に並んだ 3 次元結晶を物理系として記述して、古典力学により考察する。任意の $i_x,i_y,i_z = 1,\ldots,N$ に対してラベル $(i_x,i_y,i_z)$ を持つ原子の平衡位置は格子定数 $a$ を用いて $(ai_x,ai_y,ai_z)$ であるとする.
\begin{itembox}[l]{Q 17-6.}
  このとき 3 次元結晶の Hamilton 関数は次のように与えられる.
  \begin{align}
       & H^{3次元結晶}((q_{i_x, i_y, i_z, \alpha}, p_{i_x, i_y, i_z, \alpha})_{1\leq i_x,i_y,i_z\leq N,\alpha=x,y,z})                                                                                                                                                         \\
    := & \frac{1}{2m}\sum_{i_x=1}^{N}\sum_{i_y=1}^{N}\sum_{i_z=1}^{N}\sum_{\alpha=x,y,z}p_{i_x,i_y,i_z,\alpha}^2                                                                                                                                                          \\
    +  & \frac{1}{2}\kappa\sum_{i_x=0}^{N}\sum_{i_y=0}^{N}\sum_{i_z=0}^{N}\sum_{\alpha=x,y,z}\qty((q_{i_x,i_y,i_z,\alpha} - q_{i_x+1,i_y,i_z,\alpha})^2 + (q_{i_x,i_y,i_z,\alpha} - q_{i_x,i_y+1,i_z,\alpha})^2 + (q_{i_x,i_y,i_z,\alpha} - q_{i_x,i_y,i_z+1,\alpha})^2).
  \end{align}
  ただし $m$ は 1 個の原子の質量であり, $\kappa$ は隣り合った原子間の原子間力のバネ定数とする. また立方体の表面は固定されているとする.
  \begin{align}
    i_x = 0, N+1 \lor i_y = 0, N+1 \lor i_z = 0, N+1 \implies q_{i_x,i_y,i_z,\alpha} = 0.
  \end{align}
\end{itembox}
Q17-3 の考察から 1 次元結晶の系の Hamilton 関数は次のように与えられる.
\begin{align}
  H^{1次元結晶}(q_1,\ldots,q_N, p_1,\ldots,p_N) & := \frac{1}{2m}\sum_{i=1}^{N}p_i^2 + \frac{1}{2}\kappa\sum_{i=0}^{N}(q_i - q_{i+1})^2.
\end{align}
3 次元結晶の系は $N^3$ 個の原子と $3$ 個の自由度があり, それらの原子間力は独立にそれぞれの自由度と原子に働くと考えられる. これより 3 次元結晶の系の Hamilton 関数 $H^{3次元結晶}((q_{i_x, i_y, i_z, \alpha}, p_{i_x, i_y, i_z, \alpha})_{1\leq i_x,i_y,i_z\leq N,\alpha=x,y,z})$ は次のように書ける.
\begin{align}
     & H^{3次元結晶}((q_{i_x, i_y, i_z, \alpha}, p_{i_x, i_y, i_z, \alpha})_{1\leq i_x,i_y,i_z\leq N,\alpha=x,y,z})                                                                                                                                                         \\
  := & \frac{1}{2m}\sum_{i_x=1}^{N}\sum_{i_y=1}^{N}\sum_{i_z=1}^{N}\sum_{\alpha=x,y,z}p_{i_x,i_y,i_z,\alpha}^2                                                                                                                                                          \\
  +  & \frac{1}{2}\kappa\sum_{i_x=0}^{N}\sum_{i_y=0}^{N}\sum_{i_z=0}^{N}\sum_{\alpha=x,y,z}\qty((q_{i_x,i_y,i_z,\alpha} - q_{i_x+1,i_y,i_z,\alpha})^2 + (q_{i_x,i_y,i_z,\alpha} - q_{i_x,i_y+1,i_z,\alpha})^2 + (q_{i_x,i_y,i_z,\alpha} - q_{i_x,i_y,i_z+1,\alpha})^2).
\end{align}
ただし $m$ は 1 個の原子の質量であり, $\kappa$ は隣り合った原子間の原子間力のバネ定数とする. また立方体の表面は固定されているとする.
\begin{align}
  i_x = 0, N+1 \lor i_y = 0, N+1 \lor i_z = 0, N+1 \implies q_{i_x,i_y,i_z,\alpha} = 0.
\end{align}

\subsection{3 次元結晶における平衡位置の回りの調和振動の基準モードの計算}
固定端境界条件の 3 次元結晶の系を考えているので 1 次元の Fourier Sine 展開の基底 3 つの直積が基準振動になっていると予想できる. これより古い座標 $q_{i_x,i_y,i_z,\alpha}$ を基準振動 $q_{i_x}^{(j_x)}q_{i_y}^{(j_y)}q_{i_z}^{(j_z)}$ で展開したときの振幅を新しい座標 $Q_{j_x,j_y,j_z,\alpha}$ とする.
\begin{align}
  q_{i_x,i_y,i_z,\alpha} & = \sum_{j_x=1}^{N}\sum_{j_y=1}^{N}\sum_{j_z=1}^{N}Q_{j_x,j_y,j_z,\alpha}q_{i_x}^{(j_x)}q_{i_y}^{(j_y)}q_{i_z}^{(j_z)}.
\end{align}
この新しい座標 $Q_{j_x,j_y,j_z,\alpha}$ に対応する新しい運動量を $P_{j_x, j_y, j_z, \alpha}$ とおくと Hamilton 関数について次のように表される.

\begin{itembox}[l]{Q 17-7.}
  新しい座標と運動量 $Q_{j_x, j_y, j_z, \alpha}, P_{j_x, j_y, j_z, \alpha}$ において Hamilton 関数は次のように表される.
  \begin{align}
    H^{3次元結晶}((Q_{j_x, j_y, j_z, \alpha}, P_{j_x, j_y, j_z, \alpha})_{1\leq j_x,j_y,j_z\leq N,\alpha=x,y,z}) & = \sum_{j_x=1}^{N}\sum_{j_y=1}^{N}\sum_{j_z=1}^{N}\sum_{\alpha=x,y,z}\qty(\frac{1}{2m}P_{j_x,j_y,j_z,\alpha}^2 + \frac{1}{2}m\omega_{j_x,j_y,j_z}^2Q_{j_x,j_y,j_z,\alpha}^2).
  \end{align}
  ただし, $\omega_{j_x,j_y,j_z}$ は次のように定めた.
  \begin{align}
    \omega_{j_x,j_y,j_z} & = 2\sqrt{\frac{\kappa}{m}}\sqrt{\sin^2\qty(\frac{\pi}{2(N+1)}j_x) + \sin^2\qty(\frac{\pi}{2(N+1)}j_y) + \sin^2\qty(\frac{\pi}{2(N+1)}j_z)}.
  \end{align}
\end{itembox}

(i) Q17-1 の考察より新しい運動量を古い運動量と座標, 新しい座標から求めることができる.
\begin{align}
  P_{j_x,j_y,j_z,\alpha} & = \sum_{i_x=1}^{N}\sum_{i_y=1}^{N}\sum_{i_z=1}^{N}\pdv{q_{i_x,i_y,i_z,\alpha}}{Q_{j_x,j_y,j_z,\alpha}}p_{i_x,i_y,i_z,\alpha} \\
                         & = \sum_{i_x=1}^{N}\sum_{i_y=1}^{N}\sum_{i_z=1}^{N}q_{i_x}^{(j_x)}q_{i_y}^{(j_y)}q_{i_z}^{(j_z)}p_{i_x,i_y,i_z,\alpha}.
\end{align}

(ii) この点正準変換に対し, 運動エネルギーは新しい運動量を用いて表せられる.
\begin{align}
    & \sum_{j_x=1}^{N}\sum_{j_y=1}^{N}\sum_{j_z=1}^{N}P_{j_x,j_y,j_z,\alpha}^2                                                                                                                                                                                                                              \\
  = & \sum_{j_x=1}^{N}\sum_{j_y=1}^{N}\sum_{j_z=1}^{N}\qty(\sum_{i_x=1}^{N}\sum_{i_y=1}^{N}\sum_{i_z=1}^{N}q_{i_x}^{(j_x)}q_{i_y}^{(j_y)}q_{i_z}^{(j_z)}p_{i_x,i_y,i_z,\alpha})^2                                                                                                                           \\
  = & \sum_{j_x=1}^{N}\sum_{j_y=1}^{N}\sum_{j_z=1}^{N}\qty(\sum_{i_x=1}^{N}\sum_{i_y=1}^{N}\sum_{i_z=1}^{N}\sum_{i_x'=1}^{N}\sum_{i_y'=1}^{N}\sum_{i_z'=1}^{N}q_{i_x}^{(j_x)}q_{i_y}^{(j_y)}q_{i_z}^{(j_z)}p_{i_x,i_y,i_z,\alpha}q_{i_x'}^{(j_x)}q_{i_y'}^{(j_y)}q_{i_z'}^{(j_z)}p_{i_x',i_y',i_z',\alpha}) \\
  = & \sum_{i_x=1}^{N}\sum_{i_y=1}^{N}\sum_{i_z=1}^{N}\sum_{i_x'=1}^{N}\sum_{i_y'=1}^{N}\sum_{i_z'=1}^{N}\delta_{i_x,i_x'}\delta_{i_y,i_y'}\delta_{i_z,i_z'}p_{i_x,i_y,i_z,\alpha}p_{i_x',i_y',i_z',\alpha}                                                                                                 \\
  = & \sum_{i_x=1}^{N}\sum_{i_y=1}^{N}\sum_{i_z=1}^{N}p_{i_x,i_y,i_z,\alpha}^2.
\end{align}

(iii) またポテンシャルエネルギーについても新しい座標で表すことができる.
\begin{align}
    & \sum_{i_x=0}^{N}\sum_{i_y=0}^{N}\sum_{i_z=0}^{N}(q_{i_x,i_y,i_z,\alpha} - q_{i_x+1,i_y,i_z,\alpha})^2                                                                                                                                                                                                                                                \\
  = & \sum_{i_x=0}^{N}\sum_{i_y=0}^{N}\sum_{i_z=0}^{N}\qty(\sum_{j_x=1}^{N}\sum_{j_y=1}^{N}\sum_{j_z=1}^{N}\qty(Q_{j_x,j_y,j_z,\alpha}q_{i_x}^{(j_x)}q_{i_y}^{(j_y)}q_{i_z}^{(j_z)} - Q_{j_x,j_y,j_z,\alpha}q_{i_x+1}^{(j_x)}q_{i_y}^{(j_y)}q_{i_z}^{(j_z)}))^2                                                                                            \\
  = & \sum_{i_x=0}^{N}\sum_{i_y=0}^{N}\sum_{i_z=0}^{N}\sum_{j_x=1}^{N}\sum_{j_y=1}^{N}\sum_{j_z=1}^{N}\sum_{j_x'=1}^{N}\sum_{j_y'=1}^{N}\sum_{j_z'=1}^{N}                                                                                                                                                                                                  \\
    & \qty(Q_{j_x,j_y,j_z,\alpha}q_{i_x}^{(j_x)}q_{i_y}^{(j_y)}q_{i_z}^{(j_z)} - Q_{j_x,j_y,j_z,\alpha}q_{i_x+1}^{(j_x)}q_{i_y}^{(j_y)}q_{i_z}^{(j_z)})\qty(Q_{j_x',j_y',j_z',\alpha}q_{i_x}^{(j_x')}q_{i_y}^{(j_y')}q_{i_z}^{(j_z')} - Q_{j_x',j_y',j_z',\alpha}q_{i_x+1}^{(j_x')}q_{i_y}^{(j_y')}q_{i_z}^{(j_z')})                                       \\
  = & \sum_{i_x=0}^{N}\sum_{i_y=0}^{N}\sum_{i_z=0}^{N}\sum_{j_x=1}^{N}\sum_{j_y=1}^{N}\sum_{j_z=1}^{N}\sum_{j_x'=1}^{N}\sum_{j_y'=1}^{N}\sum_{j_z'=1}^{N}Q_{j_x,j_y,j_z,\alpha}\qty(q_{i_x}^{(j_x)} - q_{i_x+1}^{(j_x)})q_{i_y}^{(j_y)}q_{i_z}^{(j_z)}Q_{j_x',j_y',j_z',\alpha}\qty(q_{i_x}^{(j_x')} - q_{i_x+1}^{(j_x')})q_{i_y}^{(j_y')}q_{i_z}^{(j_z')} \\
  = & \sum_{j_x=1}^{N}\sum_{j_y=1}^{N}\sum_{j_z=1}^{N}\sum_{j_x'=1}^{N}\sum_{j_y'=1}^{N}\sum_{j_z'=1}^{N}B_{j_x,j_x'}\delta_{j_y,j_y'}\delta_{j_z,j_z'}Q_{j_x,j_y,j_z,\alpha}Q_{j_x',j_y',j_z',\alpha}                                                                                                                                                     \\
  = & \sum_{j_x=1}^{N}\sum_{j_y=1}^{N}\sum_{j_z=1}^{N}\sum_{j_x'=1}^{N}\sum_{j_y'=1}^{N}\sum_{j_z'=1}^{N}4\sin^2\qty(\frac{\pi}{2(N+1)}j_x)\delta_{j_x,j_x'}\delta_{j_y,j_y'}\delta_{j_z,j_z'}Q_{j_x,j_y,j_z,\alpha}Q_{j_x',j_y',j_z',\alpha}                                                                                                              \\
  = & 4\sum_{j_x=1}^{N}\sum_{j_y=1}^{N}\sum_{j_z=1}^{N}\sin^2\qty(\frac{\pi}{2(N+1)}j_x)Q_{j_x,j_y,j_z,\alpha}^2.
\end{align}

(iv) これより Hamilton 関数は新しい座標と運動量を用いて表すことができる.
\begin{align}
     & H^{3次元結晶}((q_{i_x, i_y, i_z, \alpha}, p_{i_x, i_y, i_z, \alpha})_{1\leq i_x,i_y,i_z\leq N,\alpha=x,y,z})                                                                                                                                                          \\
  := & \frac{1}{2m}\sum_{i_x=1}^{N}\sum_{i_y=1}^{N}\sum_{i_z=1}^{N}\sum_{\alpha=x,y,z}p_{i_x,i_y,i_z,\alpha}^2                                                                                                                                                           \\
  +  & \frac{1}{2}\kappa\sum_{i_x=0}^{N}\sum_{i_y=0}^{N}\sum_{i_z=0}^{N}\sum_{\alpha=x,y,z}\qty((q_{i_x,i_y,i_z,\alpha} - q_{i_x+1,i_y,i_z,\alpha})^2 + (q_{i_x,i_y,i_z,\alpha} - q_{i_x,i_y+1,i_z,\alpha})^2 + (q_{i_x,i_y,i_z,\alpha} - q_{i_x,i_y,i_z+1,\alpha})^2)   \\
  =  & \frac{1}{2m}\sum_{j_x=1}^{N}\sum_{j_y=1}^{N}\sum_{j_z=1}^{N}\sum_{\alpha=x,y,z}P_{j_x,j_y,j_z,\alpha}^2                                                                                                                                                           \\
  +  & 2\kappa\sum_{i_x=0}^{N}\sum_{i_y=0}^{N}\sum_{i_z=0}^{N}\sum_{\alpha=x,y,z}\qty(\sin^2\qty(\frac{\pi}{2(N+1)}j_x)Q_{j_x,j_y,j_z,\alpha}^2 + \sin^2\qty(\frac{\pi}{2(N+1)}j_y)Q_{j_x,j_y,j_z,\alpha}^2 + \sin^2\qty(\frac{\pi}{2(N+1)}j_z)Q_{j_x,j_y,j_z,\alpha}^2) \\
  =  & \sum_{j_x=1}^{N}\sum_{j_y=1}^{N}\sum_{j_z=1}^{N}\sum_{\alpha=x,y,z}\qty(\frac{1}{2m}P_{j_x,j_y,j_z,\alpha}^2 + 2\kappa\qty(\sin^2\qty(\frac{\pi}{2(N+1)}j_x) + \sin^2\qty(\frac{\pi}{2(N+1)}j_y) + \sin^2\qty(\frac{\pi}{2(N+1)}j_z))Q_{j_x,j_y,j_z,\alpha}^2)    \\
  =  & \sum_{j_x=1}^{N}\sum_{j_y=1}^{N}\sum_{j_z=1}^{N}\sum_{\alpha=x,y,z}\qty(\frac{1}{2m}P_{j_x,j_y,j_z,\alpha}^2 + \frac{1}{2}m\omega_{j_x,j_y,j_z}^2Q_{j_x,j_y,j_z,\alpha}^2).
\end{align}
ただし, $\omega_{j_x,j_y,j_z}$ は次のように定めた.
\begin{align}
  \omega_{j_x,j_y,j_z} & = 2\sqrt{\frac{\kappa}{m}}\sqrt{\sin^2\qty(\frac{\pi}{2(N+1)}j_x) + \sin^2\qty(\frac{\pi}{2(N+1)}j_y) + \sin^2\qty(\frac{\pi}{2(N+1)}j_z)}.
\end{align}

これより 3 次元結晶の模型の基準振動は位置や運動量に独立な角振動数 $\omega_{j_x,j_y,j_z}$ の調和振動子となることがわかった.

\begin{itembox}[l]{Q 17-8.}
  3 次元結晶中の波数 $k$ における分散関係 $\omega(k)$ は次のように表される.
  \begin{align}
    \omega(\bm{k}) & = 2\sqrt{\frac{\kappa}{m}}\sqrt{\sin^2\qty(\frac{a}{2}k_x) + \sin^2\qty(\frac{a}{2}k_y) + \sin^2\qty(\frac{a}{2}k_z)} \approx \sqrt{\frac{\kappa}{m}}a|\bm{k}| + \mathcal{O}(|\bm{k}|^3) \qquad (a|\bm{k}| \ll 1).
  \end{align}
\end{itembox}

(i) 3 次元結晶の模型の基準振動は角振動数 $\omega_{j_x,j_y,j_z}$ に依存し, それに対する波数 $\bm{k}_{j_x,j_y,j_z} = (k_{j_x}, k_{j_y}, k_{j_z})$ を考えると次のようになる.
\begin{align}
  \omega_{j_x,j_y,j_z} & = 2\sqrt{\frac{\kappa}{m}}\sqrt{\sin^2\qty(\frac{\pi}{2(N+1)}j_x) + \sin^2\qty(\frac{\pi}{2(N+1)}j_y) + \sin^2\qty(\frac{\pi}{2(N+1)}j_z)} \\
                       & = 2\sqrt{\frac{\kappa}{m}}\sqrt{\sin^2\qty(\frac{a}{2}k_{j_x}) + \sin^2\qty(\frac{a}{2}k_{j_y}) + \sin^2\qty(\frac{a}{2}k_{j_z})}.
\end{align}
これより基準振動に対する波数 $\bm{k}_{j_x,j_y,j_z}$ は次のように定められる.
\begin{align}
  \bm{k}_{j_x,j_y,j_z} & = \frac{\pi}{a(N+1)}(j_x,j_y,j_z).
\end{align}

(ii) このように定めた波数を連続的に捉え直すことで分散関係 $\omega(\bm{k})$ は波数 $\bm{k} = (k_x, k_y, k_z)$ を用いて次のようになる.
\begin{align}
  \omega(\bm{k}) & = 2\sqrt{\frac{\kappa}{m}}\sqrt{\sin^2\qty(\frac{a}{2}k_x) + \sin^2\qty(\frac{a}{2}k_y) + \sin^2\qty(\frac{a}{2}k_z)}.
\end{align}

(iii) このとき長波長 ($a|\bm{k}| \ll 1$) では分散関係は次の線形関係となることがわかる.
\begin{align}
  \omega(\bm{k}) & = 2\sqrt{\frac{\kappa}{m}}\sqrt{\sin^2\qty(\frac{a}{2}k_x) + \sin^2\qty(\frac{a}{2}k_y) + \sin^2\qty(\frac{a}{2}k_z)}                                                          \\
                 & \approx 2\sqrt{\frac{\kappa}{m}}\sqrt{\qty(\frac{a}{2}k_x + \mathcal{O}(k_x^3))^2 + \qty(\frac{a}{2}k_y + \mathcal{O}(k_y^3))^2 + \qty(\frac{a}{2}k_z + \mathcal{O}(k_z^3))^2} \\
                 & = 2\sqrt{\frac{\kappa}{m}}\sqrt{\qty(\frac{a}{2}|\bm{k}|)^2 + \mathcal{O}(|\bm{k}|^4)}                                                                                         \\
                 & = 2\sqrt{\frac{\kappa}{m}}\qty(\frac{a}{2}|\bm{k}|\sqrt{1 + \mathcal{O}(|\bm{k}|^2)})                                                                                          \\
                 & \approx \sqrt{\frac{\kappa}{m}}a|\bm{k}| + \mathcal{O}(|\bm{k}|^3) \qquad (a|\bm{k}| \ll 1).
\end{align}

(iv) これより音速 $v$ はその定義式から次のようになる.
\begin{align}
  v & = \lim_{|\bm{k}|\to 0}\frac{\omega}{|\bm{k}|} = \sqrt{\frac{\kappa}{m}}a.
\end{align}

\begin{itembox}[l]{Q 17-9.}
  3 次元結晶の模型における調和振動子の角振動数の個数分布関数 $g(\omega)$ は次のように表される.
  \begin{align}
    g(\omega) & = 3\sum_{j_x=1}^{N}\sum_{j_y=1}^{N}\sum_{j_z=1}^{N}\delta(\omega - \omega(\bm{k}_{j_x,j_y,j_z})).
  \end{align}
\end{itembox}

(i) 調和振動子の角振動数 $\omega(\bm{k}_{j_x, j_y, j_z})$ の個数分布関数 $g(\omega)$ について $\omega(\bm{k}_{j_x, j_y, j_z})$ は離散的な値を持ち, 各基準モード $(j_x, j_y, j_z, \alpha)$ によってパラメータ化されるのでデルタ関数を用いて次のように表される.
\begin{align}
  g(\omega) & = \sum_{j_x=1}^{N}\sum_{j_y=1}^{N}\sum_{j_z=1}^{N}\sum_{\alpha=x,y,z}\delta(\omega - \omega(\bm{k}_{j_x,j_y,j_z})) \\
            & = 3\sum_{j_x=1}^{N}\sum_{j_y=1}^{N}\sum_{j_z=1}^{N}\delta(\omega - \omega(\bm{k}_{j_x,j_y,j_z})).
\end{align}
また $\omega(\bm{k}_{j_x,j_y,j_z})$ は $\omega(\bm{k}_{j_x,j_y,j_z})\geq 0$ に限られるから $\omega\geq 0$ となる.

(ii) これより調和振動子の総数は次のようになる.
\begin{align}
  \int_0^\infty\dd{\omega}g(\omega) & = 3\int_0^\infty\dd{\omega}\sum_{j_x=1}^{N}\sum_{j_y=1}^{N}\sum_{j_z=1}^{N}\delta(\omega - \omega(\bm{k}_{j_x,j_y,j_z})) \\
                                    & = 3N^3.
\end{align}

ただこのような調和振動子の角振動数の個数分布関数 $g(\omega)$ をさらに簡単にすることは分散関係 $\omega(\bm{k})$ の複雑さのためにできない為, これに統計力学を適用しても計算がすぐに行き詰まる.

Debye はこの模型を修正することでこの困難を打開した. 新しい模型には解析計算ができるという要請と十分に低温であるか, あるいは十分に高温であるかという温度に関する両極端な漸近領域においてこれまでの模型と同じ結果を導くという要請をした.

\subsection{Debye模型}
以下では独立な調和振動子の角振動数に関する個数分布関数 $g(\omega)$ を解析的に計算できるよう分散関係を修正した新しい模型を考える. これを Debye 模型という.

\begin{itembox}[l]{Q 17-10.}
  十分に高温において前節の模型と新しい模型が同じ比熱の極限値を持つには独立な調和振動子の総数について一致することが必要十分である.
\end{itembox}

十分に高温ではエントロピーが高くなる為, すべての独立な調和振動子のエネルギー状態について実現確率は等分配される. このとき比熱は独立な調和振動子の総数のみに依存するから前節の模型と等しい総数となることが必要十分である.

\begin{itembox}[l]{Q 17-11.}
  十分に低温において前節の模型と新しい模型が同じ比熱の漸近的な振る舞いを示すためには分散関係の関数 $\omega(\bm{k})$ が長波長の漸近領域 $a|\bm{k}| \ll 1$ において一致することが十分である.
\end{itembox}

十分に低温ではエントロピーが低くなり, エネルギーが低い状態, つまり長波長に関する状態に実現確率が集まるので, 前節の模型と新しい模型について長波長の漸近領域において分散関係が一致するなら同じ比熱の漸近的な振る舞いとなることが言える. \\

これらより Debye 模型では独立な調和振動子の総数が $3N^3$ で調和振動子の角振動数 $\omega(\bm{k})$ は次のように定義する.
\begin{align}
  \omega(\bm{k}) & := \sqrt{\frac{\kappa}{m}}a|\bm{k}|.
\end{align}
また新しい模型の固有モードのラベルは前節と同じく $(j_x, j_y, j_z, \alpha)$ $(j_x,j_y,j_x=1,\ldots,N,\alpha=x,y,z)$ とし, 固有モード $(j_x, j_y, j_z, \alpha)$ の空間的な波数 $\bm{k}_{j_x,j_y,j_z}$ は次のように与えられる.
\begin{align}
  \bm{k}_{j_x,j_y,j_z} & = \frac{\pi}{a(N+1)}(j_x,j_y,j_z).
\end{align}

\begin{itembox}[l]{Q 17-12.}
  Debye 模型における調和振動子の角振動数の個数分布関数 $g(\omega)$ は次のように表される.
  \begin{align}
    g(\omega) & = \begin{dcases}
                    \frac{9N^3}{\omega_D}\qty(\frac{\omega}{\omega_D})^2 & (\omega\leq\omega_D) \\
                    0                                                    & (\omega > \omega_D)
                  \end{dcases} \\
    \omega_D  & = (6\pi^2)^{1/3}\sqrt{\frac{\kappa}{m}}.
  \end{align}
\end{itembox}

(i) Debye 模型における調和振動子の角振動数の個数分布関数 $g(\omega)$ は $\omega(\bm{k}_{j_x, j_y, j_z})$ が固有モード $(j_x, j_y, j_z, \alpha)$ によってパラメータ化されるのでデルタ関数を用いて次のように表される.
\begin{align}
  g(\omega) & = \sum_{j_x=1}^{N}\sum_{j_y=1}^{N}\sum_{j_z=1}^{N}\sum_{\alpha=x,y,z}\delta(\omega - \omega(\bm{k}_{j_x,j_y,j_z}))      \\
            & = 3\sum_{j_x=1}^{N}\sum_{j_y=1}^{N}\sum_{j_z=1}^{N}\delta(\omega - \omega(\bm{k}_{j_x,j_y,j_z})) \qquad (\omega\geq 0).
\end{align}

(ii) また調和振動子の総数は 3 次元結晶の模型と同様に $3N^3$ となる.
\begin{align}
  \int_0^\infty\dd{\omega}g(\omega) & = 3\int_0^\infty\dd{\omega}\sum_{j_x=1}^{N}\sum_{j_y=1}^{N}\sum_{j_z=1}^{N}\sum_{\alpha=x,y,z}\delta(\omega - \omega(\bm{k}_{j_x,j_y,j_z})) \\
                                    & = 3N^3.
\end{align}

(iii) ここでDebye 模型における調和振動子の角振動数の個数分布関数 $g(\omega)$ を具体的に計算すると次のようになる.
\begin{align}
  g(\omega) & = 3\sum_{j_x=1}^{N}\sum_{j_y=1}^{N}\sum_{j_z=1}^{N}\delta(\omega - \omega(\bm{k}_{j_x,j_y,j_z}))                                                                                  \\
            & = 3\sum_{j_x=1}^{N}\sum_{j_y=1}^{N}\sum_{j_z=1}^{N}\delta\qty(\omega - \sqrt{\frac{\kappa}{m}}a\qty|\frac{\pi}{a(N+1)}(j_x,j_y,j_z)|)                                             \\
            & = 3\sum_{j_x=1}^{N}\sum_{j_y=1}^{N}\sum_{j_z=1}^{N}\delta\qty(\omega - \sqrt{\frac{\kappa}{m}}\frac{\pi}{N+1}\sqrt{j_x^2 + j_y^2 + j_z^2})                                        \\
            & = 3\sqrt{\frac{m}{\kappa}}\frac{N+1}{\pi}\sum_{j_x=1}^{N}\sum_{j_y=1}^{N}\sum_{j_z=1}^{N}\delta\qty(\sqrt{\frac{m}{\kappa}}\frac{N+1}{\pi}\omega - \sqrt{j_x^2 + j_y^2 + j_z^2}).
\end{align}

(iv) またデルタ関数を少し広がった有限の Gauss 分布とすることで $g(\omega)$ を滑らかな分布として近似できる. これより総和は次のように積分で置き換えられることが言える.
\begin{align}
  g(\omega) & = 3\sqrt{\frac{m}{\kappa}}\frac{N+1}{\pi}\sum_{j_x=1}^{N}\sum_{j_y=1}^{N}\sum_{j_z=1}^{N}\delta\qty(\sqrt{\frac{m}{\kappa}}\frac{N+1}{\pi}\omega - \sqrt{j_x^2 + j_y^2 + j_z^2})                    \\
            & \approx 3\sqrt{\frac{m}{\kappa}}\frac{N+1}{\pi}\int_{1}^{N}\dd{j_x}\int_{1}^{N}\dd{j_y}\int_{1}^{N}\dd{j_z}\delta\qty(\sqrt{\frac{m}{\kappa}}\frac{N+1}{\pi}\omega - \sqrt{j_x^2 + j_y^2 + j_z^2}).
\end{align}

(v) ここで $\omega$ に関する次の条件が成り立つとする.
\begin{align}
  \sqrt{\frac{m}{\kappa}}\frac{N+1}{\pi}\omega \leq N. \label{omega_condition}
\end{align}
特に $g(\omega)$ の被積分関数の積分値は次のような幾何学的解釈で近似できる.
\begin{align}
          & \int_{1}^{N}\dd{j_x}\int_{1}^{N}\dd{j_y}\int_{1}^{N}\dd{j_z}\delta\qty(\sqrt{\frac{m}{\kappa}}\frac{N+1}{\pi}\omega - \sqrt{j_x^2 + j_y^2 + j_z^2})                             \\
  =       & \int_V\dd{\bm{r}}\delta\qty(|\bm{r}| - \sqrt{\frac{m}{\kappa}}\frac{N+1}{\pi}\omega) \qquad \qty(V := \lbrace (x, y, z)\mid 1\leq x\leq N, 1\leq y\leq N, 1\leq z\leq N\rbrace) \\
  \approx & \qty(半径 \sqrt{\frac{m}{\kappa}}\frac{N+1}{\pi}\omega の 2 次元球面 S_2 を第 1 象限で切り取った曲面の表面積).
\end{align}
これより $g(\omega)$ は次のように書ける.
\begin{align}
  g(\omega) & \approx 3\sqrt{\frac{m}{\kappa}}\frac{N+1}{\pi}\times\qty(半径 \sqrt{\frac{m}{\kappa}}\frac{N+1}{\pi}\omega の 2 次元球面 S_2 を第 1 象限で切り取った曲面の表面積).
\end{align}

(vi) それを具体的に計算すると次のようになる.
\begin{align}
  g(\omega) & \approx 3\sqrt{\frac{m}{\kappa}}\frac{N+1}{\pi}\times\qty(半径 \sqrt{\frac{m}{\kappa}}\frac{N+1}{\pi}\omega の 2 次元球面 S_2 を第 1 象限で切り取った曲面の表面積) \\
            & = 3\sqrt{\frac{m}{\kappa}}\frac{N+1}{\pi}\times\frac{4\pi}{8}\qty(\sqrt{\frac{m}{\kappa}}\frac{N+1}{\pi}\omega)^2                           \\
            & = \frac{3\pi}{2}\qty(\sqrt{\frac{m}{\kappa}}\frac{N+1}{\pi})^3\omega^2.
\end{align}

(vii) $\omega$ に関する条件 \eqref{omega_condition} が成り立たない場合は立方体の積分範囲と球面の表面の共通部分の面積となるので複雑な式となってしまう. ただ Debye 模型は低温における比熱の振る舞いからの要請により $\omega(\bm{k})$ が大きいときは気にしなくて良い模型でした.
これより $g(\omega)$ の $(j_x, j_y, j_z)$ に関する積分範囲を立方体から球へ修正することが許され, 次のように $g(\omega)$ は表される.
\begin{align}
  g(\omega) & = \begin{dcases}
                  \frac{3\pi}{2}\qty(\sqrt{\frac{m}{\kappa}}\frac{N}{\pi})^3\omega^2 & (\omega\leq\omega_D) \\
                  0                                                                  & (\omega > \omega_D)
                \end{dcases}.
\end{align}
ただし $N\gg 1$ であることから $N+1$ を $N$ と近似し, また打ち切る角振動数 $\omega_D$ を次のように定める.
\begin{align}
  \int_0^\infty\dd{\omega}g(\omega) & = \int_0^{\omega_D}\dd{\omega}g(\omega) = 3N^3.
\end{align}
この $\omega_D$ を Debye の角振動数という.

(viii) これより Debye の角振動数 $\omega_D$ は次のように計算される.
\begin{align}
  \int_0^{\omega_D}\dd{\omega}g(\omega) & = \int_0^{\omega_D}\dd{\omega}\frac{3\pi}{2}\qty(\sqrt{\frac{m}{\kappa}}\frac{N}{\pi})^3\omega^2 = \frac{\pi}{2}\qty(\sqrt{\frac{m}{\kappa}}\frac{N}{\pi})^3\omega_D^3 = 3N^3, \\
  \omega_D                              & = \qty(3N^3\frac{2}{\pi})^{1/3}\sqrt{\frac{\kappa}{m}}\frac{\pi}{N} = (6\pi^2)^{1/3}\sqrt{\frac{\kappa}{m}}.
\end{align}

(ix) また Debye の角振動数 $\omega_D$ を用いて $g(\omega)$ は次のように表される.
\begin{align}
  g(\omega) & = \begin{dcases}
                  \frac{3\pi}{2}\qty(\sqrt{\frac{m}{\kappa}}\frac{N}{\pi})^3\omega^2 & (\omega\leq\omega_D) \\
                  0                                                                  & (\omega > \omega_D)
                \end{dcases} \\
            & = \begin{dcases}
                  \frac{9N^3}{\omega_D}\qty(\frac{\omega}{\omega_D})^2 & (\omega\leq\omega_D) \\
                  0                                                    & (\omega > \omega_D)
                \end{dcases}.
\end{align}

現実の物質に Debye 模型を当てはめるときには, それぞれの物質は固有の Debye 角振動数 $\omega_D$ を持つことになる.

\subsection{量子論での基準モード}
今まで古典力学により行ってきた考察を量子力学に翻訳する. まず Debye 模型の Hamilton 関数は次のように与えられる.
\begin{align}
  \hat{H} & = \frac{1}{2m}\sum_{i_x=1}^{N}\sum_{i_y=1}^{N}\sum_{i_z=1}^{N}\sum_{\alpha=x,y,z}\hat{p}_{i_x,i_y,i_z,\alpha}^2                                                                                                                                                                                        \\
          & + \frac{1}{2}\kappa\sum_{i_x=0}^{N}\sum_{i_y=0}^{N}\sum_{i_z=0}^{N}\sum_{\alpha=x,y,z}\qty((\hat{q}_{i_x,i_y,i_z,\alpha} - \hat{q}_{i_x+1,i_y,i_z,\alpha})^2 + (\hat{q}_{i_x,i_y,i_z,\alpha} - \hat{q}_{i_x,i_y+1,i_z,\alpha})^2 + (\hat{q}_{i_x,i_y,i_z,\alpha} - \hat{q}_{i_x,i_y,i_z+1,\alpha})^2).
\end{align}
ただし $m$ は 1 個の原子の質量であり, $\kappa$ は隣り合った原子間の原子間力のバネ定数とする. また立方体の表面は固定されているとする.
\begin{align}
  i_x = 0, N+1\lor i_y = 0, N+1\lor i_z = 0, N+1 \implies \hat{q}_{i_x,i_y,i_z,\alpha} = 0.
\end{align}
また位置演算子 $\hat{q}_{i_x,i_y,i_z,\alpha}$ と運動量演算子 $\hat{p}_{i_x',i_y',i_z',\alpha'}$ は正準交換関係を満たす.
\begin{align}
  \qty[\hat{q}_{i_x,i_y,i_z,\alpha}, \hat{p}_{i_x',i_y',i_z',\alpha'}] & = \sqrt{-1}\hbar\delta_{i_x,i_x'}\delta_{i_y,i_y'}\delta_{i_z,i_z'}\delta_{\alpha,\alpha'}, \\
  \qty[\hat{q}_{i_x,i_y,i_z,\alpha}, \hat{q}_{i_x',i_y',i_z',\alpha'}] & = \qty[\hat{p}_{i_x,i_y,i_z,\alpha}, \hat{p}_{i_x',i_y',i_z',\alpha'}] = 0                  \\
  (1\leq i_x,i_y,i_z,i_x',i_y',i_z'                                    & \leq N, \alpha,\alpha' = x,y,z).
\end{align}
古典論での点正準変換を量子論でも行う. $(\hat{q}_{i_x,i_y,i_z,\alpha}, \hat{p}_{i_x,i_y,i_z,\alpha})_{1\leq i_x,i_y,i_z\leq N,\alpha=x,y,z}\to(\hat{Q}_{j_x,j_y,j_z,\alpha}, \hat{P}_{j_x,j_y,j_z,\alpha})_{1\leq j_x,j_y,j_z\leq N,\alpha=x,y,z}$ を次のように定める.
\begin{align}
  \hat{q}_{i_x,i_y,i_z,\alpha} & = \sum_{j_x=1}^{N}\sum_{j_y=1}^{N}\sum_{j_z=1}^{N}\hat{Q}_{j_x,j_y,j_z,\alpha}q_{i_x}^{(j_x)}q_{i_y}^{(j_y)}q_{i_z}^{(j_z)} \qquad (1\leq i_x,i_y,i_z \leq N, \alpha = x,y,z), \\
  \hat{P}_{j_x,j_y,j_z,\alpha} & = \sum_{i_x=1}^{N}\sum_{i_y=1}^{N}\sum_{i_z=1}^{N}\hat{p}_{i_x,i_y,i_z,\alpha}q_{i_x}^{(j_x)}q_{i_y}^{(j_y)}q_{i_z}^{(j_z)} \qquad (1\leq j_x,j_y,j_z \leq N, \alpha = x,y,z).
\end{align}

\begin{itembox}[l]{Q 17-13.}
  新しい位置演算子 $\hat{Q}_{j_x,j_y,j_z,\alpha}$ と運動量演算子 $\hat{P}_{j_x',j_y',j_z',\alpha'}$ について正準交換関係を満たす.
  \begin{align}
    \qty[\hat{Q}_{j_x,j_y,j_z,\alpha}, \hat{P}_{j_x',j_y',j_z',\alpha'}] & = \sqrt{-1}\hbar\delta_{j_x,j_x'}\delta_{j_y,j_y'}\delta_{j_z,j_z'}\delta_{\alpha,\alpha'}, \\
    \qty[\hat{Q}_{j_x,j_y,j_z,\alpha}, \hat{Q}_{j_x',j_y',j_z',\alpha'}] & = \qty[\hat{P}_{j_x,j_y,j_z,\alpha}, \hat{P}_{j_x',j_y',j_z',\alpha'}] = 0                  \\
    (1\leq j_x,j_y,j_z,j_x',j_y',j_z'                                    & \leq N, \alpha,\alpha' = x,y,z).
  \end{align}
\end{itembox}
まず $\hat{q}_{i_x,i_y,i_z,\alpha}$, $\hat{P}_{j_x',j_y',j_z',\alpha'}$ の交換関係について左を展開するものと右を展開するもので分けて計算すると次のようになる.
\begin{align}
  \qty[\hat{q}_{i_x,i_y,i_z,\alpha}, \hat{P}_{j_x',j_y',j_z',\alpha'}] & = \qty[\hat{q}_{i_x,i_y,i_z,\alpha}, \sum_{i_x'=1}^{N}\sum_{i_y'=1}^{N}\sum_{i_z'=1}^{N}\hat{p}_{i_x',i_y',i_z',\alpha'}q_{i_x'}^{(j_x')}q_{i_y'}^{(j_y')}q_{i_z'}^{(j_z')}]                     \\
                                                                       & = \sum_{i_x'=1}^{N}\sum_{i_y'=1}^{N}\sum_{i_z'=1}^{N}\qty[\hat{q}_{i_x,i_y,i_z,\alpha}, \hat{p}_{i_x',i_y',i_z',\alpha'}]q_{i_x'}^{(j_x')}q_{i_y'}^{(j_y')}q_{i_z'}^{(j_z')}                     \\
                                                                       & = \sum_{i_x'=1}^{N}\sum_{i_y'=1}^{N}\sum_{i_z'=1}^{N}\sqrt{-1}\hbar\delta_{i_x,i_x'}\delta_{i_y,i_y'}\delta_{i_z,i_z'}\delta_{\alpha,\alpha'}q_{i_x'}^{(j_x')}q_{i_y'}^{(j_y')}q_{i_z'}^{(j_z')} \\
                                                                       & = \sqrt{-1}\hbar\delta_{\alpha,\alpha'}q_{i_x}^{(j_x')}q_{i_y}^{(j_y')}q_{i_z}^{(j_z')},                                                                                                         \\
  \qty[\hat{q}_{i_x,i_y,i_z,\alpha}, \hat{P}_{j_x',j_y',j_z',\alpha'}] & = \qty[\sum_{j_x=1}^{N}\sum_{j_y=1}^{N}\sum_{j_z=1}^{N}\hat{Q}_{j_x,j_y,j_z,\alpha}q_{i_x}^{(j_x)}q_{i_y}^{(j_y)}q_{i_z}^{(j_z)}, \hat{P}_{j_x',j_y',j_z',\alpha'}]                              \\
                                                                       & = \sum_{j_x=1}^{N}\sum_{j_y=1}^{N}\sum_{j_z=1}^{N}\qty[\hat{Q}_{j_x,j_y,j_z,\alpha}, \hat{P}_{j_x',j_y',j_z',\alpha'}]q_{i_x}^{(j_x)}q_{i_y}^{(j_y)}q_{i_z}^{(j_z)}.
\end{align}
これより $q_{i_x}^{(j_x)}q_{i_y}^{(j_y)}q_{i_z}^{(j_z)}$ の直交性から次のことがわかる.
\begin{align}
  \qty[\hat{Q}_{j_x,j_y,j_z,\alpha}, \hat{P}_{j_x',j_y',j_z',\alpha'}] & = \sqrt{-1}\hbar\delta_{j_x,j_x'}\delta_{j_y,j_y'}\delta_{j_z,j_z'}\delta_{\alpha,\alpha'}.
\end{align}
同様に $\hat{q}_{i_x,i_y,i_z,\alpha}$ 同士, $\hat{P}_{j_x,j_y,j_z,\alpha}$ 同士の交換関係について計算すると次のようになる.
\begin{align}
  \qty[\hat{q}_{i_x,i_y,i_z,\alpha}, \hat{q}_{i_x',i_y',i_z',\alpha'}] & = \qty[\sum_{j_x=1}^{N}\sum_{j_y=1}^{N}\sum_{j_z=1}^{N}\hat{Q}_{j_x,j_y,j_z,\alpha}q_{i_x}^{(j_x)}q_{i_y}^{(j_y)}q_{i_z}^{(j_z)}, \sum_{j_x'=1}^{N}\sum_{j_y'=1}^{N}\sum_{j_z'=1}^{N}\hat{Q}_{j_x',j_y',j_z',\alpha'}q_{i_x'}^{(j_x')}q_{i_y'}^{(j_y')}q_{i_z'}^{(j_z')}] \\
                                                                       & = \sum_{j_x=1}^{N}\sum_{j_y=1}^{N}\sum_{j_z=1}^{N}\sum_{j_x'=1}^{N}\sum_{j_y'=1}^{N}\sum_{j_z'=1}^{N}\qty[\hat{Q}_{j_x,j_y,j_z,\alpha}, \hat{Q}_{j_x',j_y',j_z',\alpha'}]q_{i_x}^{(j_x)}q_{i_y}^{(j_y)}q_{i_z}^{(j_z)}q_{i_x'}^{(j_x')}q_{i_y'}^{(j_y')}q_{i_z'}^{(j_z')} \\
                                                                       & = 0,                                                                                                                                                                                                                                                                      \\
  \qty[\hat{P}_{j_x,j_y,j_z,\alpha}, \hat{P}_{j_x',j_y',j_z',\alpha'}] & = \qty[\sum_{i_x=1}^{N}\sum_{i_y=1}^{N}\sum_{i_z=1}^{N}\hat{p}_{i_x,i_y,i_z,\alpha}q_{i_x}^{(j_x)}q_{i_y}^{(j_y)}q_{i_z}^{(j_z)}, \sum_{i_x'=1}^{N}\sum_{i_y'=1}^{N}\sum_{i_z'=1}^{N}\hat{p}_{i_x',i_y',i_z',\alpha}q_{i_x'}^{(j_x')}q_{i_y'}^{(j_y')}q_{i_z'}^{(j_z')}]  \\
                                                                       & = \sum_{i_x=1}^{N}\sum_{i_y=1}^{N}\sum_{i_z=1}^{N}\sum_{i_x'=1}^{N}\sum_{i_y'=1}^{N}\sum_{i_z'=1}^{N}\qty[\hat{p}_{i_x,i_y,i_z,\alpha}, \hat{p}_{i_x',i_y',i_z',\alpha}]q_{i_x}^{(j_x)}q_{i_y}^{(j_y)}q_{i_z}^{(j_z)}q_{i_x'}^{(j_x')}q_{i_y'}^{(j_y')}q_{i_z'}^{(j_z')}  \\
                                                                       & = 0.
\end{align}
これより $q_{i_x}^{(j_x)}q_{i_y}^{(j_y)}q_{i_z}^{(j_z)}q_{i_x'}^{(j_x')}q_{i_y'}^{(j_y')}q_{i_z'}^{(j_z')}$ の直交性から次のことがわかる.
\begin{align}
  \qty[\hat{Q}_{j_x,j_y,j_z,\alpha}, \hat{Q}_{j_x',j_y',j_z',\alpha'}] = \qty[\hat{P}_{j_x,j_y,j_z,\alpha}, \hat{P}_{j_x',j_y',j_z',\alpha'}] = 0.
\end{align}
よって示された.
\begin{align}
  \qty[\hat{Q}_{j_x,j_y,j_z,\alpha}, \hat{P}_{j_x',j_y',j_z',\alpha'}] & = \sqrt{-1}\hbar\delta_{j_x,j_x'}\delta_{j_y,j_y'}\delta_{j_z,j_z'}\delta_{\alpha,\alpha'}, \\
  \qty[\hat{Q}_{j_x,j_y,j_z,\alpha}, \hat{Q}_{j_x',j_y',j_z',\alpha'}] & = \qty[\hat{P}_{j_x,j_y,j_z,\alpha}, \hat{P}_{j_x',j_y',j_z',\alpha'}] = 0                  \\
  (1\leq j_x,j_y,j_z,j_x',j_y',j_z'                                    & \leq N, \alpha,\alpha' = x,y,z).
\end{align}

\begin{itembox}[l]{Q 17-14.}
  Hamilton 演算子 $\hat{H}$ は独立な調和振動子の Hamilton 演算子の和となる.
  \begin{align}
    \hat{H} & = \sum_{j_x=1}^{N}\sum_{j_y=1}^{N}\sum_{j_z=1}^{N}\sum_{\alpha=x,y,z}\qty(\frac{1}{2m}\hat{P}_{j_x,j_y,j_z,\alpha}^2 + \frac{1}{2}m\omega_{j_x,j_y,j_z}^2\hat{Q}_{j_x,j_y,j_z,\alpha}^2).
  \end{align}
  ただし $\omega_{j_x,j_y,j_z}$ は次のように与えられる.
  \begin{align}
    \omega_{j_x,j_y,j_z} & = 2\sqrt{\frac{\kappa}{m}}\sqrt{\sin^2\qty(\frac{\pi}{2(N+1)}j_x) + \sin^2\qty(\frac{\pi}{2(N+1)}j_y) + \sin^2\qty(\frac{\pi}{2(N+1)}j_z)}.
  \end{align}
\end{itembox}
Q 17-7 で位置, 運動量が演算子だとしても同様に計算できるよう書いたので同じ結果が得られる. よって Hamilton 演算子は次のように書ける.
\begin{align}
  \hat{H} & = \sum_{j_x=1}^{N}\sum_{j_y=1}^{N}\sum_{j_z=1}^{N}\sum_{\alpha=x,y,z}\qty(\frac{1}{2m}\hat{P}_{j_x,j_y,j_z,\alpha}^2 + \frac{1}{2}m\omega_{j_x,j_y,j_z}^2\hat{Q}_{j_x,j_y,j_z,\alpha}^2).
\end{align}
ただし $\omega_{j_x,j_y,j_z}$ は次のように与えられる.
\begin{align}
  \omega_{j_x,j_y,j_z} & = 2\sqrt{\frac{\kappa}{m}}\sqrt{\sin^2\qty(\frac{\pi}{2(N+1)}j_x) + \sin^2\qty(\frac{\pi}{2(N+1)}j_y) + \sin^2\qty(\frac{\pi}{2(N+1)}j_z)}.
\end{align}

\subsection{Debye 模型による固体の比熱 $C$}
\begin{itembox}[l]{Q 17-15.}
  Debye 模型における内部エネルギーの表式は次のようになる.
  \begin{align}
    U & = U_0 + 9N^3\hbar\omega_DI(\beta\hbar\omega_D).
  \end{align}
  ただし温度 $T$ に依存しない定数のエネルギー $U_0$, $I(b)$ について次のように定められる.
  \begin{align}
    U_0  & = \frac{3}{8}(3N^3)\hbar\omega_D,       \\
    I(b) & = \int_0^1\dd{x}\frac{x^3}{e^{bx} - 1}.
  \end{align}
\end{itembox}
\begin{align}
  U & = \int_0^\infty\dd{\omega}g(\omega)u(\omega)                                                                                                        \\
    & = \int_0^{\omega_D}\dd{\omega}\frac{9N^3}{\omega_D}\qty(\frac{\omega}{\omega_D})^2\qty(\frac{1}{2} + \frac{1}{e^{\beta\hbar\omega} - 1})\hbar\omega \\
    & = 9N^3\hbar\int_0^{\omega_D}\dd{\omega}\qty(\frac{\omega}{\omega_D})^3\qty(\frac{1}{2} + \frac{1}{e^{\beta\hbar\omega} - 1})                        \\
    & = 9N^3\hbar\omega_D\int_0^1\dd{x}\qty(\frac{1}{2} + \frac{1}{e^{\beta\hbar\omega_Dx} - 1})x^3                                                       \\
    & = \frac{3}{8}(3N^3)\hbar\omega_D + 9N^3\hbar\omega_DI(\beta\hbar\omega_D)                                                                           \\
    & = U_0 + 9N^3\hbar\omega_DI(\beta\hbar\omega_D).
\end{align}
ただし温度 $T$ に依存しない定数のエネルギー $U_0$, $I(b)$ について次のように定められる.
\begin{align}
  U_0  & = \frac{3}{8}(3N^3)\hbar\omega_D,       \\
  I(b) & = \int_0^1\dd{x}\frac{x^3}{e^{bx} - 1}.
\end{align}

以下からは $b = \beta\hbar\omega_D = \hbar\omega_D/(k_BT)$ という関係を用いる.

\begin{itembox}[l]{Q 17-16.}
  Debye 模型における比熱 $C$ の表式は次のようになる.
  \begin{align}
    C & = 3nR\cdot(-3)b^2\dv{I(b)}{b}.
  \end{align}
\end{itembox}

比熱の定義式に代入することで次のようになる.
\begin{align}
  C & = \int_0^\infty\dd{\omega}g(\omega)c(\omega)                                                                                                                          \\
    & = \int_0^{\omega_D}\dd{\omega}\frac{9N^3}{\omega_D}\qty(\frac{\omega}{\omega_D})^2k_B\qty(\frac{\beta\hbar\omega e^{\beta\hbar\omega/2}}{e^{\beta\hbar\omega} - 1})^2 \\
    & = 9k_BN^3(\beta\hbar\omega_D)^2\int_0^{\omega_D}\frac{\dd{\omega}}{\omega_D}\qty(\frac{\omega}{\omega_D})^4\frac{ e^{\beta\hbar\omega}}{(e^{\beta\hbar\omega} - 1)^2} \\
    & = 3nR\cdot 3b^2\int_0^1\dd{x}\frac{x^4e^{bx}}{(e^{bx} - 1)^2}                                                                                                         \\
    & = 3nR\cdot (-3)b^2\dv{I(b)}{b}.
\end{align}

\begin{itembox}[l]{Q 17-17.}
  高温の漸近領域 $b\ll 1$ における積分 $I(b)$ は次のように評価できる.
  \begin{align}
    I(b) & = \frac{1}{3b} - \frac{1}{8} + \frac{1}{60}b - \frac{1}{5040}b^3 + \frac{1}{272160}b^5 - \cdots.
  \end{align}
\end{itembox}

(i) $x\ll 1$ において $e^x \approx 1 + x$ と近似できる. これより高温の漸近領域 $b\ll 1$ において $bx \ll 1$ であるから $I(b)$ は次のように近似できる.
\begin{align}
  I(b) & = \int_0^1\dd{x}\frac{x^3}{e^{bx} - 1} \approx \int_0^1\dd{x}\frac{x^3}{bx} = \int_0^1\dd{x}\frac{x^2}{b} = \frac{1}{3b}.
\end{align}

(ii) Bernoulli 数 $B_n$ の定義を用いて次のように計算できる.
\begin{align}
  I(b) & = \int_0^1\dd{x}\frac{x^3}{e^{bx} - 1}                                                           \\
       & = \int_0^1\dd{x}\sum_{n=0}^{\infty}\frac{B_n b^{n-1}}{n!}x^{n+2}                                 \\
       & = \sum_{n=0}^{\infty}\frac{B_n}{(n + 3)n!}b^{n-1}                                                \\
       & = \frac{1}{3b} - \frac{1}{8} + \frac{1}{60}b - \frac{1}{5040}b^3 + \frac{1}{272160}b^5 - \cdots.
\end{align}

\begin{itembox}[l]{Q 17-18.}
  高温の漸近領域 $b\ll 1$ における比熱 $C$ は次のように評価できる.
  \begin{align}
    C & = 3nR\qty(1 - \frac{1}{20}\qty(\frac{\hbar\omega_D}{k_BT})^2 + \frac{1}{560}\qty(\frac{\hbar\omega_D}{k_BT})^4 - \frac{1}{18144}\qty(\frac{\hbar\omega_D}{k_BT})^6 + \cdots).
  \end{align}
\end{itembox}

(i) まず Q 17-17(i) の結果を比熱の表式に適用すると次のようになる.
\begin{align}
  C & = 3nR\cdot (-3)b^2\dv{I(b)}{b} \approx 3nR\cdot (-3)b^2\dv{b}(\frac{1}{3b}) = 3nR.
\end{align}

(ii) 次に Q 17-17(ii) の結果を比熱の表式に適用すると次のようになる.
\begin{align}
  C & = 3nR\cdot (-3)b^2\dv{I(b)}{b}                                                                                                                                                \\
    & \approx 3nR\cdot (-3)b^2\dv{b}(\frac{1}{3b} - \frac{1}{8} + \frac{1}{60}b - \frac{1}{5040}b^3 + \frac{1}{272160}b^5 - \cdots)                                                 \\
    & = 3nR\cdot (-3)b^2\qty(-\frac{1}{3b^2} + \frac{1}{60} - \frac{1}{1680}b^2 + \frac{1}{54432}b^4 - \cdots)                                                                      \\
    & = 3nR\qty(1 - \frac{1}{20}b^2 + \frac{1}{560}b^4 - \frac{1}{18144}b^6 + \cdots)                                                                                               \\
    & = 3nR\qty(1 - \frac{1}{20}\qty(\frac{\hbar\omega_D}{k_BT})^2 + \frac{1}{560}\qty(\frac{\hbar\omega_D}{k_BT})^4 - \frac{1}{18144}\qty(\frac{\hbar\omega_D}{k_BT})^6 + \cdots).
\end{align}

\begin{itembox}[l]{Q 17-19.}
  低温の漸近領域 $b\gg 1$ における積分 $I(b)$ は次のように評価できる.
  \begin{align}
    I(b) & \approx \frac{\pi^4}{15}\frac{1}{b^4}.
  \end{align}
\end{itembox}

(i)
初項 $e^{-bx}$ 公比 $e^{-bx}$ の無限等比数列の和は $1/(e^{bx} + 1)$ である. これより $I(b)$ は次のように表される.
\begin{align}
  I(b) & = \int_0^1\dd{x}\frac{x^3}{e^{bx} - 1} = \int_0^1\dd{x}x^3\sum_{n=1}^{\infty}e^{-nbx} = \sum_{n=1}^{\infty}\int_0^1\dd{x}x^3e^{-nbx}.
\end{align}

(ii) これより
\begin{align}
  I(b) & = \sum_{n=1}^{\infty}\int_0^1\dd{x}x^3e^{-nbx}                                   \\
       & = \sum_{n=1}^{\infty}\frac{1}{(nb)^4}\int_0^{nb}\dd{t}t^3e^{-t} \qquad (t = nbx) \\
       & = \sum_{n=1}^{\infty}\frac{1}{(nb)^4}\gamma(4, nb).
\end{align}
ただし, 第一種不完全ガンマ関数 $\gamma(z, p)$ は次の式で定義される.
\begin{align}
  \gamma(z, p) & := \int_0^p\dd{t}t^{z-1}e^{-t}.
\end{align}

(iii)
さらに $I(b)$ は次のように式変形できる.
\begin{align}
  I(b) & = \sum_{n=1}^{\infty}\frac{1}{(nb)^4}\gamma(4, nb)                                                     \\
       & = \sum_{n=1}^{\infty}\frac{1}{(nb)^4}(\Gamma(4) - \Gamma(4, nb))                                       \\
       & = \frac{1}{b^4}\qty(6\sum_{n=1}^{\infty}\frac{1}{n^4} - \sum_{n=1}^{\infty}\frac{1}{n^4}\Gamma(4, nb)) \\
       & = \frac{1}{b^4}\qty(6\zeta(4) - \sum_{n=1}^{\infty}\frac{1}{n^4}\Gamma(4, nb)).
\end{align}
ただし, 第 2 種不完全ガンマ関数 $\Gamma(z, p)$, ガンマ関数 $\Gamma(z)$, ゼータ関数 $\zeta(z)$ は次のように定義される.
\begin{align}
  \Gamma(z, p) & := \int_p^\infty\dd{t}t^{z-1}e^{-t}                               \\
  \Gamma(z)    & := \int_0^\infty\dd{t}t^{z-1}e^{-t} = \gamma(z, p) + \Gamma(z, p) \\
  \zeta(s)     & := \sum_{n=1}^{\infty}\frac{1}{n^s}.
\end{align}

(iv)
ここでゼータ関数 $\zeta(4)$ の値は次の通りとなる.
\begin{align}
  \zeta(4) & = \frac{\pi^4}{90}.
\end{align}
よって $I(b)$ は次のようになる.
\begin{align}
  I(b) & = \frac{1}{b^4}\qty(6\zeta(4) - \sum_{n=1}^{\infty}\frac{1}{n^4}\Gamma(4, nb))         \\
       & = \frac{1}{b^4}\qty(\frac{\pi^4}{15} - \sum_{n=1}^{\infty}\frac{1}{n^4}\Gamma(4, nb)).
\end{align}

(v) 第二種不完全ガンマ関数 $\Gamma(z,p)$ の $p$ の極限について積分範囲が小さくなっていき, 被積分関数は発散しないので次のようになる.
\begin{align}
  \lim_{p\to+\infty}\Gamma(z, p) & = \lim_{p\to+\infty}\int_p^\infty\dd{t}t^{z-1}e^{-t} = 0.
\end{align}

(vi) 低温の漸近領域 $b\gg 1$ において (v) の考察から第二項を無視した近似を行えることがいえる. よって $I(b)$ は次の値となる.
\begin{align}
  I(b) & = \frac{1}{b^4}\qty(\frac{\pi^4}{15} - \sum_{n=1}^{\infty}\frac{1}{n^4}\Gamma(4, nb)) \approx \frac{\pi^4}{15}\frac{1}{b^4}.
\end{align}

\begin{itembox}[l]{Q 17-20.}
  低温の漸近領域 $b\gg 1$ における積分 $I(b)$ はより精密に次のように評価される.
  \begin{align}
    I(b) & \approx \frac{\pi^4}{15}\frac{1}{b^4} - b^3e^{-b}.
  \end{align}
\end{itembox}

(i)
$\Gamma(z, p)$ について部分積分することで次のように書ける.
\begin{align}
   & \quad \Gamma(z, p)                                                                                                                                                                  \\
   & = \int_p^\infty\dd{t}t^{z-1}e^{-t}                                                                                                                                                  \\
   & = -\qty[t^{z-1}e^{-t}]_p^\infty - \qty[(z-1)t^{z-2}e^{-t}]_p^\infty - \cdots - \qty[(z-1)\cdots(z-n)t^{z-n-1}e^{-t}]_p^\infty + \int_p^\infty\dd{t} (z-1)\cdots(z-n)t^{z-n-1}e^{-t} \\
   & = p^{z-1}e^{-p} + (z-1)p^{z-2}e^{-p} + \cdots + (z-1)\cdots(z-n)p^{z-n-1}e^{-p} + \int_p^\infty\dd{t} (z-1)(z-2)\cdots(z-n)t^{z-n-1}e^{-t}                                          \\
   & = p^{z-1}e^{-p}\qty(1 + \sum_{m=1}^{\infty}\frac{1}{p^m}(z-1)(z-2)\cdots(z-m)) \qquad (\because n\to\infty).
\end{align}

(ii)
(i) の結果を用いて $z = 4$ を代入すると次のようになる.
\begin{align}
  \Gamma(4, p) & = p^{3}e^{-p}\qty(1 + \frac{3}{p} + \frac{6}{p^2} + \frac{6}{p^3}) \\
               & = e^{-p}\qty(p^3 + 3p^2 + 6p + 6).
\end{align}

(iii)
これより積分 $I(b)$ の第二種不完全ガンマ関数を展開することで次のようになる.
\begin{align}
  I(b) & = \frac{1}{b^4}\qty(\frac{\pi^4}{15} - \sum_{n=1}^{\infty}\frac{1}{n^4}\Gamma(4, nb))                                                      \\
       & = \frac{1}{b^4}\qty(\frac{\pi^4}{15} - \sum_{n=1}^{\infty}\frac{1}{n^4}e^{-nb}\qty((nb)^3 + 3(nb)^2 + 6nb + 6))                            \\
       & = \frac{1}{b^4}\qty(\frac{\pi^4}{15} - \sum_{n=1}^{\infty}\qty(\frac{b^3}{n} + \frac{3b^2}{n^2} + \frac{6b}{n^3} + \frac{6}{n^4})e^{-nb}).
\end{align}

(iv)
この補正項について次のような不等式が成り立つ.
\begin{align}
  0 < \sum_{n=1}^{\infty}\qty(\frac{b^3}{n} + \frac{3b^2}{n^2} + \frac{6b}{n^3} + \frac{6}{n^4})e^{-nb} < (b^3 + 3b^2 + 6b + 6)\sum_{n=1}^{\infty}e^{-nb} = (b^3 + 3b^2 + 6b + 6)\frac{e^{-b}}{1 - e^{-b}}\sim b^3e^{-b}.
\end{align}
これより上界が指数関数的に小さくなることから $b\gg 1$ のとき $I(b)$ の最低次の漸近評価は十分正確である.
\begin{itembox}[l]{Q 17-21.}
  低温の漸近領域 $b\gg 1$ における比熱 $C$ は次のように評価される.
  \begin{align}
    C & \approx 3nR\times\frac{4\pi^4}{5}\qty(\frac{k_BT}{\hbar\omega_D})^3.
  \end{align}
\end{itembox}
Q 17-19, Q 17-20 で考察したように比熱 $C$ に $I(b)$ の値を代入すると次のようになる.
\begin{align}
  C & = 3nR\cdot(-3)b^2\dv{I(b)}{b}                                  \\
    & = 3nR\cdot(-3)b^2\qty(-\frac{\pi^4}{15}\frac{4}{b^5})          \\
    & = 3nR\times\frac{4\pi^4}{5}\qty(\frac{1}{b})^3                 \\
    & = 3nR\times\frac{4\pi^4}{5}\qty(\frac{k_BT}{\hbar\omega_D})^3.
\end{align}

よって Debye 模型の比熱は次のようにまとめられる.
\begin{itembox}[l]{Debye 模型の比熱}
  \begin{align}
    C & \approx 3nR\times\begin{dcases}
                           1                                                  & (k_BT\gg \hbar\omega_D) \\
                           \frac{4\pi^4}{5}\qty(\frac{k_BT}{\hbar\omega_D})^3 & (k_BT\ll \hbar\omega_D)
                         \end{dcases}.
  \end{align}
\end{itembox}

\section{黒体輻射}


\section{グランドカノニカル分布}

\subsection{Bose 統計と Fermi 統計}


\subsection{相転移}


\end{document}