\RequirePackage{plautopatch}
\documentclass[uplatex,diffipdfmx,a4paper,11pt]{jlreq}
\usepackage{bxpapersize}
\usepackage[utf8]{inputenc}
\usepackage{fontenc}
\usepackage{lmodern}
\usepackage{otf}
\usepackage{amsmath}
\usepackage{amssymb}
\usepackage{amsthm}
\usepackage{ascmac}
% \usepackage[hyphens]{url}
\usepackage{physics2}
\usephysicsmodule{ab, ab.braket, doubleprod, diagmat, xmat}
\usepackage{diffcoeff}
\usepackage{verbatimbox}
\usepackage{bm}
\usepackage{url}
\usepackage{siunitx}
% \usepackage[diffipdfmx,hiresbb,final]{graphicx}
\usepackage{hyperref}
\usepackage{pxjahyper}
\usepackage{tikz}\usetikzlibrary{cd}
\usepackage{listings}
\usepackage{color}
\usepackage{mathtools}
\usepackage{xspace}
\usepackage{xy}
\usepackage{xypic}
%
\title{統計力学}
\author{Anko}
\makeatletter
%
\DeclareMathOperator{\lcm}{lcm}
\DeclareMathOperator{\Kernel}{Ker}
\DeclareMathOperator{\Image}{Im}
\DeclareMathOperator{\ch}{ch}
\DeclareMathOperator{\Aut}{Aut}
\DeclareMathOperator{\Log}{Log}
\DeclareMathOperator{\Arg}{Arg}
\DeclareMathOperator{\sgn}{sgn}
%
\newcommand{\CC}{\mathbb{C}}
\newcommand{\RR}{\mathbb{R}}
\newcommand{\QQ}{\mathbb{Q}}
\newcommand{\ZZ}{\mathbb{Z}}
\newcommand{\NN}{\mathbb{N}}
\newcommand{\FF}{\mathbb{F}}
\newcommand{\PP}{\mathbb{P}}
\newcommand{\GG}{\mathbb{G}}
\newcommand{\TT}{\mathbb{T}}
\newcommand{\calB}{\mathcal{B}}
\newcommand{\calF}{\mathcal{F}}
\newcommand{\ignore}[1]{}
\newcommand{\floor}[1]{\left\lfloor #1 \right\rfloor}
% \newcommand{\abs}[1]{\left\lvert #1 \right\rvert}
\newcommand{\lt}{<}
\newcommand{\gt}{>}
\newcommand{\id}{\mathrm{id}}
\newcommand{\rot}{\curl}
\renewcommand{\angle}[1]{\left\langle #1 \right\rangle}
\newcommand{\EE}{\bm{E}}
\newcommand{\BB}{\bm{B}}
\renewcommand{\AA}{\bm{A}}
\newcommand{\rr}{\bm{r}}
\newcommand{\kk}{\bm{k}}
\newcommand{\pp}{\bm{p}}
\newcommand\mqty[1]{\begin{pmatrix}#1\end{pmatrix}}
\newcommand\vmqty[1]{\begin{vmatrix}#1\end{vmatrix}}

\let\oldcite=\cite
\renewcommand\cite[1]{\hyperlink{#1}{\oldcite{#1}}}

\let\oldbibitem=\bibitem
\renewcommand{\bibitem}[2][]{\label{#2}\oldbibitem[#1]{#2}}

% theorem環境の設定
% - 冒頭に改行
% - 末尾にdiamond (amsthm)
\theoremstyle{definition}
\newcommand*{\newscreentheoremx}[2]{
  \newenvironment{#1}[1][]{
    \begin{screen}
    \begin{#2}[##1]
      \leavevmode
      \newline
  }{
    \end{#2}
    \end{screen}
  }
}
\newcommand*{\newqedtheoremx}[2]{
  \newenvironment{#1}[1][]{
    \begin{#2}[##1]
      \leavevmode
      \newline
      \renewcommand{\qedsymbol}{\(\diamond\)}
      \pushQED{\qed}
  }{
      \qedhere
      \popQED
    \end{#2}
  }
}
\newtheorem{theorem*}{定理}

\newqedtheoremx{theorem}{theorem*}
\newcommand*\newqedtheorem@unstarred[2]{%
  \newtheorem{#1*}[theorem*]{#2}
  \newqedtheoremx{#1}{#1*}
}
\newcommand*\newqedtheorem@starred[2]{%
  \newtheorem*{#1*}{#2}
  \newqedtheoremx{#1}{#1*}
}
\newcommand*{\newqedtheorem}{\@ifstar{\newqedtheorem@starred}{\newqedtheorem@unstarred}}

\newtheorem{sctheorem*}{定理}
\newscreentheoremx{sctheorem}{sctheorem*}
\newcommand*\newscreentheorem@unstarred[2]{%
  \newtheorem{#1*}[theorem*]{#2}
  \newscreentheoremx{#1}{#1*}
}
\newcommand*\newscreentheorem@starred[2]{%
  \newtheorem*{#1*}{#2}
  \newscreentheoremx{#1}{#1*}
}
\newcommand*{\newscreentheorem}{\@ifstar{\newscreentheorem@starred}{\newscreentheorem@unstarred}}

%\newtheorem*{definition}{定義}
%\newtheorem{theorem}{定理}
%\newtheorem{proposition}[theorem]{命題}
%\newtheorem{lemma}[theorem]{補題}
%\newtheorem{corollary}[theorem]{系}

\newqedtheorem{lemma}{補題}
\newqedtheorem{corollary}{系}
\newqedtheorem{example}{例}
\newqedtheorem{proposition}{命題}
\newqedtheorem{remark}{注意}
\newqedtheorem{thesis}{主張}
\newqedtheorem{notation}{記法}
\newqedtheorem{problem}{問題}
\newqedtheorem{algorithm}{アルゴリズム}

\newscreentheorem*{axiom}{公理}
\newscreentheorem*{definition}{定義}

\renewenvironment{proof}[1][\proofname]{\par
  \normalfont
  \topsep6\p@\@plus6\p@ \trivlist
  \item[\hskip\labelsep{\bfseries #1}\@addpunct{\bfseries}]\ignorespaces\quad\par
}{%
  \qed\endtrivlist\@endpefalse
}
\renewcommand\proofname{証明}

\makeatother

\begin{document}
\maketitle
\tableofcontents
\clearpage

\section{統計力学の基礎}
エルゴード理論により次の原理が成り立つこととする。
\begin{axiom}[等確率の原理]
  孤立系を十分に長時間放置しておくと物体の実現可能な量子状態はエネルギーのゆらぎを除いてすべて等確率で実現する。
\end{axiom}

2 つの系 $A, B$ があるとする。系 $A$ のエネルギー $E_A$ と系 $B$ のエネルギー $E_B$ の和が一定で $A, B$ の間にエネルギーのやり取りができるとする。
\begin{align}
  E_A + E_B = const.
\end{align}

例えると子どもたちが 12 人居て $A$ と $B$ のグループにそれぞれ 4 人、8 人で分ける。そして 6 個あるリンゴを 1 人複数個もらっても良いとして等確率に配ったとき、それぞれのグループに配られるリンゴで最も確率の高いものは何か。
\begin{table}[hbtp]
  \label{table:micro}
  \centering
  \begin{tabular}{|c|c|l|}
    \hline
    A   & B   & 組合せ                                                         \\
    \hline
    0 個 & 6 個 & ${}_4H_0\times {}_{8}H_6 = {}_3C_0\times {}_{13}C_6 = 1716$ \\
    1 個 & 5 個 & ${}_4H_1\times {}_{8}H_5 = {}_4C_1\times {}_{12}C_5 = 3168$ \\
    2 個 & 4 個 & ${}_4H_2\times {}_{8}H_4 = {}_5C_2\times {}_{11}C_4 = 3300$ \\
    3 個 & 3 個 & ${}_4H_3\times {}_{8}H_3 = {}_6C_3\times {}_{10}C_3 = 2406$ \\
    4 個 & 2 個 & ${}_4H_4\times {}_{8}H_2 = {}_7C_4\times {}_{9}C_2 = 1260$  \\
    5 個 & 1 個 & ${}_4H_5\times {}_{8}H_1 = {}_8C_5\times {}_{8}C_1 = 448$   \\
    6 個 & 0 個 & ${}_4H_6\times {}_{8}H_0 = {}_9C_6\times {}_{7}C_0 = 84$    \\
    \hline
  \end{tabular}
  \caption{組合せ}
\end{table}

より $A$, $B$ のグループにそれぞれ 2 個、4 個で分ける確率が最も高い。
この分布を二項分布という。

\begin{proposition}
  二項分布の極限が正規分布である。
\end{proposition}
\begin{proof}
\end{proof}

\begin{definition}
  あるエネルギー $E$ のときに実現可能な量子状態数を $W(E)$ とおき、その対数を取ったものをエントロピー $S(E)$ という。
  \begin{align}
    S(E) & = k_B\log W(E)                     \\
    k_B  & = 1.380658\times 10^{-23} \si{J/K}
  \end{align}
  ただし $k_B$ をボルツマン定数 (Boltzmann constant) という。ある系 $X$ のエネルギーを $E_X$、状態数を $W_X(E_X)$、エントロピーを $S_X(E_X)$ と書くことにする。
\end{definition}
状態数で計算すると指数が出がちなのでエントロピーで計算すると簡単になる。

\begin{theorem}
  $N$ 次元の調和振動子で $E = M\hbar\omega$ とおくと状態数とエントロピーは次のように書ける。
  \begin{align}
    W(E) & = \mqty{M + N - 1                                                                       \\ N - 1} \\
    S(E) & \approx k_BN\ab(\ab(1 + \frac{M}{N})\log(1 + \frac{M}{N}) - \frac{M}{N}\log\frac{M}{N})
  \end{align}
\end{theorem}
\begin{proof}
  $N$ 次元の調和振動子系では $(n_1,\ldots,n_N)$ が全体の量子状態を決める量子数となる。このときのエネルギーは次のように表される。
  \begin{align}
    E_{(n_1,\ldots,n_N)} & = n_1\hbar\omega + \cdots + n_N\hbar\omega
  \end{align}
  等しいエネルギーの状態の条件は $M = n_1 + \cdots + n_N$ と書ける。これより状態数の組合せは次のように書ける。
  \begin{align}
    W(E) & = \mqty{M + N - 1 \\ N - 1} = \frac{(M + N - 1)!}{(N - 1)!M!}
  \end{align}
  またエントロピーは Stirling の公式 $\log n! \approx n(\log n - 1)$ を用いて
  \begin{align}
    S(E) & = k_B\log W(E)                                                                    \\
         & = k_B\log\frac{(M + N - 1)!}{(N - 1)!M!}                                          \\
         & \approx k_B\ab((N + M)(\log(N + M) - 1) - N(\log N - 1) - M(\log M - 1))          \\
         & = k_BN\ab(\ab(1 + \frac{M}{N})\log(1 + \frac{M}{N}) - \frac{M}{N}\log\frac{M}{N})
  \end{align}
\end{proof}

\begin{theorem}
  熱平衡の条件は系 $A$ の温度 $T_A$ と系 $B$ の温度 $T_B$ が一致すること。
\end{theorem}
\begin{proof}
  各系の状態数の積が全体系の状態数となるので各系と全体系のエントロピーの関係は
  \begin{align}
    S(E_A, E_B) & = k_B\log W(E_A, E_B)                 \\
                & = k_B\log W_A(E_A)W_B(E_B)            \\
                & = k_B\log W_A(E_A) + k_B\log W_B(E_B) \\
                & = S_A(E_A) + S_B(E_B)
  \end{align}
  となる。このとき熱平衡状態とはエントロピーが最大の状態であるから $\diff*{S}{E_A} = 0$ となるエネルギー $E_A, E_B$ を考えると
  \begin{align}
    \diff{S(E_A, E_B)}{E_A}   & = \diff{S_A(E_A)}{E_A} + \diff{S_A(E_B)}{E_A} = \diff{S_A(E_A)}{E_A} - \diff{S_A(E_B)}{E_B} = 0 \\
    \iff \diff{S_A(E_A)}{E_A} & = \diff{S_A(E_B)}{E_B}
  \end{align}
  よりエントロピーのエネルギー微分を温度の逆数 $1/T$ と定義すると温度が一致するときに熱平衡状態となる。
\end{proof}

\begin{definition}
  絶対温度 (absolute temperature) $T$ を次のように定義する。
  \begin{align}
    \frac{1}{T} & := \diff{S}{E}
  \end{align}
\end{definition}

この温度の定義は理想気体で正当化される。

\begin{theorem}[理想気体]
  理想気体、つまり 3 次元箱型ポテンシャル中の独立な区別できない $N$ 個の粒子について
  \begin{align}
    S & = Nk_B\ab(\frac{3}{2}\ln\frac{E}{V} + \frac{5}{2}\ln\frac{V}{N} + \ln\alpha + \mathcal{O}(N^{-1}\ln N)) & \ab(\alpha = \ab(\frac{me}{3\pi\hbar^2})^{\frac{3}{2}}) \\
    E & = \frac{3}{2}Nk_BT
  \end{align}
\end{theorem}
\begin{proof}
  部分系の固有状態と固有エネルギーが分かれば全体系のも分かる。
  \begin{align}
    E_{(n_{i, a})_{i=1,\ldots,N,a=x,y,z}}    & = E_0\sum_{i=1}^{N}\sum_{a=x,y,z}n_{i,a}^2                                               \\
    \psi_{(n_{i, a})_{i=1,\ldots,N,a=x,y,z}} & = \ab(\frac{2}{L})^{3N/2}\prod_{i=1}^{N}\prod_{a=x,y,z}\sin(\frac{n_{i,a}\pi}{L}x_{i,a})
  \end{align}
  これよりあるエネルギー $E > 0$ 以下である区別できる固有状態数 $\Omega(E)$ について
  \begin{align}
    \Omega(E) & = \ab(半径 \sqrt{\frac{E}{E_0}} の 3N 次元超球の第一象限に含まれる格子点の個数)                                                       \\
              & \approx \frac{1}{2^{3N}}\sqrt{\frac{E}{E_0}}^{3N}\frac{\pi^{3N/2}}{(3N/2)!}                                    \\
              & = \frac{1}{(3N/2)!}\frac{\pi^{3N/2}}{2^{3N}}\ab(\frac{2mL^2}{\pi^2\hbar^2})^{3N/2}E^{3N/2}                     \\
              & = \frac{1}{(3N/2)!}\ab(\frac{m}{2\pi\hbar^2})^{3N/2}E^{3N/2}V^N                                                \\
              & = \frac{1}{\sqrt{3\pi N}(3N/2)^{3N/2}e^{-3N/2}}\ab(\frac{m}{2\pi\hbar^2})^{3N/2}E^{3N/2}V^N                    \\
              & = \frac{1}{\sqrt{3\pi N}}N^{N}\ab(\frac{me}{3\pi\hbar^2})^{3N/2}\ab(\frac{E}{V})^{3N/2}\ab(\frac{V}{N})^{5N/2} \\
  \end{align}
  これを区別しないから
  \begin{align}
    \Omega^{区別できない}(E) & =\frac{1}{N!}\Omega(E)                                                                                                                         \\
                       & = \frac{1}{N!}\frac{1}{\sqrt{3\pi N}}N^{N}\ab(\frac{me}{3\pi\hbar^2})^{3N/2}\ab(\frac{E}{V})^{3N/2}\ab(\frac{V}{N})^{5N/2}                     \\
                       & = \frac{1}{\sqrt{2\pi N}N^Ne^{-N}}\frac{1}{\sqrt{3\pi N}}N^{N}\ab(\frac{me}{3\pi\hbar^2})^{3N/2}\ab(\frac{E}{V})^{3N/2}\ab(\frac{V}{N})^{5N/2} \\
                       & = \frac{e^N}{\sqrt{6}\pi N}\ab(\frac{me}{3\pi\hbar^2})^{3N/2}\ab(\frac{E}{V})^{3N/2}\ab(\frac{V}{N})^{5N/2}
  \end{align}
  これよりエントロピーは
  \begin{align}
    S(E) & = k_B\ln\Omega^{区別できない}(E)                                                                                                                    \\
         & = Nk_B\ab(\frac{3}{2}\ln\frac{E}{V} + \frac{5}{2}\ln\frac{V}{N} + \frac{3}{2}\ln(\frac{me}{3\pi\hbar^2}) - \frac{1}{N}\ln(\sqrt{6}\pi N) + 1)
  \end{align}
  よって温度を計算すると式が示せる。
  \begin{align}
    \frac{1}{T} = \diff{S}{E} & = \frac{3}{2}Nk_B\frac{1}{E} \\
    E                         & = \frac{3}{2}Nk_BT
  \end{align}
\end{proof}

\section{ミクロカノニカル分布}
\subsection{ミクロカノニカルアンサンブル}
\begin{axiom}[等重率の原理]
  孤立した物理系 $X$ において、外部から指定されたある狭いエネルギー範囲 $[U - ∆U, U]$ に固有エネルギー $E_i$ が属するような微視的なエネルギー固有状態 $\ket{\phi_i}$ のひとつひとつが実現される等しい確からしさを持っている。
\end{axiom}
エネルギーの低い順にエネルギーシェル $E$ から $E + \Delta E$ までの中の状態を 1 つのグループでまとめてラベル付けする。
\begin{align}
  N = \sum_{l}N_l, \qquad E = \sum_{l}E_lN_l, \qquad W = \prod_{l}\frac{M_l^{N_l}}{N_l!}, \qquad S = k_B\sum_{l}N_l\ab(\log\frac{M_l}{N_l} + 1)
\end{align}

\begin{align}
  \tilde{S}              & = k_B\sum_{l}N_l\ab(\log\frac{M_l}{N_l} + 1) - k_B\alpha\sum_{l}N_l - k_B\beta\sum_{l}E_lN_l \\
  \diffp{\tilde{S}}{N_l} & = 0 \iff \frac{M_l}{N_l} = e^{\alpha + \beta E_l}
\end{align}

\begin{align}
  N & = \sum_{l}M_le^{-\alpha-\beta E_l}    \\
  E & = \sum_{l}M_lE_le^{-\alpha-\beta E_l} \\
  S & = k_B\ab((1 + \alpha)N + \beta E)
\end{align}
エネルギーで微分すると
\begin{align}
  0           & = \sum_{l}M_l\ab(\diff{\alpha}{E} + \diff{\beta}{E}E_l)e^{-\alpha-\beta E_l} = \diff{\alpha}{E}N + \diff{\beta}{E}E \\
  \diff{S}{E} & = k_B\ab(\diff{\alpha}{E}N + \diff{\beta}{E}E + \beta) = k_B\beta
\end{align}
より $\alpha, \beta$ は次のように表される。
\begin{align}
  \beta       & = \frac{1}{k_BT}                            \\
  e^{-\alpha} & = \frac{N}{\sum_{i}e^{-\varepsilon_i/k_BT}}
\end{align}


\subsection{熱と仕事}
\begin{definition}
  内部エネルギー $E(S, V)$ とその束縛変数を変更させたエンタルピー $H(S, p)$ と Helmholtz 自由エネルギー $F(T, V)$ と Gibbs 自由エネルギー $G(T, p)$ を次のように定義する。
  \begin{align}
               & \qquad \dl{E} = T\dl{S} - p\dl{V}  \\
    H = E + pV & \qquad \dl{H} = T\dl{S} + V\dl{p}  \\
    F = E - TS & \qquad \dl{F} = -S\dl{T} - p\dl{V} \\
    G = F + pV & \qquad \dl{G} = -S\dl{T} + V\dl{p}
  \end{align}
\end{definition}
特に扱いやすい変数 $T$, $V$ を持つ Helmholtz 自由エネルギー $F(T, V)$ は重宝される。
\begin{proposition}
  定義より次の関係式を満たす。
  \begin{align}
    T  & = \ab(\diffp{E}{S})_V & -p & = \ab(\diffp{E}{V})_S \\
    T  & = \ab(\diffp{H}{S})_p & V  & = \ab(\diffp{H}{p})_S \\
    -S & = \ab(\diffp{F}{T})_V & -p & = \ab(\diffp{F}{V})_T \\
    -S & = \ab(\diffp{G}{T})_p & V  & = \ab(\diffp{G}{p})_T
  \end{align}
\end{proposition}

\begin{proposition}[Maxwell の関係式]
  \begin{align}
    \ab(\diffp{T}{V})_S  & = -\ab(\diffp{p}{S})_V \\
    \ab(\diffp{T}{p})_S  & = \ab(\diffp{V}{S})_p  \\
    -\ab(\diffp{S}{V})_T & = -\ab(\diffp{p}{T})_V \\
    -\ab(\diffp{S}{p})_T & = \ab(\diffp{V}{T})_p
  \end{align}
\end{proposition}
\begin{proof}

\end{proof}

\begin{theorem}[理想気体の状態方程式]
  \begin{align}
    pV & = Nk_BT
  \end{align}
\end{theorem}
\begin{align}
  S(E, V) & = Nk_B\ab(\frac{3}{2}\ln\frac{E}{V} + \frac{5}{2}\ln\frac{V}{N} + \frac{3}{2}\ln(\frac{me}{3\pi\hbar^2}) - \frac{1}{N}\ln(\sqrt{6}\pi N) + 1) \\
  0       & = Nk_B\ab(\frac{3}{2}\frac{1}{E}\ab(\diffp{E}{V})_S + \frac{1}{V})                                                                            \\
  p       & = -\ab(\diffp{E}{V})_S = \frac{2}{3}\frac{E}{V} = \frac{Nk_BT}{V}                                                                             \\
  pV      & = Nk_BT
\end{align}

\begin{align}
  C_V = \frac{3}{2}R
\end{align}

\begin{definition}[比熱]
  \begin{align}
    C   & = T\diff{S}{T}         \\
    C_X & = \ab(T\diffp{S}{T})_X
  \end{align}
\end{definition}
これ以降の話は熱力学の方で書きたい。

\section{カノニカル分布}
ある温度の環境の中で理想気体や
\subsection{ミクロカノニカル分布からカノニカル分布へ}

\begin{definition}
  \begin{align}
    \beta    & = \frac{1}{k_BT}         \\
    Z(\beta) & = \sum_{i}e^{-\beta E_i}
  \end{align}
\end{definition}

\begin{theorem}
  \begin{align}
    p & = \frac{e^{-\beta E_i}}{Z(\beta)} \\
    F & = -k_BT\ln Z                      \\
    S & =
  \end{align}
\end{theorem}

\begin{theorem}
  $N$ 個の独立な部分系からなる全体系の熱力学量は次のようになる。
  \begin{align}
    Z(\beta) = z(\beta)^N, \qquad F = Nf, \qquad S = Ns, \qquad U = Nu, \qquad C = c
  \end{align}
\end{theorem}

\subsection{二準位系}
絶対温度 $T$ の熱浴に系 $X$ が浸けられている状態として、系 $X$ の Hamilton 演算子 $\hat{h}_X$ の固有状態は $\ket{\varphi_1}$ と $\ket{\varphi_2}$ の 2 つだけであり、$\ket{\varphi_1}$ の固有エネルギーは $E_1$ であり、$\ket{\varphi_2}$ の固有エネルギーは $E_2$ であるとする:
\begin{align}
  \hat{h}_X\ket{\varphi_i} & = E_i\ket{\varphi_i} \qquad (i = 1, 2)。
\end{align}
ただし $0 < E_1 < E_2$ $\beta = 1/k_BT$ とする。

\begin{theorem}[1個の二準位系]
  二準位系における熱力学的量を考える。低温の漸近領域 ($\beta(E_2 - E_1) \gg 1$, $\beta E_1 \gg 1$) と高温の漸近領域 ($\beta(E_2 - E_1) \ll 1$, $\beta E_1 \ll 1$) は次のようになる。
  \begin{align}
    Z(\beta) & = e^{-\beta E_1} + e^{-\beta E_2}                                                                                                          \\
             & = \begin{dcases}
                   e^{-\frac{E_1}{k_BT}} \to 0      & (低温) \\
                   2 - \frac{E_1 + E_2}{k_BT} \to 2 & (高温)
                 \end{dcases}                                                                                                  \\
    F        & = -\frac{1}{\beta}\ln(e^{-\beta E_1} + e^{-\beta E_2})                                                                                     \\
             & = \begin{dcases}
                   E_1 - k_BTe^{-\frac{E_2 - E_1}{k_BT}} \to E_1  & (低温) \\
                   \frac{1}{2}(E_1 + E_2) - k_BT\ln 2 \to -\infty & (高温)
                 \end{dcases}                                                                                    \\
    S        & = k_B\ab(\ln(e^{-\beta E_1} + e^{-\beta E_2}) + \frac{\beta E_1e^{-\beta E_1} + \beta E_2e^{-\beta E_2}}{e^{-\beta E_1} + e^{-\beta E_2}}) \\
             & = \begin{dcases}
                   k_B \frac{E_2 - E_1}{k_BT}e^{- \frac{E_2 - E_1}{k_BT}} \to 0          & (低温) \\
                   k_B\ab(\ln 2 - \frac{1}{4}\ab(\frac{E_2 - E_1}{k_BT})^2) \to k_B\ln 2 & (高温)
                 \end{dcases}                                                     \\
    U        & = \frac{E_1e^{-\beta E_1} + E_2e^{-\beta E_2}}{e^{-\beta E_1} + e^{-\beta E_2}}                                                            \\
             & = \begin{dcases}
                   E_1 + (E_2 - E_1)e^{- \frac{E_2 - E_1}{k_BT}} \to E_1                                     & (低温) \\
                   \frac{1}{2}(E_1 + E_2) - \frac{1}{4}\frac{(E_2 - E_1)^2}{k_BT} \to \frac{1}{2}(E_1 + E_2) & (高温)
                 \end{dcases}                                        \\
    C        & = k_B\ab(\frac{\frac{1}{2}\beta(E_2 - E_1)}{\cosh\frac{1}{2}\beta(E_2 - E_1)})^2                                                           \\
             & = \begin{dcases}
                   k_B\ab(\frac{E_2 - E_1}{k_BT})^2e^{-\frac{E_2 - E_1}{k_BT}} \to 0 & (低温) \\
                   \frac{k_B}{4}\ab(\frac{E_2 - E_1}{k_BT})^2 \to 0                  & (高温)
                 \end{dcases}
  \end{align}
  TODO: グラフ
\end{theorem}
\begin{proof}
  $x \to 0$ において $(1 + x)^{-1} \approx 1 - x$, $e^x \approx 1 + x$ と近似できる。
  \begin{align}
    Z(\beta) & = \sum_{i} e^{-\beta E_i} = e^{-\beta E_1} + e^{-\beta E_2} \\
             & = e^{-\beta E_1}(1 + e^{-\beta (E_2 - E_1)})                \\
             & \approx \begin{dcases}
                         e^{-\frac{E_1}{k_BT}}                 \\
                         (1 - \beta E_1)(2 -\beta (E_2 - E_1)) \\
                       \end{dcases}               \\
             & = \begin{dcases}
                   e^{-\frac{E_1}{k_BT}} \to 0      & (低温) \\
                   2 - \frac{E_1 + E_2}{k_BT} \to 2 & (高温)
                 \end{dcases}
  \end{align}
  \begin{align}
    F & = -k_BT\ln Z(\beta)                                                                                                                                                                \\
      & = -\frac{1}{\beta}\ln(e^{-\beta E_1} + e^{-\beta E_2})                                                                                                                             \\
      & = \begin{dcases}
            -\frac{1}{\beta}\ln e^{-\beta E_1}(1 + e^{-\beta (E_2 - E_1)}) \\
            -\frac{1}{\beta}\ln e^{-\frac{1}{2}\beta (E_1 + E_2)}(e^{\frac{1}{2}\beta (E_2 - E_1)} + e^{-\frac{1}{2}\beta (E_2 - E_1)})
          \end{dcases} \\
      & \approx \begin{dcases}
                  E_1 - \frac{1}{\beta}e^{-\beta (E_2 - E_1)} \approx E_1 - k_BTe^{-\frac{E_2 - E_1}{k_BT}} \to E_1 \\
                  \frac{1}{2}(E_1 + E_2) - k_BT\ln 2 \to -\infty                                                    \\
                \end{dcases}
  \end{align}
  \begin{align}
    S & = - \ab(\diffp{F}{T})_{V,N} = - \ab(\diffp{F}{\beta}\diffp{\beta}{T})_{V,N} = k_B\beta^2\ab(\diffp{F}{\beta})_{V,N}                                                                                                                                                                                                                                                                   \\
      & = k_B\beta^2\ab(\frac{1}{\beta^2}\ln(e^{-\beta E_1} + e^{-\beta E_2}) - \frac{1}{\beta}\frac{-E_1e^{-\beta E_1} - E_2e^{-\beta E_2}}{e^{-\beta E_1} + e^{-\beta E_2}})                                                                                                                                                                                                                \\
      & = k_B\ab(\ln(e^{-\beta E_1} + e^{-\beta E_2}) + \frac{\beta E_1e^{-\beta E_1} + \beta E_2e^{-\beta E_2}}{e^{-\beta E_1} + e^{-\beta E_2}})                                                                                                                                                                                                                                            \\
      & = \begin{dcases}
            k_B\ab(\ln e^{-\beta E_1}(1 + e^{-\beta (E_2 - E_1)}) + \frac{\beta E_1 + \beta E_2e^{-\beta (E_2 - E_1)}}{1 + e^{-\beta (E_2 - E_1)}})                                                                                                                                                   \\
            k_B\ab(\ln e^{-\frac{1}{2}\beta (E_1 + E_2)}(e^{\frac{1}{2}\beta (E_2 - E_1)} + e^{-\frac{1}{2}\beta (E_2 - E_1)}) + \frac{\beta E_1e^{\frac{1}{2}\beta (E_2 - E_1)} + \beta E_2e^{-\frac{1}{2}\beta (E_2 - E_1)}}{e^{\frac{1}{2}\beta (E_2 - E_1)} + e^{-\frac{1}{2}\beta (E_2 - E_1)}}) \\
          \end{dcases} \\
      & \approx
    \begin{dcases}
      k_B\ab(-\beta E_1 + e^{-\beta (E_2 - E_1)} + (\beta E_1 + \beta E_2e^{-\beta (E_2 - E_1)})(1 - e^{-\beta (E_2 - E_1)}))                                \\
      k_B\ab(\ln 2 - \frac{1}{2}\beta (E_1 + E_2) + \frac{\beta}{2}\ab(E_1\ab(1 + \frac{1}{2}\beta (E_2 - E_1)) + E_2\ab(1 - \frac{1}{2}\beta (E_2 - E_1)))) \\
    \end{dcases}                                                                                                                                                                                                                        \\
      & \approx
    \begin{dcases}
      k_B \frac{E_2 - E_1}{k_BT}e^{- \frac{E_2 - E_1}{k_BT}} \to 0          \\
      k_B\ab(\ln 2 - \frac{1}{4}\ab(\frac{E_2 - E_1}{k_BT})^2) \to k_B\ln 2 \\
    \end{dcases}
  \end{align}
  \begin{align}
    U & = F + TS                                                                                                                                                                                      \\
      & = \frac{E_1e^{-\beta E_1} + E_2e^{-\beta E_2}}{e^{-\beta E_1} + e^{-\beta E_2}}                                                                                                               \\
      & = \begin{dcases}
            \frac{E_1 + E_2e^{-\beta (E_2 - E_1)}}{1 + e^{-\beta (E_2 - E_1)}}                                                                                      \\
            \frac{E_1e^{\frac{1}{2}\beta (E_2 - E_1)} + E_2e^{-\frac{1}{2}\beta (E_2 - E_1)}}{e^{\frac{1}{2}\beta (E_2 - E_1)} + e^{-\frac{1}{2}\beta (E_2 - E_1)}} \\
          \end{dcases} \\
      & \approx \begin{dcases}
                  (E_1 + E_2e^{-\beta (E_2 - E_1)})(1 - e^{-\beta (E_2 - E_1)})                                       \\
                  \frac{1}{2}\ab(E_1\ab(1 + \frac{1}{2}\beta (E_2 - E_1)) + E_2\ab(1 - \frac{1}{2}\beta (E_2 - E_1))) \\
                \end{dcases}                                                                             \\
      & \approx \begin{dcases}
                  E_1 + (E_2 - E_1)e^{- \frac{E_2 - E_1}{k_BT}} \to E_1                                     \\
                  \frac{1}{2}(E_1 + E_2) - \frac{1}{4}\frac{(E_2 - E_1)^2}{k_BT} \to \frac{1}{2}(E_1 + E_2) \\
                \end{dcases}
  \end{align}
  \begin{align}
    C & = \diffp{U}{T} = \diffp{U}{\beta}\diffp{\beta}{T} = -k_B\beta^2\diffp{U}{\beta}                                                                                                 \\
      & = -k_B\beta^2\diffp{\beta}\ab(\frac{E_1 + E_2e^{\beta(E_1 - E_2)}}{1 + e^{\beta(E_1 - E_2)}})                                                                                   \\
      & = -k_B\beta^2\frac{E_2(E_1 - E_2)e^{\beta(E_1 - E_2)}(1 + e^{\beta(E_1 - E_2)}) - (E_1 + E_2e^{\beta(E_1 - E_2)})(E_1 - E_2)e^{\beta(E_1 - E_2)}}{(1 + e^{\beta(E_1 - E_2)})^2} \\
      & = k_B\beta^2\frac{(E_2 - E_1)^2e^{\beta(E_1 - E_2)}}{(1 + e^{\beta(E_1 - E_2)})^2} = k_B\ab(\frac{\frac{1}{2}\beta(E_2 - E_1)}{\cosh\frac{1}{2}\beta(E_2 - E_1)})^2             \\
      & = \begin{dcases}
            k_B\ab(\frac{\beta(E_2 - E_1)}{1 + e^{-\beta(E_2 - E_1)}})^2e^{-\beta(E_2 - E_1)} \\
            k_B\ab(\frac{\beta(E_2 - E_1)}{e^{\frac{1}{2}\beta(E_2 - E_1)} + e^{-\frac{1}{2}\beta(E_2 - E_1)}})^2
          \end{dcases}                                                          \\
      & \approx \begin{dcases}
                  k_B\ab(\frac{E_2 - E_1}{k_BT})^2e^{-\frac{E_2 - E_1}{k_BT}} \to 0 \\
                  \frac{k_B}{4}\ab(\frac{E_2 - E_1}{k_BT})^2 \to 0
                \end{dcases}
  \end{align}
  各固有状態の実現確率について高温極限 ($\beta(E_2 - E_1) \ll 1$) のときそれぞれの固有状態は同じ確率で実現し、低温極限 ($\beta(E_2 - E_1) \gg 1$) のとき固有エネルギーの低い固有状態にほぼ確実に実現する。
  \begin{align}
    \quad p_\beta(i) & = \frac{e^{-\beta (E_i - E_1)}}{1 + e^{-\beta(E_2 - E_1)}} \approx \begin{dcases}
                                                                                            e^{-\beta (E_i - E_1)} & (\beta(E_2 - E_1) \gg 1) \\
                                                                                            \frac{1}{2}            & (\beta(E_2 - E_1) \ll 1)
                                                                                          \end{dcases}
  \end{align}
  $F = E - TS$ の最小化を考える。低温極限でエントロピーを上げるよりエネルギーが低いものを選んだ方がエネルギーが得となる為に固有エネルギーの低い状態に集まる。高温極限でエントロピーを増大させるとエネルギーが得となる為に半々となる。
\end{proof}

\begin{itembox}[l]{Q 15-2.}
  Q 15-1.では Helmholtz 自由エネルギーを計算して、後は熱力学の公式を用いて計算しましたが、今回は正準集団の理論における固有状態の実現確率を与える確率関数 $p_\beta^{正準}(i)\ (i = 1, 2)$ を計算して、内部エネルギー $u$ とエントロピー $s$ を求める。
\end{itembox}
まず確率関数 $p_\beta(i)$ は定義より次のようになる。
\begin{align}
  p_\beta(i) & = \frac{e^{-\beta E_i}}{z(\beta)}.
\end{align}
内部エネルギー $u$ はエネルギーの平均を取ることで分かる。
\begin{align}
  u & = \sum_i E_ip_\beta^{正準}(i) = \frac{E_1e^{-\beta E_1} + E_2e^{-\beta E_2}}{e^{-\beta E_1} + e^{-\beta E_2}}.
\end{align}
比熱も Q15-1.と同様に求まる。

エントロピー $s$ は Shannon のエントロピーの公式に代入することで求まる。
\begin{align}
  s & = -k_B\sum_{i = 1,2}p_\beta^{正準}(i)\ln p_\beta^{正準}(i)                                                                                      \\
    & = -k_B\sum_{i = 1,2}\frac{e^{-\beta E_i}}{z(\beta)}(- \ln z(\beta) - \beta E_i)                                                             \\
    & = k_B\ab(\ln(e^{-\beta E_1} + e^{-\beta E_2}) + \frac{\beta E_1e^{-\beta E_1} + \beta E_2e^{-\beta E_2}}{e^{-\beta E_1} + e^{-\beta E_2}}).
\end{align}

\begin{itembox}[l]{Q 15-7.}
  比熱について解析せよ。
\end{itembox}

まず比熱について次のように定義した関数 $\phi(x)$ を用いて表される。
\begin{align}
  \phi(x) & := \frac{x}{\cosh x}                                                             \\
  c       & = k_B\ab(\frac{\frac{1}{2}\beta(E_2 - E_1)}{\cosh\frac{1}{2}\beta(E_2 - E_1)})^2 \\
          & = k_B\ab(\phi\ab(\frac{1}{2}\beta(E_2 - E_1)))^2
\end{align}
ここで $x\geq 0$ の範囲において $\phi(x)$ が極大となる $x = x_0$ の値を考える。
\begin{align}
       & \left.\diff{\phi}{x}\right|_{x = x_0} = 0        \\
  \iff & \frac{\cosh x_0 - x_0\sinh x_0}{\cosh^2 x_0} = 0 \\
  \iff & x_0\tanh x_0 = 1                                 \\
  \iff & x_0 = 1.199678640257734\ldots
\end{align}

ただしプログラム \ref{newton} を用いて $x\geq 0$ の範囲で $x_0\tanh x_0 = 1$ は $x_0 = 1.199678640257734\ldots$ のとき満たすことが分かる。これより比熱 $c$ は次のように定義される $T_0$ のときに極大を取る。
\begin{align}
   & x_0 = \frac{1}{2}\beta_0(E_2 - E_1) = \frac{1}{2}\frac{E_2 - E_1}{k_BT_0} \\
   & \frac{k_BT_0}{E_2 - E_1} = \frac{1}{2x_0} =  0.41677827980048\ldots
\end{align}

低温、高温で比熱が 0 となる理由は比熱が $C = \diff{E}{T}$ であることより Q15-3, Q15-4 よりエネルギーの確率が極限的に定数となることから比熱は 0 となることが分かる。

\subsection{調和振動子系の統計力学}

\begin{theorem}
  \begin{align}
    z(\beta) & = \frac{1}{2\sinh \frac{1}{2}\beta\hbar\omega}                                                                     \\
             & \approx \begin{dcases}
                         e^{-\frac{\hbar\omega}{2k_BT}} \to 0 \\
                       \end{dcases}                                                                       \\
    f        & = \frac{1}{\beta}\ln\ab(2\sinh \frac{1}{2}\beta\hbar\omega)                                                        \\
    s        & = k_B\ab(-\ln\ab(2\sinh\frac{1}{2}\beta\hbar\omega) + \frac{1}{2}\beta\hbar\omega\coth\frac{1}{2}\beta\hbar\omega) \\
    u        & = \frac{1}{2}\hbar\omega\coth\frac{1}{2}\beta\hbar\omega                                                           \\
    c        & = k_B\ab(\frac{\frac{1}{2}\beta\hbar\omega}{\sinh\frac{1}{2}\beta\hbar\omega})^2
  \end{align}
\end{theorem}
\begin{proof}
  低温の漸近領域において Q 16-1 の結果は次のように近似できる。ただし、$x\to 0$ のとき $e^x \approx 1 + x$, $(1 + x)^{-1} \approx 1 - x$ と近似できることを用いる。
  高温の漸近領域において Q 16-1 の結果は次のように近似できる。ただし、$x\to 0$ のとき $\ln(1 + x) \approx x$ と近似できることとテイラー展開を用いる。
  \begin{align}
    z(\beta) & = \sum_{i = 0}^{\infty}e^{-\beta E_i} = \sum_{i = 0}^{\infty}e^{-\beta\ab(n + \frac{1}{2})\hbar\omega} = \frac{e^{-\frac{1}{2}\beta\hbar\omega}}{1 - e^{-\beta\hbar\omega}} = \frac{1}{2\sinh \frac{1}{2}\beta\hbar\omega} \\
             & = \begin{dcases}
                   \frac{e^{-\frac{1}{2}\beta\hbar\omega}}{1 - e^{-\beta\hbar\omega}} \\
                   \frac{e^{-\frac{1}{2}\beta\hbar\omega}}{1 - e^{-\beta\hbar\omega}} \\
                 \end{dcases}                                                                                                                                                \\
             & \approx \begin{dcases}
                         e^{-\frac{\hbar\omega}{2k_BT}} \to 0                                                                                \\
                         \frac{1 - \frac{1}{2}\beta\hbar\omega}{\beta\hbar\omega} \approx \frac{k_BT}{\hbar\omega} - \frac{1}{2} \to +\infty \\
                       \end{dcases}
  \end{align}
  \begin{align}
    f & = -k_BT\ln z(\beta)                                                                                                                                               \\
      & = -\frac{1}{\beta}\ln \frac{e^{-\frac{1}{2}\beta\hbar\omega}}{1 - e^{-\beta\hbar\omega}} = \frac{1}{\beta}\ln(1 - e^{-\beta\hbar\omega}) + \frac{1}{2}\hbar\omega \\
      & = -\frac{1}{\beta}\ln \frac{1}{2\sinh \frac{1}{2}\beta\hbar\omega} = \frac{1}{\beta}\ln(2\sinh \frac{1}{2}\beta\hbar\omega)                                       \\
      & \approx \begin{dcases}
                  \frac{1}{\beta}(-e^{-\beta\hbar\omega}) + \frac{1}{2}\hbar\omega       \\
                  \frac{1}{\beta}\ab(\ln \beta\hbar\omega + \frac{1}{2}\beta\hbar\omega) \\
                \end{dcases}                                                                                    \\
      & \approx \begin{dcases}
                  \frac{1}{2}\hbar\omega - k_BTe^{-\frac{\hbar\omega}{k_BT}} \to \frac{1}{2}\hbar\omega \\
                  -k_BT\ln\frac{k_BT}{\hbar\omega} \to - \infty                                         \\
                \end{dcases}
  \end{align}
  \begin{align}
    s & = - \ab(\diffp{f}{T})_{V,N} = k_B\beta^2\ab(\diffp{f}{\beta})_{V,N}                                                                                                                                              \\
      & = k_B\beta^2\ab(-\frac{1}{\beta^2}\ln(1 - e^{-\beta\hbar\omega}) + \frac{\hbar\omega}{\beta(1 - e^{-\beta\hbar\omega})})                                                                                         \\
      & = k_B\beta^2\ab(-\frac{1}{\beta^2}\ln\ab(2\sinh \frac{1}{2}\beta\hbar\omega) + \frac{\frac{1}{2}\hbar\omega\cosh \frac{1}{2}\beta\hbar\omega}{\beta\sinh \frac{1}{2}\beta\hbar\omega})                           \\
      & = k_B\ab(-\ln(1 - e^{-\beta\hbar\omega}) + \frac{\beta\hbar\omega}{e^{\beta\hbar\omega} - 1}) = k_B\ab(-\ln\ab(2\sinh\frac{1}{2}\beta\hbar\omega) + \frac{1}{2}\beta\hbar\omega\coth\frac{1}{2}\beta\hbar\omega) \\
      & \approx \begin{dcases}
                  k_B\ab(e^{-\beta\hbar\omega} + \beta\hbar\omega e^{-\beta\hbar\omega}(1 + e^{-\beta\hbar\omega})) \\
                  k_B\ab(-\ln\beta\hbar\omega + 1)                                                                  \\
                \end{dcases}                                                                                                        \\
      & \approx \begin{dcases}
                  k_B\frac{\hbar\omega}{k_BT}e^{- \frac{\hbar\omega}{k_BT}} \to 0 \\
                  k_B\ln\frac{k_BT}{\hbar\omega} \to +\infty                      \\
                \end{dcases}
  \end{align}
  \begin{align}
    u & = f + Ts                                                                                                                                                                                   \\
      & = \frac{1}{\beta}\ln(1 - e^{-\beta\hbar\omega}) + \frac{1}{2}\hbar\omega + \frac{1}{\beta}\ab(-\ln(1 - e^{-\beta\hbar\omega}) + \frac{\beta\hbar\omega}{e^{\beta\hbar\omega} - 1})         \\
      & = \frac{1}{\beta}\ln\ab(2\sinh \frac{1}{2}\beta\hbar\omega) + \frac{1}{\beta}\ab(-\ln\ab(2\sinh\frac{1}{2}\beta\hbar\omega) + \frac{1}{2}\beta\hbar\omega\coth\frac{1}{2}\beta\hbar\omega) \\
      & = \ab(\frac{1}{2} + \frac{1}{e^{\beta\hbar\omega} - 1})\hbar\omega = \frac{1}{2}\hbar\omega\coth\frac{1}{2}\beta\hbar\omega                                                                \\
      & \approx \begin{dcases}
                  \ab(\frac{1}{2} + e^{-\beta\hbar\omega}(1 + e^{-\beta\hbar\omega}))\hbar\omega                                        \\
                  \frac{1}{2}\hbar\omega\ab(\ab(\frac{\beta\hbar\omega}{2})^{-1} + \frac{1}{3}\ab(\frac{\beta\hbar\omega}{2}) + \cdots) \\
                \end{dcases}                                                              \\
      & \approx \begin{dcases}
                  \frac{1}{2}\hbar\omega + e^{-\beta\hbar\omega}\hbar\omega \to \frac{1}{2}\hbar\omega \\
                  k_BT\ab(1 + \frac{1}{12}\ab(\frac{\hbar\omega}{k_BT})^{2} + \cdots) \to +\infty      \\
                \end{dcases}
  \end{align}
  \begin{align}
    c & = \diffp{u}{T} = -k_B\beta^2\diffp{u}{\beta} = -k_B\beta^2\ab(-\frac{\hbar\omega e^{\beta\hbar\omega}}{(e^{\beta\hbar\omega} - 1)^2})\hbar\omega                              \\
      & = k_B\ab(\beta\hbar\omega\frac{e^{\frac{1}{2}\beta\hbar\omega}}{e^{\beta\hbar\omega} - 1})^2 = k_B\ab(\frac{\frac{1}{2}\beta\hbar\omega}{\sinh\frac{1}{2}\beta\hbar\omega})^2 \\
      & \approx \begin{dcases}
                  k_B\ab(\beta\hbar\omega(e^{-\frac{1}{2}\beta\hbar\omega})(1 + e^{-\beta\hbar\omega}))^2                                             \\
                  k_B\ab(\frac{\beta\hbar\omega}{2}\ab(\ab(\frac{\beta\hbar\omega}{2})^{-1} - \frac{1}{6}\ab(\frac{\beta\hbar\omega}{2}) + \cdots))^2 \\
                \end{dcases}                                                                               \\
      & \approx \begin{dcases}
                  k_B\ab(\frac{\hbar\omega}{k_BT})^2e^{-\frac{\hbar\omega}{k_BT}} \to 0 \\
                  k_B\ab(1 - \frac{1}{24}\ab(\frac{\hbar\omega}{k_BT}) + \cdots)^2 = k_B\ab(1 - \frac{1}{12}\ab(\frac{\hbar\omega}{k_BT}) + \cdots) \to k_B
                \end{dcases}
  \end{align}
\end{proof}


\subsection{固体の比熱の Einstein 模型}
\begin{itembox}[l]{Q 16-5.}
  独立な調和振動子の集まりの系として記述される系 $X$ において$\dl{\omega}$ が十分小さいとして、角振動数が $\omega$ から $\omega + \dl{\omega}$ の範囲にある調和振動子の個数を $g(\omega)\dl{\omega}$ と定義する。つまり $g(\omega)$ は調和振動子の角振動数の個数分布関数である。
\end{itembox}
このとき角運動量が $\omega$ である調和振動子 1 個の Helmholtz 自由エネルギー, エントロピー, 内部エネルギー, 比熱をそれぞれ $f(\omega), s(\omega), u(\omega), c(\omega)$ と書くこととすると、$\dl{\omega}$ が十分小さいことから近い角運動量の変数を個数倍して積分することで元の変数と一致する。これより次のような式が成り立つ。
\begin{align}
  F & = \int_0^\infty\dl{\omega}g(\omega)f(\omega) \\
  S & = \int_0^\infty\dl{\omega}g(\omega)s(\omega) \\
  U & = \int_0^\infty\dl{\omega}g(\omega)u(\omega) \\
  C & = \int_0^\infty\dl{\omega}g(\omega)c(\omega)
\end{align}

\begin{itembox}[l]{Q 16-6.}
  ある元素の原子 $n$ [\si{mol}] からなる個体を考える。Einstein 模型では、結晶を構成するそれぞれの原子は平衡位置の回りに独立に同一の角振動数 $\omega_E$ を持って調和振動すると考える。ここで次の観測結果に対して Einstein 模型は妥当性があることを説明せよ。
  \begin{enumerate}
    \item (高温での固体の比熱の振る舞い : Dulong-Petit の法則) 十分に高温では、$n$ [\si{mol}] の固体の比熱 $C$ は、固体を構成する物質によらずに、$3nR$ の一定値を取る。ここで、$R = 8.314\ldots$ [\si{J/(mol\cdot K)}] は気体定数である。
    \item (低温での固体の比熱の大雑把な振る舞い) 温度 $T$ が $0$ に近付くとき、固体の比熱 $C$ は小さくなっていく。温度 $T$ が $0$ に近付く極限では、比熱 $C$ はゼロになるようだ。
  \end{enumerate}
\end{itembox}
調和振動子の角振動数の個数について、各原子の自由度が $3$ であるから Avogadro 数 $N_A = 6.02\ldots\times 10^{23}$ [\si{1/mol}] を用いて全体の個数は $3N = 3nN_A$ であることが分かる。
これより Einstein 模型における調和振動子の角振動数の個数分布関数 $g(\omega)$ は次のように表される。
\begin{align}
  g(\omega) = 3N\delta(\omega - \omega_E).
\end{align}

これより比熱は次のように表される。
\begin{align}
  C & = \int_0^\infty\dl{\omega}g(\omega)c(\omega)                                           \\
    & = \int_0^\infty\dl{\omega}3N\delta(\omega - \omega_E)c(\omega)                         \\
    & = 3Nc(\omega_E)                                                                        \\
    & = 3Nk_B\ab(\frac{\frac{1}{2}\beta\hbar\omega_E}{\sinh\frac{1}{2}\beta\hbar\omega_E})^2
\end{align}
高温の漸近領域において比熱 $C$ は次のようになる。
\begin{align}
  C & = 3Nk_B\ab(\frac{\frac{1}{2}\beta\hbar\omega_E}{\sinh\frac{1}{2}\beta\hbar\omega_E})^2 \\
    & \approx 3Nk_B\ab(1 - \frac{1}{12}\ab(\frac{\hbar\omega}{k_BT})^2 + \cdots)             \\
    & \approx 3nR
\end{align}
低温の漸近領域において比熱 $C$ は次のようになる。
\begin{align}
  C & = 3Nk_B\ab(\frac{\frac{1}{2}\beta\hbar\omega_E}{\sinh\frac{1}{2}\beta\hbar\omega_E})^2 \\
    & \approx 3Nk_B\ab(\frac{\hbar\omega}{k_BT})^2e^{-\frac{\hbar\omega}{k_BT}}              \\
    & \approx 3nR\ab(\frac{\hbar\omega}{k_BT})^2e^{-\frac{\hbar\omega}{k_BT}}
\end{align}
よって低温領域で温度 $T$ が小さくなっていくとき、比熱 $C$ が小さくなる。
\begin{align}
  \lim_{T\to 0} C & = \lim_{T\to 0}3nR\ab(\frac{\hbar\omega}{k_BT})^2e^{-\frac{\hbar\omega}{k_BT}} = 0.
\end{align}
これらの結果は観測結果と一致している為、妥当性がある。

\begin{itembox}[l]{Q 16-7.}
  固体の比熱の Einstein 模型は次の実験事実と合致しないことを確認せよ。
  \begin{enumerate}
    \item (低温での固体の比熱の精密な振る舞い) 温度 $T$ が $0$ に近付くとき、固体の比熱 $C$ は $C \propto T^3$ であり、$\lim_{T\to 0} C = 0$ となる。
  \end{enumerate}
\end{itembox}
低温領域で温度 $T$ が小さくなっていくとき、比熱 $C$ は次のように小さくなる。
\begin{align}
  C & \approx 3nR\ab(\frac{\hbar\omega}{k_BT})^2e^{-\frac{\hbar\omega}{k_BT}} \\
    & \propto \frac{1}{T^{2}e^{\frac{1}{T}}}
\end{align}
これより $C \propto T^3$ とはならない為、固体の比熱の Einstein 模型は実験事実と合致しない。

\section{古典統計力学 (classical statistical mechanics) 近似}

\begin{theorem}
  \begin{align}
    Z & = \frac{1}{(2\pi\hbar)^f}\int e^{-H(p, q)/k_BT}\prod_{i=1}^{f}\dl{p_i}\dl{q_i}
  \end{align}
\end{theorem}

\subsection{振動子系の古典近似}
\subsection{理想気体の古典近似}
\subsection{非調和振動子系の古典近似}

\section{固体の比熱の Debye 模型}
ここでは固体の比熱 $C$ の Debye 模型を学ぶ. Debye 模型は高温における $C\approx 3nR$ と低温における $C\propto T^3$ の両方を正しく説明する.
\subsection{Debye 模型の基本的な考え方}
Debye 模型は Einstein 模型と同様に固体の比熱を独立な調和振動子の集まりの比熱として捉える. ただ Debye 模型は Einstein 模型に加え, 固体を構成する各原子は原子同士の原子間力によるバネにより結びついていると考える.

\subsection{解析力学の復習:点正準変換}
ある $N$ 自由度の系の一般化座標を $q_1, \ldots, q_N$ として Lagrange 形式では一般化座標 $q_i$ と一般化速度 $\dot{q}_i$ を用いて表現される. このとき一般化運動量 $p_i$ は次のように定められる.
\begin{align}
  L   & = L(q_1,\ldots,q_N,\dot{q}_1,\ldots,\dot{q}_N),                                                                                     \\
  p_i & = \ab(\diffp{L}{\dot{q}_i})_{q_1,\ldots,q_N,\dot{q}_1,\ldots,\dot{q}_{i-1},\dot{q}_{i+1},\ldots,\dot{q}_N} \qquad (i = 1,\ldots,N).
\end{align}
一方 Hamilton 形式では一般化座標 $q_i$ と一般化運動量 $p_i$ を用いて表現される.
\begin{align}
  H             & = H(q_1,\ldots,q_N,p_1,\ldots,p_N) = \sum_{i=1}^{N}p_i\dot{q}_i - L,              \\
  \diff{q_i}{t} & = \diffp{H}{p_i}, \qquad \diff{p_i}{t} = -\diffp{H}{q_i} \qquad (i = 1,\ldots,N).
\end{align}

\begin{itembox}[l]{Q 17-1.}
  Lagrange 形式での一般座標変換 $(q_1,\ldots,q_N)\to(Q_1,\ldots,Q_N)$ に対応する Hamilton 形式で正準変換を点正準変換といい, $(q_1,\ldots,q_N,p_1,\ldots,p_N)\to(Q_1,\ldots,Q_N,P_1,\ldots,P_N)$ を求める.
  \begin{align}
    q_i = f_i(Q_1,\ldots,Q_N).
  \end{align}
\end{itembox}

(i) 新しい運動量 $P_j$ は Lagrange 形式を用いて次のように求められる.
\begin{align}
  P_j & = \ab(\diffp{L}{\dot{Q}_j})_{Q_1,\ldots,Q_N,\dot{Q}_1,\ldots,\dot{Q}_{j-1},\dot{Q}_{j+1},\ldots,\dot{Q}_N} \qquad (j=1,2,\ldots,N) \\
      & = \sum_{i=1}^{N}\diffp{L}{\dot{q}_i}\diffp{\dot{q}_i}{\dot{Q}_j}                                                                   \\
      & = \sum_{i=1}^{N}p_i\diffp{q_i}{Q_j}                                                                                                \\
      & = \sum_{i=1}^{N}\diffp{f_i(Q_1,\ldots,Q_N)}{Q_j}p_i.
\end{align}

(ii) また新しい Hamilton 関数は定義式から古い Hamilton 関数と一致する.
\begin{align}
  H' = H'(Q_1,\ldots,Q_N,P_1,\ldots,P_N) = \sum_{j=1}^{N}P_j\dot{Q}_j - L = \sum_{j=1}^{N}\sum_{i=1}^{N}\diffp{f_i(Q_1,\ldots,Q_N)}{Q_j}p_i\dot{Q}_j - L = \sum_{i=1}^{N}p_i\dot{q}_i - L = H
\end{align}

\subsection{1 次元結晶における平衡位置の回りの調和振動を記述する Hamilton 関数}
直線上に等間隔の平衡位置を持って並んだ $N$ 個の原子からなる 1 次元結晶を物理系として記述して古典力学により考察する. $i$ 番目の原子の位置座標の平衡位置からのずれを $q_i$ として, その運動量を $p_i$ とする.
\begin{itembox}[l]{Q 17-2.}
  1 次元結晶の Hamilton 関数は次のように表される.
  \begin{align}
    H^{1次元結晶}(q_1,\ldots,q_N, p_1,\ldots,p_N) & := \frac{1}{2m}\sum_{i=1}^{N}p_i^2 + \frac{1}{2}\kappa\sum_{i=0}^{N}(q_i - q_{i+1})^2
  \end{align}
  ただし $\kappa$ は隣り合った原子の間の原子間力のバネ定数とし, 両端の原子は固定されている $q_0 = q_{N+1} = 0$ と仮定する.
\end{itembox}

$i$ 番目の原子の運動エネルギーは運動量 $p_i$ を用いて次のように表される.
\begin{align}
  \frac{p_i^2}{2m}.
\end{align}
また隣り合う $i, i+1$ 番目の原子の原子間力のポテンシャルエネルギーはバネ定数 $\kappa$ を用いて次のように表される.
\begin{align}
  \frac{1}{2}\kappa(q_i - q_{i+1})^2.
\end{align}
これより Hamilton 関数は次のように表される.
\begin{align}
  H^{1次元結晶}(q_1,\ldots,q_N, p_1,\ldots,p_N) & := \frac{1}{2m}\sum_{i=1}^{N}p_i^2 + \frac{1}{2}\kappa\sum_{i=0}^{N}(q_i - q_{i+1})^2.
\end{align}

\subsection{1 次元結晶における平衡位置の回りの調和振動の基準モードの計算}

\begin{itembox}[l]{Q 17-3.}
  固定端境界条件の 1 次元結晶の系を考えているので Fourier 展開した基底が基準振動となる.
  \begin{align}
    H^{1次元結晶}(Q_1,\ldots,Q_N, P_1,\ldots,P_N) & = \sum_{j=1}^{N}\ab(\frac{1}{2m}P_j^2 + \frac{1}{2}m\omega_j^2Q_j^2).
  \end{align}
  ただし, $\omega_j$ を次のように定める.
  \begin{align}
    \omega_j = 2\sqrt{\frac{\kappa}{m}}\sin\ab(\frac{\pi}{2(N+1)}j).
  \end{align}
\end{itembox}

固定端境界条件の 1 次元結晶の系を考えているので Fourier Sine 展開の基底が基準振動になっているとする.
\begin{align}
  q_i^{(j)} & = \sqrt{\frac{2}{N+1}}\sin\ab(\frac{\pi}{N+1}ji).
\end{align}
まず計算に必要な関数を定義する. \\

(i) $\alpha \neq 0 \pmod{2\pi}$ に対して $F(\alpha), G(\alpha)$ を次のように定義する.
\begin{align}
  F(\alpha) & := \sum_{i=1}^{N}\cos(\alpha i), \\
  G(\alpha) & := \sum_{i=1}^{N}\sin(\alpha i).
\end{align}
このとき $F(\alpha), G(\alpha)\in\RR$ より $F(\alpha) + \sqrt{-1}G(\alpha)\in\CC$ の実部と虚部はそれぞれ $F(\alpha), G(\alpha)$ と対応した値となる. Euler の公式を用いて次のように計算できる.
\begin{align}
  F(\alpha) + \sqrt{-1}G(\alpha) & = \sum_{i=1}^{N}e^{\sqrt{-1}\alpha i}                                                                                                                                                   \\
                                 & = \frac{e^{\sqrt{-1}\alpha} - e^{\sqrt{-1}\alpha (N+1)}}{1 - e^{\sqrt{-1}\alpha}}                                                                                                       \\
                                 & = \frac{2e^{\sqrt{-1}\alpha}e^{\sqrt{-1}\alpha \frac{N}{2}}\sin{\alpha \frac{N}{2}}}{2e^{\sqrt{-1}\alpha\frac{1}{2}}\sin{\alpha\frac{1}{2}}}                                            \\
                                 & = \frac{e^{\sqrt{-1}\frac{\alpha}{2}(N+1)}\sin{\frac{\alpha}{2}N}}{\sin{\frac{\alpha}{2}}}                                                                                              \\
                                 & = \frac{\cos\ab(\frac{\alpha}{2}(N+1))\sin{\frac{\alpha}{2}N}}{\sin{\frac{\alpha}{2}}} + \sqrt{-1}\frac{\sin\ab(\frac{\alpha}{2}(N+1))\sin{\frac{\alpha}{2}N}}{\sin{\frac{\alpha}{2}}}.
\end{align}
これより実部虚部の対応から $F(\alpha), G(\alpha)$ が求まる.
\begin{align}
  F(\alpha) & := \sum_{i=1}^{N}\cos(\alpha i) = \frac{\cos\ab(\frac{\alpha}{2}(N+1))\sin\ab(\frac{\alpha}{2}N)}{\sin{\frac{\alpha}{2}}}, \\
  G(\alpha) & := \sum_{i=1}^{N}\sin(\alpha i) = \frac{\sin\ab(\frac{\alpha}{2}(N+1))\sin\ab(\frac{\alpha}{2}N)}{\sin{\frac{\alpha}{2}}}.
\end{align}

(ii) $j, j' = 1,\ldots,N$ とすると $j - j' = -(N - 1),\ldots,N - 1$ かつ $j + j' = 2,\ldots,2N$ である. これより $j - j' = 0$ である場合に限り $j - j' = 0 \pmod{2(N+1)}$ が成り立ち, $j + j' = 0 \pmod{2(N+1)}$ が成り立つ場合は存在せず, 逆に主結合子の前件が恒偽ならばその論理式は真である. よって次の同値関係が成り立つ.
\begin{align}
   & \frac{\pi}{N+1}(j - j') = 0 \pmod{2\pi} \iff j - j' = 0 \pmod{2(N+1)} \iff j = j', \label{Q17-3. ii-1} \\
   & \frac{\pi}{N+1}(j + j') = 0 \pmod{2\pi} \iff j + j' = 0 \pmod{2(N+1)} \iff false. \label{Q17-3. ii-2}
\end{align}

(iii) $j, j' = 1,\ldots,N$ に対して次のように内積を定義する. このときこの内積の正規直交関係を示す.
\begin{align}
  (q^{(j)}, q^{(j')}) & := \sum_{i = 1}^{N}q_i^{(j)}q_i^{(j')}.
\end{align}
まず (i), (ii) を用いることで次のように式変形できる.
\begin{align}
  (q^{(j)}, q^{(j')}) & := \sum_{i = 1}^{N}q_i^{(j)}q_i^{(j')}                                                                                                                                                                                                                                             \\
                      & = \frac{2}{N+1}\sum_{i = 1}^{N}\sin\ab(\frac{\pi}{N+1}ji)\sin\ab(\frac{\pi}{N+1}j'i)                                                                                                                                                                                               \\
                      & = \frac{1}{N+1}\sum_{i = 1}^{N}\ab(\cos\ab(\frac{\pi}{N+1}(j - j')i) - \cos\ab(\frac{\pi}{N+1}(j + j')i))                                                                                                                                                                          \\
                      & = \begin{dcases}
                            \frac{1}{N+1}\ab(\frac{\cos\ab(\frac{\pi}{2}(j - j'))\sin\ab(\frac{N\pi}{2(N+1)}(j - j'))}{\sin\ab(\frac{\pi}{2(N+1)}(j - j'))} - \frac{\cos\ab(\frac{\pi}{2}(j + j'))\sin\ab(\frac{N\pi}{2(N+1)}(j + j'))}{\sin\ab(\frac{\pi}{2(N+1)}(j + j'))}) & (j \neq j') \\
                            \frac{1}{N+1}\ab(N - \frac{\cos\ab(j\pi)\sin\ab(\frac{jN}{N+1}\pi)}{\sin\ab(\frac{j}{N+1}\pi)})                                                                                                                                                   & (j = j')
                          \end{dcases}.
\end{align}
先に $j \neq j'$ の場合を考える. 括弧内を通分した分子の第一項と第二項についてそれぞれ計算する. 第一項について
\begin{align}
    & \cos\ab(\frac{\pi}{2}(j - j'))\sin\ab(\frac{N\pi}{2(N+1)}(j - j'))\sin\ab(\frac{\pi}{2(N+1)}(j + j'))                                                      \\
  = & \cos\ab(\frac{j - j'}{2}\pi)\ab(\cos\ab(\frac{(N-1)j - (N+1)j'}{2(N+1)}\pi) - \cos\ab(\frac{(N+1)j - (N-1)j'}{2(N+1)}\pi))                                 \\
  = & \cos\ab(\frac{j - j'}{2}\pi)\cos\ab(\frac{(N-1)j - (N+1)j'}{2(N+1)}\pi) - \cos\ab(\frac{j - j'}{2}\pi)\cos\ab(\frac{(N+1)j - (N-1)j'}{2(N+1)}\pi)          \\
  = & \cos\ab(\frac{j}{N+1}\pi) + \cos\ab(\frac{Nj - (N+1)j'}{N+1}\pi) - \cos\ab(\frac{j'}{N+1}\pi) - \cos\ab(\frac{(N+1)j - Nj'}{N+1}\pi). \label{Q17-3. iii 1}
\end{align}
第二項について
\begin{align}
    & \cos\ab(\frac{\pi}{2}(j + j'))\sin\ab(\frac{N\pi}{2(N+1)}(j + j'))\sin\ab(\frac{\pi}{2(N+1)}(j - j'))                                                      \\
  = & \cos\ab(\frac{j + j'}{2}\pi)\ab(\cos\ab(\frac{(N-1)j + (N+1)j'}{2(N+1)}\pi) - \cos\ab(\frac{(N+1)j + (N-1)j'}{2(N+1)}\pi))                                 \\
  = & \cos\ab(\frac{j + j'}{2}\pi)\cos\ab(\frac{(N-1)j + (N+1)j'}{2(N+1)}\pi) - \cos\ab(\frac{j + j'}{2}\pi)\cos\ab(\frac{(N+1)j + (N-1)j'}{2(N+1)}\pi)          \\
  = & \cos\ab(\frac{Nj + (N+1)j'}{N+1}\pi) + \cos\ab(\frac{j}{N+1}\pi) - \cos\ab(\frac{(N+1)j + Nj'}{N+1}\pi) - \cos\ab(\frac{j'}{N+1}\pi). \label{Q17-3. iii 2}
\end{align}
これより分子は次のようになる.
\begin{align}
  \eqref{Q17-3. iii 1} - \eqref{Q17-3. iii 2} & = \ab(\cos\frac{j}{N+1}\pi + \cos\ab(\frac{Nj}{N+1} - j')\pi - \cos\frac{j'}{N+1}\pi - \cos\ab(j - \frac{Nj'}{N+1})\pi)                 \\
                                              & - \ab(\cos\ab(\frac{Nj}{N+1} + j')\pi + \cos\frac{j}{N+1}\pi - \cos\ab(j + \frac{Nj'}{N+1})\pi - \cos\frac{j'}{N+1}\pi)                 \\
                                              & = \cos\ab(\frac{Nj}{N+1} - j')\pi - \cos\ab(\frac{Nj}{N+1} + j')\pi + \cos\ab(j + \frac{Nj'}{N+1})\pi - \cos\ab(j - \frac{Nj'}{N+1})\pi \\
                                              & = 2\sin\ab(j'\pi)\sin\ab(\frac{Nj}{N+1}\pi) - 2\sin\ab(j\pi)\sin\ab(\frac{Nj'}{N+1}\pi)                                                 \\
                                              & = 0 \qquad (\because j, j'\in\ZZ).
\end{align}
よって $j \neq j'$ のときは $(q^{(j)}, q^{(j')}) = 0$ となる.

次に $j = j'$ の場合を考える. これは $j$ が奇数か偶数かで場合分けして考える.
\begin{align}
  \frac{\cos\ab(j\pi)\sin\ab(\frac{jN}{N+1}\pi)}{\sin\ab(\frac{j}{N+1}\pi)} & =
  \begin{dcases}
    \frac{\cos\ab(2k\pi)\sin\ab(\frac{2kN}{N+1}\pi)}{\sin\ab(\frac{2k}{N+1}\pi)}           & (j = 2k, k\in\ZZ)   \\
    \frac{\cos\ab((2k-1)\pi)\sin\ab(\frac{(2k-1)N}{N+1}\pi)}{\sin\ab(\frac{2k-1}{N+1}\pi)} & (j = 2k-1, k\in\ZZ)
  \end{dcases} \\ & =
  \begin{dcases}
    \frac{1\cdot\sin\ab(2k\pi\frac{N}{N+1} - 2k\pi)}{\sin\ab(2k\pi\frac{1}{N+1})} \\
    \frac{-1 \cdot -\sin\ab((2k-1)\pi\frac{N}{N+1} - (2k-1)\pi)}{\sin\ab((2k-1)\pi\frac{1}{N+1})}
  \end{dcases}           \\
                                                                            & = -1.
\end{align}
よって $j = j'$ のときは $(q^{(j)}, q^{(j')}) = 1$ となる. これより, まとめると次の式が成り立つ.
\begin{align}
  (q^{(j)}, q^{(j')}) = \delta_{j,j'}.
\end{align}


(iv) ここで行列 $A_{ij} := q_i^{(j)}$ を定義する. このとき次の計算から $A_{ij}$ は直交行列であるとわかる.
\begin{align}
  (A^{\top}A)_{ij} & = \sum_{k=1}^{N}A_{ik}^\top A_{kj} = \sum_{k=1}^{N}A_{ki}A_{kj} = \sum_{k=1}^{N}q_k^{(i)}q_k^{(j)} = (q^{(i)}, q^{(j)}) = \delta_{i,j}.
\end{align}

(v) また $A_{ij}$ が直交行列であるから次のような正規直交関係もある.
\begin{align}
  (AA^{\top})_{ij} & = \sum_{k=1}^{N}A_{ik}A_{kj}^{\top} = \sum_{k=1}^{N}A_{ik}A_{jk} = \sum_{k=1}^{N}q_i^{(k)}q_j^{(k)} = \delta_{i,j}.
\end{align}

(vi) ここで原子の変位を表す古い座標系 $q_1, \ldots, q_N$ を $q^{(1)}, \ldots, q^{(N)}$ で離散 Fourier Sine 展開した振幅を新しい座標系 $Q_1, \ldots, Q_N$ と定義する.
\begin{align}
  q_i = \sum_{j=1}^{N}Q_jq_i^{(j)}.
\end{align}
これは点正準変換を用いて新しい運動量を古い運動量を表せられる.
\begin{align}
  P_j = \sum_{i=1}^{N}\diffp{q_i}{Q_j}p_i = \sum_{i=1}^{N}q_i^{(j)}p_i.
\end{align}

(vii) Hamilton 関数の運動エネルギーの表式の核の部分について次のように表される.
\begin{align}
  \sum_{j=1}^{N}P_j^2 = \sum_{j=1}^{N}\ab(\sum_{i=1}^{N}q_i^{(j)}p_i)^2 = \sum_{j=1}^{N}\sum_{i=1}^{N}\sum_{i'=1}^{N}(q_i^{(j)}p_i)(q_{i'}^{(j)}p_{i'}) = \sum_{i=1}^{N}p_i^2.
\end{align}

(viii) Hamilton 関数のポテンシャルエネルギーの核の部分について次のような表される.
\begin{align}
  \sum_{i=0}^{N}(q_i - q_{i+1})^2 & = \sum_{i=0}^{N}\ab(\sum_{j=1}^{N}\ab(Q_jq_i^{(j)} - Q_jq_{i+1}^{(j)}))^2                                                     \\
                                  & = \sum_{i=0}^{N}\sum_{j=1}^{N}\sum_{j'=1}^{N}\ab(Q_jq_i^{(j)} - Q_jq_{i+1}^{(j)})\ab(Q_{j'}q_i^{(j')} - Q_{j'}q_{i+1}^{(j')}) \\
                                  & = \sum_{j=1}^{N}\sum_{j'=1}^{N}\sum_{i=0}^{N}(q_i^{(j)} - q_{i+1}^{(j)})(q_i^{(j')} - q_{i+1}^{(j')})Q_jQ_{j'}                \\
                                  & = \sum_{j=1}^{N}\sum_{j'=1}^{N}B_{j,j'}Q_jQ_{j'}.
\end{align}
ただし, $B_{j,j'}$ を次のように定める.
\begin{align}
  B_{j,j'} := \sum_{i=0}^{N}(q_i^{(j)} - q_{i+1}^{(j)})(q_i^{(j')} - q_{i+1}^{(j')}).
\end{align}

(ix) 次に $B_{j,j'}$ を求める. まず $q_i^{(j)} - q_{i+1}^{(j)}$ は次のように求められる.
\begin{align}
  q_i^{(j)} - q_{i+1}^{(j)} & = \sqrt{\frac{2}{N+1}}\sin\ab(\frac{\pi}{N+1}ji) - \sqrt{\frac{2}{N+1}}\sin\ab(\frac{\pi}{N+1}j(i+1)) \\
                            & = \sqrt{\frac{2}{N+1}}\ab(\sin\ab(\frac{\pi}{N+1}ji) - \sin\ab(\frac{\pi}{N+1}j(i+1)))                \\
                            & = -2\sqrt{\frac{2}{N+1}}\cos\ab(\frac{\pi}{2}\frac{(2i+1)j}{N+1})\sin\ab(\frac{\pi}{2}\frac{j}{N+1}).
\end{align}

(x) これより $B_{j,j'}$ は次のように計算できる.
\begin{align}
  B_{j,j'} & = \sum_{i=0}^{N}(q_i^{(j)} - q_{i+1}^{(j)})(q_i^{(j')} - q_{i+1}^{(j')})                                                                                                                                                         \\
           & = \sum_{i=0}^{N}\ab(-2\sqrt{\frac{2}{N+1}}\cos\ab(\frac{\pi}{2}\frac{(2i+1)j}{N+1})\sin\ab(\frac{\pi}{2}\frac{j}{N+1}))\ab(-2\sqrt{\frac{2}{N+1}}\cos\ab(\frac{\pi}{2}\frac{(2i+1)j'}{N+1})\sin\ab(\frac{\pi}{2}\frac{j'}{N+1})) \\
           & = 4\sin\ab(\frac{\pi}{2}\frac{j}{N+1})\sin\ab(\frac{\pi}{2}\frac{j'}{N+1})\frac{2}{N+1}\sum_{i=0}^{N}\cos\ab(\frac{\pi}{N+1}j\ab(i + \frac{1}{2}))\cos\ab(\frac{\pi}{N+1}j'\ab(i + \frac{1}{2}))                                 \\
           & = 4\sin\ab(\frac{\pi}{2}\frac{j}{N+1})\sin\ab(\frac{\pi}{2}\frac{j'}{N+1})\frac{1}{N+1}\sum_{i=0}^{N}\ab(\cos\ab(\frac{\pi}{N+1}(j + j')\ab(i + \frac{1}{2})) + \cos\ab(\frac{\pi}{N+1}(j - j')\ab(i + \frac{1}{2})))            \\
           & = 4\sin\ab(\frac{\pi}{2}\frac{j}{N+1})\sin\ab(\frac{\pi}{2}\frac{j'}{N+1})\tilde{B}_{j,j'}.
\end{align}
ただし, $\tilde{B}_{j,j'}$ を次のように定める.
\begin{align}
  \tilde{B}_{j,j'} & := \frac{1}{N+1}\sum_{i=0}^{N}\ab(\cos\ab(\frac{\pi}{N+1}(j + j')\ab(i + \frac{1}{2})) + \cos\ab(\frac{\pi}{N+1}(j - j')\ab(i + \frac{1}{2}))).
\end{align}

(xi) さらに $\tilde{B}_{j,j'}$ は次のように計算できる.
\begin{align}
  \tilde{B}_{j,j'} & = \frac{1}{N+1}\sum_{i=0}^{N}\ab(\cos\ab(\pi\frac{j + j'}{N+1}\ab(i + \frac{1}{2})) + \cos\ab(\pi\frac{j - j'}{N+1}\ab(i + \frac{1}{2}))) \\
                   & \ \begin{aligned}
                         = \frac{1}{N+1}\sum_{i=0}^{N}\bigg[ & \quad\cos\ab(\frac{\pi}{2}\frac{j + j'}{N+1})\cos\ab(\pi\frac{j + j'}{N+1}i)    \\
                                                             & - \sin\ab(\frac{\pi}{2}\frac{j + j'}{N+1})\sin\ab(\pi\frac{j + j'}{N+1}i)       \\
                                                             & + \cos\ab(\frac{\pi}{2}\frac{j - j'}{N+1})\cos\ab(\pi\frac{j - j'}{N+1}i)       \\
                                                             & - \sin\ab(\frac{\pi}{2}\frac{j - j'}{N+1})\sin\ab(\pi\frac{j - j'}{N+1}i)\bigg]
                       \end{aligned}                                    \\
                   & \ \begin{aligned}
                         = \frac{1}{N+1}\bigg[ & \quad\cos\ab(\frac{\pi}{2}\frac{j + j'}{N+1})\ab(1 + F\ab(\pi\frac{j + j'}{N+1})) \\
                                               & - \sin\ab(\frac{\pi}{2}\frac{j + j'}{N+1})G\ab(\pi\frac{j + j'}{N+1})             \\
                                               & + \cos\ab(\frac{\pi}{2}\frac{j - j'}{N+1})\ab(1 + F\ab(\pi\frac{j - j'}{N+1}))    \\
                                               & - \sin\ab(\frac{\pi}{2}\frac{j - j'}{N+1})G\ab(\pi\frac{j - j'}{N+1})\bigg]
                       \end{aligned}.
\end{align}

(xii) まず $\tilde{B}_{j,j'}$ について $j = j'$ の場合を考える.
\begin{align}
  \tilde{B}_{j,j'} & = \tilde{B}_{j,j}                                                                                                                                                                                                                            \\
                   & = \frac{1}{N+1}\ab[\cos\ab(\frac{1}{N+1}j\pi)\ab(1 + F\ab(\frac{2}{N+1}j\pi)) - \sin\ab(\frac{1}{N+1}j\pi)G\ab(\frac{2}{N+1}j\pi) + \ab(1 + N) - 0]                                                                                          \\
                   & = 1 + \frac{1}{N+1}\ab(\cos\ab(\frac{1}{N+1}j\pi)\ab(1 + \frac{\cos\ab(j\pi)\sin\ab(\frac{N}{N+1}j\pi)}{\sin\ab(\frac{1}{N+1}j\pi)}) - \sin\ab(\frac{1}{N+1}j\pi)\frac{\sin\ab(j\pi)\sin\ab(\frac{N}{N+1}j\pi)}{\sin\ab(\frac{1}{N+1}j\pi)}) \\
                   & = 1 + \frac{1}{N+1}\ab(\cos\ab(\frac{1}{N+1}j\pi) + \ab(\cos\ab(\frac{1}{N+1}j\pi)\cos\ab(j\pi) - \sin\ab(\frac{1}{N+1}j\pi)\sin\ab(j\pi))\frac{\sin\ab(\frac{N}{N+1}j\pi)}{\sin\ab(\frac{1}{N+1}j\pi)})                                     \\
                   & = 1 + \frac{1}{N+1}\ab(\cos\ab(\frac{1}{N+1}j\pi) + \cos\ab(\frac{N+2}{N+1}j\pi)\frac{\sin\ab(\frac{N}{N+1}j\pi)}{\sin\ab(\frac{1}{N+1}j\pi)})                                                                                               \\
                   & = 1 + \frac{1}{N+1}\ab(\cos\ab(\frac{1}{N+1}j\pi)\sin\ab(\frac{1}{N+1}j\pi) + \cos\ab(\frac{N+2}{N+1}j\pi)\sin\ab(\frac{N}{N+1}j\pi))\bigg/\sin\ab(\frac{1}{N+1}j\pi)                                                                        \\
                   & = 1 + \frac{1}{N+1}\ab(\frac{1}{2}\sin\ab(\frac{2}{N+1}j\pi) + \frac{1}{2}\sin\ab(-\frac{2}{N+1}j\pi))\bigg/\sin\ab(\frac{1}{N+1}j\pi)                                                                                                       \\
                   & = 1.
\end{align}

(xiii) 次に $\tilde{B}_{j,j'}$ について $j \neq j'$ の場合を考える.
\begin{align}
  \tilde{B}_{j,j'} & = \tilde{B}_{j,j'}                                                                                                                                                                      \\
                   & \ \begin{aligned}
                         = \frac{1}{N+1}\bigg[ & \quad\cos\ab(\frac{\pi}{2}\frac{j + j'}{N+1})\ab(1 + F\ab(\pi\frac{j + j'}{N+1})) \\
                                               & - \sin\ab(\frac{\pi}{2}\frac{j + j'}{N+1})G\ab(\pi\frac{j + j'}{N+1})             \\
                                               & + \cos\ab(\frac{\pi}{2}\frac{j - j'}{N+1})\ab(1 + F\ab(\pi\frac{j - j'}{N+1}))    \\
                                               & - \sin\ab(\frac{\pi}{2}\frac{j - j'}{N+1})G\ab(\pi\frac{j - j'}{N+1})\bigg]
                       \end{aligned}                                                                                             \\
                   & \ \begin{aligned}
                         = \frac{1}{N+1}\Bigg[ & \quad\cos\ab(\frac{\pi}{2}\frac{j + j'}{N+1})\ab(1 + \frac{\cos\ab(\frac{1}{2}(j+j')\pi)\sin\ab(\frac{N}{2(N+1)}(j+j')\pi)}{\sin\ab(\frac{1}{2(N+1)}(j+j')\pi)}) \\
                                               & - \sin\ab(\frac{\pi}{2}\frac{j + j'}{N+1})\frac{\sin\ab(\frac{1}{2}(j+j')\pi)\sin\ab(\frac{N}{2(N+1)}(j+j')\pi)}{\sin\ab(\frac{1}{2(N+1)}(j+j')\pi)}             \\
                                               & + \cos\ab(\frac{\pi}{2}\frac{j - j'}{N+1})\ab(1 + \frac{\cos\ab(\frac{1}{2}(j-j')\pi)\sin\ab(\frac{N}{2(N+1)}(j-j')\pi)}{\sin\ab(\frac{1}{2(N+1)}(j-j')\pi)})    \\
                                               & - \sin\ab(\frac{\pi}{2}\frac{j - j'}{N+1})\frac{\sin\ab(\frac{1}{2}(j-j')\pi)\sin\ab(\frac{N}{2(N+1)}(j-j')\pi)}{\sin\ab(\frac{1}{2(N+1)}(j-j')\pi)}\Bigg]
                       \end{aligned}                                                                                              \\
                   & \ \begin{aligned}
                         = \frac{1}{N+1}\Bigg[ & \quad\cos\ab(\frac{\pi}{2}\frac{j + j'}{N+1}) + \ab(\cos\ab(\frac{\pi}{2}\frac{j + j'}{N+1})\cos\ab(\frac{j+j'}{2}\pi) - \sin\ab(\frac{\pi}{2}\frac{j + j'}{N+1})\sin\ab(\frac{j+j'}{2}\pi))\frac{\sin\ab(\frac{N(j+j')}{2(N+1)}\pi)}{\sin\ab(\frac{j+j'}{2(N+1)}\pi)}    \\
                                               & + \cos\ab(\frac{\pi}{2}\frac{j - j'}{N+1}) + \ab(\cos\ab(\frac{\pi}{2}\frac{j - j'}{N+1})\cos\ab(\frac{j-j'}{2}\pi) - \sin\ab(\frac{\pi}{2}\frac{j - j'}{N+1})\sin\ab(\frac{j-j'}{2}\pi))\frac{\sin\ab(\frac{N(j-j')}{2(N+1)}\pi)}{\sin\ab(\frac{j-j'}{2(N+1)}\pi)}\Bigg]
                       \end{aligned}                                                                          \\
                   & \ \begin{aligned}
                         = \frac{1}{N+1}\Bigg[ & \quad\cos\ab(\frac{\pi}{2}\frac{j + j'}{N+1}) + \cos\ab(\frac{N+2}{2(N+1)}(j + j')\pi)\frac{\sin\ab(\frac{N}{2(N+1)}(j+j')\pi)}{\sin\ab(\frac{1}{2(N+1)}(j+j')\pi)}    \\
                                               & + \cos\ab(\frac{\pi}{2}\frac{j - j'}{N+1}) + \cos\ab(\frac{N+2}{2(N+1)}(j - j')\pi)\frac{\sin\ab(\frac{N}{2(N+1)}(j-j')\pi)}{\sin\ab(\frac{1}{2(N+1)}(j-j')\pi)}\Bigg]
                       \end{aligned}                                                                      \\
                   & \ \begin{aligned}
                         = \frac{1}{N+1}\Bigg[ & \quad\frac{1}{2}\ab(\sin\ab(\frac{j + j'}{N+1}\pi) + \sin\ab((j + j')\pi) + \sin\ab(-\frac{j+j'}{N+1}\pi))\bigg/\sin\ab(\frac{1}{2(N+1)}(j+j')\pi)    \\
                                               & + \frac{1}{2}\ab(\sin\ab(\frac{j - j'}{N+1}\pi) + \sin\ab((j - j')\pi) + \sin\ab(-\frac{j-j'}{N+1}\pi))\bigg/\sin\ab(\frac{1}{2(N+1)}(j-j')\pi)\bigg]
                       \end{aligned} \\
                   & = 0.
\end{align}
よって (xii), (xiii) の考察から次の式が成り立つ.
\begin{align}
  \tilde{B}_{j,j'} = \delta_{j,j'}.
\end{align}

(xiv) これより $B_{j,j'}$ は (x) の考察から次のようになる.
\begin{align}
  B_{j,j'} & = 4\sin\ab(\frac{\pi}{2}\frac{j}{N+1})\sin\ab(\frac{\pi}{2}\frac{j'}{N+1})\tilde{B}_{j,j'} \\
           & = \delta_{j,j'}4\sin^2\ab(\frac{\pi}{2(N+1)}j).
\end{align}

(xv) ポテンシャルエネルギーの表式 (vii) に代入して次のようになる.
\begin{align}
  \sum_{i=0}^{N}(q_i - q_{i+1})^2 & = \sum_{j=1}^{N}\sum_{j'=1}^{N}B_{j,j'}Q_jQ_{j'}                                     \\
                                  & = \sum_{j=1}^{N}\sum_{j'=1}^{N}\delta_{j,j'}4\sin^2\ab(\frac{\pi}{2(N+1)}j)Q_jQ_{j'} \\
                                  & = 4\sum_{j=1}^{N}\sin^2\ab(\frac{\pi}{2(N+1)}j)Q_j^2.
\end{align}

(xvi) よって Hamilton 関数は (vii) (xv) から次のように表される.
\begin{align}
  H^{1次元結晶}(q_1,\ldots,q_N, p_1,\ldots,p_N) & = \frac{1}{2m}\sum_{i=1}^{N}p_i^2 + \frac{1}{2}\kappa\sum_{i=0}^{N}(q_i - q_{i+1})^2         \\
                                            & = \frac{1}{2m}\sum_{j=1}^{N}P_j^2 + 2\kappa\sum_{j=1}^{N}\sin^2\ab(\frac{\pi}{2(N+1)}j)Q_j^2 \\
  H^{1次元結晶}(Q_1,\ldots,Q_N, P_1,\ldots,P_N) & = \sum_{j=1}^{N}\ab(\frac{1}{2m}P_j^2 + \frac{1}{2}m\omega_j^2Q_j^2).
\end{align}
ただし, $\omega_j$ を次のように定めた.
\begin{align}
  \omega_j = 2\sqrt{\frac{\kappa}{m}}\sin\ab(\frac{\pi}{2(N+1)}j) \qquad (j = 1,\ldots,N).
\end{align}

\begin{itembox}[l]{Q 17-4.}
  1 次元結晶中の波数 $k$ に対する分散関係 $\omega(k)$ は次のようになる.
  \begin{align}
    \omega(k) & = 2\sqrt{\frac{\kappa}{m}}\sin\ab(\frac{1}{2}ka) \approx \sqrt{\frac{\kappa}{m}}ka + \mathcal{O}((ka)^3) \qquad (ka\ll 1).
  \end{align}
\end{itembox}

(i) $j = 1,\ldots,N$ に対して $j$ 番目の基準振動 $q_i^{(j)}$ は次のように計算される.
\begin{align}
  q_i^{(j)} & = \sqrt{\frac{2}{N+1}}\sin\ab(\frac{\pi}{N+1}ji)             \\
            & = \sqrt{\frac{2}{N+1}}\sin\ab(\frac{\pi}{a}\frac{j}{N+1}x_i) \\
            & = \sqrt{\frac{2}{N+1}}\sin\ab(k_jx_i).
\end{align}
ただし, $i$ 番目の原子の平衡位置の座標を $x_i = ai$ とし, $j$ 番目の基準振動の波数 $k_j$ を次のように定める.
\begin{align}
  k_j := \frac{\pi}{a}\frac{j}{N+1} \qquad (j = 1,\ldots,N).
\end{align}

(ii) 基準振動 $q_i^{(j)}$ の角振動数 $\omega_j$ を波数 $k_j$ の関数として次のように表される.
\begin{align}
  \omega(k_j) & = 2\sqrt{\frac{\kappa}{m}}\sin\ab(\frac{\pi}{2(N+1)}j) \\
              & = 2\sqrt{\frac{\kappa}{m}}\sin\ab(\frac{1}{2}k_ja).
\end{align}
よって分散関係 $\omega = \omega(k)$ は次のように与えられる.
\begin{align}
  \omega(k) & = 2\sqrt{\frac{\kappa}{m}}\sin\ab(\frac{1}{2}ka).
\end{align}

(iii) この 1 次元結晶を伝わる線形波動 (弾性波, 音波) が波数ごとに異なる速さを持って伝播するということから, 1次元結晶中にこれらを重ね合わせて波束が作られたとすると次第に波束の形が変化していき最終的に崩壊する.

(iv) 十分に長波長 $ka\ll 1$ のとき次のように近似することで分散関係 $\omega(k)$ は線形関係となる.
\begin{align}
  \omega(k) & = 2\sqrt{\frac{\kappa}{m}}\sin\ab(\frac{1}{2}ka)                         \\
            & \approx 2\sqrt{\frac{\kappa}{m}}\ab(\frac{1}{2}ka + \mathcal{O}((ka)^3)) \\
            & = \sqrt{\frac{\kappa}{m}}ka + \mathcal{O}((ka)^3) \qquad (ka\ll 1).
\end{align}

(v) 長波長の極限での弾性波の速さを音速という. 固体の音速 $v$ は次のようになる.
\begin{align}
  v & = \lim_{ka\to 0}\frac{\omega(k)}{k} = \sqrt{\frac{\kappa}{m}}a.
\end{align}

(vi) (iv), (v) の考察より十分に長波長のとき分散関係が線形関係となるので 1 次元結晶中では線形波動は音速 $v$ と等しい速さを持って伝搬する.

\begin{itembox}[l]{Q 17-5.}
  1 次元結晶における基準振動の角振動数 $\omega_j$ の分布を明らかにする.
\end{itembox}

(i)(ii) $\omega_j$ は次のように表されることから $j=1,\ldots,N$ に対して単調増加となる.
\begin{align}
  \omega_j & = 2\sqrt{\frac{\kappa}{m}}\sin\ab(\frac{\pi}{2(N+1)}j).
\end{align}
これより $\omega_j$ の最大値と最小値は次のようになる.
\begin{align}
  \omega_{\max} & := \max_{1\leq j\leq N}\omega_j = \omega_N = 2\sqrt{\frac{\kappa}{m}}\sin\ab(\frac{\pi N}{2(N+1)}) \approx 2\sqrt{\frac{\kappa}{m}},                                                          \\
  \omega_{\min} & := \min_{1\leq j\leq N}\omega_j = \omega_1 = 2\sqrt{\frac{\kappa}{m}}\sin\ab(\frac{\pi}{2(N+1)}) \approx 2\sqrt{\frac{\kappa}{m}}\frac{\pi}{2(N+1)} = \sqrt{\frac{\kappa}{m}}\frac{\pi}{N+1}.
\end{align}
\subsection{3 次元結晶における平衡位置の回りの調和振動を記述する Hamilton 関数}
立方格子の各点に平衡位置を持つ $N^3$ 個の原子が全体として立方体に並んだ 3 次元結晶を物理系として記述して、古典力学により考察する。任意の $i_x,i_y,i_z = 1,\ldots,N$ に対してラベル $(i_x,i_y,i_z)$ を持つ原子の平衡位置は格子定数 $a$ を用いて $(ai_x,ai_y,ai_z)$ であるとする.
\begin{itembox}[l]{Q 17-6.}
  このとき 3 次元結晶の Hamilton 関数は次のように与えられる.
  \begin{align}
       & H^{3次元結晶}((q_{i_x, i_y, i_z, \alpha}, p_{i_x, i_y, i_z, \alpha})_{1\leq i_x,i_y,i_z\leq N,\alpha=x,y,z})                                                                                                                                                        \\
    := & \frac{1}{2m}\sum_{i_x=1}^{N}\sum_{i_y=1}^{N}\sum_{i_z=1}^{N}\sum_{\alpha=x,y,z}p_{i_x,i_y,i_z,\alpha}^2                                                                                                                                                         \\
    +  & \frac{1}{2}\kappa\sum_{i_x=0}^{N}\sum_{i_y=0}^{N}\sum_{i_z=0}^{N}\sum_{\alpha=x,y,z}\ab((q_{i_x,i_y,i_z,\alpha} - q_{i_x+1,i_y,i_z,\alpha})^2 + (q_{i_x,i_y,i_z,\alpha} - q_{i_x,i_y+1,i_z,\alpha})^2 + (q_{i_x,i_y,i_z,\alpha} - q_{i_x,i_y,i_z+1,\alpha})^2).
  \end{align}
  ただし $m$ は 1 個の原子の質量であり, $\kappa$ は隣り合った原子間の原子間力のバネ定数とする. また立方体の表面は固定されているとする.
  \begin{align}
    i_x = 0, N+1 \lor i_y = 0, N+1 \lor i_z = 0, N+1 \implies q_{i_x,i_y,i_z,\alpha} = 0.
  \end{align}
\end{itembox}
Q17-3 の考察から 1 次元結晶の系の Hamilton 関数は次のように与えられる.
\begin{align}
  H^{1次元結晶}(q_1,\ldots,q_N, p_1,\ldots,p_N) & := \frac{1}{2m}\sum_{i=1}^{N}p_i^2 + \frac{1}{2}\kappa\sum_{i=0}^{N}(q_i - q_{i+1})^2.
\end{align}
3 次元結晶の系は $N^3$ 個の原子と $3$ 個の自由度があり, それらの原子間力は独立にそれぞれの自由度と原子に働くと考えられる. これより 3 次元結晶の系の Hamilton 関数 $H^{3次元結晶}((q_{i_x, i_y, i_z, \alpha}, p_{i_x, i_y, i_z, \alpha})_{1\leq i_x,i_y,i_z\leq N,\alpha=x,y,z})$ は次のように書ける.
\begin{align}
     & H^{3次元結晶}((q_{i_x, i_y, i_z, \alpha}, p_{i_x, i_y, i_z, \alpha})_{1\leq i_x,i_y,i_z\leq N,\alpha=x,y,z})                                                                                                                                                        \\
  := & \frac{1}{2m}\sum_{i_x=1}^{N}\sum_{i_y=1}^{N}\sum_{i_z=1}^{N}\sum_{\alpha=x,y,z}p_{i_x,i_y,i_z,\alpha}^2                                                                                                                                                         \\
  +  & \frac{1}{2}\kappa\sum_{i_x=0}^{N}\sum_{i_y=0}^{N}\sum_{i_z=0}^{N}\sum_{\alpha=x,y,z}\ab((q_{i_x,i_y,i_z,\alpha} - q_{i_x+1,i_y,i_z,\alpha})^2 + (q_{i_x,i_y,i_z,\alpha} - q_{i_x,i_y+1,i_z,\alpha})^2 + (q_{i_x,i_y,i_z,\alpha} - q_{i_x,i_y,i_z+1,\alpha})^2).
\end{align}
ただし $m$ は 1 個の原子の質量であり, $\kappa$ は隣り合った原子間の原子間力のバネ定数とする. また立方体の表面は固定されているとする.
\begin{align}
  i_x = 0, N+1 \lor i_y = 0, N+1 \lor i_z = 0, N+1 \implies q_{i_x,i_y,i_z,\alpha} = 0.
\end{align}

\subsection{3 次元結晶における平衡位置の回りの調和振動の基準モードの計算}
固定端境界条件の 3 次元結晶の系を考えているので 1 次元の Fourier Sine 展開の基底 3 つの直積が基準振動になっていると予想できる. これより古い座標 $q_{i_x,i_y,i_z,\alpha}$ を基準振動 $q_{i_x}^{(j_x)}q_{i_y}^{(j_y)}q_{i_z}^{(j_z)}$ で展開したときの振幅を新しい座標 $Q_{j_x,j_y,j_z,\alpha}$ とする.
\begin{align}
  q_{i_x,i_y,i_z,\alpha} & = \sum_{j_x=1}^{N}\sum_{j_y=1}^{N}\sum_{j_z=1}^{N}Q_{j_x,j_y,j_z,\alpha}q_{i_x}^{(j_x)}q_{i_y}^{(j_y)}q_{i_z}^{(j_z)}.
\end{align}
この新しい座標 $Q_{j_x,j_y,j_z,\alpha}$ に対応する新しい運動量を $P_{j_x, j_y, j_z, \alpha}$ とおくと Hamilton 関数について次のように表される.

\begin{itembox}[l]{Q 17-7.}
  新しい座標と運動量 $Q_{j_x, j_y, j_z, \alpha}, P_{j_x, j_y, j_z, \alpha}$ において Hamilton 関数は次のように表される.
  \begin{align}
    H^{3次元結晶}((Q_{j_x, j_y, j_z, \alpha}, P_{j_x, j_y, j_z, \alpha})_{1\leq j_x,j_y,j_z\leq N,\alpha=x,y,z}) & = \sum_{j_x=1}^{N}\sum_{j_y=1}^{N}\sum_{j_z=1}^{N}\sum_{\alpha=x,y,z}\ab(\frac{1}{2m}P_{j_x,j_y,j_z,\alpha}^2 + \frac{1}{2}m\omega_{j_x,j_y,j_z}^2Q_{j_x,j_y,j_z,\alpha}^2).
  \end{align}
  ただし, $\omega_{j_x,j_y,j_z}$ は次のように定めた.
  \begin{align}
    \omega_{j_x,j_y,j_z} & = 2\sqrt{\frac{\kappa}{m}}\sqrt{\sin^2\ab(\frac{\pi}{2(N+1)}j_x) + \sin^2\ab(\frac{\pi}{2(N+1)}j_y) + \sin^2\ab(\frac{\pi}{2(N+1)}j_z)}.
  \end{align}
\end{itembox}

(i) Q17-1 の考察より新しい運動量を古い運動量と座標, 新しい座標から求めることができる.
\begin{align}
  P_{j_x,j_y,j_z,\alpha} & = \sum_{i_x=1}^{N}\sum_{i_y=1}^{N}\sum_{i_z=1}^{N}\diffp{q_{i_x,i_y,i_z,\alpha}}{Q_{j_x,j_y,j_z,\alpha}}p_{i_x,i_y,i_z,\alpha} \\
                         & = \sum_{i_x=1}^{N}\sum_{i_y=1}^{N}\sum_{i_z=1}^{N}q_{i_x}^{(j_x)}q_{i_y}^{(j_y)}q_{i_z}^{(j_z)}p_{i_x,i_y,i_z,\alpha}.
\end{align}

(ii) この点正準変換に対し, 運動エネルギーは新しい運動量を用いて表せられる.
\begin{align}
    & \sum_{j_x=1}^{N}\sum_{j_y=1}^{N}\sum_{j_z=1}^{N}P_{j_x,j_y,j_z,\alpha}^2                                                                                                                                                                                                                             \\
  = & \sum_{j_x=1}^{N}\sum_{j_y=1}^{N}\sum_{j_z=1}^{N}\ab(\sum_{i_x=1}^{N}\sum_{i_y=1}^{N}\sum_{i_z=1}^{N}q_{i_x}^{(j_x)}q_{i_y}^{(j_y)}q_{i_z}^{(j_z)}p_{i_x,i_y,i_z,\alpha})^2                                                                                                                           \\
  = & \sum_{j_x=1}^{N}\sum_{j_y=1}^{N}\sum_{j_z=1}^{N}\ab(\sum_{i_x=1}^{N}\sum_{i_y=1}^{N}\sum_{i_z=1}^{N}\sum_{i_x'=1}^{N}\sum_{i_y'=1}^{N}\sum_{i_z'=1}^{N}q_{i_x}^{(j_x)}q_{i_y}^{(j_y)}q_{i_z}^{(j_z)}p_{i_x,i_y,i_z,\alpha}q_{i_x'}^{(j_x)}q_{i_y'}^{(j_y)}q_{i_z'}^{(j_z)}p_{i_x',i_y',i_z',\alpha}) \\
  = & \sum_{i_x=1}^{N}\sum_{i_y=1}^{N}\sum_{i_z=1}^{N}\sum_{i_x'=1}^{N}\sum_{i_y'=1}^{N}\sum_{i_z'=1}^{N}\delta_{i_x,i_x'}\delta_{i_y,i_y'}\delta_{i_z,i_z'}p_{i_x,i_y,i_z,\alpha}p_{i_x',i_y',i_z',\alpha}                                                                                                \\
  = & \sum_{i_x=1}^{N}\sum_{i_y=1}^{N}\sum_{i_z=1}^{N}p_{i_x,i_y,i_z,\alpha}^2.
\end{align}

(iii) またポテンシャルエネルギーについても新しい座標で表すことができる.
\begin{align}
    & \sum_{i_x=0}^{N}\sum_{i_y=0}^{N}\sum_{i_z=0}^{N}(q_{i_x,i_y,i_z,\alpha} - q_{i_x+1,i_y,i_z,\alpha})^2                                                                                                                                                                                                                                              \\
  = & \sum_{i_x=0}^{N}\sum_{i_y=0}^{N}\sum_{i_z=0}^{N}\ab(\sum_{j_x=1}^{N}\sum_{j_y=1}^{N}\sum_{j_z=1}^{N}\ab(Q_{j_x,j_y,j_z,\alpha}q_{i_x}^{(j_x)}q_{i_y}^{(j_y)}q_{i_z}^{(j_z)} - Q_{j_x,j_y,j_z,\alpha}q_{i_x+1}^{(j_x)}q_{i_y}^{(j_y)}q_{i_z}^{(j_z)}))^2                                                                                            \\
  = & \sum_{i_x=0}^{N}\sum_{i_y=0}^{N}\sum_{i_z=0}^{N}\sum_{j_x=1}^{N}\sum_{j_y=1}^{N}\sum_{j_z=1}^{N}\sum_{j_x'=1}^{N}\sum_{j_y'=1}^{N}\sum_{j_z'=1}^{N}                                                                                                                                                                                                \\
    & \ab(Q_{j_x,j_y,j_z,\alpha}q_{i_x}^{(j_x)}q_{i_y}^{(j_y)}q_{i_z}^{(j_z)} - Q_{j_x,j_y,j_z,\alpha}q_{i_x+1}^{(j_x)}q_{i_y}^{(j_y)}q_{i_z}^{(j_z)})\ab(Q_{j_x',j_y',j_z',\alpha}q_{i_x}^{(j_x')}q_{i_y}^{(j_y')}q_{i_z}^{(j_z')} - Q_{j_x',j_y',j_z',\alpha}q_{i_x+1}^{(j_x')}q_{i_y}^{(j_y')}q_{i_z}^{(j_z')})                                       \\
  = & \sum_{i_x=0}^{N}\sum_{i_y=0}^{N}\sum_{i_z=0}^{N}\sum_{j_x=1}^{N}\sum_{j_y=1}^{N}\sum_{j_z=1}^{N}\sum_{j_x'=1}^{N}\sum_{j_y'=1}^{N}\sum_{j_z'=1}^{N}Q_{j_x,j_y,j_z,\alpha}\ab(q_{i_x}^{(j_x)} - q_{i_x+1}^{(j_x)})q_{i_y}^{(j_y)}q_{i_z}^{(j_z)}Q_{j_x',j_y',j_z',\alpha}\ab(q_{i_x}^{(j_x')} - q_{i_x+1}^{(j_x')})q_{i_y}^{(j_y')}q_{i_z}^{(j_z')} \\
  = & \sum_{j_x=1}^{N}\sum_{j_y=1}^{N}\sum_{j_z=1}^{N}\sum_{j_x'=1}^{N}\sum_{j_y'=1}^{N}\sum_{j_z'=1}^{N}B_{j_x,j_x'}\delta_{j_y,j_y'}\delta_{j_z,j_z'}Q_{j_x,j_y,j_z,\alpha}Q_{j_x',j_y',j_z',\alpha}                                                                                                                                                   \\
  = & \sum_{j_x=1}^{N}\sum_{j_y=1}^{N}\sum_{j_z=1}^{N}\sum_{j_x'=1}^{N}\sum_{j_y'=1}^{N}\sum_{j_z'=1}^{N}4\sin^2\ab(\frac{\pi}{2(N+1)}j_x)\delta_{j_x,j_x'}\delta_{j_y,j_y'}\delta_{j_z,j_z'}Q_{j_x,j_y,j_z,\alpha}Q_{j_x',j_y',j_z',\alpha}                                                                                                             \\
  = & 4\sum_{j_x=1}^{N}\sum_{j_y=1}^{N}\sum_{j_z=1}^{N}\sin^2\ab(\frac{\pi}{2(N+1)}j_x)Q_{j_x,j_y,j_z,\alpha}^2.
\end{align}

(iv) これより Hamilton 関数は新しい座標と運動量を用いて表すことができる.
\begin{align}
     & H^{3次元結晶}((q_{i_x, i_y, i_z, \alpha}, p_{i_x, i_y, i_z, \alpha})_{1\leq i_x,i_y,i_z\leq N,\alpha=x,y,z})                                                                                                                                                       \\
  := & \frac{1}{2m}\sum_{i_x=1}^{N}\sum_{i_y=1}^{N}\sum_{i_z=1}^{N}\sum_{\alpha=x,y,z}p_{i_x,i_y,i_z,\alpha}^2                                                                                                                                                        \\
  +  & \frac{1}{2}\kappa\sum_{i_x=0}^{N}\sum_{i_y=0}^{N}\sum_{i_z=0}^{N}\sum_{\alpha=x,y,z}\ab((q_{i_x,i_y,i_z,\alpha} - q_{i_x+1,i_y,i_z,\alpha})^2 + (q_{i_x,i_y,i_z,\alpha} - q_{i_x,i_y+1,i_z,\alpha})^2 + (q_{i_x,i_y,i_z,\alpha} - q_{i_x,i_y,i_z+1,\alpha})^2) \\
  =  & \frac{1}{2m}\sum_{j_x=1}^{N}\sum_{j_y=1}^{N}\sum_{j_z=1}^{N}\sum_{\alpha=x,y,z}P_{j_x,j_y,j_z,\alpha}^2                                                                                                                                                        \\
  +  & 2\kappa\sum_{i_x=0}^{N}\sum_{i_y=0}^{N}\sum_{i_z=0}^{N}\sum_{\alpha=x,y,z}\ab(\sin^2\ab(\frac{\pi}{2(N+1)}j_x)Q_{j_x,j_y,j_z,\alpha}^2 + \sin^2\ab(\frac{\pi}{2(N+1)}j_y)Q_{j_x,j_y,j_z,\alpha}^2 + \sin^2\ab(\frac{\pi}{2(N+1)}j_z)Q_{j_x,j_y,j_z,\alpha}^2)  \\
  =  & \sum_{j_x=1}^{N}\sum_{j_y=1}^{N}\sum_{j_z=1}^{N}\sum_{\alpha=x,y,z}\ab(\frac{1}{2m}P_{j_x,j_y,j_z,\alpha}^2 + 2\kappa\ab(\sin^2\ab(\frac{\pi}{2(N+1)}j_x) + \sin^2\ab(\frac{\pi}{2(N+1)}j_y) + \sin^2\ab(\frac{\pi}{2(N+1)}j_z))Q_{j_x,j_y,j_z,\alpha}^2)      \\
  =  & \sum_{j_x=1}^{N}\sum_{j_y=1}^{N}\sum_{j_z=1}^{N}\sum_{\alpha=x,y,z}\ab(\frac{1}{2m}P_{j_x,j_y,j_z,\alpha}^2 + \frac{1}{2}m\omega_{j_x,j_y,j_z}^2Q_{j_x,j_y,j_z,\alpha}^2).
\end{align}
ただし, $\omega_{j_x,j_y,j_z}$ は次のように定めた.
\begin{align}
  \omega_{j_x,j_y,j_z} & = 2\sqrt{\frac{\kappa}{m}}\sqrt{\sin^2\ab(\frac{\pi}{2(N+1)}j_x) + \sin^2\ab(\frac{\pi}{2(N+1)}j_y) + \sin^2\ab(\frac{\pi}{2(N+1)}j_z)}.
\end{align}

これより 3 次元結晶の模型の基準振動は位置や運動量に独立な角振動数 $\omega_{j_x,j_y,j_z}$ の調和振動子となることがわかった.

\begin{itembox}[l]{Q 17-8.}
  3 次元結晶中の波数 $k$ における分散関係 $\omega(k)$ は次のように表される.
  \begin{align}
    \omega(\bm{k}) & = 2\sqrt{\frac{\kappa}{m}}\sqrt{\sin^2\ab(\frac{a}{2}k_x) + \sin^2\ab(\frac{a}{2}k_y) + \sin^2\ab(\frac{a}{2}k_z)} \approx \sqrt{\frac{\kappa}{m}}a|\bm{k}| + \mathcal{O}(|\bm{k}|^3) \qquad (a|\bm{k}| \ll 1).
  \end{align}
\end{itembox}

(i) 3 次元結晶の模型の基準振動は角振動数 $\omega_{j_x,j_y,j_z}$ に依存し, それに対する波数 $\bm{k}_{j_x,j_y,j_z} = (k_{j_x}, k_{j_y}, k_{j_z})$ を考えると次のようになる.
\begin{align}
  \omega_{j_x,j_y,j_z} & = 2\sqrt{\frac{\kappa}{m}}\sqrt{\sin^2\ab(\frac{\pi}{2(N+1)}j_x) + \sin^2\ab(\frac{\pi}{2(N+1)}j_y) + \sin^2\ab(\frac{\pi}{2(N+1)}j_z)} \\
                       & = 2\sqrt{\frac{\kappa}{m}}\sqrt{\sin^2\ab(\frac{a}{2}k_{j_x}) + \sin^2\ab(\frac{a}{2}k_{j_y}) + \sin^2\ab(\frac{a}{2}k_{j_z})}.
\end{align}
これより基準振動に対する波数 $\bm{k}_{j_x,j_y,j_z}$ は次のように定められる.
\begin{align}
  \bm{k}_{j_x,j_y,j_z} & = \frac{\pi}{a(N+1)}(j_x,j_y,j_z).
\end{align}

(ii) このように定めた波数を連続的に捉え直すことで分散関係 $\omega(\bm{k})$ は波数 $\bm{k} = (k_x, k_y, k_z)$ を用いて次のようになる.
\begin{align}
  \omega(\bm{k}) & = 2\sqrt{\frac{\kappa}{m}}\sqrt{\sin^2\ab(\frac{a}{2}k_x) + \sin^2\ab(\frac{a}{2}k_y) + \sin^2\ab(\frac{a}{2}k_z)}.
\end{align}

(iii) このとき長波長 ($a|\bm{k}| \ll 1$) では分散関係は次の線形関係となることがわかる.
\begin{align}
  \omega(\bm{k}) & = 2\sqrt{\frac{\kappa}{m}}\sqrt{\sin^2\ab(\frac{a}{2}k_x) + \sin^2\ab(\frac{a}{2}k_y) + \sin^2\ab(\frac{a}{2}k_z)}                                                          \\
                 & \approx 2\sqrt{\frac{\kappa}{m}}\sqrt{\ab(\frac{a}{2}k_x + \mathcal{O}(k_x^3))^2 + \ab(\frac{a}{2}k_y + \mathcal{O}(k_y^3))^2 + \ab(\frac{a}{2}k_z + \mathcal{O}(k_z^3))^2} \\
                 & = 2\sqrt{\frac{\kappa}{m}}\sqrt{\ab(\frac{a}{2}|\bm{k}|)^2 + \mathcal{O}(|\bm{k}|^4)}                                                                                       \\
                 & = 2\sqrt{\frac{\kappa}{m}}\ab(\frac{a}{2}|\bm{k}|\sqrt{1 + \mathcal{O}(|\bm{k}|^2)})                                                                                        \\
                 & \approx \sqrt{\frac{\kappa}{m}}a|\bm{k}| + \mathcal{O}(|\bm{k}|^3) \qquad (a|\bm{k}| \ll 1).
\end{align}

(iv) これより音速 $v$ はその定義式から次のようになる.
\begin{align}
  v & = \lim_{|\bm{k}|\to 0}\frac{\omega}{|\bm{k}|} = \sqrt{\frac{\kappa}{m}}a.
\end{align}

\begin{itembox}[l]{Q 17-9.}
  3 次元結晶の模型における調和振動子の角振動数の個数分布関数 $g(\omega)$ は次のように表される.
  \begin{align}
    g(\omega) & = 3\sum_{j_x=1}^{N}\sum_{j_y=1}^{N}\sum_{j_z=1}^{N}\delta(\omega - \omega(\bm{k}_{j_x,j_y,j_z})).
  \end{align}
\end{itembox}

(i) 調和振動子の角振動数 $\omega(\bm{k}_{j_x, j_y, j_z})$ の個数分布関数 $g(\omega)$ について $\omega(\bm{k}_{j_x, j_y, j_z})$ は離散的な値を持ち, 各基準モード $(j_x, j_y, j_z, \alpha)$ によってパラメータ化されるのでデルタ関数を用いて次のように表される.
\begin{align}
  g(\omega) & = \sum_{j_x=1}^{N}\sum_{j_y=1}^{N}\sum_{j_z=1}^{N}\sum_{\alpha=x,y,z}\delta(\omega - \omega(\bm{k}_{j_x,j_y,j_z})) \\
            & = 3\sum_{j_x=1}^{N}\sum_{j_y=1}^{N}\sum_{j_z=1}^{N}\delta(\omega - \omega(\bm{k}_{j_x,j_y,j_z})).
\end{align}
また $\omega(\bm{k}_{j_x,j_y,j_z})$ は $\omega(\bm{k}_{j_x,j_y,j_z})\geq 0$ に限られるから $\omega\geq 0$ となる.

(ii) これより調和振動子の総数は次のようになる.
\begin{align}
  \int_0^\infty\dl{\omega}g(\omega) & = 3\int_0^\infty\dl{\omega}\sum_{j_x=1}^{N}\sum_{j_y=1}^{N}\sum_{j_z=1}^{N}\delta(\omega - \omega(\bm{k}_{j_x,j_y,j_z})) \\
                                    & = 3N^3.
\end{align}

ただこのような調和振動子の角振動数の個数分布関数 $g(\omega)$ をさらに簡単にすることは分散関係 $\omega(\bm{k})$ の複雑さのためにできない為, これに統計力学を適用しても計算がすぐに行き詰まる.

Debye はこの模型を修正することでこの困難を打開した. 新しい模型には解析計算ができるという要請と十分に低温であるか, あるいは十分に高温であるかという温度に関する両極端な漸近領域においてこれまでの模型と同じ結果を導くという要請をした.

\subsection{Debye模型}
以下では独立な調和振動子の角振動数に関する個数分布関数 $g(\omega)$ を解析的に計算できるよう分散関係を修正した新しい模型を考える. これを Debye 模型という.

\begin{itembox}[l]{Q 17-10.}
  十分に高温において前節の模型と新しい模型が同じ比熱の極限値を持つには独立な調和振動子の総数について一致することが必要十分である.
\end{itembox}

十分に高温ではエントロピーが高くなる為, すべての独立な調和振動子のエネルギー状態について実現確率は等分配される. このとき比熱は独立な調和振動子の総数のみに依存するから前節の模型と等しい総数となることが必要十分である.

\begin{itembox}[l]{Q 17-11.}
  十分に低温において前節の模型と新しい模型が同じ比熱の漸近的な振る舞いを示すためには分散関係の関数 $\omega(\bm{k})$ が長波長の漸近領域 $a|\bm{k}| \ll 1$ において一致することが十分である.
\end{itembox}

十分に低温ではエントロピーが低くなり, エネルギーが低い状態, つまり長波長に関する状態に実現確率が集まるので, 前節の模型と新しい模型について長波長の漸近領域において分散関係が一致するなら同じ比熱の漸近的な振る舞いとなることが言える. \\

これらより Debye 模型では独立な調和振動子の総数が $3N^3$ で調和振動子の角振動数 $\omega(\bm{k})$ は次のように定義する.
\begin{align}
  \omega(\bm{k}) & := \sqrt{\frac{\kappa}{m}}a|\bm{k}|.
\end{align}
また新しい模型の固有モードのラベルは前節と同じく $(j_x, j_y, j_z, \alpha)$ $(j_x,j_y,j_x=1,\ldots,N,\alpha=x,y,z)$ とし, 固有モード $(j_x, j_y, j_z, \alpha)$ の空間的な波数 $\bm{k}_{j_x,j_y,j_z}$ は次のように与えられる.
\begin{align}
  \bm{k}_{j_x,j_y,j_z} & = \frac{\pi}{a(N+1)}(j_x,j_y,j_z).
\end{align}

\begin{itembox}[l]{Q 17-12.}
  Debye 模型における調和振動子の角振動数の個数分布関数 $g(\omega)$ は次のように表される.
  \begin{align}
    g(\omega) & = \begin{dcases}
                    \frac{9N^3}{\omega_D}\ab(\frac{\omega}{\omega_D})^2 & (\omega\leq\omega_D) \\
                    0                                                   & (\omega > \omega_D)
                  \end{dcases} \\
    \omega_D  & = (6\pi^2)^{1/3}\sqrt{\frac{\kappa}{m}}.
  \end{align}
\end{itembox}

(i) Debye 模型における調和振動子の角振動数の個数分布関数 $g(\omega)$ は $\omega(\bm{k}_{j_x, j_y, j_z})$ が固有モード $(j_x, j_y, j_z, \alpha)$ によってパラメータ化されるのでデルタ関数を用いて次のように表される.
\begin{align}
  g(\omega) & = \sum_{j_x=1}^{N}\sum_{j_y=1}^{N}\sum_{j_z=1}^{N}\sum_{\alpha=x,y,z}\delta(\omega - \omega(\bm{k}_{j_x,j_y,j_z}))      \\
            & = 3\sum_{j_x=1}^{N}\sum_{j_y=1}^{N}\sum_{j_z=1}^{N}\delta(\omega - \omega(\bm{k}_{j_x,j_y,j_z})) \qquad (\omega\geq 0).
\end{align}

(ii) また調和振動子の総数は 3 次元結晶の模型と同様に $3N^3$ となる.
\begin{align}
  \int_0^\infty\dl{\omega}g(\omega) & = 3\int_0^\infty\dl{\omega}\sum_{j_x=1}^{N}\sum_{j_y=1}^{N}\sum_{j_z=1}^{N}\sum_{\alpha=x,y,z}\delta(\omega - \omega(\bm{k}_{j_x,j_y,j_z})) \\
                                    & = 3N^3.
\end{align}

(iii) ここでDebye 模型における調和振動子の角振動数の個数分布関数 $g(\omega)$ を具体的に計算すると次のようになる.
\begin{align}
  g(\omega) & = 3\sum_{j_x=1}^{N}\sum_{j_y=1}^{N}\sum_{j_z=1}^{N}\delta(\omega - \omega(\bm{k}_{j_x,j_y,j_z}))                                                                                 \\
            & = 3\sum_{j_x=1}^{N}\sum_{j_y=1}^{N}\sum_{j_z=1}^{N}\delta\ab(\omega - \sqrt{\frac{\kappa}{m}}a\ab|\frac{\pi}{a(N+1)}(j_x,j_y,j_z)|)                                              \\
            & = 3\sum_{j_x=1}^{N}\sum_{j_y=1}^{N}\sum_{j_z=1}^{N}\delta\ab(\omega - \sqrt{\frac{\kappa}{m}}\frac{\pi}{N+1}\sqrt{j_x^2 + j_y^2 + j_z^2})                                        \\
            & = 3\sqrt{\frac{m}{\kappa}}\frac{N+1}{\pi}\sum_{j_x=1}^{N}\sum_{j_y=1}^{N}\sum_{j_z=1}^{N}\delta\ab(\sqrt{\frac{m}{\kappa}}\frac{N+1}{\pi}\omega - \sqrt{j_x^2 + j_y^2 + j_z^2}).
\end{align}

(iv) またデルタ関数を少し広がった有限の Gauss 分布とすることで $g(\omega)$ を滑らかな分布として近似できる. これより総和は次のように積分で置き換えられることが言える.
\begin{align}
  g(\omega) & = 3\sqrt{\frac{m}{\kappa}}\frac{N+1}{\pi}\sum_{j_x=1}^{N}\sum_{j_y=1}^{N}\sum_{j_z=1}^{N}\delta\ab(\sqrt{\frac{m}{\kappa}}\frac{N+1}{\pi}\omega - \sqrt{j_x^2 + j_y^2 + j_z^2})                    \\
            & \approx 3\sqrt{\frac{m}{\kappa}}\frac{N+1}{\pi}\int_{1}^{N}\dl{j_x}\int_{1}^{N}\dl{j_y}\int_{1}^{N}\dl{j_z}\delta\ab(\sqrt{\frac{m}{\kappa}}\frac{N+1}{\pi}\omega - \sqrt{j_x^2 + j_y^2 + j_z^2}).
\end{align}

(v) ここで $\omega$ に関する次の条件が成り立つとする.
\begin{align}
  \sqrt{\frac{m}{\kappa}}\frac{N+1}{\pi}\omega \leq N. \label{omega_condition}
\end{align}
特に $g(\omega)$ の被積分関数の積分値は次のような幾何学的解釈で近似できる.
\begin{align}
          & \int_{1}^{N}\dl{j_x}\int_{1}^{N}\dl{j_y}\int_{1}^{N}\dl{j_z}\delta\ab(\sqrt{\frac{m}{\kappa}}\frac{N+1}{\pi}\omega - \sqrt{j_x^2 + j_y^2 + j_z^2})                            \\
  =       & \int_V\dl{\bm{r}}\delta\ab(|\bm{r}| - \sqrt{\frac{m}{\kappa}}\frac{N+1}{\pi}\omega) \qquad \ab(V := \lbrace (x, y, z)\mid 1\leq x\leq N, 1\leq y\leq N, 1\leq z\leq N\rbrace) \\
  \approx & \ab(半径 \sqrt{\frac{m}{\kappa}}\frac{N+1}{\pi}\omega の 2 次元球面 S_2 を第 1 象限で切り取った曲面の表面積).
\end{align}
これより $g(\omega)$ は次のように書ける.
\begin{align}
  g(\omega) & \approx 3\sqrt{\frac{m}{\kappa}}\frac{N+1}{\pi}\times\ab(半径 \sqrt{\frac{m}{\kappa}}\frac{N+1}{\pi}\omega の 2 次元球面 S_2 を第 1 象限で切り取った曲面の表面積).
\end{align}

(vi) それを具体的に計算すると次のようになる.
\begin{align}
  g(\omega) & \approx 3\sqrt{\frac{m}{\kappa}}\frac{N+1}{\pi}\times\ab(半径 \sqrt{\frac{m}{\kappa}}\frac{N+1}{\pi}\omega の 2 次元球面 S_2 を第 1 象限で切り取った曲面の表面積) \\
            & = 3\sqrt{\frac{m}{\kappa}}\frac{N+1}{\pi}\times\frac{4\pi}{8}\ab(\sqrt{\frac{m}{\kappa}}\frac{N+1}{\pi}\omega)^2                           \\
            & = \frac{3\pi}{2}\ab(\sqrt{\frac{m}{\kappa}}\frac{N+1}{\pi})^3\omega^2.
\end{align}

(vii) $\omega$ に関する条件 \eqref{omega_condition} が成り立たない場合は立方体の積分範囲と球面の表面の共通部分の面積となるので複雑な式となってしまう. ただ Debye 模型は低温における比熱の振る舞いからの要請により $\omega(\bm{k})$ が大きいときは気にしなくて良い模型でした.
これより $g(\omega)$ の $(j_x, j_y, j_z)$ に関する積分範囲を立方体から球へ修正することが許され, 次のように $g(\omega)$ は表される.
\begin{align}
  g(\omega) & = \begin{dcases}
                  \frac{3\pi}{2}\ab(\sqrt{\frac{m}{\kappa}}\frac{N}{\pi})^3\omega^2 & (\omega\leq\omega_D) \\
                  0                                                                 & (\omega > \omega_D)
                \end{dcases}.
\end{align}
ただし $N\gg 1$ であることから $N+1$ を $N$ と近似し, また打ち切る角振動数 $\omega_D$ を次のように定める.
\begin{align}
  \int_0^\infty\dl{\omega}g(\omega) & = \int_0^{\omega_D}\dl{\omega}g(\omega) = 3N^3.
\end{align}
この $\omega_D$ を Debye の角振動数という.

(viii) これより Debye の角振動数 $\omega_D$ は次のように計算される.
\begin{align}
  \int_0^{\omega_D}\dl{\omega}g(\omega) & = \int_0^{\omega_D}\dl{\omega}\frac{3\pi}{2}\ab(\sqrt{\frac{m}{\kappa}}\frac{N}{\pi})^3\omega^2 = \frac{\pi}{2}\ab(\sqrt{\frac{m}{\kappa}}\frac{N}{\pi})^3\omega_D^3 = 3N^3, \\
  \omega_D                              & = \ab(3N^3\frac{2}{\pi})^{1/3}\sqrt{\frac{\kappa}{m}}\frac{\pi}{N} = (6\pi^2)^{1/3}\sqrt{\frac{\kappa}{m}}.
\end{align}

(ix) また Debye の角振動数 $\omega_D$ を用いて $g(\omega)$ は次のように表される.
\begin{align}
  g(\omega) & = \begin{dcases}
                  \frac{3\pi}{2}\ab(\sqrt{\frac{m}{\kappa}}\frac{N}{\pi})^3\omega^2 & (\omega\leq\omega_D) \\
                  0                                                                 & (\omega > \omega_D)
                \end{dcases} \\
            & = \begin{dcases}
                  \frac{9N^3}{\omega_D}\ab(\frac{\omega}{\omega_D})^2 & (\omega\leq\omega_D) \\
                  0                                                   & (\omega > \omega_D)
                \end{dcases}.
\end{align}

現実の物質に Debye 模型を当てはめるときには, それぞれの物質は固有の Debye 角振動数 $\omega_D$ を持つことになる.

\subsection{量子論での基準モード}
今まで古典力学により行ってきた考察を量子力学に翻訳する. まず Debye 模型の Hamilton 関数は次のように与えられる.
\begin{align}
  \hat{H} & = \frac{1}{2m}\sum_{i_x=1}^{N}\sum_{i_y=1}^{N}\sum_{i_z=1}^{N}\sum_{\alpha=x,y,z}\hat{p}_{i_x,i_y,i_z,\alpha}^2                                                                                                                                                                                       \\
          & + \frac{1}{2}\kappa\sum_{i_x=0}^{N}\sum_{i_y=0}^{N}\sum_{i_z=0}^{N}\sum_{\alpha=x,y,z}\ab((\hat{q}_{i_x,i_y,i_z,\alpha} - \hat{q}_{i_x+1,i_y,i_z,\alpha})^2 + (\hat{q}_{i_x,i_y,i_z,\alpha} - \hat{q}_{i_x,i_y+1,i_z,\alpha})^2 + (\hat{q}_{i_x,i_y,i_z,\alpha} - \hat{q}_{i_x,i_y,i_z+1,\alpha})^2).
\end{align}
ただし $m$ は 1 個の原子の質量であり, $\kappa$ は隣り合った原子間の原子間力のバネ定数とする. また立方体の表面は固定されているとする.
\begin{align}
  i_x = 0, N+1\lor i_y = 0, N+1\lor i_z = 0, N+1 \implies \hat{q}_{i_x,i_y,i_z,\alpha} = 0.
\end{align}
また位置演算子 $\hat{q}_{i_x,i_y,i_z,\alpha}$ と運動量演算子 $\hat{p}_{i_x',i_y',i_z',\alpha'}$ は正準交換関係を満たす.
\begin{align}
  \ab[\hat{q}_{i_x,i_y,i_z,\alpha}, \hat{p}_{i_x',i_y',i_z',\alpha'}] & = \sqrt{-1}\hbar\delta_{i_x,i_x'}\delta_{i_y,i_y'}\delta_{i_z,i_z'}\delta_{\alpha,\alpha'}, \\
  \ab[\hat{q}_{i_x,i_y,i_z,\alpha}, \hat{q}_{i_x',i_y',i_z',\alpha'}] & = \ab[\hat{p}_{i_x,i_y,i_z,\alpha}, \hat{p}_{i_x',i_y',i_z',\alpha'}] = 0                   \\
  (1\leq i_x,i_y,i_z,i_x',i_y',i_z'                                   & \leq N, \alpha,\alpha' = x,y,z).
\end{align}
古典論での点正準変換を量子論でも行う. $(\hat{q}_{i_x,i_y,i_z,\alpha}, \hat{p}_{i_x,i_y,i_z,\alpha})_{1\leq i_x,i_y,i_z\leq N,\alpha=x,y,z}\to(\hat{Q}_{j_x,j_y,j_z,\alpha}, \hat{P}_{j_x,j_y,j_z,\alpha})_{1\leq j_x,j_y,j_z\leq N,\alpha=x,y,z}$ を次のように定める.
\begin{align}
  \hat{q}_{i_x,i_y,i_z,\alpha} & = \sum_{j_x=1}^{N}\sum_{j_y=1}^{N}\sum_{j_z=1}^{N}\hat{Q}_{j_x,j_y,j_z,\alpha}q_{i_x}^{(j_x)}q_{i_y}^{(j_y)}q_{i_z}^{(j_z)} \qquad (1\leq i_x,i_y,i_z \leq N, \alpha = x,y,z), \\
  \hat{P}_{j_x,j_y,j_z,\alpha} & = \sum_{i_x=1}^{N}\sum_{i_y=1}^{N}\sum_{i_z=1}^{N}\hat{p}_{i_x,i_y,i_z,\alpha}q_{i_x}^{(j_x)}q_{i_y}^{(j_y)}q_{i_z}^{(j_z)} \qquad (1\leq j_x,j_y,j_z \leq N, \alpha = x,y,z).
\end{align}

\begin{itembox}[l]{Q 17-13.}
  新しい位置演算子 $\hat{Q}_{j_x,j_y,j_z,\alpha}$ と運動量演算子 $\hat{P}_{j_x',j_y',j_z',\alpha'}$ について正準交換関係を満たす.
  \begin{align}
    \ab[\hat{Q}_{j_x,j_y,j_z,\alpha}, \hat{P}_{j_x',j_y',j_z',\alpha'}] & = \sqrt{-1}\hbar\delta_{j_x,j_x'}\delta_{j_y,j_y'}\delta_{j_z,j_z'}\delta_{\alpha,\alpha'}, \\
    \ab[\hat{Q}_{j_x,j_y,j_z,\alpha}, \hat{Q}_{j_x',j_y',j_z',\alpha'}] & = \ab[\hat{P}_{j_x,j_y,j_z,\alpha}, \hat{P}_{j_x',j_y',j_z',\alpha'}] = 0                   \\
    (1\leq j_x,j_y,j_z,j_x',j_y',j_z'                                   & \leq N, \alpha,\alpha' = x,y,z).
  \end{align}
\end{itembox}
まず $\hat{q}_{i_x,i_y,i_z,\alpha}$, $\hat{P}_{j_x',j_y',j_z',\alpha'}$ の交換関係について左を展開するものと右を展開するもので分けて計算すると次のようになる.
\begin{align}
  \ab[\hat{q}_{i_x,i_y,i_z,\alpha}, \hat{P}_{j_x',j_y',j_z',\alpha'}] & = \ab[\hat{q}_{i_x,i_y,i_z,\alpha}, \sum_{i_x'=1}^{N}\sum_{i_y'=1}^{N}\sum_{i_z'=1}^{N}\hat{p}_{i_x',i_y',i_z',\alpha'}q_{i_x'}^{(j_x')}q_{i_y'}^{(j_y')}q_{i_z'}^{(j_z')}]                      \\
                                                                      & = \sum_{i_x'=1}^{N}\sum_{i_y'=1}^{N}\sum_{i_z'=1}^{N}\ab[\hat{q}_{i_x,i_y,i_z,\alpha}, \hat{p}_{i_x',i_y',i_z',\alpha'}]q_{i_x'}^{(j_x')}q_{i_y'}^{(j_y')}q_{i_z'}^{(j_z')}                      \\
                                                                      & = \sum_{i_x'=1}^{N}\sum_{i_y'=1}^{N}\sum_{i_z'=1}^{N}\sqrt{-1}\hbar\delta_{i_x,i_x'}\delta_{i_y,i_y'}\delta_{i_z,i_z'}\delta_{\alpha,\alpha'}q_{i_x'}^{(j_x')}q_{i_y'}^{(j_y')}q_{i_z'}^{(j_z')} \\
                                                                      & = \sqrt{-1}\hbar\delta_{\alpha,\alpha'}q_{i_x}^{(j_x')}q_{i_y}^{(j_y')}q_{i_z}^{(j_z')},                                                                                                         \\
  \ab[\hat{q}_{i_x,i_y,i_z,\alpha}, \hat{P}_{j_x',j_y',j_z',\alpha'}] & = \ab[\sum_{j_x=1}^{N}\sum_{j_y=1}^{N}\sum_{j_z=1}^{N}\hat{Q}_{j_x,j_y,j_z,\alpha}q_{i_x}^{(j_x)}q_{i_y}^{(j_y)}q_{i_z}^{(j_z)}, \hat{P}_{j_x',j_y',j_z',\alpha'}]                               \\
                                                                      & = \sum_{j_x=1}^{N}\sum_{j_y=1}^{N}\sum_{j_z=1}^{N}\ab[\hat{Q}_{j_x,j_y,j_z,\alpha}, \hat{P}_{j_x',j_y',j_z',\alpha'}]q_{i_x}^{(j_x)}q_{i_y}^{(j_y)}q_{i_z}^{(j_z)}.
\end{align}
これより $q_{i_x}^{(j_x)}q_{i_y}^{(j_y)}q_{i_z}^{(j_z)}$ の直交性から次のことがわかる.
\begin{align}
  \ab[\hat{Q}_{j_x,j_y,j_z,\alpha}, \hat{P}_{j_x',j_y',j_z',\alpha'}] & = \sqrt{-1}\hbar\delta_{j_x,j_x'}\delta_{j_y,j_y'}\delta_{j_z,j_z'}\delta_{\alpha,\alpha'}.
\end{align}
同様に $\hat{q}_{i_x,i_y,i_z,\alpha}$ 同士, $\hat{P}_{j_x,j_y,j_z,\alpha}$ 同士の交換関係について計算すると次のようになる.
\begin{align}
  \ab[\hat{q}_{i_x,i_y,i_z,\alpha}, \hat{q}_{i_x',i_y',i_z',\alpha'}] & = \ab[\sum_{j_x=1}^{N}\sum_{j_y=1}^{N}\sum_{j_z=1}^{N}\hat{Q}_{j_x,j_y,j_z,\alpha}q_{i_x}^{(j_x)}q_{i_y}^{(j_y)}q_{i_z}^{(j_z)}, \sum_{j_x'=1}^{N}\sum_{j_y'=1}^{N}\sum_{j_z'=1}^{N}\hat{Q}_{j_x',j_y',j_z',\alpha'}q_{i_x'}^{(j_x')}q_{i_y'}^{(j_y')}q_{i_z'}^{(j_z')}] \\
                                                                      & = \sum_{j_x=1}^{N}\sum_{j_y=1}^{N}\sum_{j_z=1}^{N}\sum_{j_x'=1}^{N}\sum_{j_y'=1}^{N}\sum_{j_z'=1}^{N}\ab[\hat{Q}_{j_x,j_y,j_z,\alpha}, \hat{Q}_{j_x',j_y',j_z',\alpha'}]q_{i_x}^{(j_x)}q_{i_y}^{(j_y)}q_{i_z}^{(j_z)}q_{i_x'}^{(j_x')}q_{i_y'}^{(j_y')}q_{i_z'}^{(j_z')} \\
                                                                      & = 0,                                                                                                                                                                                                                                                                     \\
  \ab[\hat{P}_{j_x,j_y,j_z,\alpha}, \hat{P}_{j_x',j_y',j_z',\alpha'}] & = \ab[\sum_{i_x=1}^{N}\sum_{i_y=1}^{N}\sum_{i_z=1}^{N}\hat{p}_{i_x,i_y,i_z,\alpha}q_{i_x}^{(j_x)}q_{i_y}^{(j_y)}q_{i_z}^{(j_z)}, \sum_{i_x'=1}^{N}\sum_{i_y'=1}^{N}\sum_{i_z'=1}^{N}\hat{p}_{i_x',i_y',i_z',\alpha}q_{i_x'}^{(j_x')}q_{i_y'}^{(j_y')}q_{i_z'}^{(j_z')}]  \\
                                                                      & = \sum_{i_x=1}^{N}\sum_{i_y=1}^{N}\sum_{i_z=1}^{N}\sum_{i_x'=1}^{N}\sum_{i_y'=1}^{N}\sum_{i_z'=1}^{N}\ab[\hat{p}_{i_x,i_y,i_z,\alpha}, \hat{p}_{i_x',i_y',i_z',\alpha}]q_{i_x}^{(j_x)}q_{i_y}^{(j_y)}q_{i_z}^{(j_z)}q_{i_x'}^{(j_x')}q_{i_y'}^{(j_y')}q_{i_z'}^{(j_z')}  \\
                                                                      & = 0.
\end{align}
これより $q_{i_x}^{(j_x)}q_{i_y}^{(j_y)}q_{i_z}^{(j_z)}q_{i_x'}^{(j_x')}q_{i_y'}^{(j_y')}q_{i_z'}^{(j_z')}$ の直交性から次のことがわかる.
\begin{align}
  \ab[\hat{Q}_{j_x,j_y,j_z,\alpha}, \hat{Q}_{j_x',j_y',j_z',\alpha'}] = \ab[\hat{P}_{j_x,j_y,j_z,\alpha}, \hat{P}_{j_x',j_y',j_z',\alpha'}] = 0.
\end{align}
よって示された.
\begin{align}
  \ab[\hat{Q}_{j_x,j_y,j_z,\alpha}, \hat{P}_{j_x',j_y',j_z',\alpha'}] & = \sqrt{-1}\hbar\delta_{j_x,j_x'}\delta_{j_y,j_y'}\delta_{j_z,j_z'}\delta_{\alpha,\alpha'}, \\
  \ab[\hat{Q}_{j_x,j_y,j_z,\alpha}, \hat{Q}_{j_x',j_y',j_z',\alpha'}] & = \ab[\hat{P}_{j_x,j_y,j_z,\alpha}, \hat{P}_{j_x',j_y',j_z',\alpha'}] = 0                   \\
  (1\leq j_x,j_y,j_z,j_x',j_y',j_z'                                   & \leq N, \alpha,\alpha' = x,y,z).
\end{align}

\begin{itembox}[l]{Q 17-14.}
  Hamilton 演算子 $\hat{H}$ は独立な調和振動子の Hamilton 演算子の和となる.
  \begin{align}
    \hat{H} & = \sum_{j_x=1}^{N}\sum_{j_y=1}^{N}\sum_{j_z=1}^{N}\sum_{\alpha=x,y,z}\ab(\frac{1}{2m}\hat{P}_{j_x,j_y,j_z,\alpha}^2 + \frac{1}{2}m\omega_{j_x,j_y,j_z}^2\hat{Q}_{j_x,j_y,j_z,\alpha}^2).
  \end{align}
  ただし $\omega_{j_x,j_y,j_z}$ は次のように与えられる.
  \begin{align}
    \omega_{j_x,j_y,j_z} & = 2\sqrt{\frac{\kappa}{m}}\sqrt{\sin^2\ab(\frac{\pi}{2(N+1)}j_x) + \sin^2\ab(\frac{\pi}{2(N+1)}j_y) + \sin^2\ab(\frac{\pi}{2(N+1)}j_z)}.
  \end{align}
\end{itembox}
Q 17-7 で位置, 運動量が演算子だとしても同様に計算できるよう書いたので同じ結果が得られる. よって Hamilton 演算子は次のように書ける.
\begin{align}
  \hat{H} & = \sum_{j_x=1}^{N}\sum_{j_y=1}^{N}\sum_{j_z=1}^{N}\sum_{\alpha=x,y,z}\ab(\frac{1}{2m}\hat{P}_{j_x,j_y,j_z,\alpha}^2 + \frac{1}{2}m\omega_{j_x,j_y,j_z}^2\hat{Q}_{j_x,j_y,j_z,\alpha}^2).
\end{align}
ただし $\omega_{j_x,j_y,j_z}$ は次のように与えられる.
\begin{align}
  \omega_{j_x,j_y,j_z} & = 2\sqrt{\frac{\kappa}{m}}\sqrt{\sin^2\ab(\frac{\pi}{2(N+1)}j_x) + \sin^2\ab(\frac{\pi}{2(N+1)}j_y) + \sin^2\ab(\frac{\pi}{2(N+1)}j_z)}.
\end{align}

\subsection{Debye 模型による固体の比熱 $C$}
\begin{itembox}[l]{Q 17-15.}
  Debye 模型における内部エネルギーの表式は次のようになる.
  \begin{align}
    U & = U_0 + 9N^3\hbar\omega_DI(\beta\hbar\omega_D).
  \end{align}
  ただし温度 $T$ に依存しない定数のエネルギー $U_0$, $I(b)$ について次のように定められる.
  \begin{align}
    U_0  & = \frac{3}{8}(3N^3)\hbar\omega_D,       \\
    I(b) & = \int_0^1\dl{x}\frac{x^3}{e^{bx} - 1}.
  \end{align}
\end{itembox}
\begin{align}
  U & = \int_0^\infty\dl{\omega}g(\omega)u(\omega)                                                                                                      \\
    & = \int_0^{\omega_D}\dl{\omega}\frac{9N^3}{\omega_D}\ab(\frac{\omega}{\omega_D})^2\ab(\frac{1}{2} + \frac{1}{e^{\beta\hbar\omega} - 1})\hbar\omega \\
    & = 9N^3\hbar\int_0^{\omega_D}\dl{\omega}\ab(\frac{\omega}{\omega_D})^3\ab(\frac{1}{2} + \frac{1}{e^{\beta\hbar\omega} - 1})                        \\
    & = 9N^3\hbar\omega_D\int_0^1\dl{x}\ab(\frac{1}{2} + \frac{1}{e^{\beta\hbar\omega_Dx} - 1})x^3                                                      \\
    & = \frac{3}{8}(3N^3)\hbar\omega_D + 9N^3\hbar\omega_DI(\beta\hbar\omega_D)                                                                         \\
    & = U_0 + 9N^3\hbar\omega_DI(\beta\hbar\omega_D).
\end{align}
ただし温度 $T$ に依存しない定数のエネルギー $U_0$, $I(b)$ について次のように定められる.
\begin{align}
  U_0  & = \frac{3}{8}(3N^3)\hbar\omega_D,       \\
  I(b) & = \int_0^1\dl{x}\frac{x^3}{e^{bx} - 1}.
\end{align}

以下からは $b = \beta\hbar\omega_D = \hbar\omega_D/(k_BT)$ という関係を用いる.

\begin{itembox}[l]{Q 17-16.}
  Debye 模型における比熱 $C$ の表式は次のようになる.
  \begin{align}
    C & = 3nR\cdot(-3)b^2\diff{I(b)}{b}.
  \end{align}
\end{itembox}

比熱の定義式に代入することで次のようになる.
\begin{align}
  C & = \int_0^\infty\dl{\omega}g(\omega)c(\omega)                                                                                                                         \\
    & = \int_0^{\omega_D}\dl{\omega}\frac{9N^3}{\omega_D}\ab(\frac{\omega}{\omega_D})^2k_B\ab(\frac{\beta\hbar\omega e^{\beta\hbar\omega/2}}{e^{\beta\hbar\omega} - 1})^2  \\
    & = 9k_BN^3(\beta\hbar\omega_D)^2\int_0^{\omega_D}\frac{\dl{\omega}}{\omega_D}\ab(\frac{\omega}{\omega_D})^4\frac{ e^{\beta\hbar\omega}}{(e^{\beta\hbar\omega} - 1)^2} \\
    & = 3nR\cdot 3b^2\int_0^1\dl{x}\frac{x^4e^{bx}}{(e^{bx} - 1)^2}                                                                                                        \\
    & = 3nR\cdot (-3)b^2\diff{I(b)}{b}.
\end{align}

\begin{itembox}[l]{Q 17-17.}
  高温の漸近領域 $b\ll 1$ における積分 $I(b)$ は次のように評価できる.
  \begin{align}
    I(b) & = \frac{1}{3b} - \frac{1}{8} + \frac{1}{60}b - \frac{1}{5040}b^3 + \frac{1}{272160}b^5 - \cdots.
  \end{align}
\end{itembox}

(i) $x\ll 1$ において $e^x \approx 1 + x$ と近似できる. これより高温の漸近領域 $b\ll 1$ において $bx \ll 1$ であるから $I(b)$ は次のように近似できる.
\begin{align}
  I(b) & = \int_0^1\dl{x}\frac{x^3}{e^{bx} - 1} \approx \int_0^1\dl{x}\frac{x^3}{bx} = \int_0^1\dl{x}\frac{x^2}{b} = \frac{1}{3b}.
\end{align}

(ii) Bernoulli 数 $B_n$ の定義を用いて次のように計算できる.
\begin{align}
  I(b) & = \int_0^1\dl{x}\frac{x^3}{e^{bx} - 1}                                                           \\
       & = \int_0^1\dl{x}\sum_{n=0}^{\infty}\frac{B_n b^{n-1}}{n!}x^{n+2}                                 \\
       & = \sum_{n=0}^{\infty}\frac{B_n}{(n + 3)n!}b^{n-1}                                                \\
       & = \frac{1}{3b} - \frac{1}{8} + \frac{1}{60}b - \frac{1}{5040}b^3 + \frac{1}{272160}b^5 - \cdots.
\end{align}

\begin{itembox}[l]{Q 17-18.}
  高温の漸近領域 $b\ll 1$ における比熱 $C$ は次のように評価できる.
  \begin{align}
    C & = 3nR\ab(1 - \frac{1}{20}\ab(\frac{\hbar\omega_D}{k_BT})^2 + \frac{1}{560}\ab(\frac{\hbar\omega_D}{k_BT})^4 - \frac{1}{18144}\ab(\frac{\hbar\omega_D}{k_BT})^6 + \cdots).
  \end{align}
\end{itembox}

(i) まず Q 17-17(i) の結果を比熱の表式に適用すると次のようになる.
\begin{align}
  C & = 3nR\cdot (-3)b^2\diff{I(b)}{b} \approx 3nR\cdot (-3)b^2\diff{b}(\frac{1}{3b}) = 3nR.
\end{align}

(ii) 次に Q 17-17(ii) の結果を比熱の表式に適用すると次のようになる.
\begin{align}
  C & = 3nR\cdot (-3)b^2\diff{I(b)}{b}                                                                                                                                          \\
    & \approx 3nR\cdot (-3)b^2\diff{b}(\frac{1}{3b} - \frac{1}{8} + \frac{1}{60}b - \frac{1}{5040}b^3 + \frac{1}{272160}b^5 - \cdots)                                           \\
    & = 3nR\cdot (-3)b^2\ab(-\frac{1}{3b^2} + \frac{1}{60} - \frac{1}{1680}b^2 + \frac{1}{54432}b^4 - \cdots)                                                                   \\
    & = 3nR\ab(1 - \frac{1}{20}b^2 + \frac{1}{560}b^4 - \frac{1}{18144}b^6 + \cdots)                                                                                            \\
    & = 3nR\ab(1 - \frac{1}{20}\ab(\frac{\hbar\omega_D}{k_BT})^2 + \frac{1}{560}\ab(\frac{\hbar\omega_D}{k_BT})^4 - \frac{1}{18144}\ab(\frac{\hbar\omega_D}{k_BT})^6 + \cdots).
\end{align}

\begin{itembox}[l]{Q 17-19.}
  低温の漸近領域 $b\gg 1$ における積分 $I(b)$ は次のように評価できる.
  \begin{align}
    I(b) & \approx \frac{\pi^4}{15}\frac{1}{b^4}.
  \end{align}
\end{itembox}

(i)
初項 $e^{-bx}$ 公比 $e^{-bx}$ の無限等比数列の和は $1/(e^{bx} + 1)$ である. これより $I(b)$ は次のように表される.
\begin{align}
  I(b) & = \int_0^1\dl{x}\frac{x^3}{e^{bx} - 1} = \int_0^1\dl{x}x^3\sum_{n=1}^{\infty}e^{-nbx} = \sum_{n=1}^{\infty}\int_0^1\dl{x}x^3e^{-nbx}.
\end{align}

(ii) これより
\begin{align}
  I(b) & = \sum_{n=1}^{\infty}\int_0^1\dl{x}x^3e^{-nbx}                                   \\
       & = \sum_{n=1}^{\infty}\frac{1}{(nb)^4}\int_0^{nb}\dl{t}t^3e^{-t} \qquad (t = nbx) \\
       & = \sum_{n=1}^{\infty}\frac{1}{(nb)^4}\gamma(4, nb).
\end{align}
ただし, 第一種不完全ガンマ関数 $\gamma(z, p)$ は次の式で定義される.
\begin{align}
  \gamma(z, p) & := \int_0^p\dl{t}t^{z-1}e^{-t}.
\end{align}

(iii)
さらに $I(b)$ は次のように式変形できる.
\begin{align}
  I(b) & = \sum_{n=1}^{\infty}\frac{1}{(nb)^4}\gamma(4, nb)                                                    \\
       & = \sum_{n=1}^{\infty}\frac{1}{(nb)^4}(\Gamma(4) - \Gamma(4, nb))                                      \\
       & = \frac{1}{b^4}\ab(6\sum_{n=1}^{\infty}\frac{1}{n^4} - \sum_{n=1}^{\infty}\frac{1}{n^4}\Gamma(4, nb)) \\
       & = \frac{1}{b^4}\ab(6\zeta(4) - \sum_{n=1}^{\infty}\frac{1}{n^4}\Gamma(4, nb)).
\end{align}
ただし, 第 2 種不完全ガンマ関数 $\Gamma(z, p)$, ガンマ関数 $\Gamma(z)$, ゼータ関数 $\zeta(z)$ は次のように定義される.
\begin{align}
  \Gamma(z, p) & := \int_p^\infty\dl{t}t^{z-1}e^{-t}                               \\
  \Gamma(z)    & := \int_0^\infty\dl{t}t^{z-1}e^{-t} = \gamma(z, p) + \Gamma(z, p) \\
  \zeta(s)     & := \sum_{n=1}^{\infty}\frac{1}{n^s}.
\end{align}

(iv)
ここでゼータ関数 $\zeta(4)$ の値は次の通りとなる.
\begin{align}
  \zeta(4) & = \frac{\pi^4}{90}.
\end{align}
よって $I(b)$ は次のようになる.
\begin{align}
  I(b) & = \frac{1}{b^4}\ab(6\zeta(4) - \sum_{n=1}^{\infty}\frac{1}{n^4}\Gamma(4, nb))         \\
       & = \frac{1}{b^4}\ab(\frac{\pi^4}{15} - \sum_{n=1}^{\infty}\frac{1}{n^4}\Gamma(4, nb)).
\end{align}

(v) 第二種不完全ガンマ関数 $\Gamma(z,p)$ の $p$ の極限について積分範囲が小さくなっていき, 被積分関数は発散しないので次のようになる.
\begin{align}
  \lim_{p\to+\infty}\Gamma(z, p) & = \lim_{p\to+\infty}\int_p^\infty\dl{t}t^{z-1}e^{-t} = 0.
\end{align}

(vi) 低温の漸近領域 $b\gg 1$ において (v) の考察から第二項を無視した近似を行えることがいえる. よって $I(b)$ は次の値となる.
\begin{align}
  I(b) & = \frac{1}{b^4}\ab(\frac{\pi^4}{15} - \sum_{n=1}^{\infty}\frac{1}{n^4}\Gamma(4, nb)) \approx \frac{\pi^4}{15}\frac{1}{b^4}.
\end{align}

\begin{itembox}[l]{Q 17-20.}
  低温の漸近領域 $b\gg 1$ における積分 $I(b)$ はより精密に次のように評価される.
  \begin{align}
    I(b) & \approx \frac{\pi^4}{15}\frac{1}{b^4} - b^3e^{-b}.
  \end{align}
\end{itembox}

(i)
$\Gamma(z, p)$ について部分積分することで次のように書ける.
\begin{align}
   & \quad \Gamma(z, p)                                                                                                                                                               \\
   & = \int_p^\infty\dl{t}t^{z-1}e^{-t}                                                                                                                                               \\
   & = -\ab[t^{z-1}e^{-t}]_p^\infty - \ab[(z-1)t^{z-2}e^{-t}]_p^\infty - \cdots - \ab[(z-1)\cdots(z-n)t^{z-n-1}e^{-t}]_p^\infty + \int_p^\infty\dl{t} (z-1)\cdots(z-n)t^{z-n-1}e^{-t} \\
   & = p^{z-1}e^{-p} + (z-1)p^{z-2}e^{-p} + \cdots + (z-1)\cdots(z-n)p^{z-n-1}e^{-p} + \int_p^\infty\dl{t} (z-1)(z-2)\cdots(z-n)t^{z-n-1}e^{-t}                                       \\
   & = p^{z-1}e^{-p}\ab(1 + \sum_{m=1}^{\infty}\frac{1}{p^m}(z-1)(z-2)\cdots(z-m)) \qquad (\because n\to\infty).
\end{align}

(ii)
(i) の結果を用いて $z = 4$ を代入すると次のようになる.
\begin{align}
  \Gamma(4, p) & = p^{3}e^{-p}\ab(1 + \frac{3}{p} + \frac{6}{p^2} + \frac{6}{p^3}) \\
               & = e^{-p}\ab(p^3 + 3p^2 + 6p + 6).
\end{align}

(iii)
これより積分 $I(b)$ の第二種不完全ガンマ関数を展開することで次のようになる.
\begin{align}
  I(b) & = \frac{1}{b^4}\ab(\frac{\pi^4}{15} - \sum_{n=1}^{\infty}\frac{1}{n^4}\Gamma(4, nb))                                                     \\
       & = \frac{1}{b^4}\ab(\frac{\pi^4}{15} - \sum_{n=1}^{\infty}\frac{1}{n^4}e^{-nb}\ab((nb)^3 + 3(nb)^2 + 6nb + 6))                            \\
       & = \frac{1}{b^4}\ab(\frac{\pi^4}{15} - \sum_{n=1}^{\infty}\ab(\frac{b^3}{n} + \frac{3b^2}{n^2} + \frac{6b}{n^3} + \frac{6}{n^4})e^{-nb}).
\end{align}

(iv)
この補正項について次のような不等式が成り立つ.
\begin{align}
  0 < \sum_{n=1}^{\infty}\ab(\frac{b^3}{n} + \frac{3b^2}{n^2} + \frac{6b}{n^3} + \frac{6}{n^4})e^{-nb} < (b^3 + 3b^2 + 6b + 6)\sum_{n=1}^{\infty}e^{-nb} = (b^3 + 3b^2 + 6b + 6)\frac{e^{-b}}{1 - e^{-b}}\sim b^3e^{-b}.
\end{align}
これより上界が指数関数的に小さくなることから $b\gg 1$ のとき $I(b)$ の最低次の漸近評価は十分正確である.
\begin{itembox}[l]{Q 17-21.}
  低温の漸近領域 $b\gg 1$ における比熱 $C$ は次のように評価される.
  \begin{align}
    C & \approx 3nR\times\frac{4\pi^4}{5}\ab(\frac{k_BT}{\hbar\omega_D})^3.
  \end{align}
\end{itembox}
Q 17-19, Q 17-20 で考察したように比熱 $C$ に $I(b)$ の値を代入すると次のようになる.
\begin{align}
  C & = 3nR\cdot(-3)b^2\diff{I(b)}{b}                               \\
    & = 3nR\cdot(-3)b^2\ab(-\frac{\pi^4}{15}\frac{4}{b^5})          \\
    & = 3nR\times\frac{4\pi^4}{5}\ab(\frac{1}{b})^3                 \\
    & = 3nR\times\frac{4\pi^4}{5}\ab(\frac{k_BT}{\hbar\omega_D})^3.
\end{align}

よって Debye 模型の比熱は次のようにまとめられる.
\begin{itembox}[l]{Debye 模型の比熱}
  \begin{align}
    C & \approx 3nR\times\begin{dcases}
                           1                                                 & (k_BT\gg \hbar\omega_D) \\
                           \frac{4\pi^4}{5}\ab(\frac{k_BT}{\hbar\omega_D})^3 & (k_BT\ll \hbar\omega_D)
                         \end{dcases}.
  \end{align}
\end{itembox}

\section{黒体輻射}


\section{グランドカノニカル分布}
\begin{definition}
  内部エネルギー $U(S, V)$ とその束縛変数を変更させたエンタルピー $H(S, p)$ と Helmholtz 自由エネルギー $F(T, V)$ と Gibbs 自由エネルギー $G(T, p)$ を次のように定義する。
  グランドポテンシャル (grand potential) または熱力学ポテンシャル (thermodynamic potential) $J(T, V, \mu)$
  \begin{align}
                 & \qquad \dl{U} = T\dl{S} - p\dl{V} + \mu\dl{N}  \\
    H = U + pV   & \qquad \dl{H} = T\dl{S} + V\dl{p} + \mu\dl{N}  \\
    F = U - TS   & \qquad \dl{F} = -S\dl{T} - p\dl{V} + \mu\dl{N} \\
    G = F + pV   & \qquad \dl{G} = -S\dl{T} + V\dl{p} + \mu\dl{N} \\
    J = F - N\mu & \qquad \dl{J} = -S\dl{T} - p\dl{V} - N\dl{\mu}
  \end{align}
\end{definition}

\begin{theorem}
  \begin{align}
    J = -pV, F = N\mu - pV, G = N\mu
  \end{align}
\end{theorem}

\begin{theorem}
  \begin{align}
    T  & = \ab(\diffp{U}{S})_{V,\mu} & -p & = \ab(\diffp{U}{V})_{S,\mu} & \mu & = \ab(\diffp{U}{\mu})_{S,V} \\
    T  & = \ab(\diffp{H}{S})_{p,\mu} & V  & = \ab(\diffp{H}{p})_{S,\mu} & \mu & = \ab(\diffp{H}{\mu})_{S,p} \\
    -S & = \ab(\diffp{F}{T})_{V,\mu} & -p & = \ab(\diffp{F}{V})_{T,\mu} & \mu & = \ab(\diffp{F}{\mu})_{T,V} \\
    -S & = \ab(\diffp{G}{T})_{p,\mu} & V  & = \ab(\diffp{G}{p})_{T,\mu} & \mu & = \ab(\diffp{G}{\mu})_{T,p} \\
    -S & = \ab(\diffp{J}{T})_{V,\mu} & -p & = \ab(\diffp{J}{V})_{T,\mu} & -N  & = \ab(\diffp{J}{\mu})_{T,V}
  \end{align}
\end{theorem}

\begin{proposition}[Maxwell の関係式]
  \begin{align}
    \ab(\diffp{T}{V})_S  & = -\ab(\diffp{p}{S})_V \\
    \ab(\diffp{T}{p})_S  & = \ab(\diffp{V}{S})_p  \\
    -\ab(\diffp{S}{V})_T & = -\ab(\diffp{p}{T})_V \\
    -\ab(\diffp{S}{p})_T & = \ab(\diffp{V}{T})_p
  \end{align}
\end{proposition}
\begin{proof}
\end{proof}

\begin{definition}[グランドカノニカル分布]
  \begin{align}
    \Xi = \sum_{n}e^{-\beta(E_n - \mu N_n)}
  \end{align}
\end{definition}

\begin{theorem}
  \begin{align}
    J & = -k_BT\ln\Xi                                   \\
    N & = \frac{1}{\beta}\ab(\diffp{\ln\Xi}{\mu})_{T,V}
  \end{align}
\end{theorem}

\begin{theorem}[粒子数の揺らぎ]
  \begin{align}
    \ab(\diffp{N}{\mu})_{T,V} = \beta\langle\Delta N^2\rangle
  \end{align}
\end{theorem}

\section{Bose 統計と Fermi 統計}
\begin{theorem}
  2 粒子の波動関数は $\varphi(\rr_1, \rr_2)$ と書かれる。
  \begin{align}
    \varphi(\rr_1, \rr_2) = \pm \varphi(\rr_2, \rr_1)
  \end{align}
\end{theorem}
\begin{proof}
  添字を交換しても物理的な状態としては同一なので定数 $\alpha$ を用いて $\varphi(\rr_1, \rr_2) = \alpha\varphi(\rr_2, \rr_1)$ と書ける。
  \begin{align}
    \alpha^2 = 1 \iff \alpha = \pm 1
  \end{align}
\end{proof}

\begin{definition}
  上の定理において $\alpha = 1$ となる粒子をボース粒子またはボゾン (boson) といい、$\alpha = -1$ となる粒子をフェルミ粒子またはフェルミオン (fermion) という。

  \begin{itemize}
    \item Fermi 粒子: 電子・陽子
    \item Bose 粒子: 光子
  \end{itemize}
\end{definition}

\begin{itemize}
  \item Fermi 統計: $e^{-\beta(\varepsilon_1 + \varepsilon_2)} + e^{-\beta(\varepsilon_2 + \varepsilon_3)} + e^{-\beta(\varepsilon_3 + \varepsilon_1)}$
  \item Bose 統計: $e^{-2\beta\varepsilon_1} + e^{-2\beta\varepsilon_2} + e^{-2\beta\varepsilon_3} + e^{-\beta(\varepsilon_1 + \varepsilon_2)} + e^{-\beta(\varepsilon_2 + \varepsilon_3)} + e^{-\beta(\varepsilon_3 + \varepsilon_1)}$
  \item ボルツマン統計: $\dfrac{1}{2!}(e^{-2\beta\varepsilon_1} + e^{-2\beta\varepsilon_2} + e^{-2\beta\varepsilon_3})$ \\
        $\dfrac{1}{2}e^{-2\beta\varepsilon_1} + \dfrac{1}{2}e^{-2\beta\varepsilon_2} + \dfrac{1}{2}e^{-2\beta\varepsilon_3} + e^{-\beta(\varepsilon_1 + \varepsilon_2)} + e^{-\beta(\varepsilon_2 + \varepsilon_3)} + e^{-\beta(\varepsilon_3 + \varepsilon_1)}$
\end{itemize}

\begin{align}
  N & = -\ab(\diffp{J}{\mu})_T = \frac{1}{\beta}\ab(\diffp{\ln\Xi(\beta,\mu)}{\mu})
\end{align}

\begin{theorem}[分配関数と分布関数]
  Fermi 統計と Bose 統計における分配関数 $\Xi(\beta,\mu)$、分布関数 $f(\varepsilon)$ は次のようになる。
  \begin{align}
    \Xi_B(\beta, \mu) & = \prod_{j=1}^{\infty}\frac{1}{1 - e^{-\beta(\varepsilon_j - \mu)}} \\
    f_B(\varepsilon)  & = \frac{1}{e^{\beta(\varepsilon_j - \mu)} - 1}                      \\
    \Xi_F(\beta, \mu) & = \prod_{j=1}^{\infty}\ab(1 + e^{-\beta(\varepsilon_j - \mu)})      \\
    f_F(\varepsilon)  & = \frac{1}{e^{\beta(\varepsilon_j - \mu)} + 1}
  \end{align}
\end{theorem}
\begin{proof}
  Fermi 統計において
  \begin{align}
    \Xi^{(j)}(\beta, \mu)                    & = \sum_{n=0}^{\infty}e^{-\beta(\varepsilon_j - \mu)n} = \frac{1}{1 - e^{-\beta(\varepsilon_j - \mu)}}                                                                                   \\
    \Xi(\beta, \mu)                          & = \prod_{j=1}^{\infty}\frac{1}{1 - e^{-\beta(\varepsilon_j - \mu)}}                                                                                                                     \\
    f_B(\varepsilon_j) := \langle n_j\rangle & = \frac{1}{\beta}\ab(\diffp{}{\mu}\ln\Xi^{(j)}(\beta,\mu)) = \frac{e^{-\beta(\varepsilon_j - \mu)}}{1 - e^{-\beta(\varepsilon_j - \mu)}} = \frac{1}{e^{\beta(\varepsilon_j - \mu)} - 1} \\
    N                                        & = \sum_{j=1}^{\infty}\frac{1}{e^{\beta(\varepsilon_j - \mu)} - 1}
  \end{align}
  Bose 統計
  \begin{align}
    \Xi^{(j)}(\beta, \mu)                    & = \sum_{n=0}^{1}e^{-\beta(\varepsilon_j - \mu)n} = 1 + e^{-\beta(\varepsilon_j - \mu)}                                                                                                  \\
    \Xi(\beta, \mu)                          & = \prod_{j=1}^{\infty}\ab(1 - e^{-\beta(\varepsilon_j - \mu)})                                                                                                                          \\
    f_F(\varepsilon_j) := \langle n_j\rangle & = \frac{1}{\beta}\ab(\diffp{}{\mu}\ln\Xi^{(j)}(\beta,\mu)) = \frac{e^{-\beta(\varepsilon_j - \mu)}}{1 + e^{-\beta(\varepsilon_j - \mu)}} = \frac{1}{e^{\beta(\varepsilon_j - \mu)} + 1} \\
    N                                        & = \sum_{j=1}^{\infty}\frac{1}{e^{\beta(\varepsilon_j - \mu)} + 1}
  \end{align}
\end{proof}

\subsection{Fermi-Dirac 統計力学}
\begin{proposition}[低温極限と高温極限での分布関数]
  低温と高温の極限において
  \begin{align}
    \lim_{T\to 0}f_F(\varepsilon)      & = \begin{dcases}
                                             1   & (\varepsilon < \mu) \\
                                             1/2 & (\varepsilon = \mu) \\
                                             0   & (\varepsilon > \mu)
                                           \end{dcases}           \\
    \lim_{T\to \infty}f_F(\varepsilon) & \approx e^{-\beta(\varepsilon - \mu)}
  \end{align}
\end{proposition}

\begin{theorem}[ゾンマーフェルト展開]
  次の積分を次のように展開できる。
  \begin{align}
    I(\beta, \mu) & := \int_{-\infty}^{\infty}\dl{\varepsilon}h(\varepsilon)f_F(\varepsilon) = \int_{-\infty}^\mu h(\varepsilon)\dl{\varepsilon} + \frac{\pi^2}{6}h'(\mu)(k_BT)^2 + O((k_BT)^{4})
  \end{align}
\end{theorem}
\begin{proof}
  \begin{align}
    g(\varepsilon) & = \int_{-\infty}^\varepsilon h(\varepsilon)\dl{\varepsilon}
  \end{align}

  \begin{align}
    I(\beta, \mu) & = \int_{-\infty}^{\infty}\dl{\varepsilon}h(\varepsilon)f_F(\varepsilon)                                                                                  \\
                  & = [g(\varepsilon)f_F(\varepsilon)]_{-\infty}^{\infty} - \int_{-\infty}^{\infty}\dl{\varepsilon}g(\varepsilon)f_F'(\varepsilon)                           \\
                  & = - \int_{-\infty}^{\infty}\dl{\varepsilon}g(\varepsilon)f_F'(\varepsilon)                                                                               \\
                  & = - \int_{-\infty}^{\infty}\dl{\varepsilon}\ab[g(\mu) + g'(\mu)(\varepsilon - \mu) + \frac{1}{2}g''(\mu)(\varepsilon - \mu)^2 + \cdots]f_F'(\varepsilon)
  \end{align}
  $x = \beta(\varepsilon - \mu)$ と変数変換すると奇関数性より
  \begin{align}
    \int_{-\infty}^{\infty}\dl{\varepsilon}f_F'(\varepsilon)                      & = [f_F(\varepsilon)]_{-\infty}^{\infty} = -1                                                                                                                        \\
    \int_{-\infty}^{\infty}\dl{\varepsilon}f_F'(\varepsilon)(\varepsilon - \mu)   & = \int_{-\infty}^{\infty}\dl{\varepsilon}\frac{x}{\beta}\frac{e^x}{(e^x + 1)^2} = \int_{-\infty}^{\infty}\dl{\varepsilon}\frac{x}{\beta}\frac{x}{4\cosh^2(x/2)} = 0 \\
    \int_{-\infty}^{\infty}\dl{\varepsilon}f_F'(\varepsilon)(\varepsilon - \mu)^2 & = \frac{1}{\beta^2}\int_{-\infty}^{\infty}\dl{\varepsilon}x^2\diff{}{x}\ab(\frac{1}{e^x + 1})                                                                       \\
                                                                                  & = \frac{2}{\beta^2}\ab[\frac{x^2}{e^x + 1}]_{0}^{\infty} - \frac{4}{\beta^2}\int_{0}^{\infty}\dl{\varepsilon}\frac{x}{e^x + 1}                                      \\
                                                                                  & = - \frac{4}{\beta^2}\int_{0}^{\infty}\dl{\varepsilon}x\sum_{n=1}^{\infty}(-1)^{n-1}e^{-nx}                                                                         \\
                                                                                  & = - \frac{4}{\beta^2}\sum_{n=1}^{\infty}\frac{(-1)^{n-1}}{2n^2}                                                                                                     \\
                                                                                  & = - \frac{4}{\beta^2}\frac{\pi^2}{12} = -\frac{\pi^2}{3\beta^2}
  \end{align}
  \begin{align}
    I(\beta, \mu) & = g(\mu) + \frac{\pi^2}{6\beta^2}g''(\mu) + O(\beta^{-4})                                            \\
                  & = \int_{-\infty}^\mu h(\varepsilon)\dl{\varepsilon} + \frac{\pi^2}{6}h'(\mu)(k_BT)^2 + O((k_BT)^{4})
  \end{align}
\end{proof}

\begin{theorem}
  \begin{align}
    \mu & \approx \varepsilon_F - \frac{\pi^2}{6}\frac{\nu'(\varepsilon_F)}{\nu(\varepsilon_F)}(k_BT)^2                                          \\
    c(T, \rho) & = \frac{\pi^2}{3}\nu(\varepsilon_F)k_B^2T
  \end{align}
\end{theorem}
\begin{proof}
  \begin{align}
    N & = \int_{-\infty}^{\infty}\dl{\varepsilon}\nu(\varepsilon)f_F(\varepsilon)                                                                                        \\
      & \approx \int_{-\infty}^{\mu}\dl{\varepsilon}\nu(\varepsilon) + \frac{\pi^2}{6}\nu'(\mu)(k_BT)^2                                                                  \\
      & = \int_{-\infty}^{\varepsilon_F}\dl{\varepsilon}\nu(\varepsilon) + \int_{\varepsilon_F}^{\mu}\dl{\varepsilon}\nu(\varepsilon) + \frac{\pi^2}{6}\nu'(\mu)(k_BT)^2 \\
      & \approx N + (\mu - \varepsilon_F)\nu(\varepsilon_F) + \frac{\pi^2}{6}\nu'(\mu)(k_BT)^2
  \end{align}
  \begin{align}
    \mu \approx \varepsilon_F - \frac{\pi^2}{6}\frac{\nu'(\varepsilon_F)}{\nu(\varepsilon_F)}(k_BT)^2
  \end{align}


  \begin{align}
    U & = \int_{-\infty}^{\infty}\dl{\varepsilon}\varepsilon\nu(\varepsilon)f_F(\varepsilon)                                                                                                                                                     \\
      & \approx \int_{-\infty}^\mu\dl{\varepsilon}\varepsilon\nu(\varepsilon) + \frac{\pi^2}{6}(\varepsilon\nu(\varepsilon))'|_{\varepsilon = \mu}(k_BT)^2                                                                                       \\
      & = \int_{-\infty}^{\varepsilon_F}\dl{\varepsilon}\varepsilon\nu(\varepsilon) + \int_{\varepsilon_F}^\mu\dl{\varepsilon}\varepsilon\nu(\varepsilon) + \frac{\pi^2}{6}(\varepsilon\nu(\varepsilon))'|_{\varepsilon = \mu}(k_BT)^2           \\
      & = \int_{-\infty}^{\varepsilon_F}\dl{\varepsilon}\varepsilon\nu(\varepsilon) + (\mu - \varepsilon_F)\varepsilon_F\nu(\varepsilon_F) + \frac{\pi^2}{6}\nu(\varepsilon_F)(k_BT)^2 + \frac{\pi^2}{6}\varepsilon_F\nu'(\varepsilon_F)(k_BT)^2 \\
      & \approx \int_{-\infty}^{\varepsilon_F}\dl{\varepsilon}\varepsilon\nu(\varepsilon) + \frac{\pi^2}{6}\nu(\varepsilon_F)(k_BT)^2
  \end{align}

  \begin{align}
    c(T, \rho) & = \diffp{U}{T} = \frac{\pi^2}{3}\nu(\varepsilon_F)k_B^2T
  \end{align}
\end{proof}

\section{相転移}


\end{document}