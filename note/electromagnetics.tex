\documentclass[uplatex,dvipdfmx,a4paper,11pt]{jlreq}
\usepackage{bxpapersize}
\usepackage[utf8]{inputenc}
\usepackage{fontenc}
\usepackage{lmodern}
\usepackage{otf}
\usepackage{amsmath}
\usepackage{amssymb}
\usepackage{amsthm}
\usepackage{ascmac}
% \usepackage[hyphens]{url}
\usepackage{physics2}
\usephysicsmodule{ab, ab.braket, doubleprod, diagmat, xmat}
\usepackage{diffcoeff}
\usepackage{braket}
\usepackage{verbatimbox}
\usepackage{bm}
\usepackage{url}
% \usepackage[dvipdfmx,hiresbb,final]{graphicx}
\usepackage{hyperref}
\usepackage{pxjahyper}
\usepackage{tikz}\usetikzlibrary{cd}
\usepackage{listings}
\usepackage{color}
\usepackage{mathtools}
\usepackage{xspace}
\usepackage{xy}
\usepackage{xypic}
%
\title{電磁気学}
\author{anko9801}
\makeatletter
%
\DeclareMathOperator{\lcm}{lcm}
\DeclareMathOperator{\Kernel}{Ker}
\DeclareMathOperator{\Image}{Im}
\DeclareMathOperator{\ch}{ch}
\DeclareMathOperator{\Aut}{Aut}
\DeclareMathOperator{\Log}{Log}
\DeclareMathOperator{\Arg}{Arg}
\DeclareMathOperator{\sgn}{sgn}
%
\newcommand{\CC}{\mathbb{C}}
\newcommand{\RR}{\mathbb{R}}
\newcommand{\QQ}{\mathbb{Q}}
\newcommand{\ZZ}{\mathbb{Z}}
\newcommand{\NN}{\mathbb{N}}
\newcommand{\FF}{\mathbb{F}}
\newcommand{\GG}{\mathbb{G}}
\newcommand{\TT}{\mathbb{T}}
\newcommand{\EE}{\bm{E}}
\newcommand{\BB}{\bm{B}}
\newcommand{\DD}{\bm{D}}
\newcommand{\HH}{\bm{H}}
\newcommand{\PP}{\bm{P}}
\newcommand{\MM}{\bm{M}}
\renewcommand{\AA}{\bm{A}}
\newcommand{\rr}{\bm{r}}
\newcommand{\kk}{\bm{k}}
\newcommand{\pp}{\bm{p}}
\newcommand{\Et}{\tilde{E}}
\newcommand{\ET}{\tilde{\bm{E}}}
\newcommand{\Ec}{\mathcal{E}}
\newcommand{\EC}{\mathcal\bm{E}}
\newcommand{\LL}{\bm{L}}
\newcommand{\ee}{\bm{\varepsilon}}
\renewcommand{\SS}{\bm{S}}
\newcommand{\JJ}{\bm{J}}
\newcommand{\vnabla}{\mathbf{\nabla}}
\newcommand{\laplacian}{\nabla^2}
\newcommand{\calB}{\mathcal{B}}
\newcommand{\calF}{\mathcal{F}}
\newcommand{\ignore}[1]{}
\newcommand{\floor}[1]{\left\lfloor #1 \right\rfloor}
% \newcommand{\abs}[1]{\left\lvert #1 \right\rvert}
\newcommand{\lt}{<}
\newcommand{\gt}{>}
\newcommand{\id}{\mathrm{id}}
\newcommand{\rot}{\curl}
\renewcommand{\angle}[1]{\left\langle #1 \right\rangle}
\newcommand\mqty[1]{\begin{pmatrix}#1\end{pmatrix}}
\newcommand\vmqty[1]{\begin{vmatrix}#1\end{vmatrix}}

\numberwithin{equation}{section}

\let\oldcite=\cite
\renewcommand\cite[1]{\hyperlink{#1}{\oldcite{#1}}}

\let\oldbibitem=\bibitem
\renewcommand{\bibitem}[2][]{\label{#2}\oldbibitem[#1]{#2}}

% theorem環境の設定
% - 冒頭に改行
% - 末尾にdiamond (amsthm)
\theoremstyle{definition}
\newcommand*{\newscreentheoremx}[2]{
  \newenvironment{#1}[1][]{
    \begin{screen}
    \begin{#2}[##1]
      \leavevmode
      \newline
  }{
    \end{#2}
    \end{screen}
  }
}
\newcommand*{\newqedtheoremx}[2]{
  \newenvironment{#1}[1][]{
    \begin{#2}[##1]
      \leavevmode
      \newline
      \renewcommand{\qedsymbol}{\(\diamond\)}
      \pushQED{\qed}
  }{
      \qedhere
      \popQED
    \end{#2}
  }
}
\newtheorem{theorem*}{定理}[section]

\newqedtheoremx{theorem}{theorem*}
\newcommand*\newqedtheorem@unstarred[2]{%
  \newtheorem{#1*}[theorem*]{#2}
  \newqedtheoremx{#1}{#1*}
}
\newcommand*\newqedtheorem@starred[2]{%
  \newtheorem*{#1*}{#2}
  \newqedtheoremx{#1}{#1*}
}
\newcommand*{\newqedtheorem}{\@ifstar{\newqedtheorem@starred}{\newqedtheorem@unstarred}}

\newtheorem{sctheorem*}{定理}
\newscreentheoremx{sctheorem}{sctheorem*}
\newcommand*\newscreentheorem@unstarred[2]{%
  \newtheorem{#1*}[theorem*]{#2}
  \newscreentheoremx{#1}{#1*}
}
\newcommand*\newscreentheorem@starred[2]{%
  \newtheorem*{#1*}{#2}
  \newscreentheoremx{#1}{#1*}
}
\newcommand*{\newscreentheorem}{\@ifstar{\newscreentheorem@starred}{\newscreentheorem@unstarred}}

%\newtheorem*{definition}{定義}
%\newtheorem{theorem}{定理}
%\newtheorem{proposition}[theorem]{命題}
%\newtheorem{lemma}[theorem]{補題}
%\newtheorem{corollary}[theorem]{系}

\newqedtheorem{lemma}{補題}
\newqedtheorem{corollary}{系}
\newqedtheorem{example}{例}
\newqedtheorem{proposition}{命題}
\newqedtheorem{remark}{注意}
\newqedtheorem{thesis}{主張}
\newqedtheorem{notation}{記法}
\newqedtheorem{problem}{問題}
\newqedtheorem{algorithm}{アルゴリズム}

\newscreentheorem*{definition}{定義}

\renewenvironment{proof}[1][\proofname]{\par
  \normalfont
  \topsep6\p@\@plus6\p@ \trivlist
  \item[\hskip\labelsep{\bfseries #1}\@addpunct{\bfseries}]\ignorespaces\quad\par
}{%
  \qed\endtrivlist\@endpefalse
}
\renewcommand\proofname{証明}

\makeatother

\begin{document}
\maketitle
\tableofcontents
\clearpage



\section{真空中の電磁気学}
電場 $\EE$ と磁束密度 $\BB$
\subsection{Maxwell 方程式}
\begin{definition}[Maxwell 方程式]
  電場 $\EE$ と磁束密度 $\BB$ に対して次のような式が成り立つ。
  \begin{align}
     & \int_{\partial V}\EE\cdot\bm{n}\dl{S} = \frac{1}{\varepsilon_0}\int_V\rho\dl{V}                                                       &  & \iff \vnabla\cdot\EE  = \frac{\rho}{\varepsilon_0}                               \\
     & \int_{\partial V}\BB\cdot\bm{n}\dl{S} = 0                                                                                             &  & \iff \vnabla\cdot\BB  = 0                                                        \\
     & \int_{\partial S}\EE\cdot\dl{\bm{l}} = -\diff{}{t}\int_S\BB\cdot\bm{n}\dl{S}                                                          &  & \iff \vnabla\times\EE = -\diffp{\BB}{t}                                          \\
     & c^2\int_{\partial S}\BB\cdot\dl{\bm{l}} = \frac{1}{\varepsilon_0}\int_S\bm{j}\cdot\bm{n}\dl{S} + \diff{}{t}\int_S\EE\cdot\bm{n}\dl{S} &  & \iff \vnabla\times\BB = \mu_0\bm{j} + \frac{1}{c^2}\diffp{\EE}{t} \label{Ampere}
  \end{align}
  ただし電荷密度 $\rho(t, \rr) = qn$, 電流密度 $\bm{j}(t, \rr) = qn\bm{v}$ とする。
  \begin{align}
    c = \frac{1}{\sqrt{\varepsilon_0\mu_0}}
  \end{align}
\end{definition}
\begin{definition}[Lorentz 力]
  電荷 $q$ の質点が電磁場から受ける力は次のように表される。
  \begin{align}
    \bm{F} & = q(\EE + \bm{v}\times\BB)
  \end{align}
  一般に次のように表される。
  \begin{align}
    \bm{F} & = \rho\EE + \bm{j}\times\BB
  \end{align}
\end{definition}
これらの法則で電磁気学が完結する。


\subsection{保存則}
\begin{theorem}[電荷の保存則]
  連続の方程式を満たし、電荷は保存する。
  \begin{align}
    \diffp{\rho}{t} + \vnabla\cdot\bm{j} = 0
  \end{align}
\end{theorem}
\begin{proof}
  式 \eqref{Ampere} の両辺の発散を計算することで連続の方程式を導出する。$\vnabla\cdot(\vnabla\times\BB) = 0$ より
  \begin{align}
    \vnabla\cdot(\vnabla\times\BB) & = \vnabla\cdot\ab(\mu_0\bm{j} + \frac{1}{c^2}\diffp{\EE}{t})        \\
                                   & = \mu_0\vnabla\cdot\bm{j} + \frac{1}{c^2}\diffp{}{t}\vnabla\cdot\EE \\
                                   & = \vnabla\cdot\bm{j} + \diffp{\rho}{t} = 0
  \end{align}
  連続の方程式の両辺を空間微分することで電荷が保存することが分かる。
  \begin{align}
     & \diff{}{t}\int_V\rho\dl{V} + \int_{\partial V}\bm{j}\cdot\dl{\bm{S}} = 0
  \end{align}
\end{proof}

\begin{theorem}[磁荷の存在]
  磁荷は存在しない。
\end{theorem}
\begin{proof}
  磁場中のガウスの法則より湧き出し源がない、つまり単極の磁荷が存在しないことが分かる。
\end{proof}

\begin{theorem}[エネルギー保存則]
  真空中での電磁場のエネルギー密度 $u_{EM}(\rr, t)$ とローレンツ力の仕事率 $u_{m}(\rr, t)$ の和を $u(\rr, t)$ とし、エネルギー流密度を $\SS(\rr, t)$ として次の連続の方程式を満たす。
  \begin{align}
    \diffp{u}{t} + \vnabla\cdot\SS & = 0
  \end{align}
\end{theorem}
\begin{proof}
  まず $u_{EM}$, $u_m$, $\SS$ についてそれぞれ
  \begin{align}
    u_{EM}         & = \frac{\varepsilon_0}{2}\EE^2 + \frac{1}{2\mu_0}\BB^2                        \\
    \diffp{u_m}{t} & = \bm{F}\cdot\bm{v} = (\rho\EE + \bm{j}\times\BB)\cdot\bm{v} = \bm{j}\cdot\EE \\
    \SS(\rr)       & = \frac{1}{\mu_0}\EE\times\BB
  \end{align}
  より連続の方程式を満たす。
  \begin{align}
    \diffp{u}{t} & = \diffp{}{t}\ab(\frac{\varepsilon_0}{2}\EE^2 + \frac{1}{2\mu_0}\BB^2) + \bm{j}\cdot\EE                              \\
                 & = \EE\cdot\varepsilon_0\diffp{\EE}{t} + \frac{\BB}{\mu_0}\cdot\diffp{\BB}{t} + \bm{j}\cdot\EE                        \\
                 & = \EE\cdot\ab(\frac{1}{\mu_0}\vnabla\times\BB - \bm{j}) + \frac{\BB}{\mu_0}\cdot(-\vnabla\times\EE) + \bm{j}\cdot\EE \\
                 & = \frac{1}{\mu_0}\ab(\EE\cdot\ab(\vnabla\times\BB) - \BB\cdot(\vnabla\times\EE))                                     \\
                 & = -\frac{1}{\mu_0}\vnabla\cdot(\EE\times\BB) = -\vnabla\cdot\SS
  \end{align}
  連続の方程式の両辺を空間微分することでエネルギーが保存することが分かる。
  \begin{align}
    \diff{}{t}\int u\dl{V} + \int_S\SS\cdot\bm{n}\dl{S} & = 0
  \end{align}
\end{proof}

\begin{definition}[Poynting ベクトル]
  エネルギーの流れの密度, 単位時間に単位面積を通過するエネルギー
  \begin{align}
    \SS(\rr) & = \frac{1}{\mu_0}\EE\times\BB
  \end{align}
\end{definition}

\begin{theorem}[運動量保存則]
  運動量密度 $g_j$ と Maxwell の応力テンソル $T_{ij}$ の反作用を電磁場の運動量流密度と考えると連続の方程式は次のようになる。
  \begin{align}
    \diffp{\bm{g}}{t} - \vnabla\cdot\bm{T} & = 0 \qquad ((\vnabla\cdot\bm{T})_j = \partial_iT_{ij})
  \end{align}
  つまりローレンツ力の反作用が電磁場の運動量の時間変化となる。
\end{theorem}
\begin{proof}
  物質の運動量密度 $\bm{g}_m(\rr, t)$、電磁場の運動量密度 $\bm{g}_{EM}(\rr, t)$、マクスウェルの応力テンソル $T_{ij}$ は次の条件を満たす。
  \begin{align}
    \diffp{\bm{g}_m}{t} & = \rho\EE + \bm{j}\times\BB                                                                         \\
    \bm{g}_{EM}         & = \varepsilon_0\EE\times\BB                                                                         \\
    T_{ij}              & = \varepsilon_0E_iE_j + \frac{1}{\mu_0}B_iB_j - \delta_{ij}u_{EM}                                   \\
    T_{ij}^{(E)}        & = \varepsilon_0\begin{pmatrix}
                                           E_x^2 - \frac{1}{2}|\EE|^2 & E_xE_y                     & E_xE_z                     \\
                                           E_xE_y                     & E_x^2 - \frac{1}{2}|\EE|^2 & E_xE_z                     \\
                                           E_xE_y                     & E_xE_z                     & E_x^2 - \frac{1}{2}|\EE|^2 \\
                                         \end{pmatrix}
  \end{align}
  これより
  \begin{align}
    \diffp{g_j}{t} & = \rho E_j + \varepsilon_{jik}j_iB_k + \diffp{}{t}(\varepsilon_0\varepsilon_{jik}E_iB_k)                                                                                                                            \\
                   & = (\varepsilon_0\partial_iE_i)E_j + \varepsilon_{jik}\ab(\frac{1}{\mu_0}\varepsilon_{ilm}\partial_lB_m - \varepsilon_0\diffp{E_i}{t})B_k + \varepsilon_0\varepsilon_{jik}\ab(\diffp{E_i}{t}B_k + E_i\diffp{B_k}{t}) \\
                   & = \varepsilon_0(\partial_iE_i)E_j + \frac{1}{\mu_0}\varepsilon_{jik}\varepsilon_{ilm}(\partial_lB_m)B_k - \varepsilon_0\varepsilon_{jik}E_i\ab(\varepsilon_{klm}\partial_lE_m)                                      \\
                   & = \varepsilon_0(\partial_iE_i)E_j + \frac{1}{\mu_0}((\partial_iB_j)B_i - (\partial_jB_i)B_i) - \varepsilon_0(E_i(\partial_jE_i) - E_i(\partial_iE_j))                                                               \\
                   & = \partial_i\ab(\varepsilon_0E_iE_j + \frac{1}{\mu_0}B_iB_j) - \partial_j\ab(\frac{1}{2\mu_0}B_i^2 + \frac{1}{2}\varepsilon_0E_i^2) \qquad (\because\partial_iB_i = 0)                                              \\
                   & = \partial_iT_{ij}
  \end{align}
  連続の方程式の両辺を空間微分することで運動量が保存することが分かる。
  \begin{align}
    \diff{}{t}\int \bm{g}\dl{V} & = \int_ST_{ij}\cdot\bm{n}\dl{S}
  \end{align}
\end{proof}

\begin{theorem}[角運動量保存則]
  \begin{align}
    \diffp{\bm{L}}{t} + \vnabla\cdot\MM = 0
  \end{align}
\end{theorem}
\begin{proof}
  \begin{align}
    \bm{L}_{EM} & = \rr\times\bm{g}_{EM} = \varepsilon_0\rr\times(\EE\times\BB) \\
    \bm{L}_m    & = \rr\times\bm{g}_m                                           \\
    \MM         & = \bm{T} \times \rr
  \end{align}
  \begin{align}
    \diffp{\bm{L}}{t} & = \rr\times\diffp{}{t}\ab(\bm{g}_{EM} + \bm{g}_m) = \rr\times(\vnabla\cdot\bm{T}) \\
                      & = \vnabla\cdot(\bm{T}\times\rr)                                                   \\
                      & = \vnabla\cdot\MM
  \end{align}
\end{proof}

\subsection{電磁ポテンシャルとゲージ変換}
\begin{definition}[電磁ポテンシャル]
  次を満たす $\phi$, $\AA$ が存在し、$\phi$ を電位、$\AA$ をベクトルポテンシャルという。
  \begin{align}
    \EE & = - \vnabla\phi - \diffp{\AA}{t} \\
    \BB & = \vnabla\times\AA
  \end{align}
\end{definition}
\begin{proof}
  Maxwell 方程式に代入すると well-defined 性を満たすことが分かる。
  \begin{align}
    \vnabla\cdot\BB                   & = \vnabla\cdot(\vnabla\times\AA) = 0                                          \\
    \vnabla\times\EE + \diffp{\BB}{t} & = \vnabla\times\ab(\EE + \diffp{\AA}{t}) = \vnabla\times\ab(-\vnabla\phi) = 0
  \end{align}
\end{proof}

\begin{theorem}
  Maxwell の方程式は電磁ポテンシャルを用いて次のように表される。
  \begin{align}
     & -\laplacian\phi - \diffp{}{t}(\vnabla\cdot\AA) = \frac{\rho}{\varepsilon_0}                                                \\
     & -\laplacian\AA + \vnabla\ab(\frac{1}{c^2}\diffp{\phi}{t} + \vnabla\cdot\AA) + \frac{1}{c^2}\diffp[2]{\AA}{t} = \mu_0\bm{j}
  \end{align}
\end{theorem}
\begin{proof}
  Maxwell の方程式に電磁ポテンシャルを代入すると次のようになる。
  \begin{align}
    \begin{dcases}
      \vnabla\cdot\EE  = \frac{\rho}{\varepsilon_0} \\
      \vnabla\cdot\BB  = 0                          \\
      \vnabla\times\EE = -\diffp{\BB}{t}            \\
      \vnabla\times\BB = \mu_0\bm{j} + \frac{1}{c^2}\diffp{\EE}{t}
    \end{dcases}
    \iff
    \begin{dcases}
      \EE = - \vnabla\phi - \diffp{\AA}{t}                                        \\
      \BB = \vnabla\times\AA                                                      \\
      -\laplacian\phi - \diffp{}{t}(\vnabla\cdot\AA) = \frac{\rho}{\varepsilon_0} \\
      \ab(\frac{1}{c^2}\diffp[2]{}{t} - \laplacian)\AA + \vnabla\ab(\frac{1}{c^2}\diffp{\phi}{t} + \vnabla\cdot\AA) = \mu_0\bm{j}
    \end{dcases}
  \end{align}
\end{proof}

\begin{theorem}[ゲージ変換]
  任意の関数 $\chi(\rr, t)$ として次のゲージ変換は不変に保つ。
  \begin{align}
    \AA  & \to \AA + \vnabla\chi      \\
    \phi & \to \phi - \diffp{\chi}{t}
  \end{align}
\end{theorem}
\begin{proof}
  電磁場に代入すると不変に保つことがわかる。
  \begin{align}
    \EE & = - \vnabla\ab(\phi - \diffp{\chi}{t}) - \diffp{}{t}(\AA + \vnabla\chi)            \\
        & = - \vnabla\phi + \vnabla\diffp{\chi}{t} - \diffp{\AA}{t} - \diffp{}{t}\vnabla\chi \\
        & = - \vnabla\phi - \diffp{\AA}{t}                                                   \\
    \BB & = \vnabla\times(\AA + \vnabla\chi)                                                 \\
        & = \vnabla\times\AA + \vnabla\times\vnabla\chi                                      \\
        & = \vnabla\times\AA
  \end{align}
\end{proof}

\begin{definition}[ゲージ]
  元々の電磁ポテンシャルに対して適切に $\chi$ を選んでゲージ変換後の電磁ポテンシャルが満たすゲージという。
  次の条件を満たすゲージをクーロンゲージ (Coulomb gauge) と呼び、主に電磁気学で用いられる。
  \begin{align}
    \vnabla\cdot\AA = 0
  \end{align}
  次の条件を満たすゲージをローレンツゲージ (Lorenz gauge) と呼び、主に相対性理論で用いられる。
  \begin{align}
    \frac{1}{c^2}\diffp{\phi}{t} + \vnabla\cdot\AA = 0
  \end{align}
\end{definition}

\begin{proposition}
  任意の状況においてクーロンゲージは存在し、静電磁場において次が成り立つ。
  \begin{align}
    \phi(\rr) & = \frac{1}{4\pi\varepsilon_0}\int_V\frac{\rho(\rr')}{|\rr - \rr'|}\dl{\rr'} \\
    \AA(\rr)  & = \frac{\mu_0}{4\pi}\int_V\frac{\bm{j}(\rr', t)}{|\rr - \rr'|}\dl{\rr'}
  \end{align}
\end{proposition}
\begin{proof}
  任意の電磁ポテンシャル $(\phi, \AA)$ に対してゲージ変換し $(\phi - \diffp{\chi}{t}, \AA + \vnabla\chi)$ がクーロンゲージを満たすような $\chi$ の表式は次のようになる。
  \begin{align}
    \laplacian\chi & = -\vnabla\cdot\AA \iff \chi(\rr, t) = \frac{1}{4\pi}\int_V\frac{\vnabla_{\rr'}\cdot\AA(\rr', t)}{|\rr - \rr'|}\dl{\rr'}
  \end{align}
  これよりクーロンゲージは存在する。クーロンゲージにおいて静電磁場における Maxwell 方程式に代入すると
  \begin{align}
    -\laplacian\phi = \frac{\rho}{\varepsilon_0} & \iff \phi(\rr) = \frac{1}{4\pi\varepsilon_0}\int_V\frac{\rho(\rr', t)}{|\rr - \rr'|}\dl{\rr'} \\
    -\laplacian\AA = \mu_0\bm{j}                 & \iff \AA(\rr)  = \frac{\mu_0}{4\pi}\int_V\frac{\bm{j}(\rr', t)}{|\rr - \rr'|}\dl{\rr'}
  \end{align}
  より電磁ポテンシャルは電荷と電流により求められることが分かる。
\end{proof}

\begin{proposition}
  任意の状況においてローレンツゲージは存在し、電磁ポテンシャルについて次の微分方程式が成り立つ。
  \begin{align}
    \ab(\frac{1}{c^2}\diffp[2]{}{t} - \laplacian)\phi & = \frac{\rho}{\varepsilon_0} \\
    \ab(\frac{1}{c^2}\diffp[2]{}{t} - \laplacian)\AA  & = \mu_0\bm{j}
  \end{align}
\end{proposition}
\begin{proof}
  任意の電磁ポテンシャル $(\phi, \AA)$ に対してゲージ変換し $(\phi - \diffp{\chi}{t}, \AA + \vnabla\chi)$ がローレンツゲージを満たすような $\chi$ の条件は次のようになる。
  \begin{align}
    \ab(\frac{1}{c^2}\diffp[2]{}{t} - \laplacian)\chi = \frac{1}{c^2}\diffp{\phi}{t} + \vnabla\cdot\AA
  \end{align}
  また Maxwell の方程式に代入することで電磁ポテンシャルは次の微分方程式を満たす。
  \begin{align}
    \ab(\frac{1}{c^2}\diffp[2]{}{t} - \laplacian)\phi & = \frac{\rho}{\varepsilon_0} \\
    \ab(\frac{1}{c^2}\diffp[2]{}{t} - \laplacian)\AA  & = \mu_0\bm{j}
  \end{align}
\end{proof}

\begin{itembox}[l]{静電磁場まとめ}
  静電磁場では電荷と電流を用いて電磁ポテンシャルと電磁場が次のように分かる。
  \begin{align}
    \phi(\rr) & = \frac{1}{4\pi\varepsilon_0}\int_V\frac{\rho(\rr')}{|\rr - \rr'|}\dl{\rr'} \\
    \AA(\rr)  & = \frac{\mu_0}{4\pi}\int_V\frac{\bm{j}(\rr')}{|\rr - \rr'|}\dl{\rr'}        \\
    \EE(\rr)  & = -\vnabla\phi                                                              \\
    \BB(\rr)  & = \vnabla\times\AA
  \end{align}
\end{itembox}


\subsection{遅延ポテンシャル}


\subsection{さまざまな環境}
\begin{theorem}[Coulomb 力]
  点電荷 $Q$ を $\rr'$ に配置したときに位置 $\rr$ での電位と電場、点電荷 $q$ に及ぼす力は次のようになる。
  \begin{align}
    \phi(\rr)   & = \frac{Q}{4\pi\varepsilon_0}\frac{1}{|\rr - \rr'|}             \\
    \EE(\rr)    & = \frac{Q}{4\pi\varepsilon_0}\frac{\rr - \rr'}{|\rr - \rr'|^3}  \\
    \bm{F}(\rr) & = \frac{Qq}{4\pi\varepsilon_0}\frac{\rr - \rr'}{|\rr - \rr'|^3}
  \end{align}
\end{theorem}
\begin{proof}
  \begin{align}
    \phi(\rr)   & = \frac{1}{4\pi\varepsilon_0}\int_V\frac{\rho(\rr')}{|\rr - \rr'|}\dl{\rr'} = \frac{1}{4\pi\varepsilon_0}\frac{Q}{|\rr - \rr'|} \\
    \EE(\rr)    & = -\vnabla\phi = \frac{Q}{4\pi\varepsilon_0}\frac{\rr - \rr'}{|\rr - \rr'|^3}                                                   \\
    \bm{F}(\rr) & = q\EE = \frac{Qq}{4\pi\varepsilon_0}\frac{\rr - \rr'}{|\rr - \rr'|^3}
  \end{align}
\end{proof}

\begin{theorem}[電気双極子]
  点電荷 $+Q, -Q$ をそれぞれ $\rr'+\bm{d}/2, \rr'-\bm{d}/2$ に配置したときに位置 $\rr$ での電位と電場は次のようになる。
  \begin{align}
    \phi(\rr) & = \frac{1}{4\pi\varepsilon_0}\frac{\pp\cdot(\rr - \rr')}{|\rr - \rr'|^3}                                       \\
    \EE(\rr)  & = \frac{1}{4\pi\varepsilon_0}\frac{\ab(3\pp\cdot(\rr - \rr'))(\rr - \rr') - (\rr - \rr')^2\pp}{|\rr - \rr'|^5}
  \end{align}
  ただし、電気双極子モーメントを $\pp = Q\bm{d}$ とおく。
\end{theorem}
\begin{proof}
  \begin{align}
    \phi(\rr) & = \frac{1}{4\pi\varepsilon_0}\ab(\frac{Q}{|\rr - \rr' - \bm{d}/2|} - \frac{Q}{|\rr - \rr' + \bm{d}/2|})        \\
              & = \frac{Q}{4\pi\varepsilon_0}\ab(\vnabla'\frac{1}{|\rr - \rr'|})\cdot\bm{d}                                    \\
              & = \frac{1}{4\pi\varepsilon_0}\frac{\pp\cdot(\rr - \rr')}{|\rr - \rr'|^3}                                       \\
    \EE(\rr)  & = -\vnabla\phi(\rr)                                                                                            \\
              & = \frac{1}{4\pi\varepsilon_0}\frac{\ab(3\pp\cdot(\rr - \rr'))(\rr - \rr') - (\rr - \rr')^2\pp}{|\rr - \rr'|^5}
  \end{align}
  \begin{align}
    \phi(r, \theta)       & = \frac{p\cos\theta}{4\pi\varepsilon_0r^2}                                                             \\
    \EE(r,\theta,\varphi) & = -\nabla\phi(r, \theta)                                                                               \\
                          & = \ab(-\diffp{\phi}{r}, -\frac{1}{r}\diffp{\phi}{\theta}, -\frac{1}{r\sin\theta}\diffp{\phi}{\varphi}) \\
                          & = \ab(\frac{p\cos\theta}{2\pi\varepsilon_0r^3}, \frac{p\sin\theta}{4\pi\varepsilon_0r^3}, 0)           \\
  \end{align}
\end{proof}

\begin{theorem}[電気双極子放射]

\end{theorem}

\begin{theorem}[電気四重極子]

\end{theorem}

\begin{theorem}[ビオ・サバールの法則]
  \begin{align}
    \dl{\BB}(\rr) & = \frac{\mu_0}{4\pi}\frac{I\dl{s}\times(\rr - \rr')}{|\rr - \rr'|^3}
  \end{align}
\end{theorem}

\begin{proposition}[ソレノイド]
\end{proposition}



\section{真空中の電磁波}


\subsection{電磁波の基礎}
\begin{proposition}
  $\rho = 0$, $\bm{j} = \bm{0}$ において $\EE, \BB$ は波動方程式を満たす。
  \begin{align}
    \nabla^2\EE = \frac{1}{c^2}\diffp[2]{}{t}\EE \\
    \nabla^2\BB = \frac{1}{c^2}\diffp[2]{}{t}\BB
  \end{align}
\end{proposition}
\begin{proof}
  $\nabla\times(\nabla\times\EE)$ を Maxwell 方程式を用いて 2 通りに計算する。$\rho = 0$, $\bm{j} = \bm{0}$ より示せる。
  \begin{align}
    \nabla\times(\nabla\times\EE) & = \nabla(\nabla\cdot\EE) - \laplacian\EE                     \\
                                  & = \frac{1}{\varepsilon_0}\nabla\rho - \laplacian\EE          \\
    \nabla\times(\nabla\times\EE) & = \nabla\times\ab(-\diffp{\BB}{t})                           \\
                                  & = -\diffp{}{t}\ab(\mu_0\bm{j} + \frac{1}{c^2}\diffp{\EE}{t}) \\
                                  & = -\mu_0\diffp{\bm{j}}{t} - \frac{1}{c^2}\diffp[2]{}{t}\EE   \\
    \nabla^2\EE                   & = \frac{1}{c^2}\diffp[2]{}{t}\EE
  \end{align}
  磁場に関しても同様にして計算すると
  \begin{align}
    \nabla\times(\nabla\times\BB) & = \nabla(\nabla\cdot\BB) - \laplacian\BB                                 \\
                                  & = - \laplacian\BB                                                        \\
    \nabla\times(\nabla\times\BB) & = \nabla\times\ab(\mu_0\bm{j} + \frac{1}{c^2}\diffp{\EE}{t})             \\
                                  & = \mu_0\nabla\times\bm{j} + \frac{1}{c^2}\diffp{}{t}\ab(-\diffp{\BB}{t}) \\
                                  & = \mu_0\nabla\times\bm{j} - \frac{1}{c^2}\diffp[2]{}{t}\BB               \\
    \nabla^2\BB                   & = \frac{1}{c^2}\diffp[2]{}{t}\BB
  \end{align}
\end{proof}

\begin{theorem}[電磁波の複素数表現]
  真空中に伝搬する電磁波の複素数解は波数 $\kk\in\RR^3$ を用いて次のように表される。
  \begin{align}
    \EE(t, \rr) & = \int_{\RR^3}\dl{\kk}\EE_0(\kk)e^{i(\kk\cdot\rr - \omega(\kk)t)} \\
    \BB(t, \rr) & = \int_{\RR^3}\dl{\kk}\BB_0(\kk)e^{i(\kk\cdot\rr - \omega(\kk)t)}
  \end{align}
  ただし電磁波の分散関係は光速度 $c$ を用いて $\omega(\kk) = c|\kk|$ と与えられ、振動方向 $\EE_0(\kk)\in\CC^2$ は進行方向 $\kk$ と直交する。
  なお、物理的な電磁場はそれらの複素表現の実部を取ることで求められる。$\EE_0$ は複素数であることに注意する。
\end{theorem}
\begin{proof}
  波動方程式に代入して成り立つことを示す。
  \begin{align}
    \nabla^2\EE & = \nabla^2\int_{\RR^3}\dl{\kk}\EE_0(\kk)e^{i(\kk\cdot\rr - \omega(\kk)t)}                      \\
                & = \int_{\RR^3}\dl{\kk}\EE_0(\kk)(-|\kk|^2)e^{i(\kk\cdot\rr - \omega(\kk)t)}                    \\
                & = \frac{1}{c^2}\int_{\RR^3}\dl{\kk}\EE_0(\kk)(-\omega^2(\kk))e^{i(\kk\cdot\rr - \omega(\kk)t)} \\
                & = \frac{1}{c^2}\diffp[2]{}{t}\int_{\RR^3}\dl{\kk}\EE_0(\kk)e^{i(\kk\cdot\rr - \omega(\kk)t)}   \\
                & = \frac{1}{c^2}\diffp[2]{}{t}\EE
  \end{align}
  磁束密度も同様にして示せる。
\end{proof}

電磁波の複素数表示について

\begin{theorem}[電磁波のエネルギー]
  電磁波のエネルギー密度と Poynting ベクトル
  \begin{align}
    \langle u_{em}\rangle & = \frac{1}{2}\varepsilon_0|\EE_0|^2 \\
    \langle\SS\rangle     & = \overline{\varepsilon}c\hat{\kk}
  \end{align}
\end{theorem}
\begin{proof}
  \begin{align}
    \Re{\EE} & = \EE_0\cos(\kk\cdot\rr - \omega t)                            \\
    \Re{\BB} & = \frac{1}{\omega}(\kk\times\EE_0)\cos(\kk\cdot\rr - \omega t)
  \end{align}
  これより電磁場のエネルギー密度は
  \begin{align}
    u_e    & = \frac{\varepsilon_0}{2}|\Re\EE|^2 = \frac{\varepsilon_0}{2}\ab|\EE_0\cos(\kk\cdot\rr - \omega t)|^2 = \frac{\varepsilon_0}{2}\ab|\EE_0|^2\cos^2(\kk\cdot\rr - \omega t)              \\
    u_m    & = \frac{1}{2\mu_0}|\Re\BB|^2 = \frac{1}{2\mu_0}\ab|\frac{1}{\omega}(\kk\times\EE_0)\cos(\kk\cdot\rr - \omega t)|^2 = \frac{\varepsilon_0}{2}\ab|\EE_0|^2\cos^2(\kk\cdot\rr - \omega t) \\
    u_{em} & = \varepsilon_0\ab|\EE_0|^2\cos^2(\kk\cdot\rr - \omega t)
  \end{align}
  時間平均を取ると
  \begin{alignat}{3}
    \langle u_e\rangle & = \frac{1}{4}\varepsilon_0|\EE_0|^2, \qquad \langle u_m\rangle = \frac{1}{4}\varepsilon_0|\EE_0|^2, \qquad \langle u_{em}\rangle = \frac{1}{2}\varepsilon_0|\EE_0|^2
  \end{alignat}
  \begin{align}
    \SS & = \frac{1}{\mu_0}\Re{\EE}\times\Re{\BB} = \frac{1}{2\mu_0}\ab|\frac{1}{\omega}(\kk\times\EE_0)\cos(\kk\cdot\rr - \omega t)|^2
  \end{align}
\end{proof}

\begin{theorem}[電磁波の運動量]
  \begin{align}
    \langle \bm{g}\rangle = \frac{1}{c^2}\langle\SS\rangle = \frac{\varepsilon_0|\EE_0|^2}{2c}\hat{\kk}
  \end{align}
\end{theorem}
\begin{proof}
\end{proof}

\begin{itembox}[l]{電磁波まとめ}
  $\kk$ について波長 $\RR$ と球面のさまざまな方向 $S^2$ での電磁波の和を取ったものが電磁波全体となる。
  \begin{align}
    \EE(t, \rr) & = \int_{\RR^3}\dl{\kk}\EE_0(\kk)e^{i(\kk\cdot\rr - \omega(\kk)t)} \\
    \BB(t, \rr) & = \int_{\RR^3}\dl{\kk}\BB_0(\kk)e^{i(\kk\cdot\rr - \omega(\kk)t)}
  \end{align}
\end{itembox}


\subsection{電磁波の伝搬}
\begin{theorem}[Helmholtz 方程式]
  \begin{align}
    (\laplacian + k^2)\EE & = 0 \\
    (\laplacian + k^2)\BB & = 0
  \end{align}
\end{theorem}


\begin{theorem}[平行導体板]
\end{theorem}

\begin{theorem}[導波管内を伝搬する電磁波]
\end{theorem}

\begin{theorem}[円形断面の導波管内を伝搬する電磁波]
\end{theorem}

\begin{theorem}[直方形型の導波管内を伝搬する電磁波]
\end{theorem}


\subsection{電磁波の回折}

\begin{theorem}[Kirchhoff の積分表示]
\end{theorem}

\begin{theorem}[Fresnel-Kirchhoff の回折積分の公式]
\end{theorem}

\begin{theorem}[Fraunhofer 回折]
\end{theorem}

\begin{theorem}[Fresnel 回折]
\end{theorem}



\section{物質中の電磁気学}
現象論になりがち

\subsection{導体の電磁気学}
\begin{definition}[導体]
  時間が経つと
  \begin{enumerate}
    \item 導体内部に電場は存在しない。
    \item 導体内部に電荷はなく、表面のみに電荷が分布する。
  \end{enumerate}
  等電位面、誘導電荷
  導体全体で電位は一定、
\end{definition}

\begin{theorem}
  \begin{align}
    E = \frac{\sigma}{\varepsilon_0}
  \end{align}
\end{theorem}

\begin{proposition}[一様に分極した誘電体球の分極電場]
\end{proposition}

\begin{proposition}[一様に分極した回転楕円体形状の誘電体の分極電場]
\end{proposition}

\begin{theorem}[静電誘導]

\end{theorem}

\begin{proposition}[半無限導体と点電荷]
  鏡像法
\end{proposition}

\begin{proposition}[一様外部電場中の導体球]
\end{proposition}

\begin{definition}[静電遮蔽]
  導体によって囲まれた空間内の電場は外部の電場に影響されず内部の電荷のみで決まる。
\end{definition}

\begin{theorem}[表面電荷による遮蔽]
\end{theorem}
\begin{proof}
  \begin{align}
    -\vnabla^2\phi = \frac{\rho}{\varepsilon_0}
  \end{align}
  \begin{align}
    n_i
  \end{align}
\end{proof}


\subsection{物質中の電磁気学の一般論}
\begin{definition}[誘電体]
  誘電体は外から電場を作用させると正の電荷と負の電荷は逆向きに変位し、電気的に分極して微視的な電気双極子を作る。これを電気分極 (electric polarization) といい、電気分極によって電場は弱められる。このとき電気分極による電気モーメント密度を $\PP(\rr)$ と表す。このとき電束密度 $\DD(\rr)$ を次のように定義する。
  \begin{align}
    \DD & = \varepsilon_0\EE + \PP
  \end{align}
  誘電体に関しては仮想的な電荷を導入することで真空中の電磁気学を近似的に適用できる。これを分極電荷 (polarization charge) という。電子やイオンに由来する自由に取り出したり、加えたりできる電荷を真電荷という。
\end{definition}

\begin{theorem}[誘電体のガウスの法則]
  誘電体による分極電荷を含むガウスの法則は次のように書ける。
  \begin{align}
    \int_{\partial V}\DD\cdot\bm{n}\dl{S} & = \int_V\rho_e(\rr)\dl{V} \iff \vnabla\cdot\DD = \rho_e
  \end{align}
\end{theorem}
\begin{proof}
  誘電体内部と誘電体表面における分極電荷の電荷密度を $\rho_P(\rr), \sigma_P(\rr)$ とおくと、電気分極 $\bm{P}$ を用いて次のように表される。
  \begin{align}
    \rho_P(\rr)   & = - \vnabla\cdot\PP \\
    \sigma_P(\rr) & = \bm{n}\cdot\PP
  \end{align}
  誘電体の中の巨視的な電場 $\EE$ は外部電場 $\EE_e$ と分極電場 $\EE_P$ からなる。
  \begin{align}
    \varepsilon_0\int_{\partial V}\EE_P\cdot\bm{n}\dl{S}       & = \int_V\rho_P\dl{V} + \int_{S_P}\sigma_P\dl{S}                         \\
                                                               & = \int_V(-\vnabla\cdot\bm{P})\dl{V} - \int_{S_P}\bm{P}\cdot\bm{n}\dl{S} \\
                                                               & = - \int_{\partial V + S_P}\bm{P}\cdot\bm{n}\dl{S}                      \\
    \int_{\partial V}(\varepsilon_0\EE + \PP)\cdot\bm{n}\dl{S} & = \int_{V}\rho_e\dl{V}
  \end{align}
  これより $\DD(\rr) = \varepsilon_0\EE(\rr) + \bm{P}(\rr)$ とおくことで誘電体におけるガウスの法則が求まる。
  \begin{align}
    \int_{\partial V}\DD\cdot\bm{n}\dl{S} & = \int_{V}\rho_e\dl{V} \iff \vnabla\cdot\DD = \rho_e
  \end{align}
\end{proof}

\begin{theorem}[誘電体のエネルギー]
  一般の誘電体において電場が作るエネルギー密度
  \begin{align}
    u_e(\rr) & = \int_0^D\EE\cdot\dl{\DD}
  \end{align}
\end{theorem}
\begin{proof}
  電場を作ってから分極させるようにしてエネルギー差を計算する。$\pp$ の電気双極子について $\Delta W = -\bm{F}\cdot\Delta\rr = q\EE\cdot\Delta\rr = \EE\cdot\Delta\pp$ となるから
  \begin{align}
    \Delta u_E & = \varepsilon_0\EE\cdot\Delta\EE                                                             \\
    \Delta u_P & = \EE\cdot\Delta\PP                                                                          \\
    \Delta u_e & = \Delta u_E + \Delta u_P = \EE\cdot(\varepsilon_0\Delta\EE + \Delta\PP) = \EE\cdot\Delta\DD
  \end{align}
  となる。これより単位体積当たりのエネルギーは
  \begin{align}
    u_e(\rr) & = \int_0^D\EE\cdot\dl{\DD}
  \end{align}
\end{proof}

\begin{theorem}[境界条件]
  誘電体の境界面において磁場がないとき次の境界条件を満たす。
  \begin{alignat}{2}
    D_{1\perp} & = D_{2\perp}, \qquad E_{1\parallel} & = E_{2\parallel}
  \end{alignat}
\end{theorem}
\begin{proof}
  Maxwell の方程式より
  \begin{alignat}{2}
    0 & = \int_S\DD\cdot\bm{n}\dl{S} = D_{1\perp}\Delta S + D_{2\perp}(-\Delta S)  \qquad & D_{1\perp} = D_{2\perp}         \\
    0 & = \int_{\partial S}\EE\cdot\dl{\bm{l}} = E_{1\parallel}l - E_{2\parallel}l \qquad & E_{1\parallel} = E_{2\parallel}
  \end{alignat}
\end{proof}

\begin{definition}[磁性体]
  磁性体は外から磁場を作用させるとスピンや原子核によって微視的な磁気双極子を作り、これを磁化 (magnetization) という。磁化 $\MM(\rr)$ を磁気双極子モーメント密度とし、磁場 $\HH(\rr)$ を次のように定義する。
  \begin{align}
    \BB & = \mu_0\HH + \MM
  \end{align}
  磁性体に関して仮想的な磁荷を導入することで真空中の電磁気学を近似的に適用できる。これを分極磁荷 (polarized magnetic charge) という。
\end{definition}

\begin{theorem}[磁場におけるガウスの法則]
  磁性体による分極磁荷を含むガウスの法則は次のように書ける。
  \begin{align}
    \int_{\partial V}\BB\cdot\bm{n}\dl{S} & = 0 \iff \vnabla\cdot\BB = 0
  \end{align}
\end{theorem}
\begin{proof}
  磁性体内部と磁性体表面における分極磁荷の密度を $\rho_P(\rr), \sigma_P(\rr)$ とおくと、磁化 $\MM$ を用いて次のように表される。
  \begin{align}
    \rho_M(\rr)   & = - \vnabla\cdot\MM \\
    \sigma_M(\rr) & = \bm{n}\cdot\MM
  \end{align}
  磁性体の中の巨視的な磁場 $\HH$ は外部磁場 $\HH_e$ と分極磁場 $\HH_M$ からなる。
  \begin{align}
    \mu_0\int_{\partial V}\HH_M\cdot\bm{n}\dl{S}       & = \int_V\rho_M\dl{V} + \int_{S_M}\sigma_M\dl{S}                   \\
                                                       & = \int_V(-\vnabla\cdot\MM)\dl{V} - \int_{S_M}\MM\cdot\bm{n}\dl{S} \\
                                                       & = -\int_{\partial V + S_M}\MM\cdot\bm{n}\dl{S}                    \\
    \int_{\partial V}(\mu_0\HH + \MM)\cdot\bm{n}\dl{S} & = \mu_0\int_{\partial V}\HH_e\cdot\bm{n}\dl{S} = 0
  \end{align}
  これより $\BB = \mu_0\HH + \MM$ とおくことで
  \begin{align}
    \int_{\partial V}\BB\cdot\bm{n}\dl{S} & = 0 \iff \vnabla\cdot\BB = 0
  \end{align}
\end{proof}

\begin{theorem}[ファラデーの電磁誘導の法則]
  \begin{align}
    \oint_{\partial S}\EE\cdot\dl{\bm{l}} = -\int_S\diffp{\BB}{t}\cdot\dl{\SS} & \iff \vnabla\times\EE = -\diffp{\BB}{t}
  \end{align}
\end{theorem}
\begin{proof}
  \begin{align}
    \int_{\partial S}\EE_P\cdot\dl{\bm{l}} = 0 \\
    \int_{\partial S}\diffp{\BB_M}{t}\cdot\dl{\bm{l}} = 0
  \end{align}
\end{proof}

\begin{theorem}[物質中のマクスウェル・アンペールの法則]
  \begin{align}
    \oint_{\partial S}\HH\cdot\dl{\bm{l}} = \int_S\ab(\bm{j} + \diffp{\DD}{t})\cdot\dl{\SS} \iff \vnabla\times\HH = \bm{j} + \diffp{\DD}{t}
  \end{align}
\end{theorem}
\begin{proof}
  原点 $O$ に $z$ 軸に平行な電気双極子があり,正電荷 $q$ と負電荷 $-q$ の $z$ 座標をそれぞれとする。正電荷と負電荷の間に電流 $I_P$ が流れ、$q$ の値が時間と共に変化するとする。
  \begin{align}
    \HH_M(\rr)                              & = \frac{I_P\bm{d}\times\rr}{4\pi r^3} = \frac{\dot{\pp}\times\rr}{4\pi r^3} \\
    \oint_{\partial S}\HH_M\cdot\dl{\bm{l}} & = \frac{\dot{p}a^2}{2(a^2 + z^2)^{3/2}}                                     \\
    \oint_{\partial S}\HH_M\cdot\dl{\bm{l}} & = \diff{\Psi_P}{t}
  \end{align}
  磁化電流
  \begin{align}
    \diff{\Psi_P}{t} & = \int\varepsilon_0\diffp{\EE}{t}\cdot\dl{\SS}
  \end{align}
  分極電荷の移動による電流 (レントゲン電流) は
  \begin{align}
    I_P & = \int\diffp{\PP}{t}\cdot\dl{\SS}
  \end{align}
  \begin{align}
    \oint_{\partial S}\HH\cdot\dl{\bm{l}} & = \diff{\Psi_P}{t} + I_P                                                        \\
                                          & = \int_S\ab(\varepsilon_0\diffp{\EE}{t} + \diffp{\PP}{t} + \bm{j})\cdot\dl{\SS} \\
                                          & = \int_S\ab(\diffp{\DD}{t} + \bm{j})\cdot\dl{\SS}                               \\
    \vnabla\times\HH                      & = \bm{j} + \diffp{\DD}{t}
  \end{align}
\end{proof}

\begin{theorem}[電荷の保存則]
  \begin{align}
    \diffp{\rho}{t} + \vnabla\cdot\bm{j} = 0
  \end{align}
\end{theorem}
\begin{proof}
  \begin{align}
    \vnabla\cdot(\vnabla\times\HH) & = \vnabla\cdot\bm{j} + \diffp{}{t}\ab(\vnabla\cdot\DD) = \diffp{\rho}{t} + \vnabla\cdot\bm{j} = 0
  \end{align}
\end{proof}

\begin{theorem}[磁性体中のエネルギー]
  \begin{align}
    u_m = \int_0^B\HH\cdot\dl{\BB}
  \end{align}
\end{theorem}
\begin{proof}
  誘電体と同様にして
  \begin{align}
    \Delta u_M & = \HH\cdot(\Delta \MM + \mu_0\Delta\HH) = \HH\cdot\Delta\BB \\
    u_M        & = \int_0^B\HH\cdot\dl{\BB}
  \end{align}
\end{proof}

\begin{theorem}[エネルギー保存則]
  エネルギー密度 $u_{em}$ は次のように表される。
  \begin{align}
    u_{em} & = \int_{0}^{\DD}\EE\cdot\dl{\DD} + \int_{0}^{\BB}\HH\cdot\dl{\BB}
  \end{align}
\end{theorem}
\begin{proof}

\end{proof}

\begin{theorem}[運動量保存則]
  常誘電体と常磁性体においてエネルギー密度を $u_{em}$、マクスウェルの応力テンソル $T_{ij}$ について
  \begin{align}
    u_{em} & = \frac{1}{2}\EE\cdot\DD + \frac{1}{2}\HH\cdot\BB \\
    T_{ij} & = E_iD_j + H_iB_j - \delta_{ij}u_{em}
  \end{align}
  物質の運動量密度を $\pp_m(\rr, t)$、電磁場の運動量密度を $\pp_{EM}(\rr, t)$ とする。
  \begin{align}
    \diffp{\pp_m}{t} & = \rho\EE + \bm{j}\times\BB \\
    \pp_{EM}         & = \DD\times\BB
  \end{align}
  このとき応力テンソル $T_{ij}$ の反作用を電磁場の運動量流密度と考えると
  \begin{align}
    \partial_i(-T_{ij}) + \diffp{p_j}{t} = 0
  \end{align}
  ローレンツ力の反作用が電磁場の運動量の時間変化となる。
\end{theorem}
\begin{proof}
  Maxwell の方程式と $\EE, \DD$ と $\BB, \HH$ の関係式から次のように求まる。
  \begin{align}
    \diffp{p_j}{t} & = \ab[\rho\EE + \bm{j}\times\BB + \diffp{}{t}(\DD\times\BB)]_j                                                                                              \\
                   & = \ab[(\vnabla\cdot\DD)\EE + \ab(\vnabla\times\HH - \diffp{\DD}{t})\times\BB + \diffp{\DD}{t}\times\BB + \DD\times\diffp{\BB}{t}]_j                         \\
                   & = \ab[\varepsilon(\vnabla\cdot\EE)\EE + \mu(\vnabla\times\HH)\times\HH - \varepsilon\EE\times(\vnabla\times\EE)]_j                                          \\
                   & = \varepsilon(\partial_iE_i)E_j + \mu\varepsilon_{jik}\varepsilon_{ilm}(\partial_lH_m)H_k - \varepsilon\varepsilon_{jik}E_i\varepsilon_{klm}(\partial_lE_m) \\
                   & = \varepsilon(\partial_iE_i)E_j + \mu(\partial_iH_j)H_i - \mu(\partial_jH_i)H_i - \varepsilon E_i(\partial_jE_i) + \varepsilon E_i(\partial_iE_j)           \\
                   & = \partial_i\ab(\varepsilon E_iE_j + \mu H_iH_j) - \partial_j\ab(\frac{1}{2}(E_i^2 + H_i^2))                                                                \\
                   & = \partial_iT_{ij}
  \end{align}
  よってこれを積分形に書き直すと Gauss の発散定理を用いて次のようになる。
  \begin{align}
    \oint_{\partial V}\bm{j}(\rr, t)\cdot\bm{n}(\rr)\dl{S} + \partial_t\int_V(\pp_m + \pp_{EM})\dl{v} = 0
  \end{align}
\end{proof}
\begin{itembox}[l]{物質中の Maxwell 方程式}
  $\DD = \varepsilon_0\EE + \PP$ $\BB = \mu_0\HH + \MM$
  \begin{alignat}{3}
     & \int_{\partial V}\DD\cdot\dl{\SS} = \int_V\rho_e\dl{V}                                              &  & \iff \vnabla\cdot\DD  = \rho_e                  \\
     & \int_{\partial V}\BB\cdot\dl{\SS} = 0                                                               &  & \iff \vnabla\cdot\BB  = 0                       \\
     & \int_{\partial S}\EE\cdot\dl{\bm{l}} = -\diff{}{t}\int_S\BB\cdot\dl{\SS}                            &  & \iff \vnabla\times\EE = -\diffp{\BB}{t}         \\
     & \int_{\partial S}\HH\cdot\dl{\bm{l}} = \int_S\bm{j}\cdot\dl{\SS} + \diff{}{t}\int_S\DD\cdot\dl{\SS} &  & \iff \vnabla\times\HH = \bm{j} + \diffp{\DD}{t}
  \end{alignat}
\end{itembox}


\subsection{誘電体の電磁気学}
\begin{definition}[常誘電体]
  電場に比例して電気分極が現れる場合を常誘電相 (paraelectric phase) といい、そのような誘電体を常誘電体という。この電気分極は電気感受率 (electric susceptibility) $\chi$ または誘電体の誘電率 (dielectric constant) という。
  \begin{align}
    \PP & = \varepsilon_0\chi\EE                        \\
    \DD & = \varepsilon_0(1 + \chi)\EE = \varepsilon\EE
  \end{align}
  電場を掛けなくても微視的な電気双極子が一方向に揃った電気分極は大きな値をとる。これを自発分極 (spontaneous polarization) といい、このような物質を強誘電体 (ferroelectrics) という。
\end{definition}
\begin{theorem}[常誘電体のエネルギー密度]
  \begin{align}
    u_e(\rr) & = \frac{1}{2}\EE\cdot\DD = \frac{\varepsilon_0}{2}\EE^2 = \frac{1}{2\varepsilon_0}\DD^2 \\
    U_e      & = \frac{1}{2}\int\phi\rho_e\dl{V}
  \end{align}
\end{theorem}
\begin{proof}
  \begin{align}
    u_e(\rr) & = \int_0^D\EE\cdot\dl{\DD} = \varepsilon\int_0^E\EE\cdot\dl{\EE} = \frac{\varepsilon\EE^2}{2} = \frac{\DD^2}{2\varepsilon} = \frac{1}{2}\EE\cdot\DD
  \end{align}
  電位と電荷密度を用いると次のようになる。
  \begin{align}
    u_e(\rr) & = \frac{1}{2}\EE\cdot\DD = \frac{1}{2}(-\vnabla\phi)\cdot\DD = \frac{1}{2}\ab(-\vnabla\cdot(\phi\DD) + \phi\vnabla\cdot\DD) = \frac{1}{2}\ab(-\vnabla\cdot(\phi\DD) + \phi\rho_e) \\
    U_e      & = \int\frac{1}{2}\ab(-\vnabla\cdot(\phi\DD) + \phi\rho_e)\dl{V} = \frac{1}{2}\int\phi\rho_e\dl{V}
  \end{align}
\end{proof}

\begin{proposition}[中心に点電荷のある誘電体球]
\end{proposition}
\begin{proof}
  半径 $r$ の閉球面 $S$ においてガウスの法則を適用する
  \begin{align}
    \oint_S\DD(\rr)\cdot\dl{\SS} & = 4\pi r^2D(r) = Q   \\
    D(r)                         & = \frac{Q}{4\pi r^2}
  \end{align}
  電束密度 $D(r)$ から電場 $E(r)$, 電位 $\phi(r)$, 電気分極 $P(r)$ を求められる。
  \begin{align}
    D(r)        & =
    \begin{dcases}
      \varepsilon E(r)   & (r < a) \\
      \varepsilon_0 E(r) & (r > a)
    \end{dcases}                                                                                                                                                             \\
    E(r)        & =
    \begin{dcases}
      \frac{Q}{4\pi\varepsilon r^2}   & (r < a) \\
      \frac{Q}{4\pi\varepsilon_0 r^2} & (r > a)
    \end{dcases}                                                                                                                                                \\
    P(r)        & = D(r) - \varepsilon_0E(r) =
    \begin{dcases}
      \frac{\varepsilon - \varepsilon_0}{\varepsilon}\frac{Q}{4\pi r^2} & (r < a) \\
      0                                                                 & (r > a)
    \end{dcases}                                                                                                              \\
    \rho_P(r)   & = -\vnabla\cdot\PP = -\frac{\varepsilon - \varepsilon_0}{\varepsilon}Q\ab(\vnabla\cdot\frac{\rr}{4\pi r^2}) = -\frac{\varepsilon - \varepsilon_0}{\varepsilon}Q\delta(\rr) \\
    \sigma_P(r) & = \bm{n}\cdot\PP = \frac{\varepsilon - \varepsilon_0}{\varepsilon}\frac{Q}{4\pi a^2}                                                                                       \\
    \phi(r)     & = \int_r^\infty E(r)\dl{r} =
    \begin{dcases}
      \frac{Q}{4\pi\varepsilon}\ab(\frac{1}{r} - \frac{1}{a}) + \frac{Q}{4\pi\varepsilon_0 a} & (r < a) \\
      \frac{Q}{4\pi\varepsilon_0 r}                                                           & (r > a)
    \end{dcases}
  \end{align}
  となる。
\end{proof}

\begin{proposition}[導体と誘電体の境界]
  導体表面での真電荷の面密度を $\sigma$ とすると
  $D = \sigma$
\end{proposition}

\begin{proposition}[誘電体境界と鏡像法]
\end{proposition}

\begin{proposition}[真電荷の周囲を誘電体で囲む]
\end{proposition}

\begin{proposition}[コンデンサーの極板]
\end{proposition}

圧電応答力による顕微鏡
\begin{definition}[強誘電体]

\end{definition}
\begin{proof}
  \begin{align}
    F(\PP) & = - \EE\cdot\PP + A(T - T_c)P^2 + BP^4 + CP^6 + \cdots
  \end{align}
\end{proof}


\subsection{磁性体の電磁気学}
\begin{theorem}[磁気現象]
  \begin{align}
    \mathcal{H} & = \frac{1}{2m}(\pp - q\AA)^2 + q\phi + g\frac{q\hbar}{2m}(\bm{s}\cdot\BB)                     \\
    \bm{m}      & = \mu_0\ab(\frac{q\hbar}{2m}\bm{l} - \frac{q^2}{2m}(\rr\times\AA) + \frac{gq\hbar}{2m}\bm{s}) \\
    \mu_B       & = \frac{e\hbar}{2m_e} = 9.274\times 10^{-24} J/T
  \end{align}
  ただし $\bm{s}$ をスピン, $g \approx 2.00232\cdots$ を g 因子
\end{theorem}
\begin{proof}
  \begin{align}
    \langle m_z\rangle & = \frac{\sum_{S_z = -S}^{S}-GS_ze^{\beta GS_zH}}{\sum_{S_z = -S}^{S}e^{\beta GS_zH}}       \\
                       & = \diffp{}{\beta}\ln\ab(\sum_{S_z = -S}^{S}\exp\ab(\frac{S_z}{S}x)) \qquad (x = \beta GSH) \\
                       & = \diffp{}{\beta}\ln\ab(\frac{\sinh\ab(\frac{2S+1}{2S}x)}{\sinh\ab(\frac{1}{2S}x)})        \\
                       & = GS\ab(\frac{2S+1}{2S}\coth\ab(\frac{2S+1}{2S}x) - \frac{1}{2S}\coth\ab(\frac{1}{2S}x))   \\
                       & = GSB_S(x) \approx \frac{G^2S(S + 1)}{3k_BT}H \qquad (x\to 0)
  \end{align}
  \begin{align}
    \mathcal{H} = -2J\SS\cdot\SS
  \end{align}
\end{proof}
\begin{definition}[強磁性体]
  磁化現象において量子力学的効果である同じ向きのスピンを持つ 2 つの電子は同一の場所には存在出来ない性質が無視できない。2つの電子完のスピン相互作用はスピンが平行と反平行の場合に $J$ 異なる交換相互作用強磁性体
\end{definition}
\begin{proof}
  \begin{align}
    F(\MM) & = -\MM\cdot\HH_e + A(T - T_c)M^2 + BM^4
  \end{align}
  $T > T_c$ のとき
  \begin{align}
    M & = \frac{H_e}{2A(T - T_c)}
  \end{align}
\end{proof}


\subsection{超伝導体の電磁気学}
\begin{definition}
  Cooper 対
  \begin{enumerate}
    \item 電気抵抗が0
    \item 超伝導体内の磁束密度は常に 0 (完全反磁性)
    \item マイスナー (Meissner) 効果
  \end{enumerate}
  内部の磁束が 0 になるように超伝導体表面には磁束を遮蔽する磁化電流 (遮蔽電流) が流れる。遮蔽電流は表面近くの領域に分布しているので、磁束は表面よりある距離 ($\lambda$) 程度侵入できる。この $\lambda$ を磁束の侵入距離という。
  遮蔽電流密度 $\bm{j}_S$ はベクトルポテンシャル $\AA$ との間にロンドン方程式と呼ばれる関係がある。
  \begin{align}
    \bm{j}_S & = - \frac{n_sq^2}{m}\AA \qquad \ab(q = -2e, m = 2m_e)
  \end{align}
\end{definition}

\begin{theorem}
\end{theorem}
\begin{proof}
  \begin{align}
    \vnabla\times\bm{j}_S           & = - \frac{n_sq^2}{m}\vnabla\times\AA = - \frac{n_sq^2}{m}\BB                                                          \\
    \vnabla\times(\vnabla\times\BB) & = \vnabla(\vnabla\cdot\BB) - \laplacian\BB = - \laplacian\BB                                                          \\
                                    & = \vnabla\times(\mu_0\bm{j}_S) = - \frac{\mu_0n_sq^2}{m}\BB                                                           \\
    \laplacian\BB                   & = \frac{\mu_0n_sq^2}{m}\BB                                                                                            \\
    \BB(x)                          & = \BB(0)\exp\ab(-\frac{x}{\lambda}) \qquad \ab(\lambda = \sqrt{\frac{m}{\mu_0n_sq^2}} \approx 10^{-8} \sim 10^{-7} m)
  \end{align}
  より
\end{proof}



\section{特殊相対論}

\end{document}