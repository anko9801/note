\RequirePackage{plautopatch}
\documentclass[uplatex,dvipdfmx,a4paper,11pt]{jlreq}
\usepackage{bxpapersize}
\usepackage[utf8]{inputenc}
\usepackage{fontenc}
\usepackage{lmodern}
\usepackage{otf}
\usepackage{amsmath}
\usepackage{amssymb}
\usepackage{amsthm}
\usepackage{ascmac}
% \usepackage[hyphens]{url}
\usepackage{physics}
\usepackage{braket}
\usepackage{verbatimbox}
\usepackage{bm}
\usepackage{url}
% \usepackage[dvipdfmx,hiresbb,final]{graphicx}
\usepackage{hyperref}
\usepackage{pxjahyper}
\usepackage{tikz}\usetikzlibrary{cd}
\usepackage{listings}
\usepackage{color}
\usepackage{mathtools}
\usepackage{xspace}
\usepackage{xy}
\usepackage{xypic}
%
\title{電磁気学}
\author{Anko}
\makeatletter
%
\DeclareMathOperator{\lcm}{lcm}
\DeclareMathOperator{\Kernel}{Ker}
\DeclareMathOperator{\Image}{Im}
\DeclareMathOperator{\ch}{ch}
\DeclareMathOperator{\Aut}{Aut}
\DeclareMathOperator{\Log}{Log}
\DeclareMathOperator{\Arg}{Arg}
\DeclareMathOperator{\sgn}{sgn}
%
\newcommand{\CC}{\mathbb{C}}
\newcommand{\RR}{\mathbb{R}}
\newcommand{\QQ}{\mathbb{Q}}
\newcommand{\ZZ}{\mathbb{Z}}
\newcommand{\NN}{\mathbb{N}}
\newcommand{\FF}{\mathbb{F}}
\newcommand{\PP}{\mathbb{P}}
\newcommand{\GG}{\mathbb{G}}
\newcommand{\TT}{\mathbb{T}}
\newcommand{\calB}{\mathcal{B}}
\newcommand{\calF}{\mathcal{F}}
\newcommand{\ignore}[1]{}
\newcommand{\floor}[1]{\left\lfloor #1 \right\rfloor}
% \newcommand{\abs}[1]{\left\lvert #1 \right\rvert}
\newcommand{\lt}{<}
\newcommand{\gt}{>}
\newcommand{\id}{\mathrm{id}}
\newcommand{\rot}{\curl}
\renewcommand{\angle}[1]{\left\langle #1 \right\rangle}
\newcommand{\EE}{\bm{E}}
\newcommand{\BB}{\bm{B}}
\renewcommand{\AA}{\bm{A}}
\newcommand{\rr}{\bm{r}}
\newcommand{\kk}{\bm{k}}
\newcommand{\pp}{\bm{p}}

\let\oldcite=\cite
\renewcommand\cite[1]{\hyperlink{#1}{\oldcite{#1}}}

\let\oldbibitem=\bibitem
\renewcommand{\bibitem}[2][]{\label{#2}\oldbibitem[#1]{#2}}

% theorem環境の設定
% - 冒頭に改行
% - 末尾にdiamond (amsthm)
\theoremstyle{definition}
\newcommand*{\newscreentheoremx}[2]{
  \newenvironment{#1}[1][]{
    \begin{screen}
    \begin{#2}[##1]
      \leavevmode
      \newline
  }{
    \end{#2}
    \end{screen}
  }
}
\newcommand*{\newqedtheoremx}[2]{
  \newenvironment{#1}[1][]{
    \begin{#2}[##1]
      \leavevmode
      \newline
      \renewcommand{\qedsymbol}{\(\diamond\)}
      \pushQED{\qed}
  }{
      \qedhere
      \popQED
    \end{#2}
  }
}
\newtheorem{theorem*}{定理}

\newqedtheoremx{theorem}{theorem*}
\newcommand*\newqedtheorem@unstarred[2]{%
  \newtheorem{#1*}[theorem*]{#2}
  \newqedtheoremx{#1}{#1*}
}
\newcommand*\newqedtheorem@starred[2]{%
  \newtheorem*{#1*}{#2}
  \newqedtheoremx{#1}{#1*}
}
\newcommand*{\newqedtheorem}{\@ifstar{\newqedtheorem@starred}{\newqedtheorem@unstarred}}

\newtheorem{sctheorem*}{定理}
\newscreentheoremx{sctheorem}{sctheorem*}
\newcommand*\newscreentheorem@unstarred[2]{%
  \newtheorem{#1*}[theorem*]{#2}
  \newscreentheoremx{#1}{#1*}
}
\newcommand*\newscreentheorem@starred[2]{%
  \newtheorem*{#1*}{#2}
  \newscreentheoremx{#1}{#1*}
}
\newcommand*{\newscreentheorem}{\@ifstar{\newscreentheorem@starred}{\newscreentheorem@unstarred}}

%\newtheorem*{definition}{定義}
%\newtheorem{theorem}{定理}
%\newtheorem{proposition}[theorem]{命題}
%\newtheorem{lemma}[theorem]{補題}
%\newtheorem{corollary}[theorem]{系}

\newqedtheorem{lemma}{補題}
\newqedtheorem{corollary}{系}
\newqedtheorem{example}{例}
\newqedtheorem{proposition}{命題}
\newqedtheorem{remark}{注意}
\newqedtheorem{thesis}{主張}
\newqedtheorem{notation}{記法}
\newqedtheorem{problem}{問題}
\newqedtheorem{algorithm}{アルゴリズム}

\newscreentheorem*{definition}{定義}

\renewenvironment{proof}[1][\proofname]{\par
  \normalfont
  \topsep6\p@\@plus6\p@ \trivlist
  \item[\hskip\labelsep{\bfseries #1}\@addpunct{\bfseries}]\ignorespaces\quad\par
}{%
  \qed\endtrivlist\@endpefalse
}
\renewcommand\proofname{証明}

\makeatother

\begin{document}
\maketitle

\section{電磁気学}
\subsection{真空中の電磁気学}
\begin{definition}[Maxwell の方程式]
  電場 $\EE$ と磁束密度 $\BB$ に対して次のような式が成り立つ。
  \begin{align}
     & \int_{\partial V}\EE\vdot\bm{n}\dd{S} = \frac{1}{\epsilon_0}\int_V\rho\dd{V}                                                   &  & \iff \vnabla\vdot\EE  = \frac{\rho}{\epsilon_0}                 \\
     & \int_{\partial S}\EE\vdot\dd{\bm{l}} = -\dv{t}\int_S\BB\vdot\bm{n}\dd{S}                                                       &  & \iff \vnabla\vdot\BB  = 0                                       \\
     & \int_{\partial V}\BB\vdot\bm{n}\dd{S} = 0                                                                                      &  & \iff \vnabla\cross\EE = -\pdv{\BB}{t}                           \\
     & c^2\int_{\partial S}\BB\vdot\dd{\bm{l}} = \frac{1}{\epsilon_0}\int_S\bm{j}\vdot\bm{n}\dd{S} + \dv{t}\int_S\EE\vdot\bm{n}\dd{S} &  & \iff \vnabla\cross\BB = \mu_0\bm{j} + \frac{1}{c^2}\pdv{\EE}{t}
  \end{align}
  ただし電荷密度 $\rho(t, \rr) = qn$ と電流密度 $\bm{j}(t, \rr) = qn\bm{v}$ とする。
  ローレンツ力
  \begin{align}
    \bm{F} = q(\EE + \bm{v}\cross\BB)
  \end{align}
\end{definition}
これらの法則で電磁気学が完結する。

\subsection{ポテンシャル}
\begin{theorem}[電位とベクトルポテンシャル]
  次を満たす $\phi$, $\AA$ が存在し、
  \begin{align}
    \EE & = - \vnabla\phi - \pdv{\AA}{t} \\
    \BB & = \vnabla\cross\AA
  \end{align}
\end{theorem}
\begin{proof}
\end{proof}

\begin{theorem}[ゲージ変換]
  任意の関数 $\chi(\rr, t)$ として次のゲージ変換は不変に保つ。
  \begin{align}
    \AA  & \to \AA + \vnabla\chi    \\
    \phi & \to \phi - \pdv{\chi}{t}
  \end{align}
  ゲージ条件
\end{theorem}
\begin{proof}
  \begin{align}
    \vnabla\vdot\BB = 0
  \end{align}
  より

\end{proof}

\begin{proposition}
  静電場においてクーロンゲージ条件を満たすとき
  \begin{align}
    \phi(\rr) & = \frac{1}{4\pi\epsilon_0}\int_V\frac{\rho(\rr')}{|\rr - \rr'|}\dd{\rr'} \\
    \AA(\rr)  & = \frac{\mu_0}{4\pi}\int_V\frac{\bm{j}(\rr')}{|\rr - \rr'|}\dd{\rr'}
  \end{align}
\end{proposition}

\begin{theorem}
  静電エネルギー
  \begin{align}
    U_e = \frac{1}{2}
  \end{align}
\end{theorem}

\subsection{電磁波}
\begin{proposition}
  $\rho = 0$ $\bm{j} = \bm{0}$ において $\EE, \BB$ は波動方程式を満たす。
\end{proposition}
\begin{align}
  \nabla^2\EE = \frac{1}{c^2}\pdv[2]{t}\EE \\
  \nabla^2\BB = \frac{1}{c^2}\pdv[2]{t}\BB
\end{align}
\begin{align}
  \EE(t, \rr) & = \EE_0e^{i(\kk\vdot\rr - \omega t)} \\
  \BB(t, \rr) & = \BB_0e^{i(\kk\vdot\rr - \omega t)}
\end{align}

\subsection{導体}
\begin{definition}[導体]
  時間が経つと
  \begin{enumerate}
    \item 導体内部に電場は存在しない。
    \item 導体内部に電荷はなく、表面のみに電荷が分布する。
  \end{enumerate}
\end{definition}
導体全体で電位は一定、

\end{document}