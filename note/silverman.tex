\documentclass[a4paper,dvipdfmx]{jsarticle}

\usepackage{okumacro}
%ルビ用%
\usepackage{indentfirst}
%字下げを保存するための設定 \parでインデント+改行%
\usepackage[dvipdfmx]{graphicx}
%画像挿入パッケージ。graphix=Windows,graphics=Mac%
\usepackage{wrapfig}
%文章を図に回り込ませるパッケージ%
\usepackage{amsfonts}
\usepackage{amssymb}
%数式色々%
\usepackage{bm}
%ベクトル%
\usepackage{url}
%url中の_や\にエラーをはかせないためのパッケージ%
\usepackage{comment}
%複数行コメントのためのパッケージ%
\usepackage{listings}
%コードのためのパッケージ(英語のみ)%
\usepackage{physics}
%物理関係のパッケージ%
\usepackage{amsmath}
\usepackage{stmaryrd}
\usepackage{amsthm}
%数学関係のパッケージ%
\usepackage{docmute}
%ファイル分割%
\lstset{
  language=C++,
  breaklines=true,
  keywordstyle = {\color[rgb]{0,0,1}},
  stringstyle = {\color[rgb]{1,0,0}},
  commentstyle = { \color[rgb]{0,1,0}},
  numbers=left,
  frame=lines
}
%各種設定%
\usepackage{color}
%色付け 使うときは\documentclass[dvipdfmx]を追加すること!%
\usepackage{ascmac}
\usepackage{otf}
%ギリシャ数字%
\usepackage{siunitx}
%SI単位系%
\usepackage{tikz}
%tikz%
\usetikzlibrary{intersections, calc, arrows, positioning, arrows.meta,automata}
%tikzlibrary%
\renewcommand{\Re}{\real}
\newcommand{\LR}{\Leftrightarrow}
%amsthm%
\theoremstyle{definition}
\newtheorem{dfn}{Definition}[section]
\newtheorem{prop}[dfn]{Proposition}
\newtheorem{lem}[dfn]{Lemma}
\newtheorem{thm}[dfn]{Theorem}
\newtheorem{cor}[dfn]{Corollary}
\newtheorem{rem}[dfn]{Remark}
\newtheorem{fact}[dfn]{Fact}
\renewcommand{\qedsymbol}{$\blacksquare$}
\DeclareMathOperator{\ch}{char}
\begin{document}
\title{}
\author{
  anko
}
\maketitle

\section{代数多様体}
$K$ 完全体 \\
$\overline{K}$ $K$ の代数閉包 \\
$G_{\overline{K}/K}$ $\overline{K}/K$ のガロア群 \\
\subsection{代数幾何学の基本}
\begin{dfn}
  環 $A$ について $A$ 係数あるいは $A$ 上の形式的べき級数 (formal power series) とは, 形式的無限和
  $$
    f(t) = a_0 + a_1t + a_2t^2 + \cdots + a_nt^n + \cdots \quad (a_0,a_1,a_2,\ldots,a_n,\ldots\in A)
  $$
  のことである. $A$ 係数の形式的べき級数全体の集合を $A\llbracket t\rrbracket$ と書く.
\end{dfn}

\begin{prop}
  $A$ を環としたとき, $f(t)\in A\llbracket x\rrbracket$ の定数項が $A$ の単元であることと $f(t)$ が $A\llbracket t\rrbracket$ の単元であることは同値である. 特に $A$ が体のとき定数項が $0$ でなければ $f(t)$ は単元である.
\end{prop}
\begin{proof}
  $g(t) = b_0 + b_1t +\cdots+ b_nt^n +\cdots\in A\llbracket t\rrbracket$ が $f(t)g(t) = 1$ を満たすとき $b_n$ の条件を書き下ろすと,
  \begin{align}
     & a_0b_0                                              = 1 \\
     & a_1b_0 + a_0b_1                                     = 0 \\
     & \vdots                                                  \\
     & a_nb_0 + a_{n-1}b_1 + \cdots + a_1b_{n-1} + a_0b_n  = 0 \\
     & \vdots
  \end{align}
  である. 1つ目の式から $f(t)$ が単元ならば $a_0$ が単元であることはわかる. 逆に $a_0$ が単元であれば $b_0 = a_0^{-1}$ とし, 帰納的に
  $$
    b_n = -a_0^{-1}(a_nb_0 + a_{n-1}b_1 + \cdots + a_1b_{n-1})
  $$
  を定めることができる. したがって $f(t)$ は単元である.
\end{proof}

\begin{prop}
  単元 $b_0\in A$ を用いて $a_0 = b_0^2$ と表されており, さらに $\frac{1}{2}\in A$ であれば $\sqrt{f(t)}\in\mathbb{A}\llbracket t\rrbracket$ である.
\end{prop}
\begin{proof}
  $g(t) = b_0 + b_1t +\cdots+ b_nt^n +\cdots\in A\llbracket t\rrbracket$ が $g(t)^2 = f(t)$ を満たすとき $a_0 = b_0^2$ とし, $a_n$ は $b_n$ を用いて次のように表される.
  \begin{align}
    a_n & = b_0b_n + b_1b_{n-1} + \cdots + b_nb_0 = 2b_0b_n + (b_1b_{n-1} + \cdots + b_{n-1}b_1) \\
    b_n & = \frac{1}{2b_0}(a_n - (b_1b_{n-1} + \cdots + b_{n-1}b_1))
  \end{align}
  これより帰納的に $b_n$ を定められる.
\end{proof}

\begin{dfn}
  $f(t)\in A\llbracket t\rrbracket$ に対して $a_n \neq 0$ となる最小の $n$ を $f(t)$ の位数 (order) と呼び, $\mathrm{ord}(f(t))$ と表す.
\end{dfn}

\begin{thm}
  $k$ を体とするとき, $k\llbracket t\rrbracket$ のイデアル $I$ は整数 $n > 0$ を用いて $(t^n)$ と表される. 特に $k\llbracket t\rrbracket$ は単項イデアル整域であり, $(t)$ は唯一の極大イデアルである.
\end{thm}
\begin{proof}
  $0$ でないイデアル $I\subset k\llbracket t\rrbracket$ に対して, 非負整数 $n$ を
  $$
    n = \min\lbrace\mathrm{ord}(f(t))\mid f(t)\in I\rbrace
  $$

\end{proof}

\subsection{アフィン多様体}

\begin{dfn}
  $n$ 次元アフィン空間は
  $$
    \mathbb{A}^n = \mathbb{A}^n(\overline{K}) = \lbrace P = (x_1,\cdots,x_n):x_i\in\overline{K}\rbrace
  $$
  同様に $\mathbb{A}^n$ の $K$ 有理点の集合を
  $$
    \mathbb{A}^n(K) = \lbrace P = (x_1,\cdots,x_n)\in\mathbb{A}^n:x_i\in K\rbrace
  $$
  ガロア群 $\sigma\in G_{\overline{K}/K}$ を $P\in\mathbb{A}^n$ に作用することを
  $$
    P^\sigma = (x_1^\sigma,\cdots,x_n^\sigma)
  $$
  と定義する.
\end{dfn}

\begin{thm}[ヒルベルト基底定理]
  環 $R$ に対して $R$ がネーター環 $\iff$ $R[X]$ がネーター環が成り立つ.
  すべてのイデアルは有限生成
\end{thm}

\subsection{特異}
Weierstrass 方程式
\begin{align}
  E: y^2 + a_1xy + a_3y = x^3 + a_2x^2 + a_4x + a_6
\end{align}
$\ch(\overline{K}) \neq 2$ のとき Weierstrass 方程式を簡約化できます。
\begin{align}
  (x, y)\mapsto\qty(x, \frac{1}{2}(y - a_1x - a_3))
\end{align}
上のように置換すると次のようになります。
\begin{align}
  \qty(\frac{1}{2}(y - a_1x - a_3))^2 & + a_1x\qty(\frac{1}{2}(y - a_1x - a_3)) + a_3 = x^3 + a_2x^2 + a_4x + a_6 \\
  y^2                                 & = 4x^3 + (a_1^2 + 4a_2)x^2 + (2a_1a_3 + 4a_4)x + (4a_6 + a_3^2)           \\
  y^2                                 & = 4x^3 + b_2x^2 + 2b_4x + b_6
\end{align}
さらに $\ch(\overline{K}) \neq 2, 3$ のときより簡約化できます。
\begin{align}
  (x, y)\mapsto\qty(\frac{x - 3b_2}{36}, \frac{y}{108})
\end{align}
上のように置換すると次のようになります。
\begin{align}
  \qty(\frac{y}{108})^2 & = 4\qty(\frac{x - 3b_2}{36})^3 + b_2\qty(\frac{x - 3b_2}{36})^2 + 2b_4\qty(\frac{x - 3b_2}{36}) + b_6 \\
  y^2                   & = (x - 3b_2)^3 + 9b_2(x - 3b_2)^2 + 648b_4(x - 3b_2) + 108^2b_6                                       \\
  y^2                   & = (x^3 - 9b_2x^2 + 27b_2^2x - 27b_2^3) + 9b_2(x^2 - 6b_2x + 9b_2^2) + 648b_4(x - 3b_2) + 108^2b_6     \\
  y^2                   & = x^3 + (-27b_2^2 + 648b_4)x + (54b_2^3 - 3\cdot 648b_2b_4 + 108^2b_6)                                \\
  y^2                   & = x^3 - 27(b_2^2 - 24b_4)x - 54(- b_2^3 + 36b_2b_4 - 216b_6)                                          \\
  y^2                   & = x^3 - 27c_4x - 54c_6
\end{align}
こうして Weierstrass 方程式は標数に応じて次のように書き表されます。
\begin{align}
  E:\begin{cases}
      y^2 + a_1xy + a_3y = x^3 + a_2x^2 + a_4x + a_6                    \\
      y^2 = 4x^3 + b_2x^2 + 2b_4x + b_6 & (\ch(\overline{K}) \neq 2)    \\
      y^2 = x^3 - 27c_4x - 54c_6        & (\ch(\overline{K}) \neq 2, 3)
    \end{cases}
\end{align}
\subsection{ヤコビアン}

\end{document}