\documentclass[12pt,a4j,draft]{ltjsarticle}
\input{preamble.tex}
\usepackage[margin=5mm]{geometry}
\setlength\parindent{0pt}
\renewcommand{\baselinestretch}{0.78}
\renewcommand{\EE}{\bm{\mathcal{E}}}

\begin{document}
\everymath{\displaystyle}
\textbf{電子}\\
群速度 $\bm{v}_e = \diffp{\omega}{\kk}$,
エネルギー $\EE(\kk) = \hbar\omega(\kk)$,
外力 $\diff{\kk}{t} = \frac{\bm{F}}{\hbar}$, $\kk(t) = \frac{\bm{F}}{\hbar}t + \kk(0)$
有効質量 $\diff{\bm{v}_e}{t} = \diff{}{t}\ab(\diffp{}{k}\frac{\EE}{\hbar}) = \frac{1}{\hbar}\diffp{\EE}{k_i,k_j}\diff{k}{t} = \frac{1}{\hbar^2}\diffp{\EE}{k_i,k_j}\bm{F} = \ab(\frac{\bm{F}}{m^*})_{ij}$
$\EE$-$k$ グラフで曲率高い方が軽くて速い。
3 p D

\end{document}