\RequirePackage{plautopatch}
\documentclass[uplatex,dvipdfmx,a4paper,11pt]{jlreq}
\usepackage{bxpapersize}
\usepackage[utf8]{inputenc}
\usepackage{fontenc}
\usepackage{lmodern}
\usepackage{otf}
\usepackage{amsmath}
\usepackage{amssymb}
\usepackage{amsthm}
\usepackage{ascmac}
% \usepackage[hyphens]{url}
\usepackage{physics2}
\usephysicsmodule{ab, ab.braket, doubleprod, diagmat, xmat}
\usepackage{diffcoeff}
\usepackage{verbatimbox}
\usepackage{bm}
\usepackage{url}
% \usepackage[dvipdfmx,hiresbb,final]{graphicx}
\usepackage{hyperref}
\usepackage{pxjahyper}
\usepackage{tikz}\usetikzlibrary{cd}
\usepackage{listings}
\usepackage{color}
\usepackage{mathtools}
\usepackage{xspace}
\usepackage{xy}
\usepackage{xypic}
%
\title{相対性理論}
\author{Anko}
\makeatletter
%
\DeclareMathOperator{\lcm}{lcm}
\DeclareMathOperator{\Kernel}{Ker}
\DeclareMathOperator{\Image}{Im}
\DeclareMathOperator{\ch}{ch}
\DeclareMathOperator{\Aut}{Aut}
\DeclareMathOperator{\Log}{Log}
\DeclareMathOperator{\Arg}{Arg}
\DeclareMathOperator{\sgn}{sgn}
%
\newcommand{\CC}{\mathbb{C}}
\newcommand{\RR}{\mathbb{R}}
\newcommand{\QQ}{\mathbb{Q}}
\newcommand{\ZZ}{\mathbb{Z}}
\newcommand{\NN}{\mathbb{N}}
\newcommand{\FF}{\mathbb{F}}
\newcommand{\PP}{\mathbb{P}}
\newcommand{\GG}{\mathbb{G}}
\newcommand{\TT}{\mathbb{T}}
\newcommand{\calB}{\mathcal{B}}
\newcommand{\calF}{\mathcal{F}}
\newcommand{\ignore}[1]{}
\newcommand{\floor}[1]{\left\lfloor #1 \right\rfloor}
% \newcommand{\abs}[1]{\left\lvert #1 \right\rvert}
\newcommand{\lt}{<}
\newcommand{\gt}{>}
\newcommand{\id}{\mathrm{id}}
\newcommand{\rot}{\curl}
\renewcommand{\angle}[1]{\left\langle #1 \right\rangle}
\renewcommand{\AA}{\bm{A}}
\newcommand{\BB}{\bm{B}}
\newcommand{\ee}{\bm{e}}
\newcommand{\kk}{\bm{k}}
\newcommand{\pp}{\bm{p}}
\newcommand{\rr}{\bm{r}}

\let\oldcite=\cite
\renewcommand\cite[1]{\hyperlink{#1}{\oldcite{#1}}}

\let\oldbibitem=\bibitem
\renewcommand{\bibitem}[2][]{\label{#2}\oldbibitem[#1]{#2}}

% theorem環境の設定
% - 冒頭に改行
% - 末尾にdiamond (amsthm)
\theoremstyle{definition}
\newcommand*{\newscreentheoremx}[2]{
  \newenvironment{#1}[1][]{
    \begin{screen}
    \begin{#2}[##1]
      \leavevmode
      \newline
  }{
    \end{#2}
    \end{screen}
  }
}
\newcommand*{\newqedtheoremx}[2]{
  \newenvironment{#1}[1][]{
    \begin{#2}[##1]
      \leavevmode
      \newline
      \renewcommand{\qedsymbol}{\(\diamond\)}
      \pushQED{\qed}
  }{
      \qedhere
      \popQED
    \end{#2}
  }
}
\newtheorem{theorem*}{定理}

\newqedtheoremx{theorem}{theorem*}
\newcommand*\newqedtheorem@unstarred[2]{%
  \newtheorem{#1*}[theorem*]{#2}
  \newqedtheoremx{#1}{#1*}
}
\newcommand*\newqedtheorem@starred[2]{%
  \newtheorem*{#1*}{#2}
  \newqedtheoremx{#1}{#1*}
}
\newcommand*{\newqedtheorem}{\@ifstar{\newqedtheorem@starred}{\newqedtheorem@unstarred}}

\newtheorem{sctheorem*}{定理}
\newscreentheoremx{sctheorem}{sctheorem*}
\newcommand*\newscreentheorem@unstarred[2]{%
  \newtheorem{#1*}[theorem*]{#2}
  \newscreentheoremx{#1}{#1*}
}
\newcommand*\newscreentheorem@starred[2]{%
  \newtheorem*{#1*}{#2}
  \newscreentheoremx{#1}{#1*}
}
\newcommand*{\newscreentheorem}{\@ifstar{\newscreentheorem@starred}{\newscreentheorem@unstarred}}

%\newtheorem*{definition}{定義}
%\newtheorem{theorem}{定理}
%\newtheorem{proposition}[theorem]{命題}
%\newtheorem{lemma}[theorem]{補題}
%\newtheorem{corollary}[theorem]{系}

\newqedtheorem{lemma}{補題}
\newqedtheorem{corollary}{系}
\newqedtheorem{example}{例}
\newqedtheorem{proposition}{命題}
\newqedtheorem{remark}{注意}
\newqedtheorem{thesis}{主張}
\newqedtheorem{notation}{記法}
\newqedtheorem{problem}{問題}
\newqedtheorem{algorithm}{アルゴリズム}

\newscreentheorem*{axiom}{公理}
\newscreentheorem*{definition}{定義}

\renewenvironment{proof}[1][\proofname]{\par
  \normalfont
  \topsep6\p@\@plus6\p@ \trivlist
  \item[\hskip\labelsep{\bfseries #1}\@addpunct{\bfseries}]\ignorespaces\quad\par
}{%
  \qed\endtrivlist\@endpefalse
}
\renewcommand\proofname{証明}

\makeatother

\begin{document}
\maketitle
\tableofcontents
\clearpage

\section{特殊相対性理論}
\begin{definition}[Einstein の相対性原理]
  自然法則は全ての慣性系において同じ形になる。
\end{definition}
\begin{definition}[光速度不変の原理]
  光の速度は全ての慣性系で、光源の速度によらず一定である。
\end{definition}

\begin{definition}[共変ベクトル・反変ベクトル]
  計量テンソル $g_{\mu\nu}$ とその逆行列 $g^{\mu\nu}$
  \begin{align}
    g_{\mu\nu} = g^{\mu\nu} = \begin{pmatrix}
                                1 & 0 & 0 & 0 \\ 0 & -1 & 0 & 0 \\ 0 & 0 & -1 & 0 \\ 0 & 0 & 0 & -1
                              \end{pmatrix}
  \end{align}
  時刻 $t\in\RR$ と場所 $\bm{x} = (x^1, x^2, x^3)\in\RR^3$ をまとめた時空点 $x = (x^\mu)$ を次の反変ベクトルで表す。
  \begin{align}
    x^\mu & = (x^0, x^1, x^2, x^3) := (ct, \rr(t))
  \end{align}
  $x$ による微分を共変ベクトルとして次のように定義する。
  \begin{align}
    \partial_\mu & = (\partial_0, \partial_1, \partial_2, \partial_3) := \ab(\frac{1}{c}\diffp{}{t}, \nabla)
  \end{align}
\end{definition}

\begin{proposition}
  $g < 0$
\end{proposition}
\begin{proof}
  \begin{align}
    g_{\mu\nu} & = M\eta_{\mu\nu}M^\top                          \\
    g          & = \det(\eta_{\mu\nu})\det(M)^2 = -\det(M)^2 < 0
  \end{align}
\end{proof}

\begin{definition}[世界間隔]
  \begin{align}
    \dl{s}^2 & = (c\dl{t})^2 - \dl{x}^2 - \dl{y}^2 - \dl{z}^2 = (c\dl{\tau})^2
  \end{align}
  これらを元に四元速度、四元加速度、四元運動量
  \begin{align}
    u^\mu(t) & = \diff{x^\mu}{\tau} = \gamma(c, \dot{\rr})     \\
    a^\mu(t) & = \diff[2]{x^\mu}{\tau} = \gamma(0, \ddot{\rr}) \\
    p^\mu    & = mu^\mu = m\gamma(c, \dot{\rr})                \\
    j^\mu(x) & := (c\rho, \bm{j})                              \\
  \end{align}
\end{definition}

運動方程式
\begin{align}
  m\diff[2]{x^\mu}{\tau} = 0
\end{align}

作用
\begin{align}
  \delta S & = \int \varepsilon^\mu\delta x_\mu\dl{\tau} = 0
\end{align}

\section{一般相対性理論}
\begin{axiom}[アインシュタインの等価原理]
  加速系と重力場の系は局所的には原理的に区別できない。
\end{axiom}
例えば宇宙人によって部屋に閉じ込められたとき地球と同じ重力があるからといって地球にいるとは限らない。

\subsection{座標系}
\begin{definition}[デカルト座標]
  \begin{align}
    \ee_x\cdot\ee_x & = \ee_y\cdot\ee_y = 1, \qquad \ee_x\cdot\ee_y = 0
  \end{align}
\end{definition}

\begin{definition}[極座標]
  ユークリッド平面においてデカルト座標 $(x, y)$ から極座標 $(r, \theta)$ への変換は次のように定義する。
  \begin{align}
    r = \sqrt{x^2 + y^2}, \qquad \theta = \arctan\frac{y}{x}
  \end{align}
  逆に極座標 $(r, \theta)$ からデカルト座標 $(x, y)$ へは次のようになる。
  \begin{align}
    x = r\cos\theta, \qquad y = r\sin\theta
  \end{align}
\end{definition}

\begin{align}
  \ee_r      & = \diffp{x}{r}\ee_x + \diffp{y}{r}\ee_y = \cos\theta\ee_x + \sin\theta\ee_y              \\
  \ee_\theta & = \diffp{x}{\theta}\ee_x + \diffp{y}{\theta}\ee_y = -r\sin\theta\ee_x + r\cos\theta\ee_y
\end{align}
\begin{align}
  \ee_r\cdot\ee_r           & = (\cos\theta\ee_x + \sin\theta\ee_y)\cdot(\cos\theta\ee_x + \sin\theta\ee_y)       \\
                            & = \cos^2\theta + \sin^2\theta = 1                                                   \\
  \ee_\theta\cdot\ee_\theta & = (-r\sin\theta\ee_x + r\cos\theta\ee_y)\cdot(-r\sin\theta\ee_x + r\cos\theta\ee_y) \\
                            & = r^2\sin^2\theta + r^2\cos^2\theta = r^2                                           \\
  \ee_r\cdot\ee_\theta      & = (\cos\theta\ee_x + \sin\theta\ee_y)\cdot(-r\sin\theta\ee_x + r\cos\theta\ee_y)    \\
                            & = -r\sin\theta\cos\theta + r\sin\theta\cos\theta = 0
\end{align}


\subsection{様々な座標系における演算}
どのような座標系においても同じ形の式を作る。
異なる座標系での内積
\begin{definition}[座標系の基底ベクトル]
  座標変換によって基底ベクトルによる
  \begin{align}
    \ee_{\alpha'} & = \Lambda_{\ \alpha'}^\beta\ee_\beta = \diffp{x^\beta}{x^{\alpha'}}\ee_\beta
  \end{align}
  ある座標系において基底ベクトル同士の内積を計量テンソル $g_{\alpha\beta}$ といい、これが座標系を表現する。
  \begin{align}
    \ee_\alpha\cdot\ee_\beta & = g_{\alpha\beta}
  \end{align}
  また基底ベクトルの微分による係数をクリストッフェル記号といい、$\Gamma^\mu_{\ \alpha\beta}$ と書く。
  \begin{align}
    \diffp{\ee_\alpha}{x^\beta} & = \Gamma^\mu_{\ \alpha\beta}\ee_\mu
  \end{align}
\end{definition}
\begin{align}
  g^{\alpha\beta} & = g_{\alpha\beta}^{-1}                                       \\
  g_{\alpha\beta} & = g_{\beta\alpha}                                            \\
  g_{\mu\nu}      & = \Lambda^\alpha_{\ \mu}\Lambda^\beta_{\ \nu}g_{\alpha\beta}
\end{align}
ベクトル $\AA = A^\alpha\ee_\alpha$ を基底で微分 $\nabla_\beta\AA = A^\alpha_{\ ;\beta}\ee_\alpha$ について
\begin{align}
  A^\alpha_{\ ;\beta}\ee_\alpha & = \nabla_\beta\AA = \diffp{\AA}{x^\beta} = \diffp{}{x^\beta}(A^\alpha\ee_\alpha) = \diffp{A^\alpha}{x^\beta}\ee_\alpha + A^\alpha\diffp{\ee_\alpha}{x^\beta} = A^\alpha_{\ ,\beta}\ee_\alpha + A^\alpha\Gamma^\mu_{\ \alpha\beta}\ee_\mu
\end{align}
より
\begin{align}
  A^\alpha_{\ ;\beta} & = A^\alpha_{\ ,\beta} + A^\mu\Gamma^\alpha_{\ \mu\beta}
\end{align}
これを共変微分という。共変ベクトルについても共変微分すると
\begin{align}
  A_{\alpha;\mu} & = (g_{\alpha\beta}A^{\beta})_{;\mu} = g_{\alpha\beta;\mu}A^{\beta} + g_{\alpha\beta}A^{\beta}_{\ ;\mu} = g_{\alpha\beta;\mu}A^{\beta} + A_{\alpha;\mu}
\end{align}
より次のテンソル方程式が成り立つ。
\begin{align}
  g_{\alpha\beta;\mu} & = 0
\end{align}
またスカラー場 $\phi$ について 2 階微分することで
\begin{align}
  \phi_{,\alpha;\beta} & = \phi_{,\alpha,\beta} + \phi_\mu\Gamma^\mu_{\ \alpha\beta}                                                             \\
  \phi_{,\beta;\alpha} & = \phi_{,\beta,\alpha} + \phi_\mu\Gamma^\mu_{\ \beta\alpha} = \phi_{,\alpha,\beta} + \phi_\mu\Gamma^\mu_{\ \beta\alpha}
\end{align}
より $\phi_{,\alpha;\beta} = \phi_{,\beta;\alpha}$ であるから
\begin{align}
  \Gamma^\mu_{\ \alpha\beta} & = \Gamma^\mu_{\ \beta\alpha}
\end{align}
となる。
計量テンソルの共変微分を計算すると
\begin{align}
  g_{\alpha\beta;\mu} & = g_{\alpha\beta,\mu} - g_{\nu\beta}\Gamma^\nu_{\ \alpha\mu} - g_{\alpha\nu}\Gamma^\nu_{\ \mu\beta} = 0
\end{align}
これを添字を変えたものを適切に足し引きして両辺に $\frac{1}{2}g^{\alpha\nu}$ を掛けると
\begin{align}
  g_{\alpha\beta,\mu}                                                                          & = g_{\nu\beta}\Gamma^\nu_{\ \alpha\mu} + g_{\alpha\nu}\Gamma^\nu_{\ \mu\beta} & \\
  g_{\mu\alpha,\beta}                                                                          & = g_{\nu\alpha}\Gamma^\nu_{\ \mu\beta} + g_{\mu\nu}\Gamma^\nu_{\ \beta\alpha}   \\
  g_{\beta\mu,\alpha}                                                                          & = g_{\nu\mu}\Gamma^\nu_{\ \beta\alpha} + g_{\beta\nu}\Gamma^\nu_{\ \alpha\mu}   \\
  g_{\alpha\beta,\mu} + g_{\mu\alpha,\beta} - g_{\beta\mu,\alpha}                              & = 2g_{\alpha\nu}\Gamma^\nu_{\ \beta\mu}                                         \\
  \frac{1}{2}g^{\alpha\nu}\ab(g_{\alpha\beta,\mu} + g_{\mu\alpha,\beta} - g_{\beta\mu,\alpha}) & = \Gamma^\nu_{\ \beta\mu}
\end{align}
よりクリストッフェル記号を計量テンソルで表すことができる。
\begin{align}
  \Gamma^\mu_{\ \alpha\beta} & = \frac{1}{2}g^{\mu\nu}(g_{\alpha\nu,\beta} + g_{\beta\nu,\alpha} - g_{\alpha\beta,\nu})
\end{align}

\begin{itembox}[l]{テンソル方程式}
  \begin{align}
    A^\alpha_{\ ;\beta}        & = A^\alpha_{\ ,\beta} + A^\mu\Gamma^\alpha_{\ \mu\beta}                                  \\
    g_{\alpha\beta;\mu}        & = 0                                                                                      \\
    \Gamma^\mu_{\ \alpha\beta} & = \Gamma^\mu_{\ \beta\alpha}                                                             \\
    \Gamma^\mu_{\ \alpha\beta} & = \frac{1}{2}g^{\mu\nu}(g_{\alpha\nu,\beta} + g_{\beta\nu,\alpha} - g_{\alpha\beta,\nu})
  \end{align}
\end{itembox}



\begin{example}[デカルト座標]
  \begin{align}
    g_{\alpha\beta}             & = \delta_{\alpha\beta}                   \\
    \diffp{\ee_\alpha}{x^\beta} & = \Gamma^\mu_{\ \alpha\beta}\ee_\mu = 0  \\
    A^\alpha_{\ ;\beta}         & = A^\alpha_{\ ,\beta} = A_{\alpha,\beta} \\
    A^\alpha_{\ ;\alpha}        & = \diffp{A^x}{x} + \diffp{A^y}{y}
  \end{align}
\end{example}
\begin{example}[極座標]
  \begin{align}
    \Lambda_{\ \alpha}^\beta & =
    \begin{pmatrix}
      \Lambda_{\ r}^{x}      & \Lambda_{\ r}^{y}      \\
      \Lambda_{\ \theta}^{x} & \Lambda_{\ \theta}^{y}
    \end{pmatrix}
    =
    \begin{pmatrix}
      \diffp{x}{r}      & \diffp{y}{r}      \\
      \diffp{x}{\theta} & \diffp{y}{\theta}
    \end{pmatrix}
    =
    \begin{pmatrix}
      \cos\theta   & \sin\theta  \\
      -r\sin\theta & r\cos\theta
    \end{pmatrix}                                        \\
    \ee_r                    & = \cos\theta\ee_x + \sin\theta\ee_y    \\
    \ee_\theta               & = -r\sin\theta\ee_x + r\cos\theta\ee_y
  \end{align}
  \begin{align}
    g_{\alpha\beta} & = \begin{pmatrix}
                          1 & 0   \\
                          0 & r^2
                        \end{pmatrix}
  \end{align}
  \begin{align}
    \diffp{\ee_r}{r}           & = \Gamma^{r}_{\ rr}\ee_r + \Gamma^{\theta}_{\ rr}\ee_\theta = 0                                 \\
    \diffp{\ee_r}{\theta}      & = \Gamma^{r}_{\ r\theta}\ee_r + \Gamma^{\theta}_{\ r\theta}\ee_\theta = \frac{1}{r}\ee_\theta   \\
    \diffp{\ee_\theta}{r}      & = \Gamma^{r}_{\ \theta r}\ee_r + \Gamma^{\theta}_{\ \theta r}\ee_\theta = \frac{1}{r}\ee_\theta \\
    \diffp{\ee_\theta}{\theta} & = \Gamma^{r}_{\ \theta\theta}\ee_r + \Gamma^{\theta}_{\ \theta\theta}\ee_\theta = -r\ee_r
  \end{align}
  \begin{align}
    A^\alpha_{\ ;\alpha} & = \diffp{A^\alpha}{\alpha} + \Gamma^\alpha_{\ \mu\alpha}A^\mu \\
                         & = \diffp{A^r}{r} + \diffp{A^\theta}{\theta} + \frac{1}{r}A^r  \\
                         & = \frac{1}{r}\diffp{}{r}(rA^r) + \diffp{}{\theta}A^\theta
  \end{align}
\end{example}

\subsection{局所平坦性定理}
\begin{align}
  \eta_{\mu\nu} = \eta^{\mu\nu} = \begin{pmatrix}
                                    -1 & 0 & 0 & 0 \\ 0 & 1 & 0 & 0 \\ 0 & 0 & 1 & 0 \\ 0 & 0 & 0 & 1
                                  \end{pmatrix}
\end{align}
\begin{align}
  \dl{x^0}\dl{x^1}\dl{x^2}\dl{x^3} & = \diffp{(x^0, x^1, x^2, x^3)}{(x'^0, x'^1, x'^2, x'^3)}\dl{x'^0}\dl{x'^1}\dl{x'^2}\dl{x'^3} \\
                                   & = \det(\Lambda^\alpha_{\ \beta})\dl{x'^0}\dl{x'^1}\dl{x'^2}\dl{x'^3}                         \\
                                   & = \sqrt{-g}\dl{x'^0}\dl{x'^1}\dl{x'^2}\dl{x'^3}
\end{align}

\begin{align}
  (g_{\alpha\beta})              & = (\Lambda^\alpha_{\ \beta})(\eta_{\alpha\beta})(\Lambda^\alpha_{\ \beta})^T               \\
  g = \det(g_{\alpha\beta})      & = \det(\Lambda^\alpha_{\ \beta})\det(\eta_{\alpha\beta})\det((\Lambda^\alpha_{\ \beta})^T) \\
                                 & = -\det(\Lambda^\alpha_{\ \beta})^2                                                        \\
  \det(\Lambda^\alpha_{\ \beta}) & = \sqrt{-g}
\end{align}
\begin{theorem}[発散の公式]
  \begin{align}
    A^\alpha_{\ ;\alpha} & = \frac{1}{\sqrt{-g}}(\sqrt{-g}A^\mu)_{,\mu}
  \end{align}
\end{theorem}
\begin{proof}
  \begin{align}
    \Gamma^\alpha_{\ \mu\alpha} & = \frac{1}{2}g^{\alpha\beta}(g_{\mu\beta,\alpha} + g_{\alpha\beta,\mu} - g_{\mu\alpha,\beta})                                                                                                                     \\
                                & = \frac{1}{2}\underbrace{g^{\alpha\beta}}_{対称}\underbrace{(g_{\mu\beta,\alpha} - g_{\mu\alpha,\beta})}_{反対称} + \frac{1}{2}g^{\alpha\beta}g_{\alpha\beta,\mu}                                                      \\
                                & = \frac{1}{2}g^{\alpha\beta}g_{\alpha\beta,\mu}                                                                                                                                                                   \\
                                & = \frac{(\sqrt{-g})_{,\mu}}{\sqrt{-g}}                                                                                                                       & (g_{,\mu} & = gg^{\alpha\beta}g_{\alpha\beta,\mu}) \\
  \end{align}
  \begin{align}
    A^\alpha_{\ ;\alpha} & = A^\alpha_{\ ,\alpha} + A^\mu\Gamma^\alpha_{\ \mu\alpha}           \\
                         & = A^\alpha_{\ ,\alpha} + \frac{1}{\sqrt{-g}}A^\mu(\sqrt{-g})_{,\mu} \\
                         & = \frac{1}{\sqrt{-g}}(\sqrt{-g}A^\mu)_{,\mu}
  \end{align}

\end{proof}
\begin{align}
  V^\alpha(B) - V^\alpha(A) & = \int_A^B\diffp{V^\alpha}{x^1}\dl{x^1} = -\int_{x^2 = b}\Gamma^\alpha_{\ \mu 1}V^\mu\dl{x^1}
\end{align}
\begin{align}
  \delta V^\alpha & = 最初に \delta a\ee_\sigma, 次に \delta b \ee_\lambda, そして -\delta a\ee_\sigma, 最後に -\delta b\ee_\lambda の移動による V^\alpha の変化                                                                                                                                                                                 \\
                  & = -\int_{x^\lambda = b}\Gamma^\alpha_{\ \mu\sigma}V^\mu\dl{x^\sigma} - \int_{x^\sigma = a + \delta a}\Gamma^\alpha_{\ \mu\lambda}V^\mu\dl{x^\lambda} + \int_{x^\lambda = b + \delta b}\Gamma^\alpha_{\ \mu\sigma}V^\mu\dl{x^\sigma} + \int_{x^\sigma = a}\Gamma^\alpha_{\ \mu\lambda}V^\mu\dl{x^\lambda} \\
                  & \approx \int_{a}^{a + \delta a}\delta b\diffp{}{x^\lambda}(\Gamma^\alpha_{\ \mu\sigma}V^\mu)\dl{x^\sigma} + \int_{b}^{b + \delta b}\delta a\diffp{}{x^\sigma}(\Gamma^\alpha_{\ \mu\lambda}V^\mu)\dl{x^\lambda}                                                                                           \\
                  & \approx \delta a\delta b\ab[\diffp{}{x^\lambda}(\Gamma^\alpha_{\ \mu\sigma}V^\mu) - \diffp{}{x^\sigma}(\Gamma^\alpha_{\ \mu\lambda}V^\mu)]                                                                                                                                                               \\
                  & = \delta a\delta b[\Gamma^\alpha_{\ \mu\sigma,\lambda} - \Gamma^\alpha_{\ \mu\lambda,\sigma} + \Gamma^\alpha_{\ \nu\lambda}\Gamma^\nu_{\ \mu\sigma} - \Gamma^\alpha_{\ \nu\sigma}\Gamma^\nu_{\ \mu\lambda}]V^\mu
\end{align}
\begin{definition}[リーマンの曲率テンソル]
  \begin{align}
    R^\alpha_{\ \beta\mu\nu} & := \Gamma^\alpha_{\ \beta\nu,\mu} - \Gamma^\alpha_{\ \beta\mu,\nu} + \Gamma^\alpha_{\ \sigma\mu}\Gamma^\sigma_{\ \beta\nu} - \Gamma^\alpha_{\ \sigma\nu}\Gamma^\nu_{\ \beta\mu}
  \end{align}
\end{definition}
局所慣性系において $\Gamma^\alpha_{\ \mu\nu} = 0$ であるから
\begin{align}
  R^\alpha_{\ \beta\mu\nu} & = \Gamma^\alpha_{\ \beta\nu,\mu} - \Gamma^\alpha_{\ \beta\mu,\nu}                                                                                                                                               \\
                           & = \frac{1}{2}g^{\alpha\sigma}(g_{\sigma\beta,\mu\nu} + g_{\sigma\nu,\beta\mu} - g_{\beta\nu,\sigma\mu}) - \frac{1}{2}g^{\alpha\sigma}(g_{\sigma\beta,\nu\mu} + g_{\sigma\mu,\beta\nu} - g_{\beta\mu,\sigma\nu}) \\
                           & = \frac{1}{2}g^{\alpha\sigma}\ab(g_{\sigma\nu,\beta\mu} - g_{\sigma\mu,\beta\nu} + g_{\beta\mu,\sigma\nu} - g_{\beta\nu,\sigma\mu})
\end{align}

\begin{align}
  R_{\alpha\beta\mu\nu} & = g_{\alpha\lambda}R^\lambda_{\ \beta\mu\nu} = \frac{1}{2}\ab(g_{\alpha\nu,\beta\mu} - g_{\alpha\mu,\beta\nu} + g_{\beta\mu,\alpha\nu} - g_{\beta\nu,\alpha\mu})
\end{align}
\begin{align}
   & R_{\alpha\beta\mu\nu} = -R_{\beta\alpha\mu\nu} = -R_{\alpha\beta\nu\mu} = R_{\mu\nu\alpha\beta} \\
   & R_{\alpha\beta\mu\nu} + R_{\alpha\nu\beta\mu} + R_{\alpha\mu\nu\beta} = 0
\end{align}




\end{document}
