\RequirePackage{plautopatch}
\documentclass[uplatex,dvipdfmx,a4paper,11pt]{jlreq}
\usepackage{bxpapersize}
\usepackage[utf8]{inputenc}
\usepackage{fontenc}
\usepackage{lmodern}
\usepackage{otf}
\usepackage{amsmath}
\usepackage{amssymb}
\usepackage{amsthm}
\usepackage{ascmac}
% \usepackage[hyphens]{url}
\usepackage{physics}
\usepackage{braket}
\usepackage{verbatimbox}
\usepackage{bm}
\usepackage{url}
% \usepackage[dvipdfmx,hiresbb,final]{graphicx}
\usepackage{hyperref}
\usepackage{pxjahyper}
\usepackage{tikz}\usetikzlibrary{cd}
\usepackage{listings}
\usepackage{color}
\usepackage{mathtools}
\usepackage{xspace}
\usepackage{xy}
\usepackage{xypic}
%
\title{複素解析}
\author{Anko}
\makeatletter
%
\DeclareMathOperator{\lcm}{lcm}
\DeclareMathOperator{\Kernel}{Ker}
\DeclareMathOperator{\Image}{Im}
\DeclareMathOperator{\ch}{ch}
\DeclareMathOperator{\Aut}{Aut}
\DeclareMathOperator{\Log}{Log}
\DeclareMathOperator{\Arg}{Arg}
\DeclareMathOperator{\sgn}{sgn}
%
\newcommand{\CC}{\mathbb{C}}
\newcommand{\RR}{\mathbb{R}}
\newcommand{\QQ}{\mathbb{Q}}
\newcommand{\ZZ}{\mathbb{Z}}
\newcommand{\NN}{\mathbb{N}}
\newcommand{\FF}{\mathbb{F}}
\newcommand{\PP}{\mathbb{P}}
\newcommand{\GG}{\mathbb{G}}
\newcommand{\TT}{\mathbb{T}}
\newcommand{\calB}{\mathcal{B}}
\newcommand{\calF}{\mathcal{F}}
\newcommand{\ignore}[1]{}
\newcommand{\floor}[1]{\left\lfloor #1 \right\rfloor}
% \newcommand{\abs}[1]{\left\lvert #1 \right\rvert}
\newcommand{\lt}{<}
\newcommand{\gt}{>}
\newcommand{\id}{\mathrm{id}}
\newcommand{\rot}{\curl}
\renewcommand{\angle}[1]{\left\langle #1 \right\rangle}
\newcommand{\EE}{\bm{E}}
\newcommand{\BB}{\bm{B}}
\renewcommand{\AA}{\bm{A}}
\newcommand{\rr}{\bm{r}}
\newcommand{\kk}{\bm{k}}
\newcommand{\pp}{\bm{p}}

\let\oldcite=\cite
\renewcommand\cite[1]{\hyperlink{#1}{\oldcite{#1}}}

\let\oldbibitem=\bibitem
\renewcommand{\bibitem}[2][]{\label{#2}\oldbibitem[#1]{#2}}

% theorem環境の設定
% - 冒頭に改行
% - 末尾にdiamond (amsthm)
\theoremstyle{definition}
\newcommand*{\newscreentheoremx}[2]{
  \newenvironment{#1}[1][]{
    \begin{screen}
    \begin{#2}[##1]
      \leavevmode
      \newline
  }{
    \end{#2}
    \end{screen}
  }
}
\newcommand*{\newqedtheoremx}[2]{
  \newenvironment{#1}[1][]{
    \begin{#2}[##1]
      \leavevmode
      \newline
      \renewcommand{\qedsymbol}{\(\diamond\)}
      \pushQED{\qed}
  }{
      \qedhere
      \popQED
    \end{#2}
  }
}
\newtheorem{theorem*}{定理}

\newqedtheoremx{theorem}{theorem*}
\newcommand*\newqedtheorem@unstarred[2]{%
  \newtheorem{#1*}[theorem*]{#2}
  \newqedtheoremx{#1}{#1*}
}
\newcommand*\newqedtheorem@starred[2]{%
  \newtheorem*{#1*}{#2}
  \newqedtheoremx{#1}{#1*}
}
\newcommand*{\newqedtheorem}{\@ifstar{\newqedtheorem@starred}{\newqedtheorem@unstarred}}

\newtheorem{sctheorem*}{定理}
\newscreentheoremx{sctheorem}{sctheorem*}
\newcommand*\newscreentheorem@unstarred[2]{%
  \newtheorem{#1*}[theorem*]{#2}
  \newscreentheoremx{#1}{#1*}
}
\newcommand*\newscreentheorem@starred[2]{%
  \newtheorem*{#1*}{#2}
  \newscreentheoremx{#1}{#1*}
}
\newcommand*{\newscreentheorem}{\@ifstar{\newscreentheorem@starred}{\newscreentheorem@unstarred}}

%\newtheorem*{definition}{定義}
%\newtheorem{theorem}{定理}
%\newtheorem{proposition}[theorem]{命題}
%\newtheorem{lemma}[theorem]{補題}
%\newtheorem{corollary}[theorem]{系}

\newqedtheorem{lemma}{補題}
\newqedtheorem{corollary}{系}
\newqedtheorem{example}{例}
\newqedtheorem{proposition}{命題}
\newqedtheorem{remark}{注意}
\newqedtheorem{thesis}{主張}
\newqedtheorem{notation}{記法}
\newqedtheorem{problem}{問題}
\newqedtheorem{algorithm}{アルゴリズム}

\newscreentheorem*{definition}{定義}

\renewenvironment{proof}[1][\proofname]{\par
  \normalfont
  \topsep6\p@\@plus6\p@ \trivlist
  \item[\hskip\labelsep{\bfseries #1}\@addpunct{\bfseries}]\ignorespaces\quad\par
}{%
  \qed\endtrivlist\@endpefalse
}
\renewcommand\proofname{証明}

\makeatother

\begin{document}
\maketitle
\tableofcontents
\clearpage

\section{複素関数}
この章では
\subsection{複素数}
高校では $i^2 = -1$ を満たす $i$ を虚数単位といってこれを実数に付け加えたものを複素数と言っていました。これを代数の知識を用いて定義し直します。

\begin{definition}
  $\RR[x]/(x^2 + 1)$ を複素数体といい、 $\CC$ と書く。この元を複素数といい、 $x$ を虚数単位 (imaginary unit) といい、$i$ と書くことにする。このとき任意の元 $z\in\CC$ は 2 つの実数 $x, y\in\RR$ を用いて $z = x + iy$ と表される。このとき関数 $\Re, \Im$ を次のように定義する。
  \begin{align}
    \begin{cases}
      x = \Re z \\
      y = \Im z
    \end{cases}
  \end{align}
\end{definition}

\begin{proposition}
  複素数 $z_1 = x_1 + iy_1$, $z_2 = x_2 + iy_2$, $z = x + iy$ とすると次が成り立つ。
  \begin{align}
     & z_1 = z_2 \iff x_1 = x_2 \land y_1 = y_2            \\
     & z_1 + z_2 = (x_1 + x_2) + i(y_1 + y_2)              \\
     & z_1z_2 = (x_1x_2 - y_1y_2) + i(x_1y_2 + x_2y_1)     \\
     & z^{-1} = \frac{x}{x^2 + y^2} - i\frac{y}{x^2 + y^2}
  \end{align}
\end{proposition}
\begin{proof}
  複素数体 $\CC$ において $\lbrace 1, i\rbrace$ は基底となるから線形独立であり $x + iy = 0$ ならば $x = 0$ かつ $y = 0$ である。
  これより次のようになる。
  \begin{align}
    z_1 = z_2 & \iff z_1 - z_2 = 0 \iff (x_1 - x_2) + i(y_1 - y_2) = 0                \\
              & \iff x_1 - x_2 = 0 \land y_1 - y_2 = 0 \iff x_1 = x_2 \land y_1 = y_2
  \end{align}
  他の等式は計算すれば満たすことが分かる。
\end{proof}

また交換法則、結合法則、分配法則を満たすことが分かるので $\CC$ は可換体です。
\begin{definition}
  複素数 $z = x + iy$ の絶対値 (absolute value)、共役複素数 (complex conjugate) をそれぞれ次のように定義する。
  \begin{align}
    |z|          & = \sqrt{x^2 + y^2} \\
    \overline{z} & = x - iy
  \end{align}
\end{definition}
\begin{proposition}
  複素数 $z, z_1, z_2\in\CC$ に対して次の式が成り立つ。
  \begin{align}
     & |z|^2 = z\overline{z}                                   \\
     & \qty||z_1| - |z_2|| \leq |z_1 + z_2| \leq |z_1| + |z_2|
  \end{align}
\end{proposition}
\begin{proof}
  最初の式については $z = x + iy$ と置くと次のように表される。
  \begin{align}
    |z|^2 = x^2 + y^2 = (x + iy)(x - iy) = z\overline{z}
  \end{align}
  次に第二式の各項についてそれぞれを二乗した値を比較する。
  \begin{align}
    \qty(|z_1| - |z_2|)^2 & = \qty(\sqrt{x_1^2 + y_1^2} - \sqrt{x_2^2 + y_2^2})^2 = x_1^2 + y_1^2 + x_2^2 + y_2^2 - 2\sqrt{(x_1^2 + y_1^2)(x_2^2 + y_2^2)} \\
    |z_1 + z_2|^2         & = (x_1 + x_2)^2 + (y_1 + y_2)^2 = x_1^2 + x_2^2 + y_1^2 + y_2^2 + 2(x_1x_2 + y_1y_2)                                           \\
    (|z_1| + |z_2|)^2     & = |z_1|^2 + |z_2|^2 + 2|z_1||z_2| = x_1^2 + y_1^2 + x_2^2 + y_2^2 + 2\sqrt{(x_1^2 + y_1^2)(x_2^2 + y_2^2)}
  \end{align}
  ここで $x_1^2y_2^2 + x_2^2y_1^2 \geq 2x_1x_2y_1y_2$ であるから $x^2 \leq y^2\implies |x|\leq |y|$ より第二式が成り立つ。
\end{proof}

無限遠点

リーマン面
リーマン球面

\subsection{複素変数の関数}

\begin{theorem}[コーシー・リーマンの方程式]
  複素変数の関数 $f(z)\in\CC(z) = f(x + iy) = u(x, y) + iv(x, y)$ が $z = z_0$ で微分可能であるとき導関数 $f'(z_0)$ は次のように書くことができる。
  \begin{align}
    f'(z_0) = \lim_{\Delta z\to 0}\frac{f(z_0 + \Delta z) - f(z_0)}{\Delta z}
  \end{align}
  このとき実関数 $u, v$ は次の式を満たす。
  \begin{align}
    \pdv{u}{x} = \pdv{v}{y}, \pdv{u}{y} = -\pdv{v}{x}
  \end{align}
\end{theorem}
\begin{proof}
  $\Delta z = \Delta x + 0i$
  \begin{align}
    f'(z_0) & = \lim_{\Delta z\to 0}\frac{f(z_0 + \Delta z) - f(z_0)}{\Delta z}                                                     \\
            & = \lim_{\Delta x\to 0}\frac{u(x_0 + \Delta x, y_0) - u(x_0, y_0) + i(v(x_0 + \Delta x, y_0) - v(x_0, y_0))}{\Delta x} \\
            & = \pdv{u(x_0, y_0)}{x} + i\pdv{v(x_0, y_0)}{x}
  \end{align}
  $\Delta z = 0 + i\Delta y$
  \begin{align}
    f'(z_0) & = \lim_{\Delta z\to 0}\frac{f(z_0 + \Delta z) - f(z_0)}{\Delta z}                                                      \\
            & = \lim_{\Delta y\to 0}\frac{u(x_0, y_0 + \Delta y) - u(x_0, y_0) + i(v(x_0, y_0 + \Delta y) - v(x_0, y_0))}{i\Delta y} \\
            & = \pdv{v(x_0, y_0)}{y} - i\pdv{u(x_0, y_0)}{y}
  \end{align}
  これより領域において次が成り立つ
  \begin{align}
    \pdv{u}{x} = \pdv{v}{y}, \pdv{u}{y} = -\pdv{v}{x}
  \end{align}
\end{proof}


\subsection{正則関数}
\begin{theorem}
  Riemman Roch の定理
\end{theorem}

\begin{definition}[初等関数]
  指数関数 $2\pi i$ 周期で同じ
  \begin{align}
    f(z) = e^z = e^{x + iy} = e^x(\cos y + i\sin y)
  \end{align}
  三角関数
  \begin{align}
    \sin z & := \frac{e^{iz} - e^{-iz}}{2i}                          \\
    \cos z & := \frac{e^{iz} + e^{-iz}}{2}                           \\
    \tan z & := \frac{1}{i}\frac{e^{iz} - e^{-iz}}{e^{iz} + e^{-iz}}
  \end{align}

  双曲線関数
  \begin{align}
    \sinh z & := \frac{e^{z} - e^{-z}}{2}              \\
    \cosh z & := \frac{e^{z} + e^{-z}}{2}              \\
    \tanh z & := \frac{e^{z} - e^{-z}}{e^{z} + e^{-z}}
  \end{align}

  対数関数
  \begin{align}
    \log z & := \log r + i\theta \qquad (z\neq 0) \\
    \theta & = \arg z = \Arg z + 2n\pi            \\
    \Log z & := \log r + i\Arg z
  \end{align}
\end{definition}
\begin{proposition}
  \begin{align}
    \sin^{-1} z = -i\log(iz + (1 - z^2)^{1/2})
  \end{align}
\end{proposition}

\begin{theorem}[一致の定理]
  領域 $D$ で正則な関数 $f(z), g(z)$ があり、$D$ の小領域もしくは曲線上で一致しているとき、領域 $D$ 全体で $f(z) = g(z)$ が成り立つ。
\end{theorem}
\begin{proof}
  \begin{align}
    f(z) & = \sum_{n=0}^{\infty}\frac{f^{(n)}(z_0)}{n!}(z - z_0)^n \\
    g(z) & = \sum_{n=0}^{\infty}\frac{g^{(n)}(z_0)}{n!}(z - z_0)^n
  \end{align}
\end{proof}
\begin{theorem}[解析接続]
\end{theorem}


\end{document}