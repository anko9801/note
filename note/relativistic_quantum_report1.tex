\RequirePackage{plautopatch}
\documentclass[uplatex,dvipdfmx,a4paper,11pt]{jlreq}
\usepackage{bxpapersize}
\usepackage[utf8]{inputenc}
\usepackage{fontenc}
\usepackage{lmodern}
\usepackage{otf}
\usepackage{amsmath}
\usepackage{amssymb}
\usepackage{amsthm}
\usepackage{ascmac}
% \usepackage[hyphens]{url}
\usepackage{physics}
\usepackage{braket}
\usepackage{verbatimbox}
\usepackage{bm}
\usepackage{url}
% \usepackage[dvipdfmx,hiresbb,final]{graphicx}
\usepackage{hyperref}
\usepackage{pxjahyper}
\usepackage{tikz}\usetikzlibrary{cd}
\usepackage{listings}
\usepackage{color}
\usepackage{mathtools}
\usepackage{xspace}
\usepackage{xy}
\usepackage{xypic}
%
\title{相対論的量子力学 レポート課題 1}
\author{宇佐見大希}
\makeatletter
%
\DeclareMathOperator{\lcm}{lcm}
\DeclareMathOperator{\Kernel}{Ker}
\DeclareMathOperator{\Image}{Im}
\DeclareMathOperator{\ch}{ch}
\DeclareMathOperator{\Aut}{Aut}
\DeclareMathOperator{\Log}{Log}
\DeclareMathOperator{\Arg}{Arg}
\DeclareMathOperator{\sgn}{sgn}
%
\newcommand{\CC}{\mathbb{C}}
\newcommand{\RR}{\mathbb{R}}
\newcommand{\QQ}{\mathbb{Q}}
\newcommand{\ZZ}{\mathbb{Z}}
\newcommand{\NN}{\mathbb{N}}
\newcommand{\FF}{\mathbb{F}}
\newcommand{\PP}{\mathbb{P}}
\newcommand{\GG}{\mathbb{G}}
\newcommand{\TT}{\mathbb{T}}
\newcommand{\calB}{\mathcal{B}}
\newcommand{\calF}{\mathcal{F}}
\newcommand{\ignore}[1]{}
\newcommand{\floor}[1]{\left\lfloor #1 \right\rfloor}
% \newcommand{\abs}[1]{\left\lvert #1 \right\rvert}
\newcommand{\lt}{<}
\newcommand{\gt}{>}
\newcommand{\id}{\mathrm{id}}
\newcommand{\rot}{\curl}
\renewcommand{\angle}[1]{\left\langle #1 \right\rangle}
\newcommand{\EE}{\bm{E}}
\newcommand{\BB}{\bm{B}}
\renewcommand{\AA}{\bm{A}}
\newcommand{\rr}{\bm{r}}
\newcommand{\kk}{\bm{k}}
\newcommand{\pp}{\bm{p}}

\let\oldcite=\cite
\renewcommand\cite[1]{\hyperlink{#1}{\oldcite{#1}}}

\let\oldbibitem=\bibitem
\renewcommand{\bibitem}[2][]{\label{#2}\oldbibitem[#1]{#2}}

% theorem環境の設定
% - 冒頭に改行
% - 末尾にdiamond (amsthm)
\theoremstyle{definition}
\newcommand*{\newscreentheoremx}[2]{
  \newenvironment{#1}[1][]{
    \begin{screen}
    \begin{#2}[##1]
      \leavevmode
      \newline
  }{
    \end{#2}
    \end{screen}
  }
}
\newcommand*{\newqedtheoremx}[2]{
  \newenvironment{#1}[1][]{
    \begin{#2}[##1]
      \leavevmode
      \newline
      \renewcommand{\qedsymbol}{\(\diamond\)}
      \pushQED{\qed}
  }{
      \qedhere
      \popQED
    \end{#2}
  }
}
\newtheorem{theorem*}{定理}

\newqedtheoremx{theorem}{theorem*}
\newcommand*\newqedtheorem@unstarred[2]{%
  \newtheorem{#1*}[theorem*]{#2}
  \newqedtheoremx{#1}{#1*}
}
\newcommand*\newqedtheorem@starred[2]{%
  \newtheorem*{#1*}{#2}
  \newqedtheoremx{#1}{#1*}
}
\newcommand*{\newqedtheorem}{\@ifstar{\newqedtheorem@starred}{\newqedtheorem@unstarred}}

\newtheorem{sctheorem*}{定理}
\newscreentheoremx{sctheorem}{sctheorem*}
\newcommand*\newscreentheorem@unstarred[2]{%
  \newtheorem{#1*}[theorem*]{#2}
  \newscreentheoremx{#1}{#1*}
}
\newcommand*\newscreentheorem@starred[2]{%
  \newtheorem*{#1*}{#2}
  \newscreentheoremx{#1}{#1*}
}
\newcommand*{\newscreentheorem}{\@ifstar{\newscreentheorem@starred}{\newscreentheorem@unstarred}}

%\newtheorem*{definition}{定義}
%\newtheorem{theorem}{定理}
%\newtheorem{proposition}[theorem]{命題}
%\newtheorem{lemma}[theorem]{補題}
%\newtheorem{corollary}[theorem]{系}

\newqedtheorem{lemma}{補題}
\newqedtheorem{corollary}{系}
\newqedtheorem{example}{例}
\newqedtheorem{proposition}{命題}
\newqedtheorem{remark}{注意}
\newqedtheorem{thesis}{主張}
\newqedtheorem{notation}{記法}
\newqedtheorem{problem}{問題}
\newqedtheorem{algorithm}{アルゴリズム}

\newscreentheorem*{definition}{定義}

\renewenvironment{proof}[1][\proofname]{\par
  \normalfont
  \topsep6\p@\@plus6\p@ \trivlist
  \item[\hskip\labelsep{\bfseries #1}\@addpunct{\bfseries}]\ignorespaces\quad\par
}{%
  \qed\endtrivlist\@endpefalse
}
\renewcommand\proofname{証明}

\makeatother

\begin{document}
\maketitle

\begin{problem}
$\alpha^i$ と $\beta$ を次を満たすエルミート行列とする。
\begin{align}
  \lbrace\alpha^i,\alpha^j\rbrace = 2\delta^{ij}, \qquad \beta^2 = 1, \qquad \lbrace\alpha^i,\beta\rbrace = 0 \label{condition}
\end{align}
\end{problem}
(1) $\gamma^0, \gamma^i$ を次のように定義する。
\begin{align}
  \gamma^0 := \beta, \qquad \gamma^i := \beta\alpha^i
\end{align}
このとき次の式を示せ。
\begin{align}
  (\gamma^0)^\dagger = \gamma^0, \qquad (\gamma^i)^\dagger = \gamma^i, \qquad \lbrace\gamma^\mu, \gamma^\nu\rbrace = 2g^{\mu\nu}
\end{align}
\begin{proof}
  $\alpha^i$, $\beta$ がエルミート行列であることと式 \eqref{condition} より次の式が成り立つ。
  \begin{align}
    (\alpha^i)^\dagger & = \alpha^i                            \\
    \beta^\dagger      & = \beta                               \\
    \alpha^i\alpha^j   & = - \alpha^j\alpha^i \qquad (i\neq j) \\
    \alpha^i\beta      & = -\beta\alpha^i
  \end{align}
  これらよりガンマ行列のエルミート共役が分かる。
  \begin{align}
    (\gamma^0)^\dagger & = \beta^\dagger = \beta = \gamma^0                                     \\
    (\gamma^i)^\dagger & = (\beta\alpha^i)^\dagger = \alpha^i\beta = -\beta\alpha^i = -\gamma^i
  \end{align}
  次に $\lbrace\gamma^\mu, \gamma^\nu\rbrace = 2g^{\mu\nu}$ を示す。$\lbrace\gamma^\mu, \gamma^\nu\rbrace = \lbrace\gamma^\nu, \gamma^\mu\rbrace$ より $\mu \leq \nu$ を示せばよい。 \\
  $\mu > 0$ のとき
  \begin{align}
    \lbrace\gamma^\mu, \gamma^\nu\rbrace & = \beta\alpha^\mu\beta\alpha^\nu + \beta\alpha^\nu\beta\alpha^\mu \\
                                         & = -\beta^2(\alpha^\mu\alpha^\nu + \alpha^\nu\alpha^\mu)           \\
                                         & = -2\delta^{\mu\nu}
  \end{align}
  $\mu = 0$ かつ $\mu < \nu$ のとき
  \begin{align}
    \lbrace\gamma^\mu, \gamma^\nu\rbrace & = \beta\beta\alpha^\nu + \beta\alpha^\nu\beta \\
                                         & = \beta^2\alpha^\nu - \beta^2\alpha^\nu       \\
                                         & = 0
  \end{align}
  $\mu = \nu = 0$ のとき
  \begin{align}
    \lbrace\gamma^\mu, \gamma^\nu\rbrace & = 2\beta^2 \\
                                         & = 2
  \end{align}
  よって $\lbrace\gamma^\mu, \gamma^\nu\rbrace = 2g^{\mu\nu}$ と書ける。
\end{proof}

(2) $\gamma^5$ を次のように定義する。
\begin{align}
  \gamma^5 := i\gamma^0\gamma^1\gamma^2\gamma^3
\end{align}
このとき次の式を示せ。
\begin{align}
  (\gamma^5)^\dagger = \gamma^5, \qquad (\gamma^5)^2 = 1, \qquad \lbrace\gamma^\mu, \gamma^5\rbrace = 0
\end{align}
\begin{proof}
  \begin{align}
    (\gamma^5)^\dagger & = -i(-\gamma^3)(-\gamma^2)(-\gamma^1)\gamma^0                                   \\
                       & = i\gamma^3\gamma^2\gamma^1\gamma^0                                             \\
                       & = (-1)^6i\gamma^0\gamma^1\gamma^2\gamma^3                                       \\
                       & = \gamma^5                                                                      \\
    (\gamma^5)^2       & = -(\gamma^0\gamma^1\gamma^2\gamma^3)(\gamma^0\gamma^1\gamma^2\gamma^3)^\dagger \\
                       & = - (\gamma^0)^2(\gamma^1)^2(\gamma^2)^2(\gamma^3)^2                            \\
                       & = 1
  \end{align}
  \begin{align}
    \lbrace\gamma^\mu, \gamma^5\rbrace & = i(\gamma^\mu\gamma^0\gamma^1\gamma^2\gamma^3 + \gamma^0\gamma^1\gamma^2\gamma^3\gamma^\mu) \\
                                       & = i((-1)^\mu + (-1)^{3-\mu})\gamma^0\cdots(\gamma^\mu)^2\cdots\gamma^3                       \\
                                       & = 0
  \end{align}
\end{proof}

(3) $\Sigma^i$ を次のように定義する。
\begin{align}
  \Sigma^i := \frac{i}{2}\sum_{j,k}\varepsilon^{ijk}\gamma^j\gamma^k
\end{align}
このとき次の式を示せ。
\begin{align}
  (\Sigma^i)^\dagger = \Sigma^i, \qquad \lbrace\Sigma^i, \Sigma^j\rbrace = 2\delta^{ij}
\end{align}
\begin{proof}

  \begin{align}
    (\Sigma^i)^\dagger & = -\frac{i}{2}\sum_{j,k}(\varepsilon^{ijk}\gamma^j\gamma^k)^\dagger \\
                       & = -\frac{i}{2}\sum_{j,k}\varepsilon^{ijk}\gamma^k\gamma^j           \\
                       & = -\frac{i}{2}\sum_{j,k}-\varepsilon^{ikj}\gamma^k\gamma^j          \\
                       & = \Sigma^i
  \end{align}
  次に $\lbrace\Sigma^i, \Sigma^j\rbrace = 2\delta^{ij}$ を示す。 \\
  $i = j$ のとき、ある $k, l$ が存在して
  \begin{align}
    \lbrace\Sigma^i, \Sigma^j\rbrace & = -\frac{1}{2}\qty(\sum_{a,b}\varepsilon^{iab}\gamma^a\gamma^b)^2                                                                                             \\
                                     & = -\frac{1}{2}\qty(\gamma^k\gamma^l - \gamma^l\gamma^k)^2                                                                                                     \\
                                     & = -\frac{1}{2}\qty(\gamma^k\gamma^l\gamma^k\gamma^l - \gamma^k\gamma^l\gamma^l\gamma^k - \gamma^l\gamma^k\gamma^k\gamma^l + \gamma^l\gamma^k\gamma^l\gamma^k) \\
                                     & = 2
  \end{align}
  $i \neq j$ のとき $a, b, c, d$ のいづれか 2 つ 1 組は同じであるから
  \begin{align}
    \lbrace\Sigma^i, \Sigma^j\rbrace & = -\frac{1}{4}\sum_{a,b}\sum_{c,d}(\varepsilon^{iab}\gamma^a\gamma^b\varepsilon^{jcd}\gamma^c\gamma^d + \varepsilon^{jcd}\gamma^c\gamma^d\varepsilon^{iab}\gamma^a\gamma^b) \\
                                     & = -\frac{1}{4}\sum_{a,b}\sum_{c,d}\varepsilon^{iab}\varepsilon^{jcd}(1 + (-1)^3)\gamma^a\gamma^b\gamma^c\gamma^d                                                            \\
                                     & = 0
  \end{align}
  よって $\lbrace\Sigma^i, \Sigma^j\rbrace = 2\delta^{ij}$ である。
\end{proof}

(4) $\bm{\Sigma} = (\Sigma^1, \Sigma^2, \Sigma^3)$ と任意の $\bm{v} = (v^1, v^2, v^3)$ に対して次が成り立つことを示せ。
\begin{align}
  (\bm{v}\vdot\bm{\Sigma})^2 = \bm{v}^2
\end{align}
\begin{proof}
  \begin{align}
    (\bm{v}\vdot\bm{\Sigma})^2 & = (v^i\Sigma^i)(v^j\Sigma^j)^\dagger \\
                               & = v^i\Sigma^i\Sigma^j(v^j)^\dagger   \\
                               & = \delta^{ij}v^i(v^j)^\dagger        \\
                               & = \bm{v}^2
  \end{align}
\end{proof}

(5) $\alpha^i, \beta$ の具体例を 1 つ挙げて性質を満たすことを確認せよ。
\begin{proof}
  次のディラック表示を用いる。
  \begin{align}
    \alpha^i = \begin{pmatrix}
                 0        & \sigma^i \\
                 \sigma^i & 0
               \end{pmatrix}, \qquad
    \beta = \begin{pmatrix}
              \sigma^0 & 0         \\
              0        & -\sigma^0
            \end{pmatrix}
  \end{align}
  このとき
  \begin{align}
    (\alpha^i)^\dagger & = \begin{pmatrix}
                             0        & \sigma^i \\
                             \sigma^i & 0
                           \end{pmatrix} = \alpha^i \\
    \beta^\dagger      & = \begin{pmatrix}
                             \sigma^0 & 0         \\
                             0        & -\sigma^0
                           \end{pmatrix} = \beta
  \end{align}
  \begin{align}
    \lbrace\alpha^i, \alpha^j\rbrace & = \begin{pmatrix}
                                           0        & \sigma^i \\
                                           \sigma^i & 0
                                         \end{pmatrix}
    \begin{pmatrix}
      0        & \sigma^j \\
      \sigma^j & 0
    \end{pmatrix} +
    \begin{pmatrix}
      0        & \sigma^j \\
      \sigma^j & 0
    \end{pmatrix}
    \begin{pmatrix}
      0        & \sigma^i \\
      \sigma^i & 0
    \end{pmatrix}                                                                                                                           \\
                                     & = (\sigma^i\sigma^j + \sigma^j\sigma^i)\otimes I                                                           \\
                                     & = \qty((\delta^{ij}I + i\varepsilon^{ijk}\sigma^k) + (\delta^{ji}I + i\varepsilon^{jik}\sigma^k))\otimes I \\
                                     & = 2\delta^{ij}                                                                                             \\
    \beta^2                          & = \begin{pmatrix}
                                           \sigma^0 & 0         \\
                                           0        & -\sigma^0
                                         \end{pmatrix}^2 = 1                                                                                     \\
    \lbrace\alpha^i, \beta\rbrace    & = \begin{pmatrix}
                                           0        & \sigma^i \\
                                           \sigma^i & 0
                                         \end{pmatrix}
    \begin{pmatrix}
      \sigma^0 & 0         \\
      0        & -\sigma^0
    \end{pmatrix} +
    \begin{pmatrix}
      \sigma^0 & 0         \\
      0        & -\sigma^0
    \end{pmatrix}
    \begin{pmatrix}
      0        & \sigma^i \\
      \sigma^i & 0
    \end{pmatrix}                                                                                                                           \\
                                     & = (\sigma^i\sigma^0 - \sigma^0\sigma^i)\otimes\begin{pmatrix}
                                                                                       0 & -1 \\
                                                                                       1 & 0
                                                                                     \end{pmatrix}                                               \\
                                     & = 0
  \end{align}
\end{proof}

\begin{problem}
共変微分 $D_\mu$ を次のように定義する。
\begin{align}
  D_\mu := \partial_\mu - \frac{q}{i\hbar}A_\mu(x)
\end{align}
\end{problem}
(1) $\bm{D} = (D_1, D_2, D_3)$ に対して
\begin{align}
  (\bm{D}\vdot\bm{\sigma})^2 & = \bm{D}^2 + \frac{q}{\hbar}\bm{B}(x)\vdot\bm{\sigma}
\end{align}
を示せ。
\begin{proof}
  \begin{align}
    D_\mu D_\nu^\dagger & = \qty(\partial_\mu - \frac{q}{i\hbar}A_\mu(x))\qty(\partial_\nu + \frac{q}{i\hbar}A_\nu(x))                                                    \\
                        & = \partial_\mu\partial_\nu + \frac{q}{i\hbar}\partial_\mu A_\nu(x) - \frac{q}{i\hbar}A_\mu(x)\partial_\nu + \frac{q^2}{\hbar^2}A_\mu(x)A_\nu(x)
  \end{align}
  \begin{align}
    (\bm{D}\vdot\bm{\sigma})^2 & = \sum_{i,j}(D_i\sigma^i)(D_j\sigma^j)^\dagger                                      \\
                               & = \sum_{i,j}\qty(\delta^{ij}I + i\sum_{k}\varepsilon^{ijk}\sigma^k)D_iD_j^\dagger   \\
                               & = \sum_{i}D_i^2 + i\sum_{i,j,k}\varepsilon^{ijk}\sigma^kD_iD_j^\dagger              \\
                               & = \bm{D}^2 + i\sum_{i,j,k}\varepsilon^{ijk}\sigma^k\frac{q}{i\hbar}\partial_iA_j(x) \\
                               & = \bm{D}^2 + \frac{q}{\hbar}\bm{\sigma}\vdot(\vnabla\cross\bm{A}(x))                \\
                               & = \bm{D}^2 + \frac{q}{\hbar}\bm{B}(x)\vdot\bm{\sigma}
  \end{align}
\end{proof}

(2) 荷電粒子に対するクライン・ゴルドン方程式
\begin{align}
  [\hbar^2D^2 + (mc)^2]\phi(x) = 0
\end{align}
が非相対論的極限 $mc^2\to\infty$ において、シュレーディンガー方程式に帰着することを示せ
\begin{proof}
  $A^\mu = (\phi/c, \bm{A})$ であり $\phi(x) = e^{-i(mc^2)t/\hbar}\varphi(x)$ とおくと
  \begin{align}
     & [\hbar^2D^2 + (mc)^2]\phi(x) = 0                                                                                                                                                                                                   \\
     & = \qty[\hbar^2\qty(\partial_\mu - \frac{q}{i\hbar}A_\mu(x))\qty(\partial^\nu - \frac{q}{i\hbar}A^\nu(x)) + (mc)^2]\phi(x)                                                                                                          \\
     & = \qty[\qty(\hbar^2\partial_\mu\partial^\nu + i\hbar\partial_\mu qA^\nu(x)  + qA_\mu(x)i\hbar\partial^\nu - q^2A_\mu(x)A^\nu(x)) + (mc)^2]\phi(x)                                                                                  \\
     & = \qty[\qty(\frac{\hbar^2}{c^2}\pdv[2]{t} + \pp^2) + q\qty(\frac{i\hbar}{c^2}\pdv{t}\phi - \pp\vdot\bm{A}(x)) + q\qty(\frac{i\hbar}{c^2}\phi\pdv{t} - \bm{A}(x)\vdot\pp) - q^2\qty(\frac{\phi^2}{c^2} - \bm{A}^2) + (mc)^2]\phi(x) \\
     & = \qty[\frac{\hbar^2}{c^2}\pdv[2]{t} + q\frac{i\hbar}{c^2}\pdv{t}\phi + q\frac{i\hbar}{c^2}\phi\pdv{t} - q^2\frac{\phi^2}{c^2} + (\pp - q\bm{A}(x))^2 + (mc)^2]e^{-i(mc^2)t/\hbar}\varphi(x)                                       \\
     & = \qty[-(mc)^2 + q\frac{i\hbar}{c^2}\pdv{\phi}{t} + 2mq\phi - \frac{q^2\phi^2}{c^2} + (\pp - q\bm{A}(x))^2 + (mc)^2]e^{-i(mc^2)t/\hbar}\varphi(x)                                                                                  \\
     & = 2m\qty[-\frac{q^2\phi^2}{2mc^2} + \frac{(\pp - q\bm{A}(x))^2}{2m} + q\phi(x)]e^{-i(mc^2)t/\hbar}\varphi(x)
  \end{align}
  非相対論的極限 $mc^2\to\infty$ のとき $\phi(x)$ は時間に依存しないからシュレーディンガー方程式となる。
  \begin{align}
    i\hbar\pdv{t}\phi(x) & = \qty(\frac{(\pp - q\bm{A}(x))^2}{2m} + q\phi)\phi(x)
  \end{align}
\end{proof}
(3)
\begin{align}
  (i\hbar\gamma^\mu D_\mu - mc)\psi(x) = 0
\end{align}
\begin{proof}
  まず次の式が成り立つことを確認する。
  \begin{align}
    \gamma^0\gamma^\mu\gamma^5 & = \gamma^0\gamma^0\gamma^5 + \gamma^0\gamma^i\gamma^5 \\
                               & = \gamma^0\gamma^0\gamma^5 - \gamma^i\gamma^0\gamma^5 \\
                               & = (\gamma^\mu)^\dagger\gamma^0\gamma^5
  \end{align}
  よってディラック方程式より次のようになる。
  \begin{align}
    \partial_\mu j_A^\mu(x) & = \partial_\mu(\bar{\psi}(x)\gamma^\mu\gamma^5\psi(x))                                                                                                                                      \\
                            & = \partial_\mu(\psi^\dagger(x)\gamma^0\gamma^\mu\gamma^5\psi(x))                                                                                                                            \\
                            & = (\partial_\mu\psi(x))^\dagger\gamma^0\gamma^\mu\gamma^5\psi(x) + \psi^\dagger(x)\gamma^0\gamma^\mu\gamma^5\partial_\mu\psi(x)                                                             \\
                            & = (\gamma^\mu\partial_\mu\psi(x))^\dagger\gamma^0\gamma^5\psi(x) - \psi^\dagger(x)\gamma^0\gamma^5(\gamma^\mu\partial_\mu\psi(x))                                                           \\
                            & = \qty(\frac{1}{i\hbar}\qty(\gamma^\mu qA_\mu(x) + mc)\psi(x))^\dagger\gamma^0\gamma^5\psi(x) - \psi^\dagger(x)\gamma^0\gamma^5\qty(\frac{1}{i\hbar}\qty(\gamma^\mu qA_\mu(x) + mc)\psi(x)) \\
                            & = -\frac{1}{i\hbar}\qty(qA_\mu(x)(\psi^\dagger(x)(\gamma^\mu)^\dagger\gamma^0\gamma^5\psi + \psi^\dagger(x)\gamma^0\gamma^5\gamma^\mu \psi(x)) + 2mc\psi^\dagger(x)\gamma^0\gamma^5\psi(x)) \\
                            & = -\frac{1}{i\hbar}\qty(qA_\mu(x)(\psi^\dagger(x)\gamma^0\gamma^\mu\gamma^5\psi - \psi^\dagger(x)\gamma^0\gamma^\mu\gamma^5 \psi(x)) + 2mc\psi^\dagger(x)\gamma^0\gamma^5\psi(x))           \\
                            & = -\frac{2mc}{i\hbar}\bar{\psi}(x)\gamma^5\psi(x)
  \end{align}
\end{proof}

\begin{problem}
中心力ポテンシャル $V(r)$ を持つハミルトニアン $\hat{H} = c\bm{\alpha}\vdot\hat{\pp} + \beta mc^2 + V(r)$ を考える.
\end{problem}
(1) $\hat{H}$ と軌道角運動量 $\hat{L} = \rr\cross\hat{\pp}$ との交換関係を求めよ。
\begin{proof}
  \begin{align}
    [\hat{H}, \hat{L}^i] & = [c\bm{\alpha}\vdot\hat{\pp} + \beta mc^2 + V(r), \varepsilon^{ijk}r^j\hat{p}^k]                                               \\
                         & = \varepsilon^{ijk}\qty(c[\alpha^\mu\hat{p}^\mu, r^j\hat{p}^k] + mc^2[\beta, r^j\hat{p}^k] + [V(r), r^j\hat{p}^k])              \\
                         & = \varepsilon^{ijk}\qty(c\alpha^\mu(p^\mu r^jp^k - r^jp^kp^\mu) + mc^2(\beta r^jp^k - r^jp^k\beta) + (V(r)r^jp^k - r^jp^kV(r))) \\
                         & = \varepsilon^{ijk}\qty(c\alpha^\mu (-i\hbar\delta^{\mu j})p^k + 0 + r^j(-i\hbar\partial^kV(r)))                                \\
                         & = -i\hbar c\varepsilon^{ijk}\alpha^jp^k - i\hbar\qty(\rr\cross\dv{V}{r}\pdv{r}{\rr})_i                                          \\
                         & = -i\hbar c\varepsilon^{ijk}\alpha^jp^k - i\hbar\dv{V}{r}\qty(\rr\cross\frac{\rr}{r})_i                                         \\
                         & = -i\hbar c\varepsilon^{ijk}\alpha^jp^k
  \end{align}
\end{proof}
(2) $\Sigma^i := -\dfrac{i}{2}\sum_{j,k}\varepsilon^{ijk}\alpha^j\alpha^k$ とするとき、$\hat{H}$ とスピン角運動量 $\hat{\bm{S}} = \dfrac{\hbar}{2}\bm{\Sigma}$ との交換関係を求めよ。
\begin{proof}
  \begin{align}
    [\hat{H}, \hat{S}^i] & = \qty[c\bm{\alpha}\vdot\hat{\pp} + \beta mc^2 + V(r), -\frac{i\hbar}{4}\varepsilon^{ijk}\alpha^j\alpha^k]                                                          \\
                         & = -\frac{i\hbar}{4}\varepsilon^{ijk}\qty(c[\alpha^\mu, \alpha^j\alpha^k]p^\mu + mc^2[\beta, \alpha^j\alpha^k] + [V(r), \alpha^j\alpha^k])                           \\
                         & = -\frac{i\hbar}{4}\varepsilon^{ijk}\qty(c(\alpha^\mu\alpha^j\alpha^k - \alpha^j\alpha^k\alpha^\mu)p^\mu + mc^2(\beta\alpha^j\alpha^k - \alpha^j\alpha^k\beta) + 0) \\
                         & = -\frac{i\hbar}{4}\varepsilon^{ijk}\qty(c((-\alpha^j\alpha^\mu + 2\delta^{\mu j})\alpha^k - \alpha^j(-\alpha^\mu\alpha^k + 2\delta^{k\mu}))p^\mu + 0 + 0)          \\
                         & = -\frac{i\hbar c}{2}\varepsilon^{ijk}(\alpha^kp^j - \alpha^jp^k)                                                                                                   \\
                         & = i\hbar c\varepsilon^{ijk}\alpha^jp^k
  \end{align}
\end{proof}
(3) 全角運動量が保存量となることを示せ。
\begin{proof}
  \begin{align}
    [\hat{H}, \hat{J}^i] & = [\hat{H}, \hat{L}^i + \hat{S}^i] = 0
  \end{align}
\end{proof}

\begin{problem}
$\bm{\sigma}$ をパウリ行列として、任意のベクトル $\pp$ に対する 2 行 2 列の行列 $\bm{\sigma}\vdot\pp$ を考える。
\end{problem}
(1) $(\bm{\sigma}\vdot\pp)^2$ を求めよ。
\begin{proof}
  \begin{align}
    (\bm{\sigma}\vdot\pp)^2 & = (\sigma^ip^i)(\sigma^jp^j)^\dagger                    \\
                            & = p^ip^j\qty(\delta^{ij}I + i\varepsilon^{ijk}\sigma^k) \\
                            & = \pp^2
  \end{align}
\end{proof}
(2) $\trace(\bm{\sigma}\vdot\pp)$ を求めよ。
\begin{proof}
  $\pp = (p^1, p^2, p^3)$ とすると
  \begin{align}
    \trace(\bm{\sigma}\vdot\pp) & = \trace\begin{pmatrix}
                                            p^3        & p^1 - ip^2 \\
                                            p^1 + ip^2 & -p^3
                                          \end{pmatrix} = 0
  \end{align}
\end{proof}
(3) $\bm{\sigma}\vdot\pp$ の固有値を求めよ。
\begin{proof}
  $\bm{\sigma}\vdot\pp$ の固有値を $\lambda_1, \lambda_2$ とおくと
  \begin{align}
    \lambda_1 + \lambda_2     & = \trace(\bm{\sigma}\vdot\pp) = 0 \\
    \lambda_1^2 = \lambda_2^2 & = (\bm{\sigma}\vdot\pp)^2 = \pp^2
  \end{align}
  よって固有値は $\pm|\pp|$ である。
\end{proof}
(4) $\pp := |\pp|(\sin\theta\cos\phi, \sin\theta\sin\phi, \cos\theta)$ とするとき、固有ベクトルを求めよ。
\begin{proof}
  \begin{align}
    (\bm{\sigma}\vdot\pp \mp |\pp|I)\bm{v} & = |\pp|\begin{pmatrix}
                                                      \cos\theta \mp 1                         & \sin\theta\cos\phi - i\sin\theta\sin\phi \\
                                                      \sin\theta\cos\phi + i\sin\theta\sin\phi & -\cos\theta \mp 1
                                                    \end{pmatrix}\bm{v} \\
                                           & = |\pp|\begin{pmatrix}
                                                      \cos\theta \mp 1     & \sin\theta e^{-i\phi} \\
                                                      \sin\theta e^{i\phi} & -\cos\theta \mp 1
                                                    \end{pmatrix}\bm{v}
  \end{align}
  より固有値 $\pm|\pp|$ に対する固有ベクトルはそれぞれ次のベクトルの定数倍である。

  \begin{align}
    \bm{v} = \begin{pmatrix}
               \sin\theta e^{-i\phi} \\
               -\cos\theta \pm 1
             \end{pmatrix}
  \end{align}

\end{proof}

\end{document}
