\RequirePackage{plautopatch}
\documentclass[uplatex,dvipdfmx,a4paper,11pt]{jlreq}
\usepackage{bxpapersize}
\usepackage[utf8]{inputenc}
\usepackage{fontenc}
\usepackage{lmodern}
\usepackage{otf}
\usepackage{amsmath}
\usepackage{amssymb}
\usepackage{amsthm}
\usepackage{ascmac}
% \usepackage[hyphens]{url}
\usepackage{physics}
\usepackage{braket}
\usepackage{verbatimbox}
\usepackage{bm}
\usepackage{url}
% \usepackage[dvipdfmx,hiresbb,final]{graphicx}
\usepackage{hyperref}
\usepackage{pxjahyper}
\usepackage{tikz}\usetikzlibrary{cd}
\usepackage{listings}
\usepackage{color}
\usepackage{mathtools}
\usepackage{xspace}
\usepackage{xy}
\usepackage{xypic}
%
\title{量子力学}
\author{Anko}
\makeatletter
%
\DeclareMathOperator{\lcm}{lcm}
\DeclareMathOperator{\Kernel}{Ker}
\DeclareMathOperator{\Image}{Im}
\DeclareMathOperator{\ch}{ch}
\DeclareMathOperator{\Aut}{Aut}
\DeclareMathOperator{\Log}{Log}
\DeclareMathOperator{\Arg}{Arg}
\DeclareMathOperator{\sgn}{sgn}
%
\newcommand{\CC}{\mathbb{C}}
\newcommand{\RR}{\mathbb{R}}
\newcommand{\QQ}{\mathbb{Q}}
\newcommand{\ZZ}{\mathbb{Z}}
\newcommand{\NN}{\mathbb{N}}
\newcommand{\FF}{\mathbb{F}}
\newcommand{\PP}{\mathbb{P}}
\newcommand{\GG}{\mathbb{G}}
\newcommand{\TT}{\mathbb{T}}
\newcommand{\calB}{\mathcal{B}}
\newcommand{\calF}{\mathcal{F}}
\newcommand{\ignore}[1]{}
\newcommand{\floor}[1]{\left\lfloor #1 \right\rfloor}
% \newcommand{\abs}[1]{\left\lvert #1 \right\rvert}
\newcommand{\lt}{<}
\newcommand{\gt}{>}
\newcommand{\id}{\mathrm{id}}
\newcommand{\rot}{\curl}
\renewcommand{\angle}[1]{\left\langle #1 \right\rangle}
\newcommand{\EE}{\bm{E}}
\newcommand{\BB}{\bm{B}}
\renewcommand{\AA}{\bm{A}}
\newcommand{\rr}{\bm{r}}
\newcommand{\kk}{\bm{k}}
\newcommand{\pp}{\bm{p}}

\let\oldcite=\cite
\renewcommand\cite[1]{\hyperlink{#1}{\oldcite{#1}}}

\let\oldbibitem=\bibitem
\renewcommand{\bibitem}[2][]{\label{#2}\oldbibitem[#1]{#2}}

% theorem環境の設定
% - 冒頭に改行
% - 末尾にdiamond (amsthm)
\theoremstyle{definition}
\newcommand*{\newscreentheoremx}[2]{
  \newenvironment{#1}[1][]{
    \begin{screen}
    \begin{#2}[##1]
      \leavevmode
      \newline
  }{
    \end{#2}
    \end{screen}
  }
}
\newcommand*{\newqedtheoremx}[2]{
  \newenvironment{#1}[1][]{
    \begin{#2}[##1]
      \leavevmode
      \newline
      \renewcommand{\qedsymbol}{\(\diamond\)}
      \pushQED{\qed}
  }{
      \qedhere
      \popQED
    \end{#2}
  }
}
\newtheorem{theorem*}{定理}

\newqedtheoremx{theorem}{theorem*}
\newcommand*\newqedtheorem@unstarred[2]{%
  \newtheorem{#1*}[theorem*]{#2}
  \newqedtheoremx{#1}{#1*}
}
\newcommand*\newqedtheorem@starred[2]{%
  \newtheorem*{#1*}{#2}
  \newqedtheoremx{#1}{#1*}
}
\newcommand*{\newqedtheorem}{\@ifstar{\newqedtheorem@starred}{\newqedtheorem@unstarred}}

\newtheorem{sctheorem*}{定理}
\newscreentheoremx{sctheorem}{sctheorem*}
\newcommand*\newscreentheorem@unstarred[2]{%
  \newtheorem{#1*}[theorem*]{#2}
  \newscreentheoremx{#1}{#1*}
}
\newcommand*\newscreentheorem@starred[2]{%
  \newtheorem*{#1*}{#2}
  \newscreentheoremx{#1}{#1*}
}
\newcommand*{\newscreentheorem}{\@ifstar{\newscreentheorem@starred}{\newscreentheorem@unstarred}}

%\newtheorem*{definition}{定義}
%\newtheorem{theorem}{定理}
%\newtheorem{proposition}[theorem]{命題}
%\newtheorem{lemma}[theorem]{補題}
%\newtheorem{corollary}[theorem]{系}

\newqedtheorem{lemma}{補題}
\newqedtheorem{corollary}{系}
\newqedtheorem{example}{例}
\newqedtheorem{proposition}{命題}
\newqedtheorem{remark}{注意}
\newqedtheorem{thesis}{主張}
\newqedtheorem{notation}{記法}
\newqedtheorem{problem}{問題}
\newqedtheorem{algorithm}{アルゴリズム}

\newscreentheorem*{axiom}{公理}
\newscreentheorem*{definition}{定義}

\renewenvironment{proof}[1][\proofname]{\par
  \normalfont
  \topsep6\p@\@plus6\p@ \trivlist
  \item[\hskip\labelsep{\bfseries #1}\@addpunct{\bfseries}]\ignorespaces\quad\par
}{%
  \qed\endtrivlist\@endpefalse
}
\renewcommand\proofname{証明}

\makeatother

\begin{document}
\maketitle
\tableofcontents
\clearpage

\subsection{電磁場中の荷電粒子}
スカラーポテンシャル $\phi$, ベクトルポテンシャル $\bm{A}$ の中での電荷 $q$ を持つ粒子の運動は次の置き換えで記述できる。
\begin{align}
  H      & \mapsto H - q\phi(\rr, t)        \\
  \bm{p} & \mapsto \bm{p} - q\bm{A}(\rr, t)
\end{align}
一様な磁場 $\bm{B}$ の場合, ベクトルポテンシャルは $\bm{A} = \dfrac{1}{2}\bm{B}\cross\rr$ (対称ゲージ) と置くことができるので
\begin{align}
  \frac{1}{2m}(\bm{p} - q\bm{A}) = \frac{\bm{p}^2}{2m} - \frac{q}{2m}\bm{B}\vdot\bm{L} + \frac{q^2}{8m}(\bm{B}^2\rr^2 - (\bm{B}\vdot\rr)^2)
\end{align}
$\sigma_i\sigma_j = \delta_{ij} + i\epsilon_{ijk}\sigma_k$ となるので
\begin{align}
  H & = \frac{(\bm{\sigma}\vdot\bm{p})^2}{2m}                                                                         \\
    & = \frac{(\bm{\sigma}\vdot\bm{p - q\bm{A}})^2}{2m}                                                               \\
    & = \frac{1}{2m}(\bm{p - q\bm{A}})^2 - 2\frac{q\hbar}{2m}\bm{B}\vdot\bm{s} & \qty(\bm{s} = \frac{\bm{\sigma}}{2})
\end{align}
となる。
ゼーマン相互作用

\section{相対論的量子力学}
\begin{theorem}
  \begin{align}
    L           & = \frac{m}{2}\dot{\rr}^2 - q\phi(t, \rr) + q\dot{\rr}\vdot\AA(t, \rr) \\
    m\ddot{\rr} & = q\EE(t, \rr) + q\dot{\rr}\cross\BB(t, \rr)                          \\
    \pp         & = m\dot{\rr} + q\AA(t, \rr)                                           \\
    H           & = \frac{(\pp - q\AA(t, \rr))^2}{2m} + q\phi(t, \rr)
  \end{align}
\end{theorem}
\begin{proof}
  \begin{align}
    \pdv{t}(\pdv{L}{\dot{\rr}}) - \pdv{L}{\rr} & = 0                                                                         \\
    m\ddot{\rr}                                & = q\EE(t, \rr) + q\dot{\rr}\cross\BB(t, \rr)                                \\
    \pp                                        & = \pdv{L}{\dot{\rr}} = m\dot{\rr} + q\AA(t, \rr)                            \\
    H                                          & = \pp\vdot\dot{\rr} - L = \frac{(\pp - q\AA(t, \rr))^2}{2m} + q\phi(t, \rr)
  \end{align}
\end{proof}
\begin{align}
  i\hbar\pdv{t}\psi(t, \rr)                       & = \qty(\frac{(\hat{\pp} - q\AA(t, \rr))^2}{2m} + q\phi(t, \rr))\psi(t, \rr) \\
  \qty(i\hbar\pdv{t} - q\phi(t, \rr))\psi(t, \rr) & = \frac{(-i\hbar\vnabla - q\AA(t, \rr))^2}{2m}
\end{align}
\begin{align}
  i\hbar\pdv{t}  & \to i\hbar\pdv{t} - q\phi(t, \rr) \\
  -i\hbar\vnabla & \to -i\hbar\vnabla - q\AA(t, \rr)
\end{align}
\begin{theorem}
  \begin{align}
    \pdv{\rho(t,\rr)}{t} + \vnabla\vdot\bm{j}(t,\rr) = 0
  \end{align}
\end{theorem}
\begin{proof}
  \begin{align}
    i\hbar\pdv{t}\psi(t, \rr) & = \qty(\frac{(\hat{\pp} - q\AA(t, \rr))^2}{2m} + q\phi(t, \rr))\psi(t, \rr) \\
  \end{align}
\end{proof}
\end{document}