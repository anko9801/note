\RequirePackage{plautopatch}
\documentclass[uplatex,dvipdfmx,a4paper,11pt]{jlreq}
\usepackage{bxpapersize}
\usepackage[utf8]{inputenc}
\usepackage{fontenc}
\usepackage{lmodern}
\usepackage{otf}
\usepackage{amsmath}
\usepackage{amssymb}
\usepackage{amsthm}
\usepackage{ascmac}
\usepackage{physics2}
\usephysicsmodule{ab, ab.braket, doubleprod, diagmat, xmat}
\usepackage{diffcoeff}
\usepackage{bm}
\usepackage{url}
% \usepackage[dvipdfmx,hiresbb,final]{graphicx}
\usepackage{hyperref}
\usepackage{pxjahyper}
\usepackage{tikz}\usetikzlibrary{cd}
\usepackage{color}
\usepackage{mathtools}
\usepackage{xspace}
\usepackage{xy}
\usepackage{xypic}

%
\title{相対論的量子力学}
\author{anko9801}
\makeatletter
%
% theorem環境の設定
% - 冒頭に改行
% - 末尾にdiamond (amsthm)
\theoremstyle{definition}
\newcommand*{\newscreentheoremx}[2]{
  \newenvironment{#1}[1][]{
    \begin{screen}
    \begin{#2}[##1]
      \leavevmode
      \newline
  }{
    \end{#2}
    \end{screen}
  }
}
\newcommand*{\newqedtheoremx}[2]{
  \newenvironment{#1}[1][]{
    \begin{#2}[##1]
      \leavevmode
      \newline
      \renewcommand{\qedsymbol}{\(\diamond\)}
      \pushQED{\qed}
  }{
      \qedhere
      \popQED
    \end{#2}
  }
}
\newtheorem{theorem*}{定理}[section]

\newqedtheoremx{theorem}{theorem*}
\newcommand*\newqedtheorem@unstarred[2]{%
  \newtheorem{#1*}[theorem*]{#2}
  \newqedtheoremx{#1}{#1*}
}
\newcommand*\newqedtheorem@starred[2]{%
  \newtheorem*{#1*}{#2}
  \newqedtheoremx{#1}{#1*}
}
\newcommand*{\newqedtheorem}{\@ifstar{\newqedtheorem@starred}{\newqedtheorem@unstarred}}

\newtheorem{sctheorem*}{定理}[section]
\newscreentheoremx{sctheorem}{sctheorem*}
\newcommand*\newscreentheorem@unstarred[2]{%
  \newtheorem{#1*}[theorem*]{#2}
  \newscreentheoremx{#1}{#1*}
}
\newcommand*\newscreentheorem@starred[2]{%
  \newtheorem*{#1*}{#2}
  \newscreentheoremx{#1}{#1*}
}
\newcommand*{\newscreentheorem}{\@ifstar{\newscreentheorem@starred}{\newscreentheorem@unstarred}}

%\newtheorem*{definition}{定義}
%\newtheorem{theorem}{定理}
%\newtheorem{proposition}[theorem]{命題}
%\newtheorem{lemma}[theorem]{補題}
%\newtheorem{corollary}[theorem]{系}

\newqedtheorem{lemma}{補題}
\newqedtheorem{corollary}{系}
\newqedtheorem{example}{例}
\newqedtheorem{proposition}{命題}
\newqedtheorem{remark}{注意}
\newqedtheorem{thesis}{主張}
\newqedtheorem{notation}{記法}
\newqedtheorem{problem}{問題}
\newqedtheorem{algorithm}{アルゴリズム}

\newscreentheorem*{axiom}{公理}
\newscreentheorem*{definition}{定義}

\renewenvironment{proof}[1][\proofname]{\par
  \normalfont
  \topsep6\p@\@plus6\p@ \trivlist
  \item[\hskip\labelsep{\bfseries #1}\@addpunct{\bfseries}]\ignorespaces\quad\par
}{%
  \qed\endtrivlist\@endpefalse
}
\renewcommand\proofname{証明}
%
\csname endofdump\endcsname
\DeclareMathOperator{\lcm}{lcm}
\DeclareMathOperator{\Kernel}{Ker}
\DeclareMathOperator{\Image}{Im}
\DeclareMathOperator{\ch}{ch}
\DeclareMathOperator{\Aut}{Aut}
\DeclareMathOperator{\Log}{Log}
\DeclareMathOperator{\Arg}{Arg}
\DeclareMathOperator{\sgn}{sgn}
%
\newcommand{\CC}{\mathbb{C}}
\newcommand{\RR}{\mathbb{R}}
\newcommand{\QQ}{\mathbb{Q}}
\newcommand{\ZZ}{\mathbb{Z}}
\newcommand{\NN}{\mathbb{N}}
\newcommand{\FF}{\mathbb{F}}
\newcommand{\PP}{\mathbb{P}}
\newcommand{\GG}{\mathbb{G}}
\newcommand{\TT}{\mathbb{T}}
\newcommand{\EE}{\bm{E}}
\newcommand{\BB}{\bm{B}}
\renewcommand{\AA}{\hat{A}}
\renewcommand{\BB}{\hat{B}}
\renewcommand{\AA}{\bm{A}}
\newcommand{\rr}{\bm{r}}
\newcommand{\kk}{\bm{k}}
\newcommand{\pp}{\bm{p}}
\newcommand{\calB}{\mathcal{B}}
\newcommand{\calF}{\mathcal{F}}
\newcommand{\vnabla}{\mathbf{\nabla}}
\newcommand{\laplacian}{\nabla^2}
\newcommand{\ignore}[1]{}
\newcommand{\floor}[1]{\left\lfloor #1 \right\rfloor}
% \newcommand{\abs}[1]{\left\lvert #1 \right\rvert}
\newcommand{\lt}{<}
\newcommand{\gt}{>}
\newcommand{\id}{\mathrm{id}}
\newcommand{\rot}{\curl}
\renewcommand{\angle}[1]{\left\langle #1 \right\rangle}
\newcommand\mab[1]{\begin{pmatrix}#1\end{pmatrix}}
\newcommand\vmab[1]{\begin{vmatrix}#1\end{vmatrix}}
\numberwithin{equation}{section}

\let\oldcite=\cite
\renewcommand\cite[1]{\hyperlink{#1}{\oldcite{#1}}}

\let\oldbibitem=\bibitem
\renewcommand{\bibitem}[2][]{\label{#2}\oldbibitem[#1]{#2}}

\makeatother

\begin{document}
\maketitle
\tableofcontents
\clearpage

\subsection{電磁場中の荷電粒子}
スカラーポテンシャル $\phi$, ベクトルポテンシャル $\bm{A}$ の中での電荷 $q$ を持つ粒子の運動は次の置き換えで記述できる。
\begin{align}
  H      & \mapsto H - q\phi(\rr, t)        \\
  \bm{p} & \mapsto \bm{p} - q\bm{A}(\rr, t)
\end{align}
一様な磁場 $\bm{B}$ の場合, ベクトルポテンシャルは $\bm{A} = \dfrac{1}{2}\bm{B}\times\rr$ (対称ゲージ) と置くことができるので
\begin{align}
  \frac{1}{2m}(\bm{p} - q\bm{A}) = \frac{\bm{p}^2}{2m} - \frac{q}{2m}\bm{B}\cdot\bm{L} + \frac{q^2}{8m}(\bm{B}^2\rr^2 - (\bm{B}\cdot\rr)^2)
\end{align}
$\sigma_i\sigma_j = \delta_{ij} + i\epsilon_{ijk}\sigma_k$ となるので
\begin{align}
  H & = \frac{(\bm{\sigma}\cdot\bm{p})^2}{2m}                                                                        \\
    & = \frac{(\bm{\sigma}\cdot\bm{p - q\bm{A}})^2}{2m}                                                              \\
    & = \frac{1}{2m}(\bm{p - q\bm{A}})^2 - 2\frac{q\hbar}{2m}\bm{B}\cdot\bm{s} & \ab(\bm{s} = \frac{\bm{\sigma}}{2})
\end{align}
となる。
ゼーマン相互作用

\section{相対論的波動方程式}
\subsection{クライン・ゴルドン方程式}
相対論的力学におけるエネルギーと運動量の関係に基づいてローレンツ変換のもとで共変となる相対論的波動方程式を構築する。
\begin{align}
  i\hbar\diffp{}{t}\psi(t, \rr) = -\frac{\hbar^2}{2m}\nabla^2\psi(t, \rr)
\end{align}
相対論的力学における自由粒子のエネルギーと運動量の関係 $E = \pm\sqrt{(mc^2)^2 + (c\pp)^2}$ より
\begin{align}
  i\hbar\diffp{}{t}\psi(x) & = \pm\sqrt{(mc^2)^2 + (c\hat{\pp})^2}\psi(x)
\end{align}
となるが
\begin{enumerate}
  \item 時間微分と空間微分について非対称であり, ローレンツ変換のもとでの共変性が見えない。
  \item 実際に, 光速よりも速く情報が伝播しないという, 相対論的な因果律を破る。
  \item 空間微分が平方根の中に入っているため連続の方程式を導くことができず, 波動関数の確率解釈ができない。
\end{enumerate}
上記の波動方程式では, 時間に関して 1 階微分, 空間に関して 2 階微分が平方根の中に入っている. この非対称性を解消するため, 時間微分を両辺に作用させると
\begin{align}
  p^\mu & = \ab(\frac{E}{c}, \pp) = \ab(i\hbar\frac{1}{c}\diffp{}{t}, -i\hbar\nabla) = i\hbar\partial^\mu
\end{align}
\begin{align}
  (mc^2)^2 = E^2 - (c\pp)^2 & = c^2p_\mu p^\mu = -(\hbar c)^2\partial_\mu\partial^\mu
\end{align}
\begin{align}
  \ab[\partial_\mu\partial^\mu + \ab(\frac{mc}{\hbar})^2]\psi(x) = 0
\end{align}
を得る。これをクライン・ゴルドン方程式と呼ぶ。
\begin{definition}[クライン・ゴルドン方程式]
  \begin{align}
    \ab[\partial_\mu\partial^\mu + \ab(\frac{mc}{\hbar})^2]\psi(x) = 0
  \end{align}
\end{definition}
クライン・ゴルドン方程式から
\begin{align}
  \psi^*(x)\ab[\partial_\mu\partial^\mu + \ab(\frac{mc}{\hbar})^2]\psi(x) = 0 \\
  \psi(x)\ab[\partial_\mu\partial^\mu + \ab(\frac{mc}{\hbar})^2]\psi^*(x) = 0
\end{align}
が得られるので両辺の差を取って
\begin{align}
   & \psi^*(x)\partial_\mu\partial^\mu\psi(x) - \psi(x)\partial_\mu\partial^\mu\psi^*(x) = 0 \\
   & \partial_\mu[\psi^*(x)\partial^\mu\psi(x) - \psi(x)\partial^\mu\psi^*(x)] = 0           \\
   & \partial_\mu j^\mu(x) = 0
\end{align}
連続の方程式が成り立つ。
\begin{align}
  j^\mu(x) := \psi^*(x)\partial^\mu\psi(x) - \psi(x)\partial^\mu\psi^*(x)
\end{align}
これより保存則が成り立つ。
となる。ただし $\rho(x) = j^0(x)/c$ は非負とは限らないため粒子の存在確率密度と解釈することはできない。

クライン・ゴルドン方程式では時間に関して 2 階微分を含むため、$j^0(x)$ に時間微分が残り、確率解釈ができなかった。時間に関して 1 階微分のみを含む相対論的波動方程式を構築したい。また共変性を満足するためには空間に関しても 1 階微分のみを含む必要がある。そこで

\subsection{ディラック方程式}
確率解釈できる相対論的波動方程式を構築する。

次の形となることを仮定する。
ただし、$\alpha^i, \beta$ は次を満たす無次元の未知係数である。
これをディラック方程式という。
\begin{align}
  i\hbar\diffp{}{t}\psi(x) & = \ab(c\alpha^ip_i + \beta mc^2)\psi(x)
\end{align}
これがクライン・ゴルドン方程式を満たすことから
\begin{align}
  \ab(i\hbar\diffp{}{t})^2\psi(x) & = \ab(c\alpha^ip_i + \beta mc^2)^2\psi(x)                                                          \\
                                  & = \ab[c^2\alpha^ip_i\alpha^jp_j + \beta^2(mc^2)^2 + (\alpha^i\beta + \beta\alpha^i)p_i(mc)]\psi(x) \\
                                  & = [(c\pp)^2 + (mc^2)^2]\psi(x)
\end{align}
係数を比較することによって $\alpha^i$ と $\beta$ は次を満たすエルミート行列であることがわかる。
\begin{align}
  \lbrace\alpha^i,\alpha^j\rbrace = 2\delta^{ij}, \qquad \beta^2 = 1, \qquad \lbrace\alpha^i,\beta\rbrace = 0 \label{condition}
\end{align}
さらにガンマ行列 $\gamma^\mu := (\beta, \beta\bm{\alpha})$ を定義する。
両辺に左から $\beta/c$ を掛けてディラック方程式は次のようになる。
\begin{align}
   & i\hbar\gamma^0\partial_0\psi(x) = \ab(-i\hbar\gamma^i\partial_i + mc)\psi(x) \\
   & \ab(i\hbar\gamma^\mu\partial_\mu - mc)\psi(x) = 0
\end{align}
ガンマ行列の同値な条件は次のようになる。
\begin{align}
  (\gamma^0)^\dagger                   & = \beta^\dagger = \beta = \gamma^0                                                                                        \\
  (\gamma^i)^\dagger                   & = (\beta\alpha^i)^\dagger = \alpha^i\beta = -\beta\alpha^i = -\gamma^i                                                    \\
  \lbrace\gamma^\mu, \gamma^\nu\rbrace & =
  \begin{dcases}
    \beta\alpha^\mu\beta\alpha^\nu + \beta\alpha^\nu\beta\alpha^\mu = -\beta^2(\alpha^\mu\alpha^\nu + \alpha^\nu\alpha^\mu) = -2\delta^{\mu\nu} & (\mu > 0, \nu > 0) \\
    \beta\beta\alpha^\nu + \beta\alpha^\nu\beta = \beta^2\alpha^\nu - \beta^2\alpha^\nu = 0                                                     & (\mu = 0, \nu > 0) \\
    2\beta^2 = 2                                                                                                                                & (\mu = \nu = 0)
  \end{dcases} \\
                                       & = 2g^{\mu\nu}
\end{align}
より $\gamma^0$ はエルミート行列で $\gamma^i$ は反エルミート行列である。$\lbrace\gamma^\mu, \gamma^\nu\rbrace = \lbrace\gamma^\nu, \gamma^\mu\rbrace$ より $\mu \leq \nu$ のときを示せばよい。

\begin{definition}[ディラック方程式]
  \begin{align}
     & (i\hbar\gamma^\mu \partial_\mu - mc)\psi(x) = 0
  \end{align}
  \begin{align}
     & (\gamma^0)^\dagger = \gamma^0, \qquad (\gamma^i)^\dagger = -\gamma^i, \qquad \lbrace\gamma^\mu, \gamma^\nu\rbrace = 2g^{\mu\nu}
  \end{align}
\end{definition}
式 \eqref{condition} を満たす行列は例えばディラック表示がある。
\begin{align}
  \alpha^i = \sigma^i \otimes \sigma^1 = \begin{pmatrix}
                                           0        & \sigma^i \\
                                           \sigma^i & 0
                                         \end{pmatrix}, \qquad
  \beta = \sigma^0 \otimes \sigma^3 = \begin{pmatrix}
                                        \sigma^0 & 0         \\
                                        0        & -\sigma^0
                                      \end{pmatrix}
\end{align}
ただしパウリ行列 $\bm{\sigma}$ は次のように定義される。
\begin{align}
  \sigma_0 = \begin{pmatrix}
               1 & 0 \\
               0 & 1
             \end{pmatrix}, \qquad
  \sigma_1 = \begin{pmatrix}
               0 & 1 \\
               1 & 0
             \end{pmatrix}, \qquad
  \sigma_2 = \begin{pmatrix}
               0 & -i \\
               i & 0
             \end{pmatrix}, \qquad
  \sigma_3 = \begin{pmatrix}
               1 & 0  \\
               0 & -1
             \end{pmatrix}
\end{align}
パウリ行列は Hermite 行列 $\sigma_i^\dagger = \sigma_i$ かつユニタリ行列 $\sigma_i^\dagger\sigma_i = \sigma_i\sigma_i^\dagger = I$ で次の性質を満たす。
\begin{align}
  \sigma_i\sigma_j & = \delta_{ij}I + i\sum_{k}\varepsilon_{ijk}\sigma_k
\end{align}
これより
\begin{align}
  (\alpha^i)^\dagger     & = \alpha^i, \qquad \beta^\dagger = \beta, \qquad \beta^2 = (\sigma^0\otimes\sigma^3)^2 = 1                                       \\
  \{\alpha^i, \alpha^j\} & = \{\sigma^i\otimes\sigma^1, \sigma^j\otimes\sigma^1\} = (\sigma^i\sigma^j + \sigma^j\sigma^i)\otimes(\sigma^1)^2 = 2\delta^{ij} \\
  \{\alpha^i, \beta\}    & = \{\sigma^i\otimes\sigma^1, \sigma^0\otimes\sigma^3\} = (\sigma^i\sigma^0 - \sigma^0\sigma^i)\otimes(\sigma^1\sigma^3) = 0
\end{align}

\begin{align}
  i\hbar\diffp{}{t}\psi(x) & = \ab(c\bm{\alpha}\cdot(\pp - q\AA(x)) + \beta mc^2 + q\phi(x))\psi(x) \\
                           & = \begin{pmatrix}
                                 +mc^2 + q\phi(x)                 & c\bm{\sigma}\cdot[\pp - q\AA(x)] \\
                                 c\bm{\sigma}\cdot[\pp - q\AA(x)] & -mc^2 + q\phi(x)
                               \end{pmatrix}
  \begin{pmatrix}
    \psi_+(x) \\
    \psi_-(x)
  \end{pmatrix}
\end{align}

\subsection{ディラック方程式の共変性}
\begin{align}
  A_\mu(t, \rr) = \ab(\frac{\phi(t, \rr)}{c}, \bm{A}(t, \rr))
\end{align}
共変微分 $D_\mu$ を次のように定義する。
\begin{align}
  D_\mu := \partial_\mu - \frac{q}{i\hbar}A_\mu(x) = \ab(\frac{1}{c}\diffp{}{t} - \frac{q}{i\hbar}\frac{\phi(t, \rr)}{c} + \frac{q}{i\hbar}\bm{A}(t, \rr))
\end{align}
\begin{align}
  (i\hbar\gamma^\mu D_\mu - mc)\psi(x) = 0
\end{align}
まず次の式が成り立つことを確認する。
\begin{align}
  D_\mu D_\nu^\dagger & = \ab(\partial_\mu - \frac{q}{i\hbar}A_\mu(x))\ab(\partial_\nu + \frac{q}{i\hbar}A_\nu(x))                                                      \\
                      & = \partial_\mu\partial_\nu + \frac{q}{i\hbar}\partial_\mu A_\nu(x) - \frac{q}{i\hbar}A_\mu(x)\partial_\nu + \frac{q^2}{\hbar^2}A_\mu(x)A_\nu(x)
\end{align}
よって次のように示せる。
\begin{align}
  (\bm{D}\cdot\bm{\sigma})^2 & = \sum_{i,j}(D_i\sigma^i)(D_j\sigma^j)^\dagger                                      \\
                             & = \sum_{i,j}\ab(\delta^{ij}I + i\sum_{k}\varepsilon^{ijk}\sigma^k)D_iD_j^\dagger    \\
                             & = \sum_{i}D_i^2 + i\sum_{i,j,k}\varepsilon^{ijk}\sigma^kD_iD_j^\dagger              \\
                             & = \bm{D}^2 + i\sum_{i,j,k}\varepsilon^{ijk}\sigma^k\frac{q}{i\hbar}\partial_iA_j(x) \\
                             & = \bm{D}^2 + \frac{q}{\hbar}\bm{\sigma}\cdot(\vnabla\times\bm{A}(x))                \\
                             & = \bm{D}^2 + \frac{q}{\hbar}\bm{B}(x)\cdot\bm{\sigma}
\end{align}

慣性系 $x^\mu$ と共変微分 $D_\mu$ におけるローレンツ共変性は次のようになる。
\begin{align}
  x'^\mu & = \Lambda^\mu_{\ \nu}x^\nu \\
  D'_\mu & = \Lambda_\mu^{\ \nu}D_\nu
\end{align}
波動関数 $\psi(x)$ のローレンツ共変性
\begin{align}
  \psi'(x') & = S_\Lambda\psi(x)
\end{align}
このときディラック方程式は次のように変換でき、共変性を満たす。
$S_\Lambda^{-1}\gamma^\mu S_\Lambda = \Lambda_\nu^\mu\gamma^\nu$ を満たすように定めた $S_\Lambda$
\begin{align}
  (i\hbar\gamma^\mu D'_\mu - mc)\psi'(x') & = (i\hbar\gamma^\mu \Lambda^\nu_\mu D_\mu - mc)S_\Lambda\psi(x)                           \\
                                          & = S_\Lambda(i\hbar S_\Lambda^{-1}\gamma^\mu S_\Lambda\Lambda_\mu^\nu D_\mu - mc)\psi(x)   \\
                                          & = S_\Lambda(i\hbar\Lambda_{\lambda}^{\mu}\gamma^\lambda\Lambda_\mu^\nu D_\mu - mc)\psi(x) \\
                                          & = S_\Lambda(i\hbar\delta_{\lambda}^\nu\gamma^\lambda D_\mu - mc)\psi(x)                   \\
                                          & = S_\Lambda(i\hbar\gamma^\nu D_\mu - mc)\psi(x) = 0
\end{align}

\subsection{無限小変換}
無限小変換を考える。
\begin{align}
  \Lambda_{\ \nu}^\mu = \delta_\nu^\mu + \omega_{\ \nu}^\mu
\end{align}
\begin{align}
  \omega_{\ \nu}^\mu & = \begin{pmatrix}
                           0             & -\Delta\eta_x   & -\Delta\eta_y   & -\Delta\eta_z   \\
                           -\Delta\eta_x & 0               & \Delta\theta_z  & -\Delta\theta_y \\
                           -\Delta\eta_y & -\Delta\theta_z & 0               & \Delta\theta_x  \\
                           -\Delta\eta_z & \Delta\theta_y  & -\Delta\theta_x & 0
                         \end{pmatrix}
\end{align}
\begin{align}
  \Lambda_{\ \nu}^\mu\Lambda^{\ \lambda}_\mu & = \delta_\nu^\lambda                                                                                                                                        \\
  \Lambda_{\ \nu}^\mu\Lambda^{\ \lambda}_\mu & = (\delta_\nu^\mu + \omega_{\ \nu}^\mu)(\delta^\lambda_\mu + \omega^{\ \lambda}_\mu) = \delta^\lambda_\nu + \omega^{\ \lambda}_\nu + \omega_{\ \nu}^\lambda
\end{align}
$\omega_{\nu\lambda} = -\omega_{\lambda\nu}$ と反対称となる。
そして $S_\Lambda$ を次のように定義する。
\begin{align}
  S_\Lambda = 1 + \omega_{\mu\nu}\Gamma^{\mu\nu}, \qquad S_\Lambda^{-1} = 1 - \omega_{\mu\nu}\Gamma^{\mu\nu}
\end{align}
対称と反対称だと $0$ になるから $\Gamma^{\mu\nu}$ は反対称としても構わないが今回は $\Gamma^{\mu\nu} = c\gamma^\mu\gamma^\nu$ とする。
\begin{align}
  S_\Lambda^{-1}\gamma^\mu S_\Lambda                                                                                           & = \Lambda^\mu_{\ \nu}\gamma^\nu                                                                                                                                                               \\
  (1 - \omega_{\kappa\lambda}\Gamma^{\kappa\lambda})\gamma^\mu(1 + \omega_{\kappa\lambda}\Gamma^{\kappa\lambda})               & = (\delta^\mu_\nu + \omega^\mu_{\ \nu})\gamma^\nu                                                                                                                                             \\
  \gamma^\mu - \omega_{\kappa\lambda}\Gamma^{\kappa\lambda}\gamma^\mu + \omega_{\kappa\lambda}\gamma^\mu\Gamma^{\kappa\lambda} & = \gamma^\mu + \omega^\mu_{\ \nu}\gamma^\nu                                                                                                                                                   \\
  \omega_{\kappa\lambda}\gamma^\mu\Gamma^{\kappa\lambda} - \omega_{\kappa\lambda}\Gamma^{\kappa\lambda}\gamma^\mu              & = \omega^\mu_{\ \nu}\gamma^\nu                                                                                                                                                                \\
  c\omega_{\kappa\lambda}\gamma^\mu\gamma^\kappa\gamma^\lambda - c\omega_{\kappa\lambda}\gamma^\kappa\gamma^\lambda\gamma^\mu  & = c\omega_{\kappa\lambda}\gamma^\mu\gamma^\kappa\gamma^\lambda - c\omega_{\kappa\lambda}(2g^{\mu\lambda}\gamma^\kappa - 2g^{\mu\kappa}\gamma^\lambda + \gamma^\mu\gamma^\kappa\gamma^\lambda) \\
                                                                                                                               & = c(-2\omega_{\kappa}^{\ \mu}\gamma^\kappa + 2\omega^\mu_{\ \lambda}\gamma^\lambda) = 4c\omega_{\ \nu}^{\mu}\gamma^\nu = \omega^\mu_{\ \nu}\gamma^\nu
\end{align}
\begin{align}
  S_\Lambda = 1 + \frac{1}{4}\omega_{\mu\nu}\gamma^\mu\gamma^\nu
\end{align}



\subsection{有限ローレンツ変換}
\begin{align}
  N\omega_{\mu\nu} & = Ng_{\mu\lambda}\omega_{\ \nu}^\lambda = \begin{pmatrix}
                                                                 0      & -\eta_x   & -\eta_y   & -\eta_z   \\
                                                                 \eta_x & 0         & -\theta_z & \theta_y  \\
                                                                 \eta_y & \theta_z  & 0         & -\theta_x \\
                                                                 \eta_z & -\theta_y & \theta_x  & 0
                                                               \end{pmatrix}
\end{align}
\begin{align}
  S_\Lambda & = \lim_{N\to\infty}[1 + \frac{\omega_{\mu\nu}}{4}\gamma^\mu\gamma^\nu]^N = \lim_{N\to\infty}\ab[\exp\ab(\frac{\omega_{\mu\nu}}{4}\gamma^\mu\gamma^\nu)]^N = \exp\ab(\frac{N\omega_{\mu\nu}}{4}\gamma^\mu\gamma^\nu)                \\
            & = \exp\ab(- \frac{\eta_x}{2}\gamma^0\gamma^1 - \frac{\eta_y}{2}\gamma^0\gamma^2 - \frac{\eta_z}{2}\gamma^0\gamma^3 - \frac{\theta_x}{2}\gamma^2\gamma^3 - \frac{\theta_y}{2}\gamma^3\gamma^1 - \frac{\theta_z}{2}\gamma^1\gamma^2) \\
            & = \exp\ab(- \frac{\eta_x}{2}\alpha^1 - \frac{\eta_y}{2}\alpha^2 - \frac{\eta_z}{2}\alpha^3 + \frac{\theta_x}{2}\alpha^2\alpha^3 + \frac{\theta_y}{2}\alpha^3\alpha^1 + \frac{\theta_z}{2}\alpha^1\alpha^2)                         \\
            & = \exp\ab(-\frac{\bm{\eta}}{2}\cdot\bm{\alpha} + i\frac{\bm{\theta}}{2}\cdot\bm{\Sigma})                                                                                                                                           \\
            & = \ab[\cosh\ab(-\frac{\bm{\eta}}{2}\cdot\bm{\alpha}) - \sinh\ab(-\frac{\bm{\eta}}{2}\cdot\bm{\alpha})] + \exp\ab(i\frac{\bm{\theta}}{2}\cdot\bm{\Sigma})
\end{align}
$\Sigma^i$ を次のように定義する。
$\bm{\Sigma} = (\Sigma^1, \Sigma^2, \Sigma^3)$ と任意の $\bm{v} = (v^1, v^2, v^3)$ に対して次が成り立つことを示せ。
\begin{align}
  \Sigma^i := -\frac{i}{2}\sum_{j,k = 1}^{3}\varepsilon^{ijk}\alpha^j\alpha^k = -i(\alpha^2\alpha^3, \alpha^3\alpha^1, \alpha^1\alpha^2)
\end{align}
\begin{align}
  (\Sigma^i)^\dagger & = -\frac{i}{2}\sum_{j,k}\varepsilon^{ijk}\gamma^k\gamma^j = \Sigma^i
\end{align}
\begin{align}
  \lbrace\Sigma^i, \Sigma^j\rbrace = 2\delta^{ij}
\end{align}
$i = j$ のとき、ある $k, l$ が存在して
\begin{align}
  \lbrace\Sigma^i, \Sigma^j\rbrace & = -\frac{1}{2}\ab(\sum_{a,b}\varepsilon^{iab}\gamma^a\gamma^b)^2                                                                                             \\
                                   & = -\frac{1}{2}\ab(\gamma^k\gamma^l - \gamma^l\gamma^k)^2                                                                                                     \\
                                   & = -\frac{1}{2}\ab(\gamma^k\gamma^l\gamma^k\gamma^l - \gamma^k\gamma^l\gamma^l\gamma^k - \gamma^l\gamma^k\gamma^k\gamma^l + \gamma^l\gamma^k\gamma^l\gamma^k) \\
                                   & = 2
\end{align}
$i \neq j$ のとき $a, b, c, d$ のいづれか 2 つ 1 組は同じであるから
\begin{align}
  \lbrace\Sigma^i, \Sigma^j\rbrace & = -\frac{1}{4}\sum_{a,b}\sum_{c,d}(\varepsilon^{iab}\gamma^a\gamma^b\varepsilon^{jcd}\gamma^c\gamma^d + \varepsilon^{jcd}\gamma^c\gamma^d\varepsilon^{iab}\gamma^a\gamma^b) \\
                                   & = -\frac{1}{4}\sum_{a,b}\sum_{c,d}\varepsilon^{iab}\varepsilon^{jcd}(1 + (-1)^3)\gamma^a\gamma^b\gamma^c\gamma^d                                                            \\
                                   & = 0
\end{align}

\begin{align}
  (\bm{v}\cdot\bm{\Sigma})^2 & = (v^i\Sigma^i)(v^j\Sigma^j)^\dagger = v^i\Sigma^i\Sigma^j(v^j)^\dagger = \delta^{ij}v^i(v^j)^\dagger = \bm{v}^2
\end{align}

\subsection{非相対論的極限}
ディラック表示のディラック方程式


非相対論的極限 $mc^2\to\infty$ において、シュレーディンガー方程式に帰着する。
荷電粒子に対するクライン・ゴルドン方程式
\begin{align}
  [\hbar^2D^2 + (mc)^2]\psi(x) = 0
\end{align}
\begin{align}
  \psi(x) = e^{-i(mc^2)t/\hbar}\varphi(x)
\end{align}
とおくと
\begin{align}
  \diffp{}{t}\psi(x) & = \frac{mc^2}{i\hbar}e^{-i(mc^2)t/\hbar}\varphi(x) = \frac{mc^2}{i\hbar}\psi(x)
\end{align}
\begin{align}
  D_\mu = \partial_\mu - \frac{q}{i\hbar}A_\mu(x) & = \frac{1}{i\hbar}\ab(p_\mu - qA_\mu(x)) = \frac{1}{i\hbar}\ab(\frac{i\hbar}{c}\diffp{}{t} - \frac{q}{c}\phi(x), \pp - q\AA(x)) \\
  D^2 = D_\mu D^\mu                               & = -\frac{1}{\hbar^2}\ab[\ab(\frac{i\hbar}{c}\diffp{}{t} - \frac{q}{c}\phi(x))^2 - \ab(\pp - q\AA(x))^2]                         \\
  [\hbar^2D^2 + (mc)^2]\psi(x)                    & = \ab[(\pp - q\AA(x))^2 - \ab(\frac{i\hbar}{c}\diffp{}{t} - \frac{q}{c}\phi(x))^2 + (mc)^2]\psi(x)                              \\
                                                  & = \ab[(\pp - q\AA(x))^2 - \ab(mc - \frac{q}{c}\phi(x))^2 + (mc)^2]\psi(x)                                                       \\
                                                  & = \ab[(\pp - q\AA(x))^2 + 2mq\phi(x) - \ab(\frac{q}{c}\phi(x))^2]\psi(x)                                                        \\
                                                  & = 2m\ab[-\frac{q^2\phi^2}{2mc^2} + \frac{(\pp - q\bm{A}(x))^2}{2m} + q\phi(x)]\psi(x)
\end{align}
非相対論的極限 $mc^2\to\infty$ のとき $\phi(x)$ は時間に依存しないからシュレーディンガー方程式となる。
\begin{align}
  i\hbar\diffp{}{t}\psi(x) & = \ab(\frac{(\pp - q\bm{A}(x))^2}{2m} + q\phi)\psi(x)
\end{align}



\subsection{双線型形式の変換性}
ローレンツ変換 $S_\Lambda$ の中で本義ローレンツ変換 $S_L$ と空間反転 $S_P$
\begin{align}
  S_L & = \exp\ab(-\frac{\bm{\eta}}{2}\cdot\bm{\alpha} + i\frac{\bm{\theta}}{2}\cdot\bm{\Sigma}) \\
  S_P & = \gamma^0
\end{align}
\begin{align}
  S_L^\dagger                 & = \exp\ab(-\frac{\bm{\eta}}{2}\cdot\bm{\alpha} - i\frac{\bm{\theta}}{2}\cdot\bm{\Sigma}) \\
  \gamma^0S_L^\dagger\gamma^0 & = \exp\ab(+\frac{\bm{\eta}}{2}\cdot\bm{\alpha} - i\frac{\bm{\theta}}{2}\cdot\bm{\Sigma}) \\
  S_L^{-1}                    & = \exp\ab(+\frac{\bm{\eta}}{2}\cdot\bm{\alpha} - i\frac{\bm{\theta}}{2}\cdot\bm{\Sigma})
\end{align}
より $S_L^\dagger S_L \neq 1$ となる。
\begin{align}
  \psi^{\prime\dagger}(x')\psi'(x') = \psi^\dagger(x)S_L^\dagger S_L\psi(x) \neq \psi^\dagger(x)\psi(x)
\end{align}
\begin{definition}
  \begin{align}
    \bar{\psi}(x) := \psi^\dagger(x)\gamma^0
  \end{align}
\end{definition}
\begin{align}
   & \bar{\psi}'(x') = \psi^{\prime\dagger}(x')\gamma^0 = \psi^\dagger(x)S_L^\dagger\gamma^0 = \psi^\dagger(x)\gamma^0S_L^{-1} = \bar{\psi}(x)S_L^{-1} \\
   & \bar{\psi}'(x')\psi'(x') = \psi^\dagger(x)S_L^{-1} S_L\psi(x) = \psi^\dagger(x)\psi(x)
\end{align}
カイラリティ $\gamma^5$ を次のように定義する。
\begin{align}
  \gamma^5 := i\gamma^0\gamma^1\gamma^2\gamma^3
\end{align}
\begin{align}
  (\gamma^5)^\dagger                 & = -i(-\gamma^3)(-\gamma^2)(-\gamma^1)\gamma^0 = i\gamma^3\gamma^2\gamma^1\gamma^0 = i\gamma^0\gamma^1\gamma^2\gamma^3 = \gamma^5                                        \\
  (\gamma^5)^2                       & = -(\gamma^0\gamma^1\gamma^2\gamma^3)(\gamma^0\gamma^1\gamma^2\gamma^3)^\dagger = - (\gamma^0)^2(\gamma^1)^2(\gamma^2)^2(\gamma^3)^2 = 1                                \\
  \lbrace\gamma^\mu, \gamma^5\rbrace & = i(\gamma^\mu\gamma^0\gamma^1\gamma^2\gamma^3 + \gamma^0\gamma^1\gamma^2\gamma^3\gamma^\mu) = i((-1)^\mu + (-1)^{3-\mu})\gamma^0\cdots(\gamma^\mu)^2\cdots\gamma^3 = 0
\end{align}

\begin{align}
  \bar\psi'(x)\gamma^5\psi'(x) & = \psi^\dagger(x)S_L^{-1}\gamma^5S_L\psi(x) = \psi^\dagger(x)\gamma^5\psi(x)  \\
  \bar\psi'(x)\gamma^5\psi'(x) & = \psi^\dagger(x)S_P^{-1}\gamma^5S_P\psi(x) = -\psi^\dagger(x)\gamma^5\psi(x)
\end{align}

荷電粒子に対するディラック方程式
\begin{align}
  (i\hbar\gamma^\mu D_\mu - mc)\psi(x) = 0
\end{align}
を用いて, 軸性ベクトル $j_A^\mu(x) := \bar{\psi}(x)\gamma^\mu\gamma^5\psi(x)$ の発散 $\partial_\mu j_A^\mu(x)$ を計算し、微分を含まない形で表せ。
まず次の式が成り立つことを確認する。
\begin{align}
  \gamma^0\gamma^\mu\gamma^5 & = \gamma^0\gamma^0\gamma^5 + \gamma^0\gamma^i\gamma^5 \\
                             & = \gamma^0\gamma^0\gamma^5 - \gamma^i\gamma^0\gamma^5 \\
                             & = (\gamma^\mu)^\dagger\gamma^0\gamma^5
\end{align}
よってディラック方程式より次のようになる。
\begin{align}
   & \partial_\mu j_A^\mu(x)                                                                                                                                                                    \\
   & = \partial_\mu(\bar{\psi}(x)\gamma^\mu\gamma^5\psi(x))                                                                                                                                     \\
   & = \partial_\mu(\psi^\dagger(x)\gamma^0\gamma^\mu\gamma^5\psi(x))                                                                                                                           \\
   & = (\partial_\mu\psi(x))^\dagger\gamma^0\gamma^\mu\gamma^5\psi(x) + \psi^\dagger(x)\gamma^0\gamma^\mu\gamma^5\partial_\mu\psi(x)                                                            \\
   & = (\gamma^\mu\partial_\mu\psi(x))^\dagger\gamma^0\gamma^5\psi(x) - \psi^\dagger(x)\gamma^0\gamma^5(\gamma^\mu\partial_\mu\psi(x))                                                          \\
   & = \ab(\frac{1}{i\hbar}\ab(\gamma^\mu qA_\mu(x) + mc)\psi(x))^\dagger\gamma^0\gamma^5\psi(x) - \psi^\dagger(x)\gamma^0\gamma^5\ab(\frac{1}{i\hbar}\ab(\gamma^\mu qA_\mu(x) + mc)\psi(x))    \\
   & = -\frac{1}{i\hbar}\ab(qA_\mu(x)(\psi^\dagger(x)(\gamma^\mu)^\dagger\gamma^0\gamma^5\psi + \psi^\dagger(x)\gamma^0\gamma^5\gamma^\mu \psi(x)) + 2mc\psi^\dagger(x)\gamma^0\gamma^5\psi(x)) \\
   & = -\frac{1}{i\hbar}\ab(qA_\mu(x)(\psi^\dagger(x)\gamma^0\gamma^\mu\gamma^5\psi - \psi^\dagger(x)\gamma^0\gamma^\mu\gamma^5 \psi(x)) + 2mc\psi^\dagger(x)\gamma^0\gamma^5\psi(x))           \\
   & = -\frac{2mc}{i\hbar}\bar{\psi}(x)\gamma^5\psi(x)
\end{align}



\begin{problem}
中心力ポテンシャル $V(r)$ を持つハミルトニアン $\hat{H} = c\bm{\alpha}\cdot\hat{\pp} + \beta mc^2 + V(r)$ を考える.
\end{problem}
(1) $\hat{H}$ と軌道角運動量 $\hat{L} = \rr\times\hat{\pp}$ との交換関係を求めよ。
\begin{proof}
  \begin{align}
    [\hat{H}, \hat{L}^i] & = [c\bm{\alpha}\cdot\hat{\pp} + \beta mc^2 + V(r), \varepsilon^{ijk}r^j\hat{p}^k]                                              \\
                         & = \varepsilon^{ijk}\ab(c[\alpha^\mu\hat{p}^\mu, r^j\hat{p}^k] + mc^2[\beta, r^j\hat{p}^k] + [V(r), r^j\hat{p}^k])              \\
                         & = \varepsilon^{ijk}\ab(c\alpha^\mu(p^\mu r^jp^k - r^jp^kp^\mu) + mc^2(\beta r^jp^k - r^jp^k\beta) + (V(r)r^jp^k - r^jp^kV(r))) \\
                         & = \varepsilon^{ijk}\ab(c\alpha^\mu (-i\hbar\delta^{\mu j})p^k + 0 + r^j(-i\hbar\partial^kV(r)))                                \\
                         & = -i\hbar c\varepsilon^{ijk}\alpha^jp^k - i\hbar\ab(\rr\times\diff{V}{r}\diffp{r}{\rr})_i                                      \\
                         & = -i\hbar c\varepsilon^{ijk}\alpha^jp^k - i\hbar\diff{V}{r}\ab(\rr\times\frac{\rr}{r})_i                                       \\
                         & = -i\hbar c\varepsilon^{ijk}\alpha^jp^k
  \end{align}
\end{proof}
(2) $\Sigma^i := -\dfrac{i}{2}\sum_{j,k}\varepsilon^{ijk}\alpha^j\alpha^k$ とするとき、$\hat{H}$ とスピン角運動量 $\hat{\bm{S}} = \dfrac{\hbar}{2}\bm{\Sigma}$ との交換関係を求めよ。
\begin{proof}
  \begin{align}
    [\hat{H}, \hat{S}^i] & = \ab[c\bm{\alpha}\cdot\hat{\pp} + \beta mc^2 + V(r), -\frac{i\hbar}{4}\varepsilon^{ijk}\alpha^j\alpha^k]                                                          \\
                         & = -\frac{i\hbar}{4}\varepsilon^{ijk}\ab(c[\alpha^\mu, \alpha^j\alpha^k]p^\mu + mc^2[\beta, \alpha^j\alpha^k] + [V(r), \alpha^j\alpha^k])                           \\
                         & = -\frac{i\hbar}{4}\varepsilon^{ijk}\ab(c(\alpha^\mu\alpha^j\alpha^k - \alpha^j\alpha^k\alpha^\mu)p^\mu + mc^2(\beta\alpha^j\alpha^k - \alpha^j\alpha^k\beta) + 0) \\
                         & = -\frac{i\hbar}{4}\varepsilon^{ijk}\ab(c((-\alpha^j\alpha^\mu + 2\delta^{\mu j})\alpha^k - \alpha^j(-\alpha^\mu\alpha^k + 2\delta^{k\mu}))p^\mu + 0 + 0)          \\
                         & = -\frac{i\hbar c}{2}\varepsilon^{ijk}(\alpha^kp^j - \alpha^jp^k)                                                                                                  \\
                         & = i\hbar c\varepsilon^{ijk}\alpha^jp^k
  \end{align}
\end{proof}
(3) 全角運動量が保存量となることを示せ。
\begin{proof}
  全角運動量がハミルトニアンと交換するから保存量となる。
  \begin{align}
    [\hat{H}, \hat{J}^i] & = [\hat{H}, \hat{L}^i + \hat{S}^i] = 0
  \end{align}
\end{proof}

\begin{problem}
$\bm{\sigma}$ をパウリ行列として、任意のベクトル $\pp$ に対する 2 行 2 列の行列 $\bm{\sigma}\cdot\pp$ を考える。
\end{problem}
(1) $(\bm{\sigma}\cdot\pp)^2$ を求めよ。
\begin{proof}
  \begin{align}
    (\bm{\sigma}\cdot\pp)^2 & = (\sigma^ip^i)(\sigma^jp^j)^\dagger                   \\
                            & = p^ip^j\ab(\delta^{ij}I + i\varepsilon^{ijk}\sigma^k) \\
                            & = \pp^2
  \end{align}
\end{proof}
(2) $Tr(\bm{\sigma}\cdot\pp)$ を求めよ。
\begin{proof}
  $\pp = (p^1, p^2, p^3)$ とすると
  \begin{align}
    Tr(\bm{\sigma}\cdot\pp) & = Tr\begin{pmatrix}
                                    p^3        & p^1 - ip^2 \\
                                    p^1 + ip^2 & -p^3
                                  \end{pmatrix} = 0
  \end{align}
\end{proof}
(3) $\bm{\sigma}\cdot\pp$ の固有値を求めよ。
\begin{proof}
  $\bm{\sigma}\cdot\pp$ の固有値を $\lambda_1, \lambda_2$ とおくと
  \begin{align}
    \lambda_1 + \lambda_2     & = Tr(\bm{\sigma}\cdot\pp) = 0     \\
    \lambda_1^2 = \lambda_2^2 & = (\bm{\sigma}\cdot\pp)^2 = \pp^2
  \end{align}
  よって固有値は $\pm|\pp|$ である。
\end{proof}
(4) $\pp := |\pp|(\sin\theta\cos\phi, \sin\theta\sin\phi, \cos\theta)$ とするとき、固有ベクトルを求めよ。
\begin{proof}
  \begin{align}
    (\bm{\sigma}\cdot\pp \mp |\pp|I)\bm{v} & = |\pp|\begin{pmatrix}
                                                      \cos\theta \mp 1                         & \sin\theta\cos\phi - i\sin\theta\sin\phi \\
                                                      \sin\theta\cos\phi + i\sin\theta\sin\phi & -\cos\theta \mp 1
                                                    \end{pmatrix}\bm{v} \\
                                           & = |\pp|\begin{pmatrix}
                                                      \cos\theta \mp 1     & \sin\theta e^{-i\phi} \\
                                                      \sin\theta e^{i\phi} & -\cos\theta \mp 1
                                                    \end{pmatrix}\bm{v}
  \end{align}
  より固有値 $\pm|\pp|$ に対する固有ベクトルはそれぞれ次のベクトルの定数倍である。

  \begin{align}
    \bm{v} = \begin{pmatrix}
               \sin\theta e^{-i\phi} \\
               -\cos\theta \pm 1
             \end{pmatrix}
  \end{align}

\end{proof}

\subsection{2次元時空におけるディラック方程式}
ガンマ行列についても
\begin{align}
  \gamma^0 = \sigma_1, \qquad \gamma^1 = i\sigma_2, \qquad \gamma^5 = -\sigma_3
\end{align}
2次元時空におけるディラック方程式は次のように考えられる。
\begin{align}
  (i\hbar\gamma^\mu\partial_\mu - mc)\psi(x) = 0
\end{align}
このときガンマ行列 $\gamma^0, \gamma^1$ は次を満たす。
\begin{align}
  \lbrace\gamma^\mu, \gamma^\nu\rbrace = 2g^{\mu\nu}, \qquad (\gamma^0)^\dagger = \gamma^0, \qquad (\gamma^1)^\dagger = - \gamma^1
\end{align}
またカイラリティ $\gamma^5$ は次を満たす。
\begin{align}
  (\gamma^5)^\dagger = \gamma^5, \qquad (\gamma^5)^2 = 1, \qquad \lbrace\gamma^\mu, \gamma^5\rbrace = 0
\end{align}
カイラリティ $\gamma^5$ がガンマ行列の複素数係数多項式で表されるとするとガンマ行列の性質より次のように書ける。
\begin{align}
  \gamma^5 & = \sum_{e_0, e_1}a_{e_0,e_1}(\gamma^0)^{e_0}(\gamma^1)^{e_1} = \alpha_0 + \alpha_1\gamma^0 + \alpha_2\gamma^1 + \alpha_3\gamma^0\gamma^1 & \ab(a_{e_0, e_1}, \alpha_i\in\CC)
\end{align}
これを代入するとガンマ行列の直交性より
\begin{align}
  (\gamma^5)^\dagger = \gamma^5          & \iff \alpha_0\in\RR, \alpha_1\in\RR, \alpha_2\in i\RR, \alpha_3\in\RR \\
  \lbrace\gamma^\mu, \gamma^5\rbrace = 0 & \implies \alpha_0 = \alpha_1 = \alpha_2 = 0                           \\
  (\gamma^5)^2 = 1                       & \implies \alpha_4 = \pm 1
\end{align}
となる。よって $\gamma^5 = \pm \gamma^0\gamma^1$ となる。ここでは特に $\gamma^5 = \gamma^0\gamma^1$ とする。

\begin{problem}
$\gamma_\pm = \dfrac{1 \pm \gamma^5}{2}$ とするとき $(\gamma_+)^a$, $(\gamma_-)^b$, $(\gamma_+)^a(\gamma_-)^b$, $(\gamma_-)^b(\gamma_+)^a$ ($a, b\in\ZZ_{>0}$) を $\gamma_\pm$ を用いて表わせ。
\end{problem}
\begin{proof}
  \begin{align}
    (\gamma_\pm)^2   & = \frac{1 \pm 2\gamma^5 + (\gamma^5)^2}{2^2} = \frac{1 \pm \gamma^5}{2} = \gamma_\pm \\
    \gamma_+\gamma_- & = \gamma_-\gamma_+ = \frac{1 + \gamma^5 - \gamma^5 - (\gamma^5)^2}{2^2} = 0
  \end{align}
  より帰納法から次が示せる。
  \begin{align}
    (\gamma_+)^a = \gamma_+, \qquad (\gamma_-)^b = \gamma_-, \qquad (\gamma_+)^a(\gamma_-)^b = 0, \qquad (\gamma_-)^b(\gamma_+)^a = 0
  \end{align}
\end{proof}

\begin{problem}
$\psi_{\pm}(x) = \gamma_{\pm}\psi(x)$ は $\gamma^5$ の固有関数である。それぞれの固有値を求めよ。
\end{problem}
\begin{proof}
  カイラリティを作用させることで固有関数 $\psi_\pm(x)$ の固有値は $\pm 1$ となる。
  \begin{align}
    \gamma^5\psi_\pm(x) & = \gamma^5\gamma_\pm\psi(x) = \frac{\gamma^5 \pm 1}{2}\psi(x) = \pm\gamma^5\psi(x)
  \end{align}
\end{proof}

\begin{problem}
$\psi_\pm(x)$ が満たす連立微分方程式をディラック方程式から求めよ。
\end{problem}
\begin{proof}
  $\lbrace\gamma^\mu, \gamma^5\rbrace = 0$ より $\gamma^\mu\gamma_\pm = \gamma_\mp\gamma^\mu$ となる。よって
  \begin{align}
         & \begin{dcases}
             \gamma_+(i\hbar\gamma^\mu\partial_\mu - mc)\psi(x) = 0 \\
             \gamma_-(i\hbar\gamma^\mu\partial_\mu - mc)\psi(x) = 0 \\
           \end{dcases} \\
    \iff & \begin{dcases}
             i\hbar\gamma^\mu\partial_\mu\psi_-(x) = mc\psi_+(x) \\
             i\hbar\gamma^\mu\partial_\mu\psi_+(x) = mc\psi_-(x) \\
           \end{dcases}
  \end{align}
  となる。
\end{proof}

\begin{problem}
$m = 0$ の場合に $\psi_+(x) \propto e^{-iEt/\hbar+ipx/\hbar}$ が解となるとき、$E, p$ が満たす関係式を求めよ。
\end{problem}
\begin{proof}
  $m = 0$ のとき $i\hbar\gamma^\mu\partial_\mu\psi_+(x) = 0$ となる。 $\gamma^1 = \gamma^0(\gamma_+ - \gamma_-)$ より
  \begin{align}
    i\hbar\gamma^\mu\partial_\mu\psi_+(x) & = i\hbar(\gamma^0c\partial_t + \gamma^1\partial_x)\psi_+(x) \\
                                          & = (\gamma^0Ec - \gamma^1p)\psi_+(x)                         \\
                                          & = \gamma^0(Ec - (\gamma_+ - \gamma_-)p)\psi_+(x)            \\
                                          & = \gamma^0(Ec - p)\psi_+(x) = 0
  \end{align}
  となる。よって $Ec = p$ を満たす。
\end{proof}

\begin{problem}
$m = 0$ の場合に $\psi_-(x) \propto e^{-iEt/\hbar+ipx/\hbar}$ が解となるとき、$E, p$ が満たす関係式を求めよ。
\end{problem}
\begin{proof}
  $m = 0$ のとき $i\hbar\gamma^\mu\partial_\mu\psi_-(x) = 0$ となる。 $\gamma^1 = \gamma^0(\gamma_+ - \gamma_-)$ より
  \begin{align}
    i\hbar\gamma^\mu\partial_\mu\psi_-(x) & = i\hbar(\gamma^0c\partial_t + \gamma^1\partial_x)\psi_-(x) \\
                                          & = (\gamma^0Ec - \gamma^1p)\psi_-(x)                         \\
                                          & = \gamma^0(Ec - (\gamma_+ - \gamma_-)p)\psi_-(x)            \\
                                          & = \gamma^0(Ec + p)\psi_-(x) = 0
  \end{align}
  となる。よって $Ec = -p$ を満たす。
\end{proof}

\section{指数関数}

\begin{problem}
次の式を示せ。
\begin{align}
  e^{i\BB}\AA e^{-i\BB} = \sum_{n=0}^{\infty}\frac{i^n}{n!}\underbrace{[\BB, \ldots [\BB, [\BB}_{n}, \AA]] \ldots]
\end{align}
\end{problem}
\begin{proof}
  $e^{i\lambda\BB}\AA e^{-i\lambda\BB}$ について考える。これを $\lambda$ について展開すると
  \begin{align}
    e^{i\lambda\BB}\AA e^{-i\lambda\BB} & = \sum_{n=0}^{\infty}\frac{\lambda^n}{n!}\ab[\diff[n]{}{\lambda}e^{i\lambda\BB}\AA e^{i\lambda\BB}]_{\lambda = 0}                                         \\
                                        & = \sum_{n=0}^{\infty}\frac{\lambda^n}{n!}\ab[\diff[n-1]{}{\lambda}e^{i\lambda\BB}i[\BB, \AA]e^{i\lambda\BB}]_{\lambda = 0}                                \\
                                        & = \sum_{n=0}^{\infty}\frac{\lambda^n}{n!}\Big[e^{i\lambda\BB}i^n\underbrace{[\BB, \ldots [\BB, [\BB}_{n}, \AA]] \ldots]e^{i\lambda\BB}\Big]_{\lambda = 0} \\
                                        & = \sum_{n=0}^{\infty}\frac{(i\lambda)^n}{n!}\underbrace{[\BB, \ldots [\BB, [\BB}_{n}, \AA]] \ldots]
  \end{align}
  よって $\lambda = 1$ を代入することで示せる。
  \begin{align}
    e^{i\BB}\AA e^{-i\BB} & = \sum_{n=0}^{\infty}\frac{i^n}{n!}\underbrace{[\BB, \ldots [\BB, [\BB}_{n}, \AA]] \ldots]
  \end{align}
\end{proof}

\begin{problem}
$\partial$ を微分演算子とするとき、$(\partial e^{i\BB})e^{-i\BB} = -e^{i\BB}(\partial e^{-i\BB})$ を示せ。
\end{problem}
\begin{proof}
  $1 = e^{i\BB}e^{-i\BB}$ に微分演算子を作用させることで示せる。
  \begin{align}
    0 = \partial 1 = \partial(e^{i\BB}e^{-i\BB}) & = (\partial e^{i\BB})e^{-i\BB} + e^{i\BB}(\partial e^{-i\BB}) \\
    (\partial e^{i\BB})e^{-i\BB}                 & = -e^{i\BB}(\partial e^{-i\BB})
  \end{align}
\end{proof}

\begin{problem}
次の式を示せ。
\begin{align}
  e^{i\BB}(\partial e^{-i\BB}) = -\sum_{n=1}^{\infty}\frac{i^n}{n!}\underbrace{[\BB, \ldots [\BB, [\BB}_{n-1}, \partial\BB]] \ldots]
\end{align}
\end{problem}
\begin{proof}
  (1) において $\AA = \partial$ を代入して示せる。
  \begin{align}
    e^{i\BB}(\partial e^{-i\BB}) & = \sum_{n=0}^{\infty}\frac{i^n}{n!}\underbrace{[\BB, \ldots [\BB, [\BB}_{n}, \partial]] \ldots]            \\
                                 & = \partial + \sum_{n=1}^{\infty}\frac{i^n}{n!}\underbrace{[\BB, \ldots [\BB, [\BB}_{n}, \partial]] \ldots] \\
                                 & = -\sum_{n=1}^{\infty}\frac{i^n}{n!}\underbrace{[\BB, \ldots [\BB, [\BB}_{n-1}, \partial\BB]] \ldots]      \\
  \end{align}
\end{proof}

\section{スピノル球関数}
\begin{problem}
スピノル球関数を球面調和関数を用いて次のように定義する。
\begin{align}
  \mathcal{Y}_{j, m}^\pm(\theta, \phi) & = \frac{1}{\sqrt{2l + 1}}\begin{pmatrix}
                                                                    \sqrt{l + \frac{1}{2} \pm m}Y_l^{m - 1/2}(\theta, \phi)    \\
                                                                    \pm\sqrt{l + \frac{1}{2} \mp m}Y_l^{m + 1/2}(\theta, \phi) \\
                                                                  \end{pmatrix}_{j = l\pm 1/2}
\end{align}
軌道角運動量 $\hat{\bm{L}}$, スピン角運動量 $\hat{\bm{S}}$, 全角運動量 $\hat{\bm{J}}$ とする。

$\mathcal{Y}_{j, m}^\pm(\theta, \phi)$ が持つパリティを求めよ。
\end{problem}
\begin{proof}
  球面調和関数におけるパリティは $(-1)^l$ となるからスピノル球関数のパリティは $(-1)^l$ となる。
  \begin{align}
    Y_l^m(\pi - \theta, \pi + \phi)                  & = (-1)^lY_l^m(\theta, \phi)                  \\
    \mathcal{Y}_{j, m}^\pm(\pi - \theta, \pi + \phi) & = (-1)^l\mathcal{Y}_{j, m}^\pm(\theta, \phi)
  \end{align}
\end{proof}

\begin{problem}
$\mathcal{Y}_{j, m}^\pm(\theta, \phi)$ が持つ $\hat{J}_z$ の固有値を求めよ。
\end{problem}
\begin{proof}
  球面調和関数における固有値からスピノル球関数の $\hat{J}_z$ の固有値は $m\hbar$ となる。
  \begin{align}
    \hat{J}_zY_l^m(\theta, \phi)                  & = m\hbar Y_l^m(\theta, \phi)                  \\
    \hat{J}_z\mathcal{Y}_{j, m}^\pm(\theta, \phi) & = m\hbar \mathcal{Y}_{j, m}^\pm(\theta, \phi)
  \end{align}
\end{proof}

\begin{problem}
$\mathcal{Y}_{j, m}^\pm(\theta, \phi)$ が持つ $\hat{\bm{L}}^2$ の固有値を求めよ。
\end{problem}
\begin{proof}
  球面調和関数における固有値からスピノル球関数の $\hat{\bm{L}}^2$ の固有値は $\hbar^2l(l+1)$ となる。
  \begin{align}
    \hat{\bm{L}}^2Y_l^m(\theta, \phi)                  & = \hbar^2l(l+1) Y_l^m(\theta, \phi)                  \\
    \hat{\bm{L}}^2\mathcal{Y}_{j, m}^\pm(\theta, \phi) & = \hbar^2l(l+1) \mathcal{Y}_{j, m}^\pm(\theta, \phi)
  \end{align}
\end{proof}

\begin{problem}
$\mathcal{Y}_{j, m}^\pm(\theta, \phi)$ が持つ $\hat{\bm{L}}\cdot\hat{\bm{S}}$ の固有値を求めよ。
\end{problem}
\begin{proof}
  $\mathcal{Y}_{j, m}^\pm(\theta, \phi)$ に $\hat{\bm{L}}\cdot\hat{\bm{S}}$ を作用させると
  \begin{align}
    \hat{\bm{L}}\cdot\hat{\bm{S}}\mathcal{Y}_{j, m}^\pm(\theta, \phi) & = \frac{1}{2}(\hat{\bm{J}}^2 - \hat{\bm{L}}^2 - \hat{\bm{S}}^2)\mathcal{Y}_{j, m}^\pm(\theta, \phi)                             \\
                                                                      & = \frac{\hbar^2}{2}\ab(j(j+1) - l(l+1) - \frac{3}{4})\mathcal{Y}_{j, m}^\pm(\theta, \phi)                                       \\
                                                                      & = \frac{\hbar^2}{2}\ab(\ab(l\pm \frac{1}{2})\ab(l\pm \frac{1}{2}+1) - l(l+1) - \frac{3}{4})\mathcal{Y}_{j, m}^\pm(\theta, \phi) \\
                                                                      & = \frac{\hbar^2}{2}\ab(\pm\ab(l + \frac{1}{2}) - \frac{1}{2})\mathcal{Y}_{j, m}^\pm(\theta, \phi)                               \\
  \end{align}
  より固有値は $\dfrac{\hbar^2l}{2}, -\dfrac{\hbar^2(l+1)}{2}$ となる。
\end{proof}

\begin{problem}
$\mathcal{Y}_{j, m}^\pm(\theta, \phi)$ が持つ $\hat{\bm{J}}^2$ の固有値を求めよ。
\end{problem}
\begin{proof}
  $\mathcal{Y}_{j, m}^\pm(\theta, \phi)$ に $\hat{\bm{J}}^2$ を作用させると
  \begin{align}
    \hat{\bm{J}}^2\mathcal{Y}_{j, m}^\pm(\theta, \phi) & = \hbar^2j(j+1) \mathcal{Y}_{j, m}^\pm(\theta, \phi)                                                   \\
                                                       & = \hbar^2\ab(l \pm \frac{1}{2})\ab(\ab(l \pm \frac{1}{2}) + 1)\mathcal{Y}_{j, m}^\pm(\theta, \phi)     \\
                                                       & = \hbar^2\ab(\ab(l \pm \frac{1}{2} + \frac{1}{2})^2 - \frac{1}{4})\mathcal{Y}_{j, m}^\pm(\theta, \phi)
  \end{align}
  より固有値は $l^2 - \dfrac{1}{4}, (l + 1)^2 - \dfrac{1}{4}$ となる。
\end{proof}

\begin{problem}
パウリ行列 $\bm{\sigma}$ と位置ベクトル $\rr = r(\sin\theta\cos\phi, \sin\theta\sin\phi, \cos\theta)$ に対して $\bm{\sigma}\cdot\dfrac{\rr}{r}\mathcal{Y}_{j, m}^\pm(\theta, \phi)$ を計算し、スピノル球関数のみを用いて表せ。
\end{problem}
\begin{proof}
  まず演算子を計算すると
  \begin{align}
    \bm{\sigma}\cdot\dfrac{\rr}{r} & = \bm{\sigma}\cdot (\sin\theta\cos\phi, \sin\theta\sin\phi, \cos\theta) =
    \begin{pmatrix}
      \cos\theta           & \sin\theta e^{-i\phi} \\
      \sin\theta e^{i\phi} & -\cos\theta           \\
    \end{pmatrix}
  \end{align}
  となる。三角関数を球面調和関数に作用させたときの固有値は次のようになるから
  \begin{align}
    \cos\theta Y_{l}^m(\theta, \phi)            & = \sqrt{\frac{(l + m + 1)(l - m + 1)}{(2l + 1)(2l + 3)}}Y_{l+1}^m(\theta, \phi) + \sqrt{\frac{(l + m)(l - m)}{(2l + 1)(2l - 1)}}Y_{l-1}^m(\theta, \phi)              \\
    \sin\theta e^{i\phi} Y_{l}^m(\theta, \phi)  & = -\sqrt{\frac{(l + m + 1)(l + m + 2)}{(2l + 1)(2l + 3)}}Y_{l+1}^{m+1}(\theta, \phi) + \sqrt{\frac{(l - m)(l - m - 1)}{(2l + 1)(2l - 1)}}Y_{l-1}^{m+1}(\theta, \phi) \\
    \sin\theta e^{-i\phi} Y_{l}^m(\theta, \phi) & = \sqrt{\frac{(l - m + 1)(l - m + 2)}{(2l + 1)(2l + 3)}}Y_{l+1}^{m-1}(\theta, \phi) + \sqrt{\frac{(l + m)(l + m - 1)}{(2l + 1)(2l - 1)}}Y_{l-1}^{m-1}(\theta, \phi)
  \end{align}
  次のように計算できる。
  \begin{alignat}{3}
    \bm{\sigma}\cdot\dfrac{\rr}{r}\mathcal{Y}_{j, m}^\pm(\theta, \phi) & = \frac{1}{\sqrt{2l + 1}}       &  &
    \begin{pmatrix}
      \cos\theta           & \sin\theta e^{-i\phi} \\
      \sin\theta e^{i\phi} & -\cos\theta           \\
    \end{pmatrix}
    \begin{pmatrix}
      \sqrt{l + \frac{1}{2} \pm m}Y_l^{m - 1/2}(\theta, \phi)    \\
      \pm\sqrt{l + \frac{1}{2} \mp m}Y_l^{m + 1/2}(\theta, \phi) \\
    \end{pmatrix}_{j = l\pm 1/2}                                                                                                                                                                                                             \\
                                                                       & = \frac{1}{\sqrt{2l + 1}}       &  & \Bigg(\sqrt{l + \frac{1}{2} \pm m}\cos\theta Y_l^{m - 1/2}(\theta, \phi) \pm\sqrt{l + \frac{1}{2} \mp m}\sin\theta e^{-i\phi}Y_l^{m + 1/2}(\theta, \phi)                 \\
                                                                       &                                 &  & , \sqrt{l + \frac{1}{2} \pm m}\sin\theta e^{i\phi}Y_l^{m - 1/2}(\theta, \phi) \mp\sqrt{l + \frac{1}{2} \mp m}\cos\theta Y_l^{m + 1/2}(\theta, \phi)\Bigg)_{j = l\pm 1/2} \\
                                                                       & = \frac{1}{\sqrt{2l + 1}}       &  & \Bigg(\sqrt{l + \frac{1}{2} \pm m}\sqrt{\frac{(l + m + \frac{1}{2})(l - m + \frac{3}{2})}{(2l + 1)(2l + 3)}}Y_{l+1}^{m-1/2}(\theta, \phi)                                \\
                                                                       &                                 &  & + \sqrt{l + \frac{1}{2} \pm m}\sqrt{\frac{(l + m - \frac{1}{2})(l - m + \frac{1}{2})}{(2l + 1)(2l - 1)}}Y_{l-1}^{m-1/2}(\theta, \phi)                                    \\
                                                                       &                                 &  & \pm\sqrt{l + \frac{1}{2} \mp m}\sqrt{\frac{(l - m + \frac{1}{2})(l - m + \frac{3}{2})}{(2l + 1)(2l + 3)}}Y_{l+1}^{m-1/2}(\theta, \phi)                                   \\
                                                                       &                                 &  & \pm\sqrt{l + \frac{1}{2} \mp m}\sqrt{\frac{(l + m + \frac{1}{2})(l + m - \frac{1}{2})}{(2l + 1)(2l - 1)}}Y_{l-1}^{m-1/2}(\theta, \phi)                                   \\
                                                                       &                                 &  & , -\sqrt{l + \frac{1}{2} \pm m}\sqrt{\frac{(l + m + \frac{1}{2})(l + m + \frac{3}{2})}{(2l + 1)(2l + 3)}}Y_{l+1}^{m+1/2}(\theta, \phi)                                   \\
                                                                       &                                 &  & + \sqrt{l + \frac{1}{2} \pm m}\sqrt{\frac{(l - m + \frac{1}{2})(l - m - \frac{1}{2})}{(2l + 1)(2l - 1)}}Y_{l-1}^{m+1/2}(\theta, \phi)                                    \\
                                                                       &                                 &  & \mp\sqrt{l + \frac{1}{2} \mp m}\sqrt{\frac{(l + m + \frac{3}{2})(l - m + \frac{1}{2})}{(2l + 1)(2l + 3)}}Y_{l+1}^{m+1/2}(\theta, \phi)                                   \\
                                                                       &                                 &  & \mp\sqrt{l + \frac{1}{2} \mp m}\sqrt{\frac{(l + m + \frac{1}{2})(l - m - \frac{1}{2})}{(2l + 1)(2l - 1)}}Y_{l-1}^{m+1/2}(\theta, \phi)\Bigg)_{j = l\pm 1/2}              \\
                                                                       & = \frac{1}{\sqrt{2l + 1}}       &  & \Bigg((2l + 1)\sqrt{\frac{l \mp m + \frac{1}{2} \pm 1}{(2l + 1)(2l + 1 \pm 2)}}Y_{l\pm 1}^{m-1/2}(\theta, \phi)                                                          \\
                                                                       &                                 &  & ,\mp(2l + 1)\sqrt{\frac{l \pm m + \frac{1}{2} \pm 1}{(2l + 1)(2l + 1 \pm 2)}}Y_{l\pm 1}^{m+1/2}(\theta, \phi)\Bigg)_{j = l\pm 1/2}                                       \\
                                                                       & = \frac{1}{\sqrt{2j + 1 \pm 1}} &  & \Bigg(\sqrt{j + \frac{1}{2} \mp \ab(m - \frac{1}{2})}Y_{j\pm 1/2}^{m-1/2}(\theta, \phi)                                                                                  \\
                                                                       &                                 &  & ,\mp\sqrt{j + \frac{1}{2} \pm \ab(m + \frac{1}{2})}Y_{j\pm 1/2}^{m+1/2}(\theta, \phi)\Bigg) = \mathcal{Y}_{j, m}^\mp(\theta, \phi)
  \end{alignat}
  よって次の式となる。
  \begin{align}
    \bm{\sigma}\cdot\dfrac{\rr}{r}\mathcal{Y}_{j, m}^\pm(\theta, \phi) & = \mathcal{Y}_{j, m}^\mp(\theta, \phi)
  \end{align}
\end{proof}

\section{水素原子における電子のエネルギー準位}
\begin{problem}
中心力ポテンシャル $V(r) = -\dfrac{\alpha\hbar c}{r}$ のもとでディラック方程式を解くことにより得られる水素原子中の電子のエネルギー準位を考える。

主量子数 $n$ が与えられたとき、全角運動量 $j$ が取り得る値を答えよ。
\end{problem}
\begin{proof}
  $n$ と $j$ に関して $n = j + n' + 1/2$ という関係があるから $j = 1/2,\ldots,(2n-1)/2$ を取る。
\end{proof}

\begin{problem}
主量子数 $n$, 全角運動量 $j$ を持つ状態のエネルギー固有値の表式を書き下せ。また、そのエネルギー固有値の縮重度を答えよ。
\end{problem}
\begin{proof}
  主量子数 $n$, 全角運動量 $j$ を持つ状態のエネルギー固有値の表式は次のようになる。
  \begin{align}
    E & = \frac{mc^2}{\sqrt{1 + \frac{(Z\alpha)^2}{\ab(n - \ab(j + \frac{1}{2}) + \sqrt{(j + \frac{1}{2})^2 - (Z\alpha)^2})^2}}}
  \end{align}
  また縮重度は $j = l \pm 1/2$ より $n' = 0$ において $2j + 1$、$n' > 0$ において $2(2j + 1)$ となる。
\end{proof}

\begin{problem}
主量子数 $n$ を持つ状態の総数を求めよ。
\end{problem}
\begin{proof}
  $n = j + n' + 1/2$ と $j = l \pm 1/2$ より状態の総数は $2n^2$ となる。
  \begin{align}
    2n + 2\times\sum_{n'=1}^{n-1}2(n-n') & = 2n^2
  \end{align}
\end{proof}


\begin{problem}
電子の静止エネルギーから測った束縛エネルギーの大きさが縮退を除いて 7 番目と 10 番目に大きい準位の主量子数 $n$ と全角運動量 $j$ をそれぞれ答えよ。また、それらの準位の束縛エネルギーの大きさを有効数字 6 桁で求めよ。
\end{problem}
\begin{proof}
  7 番目に大きい準位は $n = 4, j = 1/2$ で 10 番目に大きい準位は $n = 4, j = 7/2$ である。
  またそれぞれの束縛エネルギーは
  \begin{align}
    \mathcal{E}         & \approx -\frac{\alpha^2 mc^2}{2n^2} - \ab(\frac{1}{j + 1/2} - \frac{3}{4n})\frac{\alpha^4mc^2}{2n^3}   \\
                        & \approx -\frac{13.60569}{n^2} - \ab(\frac{1}{j + 1/2} - \frac{3}{4n})\frac{7.249022\times10^{-4}}{n^3} \\
    \mathcal{E}_{4,1/2} & \approx 0.850365                                                                                       \\
    \mathcal{E}_{4,7/2} & \approx 0.850356
  \end{align}
  となる。
\end{proof}

\begin{problem}
同じ主量子数 $n$ を持つ状態でも全角運動量 $j$ に依存してエネルギー準位が分裂する. この現象を表す名称を答えよ.
\end{problem}
\begin{proof}
  微細構造
\end{proof}

\end{document}