\RequirePackage{plautopatch}
\documentclass[uplatex,diffipdfmx,a4paper,11pt]{jlreq}
\input{preamble.tex}
\newcommand{\LL}{\mathcal{L}}

\title{微分方程式}
\author{anko9801}
\begin{document}
\maketitle
\tableofcontents
\clearpage


\section{ラプラス変換}
\begin{definition}[ラプラス変換]
  \begin{align}
    F(s) = \LL[f(t)]      & = \int_0^\infty f(t)e^{-st}\dl{t}                                     \\
    f(t) = \LL^{-1}[F(s)] & = \lim_{p\to\infty}\frac{1}{2\pi i}\int_{c-ip}^{c+ip}F(s)e^{st}\dl{s}
  \end{align}
\end{definition}
\begin{theorem}
  \begin{align}
    \LL[f'(t)]                 & = s\LL[f(t)] - f(0) \\
    \LL\ab[\int_0^tf(u)\dl{u}] & =
  \end{align}
\end{theorem}
\begin{proof}
  テスト
  \begin{align}
    \LL[f'(t)]                 & = \int_0^\infty f'(t)e^{-st}\dl{t} = \ab[f(t)e^{-st}]_0^\infty + s\int_0^\infty f(t)e^{-st}\dl{t} = s\LL[f(t)] - f(0) \\
    \LL\ab[\int_0^tf(u)\dl{u}] & =
  \end{align}
\end{proof}
例えば
\begin{align}
  \LL[1]   & = \int_0^\infty e^{-st}\dl{t} = \ab[\frac{e^{-st}}{-s}]_0^\infty = \frac{1}{s}                                                                                                                                     \\
  \LL[t]   & = \int_0^\infty te^{-st}\dl{t} = \ab[t\frac{e^{-st}}{-s}]_0^\infty - \ab[\frac{e^{-st}}{(-s)^2}]_0^\infty = \frac{1}{s^2}                                                                                          \\
  \LL[t^2] & = \int_0^\infty t^2e^{-st}\dl{t} = \ab[t^2\frac{e^{-st}}{-s}]_0^\infty - \ab[2t\frac{e^{-st}}{(-s)^2}]_0^\infty + \ab[2\frac{e^{-st}}{(-s)^3}]_0^\infty = \frac{2}{s^3}                                            \\
  \LL[t^3] & = \int_0^\infty t^3e^{-st}\dl{t} = \ab[t^3\frac{e^{-st}}{-s}]_0^\infty - \ab[3t^2\frac{e^{-st}}{(-s)^2}]_0^\infty + \ab[6t\frac{e^{-st}}{(-s)^3}]_0^\infty - \ab[6\frac{e^{-st}}{(-s)^4}]_0^\infty = \frac{6}{s^4} \\
  \LL[t^n] & = \int_0^\infty t^ne^{-st}\dl{t} = \sum_{i = 0}^{n}(-1)^i\ab[\diff[i]{t^n}{t}\frac{e^{-st}}{(-s)^{i+1}}]_0^\infty = \frac{n!}{s^{n+1}}
\end{align}

\begin{align}
  \LL[e^{s_0t}] & = \int_0^\infty e^{(s_0-s)t}\dl{t} = \frac{1}{s - s_0}
\end{align}
\begin{align}
  \LL[\sin\omega t] & = \int_0^\infty \sin\omega te^{-st}\dl{t} = \frac{1}{2i}\int_0^\infty \ab(e^{i\omega t} - e^{-i\omega t})e^{-st}\dl{t} \\
                    & = -\frac{1}{2i}\ab(\frac{1}{i\omega - s} - \frac{1}{-i\omega - s}) = \frac{\omega}{s^2 + \omega^2}                     \\
  \LL[\cos\omega t] & = \int_0^\infty \cos\omega te^{-st}\dl{t} = \frac{1}{2}\int_0^\infty \ab(e^{i\omega t} + e^{-i\omega t})e^{-st}\dl{t}  \\
                    & = -\frac{1}{2}\ab(\frac{1}{i\omega - s} + \frac{1}{-i\omega - s}) = \frac{s}{s^2 + \omega^2}
\end{align}
\begin{align}
  \LL[\sinh at] & = \int_0^\infty \sinh ate^{-st}\dl{t} = \frac{1}{2}\int_0^\infty \ab(e^{at} - e^{-at})e^{-st}\dl{t} \\
                & = -\frac{1}{2}\ab(\frac{1}{a - s} - \frac{1}{-a - s}) = \frac{a}{s^2 - a^2}                         \\
  \LL[\cosh at] & = \int_0^\infty \cosh ate^{-st}\dl{t} = \frac{1}{2}\int_0^\infty \ab(e^{at} + e^{-at})e^{-st}\dl{t} \\
                & = -\frac{1}{2}\ab(\frac{1}{a - s} + \frac{1}{-a - s}) = \frac{s}{s^2 - a^2}
\end{align}
\begin{align}
  \LL[\delta(t - a)] & = \int_0^\infty \delta(t - a)e^{-st}\dl{t} = e^{-sa}
\end{align}

\section{特殊関数}
\subsection{ガウス積分}
\begin{theorem}[Gauss 積分]
  \begin{align}
    \int_{-\infty}^{\infty} e^{-\alpha x^2}\dl{x} = \sqrt{\frac{\pi}{\alpha}} \qquad (\real a > 0)
  \end{align}
\end{theorem}
\begin{proof}
  まず積分値を $I$ とおく。
  \begin{align}
    I & := \int_{-\infty}^{\infty} e^{-\alpha x^2}\dl{x}
  \end{align}
  ここで $I^2$ を変数変換して計算する。
  \begin{align}
    I^2 & = \int_{-\infty}^{\infty} e^{-\alpha x^2}\dl{x}\int_{-\infty}^{\infty} e^{-\alpha y^2}\dl{y} \\
        & = \int_{-\infty}^{\infty}\int_{-\infty}^{\infty} e^{-\alpha(x^2 + y^2)}\dl{x}\dl{y}          \\
        & = \int_0^{\infty}\int_0^{2\pi}e^{-\alpha r^2}r\dl{\theta}\dl{r}                              \\
        & = 2\pi\ab[\frac{e^{-\alpha r^2}}{-2\alpha}]_0^{\infty}                                       \\
        & = \frac{\pi}{\alpha}
  \end{align}
  よって示される。
  \begin{align}
    \int_{-\infty}^{\infty} e^{-\alpha x^2}\dl{x} = \sqrt{\frac{\pi}{\alpha}}
  \end{align}
\end{proof}

\begin{theorem}[Gauss 積分]
  \begin{align}
     & \int_0^{\infty}x^{2n}e^{-x^2/a^2}\dl{x} = \sqrt{\pi}(2n - 1)!!\frac{a^{2n + 1}}{2^{n+1}} \\
     & \int_0^{\infty}x^{2n + 1}e^{-x^2/a^2}\dl{x} = \frac{n!}{2}a^{2n + 2}                     \\
     & \int_{-\infty}^{\infty}e^{-k^2/4}e^{ikx}\dl{k} = 2\sqrt{\pi}e^{-x^2}
  \end{align}
\end{theorem}
\begin{proof}
  \begin{align}
    \int_0^{\infty} x^{2n}e^{-\alpha x^2}\dl{x} & = (-1)^n\int_0^{\infty} \diffp[n]{\alpha}e^{-\alpha x^2}\dl{x}     \\
                                                & = (-1)^n\diffp[n]{\alpha}\int_0^{\infty} e^{-\alpha x^2}\dl{x}     \\
                                                & = (-1)^n\diffp[n]{\alpha}\ab(\frac{1}{2}\sqrt{\frac{\pi}{\alpha}}) \\
                                                & = \sqrt{\pi}\frac{(2n - 1)!!}{2^{n+1}}\alpha^{-(2n + 1)/2}
  \end{align}
  \begin{align}
    \int_0^{\infty} x^{2n + 1}e^{-\alpha x^2}\dl{x} & = (-1)^n\int_0^{\infty}\diffp[n]{}{\alpha}xe^{-\alpha x^2}\dl{x} \\
                                                    & = (-1)^n\diffp[n]{}{\alpha}\int_0^{\infty}xe^{-\alpha x^2}\dl{x} \\
                                                    & = (-1)^n\diffp[n]{}{\alpha}\frac{1}{2\alpha}                     \\
                                                    & = \frac{n!}{2}\alpha^{-(n + 1)}
  \end{align}
\end{proof}

\subsection{ガンマ関数}
\begin{definition}
  複素平面上で $\Re z > 1$ を満たす領域内にある閉曲線 $C$ 上の点 $z$ に対して次の関数は一様収束し正則な関数となる.
  \begin{align}
    \Gamma(z) := \int_0^\infty t^{z-1}e^{-t}\dl{t}
  \end{align}
\end{definition}

\begin{proposition}
  \begin{align}
    \Gamma(z + 1) & = z\Gamma(z), \qquad \Gamma(n + 1) = n!, \qquad \Gamma(1) = 1, \qquad \Gamma\ab(\frac{1}{2}) = \sqrt{\pi}
  \end{align}
\end{proposition}
\begin{proof}
  \begin{align}
    \Gamma(z + 1)          & = \int_0^\infty t^{z}e^{-t}\dl{t}                              \\
                           & = [-t^{z}e^{-t}]_0^\infty + z\int_0^\infty t^{z-1}e^{-t}\dl{t} \\
                           & = z\Gamma(z)                                                   \\
    \Gamma(1)              & = \int_0^\infty e^{-t}\dl{t}                                   \\
                           & = [-e^{-t}]_0^\infty                                           \\
                           & = 1                                                            \\
    \Gamma\ab(\frac{1}{2}) & = \int_0^\infty t^{-1/2}e^{-t}\dl{t}                           \\
                           & = \int_0^\infty s^{-1}e^{-s^2}2s\dl{s}                         \\
                           & = 2\int_0^\infty e^{-s^2}\dl{s}                                \\
                           & = \sqrt{\pi}
  \end{align}
\end{proof}

\begin{proposition}[スターリングの公式 (Stirling's formula)]
  \begin{align}
    \Gamma(x + 1) & = \sqrt{2\pi x}e^{-x}x^x \qquad (x \gg 1)
  \end{align}
\end{proposition}
\begin{proof}
  \begin{align}
    \Gamma(x) & = \int_0^\infty t^{x - 1}e^{-t}\dl{t}                                                    \\
              & = \int_0^\infty e^{s(x - 1)}e^{-e^s}e^s\dl{s}                                            \\
              & = \int_0^\infty e^{sx - e^s}\dl{s}                                                       \\
              & \approx \int_0^\infty e^{(x\ln x - x) - \frac{x}{2}(s - \ln x)^2}\dl{s} \qquad (x \gg 1) \\
              & = x^xe^{-x}\int_0^\infty e^{-\frac{x}{2}(s - \ln x)^2}\dl{s}                             \\
              & = x^xe^{-x}\int_0^\infty e^{-\frac{x}{2}s^2}\dl{s}                                       \\
              & = \sqrt{\frac{2\pi}{x}}x^xe^{-x}
  \end{align}
  これより
  \begin{align}
    \Gamma(x + 1) & = x\Gamma(x) = \sqrt{2\pi x}x^xe^{-x}
  \end{align}
\end{proof}

\begin{proposition}[ガンマ関数の極と零点]
  \begin{align}
    \Gamma(z)                                    & = \infty \iff z = 0, -1, -2, \ldots           \\
    \Res[\Gamma(z); z = -n]                      & = \frac{(-1)^n}{n!} \qquad (n = 0,1,2,\ldots) \\
    \lbrace\Gamma(s) = 0\mid |s| < \infty\rbrace & = \emptyset
  \end{align}
\end{proposition}

\begin{proposition}[ワイエルシュトラスの公式 (Weierstrass' formula)]
  $\gamma$ はオイラーの定数 (Euler's constant) とする.
  \begin{align}
    \frac{1}{\Gamma(z)} & = ze^{\gamma z}\prod_{n=1}^{\infty}\ab(1 + \frac{z}{n})e^{-z/n}              \\
    \gamma              & := \lim_{n\to\infty}\ab(\sum_{m=1}^{n}\frac{1}{m} - \log n) = 0.577216\cdots
  \end{align}
\end{proposition}


\subsection{ベータ関数}
\begin{definition}
  ベータ関数 (Beta function)
  \begin{align}
    B(m, n) := \int_{0}^{1}t^{m-1}(1-t)^{n-1}\dl{t}
  \end{align}
\end{definition}

\begin{proposition}
  \begin{align}
    B(m, n) & = B(n, m)                                                          \\
    B(m, n) & = 2\int_0^{\pi/2}\sin^{2m - 1}\theta\cos^{2n - 1}\theta\dl{\theta} \\
    B(m, n) & = \int_0^{\infty}\frac{u^{m-1}}{(1 + u)^{m + n}}\dl{\theta}
  \end{align}
\end{proposition}
\begin{proof}
  \begin{align}
    B(m, n) & = \int_{0}^{1}t^{m-1}(1-t)^{n-1}\dl{t}                                                                             \\
            & = \int_{1}^{0}(1 - s)^{m-1}s^{n-1}(-\dl{t}) \qquad (s = 1 - t)                                                     \\
            & = \int_{0}^{1}s^{n-1}(1 - s)^{m-1}\dl{t}                                                                           \\
            & = B(n, m)                                                                                                          \\
    B(m, n) & = \int_{0}^{1}t^{m-1}(1-t)^{n-1}\dl{t}                                                                             \\
            & = \int_{0}^{\pi/2}\sin^{2(m-1)}\theta\cos^{2(n-1)}\theta2\sin\theta\cos\theta\dl{\theta} \qquad (t = \sin^2\theta) \\
            & = 2\int_0^{\pi/2}\sin^{2m - 1}\theta\cos^{2n - 1}\theta\dl{\theta}                                                 \\
    B(m, n) & = \int_{0}^{1}t^{m-1}(1-t)^{n-1}\dl{t}                                                                             \\
  \end{align}
\end{proof}

\begin{theorem}
  \begin{align}
    B(m, n)     & = \frac{n - 1}{m}B(m + 1, n - 1)  \\
    B(m, n + 1) & = \frac{n!}{\prod_{k=0}^n(m + k)}
  \end{align}
\end{theorem}
\begin{proof}
  \begin{align}
    B(m, n)     & = \int_{0}^{1}t^{m-1}(1-t)^{n-1}\dl{t}                                               \\
                & = \ab[\frac{1}{m}t^m(1-t)^{n-1}]_0^1 + \frac{n-1}{m}\int_{0}^{1}t^m(1-t)^{n-2}\dl{t} \\
                & = \frac{n - 1}{m}B(m + 1, n - 1)                                                     \\
    B(m, n + 1) & = \frac{n\cdot(n - 1)\cdots 1}{a\cdot(m + 1)\cdots(m + n - 1)}B(m + n, 1)            \\
                & = \frac{n!}{\prod_{k = 0}^{n-1}(m + k)}\frac{1}{m + n}                               \\
                & = \frac{n!}{\prod_{k = 0}^{n}(m + k)}
  \end{align}
\end{proof}

\begin{proposition}[ガンマ関数とベータ関数との関係]
  \begin{align}
    B(m, n)                & = \frac{\Gamma(m)\Gamma(n)}{\Gamma(m + n)} \\
    \Gamma(z)\Gamma(1 - z) & = \frac{\pi}{\sin\pi z}
  \end{align}
\end{proposition}
\begin{proof}
  \begin{align}
    \Gamma(m)\Gamma(n) & = \int_0^\infty s^{m-1}e^{-s}\dl{s}\int_0^\infty t^{n-1}e^{-t}\dl{t}                                                                                      \\
                       & = \int_0^\infty q^{2(m-1)}e^{-q^2}2q\dl{q}\int_0^\infty p^{2(n-1)}e^{-p^2}2p\dl{p}                                   & (s = q^2, t = p^2)                 \\
                       & = 4\int_0^\infty \int_0^\infty e^{-(q^2 + p^2)}q^{2m-1}p^{2n-1}\dl{p}\dl{q}                                                                               \\
                       & = 4\int_0^{\frac{\pi}{2}}\int_0^\infty e^{-r^2}r^{2(m + n - 1)}\cos^{2m-1}\theta\sin^{2n-1}\theta r\dl{r}\dl{\theta} & (q = r\cos\theta, p = r\sin\theta) \\
                       & = 4\int_0^\infty e^{-r^2}r^{2(m + n) - 1}\dl{r}\int_0^{\frac{\pi}{2}}\cos^{2m-1}\theta\sin^{2n-1}\theta\dl{\theta}                                        \\
                       & = 2\int_0^\infty e^{-R}R^{m + n - 1}r\frac{1}{2r}\dl{R}B(m, n)                                                       & (R = r^2)                          \\
                       & = \int_0^\infty e^{-R}R^{m + n - 1}\dl{R}B(m, n)                                                                                                          \\
                       & = \Gamma(m + n)B(m, n)
  \end{align}
  これより
  \begin{align}
    B(m, n) & = \frac{\Gamma(m)\Gamma(n)}{\Gamma(m + n)}
  \end{align}
  \begin{align}
    \Gamma(z)\Gamma(1 - z) & = \Gamma(1)B(z, 1 - z)                \\
                           & = \int_{0}^{1}t^{z-1}(1-t)^{-z}\dl{t} \\
                           & = \frac{\pi}{\sin\pi z}
  \end{align}
\end{proof}

\begin{proposition}[ガウスの公式 (Gauss's formula)]
  \begin{align}
    \Gamma(z) & = \lim_{n\to\infty}\frac{n!n^z}{z(z+1)\cdots(z+n)} \\
  \end{align}
\end{proposition}
\begin{proof}
\end{proof}

\begin{proposition}[Legendre の倍数公式]
  \begin{align}
    \Gamma(2z) & = \frac{2^{2z}}{2\sqrt{\pi}}\Gamma(z)\Gamma\ab(z + \frac{1}{2})
  \end{align}
\end{proposition}


\subsection{$n$ 次元超球の体積と表面積}
\begin{theorem}
  \begin{align}
    V_n(R) & = R^n\frac{\pi^{\frac{n}{2}}}{\Gamma(\frac{n}{2}+1)} = R^n\frac{\pi^{\frac{n}{2}}}{(\frac{n}{2})!}
  \end{align}
\end{theorem}
\begin{proof}
  $n$ 次元超球の体積は次のように表現できる。
  \begin{align}
    V_n(R) = \int_{x_1^2 + x_2^2 + \cdots + x_n^2\leq R^2}\dl{x_1}\dl{x_2}\cdots\dl{x_n}
  \end{align}
  各 $x_i$ を $R$ 倍することで
  \begin{align}
    V_n(R) & = \int_{x_1^2 + x_2^2 + \cdots + x_n^2\leq R^2}\dl{x_1}\dl{x_2}\cdots\dl{x_n}  \\
           & = R^n\int_{x_1^2 + x_2^2 + \cdots + x_n^2\leq 1}\dl{x_1}\dl{x_2}\cdots\dl{x_n} \\
           & = R^nV_n(1)
  \end{align}
  これより $V_n(1)$ を求めればよい。$V_n(1)$ と $V_{n-1}(1)$ の関係を求める。
  \begin{align}
    V_n(1) & = \int_{x_1^2 + x_2^2 + \cdots + x_{n-1}^2\leq 1 - x_n^2}\dl{x_1}\dl{x_2}\cdots\dl{x_n} \\
           & = V_{n-1}(1)\int_{-1}^1\ab(1 - x_n^2)^{\frac{n-1}{2}}\dl{x_n}                           \\
           & = 2V_{n-1}(1)\int_0^{\pi/2}\cos^{n}\theta\dl{\theta}                                    \\
           & = V_{n-1}(1)B\ab(\frac{1}{2}, \frac{n+1}{2})                                            \\
           & = V_{n-1}(1)\frac{\Gamma(\frac{1}{2})\Gamma(\frac{n+1}{2})}{\Gamma(\frac{n}{2}+1)}      \\
           & = V_{n-1}(1)\sqrt{\pi}\frac{\Gamma(\frac{n+1}{2})}{\Gamma(\frac{n+2}{2})}
  \end{align}
  これより連続に適用することで
  \begin{align}
    V_n(1) & = V_{1}(1)\pi^{\frac{n-1}{2}}\frac{\Gamma(\frac{3}{2})}{\Gamma(\frac{4}{2})}\frac{\Gamma(\frac{4}{2})}{\Gamma(\frac{5}{2})}\cdots\frac{\Gamma(\frac{n+1}{2})}{\Gamma(\frac{n+2}{2})} \\
           & = 2\pi^{\frac{n-1}{2}}\frac{\Gamma(\frac{3}{2})}{\Gamma(\frac{n+2}{2})}                                                                                                              \\
           & = \frac{\pi^{\frac{n}{2}}}{\Gamma(\frac{n}{2}+1)} = \frac{\pi^{\frac{n}{2}}}{(\frac{n}{2})!}
  \end{align}
  となり、超球の体積は
  \begin{align}
    V_n(R) & = R^n\frac{\pi^{\frac{n}{2}}}{\Gamma(\frac{n}{2}+1)} = R^n\frac{\pi^{\frac{n}{2}}}{(\frac{n}{2})!}
  \end{align}
\end{proof}

\begin{theorem}
  表面積
  \begin{align}
    S_n(R) = R^{n-1}\frac{2\pi^{\frac{\pi}{2}}}{\Gamma(\frac{n}{2})}
  \end{align}
\end{theorem}
\begin{proof}

\end{proof}

\begin{proposition}

\end{proposition}

\subsection{超幾何関数}
\begin{definition}
  超幾何関数
  \begin{align}
    x(1 - x)y'' + [c - (a + b + 1)x]y' - aby = 0
  \end{align}
\end{definition}
\begin{proposition}
  \begin{align}
    e^x         & = \lim_{b\to\infty}{}_2F_1\ab(1,b,1;\frac{x}{b}) \\
    \log(1 + x) & = x\cdot{}_2F_1(1,1,2;-x)
  \end{align}
\end{proposition}

\subsection{Bernoulli 数}
\begin{definition}[Bernoulli 数]
  Bernoulli 数 $B_n$ は以下の正則関数の多項式展開の係数として定義される.
  \begin{align}
    \frac{x}{e^x - 1} = \sum_{n=0}^{\infty}\frac{B_n}{n!}x^n.
  \end{align}
\end{definition}

\begin{proposition}
  \begin{align}
    B_1 = -\frac{1}{2}, B_{2n+1} = 0 \qquad (n = 1,2,3,\ldots).
  \end{align}
\end{proposition}
\begin{proof}
  まず Bernoulli の定義式の両辺に $x/2$ を加える。
  \begin{align}
    \frac{x}{e^x - 1} + \frac{x}{2} = \frac{x}{2} + \sum_{n=0}^{\infty}\frac{B_n}{n!}x^n
  \end{align}
  このとき左辺は偶関数となる。
  \begin{align}
    \frac{x}{e^x - 1} + \frac{x}{2} & = \frac{x(e^x + 1)}{2(e^x - 1)} = \frac{x}{2}\frac{e^{x/2} + e^{-x/2}}{e^{x/2} - e^{-x/2}} = \frac{x}{2}\coth(\frac{x}{2})                             \\
    \frac{-x}{2}\coth(\frac{-x}{2}) & = \frac{-x}{2}\frac{e^{-x/2} + e^{x/2}}{e^{-x/2} - e^{x/2}} = \frac{x}{2}\frac{e^{x/2} + e^{-x/2}}{e^{x/2} - e^{-x/2}} = \frac{x}{2}\coth(\frac{x}{2})
  \end{align}
  これより次の右辺も偶関数であることがわかり、一致の定理から右辺について奇数次の項は現れない。よって 3 以上の奇数を添え字に持つ Bernoulli 数はゼロとなる。
  \begin{align}
    B_1 & = -\frac{1}{2}, \qquad B_{2n+1} = 0 \qquad (n = 1,2,3,\ldots)
  \end{align}
\end{proof}
\begin{theorem}
  \begin{align}
    \sum_{m=0}^{n-1}\frac{B_n}{(n - m)!m!}x^n = \delta_{n,1} \qquad (n = 1,2,3,\ldots).
  \end{align}
\end{theorem}
\begin{proof}
  定義式の左辺の分母を払って展開すると
  \begin{align}
    x & = (e^x - 1)\sum_{n=0}^{\infty}\frac{B_n}{n!}x^n                                   \\
      & = \ab(\sum_{k=1}^{\infty}\frac{x^k}{k!})\ab(\sum_{n=0}^{\infty}\frac{B_n}{n!}x^n) \\
      & = \sum_{k=1}^{\infty}\sum_{n=0}^{\infty}\frac{B_n}{k!n!}x^{k+n}                   \\
      & = \sum_{n=1}^{\infty}\sum_{m=0}^{n-1}\frac{B_n}{(n - m)!m!}x^n.
  \end{align}
  となり両辺の係数を比較することで次のようになる。
  \begin{align}
    \sum_{m=0}^{n-1}\frac{B_n}{(n - m)!m!}x^n = \delta_{n,1} \qquad (n = 1,2,3,\ldots).
  \end{align}
\end{proof}

\begin{proposition}
  \begin{align}
    B_0 = 1, B_1 = -\frac{1}{2}, B_2 = \frac{1}{6}, B_3 = 0, B_4 = -\frac{1}{30}, B_5 = 0, B_6 = \frac{1}{42}, \cdots
  \end{align}
\end{proposition}
\begin{proof}
  上の定理について具体的式を求めると
  \begin{align}
     & B_0 = 1                                                                                                                                 \\
     & \frac{1}{2}B_0 + B_1 = 0                                                                                                                \\
     & \frac{1}{6}B_0 + \frac{1}{2}B_1 + \frac{1}{2}B_2 = 0                                                                                    \\
     & \frac{1}{24}B_0 + \frac{1}{6}B_1 + \frac{1}{4}B_2 + \frac{1}{6}B_3 = 0                                                                  \\
     & \frac{1}{120}B_0 + \frac{1}{24}B_1 + \frac{1}{12}B_2 + \frac{1}{12}B_3 + \frac{1}{24}B_4 = 0                                            \\
     & \frac{1}{720}B_0 + \frac{1}{120}B_1 + \frac{1}{48}B_2 + \frac{1}{36}B_3 + \frac{1}{48}B_4 + \frac{1}{120}B_5 = 0                        \\
     & \frac{1}{5040}B_0 + \frac{1}{720}B_1 + \frac{1}{240}B_2 + \frac{1}{144}B_3 + \frac{1}{144}B_4 + \frac{1}{240}B_5 + \frac{1}{720}B_6 = 0 \\
     & \cdots
  \end{align}
  より添字が奇数のときを代入することで求まる。
  \begin{align}
    B_0 = 1, B_1 = -\frac{1}{2}, B_2 = \frac{1}{6}, B_3 = 0, B_4 = -\frac{1}{30}, B_5 = 0, B_6 = \frac{1}{42}, \cdots.
  \end{align}
\end{proof}



\subsection{ゼータ関数 $\zeta(s)$}
\begin{definition}[ゼータ関数]
  ゼータ関数 $\zeta(s)$ は次のように定義される.
  \begin{align}
    \zeta(s) & := \sum_{n=1}^{\infty}\frac{1}{n^s} \qquad (\real s > 1).
  \end{align}
\end{definition}

\begin{proposition}
  $\zeta(s)$ が $\real s > 1$ において一様絶対収束することを示す.
\end{proposition}
\begin{proof}
  $s = a + bi\ (a > 1)$ とおく. すると次のようになる.
  \begin{align}
    |\zeta(s)| & \leq \sum_{n=1}^{\infty}\ab|\frac{1}{n^s}| = \sum_{n=1}^{\infty}\frac{1}{n^a} \approx \int_{1}^{\infty}\dl{x}x^{-a} = \ab[\frac{1}{1 - a}x^{1-a}]_1^\infty < \infty.
  \end{align}
  よってゼータ関数 $\zeta(s)$ は一様絶対収束する.
\end{proof}

\begin{proposition}
  \begin{align}
    \zeta(s) & = \prod_{p:prime}\frac{1}{1 - p^{-s}} \qquad (\real s > 1).
  \end{align}
\end{proposition}
\begin{proof}
  素因数分解の一意性より次のようにゼータ関数 $\zeta(s)$ は式変形できる.
  \begin{align}
    \zeta(s) & = \sum_{n=1}^{\infty}\frac{1}{n^s}                                                                                   \\
             & = \frac{1}{1^s} + \frac{1}{2^s} + \frac{1}{3^s} + \frac{1}{2^{2s}} + \frac{1}{5^s} + \frac{1}{(2\cdot 3)^s} + \cdots \\
             & = \ab(1 + 2^{-s} + 2^{-2s} + \cdots)\ab(1 + 3^{-s} + 3^{-2s} + \cdots)\ab(1 + 5^{-s} + 5^{-2s} + \cdots)\cdots       \\
             & = \prod_{p:prime}(1 + p^{-s} + p^{-2s} + \cdots)                                                                     \\
             & = \prod_{p:prime}\frac{1}{1 - p^{-s}}.
  \end{align}
\end{proof}

\begin{proposition}
  \begin{align}
    \zeta(s) & = 0 \implies \real s \leq 1.
  \end{align}
\end{proposition}
\begin{proof}
  $\real s > 1$ において $s = a + b\sqrt{-1}\ (a > 1)$ とおくと $p^{-s}$ の大きさは次のように評価される.
  \begin{align}
    |p^{-s}| = |p^{-a-b\sqrt{-1}}| = |p^{-a}|\cdot|e^{-\sqrt{-1}b\ln p}| = p^{-a}.
  \end{align}
  これより $\zeta(s)$ の大きさは次のように評価される.
  \begin{align}
    |\zeta(s)| & = \ab|\prod_{p:prime}\frac{1}{1 - p^{-s}}| \geq \prod_{p:prime}\frac{1}{1 - |p^{-s}|} = \prod_{p:prime}\frac{1}{1 - p^{-a}} > 0.
  \end{align}
  よって $\real s > 1$ において $\zeta(s)$ はゼロとならない. つまり次のようになる.
  \begin{align}
    \zeta(s) & = 0 \implies \real s \leq 1.
  \end{align}
\end{proof}

\begin{proposition}
  素数が無限に存在することを示す.
\end{proposition}
\begin{proof}
  ゼータ関数 $\zeta(s)\ (\real s > 1)$ について $s\to 1$ の極限を取ると発散する.
  \begin{align}
    \lim_{s\to 1}\zeta(s) & = \lim_{s\to 1}\sum_{n=1}^{\infty}\frac{1}{n^s} = \infty.
  \end{align}
  また Euler 積表示についても極限を取る.
  \begin{align}
    \lim_{s\to 1}\zeta(s) & = \prod_{p:prime}\frac{1}{1 - 1/p}.
  \end{align}
  ここで素数が有限個しかないならば発散しない. ただゼータ関数は極限を取ると発散するので素数は無限個存在する.
\end{proof}

\begin{proposition}
  \begin{align}
    \Gamma(s)\zeta(s) = \int_0^\infty\dl{x}\frac{x^{s-1}}{e^x - 1} \qquad (\real s > 1).
  \end{align}
\end{proposition}
\begin{proof}
  ガンマ関数の定義式について $x := nx$ と置換積分することで次のように式変形できる.
  \begin{align}
    \Gamma(s)         & = \int_0^\infty \dl{x}x^{s-1}e^{-x}                     \\
                      & = \int_0^\infty n\dl{x}\ab(nx)^{s-1}e^{-nx},            \\
    \Gamma(s)\zeta(s) & = \sum_{n=1}^{\infty}\frac{\Gamma(s)}{n^s}              \\
                      & = \sum_{n=1}^{\infty}\int_0^\infty \dl{x}x^{s-1}e^{-nx} \\
                      & = \int_0^\infty\dl{x}\frac{x^{s-1}}{e^x - 1}.
  \end{align}
\end{proof}

\begin{proposition}
  この積分値を求める為に複素解析を用いる. 積分路 $C$ を $C = C(\delta) = C_+(\delta) + C_0(\delta) + C_+(\delta)$ として $C_+(\delta)$ は実軸上無限遠から原点から $\delta$ の距離にある点まで, $C_0(\delta)$ は中心を原点とする半径 $\delta$ の円を反時計回りに 1 周し, $C_-(\delta)$ は実軸上原点から $\delta$ の距離にある点から無限遠までを積分する. また次の関数 $I(s; C)$ を定義しておく.
  \begin{align}
    I(s; C) & := \int_C\dl{z}\frac{z^{s-1}}{e^z - 1}.
  \end{align}
  $0 <\delta < 2\pi$ を満たす範囲で $\delta$ を動かしても積分値は一定である.
  $\real s > 1$ のとき $\delta\to 0$ とすると $C_0(\delta)$ に沿った積分 $I(s;C_0(\delta))$ がゼロになる.
\end{proposition}
\begin{proof}
  被積分関数は $2n\pi\sqrt{-1}$ について 1 位の極がある. これより留数定理から積分路の内部の極の数が変化しないなら積分値は一定である. よって $0 <\delta < 2\pi$ を満たす範囲で $\delta$ を動かしても極の数は変化しないから積分値は一定である.
  \begin{align}
    |I(s;C_0(\delta))| & = \ab|\int_{C_0(\delta)}\dl{z}\frac{z^{s-1}}{e^z - 1}|                                                                        \\
                       & = \ab|\int_0^{2\pi}\delta ie^{i\theta}\dl{\theta}\frac{(\delta e^{i\theta})^{s-1}}{e^{\delta(\cos\theta + i\sin\theta)} - 1}| \\
                       & \leq \int_0^{2\pi}\dl{\theta}\frac{|\delta^s|}{e^{\delta\cos\theta} - 1}                                                      \\
                       & < |\delta^{s-1}|\pi.
  \end{align}
  これより $\delta\to 0$ のとき積分値 $I(s; C_0(\delta))$ は $0$ となる.
\end{proof}

\begin{proposition}
  \begin{align}
    I(s; C) & = (e^{2\pi is} - 1)\int_0^\infty\dl{x}\frac{x^{s-1}}{e^x - 1}.
  \end{align}
\end{proposition}
\begin{proof}
  Q 17A-10 の考察から $\delta\to 0$ の極限において積分 $I(s; C)$ を考える.
  \begin{align}
    I(s; C) & = \int_{C(\delta)}\dl{z}\frac{z^{s-1}}{e^z - 1}                                                                               \\
            & = \int_{C_- + C_0 + C_+}\dl{z}\frac{z^{s-1}}{e^z - 1}                                                                         \\
            & = \int_{C_-}\dl{z}\frac{z^{s-1}}{e^z - 1} + \int_{C_0}\dl{z}\frac{z^{s-1}}{e^z - 1} + \int_{C_+}\dl{z}\frac{z^{s-1}}{e^z - 1} \\
            & = e^{2\pi is}\int_{C_+}\dl{z}\frac{z^{s-1}}{e^z - 1} + 0 + \int_{C_+}\dl{z}\frac{z^{s-1}}{e^z - 1}                            \\
            & = (e^{2\pi is} - 1)\int_{0}^\infty\dl{x}\frac{x^{s-1}}{e^x - 1}.
  \end{align}
\end{proof}

\begin{itembox}[l]{Q 17A-12.}
  \begin{align}
    \zeta(s) & = \frac{1}{(e^{2\pi is} - 1)\Gamma(s)}I(s; C).
  \end{align}
\end{itembox}

(i) 17A-11 より$\real s > 1$ において次が成り立つ.
\begin{align}
  \Gamma(s)\zeta(s) & = \int_0^\infty\dl{x}\frac{x^{s-1}}{e^x - 1}                        \\
                    & = \frac{I(s; C)}{e^{2\pi is} - 1},                                  \\
  \zeta(s)          & = \frac{1}{(e^{2\pi is} - 1)\Gamma(s)}I(s; C) \qquad (\real s > 1).
\end{align}

(ii)
$I(s; C)$ は次のように定義された.
\begin{align}
  I(s; C) & = \int_{C(\delta)}\dl{z}\frac{z^{s-1}}{e^z - 1}.
\end{align}
これは複素平面全体 $s\in\CC$ に対して正則である. よって (i) で求めた式は $\real s > 1$ の条件を取り外すことができ, 解析接続となる.

\begin{itembox}[l]{Q 17A-13.}
  \begin{align}
    \zeta(s) & = e^{-\pi is}\Gamma(1 - s)\frac{1}{2\pi i}I(s; C).
  \end{align}
\end{itembox}

さらに次のガンマ関数 $\Gamma(s)$ の反転公式より
\begin{align}
  \Gamma(s)\Gamma(1-s) & = \frac{\pi}{\sin\pi s}.
\end{align}
ゼータ関数は次のように表される.
\begin{align}
  \zeta(s) & = \frac{1}{(e^{2\pi is} - 1)\Gamma(s)}I(s; C)                                          \\
           & = \frac{\sin\pi s}{\pi(e^{2\pi is} - 1)}\Gamma(1 - s)I(s; C)                           \\
           & = \frac{e^{i\pi s} - e^{-i\pi s}}{e^{2\pi is} - 1}\Gamma(1 - s)\frac{1}{2\pi i}I(s; C) \\
           & = e^{-\pi is}\Gamma(1 - s)\frac{1}{2\pi i}I(s; C).
\end{align}



\section{微分方程式}


\subsection{エルミート多項式}
\begin{definition}[エルミート多項式]
  次の級数展開の右辺に現れる $H_n(x)$ をエルミート多項式 (Hermite polynomials) という。
  \begin{align}
    e^{-t^2 + 2tx} & = \sum_{n=0}^{\infty}\frac{1}{n!}H_n(x)t^n
  \end{align}
  また左辺の関数はエルミート多項式の母関数 (generating function) という。
\end{definition}

\begin{theorem}[ロドリグの公式 (Rodrigues's formula)]
  \begin{align}
    H_n(x) & = (-1)^ne^{x^2}\diff[n]{x}e^{-x^2}
  \end{align}
\end{theorem}
\begin{proof}
  両辺を $t$ で $n$ 階微分する。
  \begin{align}
    \diffp[n]{t}(左辺) & = e^{x^2}\diffp[n]{t}e^{-(t - x)^2} = -e^{x^2}\diffp[n]{x}e^{-(t - x)^2} \\
    \diffp[n]{t}(右辺) & = \sum_{m=n}^{\infty}\frac{1}{(m-n)!}H_{m}(x)t^{m-n}
  \end{align}
  $t = 0$ とすると示せる。
  \begin{align}
    H_{n}(x) = -e^{x^2}\diffp[n]{x}e^{-x^2}
  \end{align}
\end{proof}

\begin{proposition}
  \begin{align}
    H_n(x) & = \frac{n!}{2\pi i}\int_C\frac{e^{-z^2 + 2zx}}{z^{n+1}}\dl{z}                               \\
    H_n(x) & = \frac{1}{2\sqrt{\pi}}(-i)^n\int_{-\infty + 2ix}^{\infty + 2ix}e^{-q^2/4}(q + 2ix)^n\dl{q} \\
    H_n(x) & = \sum_{l=0}^{[n/2]}(-1)^l\frac{n!}{(n - 2l)!l!}(2x)^{n - 2l}
  \end{align}
\end{proposition}

\begin{proposition}
  \begin{align}
    H_n'(x)    & = 2xH_n(x) - H_{n+1}(x)   \\
    H_{n+1}(x) & = 2xH_n(x) - 2nH_{n-1}(x) \\
    H_n'(x)    & = 2nH_{n-1}(x)
  \end{align}
\end{proposition}
\begin{proof}

\end{proof}

\begin{theorem}
  \begin{align}
    \diff[2]{x}f(x) - 2x\diff{x}f(x) + 2nf(x) = 0
  \end{align}
\end{theorem}
\begin{proof}

\end{proof}

\begin{theorem}
  \begin{align}
    \int_{-\infty}^\infty H_m(x)H_n(x)e^{-x^2}\dl{x} & = 2^nn!\sqrt{\pi}\delta_{mn}
  \end{align}
\end{theorem}
\begin{proof}

\end{proof}


\subsection{ルジャンドル微分方程式}
\begin{definition}[ルジャンドル微分方程式]
  \begin{align}
    (1 - x^2)y'' - 2xy' + \lambda y = 0
  \end{align}
\end{definition}

\begin{align}
  y = \sum_{j=0}^{\infty}a_jx^j
\end{align}

\begin{definition}[ルジャンドルの陪微分方程式]
  ルジャンドルの陪微分方程式
  \begin{align}
    (1 - x^2)y'' - 2xy' + \ab(n(n + 1) - \frac{m^2}{1 - x^2})y = 0
  \end{align}
  これを満たす独立な 2 つの解 $P_n^m(x)$ と $Q_n^m(x)$ を第一種および第二種ルジャンドル陪関数はルジャンドル関数で表される。
\end{definition}

\subsection{ベッセルの微分方程式}
\begin{definition}
  ベッセルの微分方程式 (Bessel's equation)
  \begin{align}
    x^2y'' + xy' + (x^2 - \nu^2)y = 0
  \end{align}
\end{definition}
\begin{definition}
  ベッセルの微分方程式 (Bessel's equation)
  \begin{align}
    x^2y'' + xy' + (x^2 - \nu^2)y = 0
  \end{align}
\end{definition}

\subsection{ラゲール多項式}
\begin{definition}
  ラゲール多項式
  \begin{align}
    \frac{e^{-xz/(1-z)}}{1-z} & = \sum_{n=0}^{\infty}L_n(x)\frac{z^n}{n!}
  \end{align}
\end{definition}
\begin{proposition}
  \begin{align}
    L_n(x)     & = e^x\diff[n]{x}(x^ne^{-x})                          \\
    L_n(x)     & = \sum_{l=0}^{n}\frac{(-1)^l(n!)^2}{(l!)^2(n-l)!}x^l \\
    L_{n+1}(x) & = (2n + 1 - x)L_n(x) - n^2L_{n-1}(x)                 \\
    xL_n'(x)   & = nL_n(x) - n^2L_{n-1}(x)                            \\
    L_n(0)     & = n!
  \end{align}
\end{proposition}

\subsection{ポアソン方程式}

\subsection{変数分離}

\subsection{境界値問題}
\begin{definition}
  ラプラス方程式 (Laplace equation)
  \begin{align}
    \diffp[2]{u}{x} + \diffp[2]{u}{y} = 0
  \end{align}
  ポアソン方程式 (Poisson equation)
  \begin{align}
    \diffp[2]{u}{x} + \diffp[2]{u}{y} = -\rho(x, y)
  \end{align}
  波動方程式 (wave equation)
  \begin{align}
    \frac{1}{c^2}\diffp[2]{u}{t} = \diffp[2]{u}{x}
  \end{align}
  熱伝導方程式 (heat conduction equation) \\
  $\kappa$ を熱伝導率 (thermal conductivity)
  \begin{align}
    \diffp{u}{t} & = \kappa\diffp[2]{u}{x} + q(x)
  \end{align}
\end{definition}

\begin{proposition}
  ラプラス方程式を満たし
  \begin{align}
    \diffp[2]{u}{x} + \diffp[2]{u}{y} = 0
  \end{align}
  次の境界条件を満たす関数 $u(x, y)$ を求める。
  \begin{align}
    u(0, y) = 0, u(a, y) = 0, u(x, 0) = f(x), u(x, b) = 0
  \end{align}
\end{proposition}
\begin{proof}
  これは変数分離法が使えないと思う。
  \begin{align}
    u(x, y) = X(x)Y(y)
  \end{align}
  ラプラス方程式
  \begin{align}
    X''(x)Y(y)          & + X(x)Y''(y) = 0                                                      \\
    \frac{X''(x)}{X(x)} & = - \frac{Y''(y)}{Y(y)}                                               \\
    X''(x)              & = - \lambda^2X(x)                                                     \\
    Y''(y)              & = \lambda^2Y(y)                                                       \\
    X(x)                & = \sin(\frac{n\pi x}{a})                                              \\
    \lambda             & = \frac{n\pi}{a}                                                      \\
    f(x)                & = \sum_{n=1}^{\infty}A_n\sin(\frac{n\pi x}{a})\sinh(\frac{n\pi b}{a})
  \end{align}
\end{proof}

\end{document}