\RequirePackage{plautopatch}
\documentclass[uplatex,dvipdfmx,a4paper,11pt]{jlreq}
\usepackage{bxpapersize}
\usepackage[utf8]{inputenc}
\usepackage{fontenc}
\usepackage{lmodern}
\usepackage{otf}
\usepackage{amsmath}
\usepackage{amssymb}
\usepackage{amsthm}
\usepackage{ascmac}
% \usepackage[hyphens]{url}
\usepackage{physics}
\usepackage{braket}
\usepackage{verbatimbox}
\usepackage{bm}
\usepackage{url}
% \usepackage[dvipdfmx,hiresbb,final]{graphicx}
\usepackage{hyperref}
\usepackage{pxjahyper}
\usepackage{tikz}\usetikzlibrary{cd}
\usepackage{listings}
\usepackage{color}
\usepackage{mathtools}
\usepackage{xspace}
\usepackage{xy}
\usepackage{xypic}
%
\title{量子力学}
\author{Anko}
\makeatletter
%
\DeclareMathOperator{\lcm}{lcm}
\DeclareMathOperator{\Kernel}{Ker}
\DeclareMathOperator{\Image}{Im}
\DeclareMathOperator{\ch}{ch}
\DeclareMathOperator{\Aut}{Aut}
\DeclareMathOperator{\Log}{Log}
\DeclareMathOperator{\Arg}{Arg}
\DeclareMathOperator{\sgn}{sgn}
%
\newcommand{\CC}{\mathbb{C}}
\newcommand{\RR}{\mathbb{R}}
\newcommand{\QQ}{\mathbb{Q}}
\newcommand{\ZZ}{\mathbb{Z}}
\newcommand{\NN}{\mathbb{N}}
\newcommand{\FF}{\mathbb{F}}
\newcommand{\PP}{\mathbb{P}}
\newcommand{\GG}{\mathbb{G}}
\newcommand{\TT}{\mathbb{T}}
\newcommand{\calB}{\mathcal{B}}
\newcommand{\calF}{\mathcal{F}}
\newcommand{\ignore}[1]{}
\newcommand{\floor}[1]{\left\lfloor #1 \right\rfloor}
% \newcommand{\abs}[1]{\left\lvert #1 \right\rvert}
\newcommand{\lt}{<}
\newcommand{\gt}{>}
\newcommand{\id}{\mathrm{id}}
\newcommand{\rot}{\curl}
\renewcommand{\angle}[1]{\left\langle #1 \right\rangle}
\newcommand{\EE}{\bm{E}}
\newcommand{\BB}{\bm{B}}
\renewcommand{\AA}{\bm{A}}
\newcommand{\rr}{\bm{r}}
\newcommand{\kk}{\bm{k}}
\newcommand{\pp}{\bm{p}}

\let\oldcite=\cite
\renewcommand\cite[1]{\hyperlink{#1}{\oldcite{#1}}}

\let\oldbibitem=\bibitem
\renewcommand{\bibitem}[2][]{\label{#2}\oldbibitem[#1]{#2}}

% theorem環境の設定
% - 冒頭に改行
% - 末尾にdiamond (amsthm)
\theoremstyle{definition}
\newcommand*{\newscreentheoremx}[2]{
  \newenvironment{#1}[1][]{
    \begin{screen}
    \begin{#2}[##1]
      \leavevmode
      \newline
  }{
    \end{#2}
    \end{screen}
  }
}
\newcommand*{\newqedtheoremx}[2]{
  \newenvironment{#1}[1][]{
    \begin{#2}[##1]
      \leavevmode
      \newline
      \renewcommand{\qedsymbol}{\(\diamond\)}
      \pushQED{\qed}
  }{
      \qedhere
      \popQED
    \end{#2}
  }
}
\newtheorem{theorem*}{定理}

\newqedtheoremx{theorem}{theorem*}
\newcommand*\newqedtheorem@unstarred[2]{%
  \newtheorem{#1*}[theorem*]{#2}
  \newqedtheoremx{#1}{#1*}
}
\newcommand*\newqedtheorem@starred[2]{%
  \newtheorem*{#1*}{#2}
  \newqedtheoremx{#1}{#1*}
}
\newcommand*{\newqedtheorem}{\@ifstar{\newqedtheorem@starred}{\newqedtheorem@unstarred}}

\newtheorem{sctheorem*}{定理}
\newscreentheoremx{sctheorem}{sctheorem*}
\newcommand*\newscreentheorem@unstarred[2]{%
  \newtheorem{#1*}[theorem*]{#2}
  \newscreentheoremx{#1}{#1*}
}
\newcommand*\newscreentheorem@starred[2]{%
  \newtheorem*{#1*}{#2}
  \newscreentheoremx{#1}{#1*}
}
\newcommand*{\newscreentheorem}{\@ifstar{\newscreentheorem@starred}{\newscreentheorem@unstarred}}

%\newtheorem*{definition}{定義}
%\newtheorem{theorem}{定理}
%\newtheorem{proposition}[theorem]{命題}
%\newtheorem{lemma}[theorem]{補題}
%\newtheorem{corollary}[theorem]{系}

\newqedtheorem{lemma}{補題}
\newqedtheorem{corollary}{系}
\newqedtheorem{example}{例}
\newqedtheorem{proposition}{命題}
\newqedtheorem{remark}{注意}
\newqedtheorem{thesis}{主張}
\newqedtheorem{notation}{記法}
\newqedtheorem{problem}{問題}
\newqedtheorem{algorithm}{アルゴリズム}

\newscreentheorem*{axiom}{公理}
\newscreentheorem*{definition}{定義}

\renewenvironment{proof}[1][\proofname]{\par
  \normalfont
  \topsep6\p@\@plus6\p@ \trivlist
  \item[\hskip\labelsep{\bfseries #1}\@addpunct{\bfseries}]\ignorespaces\quad\par
}{%
  \qed\endtrivlist\@endpefalse
}
\renewcommand\proofname{証明}

\makeatother

\begin{document}
\maketitle
\tableofcontents
\clearpage

\section{量子力学の基礎}
TODO: 実験的背景: 電子線を用いた二重スリット実験など
\begin{axiom}[粒子の波動性]
  すべての粒子は波動性を持つ。
\end{axiom}

\subsection{波動関数}
\begin{definition}[波動関数]
  波動方程式を満たす関数 $\psi(\rr, t)\in C^1(\CC)$ を粒子の場とし、これを波動関数 (wave function) という。また $\rho(\rr, t) = |\psi(\rr, t)|^2$ を粒子の確率密度 (probability density) と解釈し、次の規格化条件を満たすように波動関数を定義する。
  \begin{align}
    \int\rho(\rr, t)\dd{\rr} = 1
  \end{align}
\end{definition}

光子や電子の性質から粒子の性質と対応付ける。
\begin{definition}[ド・ブロイの関係式]
  光子と同様に任意の粒子は次のような関係式が成り立つとする。
  \begin{align}
    E & = \hbar\omega, \qquad p = \hbar k
  \end{align}
  また質量 $m$ の粒子の持つ力学的エネルギーは運動エネルギーとポテンシャルエネルギーの和で与えられる。
  \begin{align}
    E & = \frac{p^2}{2m} + V(\rr)
  \end{align}
\end{definition}

\begin{theorem}[Schrödinger の方程式]
  このとき波動関数 $\psi(\rr, t)$ について次の関係式が成り立つ。
  \begin{align}
    i\hbar\pdv{t}\psi(\rr, t) & = \qty(-\frac{\hbar^2}{2m}\nabla^2 + V(\rr))\psi(\rr, t)
  \end{align}
\end{theorem}
\begin{proof}
  波動関数は波動方程式を満たすので次のように書ける。
  \begin{align}
    \frac{k^2}{\omega^2}\pdv[2]{\psi(\rr, t)}{t} & = \nabla^2\psi(\rr, t)
  \end{align}
  このときダランベールの解より $|\kk|^2 = k^2$ を満たす $\kk$ を用いて波動関数は $f(\kk\vdot\rr - \omega t)$ の重ね合わせとなる。ここでは特に次の関数となると考える。
  \begin{align}
    \psi(\rr, t) & = \int_{\kk}\tilde{\varphi}(\kk)e^{i(\kk\vdot\rr - \omega t)}\dd{\kk}
  \end{align}
  よって波動方程式は次のようになる。
  \begin{align}
    -k^2\psi(\rr, t)          & = \nabla^2\psi(\rr, t)                                                      \\
    k^2 = \frac{p^2}{\hbar^2} & = \frac{2m}{\hbar^2}(E - V(\rr)) = \frac{2m}{\hbar^2}(\hbar\omega - V(\rr)) \\
    \hbar\omega\psi(\rr, t)   & = \qty(-\frac{\hbar^2}{2m}\nabla^2 + V(\rr))\psi(\rr, t)                    \\
    i\hbar\pdv{t}\psi(\rr, t) & = \qty(-\frac{\hbar^2}{2m}\nabla^2 + V(\rr))\psi(\rr, t)
  \end{align}
\end{proof}

\begin{theorem}
  粒子の確率密度について連続の方程式を満たす。
  \begin{align}
    \pdv{t}\rho(\rr, t) + \vnabla\vdot\bm{j}(\rr, t) = 0
  \end{align}
\end{theorem}
\begin{proof}
  確率密度の時間微分を考えると
  \begin{align}
    \pdv{t}\rho(\rr, t) & = \psi^*(\rr, t)\qty(\pdv{t}\psi(\rr, t)) + \qty(\pdv{t}\psi^*(\rr, t))\psi(\rr, t)                                                                                                      \\
                        & = \psi^*(\rr, t)\qty(-\frac{i}{\hbar}\qty(-\frac{\hbar^2}{2m}\nabla^2 + V(\rr))\psi(\rr, t)) + \qty(\frac{i}{\hbar}\qty(-\frac{\hbar^2}{2m}\nabla^2 + V(\rr))\psi^*(\rr, t))\psi(\rr, t) \\
                        & = \frac{i\hbar}{2m}\qty(\psi^*(\rr, t)\nabla^2\psi(\rr, t) - \nabla^2\psi^*(\rr, t)\psi(\rr, t))                                                                                         \\
                        & = \frac{i\hbar}{2m}\vnabla\vdot\qty(\psi^*(\rr, t)\vnabla\psi(\rr, t) - \vnabla\psi^*(\rr, t)\psi(\rr, t))
  \end{align}
  より確率の流れ $\bm{j}(\rr, t)$ を次のように解釈する。
  \begin{align}
    \bm{j}(\rr, t) & := -\frac{i\hbar}{2m}\qty(\psi^*(\rr, t)\vnabla\psi(\rr, t) - \vnabla\psi^*(\rr, t)\psi(\rr, t))
  \end{align}
  これより連続の方程式を満たす。
  \begin{align}
    \pdv{t}\rho(\rr, t) + \vnabla\vdot\bm{j}(\rr, t) = 0
  \end{align}
\end{proof}

\begin{theorem}
  粒子の全存在確率は保存する。
\end{theorem}
\begin{proof}
  \begin{align}
    \dv{t}\int\rho(\rr, t)\dd{V} & = -\int\vnabla\vdot\bm{j}(\rr, t)\dd{V} = -\lim_{|\rr|\to\infty}\int\bm{j}(\rr, t)\vdot\bm{n}\dd{S} = 0
  \end{align}
\end{proof}

\subsection{期待値と演算子}
\begin{definition}
  物理量 $F$ に対する期待値を次のように定義する。
  \begin{align}
    \ev{F} & := \int\psi^*(\rr, t)F\psi(\rr, t)\dd{\rr}
  \end{align}
\end{definition}
\begin{theorem}
  このとき以下の物理量の期待値は次のようになる。
  \begin{align}
    \ev{\rr} = \ev{\rr}, \qquad \ev{\pp} = \ev{-i\hbar\vnabla}, \qquad m\dv[2]{\ev{\rr}}{t} = - \ev{\vnabla V(\rr, t)}
  \end{align}
\end{theorem}
\begin{proof}
  \begin{align}
    \ev{\rr} & = \int\psi^*(\rr, t)\rr\psi(\rr, t)\dd{\rr} = \int\rr\rho(\rr, t)\dd{\rr}
  \end{align}
  \begin{align}
    \ev{\pp} & = m\dv{\ev{\rr}}{t} = m\dv{t}\int\psi^*(\rr, t)\rr\psi(\rr, t)\dd{\rr}                                                                                                                                       \\
             & = m\int\qty(\psi^*(\rr, t)\rr\pdv{t}\psi(\rr, t) + \pdv{t}\psi^*(\rr, t)\rr\psi(\rr, t))\dd{\rr}                                                                                                             \\
             & = -m\int\psi^*(\rr, t)\rr\frac{i}{\hbar}\qty(-\frac{\hbar^2}{2m}\nabla^2 + V(\rr))\psi(\rr, t)\dd{\rr} + m\int\frac{i}{\hbar}\qty(-\frac{\hbar^2}{2m}\nabla^2 + V(\rr))\psi^*(\rr, t)\rr\psi(\rr, t)\dd{\rr} \\
             & = \frac{i\hbar}{2}\int\psi^*(\rr, t)\rr\nabla^2\psi(\rr, t)\dd{\rr} - \frac{i\hbar}{2}\int\nabla^2\psi^*(\rr, t)\rr\psi(\rr, t)\dd{\rr}                                                                      \\
             & = \frac{i\hbar}{2}\int\psi^*(\rr, t)\rr\nabla^2\psi(\rr, t)\dd{\rr} - \frac{i\hbar}{2}\int\psi^*(\rr, t)(\nabla^2\rr\psi(\rr, t))\dd{\rr}                                                                    \\
             & = -i\hbar\int\psi^*(\rr, t)\vnabla\psi(\rr, t)\dd{\rr}                                                                                                                                                       \\
             & = \ev{-i\hbar\vnabla}
  \end{align}
  \begin{align}
    m\dv[2]{\ev{\rr}}{t} & = \dv{\ev{\pp}}{t} = -i\hbar\dv{t}\int\psi^*(\rr, t)\vnabla\psi(\rr, t)\dd{\rr}                                                                                                                  \\
                         & = -\int\psi^*(\rr, t)\vnabla\qty(i\hbar\pdv{t}\psi(\rr, t))\dd{\rr} + \int\qty(-i\hbar\pdv{t}\psi^*(\rr, t))\vnabla\psi(\rr, t)\dd{\rr}                                                          \\
                         & = -\int\psi^*(\rr, t)\vnabla\qty(\qty(-\frac{\hbar^2}{2m}\nabla^2 + V(\rr))\psi(\rr, t))\dd{\rr} + \int\qty(\qty(-\frac{\hbar^2}{2m}\nabla^2 + V(\rr))\psi^*(\rr, t))\vnabla\psi(\rr, t)\dd{\rr} \\
                         & = \frac{\hbar^2}{2m}\int\psi^*(\rr, t)\vnabla\qty(\nabla^2\psi(\rr, t))\dd{\rr} - \int\psi^*(\rr, t)\vnabla\qty(V(\rr)\psi(\rr, t))\dd{\rr}                                                      \\
                         & - \frac{\hbar^2}{2m}\int\qty(\nabla^2\psi^*(\rr, t))\vnabla\psi(\rr, t)\dd{\rr} + \int\qty(V(\rr)\psi^*(\rr, t))\vnabla\psi(\rr, t)\dd{\rr}                                                      \\
                         & = - \int\psi^*(\rr, t)\vnabla V(\rr, t)\psi(\rr, t)\dd{\rr}                                                                                                                                      \\
                         & = - \ev{\vnabla V(\rr, t)}
  \end{align}
\end{proof}

このようなことから物理量に対して演算子を定義する。
\begin{definition}
  位置演算子 $\hat{\rr}$、運動量演算子 $\hat{\pp}$、ハミルトニアン $\hat{H}$ を次のように定義する。
  \begin{align}
    \hat{\rr} := \rr, \qquad \hat{\pp} := -i\hbar\vnabla, \qquad \hat{H} := \frac{\hat{\pp}^2}{2m} + V(\hat{\rr}, t)
  \end{align}
  ただし任意の演算子はエルミート演算子であるとする。
  \begin{align}
    \int\phi^*(\rr, t)\hat{F}\psi(\rr, t)\dd{\rr} = \int(\hat{F}\phi(\rr, t))^*\psi(\rr, t)\dd{\rr}
  \end{align}
  これより期待値は実数である。
  \begin{align}
    \hat{H}\psi(\rr) = E\psi(\rr)
  \end{align}
\end{definition}

\begin{definition}[固有関数、固有値]
  次のように演算子 $\hat{F}$ に対して定数倍を除いて波動関数が変化しないとき、波動関数 $\psi_f(\rr, t)$ を演算子 $\hat{F}$ の固有関数、定数 $f$ を固有値と呼ぶ。
  \begin{align}
    \hat{F}\psi_f(\rr, t) & = f\psi_f(\rr, t)
  \end{align}
\end{definition}

\begin{theorem}
  エルミート演算子 $\hat{F}$ において異なる固有値 $f, f'$ を持つ固有関数 $\psi_f(\rr, t), \psi_{f'}(\rr, t)$ は直交する。
\end{theorem}
\begin{proof}
  エルミート演算子の性質より
  \begin{align}
    \int\psi_{f'}^*(\rr, t)\hat{F}\psi_f(\rr, t)  & = \int\qty(\hat{F}\psi_{f'}(\rr, t))^*\psi_f(\rr, t) \\
    f\int\psi_{f'}^*(\rr, t)\psi_f(\rr, t)        & = f'\int\psi_{f'}^*(\rr, t)\psi_f(\rr, t)            \\
    (f - f')\int\psi_{f'}^*(\rr, t)\psi_f(\rr, t) & = 0                                                  \\
    \int\psi_{f'}^*(\rr, t)\psi_f(\rr, t)         & = 0
  \end{align}
\end{proof}

\begin{theorem}[不確定性原理]
  ある波動関数においてある2つの物理量の標準偏差の積は一定値以上である。
  \begin{align}
    \Delta r_i\Delta p_j \geq \frac{\hbar}{2}\delta_{ij}
  \end{align}
\end{theorem}
\begin{proof}
  波動関数が次のような関数のとき
  \begin{align}
    \Psi(\rr, t) & := (is(\hat{r}_i - \ev{\hat{r}_i}) + (\hat{p}_j - \ev{\hat{p}_j}))\psi(\rr, t)
  \end{align}

  \begin{align}
    \int|\Psi(\rr, t)|^2\dd{\rr} & = \int\Psi^*(\rr, t)\Psi(\rr, t)\dd{\rr}                                                                                                                                                                                               \\
                                 & = \int(is(\hat{r}_i - \ev{\hat{r}_i}) + (\hat{p}_j - \ev{\hat{p}_j})\psi(\rr, t))^*(is(\hat{r}_i - \ev{\hat{r}_i}) + (\hat{p}_j - \ev{\hat{p}_j})\psi(\rr, t))\dd{\rr}                                                                 \\
                                 & = \int\psi^*(\rr, t)(-is(\hat{r}_i - \ev{\hat{r}_i}) + (\hat{p}_j - \ev{\hat{p}_j}))(is(\hat{r}_i - \ev{\hat{r}_i}) + (\hat{p}_j - \ev{\hat{p}_j}))\psi(\rr, t)\dd{\rr}                                                                \\
                                 & = \int\psi^*(\rr, t)(s^2(\hat{r}_i - \ev{\hat{r}_i})^2 - is(\hat{r}_i - \ev{\hat{r}_i})(\hat{p}_j - \ev{\hat{p}_j}) + is(\hat{p}_j - \ev{\hat{p}_j})(\hat{r}_i - \ev{\hat{r}_i}) + (\hat{p}_j - \ev{\hat{p}_j})^2)\psi(\rr, t)\dd{\rr} \\
                                 & = s^2\ev{(\hat{r}_i - \ev{\hat{r}_i})^2} + s\hbar\delta_{ij} + \ev{(\hat{p}_j - \ev{\hat{p}_j})^2}                                                                                                                                     \\
                                 & = s^2\Delta r_i^2 + s\hbar\delta_{ij} + \Delta p_j^2                                                                                                                                                                                   \\
                                 & = \qty(s + \frac{\hbar\delta_{ij}}{2\Delta r_i^2})^2\Delta r_i^2 - \frac{\hbar^2\delta_{ij}}{4\Delta r_i^2} + \Delta p_j^2 \geq 0                                                                                                      \\
  \end{align}
  $s = \frac{\hbar\delta_{ij}}{2\Delta r_i^2}$ と代入すると
  \begin{align}
    \Delta r_i\Delta p_j \geq \frac{\hbar}{2}\delta_{ij}
  \end{align}
\end{proof}

\section{時間に依存しないポテンシャル}
ここではポテンシャルが時間に依存せず、シュレーディンガー方程式が時間に依存しないときを考える。時間成分について波動関数は $\psi(\rr, t) = \varphi(\rr)e^{-i\omega t}$ と分けられるからシュレーディンガー方程式は次のように書ける。
\begin{align}
  \hbar\omega\varphi(\rr) & = \qty(-\frac{\hbar^2}{2m}\vnabla^2 + V(\rr))\varphi(\rr) \\
  \hat{H}\varphi(\rr)     & = E\varphi(\rr)
\end{align}

\begin{definition}[エネルギーの縮退]
  基底状態はエネルギーが最小となるパラメータの状態、第 $n$ 励起状態は基底状態の次に $n$ 番目に低いエネルギーの状態である。3 次元系の場合、異なる状態で同じエネルギーを持つことがある。これをエネルギーの縮退という。
\end{definition}

\subsection{有限ポテンシャル}
ポテンシャルが有限のとき
\begin{align}
  \lim_{\epsilon\to +0}\vnabla\psi(\rr, t)|_{\rr_0}^{\rr_0 + \epsilon} & = \lim_{\epsilon\to +0}\int_{\rr_0}^{\rr_0 + \epsilon}\nabla^2\psi(\rr, t)                        \\
                                                                       & = -\frac{2m}{\hbar^2}\lim_{\epsilon\to +0}\int_{\rr_0}^{\rr_0 + \epsilon}(E - V(\rr))\psi(\rr, t) \\
                                                                       & \to 0
\end{align}
ポテンシャルが空間反転対称性をもつとき
\begin{align}
  E\varphi(-\rr) & = \qty(-\frac{\hbar^2}{2m}(-\nabla)^2 + V(-\rr))\varphi(-\rr) \\
                 & = \qty(-\frac{\hbar^2}{2m}\nabla^2 + V(\rr))\varphi(-\rr)
\end{align}
より $\varphi(\rr), \varphi(-\rr)$ は解となる。線形従属、線形独立のときを考えると偶関数または奇関数としても一般性は失われない。

\subsection{平面波}
ポテンシャルが全くないとき平面波となる。
\begin{proposition}
  ポテンシャルがないときを考える。
  \begin{align}
    V(x) = 0
  \end{align}
  このとき波数 $\kk$ の波動関数は次のようになる。
  \begin{align}
    \psi_{\kk}(\rr, t) & = \frac{e^{i(\kk\vdot\rr - \omega_{\kk}t)}}{(2\pi)^{3/2}} \qquad \qty(\hbar\omega_{\kk} = \frac{\hbar^2k^2}{2m})
  \end{align}
\end{proposition}
\begin{proof}
  このとき波数 $\kk$ の波動関数は次のようになる。
  \begin{align}
    \psi_{\kk}(\rr, t) & = Ce^{i(\kk\vdot\rr - \omega_{\kk}t)} \qquad \qty(\hbar\omega_{\kk} = \frac{\hbar^2k^2}{2m})
  \end{align}
  一辺の長さ $L$ の箱の中に閉じ込めるという周期境界条件を考える。
  \begin{align}
    \psi(\rr, t) = \psi(\rr + L\bm{e}_i, t) \iff e^{ik_iL} = 1 \iff k_i = \frac{2\pi n_i}{L} \qquad (n_i\in\ZZ)
  \end{align}
  また規格化条件より次のようになる。
  \begin{align}
    \int|\psi_{\kk}(\rr, t)|^2\dd{\rr} = |C|^2L^3 = 1 \iff |C| & = \frac{1}{L^{3/2}}
  \end{align}
  また正規直交関係式より
  \begin{align}
    \int\psi_{\kk'}^*(\rr, t)\psi_{\kk}(\rr, t)\dd{\rr} & = \delta_{\kk\kk'}
  \end{align}
  一辺の長さが無限大の箱を考えるときディラックのデルタ関数を用いると
  \begin{align}
    \lim_{L\to\infty}\int\psi_{\kk'}^*(\rr, t)\psi_{\kk}(\rr, t)\dd{\rr} & = |C|^2e^{-i(\omega_{\kk} - \omega_{\kk'})t}\lim_{L\to\infty}\int e^{i(\kk - \kk')\vdot\rr}\dd{\rr}                                            \\
                                                                         & = |C|^2e^{-i(\omega_{\kk} - \omega_{\kk'})t}\prod_{i=x,y,z}\lim_{L\to\infty}\int_{-L/2}^{L/2} e^{i(k_i - k_i')r_i}\dd{r_i}                     \\
                                                                         & = |C|^2e^{-i(\omega_{\kk} - \omega_{\kk'})t}\prod_{i=x,y,z}\lim_{L\to\infty}\frac{e^{i(k_i - k_i')L/2} - e^{-i(k_i - k_i')L/2}}{i(k_i - k_i')} \\
                                                                         & = |C|^2e^{-i(\omega_{\kk} - \omega_{\kk'})t}\prod_{i=x,y,z}2\pi\lim_{L\to\infty}\frac{\sin((k_i - k_i')L/2)}{\pi(k_i - k_i')}                  \\
                                                                         & = |C|^2e^{-i(\omega_{\kk} - \omega_{\kk'})t}\prod_{i=x,y,z}2\pi\delta(k_i - k_i')                                                              \\
                                                                         & = |C|^2e^{-i(\omega_{\kk} - \omega_{\kk'})t}(2\pi)^3\delta(\kk - \kk')                                                                         \\
                                                                         & = e^{-i(\omega_{\kk} - \omega_{\kk'})t}\delta(\kk - \kk') \qquad \qty(\because |C| = \frac{1}{(2\pi)^{3/2}})                                   \\
                                                                         & = \delta(\kk - \kk')
  \end{align}
\end{proof}


\subsection{剛体壁ポテンシャル}
\begin{proposition}
  中心から距離 $L$ 以降には粒子が入れないような 1 次元ポテンシャルを考える。
  \begin{align}
    V(x) & = \begin{cases}
               + \infty & (|x| > L) \\
               0        & (|x| < L)
             \end{cases}
  \end{align}
  このとき固有関数、固有エネルギーは次のようになる。
  \begin{align}
    \varphi_n(x) & = \begin{dcases}
                       \frac{1}{\sqrt{L}}\cos(\frac{n\pi}{2L}x) & (n = 1,3,5,\ldots) \\
                       \frac{1}{\sqrt{L}}\sin(\frac{n\pi}{2L}x) & (n = 2,4,6,\ldots)
                     \end{dcases} \\
    E_n          & = \frac{\hbar^2}{2m}\qty(\frac{n\pi}{2L})^2
  \end{align}
\end{proposition}
\begin{proof}
  $|x| > L$ においてポテンシャルの深さが無限大となるので粒子は侵入出来ない為に波動関数はゼロとなる。また $|x| < L$ においては次の微分方程式となる。
  \begin{align}
    -\frac{\hbar^2}{2m}\dv[2]{x}\varphi(x) & = E\varphi(x) & (|x| < L)
  \end{align}
  これより波動関数の解は次のようになる。
  \begin{align}
    \varphi(x) & = \begin{dcases}
                     0                    & (|x| > L) \\
                     Ae^{ikx} + Be^{-ikx} & (|x| < L)
                   \end{dcases} \\
    E          & = \frac{\hbar^2k^2}{2m} > 0
  \end{align}
  波動関数は連続的につながっていなければならないので
  \begin{align}
    \varphi(\pm L) = 0 & \iff \begin{cases}
                                Ae^{ikL} + Be^{-ikL} = 0 \\
                                Ae^{-ikL} + Be^{ikL} = 0
                              \end{cases}  \\
                       & \iff \begin{cases}
                                Ae^{2ikL} + B = 0 \\
                                Ae^{-2ikL} + B = 0
                              \end{cases}         \\
                       & \iff \begin{cases}
                                Ae^{2ikL} + B = 0 \\
                                Ae^{-2ikL} - Ae^{2ikL} = 0
                              \end{cases} \\
                       & \iff \begin{cases}
                                B = -Ae^{2ikL} \\
                                A(e^{4ikL} - 1) = 0
                              \end{cases}
  \end{align}
  ここで $A = B = 0$ となる解は意味を成さないので排除すると次のように $k$ が離散化される。
  \begin{align}
    e^{4ikL} = 1 & \iff k = \frac{n\pi}{2L} \qquad (n = 1,2,\cdots)
  \end{align}
  これより波動関数は次のようになる。
  \begin{align}
    B = - Ae^{in\pi}                                                & = \begin{cases}
                                                                          +A & (n = 1,3,5,\ldots) \\
                                                                          -A & (n = 2,4,6,\ldots)
                                                                        \end{cases}                        \\
    \varphi_n(x) = Ae^{i\frac{n\pi}{2L}x} + Be^{-i\frac{n\pi}{2L}x} & = \begin{dcases}
                                                                          2A\cos(\frac{n\pi}{2L}x)  & (n = 1,3,5,\ldots) \\
                                                                          2Ai\sin(\frac{n\pi}{2L}x) & (n = 2,4,6,\ldots)
                                                                        \end{dcases}
  \end{align}
  最後に $A$ を規格化条件
  \begin{align}
    \int_{-\infty}^\infty |\varphi_n(x)|^2 = \int_{-L}^L |\varphi_n(x)|^2 = (2A)^2L = 1
  \end{align}
  より決定すると、固有関数とエネルギー固有値は
  \begin{align}
    \varphi_n(x) & = \begin{dcases}
                       \frac{1}{\sqrt{L}}\cos(\frac{n\pi}{2L}x) & (n = 1,3,5,\ldots) \\
                       \frac{1}{\sqrt{L}}\sin(\frac{n\pi}{2L}x) & (n = 2,4,6,\ldots)
                     \end{dcases} \\
    E_n          & = \frac{\hbar^2}{2m}\qty(\frac{n\pi}{2L})^2
  \end{align}
  のように離散化される。
\end{proof}


\subsection{立方体剛体壁ポテンシャル}
\begin{proposition}
  立方体中にしか粒子が存在しないようなポテンシャルを考える。
  \begin{align}
    V(x) & = \begin{cases}
               0        & (0 < x, y, z < L) \\
               + \infty & (otherwise)
             \end{cases}
  \end{align}
  このとき固有関数、固有エネルギーは次のようになる。
  \begin{align}
    \varphi(x, y, z) & = \qty(\frac{2}{L})^{3/2}\sin(\frac{n_x\pi}{L}x)\sin(\frac{n_y\pi}{L}y)\sin(\frac{n_z\pi}{L}z) \\
    E                & = \frac{\pi^2\hbar^2}{2mL^2}\qty(n_x^2 + n_y^2 + n_z^2)
  \end{align}
\end{proposition}
\begin{proof}
  波動関数を $\psi(\rr) = X(x)Y(y)Z(z)$ と変数分離すると剛体壁ポテンシャルと同様に解ける。
\end{proof}

状態 $(n_x, n_y, n_z)$ について基底状態は $(1, 1, 1)$ の状態であり、第 1 励起状態は $(2, 1, 1), (1, 2, 1), (1, 1, 2)$ の 3 つの状態があり、エネルギーの縮退を起こしている。

\subsection{井戸型ポテンシャル}
\begin{proposition}
  \begin{align}
    V(x) =
    \begin{cases}
      V_0 & (|x| < L) \\
      0   & (|x| > L)
    \end{cases}
  \end{align}
  $V_0 < E < 0$ のとき
\end{proposition}
\begin{proof}
  空間反転対称性より偶関数と奇関数 $\varphi_+(x), \varphi_-(x)$ としてよい。
  \begin{align}
    -\frac{\hbar^2}{2m}\dv[2]{x}\varphi_{\pm}(x) & = (E - V_0)\varphi_{\pm}(x) & (|x| < L) \\
    -\frac{\hbar^2}{2m}\dv[2]{x}\varphi_{\pm}(x) & = E\varphi_{\pm}(x)         & (|x| > L)
  \end{align}
  まず $V_0 < E < 0$ となる場合を考える。
  \begin{align}
    \varphi_{\pm}(x) & = \begin{dcases}
                           Ae^{ikx} + Be^{-ikx}                               & (|x| < L) \\
                           C_\pm e^{\kappa(x - L)} + D_\pm e^{-\kappa(x - L)} & (|x| > L)
                         \end{dcases} \\
    E - V_0          & = \frac{\hbar^2k^2}{2m} > 0, \qquad E = -\frac{\hbar^2\kappa^2}{2m} < 0
  \end{align}
  $|x| < L$ において偶奇性より
  \begin{align}
    \begin{dcases}
      \varphi_+(x) = A_+\cos(kx) \\
      \varphi_-(x) = A_-\sin(kx)
    \end{dcases}
  \end{align}
  境界における波動関数とその微分係数の連続性を要請すると
  \begin{align}
    \begin{dcases}
      \varphi_+(L) = A_+\cos(kL) = C_+ + D_+                \\
      \varphi_-(L) = A_-\sin(kL) = C_- + D_-                \\
      \varphi_+'(L) = -A_+k\sin(kL) = C_+\kappa - D_+\kappa \\
      \varphi_-'(L) = A_-k\cos(kL) = C_+\kappa - D_+\kappa  \\
    \end{dcases}
    \iff
    \begin{dcases}
      2C_+ = A_+\cos(kL) - A_+\frac{k}{\kappa}\sin(kL) \\
      2D_+ = A_+\cos(kL) + A_+\frac{k}{\kappa}\sin(kL) \\
      2C_- = A_-\sin(kL) + A_-\frac{k}{\kappa}\cos(kL) \\
      2D_- = A_-\sin(kL) - A_-\frac{k}{\kappa}\cos(kL) \\
    \end{dcases}
  \end{align}
\end{proof}



\subsection{1次元調和振動子}
\begin{proposition}
  ポテンシャルが質点の遠心力を仕事とした調和振動子とする。
  \begin{align}
    V(x) = \frac{1}{2}m\omega^2x^2
  \end{align}
  固有関数と固有エネルギーは次のようになる。
  \begin{align}
    \psi_n(x) & = \sqrt{\frac{1}{2^nn!\sqrt{\pi}}}H_n\qty(\sqrt{\frac{m\omega}{\hbar}}x)e^{-\frac{m\omega x^2}{2\hbar}} \\
    E_n       & = \hbar\omega\qty(n + \frac{1}{2})
  \end{align}
\end{proposition}
\begin{proof}
  \begin{align}
    H = \frac{p^2}{2m} + \frac{1}{2}m\omega^2x^2
  \end{align}
  $\xi = \sqrt{m\omega/\hbar}x$ とおくと
  \begin{align}
    H\psi(x) & = \qty(-\frac{\hbar^2}{2m}\dv[2]{x} + \frac{1}{2}m\omega^2x^2)\psi(x) \\
             & = \frac{\hbar\omega}{2}\qty(-\dv[2]{\xi} + \xi^2)\psi(\xi)
  \end{align}
  より $\epsilon = 2E/\hbar\omega$ とおくと
  \begin{align}
    \psi'' + \qty(\epsilon - \xi^2)\psi(\xi) = 0
  \end{align}
  となる。この解は $\psi(\xi) = X(\xi)e^{\pm\frac{\xi^2}{2}}$ と予測されるのでこれを微分方程式に代入とすると
  \begin{align}
    X'' \pm 2\xi X' + (\epsilon\pm 1)X = 0
  \end{align}
  よりこの微分方程式の解 $X(\xi)$ はエルミート多項式の定数倍 $cH_n(\xi)$ となる。このとき無限大で発散する $\psi(\xi) = X(\xi)e^{\frac{\xi^2}{2}}$ は不適。
  // TODO なぜ $+$ の場合を排除できるのかを明確に記す。
  これより $\psi_n(\xi) = cH_n(\xi)e^{-\frac{\xi^2}{2}}$ となる。規格化条件を考えると
  \begin{align}
    \int_\RR \psi_m^*\psi_n\dd{\xi} & = c^2\int_\RR \qty(H_m(\xi)e^{-\frac{\xi^2}{2}})^*H_n(\xi)e^{-\frac{\xi^2}{2}}\dd{\xi} \\
                                    & = c^2\int_\RR H_m(\xi)H_n(\xi)e^{-\xi^2}\dd{\xi}                                       \\
                                    & = 2^nn!\sqrt{\pi}c^2\delta_{m, n}                                                      \\
                                    & = \delta_{m, n}
  \end{align}
  よって次のようになる。
  \begin{align}
    \psi_n(\xi) & = \sqrt{\frac{1}{2^nn!\sqrt{\pi}}}H_n(\xi)e^{-\frac{\xi^2}{2}}                                          \\
    \psi_n(x)   & = \sqrt{\frac{1}{2^nn!\sqrt{\pi}}}H_n\qty(\sqrt{\frac{m\omega}{\hbar}}x)e^{-\frac{m\omega x^2}{2\hbar}} \\
    E_n         & = \hbar\omega\qty(n + \frac{1}{2})
  \end{align}
\end{proof}

\begin{definition}
  上昇演算子 $\hat{a}^\dagger$, 下降演算子 $\hat{a}$, 数演算子 $\hat{N}$ を次のように定義する。
  \begin{align}
    \hat{a}^\dagger & = \frac{1}{\sqrt{2}}\qty(\xi - \dv{\xi}) & \hat{a} & = \frac{1}{\sqrt{2}}\qty(\xi + \dv{\xi}) & \hat{N} & = \hat{a}^\dagger\hat{a}
  \end{align}
\end{definition}

\begin{proposition}
  上昇・下降演算子により
  \begin{align}
    \hat{a}^\dagger\psi_n(\xi) & = \sqrt{n+1}\psi_{n+1}(\xi) \\
    \hat{a}\psi_n(\xi)         & = \sqrt{n}\psi_{n-1}(\xi)   \\
    \hat{N}\psi_n(\xi)         & = n\psi_n(\xi)
  \end{align}
\end{proposition}
\begin{proof}
  これらを波動関数に掛けると
  \begin{align}
    \hat{a}^\dagger\psi_n(\xi) & = \frac{1}{\sqrt{2}}\qty(\xi - \dv{\xi})\qty(\sqrt{\frac{1}{2^nn!\sqrt{\pi}}}H_n(\xi)e^{-\frac{\xi^2}{2}})   \\
                               & = \sqrt{\frac{n+1}{2^{n+1}(n+1)!\sqrt{\pi}}}\qty(\xi H_n(\xi) - (H_n'(\xi) - \xi H_n(\xi)))                  \\
                               & = \sqrt{n+1}\psi_{n+1}(\xi)                                                                                  \\
    \hat{a}\psi_n(\xi)         & = \frac{1}{\sqrt{2}}\qty(\xi + \dv{\xi})\qty(\sqrt{\frac{1}{2^nn!\sqrt{\pi}}}H_n(\xi)e^{-\frac{\xi^2}{2}})   \\
                               & = \sqrt{\frac{1}{2^{n-1}(n-1)!\sqrt{\pi}}}\frac{1}{2\sqrt{n}}\qty(\xi H_n(\xi) + (H_n'(\xi) - \xi H_n(\xi))) \\
                               & = \sqrt{n}\psi_{n-1}(\xi)                                                                                    \\
    \hat{N}\psi_n(\xi)         & = n\psi_n(\xi)
  \end{align}
  となる。
\end{proof}

\begin{proposition}
  \begin{align}
    [\hat{a}, \hat{a}^\dagger] & = 1                                      \\
    [\hat{N}, \hat{a}]         & = -\hat{a}^\dagger                       \\
    [\hat{N}, \hat{a}^\dagger] & = \hat{a}                                \\
    \hat{H}                    & = \hbar\omega\qty(\hat{N} + \frac{1}{2})
  \end{align}
\end{proposition}
\begin{proof}
  \begin{align}
    [\hat{a}, \hat{a}^\dagger] & = \hat{a}\hat{a}^\dagger - \hat{a}^\dagger\hat{a}                                                      \\
                               & = \frac{1}{2}\qty(\qty(\xi + \dv{\xi})\qty(\xi - \dv{\xi}) - \qty(\xi - \dv{\xi})\qty(\xi + \dv{\xi})) \\
                               & = \frac{1}{2}\qty(\qty(\xi^2 + 1 - \dv[2]{\xi}) - \qty(\xi^2 - 1 - \dv[2]{\xi}))                       \\
                               & = 1                                                                                                    \\
    [\hat{N}, \hat{a}]         & = [\hat{a}^\dagger\hat{a}, \hat{a}] =                                                                  \\
    [\hat{N}, \hat{a}^\dagger] & = \hat{a}                                                                                              \\
  \end{align}
\end{proof}

\subsection{3次元調和振動子}
\begin{proposition}
  3次元等方調和振動子について
  \begin{align}
    \hat{H} = \frac{\hat{\pp}}{2m} + \frac{1}{2}m\omega^2\rr^2
  \end{align}
  固有関数、固有エネルギーは次のようになる。
  \begin{align}
    \psi_{n_1,n_2,n_3}(\rr) & = \prod_{i = 1}^3\sqrt{\frac{1}{2^{n_i}n_i!\sqrt{\pi}}}H_{n_i}\qty(\sqrt{\frac{m\omega}{\hbar}}r_i)e^{-\frac{m\omega r_i^2}{2\hbar}} \\
    E_{n_1,n_2,n_3}         & = \sum_{i=1}^{3}\hbar\omega\qty(n_i + \frac{1}{2})
  \end{align}
\end{proposition}
\begin{proof}
  波動関数を $\psi(\rr) = X_1(r_1)X_2(r_2)X_3(r_3)$ と変数分離すると1次元調和振動子と同様に解ける。
  \begin{align}
    E_iX_i(r_i) & = \qty(\frac{\hat{p}_i^2}{2m} + \frac{1}{2}m\omega^2r_i^2)X_i(r_i)                                                           \\
    X_i(r_i)    & = \sqrt{\frac{1}{2^{n_i}n_i!\sqrt{\pi}}}H_{n_i}(\xi_i)e^{-\frac{\xi_i^2}{2}} & \qty(\xi_i = \sqrt{\frac{m\omega}{\hbar}}r_i) \\
    E_i         & = \hbar\omega\qty(n_i + \frac{1}{2})
  \end{align}
  より
  \begin{align}
    \psi_{n_1,n_2,n_3}(r_1,r_2,r_3) & = \prod_{i = 1}^3\sqrt{\frac{1}{2^{n_i}n_i!\sqrt{\pi}}}H_{n_i}(\xi_i)e^{-\frac{\xi_i^2}{2}} \\
    E_{n_1,n_2,n_3}                 & = \sum_{i=1}^{3}\hbar\omega\qty(n_i + \frac{3}{2})
  \end{align}
  となる。
\end{proof}



\subsection{2次元中心力ポテンシャル}
\begin{proposition}
  2次元中心力ポテンシャルのとき、波動関数は $\psi(r, \theta) = R(r)e^{i\mu\theta}$ として $R(r)$ は次の微分方程式を満たす関数である。
  \begin{align}
    R'' + \frac{1}{r}R' - \qty(\frac{2m(V(r) - E)}{\hbar^2} + \mu^2)R = 0
  \end{align}
\end{proposition}
\begin{proof}
  極座標
  \begin{align}
    \hat{H} & = -\frac{\hbar^2}{2m}\laplacian + V(r)                                                                               \\
            & = -\frac{\hbar^2}{2m}\qty(\pdv[2]{r} + \frac{1}{r}\pdv{r} + \frac{1}{r^2}\pdv[2]{\theta}) + V(r)                     \\
    0       & = \qty(\pdv[2]{r} + \frac{1}{r}\pdv{r} + \frac{1}{r^2}\pdv[2]{\theta} + \frac{2m(E - V(r))}{\hbar^2})\psi(r, \theta)
  \end{align}
  波動関数を $\psi(r, \theta) = R(r)\Theta(\theta)$ と変数分離する。
  \begin{align}
    \frac{R''}{R} + \frac{1}{r}\frac{R'}{R} + \frac{1}{r^2}\frac{\Theta''}{\Theta} + \frac{2m(E - V(r))}{\hbar^2} = 0
  \end{align}
  依存する変数を分けることで定数 $\mu$ を用いて次のようになる。
  \begin{align}
    \begin{dcases}
      R'' + \frac{1}{r}R' + \frac{2m(E - V(r))}{\hbar^2}R = \mu^2R \\
      \Theta'' = -\mu^2\Theta
    \end{dcases}
  \end{align}
  $\Theta(\theta)$ については次のように解ける。
  \begin{align}
    \Theta(\theta) = \begin{cases}
                       Ae^{i|\mu|\theta} + Be^{-i|\mu|\theta} & (\mu^2 \neq 0) \\
                       C\theta + D                            & (\mu^2 = 0)    \\
                     \end{cases}
  \end{align}
  波動関数は連続であるから $\Theta(0) = \Theta(2\pi)$ であり、規格化条件を満たす。これより $C = D = 0$ となる解は意味を成さず、$m\in\ZZ$ となる。
  \begin{align}
    \Theta(\theta) = \frac{1}{\sqrt{2\pi}}e^{i\mu\theta} \qquad (\mu\in\ZZ)
  \end{align}
  よって波動関数は $\psi(r, \theta) = R(r)e^{i\mu\theta}$ として $R(r)$ は次の微分方程式を満たす関数である。
  \begin{align}
    R'' + \frac{1}{r}R' + \qty(\frac{2m(E - V(r))}{\hbar^2} - \mu^2)R = 0
  \end{align}
\end{proof}



\subsection{2次元等方調和振動子}
\begin{proposition}
  2次元等方調和振動子のポテンシャルにおいて固有関数と固有エネルギーは次のようになる。
  \begin{align}
    \psi(\rho, \theta) & = \rho^{|\mu|}e^{-\frac{\rho^2}{2}}L_n^{|\mu|}(\rho)e^{i\mu\theta} \\
    E_{n, \mu}         & =
  \end{align}
\end{proposition}
\begin{proof}
  極座標で2次元等方調和振動子を考える。まず $r$ を無次元化すると
  \begin{align}
    \hat{H} & = -\frac{\hbar^2}{2m}\laplacian + \frac{1}{2}m\omega^2r^2                                                                                                        \\
            & = -\frac{\hbar^2}{2m}\qty(\pdv[2]{r} + \frac{1}{r}\pdv{r} + \frac{1}{r^2}\pdv[2]{\theta}) + \frac{1}{2}m\omega^2r^2                                              \\
            & = -\frac{\hbar\omega}{2}\qty(\pdv[2]{\rho} + \frac{1}{\rho}\pdv{\rho} + \frac{1}{\rho^2}\pdv[2]{\theta} - \rho^2)   & \qty(\rho = \sqrt{\frac{m\omega}{\hbar}}r)
  \end{align}
  波動関数を $\psi(\rho, \theta) = R(\rho)e^{i\mu\theta}$ と変数分離する。
  \begin{align}
    R'' + \frac{1}{\rho}R' + \qty(\frac{2E}{\hbar\omega} - \rho^2 - \frac{\mu^2}{\rho^2})R = 0
  \end{align}
  $\rho\to 0$ のとき $R(\rho) = \rho^s$ とおくと $R(\rho) = \rho^{|\mu|}$ が適する。
  \begin{align}
     & \rho^2R'' + \rho R' - \mu^2R = 0                              & (\rho\to 0) \\
     & \rho^2s(s - 1)\rho^{s-2} + \rho s\rho^{s-1} - \mu^2\rho^s = 0               \\
     & (s^2 - \mu^2)\rho^s = 0
  \end{align}
  $\rho\to\infty$ のとき $R = e^{-\frac{\rho^2}{2}}$ が適する。
  \begin{align}
     & \rho R'' + R' - \rho^3R = 0                                                                            & (\rho\to\infty) \\
     & \rho (-1 + \rho^2)e^{-\frac{\rho^2}{2}} - \rho e^{-\frac{\rho^2}{2}} - \rho^3e^{-\frac{\rho^2}{2}} = 0
  \end{align}
  この結果を用いて微分方程式に代入するとそれらはラゲールの陪関数によって補完されることが分かる。
  \begin{align}
    R(\rho) & = \rho^{|\mu|}e^{-\frac{\rho^2}{2}}L_n^{|\mu|}(\rho) \qquad (|\mu|\leq n\in\ZZ)
  \end{align}
  \begin{align}
    \psi(\rho, \theta) & = \rho^{|\mu|}e^{-\frac{\rho^2}{2}}L_n^{|\mu|}(\rho)e^{i\mu\theta} \\
    E_{n, \mu}         & =
  \end{align}
\end{proof}



\subsection{3次元中心力(球対称)ポテンシャル}
\begin{proposition}
  3次元中心力ポテンシャルのとき、波動関数は $\psi_{lm}(r, \theta, \phi) = R_l(r)\Theta_{lm}(\theta)\Phi_m(\phi)$ と変数分離するとそれぞれ次のようになる。
  \begin{align}
    \Phi_m(\phi)                               & = \frac{1}{\sqrt{2\pi}}e^{im\phi}                                                                        & (m\in\ZZ) \\
    \Theta_{lm}(\theta)                        & = (-1)^{\frac{m + |m|}{2}}\sqrt{\qty(l + \frac{1}{2})\frac{(l - |m|)!}{(l + |m|)!}}P_l^{|m|}(\cos\theta) & (l\in\ZZ) \\
    - \frac{\hbar^2}{2\mu r}\dv[2]{r}(rR_l(r)) & + \qty(V(r) + \frac{l(l+1)\hbar^2}{2\mu r^2})rR_l(r) = ErR_l(r)
  \end{align}
\end{proposition}
\begin{proof}
  動径方向のみに依存するポテンシャル $V(r)$ を考える。
  \begin{align}
    \hat{H} & = -\frac{\hbar^2}{2\mu}\laplacian + V(r)                                                                                                                                                       \\
            & = -\frac{\hbar^2}{2\mu}\qty(\frac{1}{r^2}\pdv{r}\qty(r^2\pdv{r}) + \frac{1}{r^2\sin\theta}\pdv{\theta}\qty(\sin\theta\pdv{\theta}) + \frac{1}{r^2\sin^2\theta}\pdv[2]{\phi}) + V(r)            \\
    0       & = \qty(\pdv{r}\qty(r^2\pdv{r}) + \frac{1}{\sin\theta}\pdv{\theta}\qty(\sin\theta\pdv{\theta}) + \frac{1}{\sin^2\theta}\pdv[2]{\phi} + \frac{2\mu r^2(E - V(r))}{\hbar^2})\psi(r, \theta, \phi)
  \end{align}
  と書ける。波動関数 $\psi(r, \theta, \phi)$ を $\psi(r, \theta, \phi) = R(r)Y(\theta, \phi)$ と変数分離すると定数 $\lambda$ を用いて
  \begin{align}
     & \qty(\pdv{r}\qty(r^2\pdv{r}) + \frac{2\mu r^2(E - V(r))}{\hbar^2})R(r) = \lambda R(r)                                                               \\
     & \qty(\frac{1}{\sin\theta}\pdv{\theta}\qty(\sin\theta\pdv{\theta}) + \frac{1}{\sin^2\theta}\pdv[2]{\phi}) Y(\theta, \phi) = -\lambda Y(\theta, \phi)
  \end{align}
  となる。また $Y(\theta, \phi) = \Theta(\theta)\Phi(\phi)$ と変数分離すると定数 $m$ を用いて
  \begin{align}
     & \qty(\sin\theta\pdv{\theta}\qty(\sin\theta\pdv{\theta}) + \lambda \sin^2\theta)\Theta(\theta) = m^2\Theta(\theta) \\
     & \dv[2]{\Phi(\phi)}{\phi} = -m^2\Phi(\phi)
  \end{align}
  となる。よって次の 3 式を解けばよい。
  \begin{align}
     & \qty(\dv{r}\qty(r^2\dv{r}) + \frac{2\mu r^2(E - V(r))}{\hbar^2})R(r) = \lambda R(r)                              \\
     & \qty(\sin\theta\dv{\theta}\qty(\sin\theta\dv{\theta}) + \lambda \sin^2\theta)\Theta(\theta)  = m^2\Theta(\theta) \\
     & \dv[2]{\Phi(\phi)}{\phi} = -m^2\Phi(\phi)
  \end{align}
  まず $\Phi(\phi)$ の一般解は次のようになる。
  \begin{align}
     & \dv[2]{\Phi(\phi)}{\phi} + m^2\Phi(\phi) = 0               \\
     & \Phi(\phi) = \begin{cases}
                      Ae^{i|m|\phi} + Be^{-i|m|\phi} & (m^2 \neq 0) \\
                      C\phi + D                      & (m^2 = 0)    \\
                    \end{cases}
  \end{align}
  波動関数は連続であるから $\Phi(0) = \Phi(2\pi)$ であり、規格化条件を満たす。$C = D = 0$ となる解は意味を成さず、$m\in\ZZ$ となる。$L_z$ の固有関数となることから
  \begin{align}
    \Phi(\phi) & = \frac{1}{\sqrt{2\pi}}e^{im\phi} \qquad (m\in\ZZ)
  \end{align}
  となる。次に $\Theta(\theta)$ について解く。$z = \cos\theta$ とおくと,
  \begin{align}
    \qty(\sin\theta\dv{\theta}\qty(\sin\theta\dv{\theta}) + \lambda \sin^2\theta)\Theta(\theta) & = m^2\Theta(\theta) \\
    \dv{z}\qty((1 - z^2)\dv{\Theta}{z}) + \qty(\lambda - \frac{m^2}{1 - z^2})\Theta(z)          & = 0
  \end{align}
  となる。$m = 0$ において $\Theta(z)$ はルジャンドルの微分方程式を満たす。$\Theta(z)$ をべき展開することで
  \begin{align}
     & (1 - z^2)\Theta'' - 2z\Theta' + \lambda\Theta = 0, \qquad \Theta(z) = \sum_{k = 0}^\infty a_kz^k                          \\
     & (1 - z^2)\sum_{k = 2}^\infty k(k-1)a_kz^{k-2} - 2z\sum_{k = 1}^\infty ka_kz^{k-1} + \lambda\sum_{k = 0}^\infty a_kz^k = 0 \\
     & \sum_{k = 0}^\infty \qty((k+1)(k+2)a_{k+2} + \qty(\lambda - k(k+1))a_k)z^k + \mathcal{O}(z) = 0                           \\
     & a_{k+2} = \frac{k(k+1) - \lambda}{(k+2)(k+1)}a_k
  \end{align}
  となる。よって $z$ について一般に発散しない為には $\lambda = l(l+1)\ (l\in\ZZ_{>0})$ とならければならない。すると $m\neq 0$ のときはルジャンドルの陪微分方程式となる。
  \begin{align}
    \dv{z}\qty((1 - z^2)\dv{\Theta}{z}) + \qty(l(l+1) - \frac{m^2}{1 - z^2})\Theta(z) & = 0
  \end{align}
  これよりルジャンドルの陪関数 $P_l^m(z)$ と規格化条件から
  \begin{align}
    \Theta_{lm}(\theta) & = (-1)^{\frac{m + |m|}{2}}\sqrt{\qty(l + \frac{1}{2})\frac{(l - |m|)!}{(l + |m|)!}}P_l^{|m|}(\cos\theta) \\
    P_l^m(z)            & = (1 - z^2)^{\frac{m}{2}}\dv[m]{P_l(z)}{z}                                                               \\
    P_l^{-m}(z)         & = (-1)^m\frac{(l - |m|)!}{(l + |m|)!}P_l^m(z)                                                            \\
    P_l(z)              & = \frac{1}{2^l}\dv[l]{z}(z^2 - 1)^l
  \end{align}
  と書ける。また $R_l(r)$ については $R_l(r) = \dfrac{\chi_l(r)}{r}$ とおくと
  \begin{align}
    - \frac{\hbar^2}{2\mu r}\dv[2]{r}\chi_l(r) + \qty(V(r) + \frac{l(l+1)\hbar^2}{2\mu r^2})\chi_l(r) = E\chi_l(r)
  \end{align}
  となり, 1 次元のシュレーディンガー方程式に帰着する。
\end{proof}



\subsection{自由な 3 次元系}
\begin{proposition}
  ポテンシャルが球対称に無いとき
  \begin{align}
    V(r) = 0
  \end{align}
  球ベッセル関数 $j_l(\xi)$ と球ノイマン関数 $n_l(\xi)$ の線形結合で書かれる。
  \begin{align}
    R_l(\xi) & = \alpha j_l(\xi) + \beta n_l(\xi)
  \end{align}
\end{proposition}
\begin{proof}
  動径方向のシュレーディンガー方程式について $k^2 = \dfrac{2\mu E}{\hbar^2}$, $\xi = kr$ とすると
  \begin{align}
     & \dv[2]{r}R_l(r) + \frac{2}{r}\dv{r}R_l(r) + \qty(\frac{2\mu E}{\hbar^2} - \frac{l(l+1)}{r^2})R_l(r) = 0 \\
     & \dv[2]{\xi}R_l(\xi) + \frac{2}{\xi}\dv{\xi}R_l(\xi) + \qty(1 - \frac{l(l+1)}{\xi^2})R_l(\xi) = 0
  \end{align}
  となり, 一般解は球ベッセル関数 $j_l(\xi)$ と球ノイマン関数 $n_l(\xi)$ の線形結合で書かれる。
  \begin{align}
    R_l(\xi) & = \alpha j_l(\xi) + \beta n_l(\xi)
  \end{align}
  球ノイマン関数は原点に極を持つので大体の場合排除される。

  例えば球面波のとき $\psi_{lm}(r, \theta, \phi) = j_l(kr)Y_l^m(\theta,\phi)$ となる。

  平面波のとき $\psi_{lm}(r, \theta, \phi) = e^{i\bm{k}\vdot\bm{r}}$ となる。特に $z$ 方向のとき次のようになるらしい。
  \begin{align}
    e^{i\bm{k}\vdot\bm{z}} = e^{ikr\cos\theta} = \sum_{l = 0}^\infty (2l + 1)i^lj_l(kr)P_l(\cos\theta)
  \end{align}
\end{proof}



\subsection{球対称剛体壁ポテンシャル}
\begin{proposition}
  次のようなポテンシャルのとき
  \begin{align}
    V(r) = \begin{dcases}
             0      & (0\leq r\leq L) \\
             \infty & (L < r)
           \end{dcases}
  \end{align}
  固有関数と固有エネルギーは次のようになる。
  \begin{align}
    \psi_{nlm}(r, \theta, \phi) & = C_{nl}j_l(\xi_{nl})Y_l^m(\theta,\phi) \\
    E_{ln}                      & = \dfrac{\hbar^2}{2\mu L^2}\xi_{ln}^2
  \end{align}
\end{proposition}
\begin{proof}
  $r > L$ のとき $\xi_{nl} = 0$ となる。$0 \leq r < L$ において考える。

  $l = 0$ のとき
  \begin{align}
    - \frac{\hbar^2}{2\mu r}\dv[2]{r}\chi_0(r) + V(r)\chi_0(r) = E\chi_0(r)
  \end{align}
  となるので境界条件と規格化条件より
  \begin{align}
    R_{n0}(r) = \frac{\chi_{n0}(r)}{r} & = \begin{dcases}
                                             \frac{1}{r}\sqrt{\frac{2}{L}}\cos(\frac{n\pi}{2L}r) & (n: 奇数) \\
                                             \frac{1}{r}\sqrt{\frac{2}{L}}\sin(\frac{n\pi}{2L}r) & (n: 偶数)
                                           \end{dcases} \\
    E_{n0}                             & = \dfrac{\hbar^2}{2\mu}\qty(\frac{n\pi}{2L})^2
  \end{align}
  $n$ が奇数のときは $r\to 0$ で発散する。これより $n$ が偶数のときに限る。
  \begin{align}
    R_{n0}(r) & = \frac{1}{r}\sqrt{\frac{2}{L}}\sin(\frac{n\pi}{2L}r) \qquad (n: 偶数)
  \end{align}
  $l \neq 0$ のとき $k^2 = \dfrac{2\mu E}{\hbar^2}$, $\xi = kr$ とおくと $R_{nl}(r)$ について球ベッセル微分方程式となる。
  \begin{align}
    \xi^2\dv[2]{R_l}{\xi} + 2\xi\dv{R_l}{\xi} + (\xi^2 - l(l + 1))R_l(\xi) = 0
  \end{align}
  これより境界条件 $\xi_{nl}$ を定めて となる。
  球ベッセル関数 $j_l(\xi)$ のゼロ点 $\xi_{l,n}$ におけるエネルギー固有値を $E_{l,n}$ とおくと
  \begin{align}
    R_l(r) & =  j_l(\xi_{nl})                      \\
    E_{ln} & = \dfrac{\hbar^2\xi_{ln}^2}{2\mu L^2}
  \end{align}
\end{proof}



\subsection{3次元等方調和振動子}
\begin{proposition}
  ポテンシャルが次のような 3 次元等方調和振動子のとき
  \begin{align}
    V(r) = \frac{1}{2}\mu\omega^2r^2
  \end{align}
  半径成分の波動方程式は次のようになる。
  \begin{align}
    R_l(x) = x^{l/2}e^{-x/2}S_n^\alpha(x)
  \end{align}
\end{proposition}
\begin{proof}
  \begin{align}
    \dv[2]{r}R_l(r) + \frac{2}{r}\dv{r}R_l(r) + \frac{2\mu}{\hbar^2}\qty(E - \frac{1}{2}\mu\omega^2r^2 - \frac{l(l+1)\hbar^2}{2\mu r^2})R_l(r) = 0
  \end{align}
  まず $\rho = \sqrt{\dfrac{\mu\omega}{\hbar}}r$ と無次元化する。
  \begin{align}
    \dv[2]{\rho}R_l(\rho) + \frac{2}{\rho}\dv{\rho}R_l(\rho) + \qty(\lambda + \rho^2 - \frac{l(l+1)}{\rho^2})R_l(\rho) & = 0 & \qty(\lambda = \frac{2E}{\hbar\omega})
  \end{align}
  となる。$x = \rho^2$ と変数変換すると
  \begin{align}
     & x\dv[2]{x}R_l(x) + \frac{3}{2}\dv{x}R_l(x) + \frac{1}{4}\qty(\lambda + x - \frac{l(l+1)}{x})R_l(x) = 0
  \end{align}
  となり, 級数展開法より $\rho\to\infty$ で発散しない為には $n$ を非負整数として $\lambda = 4n + 2l + 3$ となる。
  $\rho\to\infty$, $\rho\to 0$ のときの漸近解はそれぞれ $e^{-x/2}$, $x^{l/2}$ となるので $R_l(x) = x^{l/2}e^{-x/2}S_n^\alpha(x)$ と分離すると
  \begin{align}
    x\dv[2]{x}S_n^\alpha + (\alpha + 1 - x)\dv{x}S_n^\alpha + nS_n^\alpha = 0
  \end{align}
  ソニンの多項式となるので解はラゲールの陪関数を用いて $L_{n + \alpha}^\alpha$ と書ける。
\end{proof}



\subsection{水素原子}
\begin{proposition}
  \begin{align}
    V(r) = -\frac{e^2}{4\pi\epsilon_0r}
  \end{align}
  固有関数と固有エネルギーは次のようになる。
  \begin{align}
    R_{nl}(\rho) & = \qty(\frac{2}{na_0})^{3/2}\sqrt{\frac{(n-l-1)!}{2n((n+l)!)^3}}\rho^le^{-\rho/2}L_{n+l}^{2l+1}(\rho) \\
    E_n          & = - \frac{e^2}{8\pi\epsilon_0a_B}\frac{1}{n^2}
  \end{align}
\end{proposition}
\begin{proof}
  \begin{align}
    \dv[2]{r}R_l(r) + \frac{2}{r}\dv{r}R_l(r) + \frac{2\mu}{\hbar^2}\qty(E + \frac{e^2}{4\pi\epsilon_0r} - \frac{l(l+1)\hbar^2}{2\mu r^2})R_l(r) = 0
  \end{align}
  まず $\rho = \alpha r$, $\alpha = 2\sqrt{\dfrac{2\mu|E|}{\hbar^2}}$ と無次元化する。
  \begin{align}
    \dv[2]{\rho}R_l(\rho) + \frac{2}{\rho}\dv{\rho}R_l(\rho) + \qty(\frac{1}{4} + \frac{\lambda}{\rho} - \frac{l(l+1)}{\rho^2})R_l(\rho) & = 0 & \qty(\lambda = \frac{e^2}{8\pi\epsilon_0E}\sqrt{\dfrac{2\mu|E|}{\hbar^2}})
  \end{align}
  $\rho\to\infty$, $\rho\to 0$ のときの漸近解はそれぞれ $e^{-\rho/2}$, $\rho^l$ となるので $R_l(\rho) = \rho^le^{-\rho/2}L(\rho)$ と分離すると
  \begin{align}
    \rho\dv[2]{L}{\rho} + (2l + 2 - \rho)\dv{L}{\rho} + (\lambda - 1 - l)L = 0
  \end{align}
  となりラゲールの陪多項式となる。ここで級数展開法より $r\to\infty$ で発散しない為には非負整数 $n$ を用いて $\lambda = n + l + 1$ とかける。これより水素原子のエネルギー準位はボーア半径 $a_B$ を用いて
  \begin{align}
    E_n & = - \frac{\mu e^4}{2(4\pi\epsilon_0)^2\hbar^2}\frac{1}{n^2} =  - \frac{e^2}{8\pi\epsilon_0a_B}\frac{1}{n^2} \\
    a_B & = \frac{4\pi\epsilon_0\hbar^2}{\mu e^2}
  \end{align}
  とかける。よって規格化条件を加えると
  \begin{align}
    R_{nl}(\rho) = \qty(\frac{2}{na_0})^{3/2}\sqrt{\frac{(n-l-1)!}{2n((n+l)!)^3}}\rho^le^{-\rho/2}L_{n+l}^{2l+1}(\rho)
  \end{align}
  となり $0\leq l < n$ を満たす。
\end{proof}



\section{ヒルベルト空間}
これからは固有関数を状態として抽象化を行う。
\begin{definition}
  複素内積空間で内積によって誘導される距離関数に関して完備距離空間となるときヒルベルト空間という。
\end{definition}
\begin{definition}[ブラケット]
  すべての量子状態はヒルベルト空間上のベクトルに対応する。このベクトルをケット (ket) と呼び、$\ket{\alpha}$ と記す。ベクトル空間よりケットの線形性は成り立ち、ケットのスカラー倍は同じ状態を表すと要請する。

  ケットの双対としてブラ (bra) をエルミート共役 (Hermitian adjoint) を用いて次のように定義する。
  \begin{align}
    \bra{\alpha}                         & := \ket{\alpha}^\dagger                       \\
    c_1^*\bra{\alpha} + c_2^*\bra{\beta} & := (c_1\ket{\alpha} + c_2\ket{\beta})^\dagger \\
    \bra{\alpha}X^\dagger                & := (X\ket{\alpha})^\dagger
  \end{align}
  ブラ $\bra{\beta}$ とケット $\ket{\alpha}$ との内積 (inner product) は複素数を返し、$\braket{\beta|\alpha}\in\CC$ と記す。ただし交換すると複素共役となり、正値計量の要請を満たすとする。
  \begin{align}
    \braket{\beta|\alpha}  & = \braket{\alpha|\beta}^* \\
    \braket{\alpha|\alpha} & \geq 0
  \end{align}
  ケット $\ket{\beta}$ とブラ $\bra{\alpha}$ との外積 (outer product) は演算子を返し、$\ket{\beta}\bra{\alpha}$ と記す。
\end{definition}
\begin{definition}[演算子]
  演算子 $A$ はケット $\ket{\alpha}$ に左から作用して別のケット $A\ket{\alpha}$ となる。演算子同士の同値性、和、積を次のように定義される。
  \begin{align}
    A = B               & \iff \forall \ket{\alpha}, A\ket{\alpha} = B\ket{\alpha} \\
    (A + B)\ket{\alpha} & := A\ket{\alpha} + B\ket{\alpha}                         \\
    (AB)\ket{\alpha}    & := A(B\ket{\alpha})
  \end{align}
  演算子 $A$ に対して固有ケット(eigenkets) と呼ばれる特別なケット $\ket{a^{(n)}}$ があり、スカラー値 $a^{(n)}\in\CC$ を用いて次のような関係が成り立つ。
  \begin{align}
    A\ket{a^{(n)}} = a^{(n)}\ket{a^{(n)}}
  \end{align}
  固有ケットに対応する物理的状態を固有状態 (eigenstate) という。
\end{definition}
観測可能量 (observable) は演算子 (operator) で表される。

\begin{proposition}
  \begin{align}
    \braket{\alpha|\alpha}            & \in\RR                               \\
    (XY)^\dagger                      & = Y^\dagger X^\dagger                \\
    (\ket{\beta}\bra{\alpha})^\dagger & = \ket{\alpha}\bra{\beta}            \\
    \bra{\beta}X\ket{\alpha}          & = \bra{\alpha}X^\dagger\ket{\beta}^*
  \end{align}
\end{proposition}
\begin{proof}
  $\braket{\alpha|\alpha} = \braket{\alpha|\alpha}^*$ より $\braket{\alpha|\alpha}\in\RR$ となる。
  \begin{align}
    (XY\ket{\alpha})^\dagger                    & = (X(Y\ket{\alpha}))^\dagger = (\bra{\alpha}Y^\dagger)X^\dagger = \bra{\alpha}Y^\dagger X^\dagger                 \\
    (\ket{\beta}\braket{\alpha|\gamma})^\dagger & = \braket{\alpha|\gamma}^*\bra{\beta} = \braket{\gamma|\alpha}\bra{\beta} = \bra{\gamma}(\ket{\alpha}\bra{\beta}) \\
    \bra{\beta}X\ket{\alpha}                    & = \bra{\beta}(X\ket{\alpha}) = \qty{(\bra{\alpha}X^\dagger)\ket{\beta}}^* = \bra{\alpha}X^\dagger\ket{\beta}^*
  \end{align}
\end{proof}

\begin{theorem}[エルミート演算子の固有ケットの直交性]
  エルミート演算子の固有値は実数であり、異なる固有値を持つ固有ケットは互いに直交する。
\end{theorem}
\begin{proof}
  \begin{align}
    A\ket{a'} = a'\ket{a'}      & \iff \bra{a''}A\ket{a'} = a'\braket{a''|a'}    \\
    \bra{a''}A = a''^*\bra{a''} & \iff \bra{a''}A\ket{a'} = a''^*\braket{a''|a'} \\
    (a' - a''^*)\braket{a''|a'} & = 0
  \end{align}
  $a' = a''$ かつ $\braket{a'|a'} \neq 0$ のとき $a', a''\in\RR$ である。これより
  \begin{align}
    (a' - a'')\braket{a''|a'} & = 0
  \end{align}
  となる。よって $a' \neq a''$ のとき $\braket{a''|a'} = 0$ である。
\end{proof}

これよりエルミート演算子の固有ケットは完全系を成す。状態は観測でしか分からないので固有ケットにより全てのケットは生成される。

\begin{align}
  \ket{\alpha} & = \sum_{a'}c_{a'}\ket{a'}             \\
               & = \sum_{a'}\ket{a'}\braket{a'|\alpha}
\end{align}
これより完備関係式 (completeness relation) と呼ばれる式が成り立つ。
\begin{align}
  \sum_{a'}\ket{a'}\bra{a'} = 1
\end{align}

\begin{proposition}[行列表現]
  演算子やブラケットは行列で表現できる。
\end{proposition}
\begin{proof}
  ある演算子の固有ケットによる完備関係式を用いることで演算子とその和積、ブラケット $\ket{\alpha}$, $\bra{\alpha}$ は次のように書ける。
  \begin{align}
    X                        & = \sum_{a''}\sum_{a'}\ket{a''}\bra{a''}X\ket{a'}\bra{a'} \\
    \bra{a''}(X + Y)\ket{a'} & = \bra{a''}X\ket{a'} + \bra{a''}Y\ket{a'}                \\
    \bra{a''}XY\ket{a'}      & = \sum_{a'''}\bra{a''}X\ket{a'''}\bra{a'''}Y\ket{a'}     \\
    \bra{a'}X\ket{\alpha}    & = \sum_{a''}\bra{a'}X\ket{a''}\braket{a''|\alpha}        \\
    \bra{\alpha}X\ket{a'}    & = \sum_{a''}\braket{\alpha|a''}\bra{a''}X\ket{a'}
  \end{align}
  これより行列として表現すると次のようになる。
  \begin{align}
    X                                            & = \mqty(\bra{a^{(1)}}X\ket{a^{(1)}} & \bra{a^{(1)}}X\ket{a^{(2)}} & \cdots \\
    \bra{a^{(2)}}X\ket{a^{(1)}}                  & \bra{a^{(2)}}X\ket{a^{(2)}}         & \cdots                               \\
    \vdots                                       & \vdots                              & \ddots)                              \\
    \ket{\alpha}                                 & = \mqty(\braket{a^{(1)}|\alpha}                                            \\ \braket{a^{(2)}|\alpha} \\ \vdots), \qquad
    \bra{\alpha} = \mqty(\braket{\alpha|a^{(1)}} & \braket{\alpha|a^{(2)}}             & \cdots)
  \end{align}
\end{proof}

\begin{proposition}
  \begin{align}
    \ket{\alpha}           & = \int\dd{x'}\ket{x'}\braket{x'|\alpha}           \\
    \braket{x'|\alpha}     & = \psi_\alpha(x')                                 \\
    \braket{\beta|\alpha}  & = \int\dd{x'}\braket{\beta|x'}\braket{x'|\alpha}  \\
                           & = \int\dd{x'}\psi_\beta^*(x')\psi_\alpha(x')      \\
    \braket{\alpha|\alpha} & = \int\dd{x'}\braket{\alpha|x'}\braket{x'|\alpha} \\
                           & = \int\dd{x'}\psi_\alpha^*(x')\psi_\alpha(x')     \\
                           & = 1
  \end{align}
\end{proposition}
\begin{definition}[平行移動]
  無限小平行移動演算子
  \begin{align}
    \mathfrak{J}(\dd{\rr'})\ket{\rr'} = \ket{\rr' + \dd{\rr'}}
  \end{align}
  平行移動生成演算子
\end{definition}
\begin{align}
  \mathfrak{J}(\dd{\rr'})\ket{\alpha} = \mathfrak{J}(\dd{\rr'})\int\dd{\rr'}\ket{\rr'}\braket{\rr'|\alpha} = \int\dd{\rr'}\ket{\rr' + \dd{\rr'}}\braket{\rr'|\alpha} = \int\dd{\rr'}\ket{\rr'}\braket{\rr' - \dd{\rr'}|\alpha}
\end{align}

\begin{definition}[交換関係・反交換関係]
  2 つの演算子 $\hat{A}$, $\hat{B}$ について交換関係 (commutation relation) $[\hat{A}, \hat{B}]$ と反交換関係 (anticommutation relations) $\lbrace\hat{A}, \hat{B}\rbrace$ は次のように定義する。
  \begin{align}
    \qty[\hat{A}, \hat{B}] & = \hat{A}\hat{B} - \hat{B}\hat{A} \\
    \qty{\hat{A}, \hat{B}} & = \hat{A}\hat{B} + \hat{B}\hat{A}
  \end{align}
  $[\hat{A}, \hat{B}] = 0$ のとき $\hat{A}$ と $\hat{B}$ は交換可能であるという。このとき $\hat{A}$ の固有状態であり、同時に $\hat{B}$ の固有状態でもある状態を作ることができる。これを同時対角化可能という。
\end{definition}

\begin{proposition}
  ヤコビ恒等式
  \begin{align}
    [A, A]      & = 0                             \\
    [A, B]      & = - [B, A]                      \\
    [A, c]      & = 0                             \\
    [A + B, C]  & = [A, C] + [B, C]               \\
    [A, BC]     & = [A, B]C + B[A, C]             \\
    [A, [B, C]] & + [B, [C, A]] + [C, [A, B]] = 0
  \end{align}
\end{proposition}
\begin{proof}
  \begin{align}
    [A, A]     & = AA - AA = 0             \\
    [A, B]     & = AB - BA = - [B, A]      \\
    [A, c]     & = Ac - cA = 0             \\
    [A + B, C] & = (A + B)C - C(A + B)     \\
               & = (AC - CA) + (BC - CB)   \\
               & = [A, C] + [B, C]         \\
    [A, BC]    & = ABC - BCA               \\
               & = (AB - BA)C + B(AC - CA) \\
               & = [A, B]C + B[A, C]
  \end{align}
  \begin{align}
     & [A, [B, C]] + [B, [C, A]] + [C, [A, B]]                                       \\
     & = A(BC - CB) - (BC - CB)A + B(CA - AC) - (CA - AC)B + C(AB - BA) - (AB - BA)C \\
     & = 0
  \end{align}
\end{proof}

\section{時間発展のあるシュレーディンガー方程式}
\begin{align}
  i\hbar\pdv{t}\ket{\phi(t)} & = \hat{H}\ket{\phi(t)}
\end{align}
$\hat{H}\ket{\phi_n} = E_n\ket{\phi_n}$ としたとき, $\ket{\phi_n}$ は完全系をなす。これで展開して代入すると
\begin{align}
  \ket{\phi(t)}      & = \sum_nc_n(t)\ket{\phi_n}                           & \qty(c_n(t) = \braket{\phi_n}{\phi(t)}) \\
  i\hbar\dv{t}c_n(t) & = E_nc_n(t)                                                                                    \\
  c_n(t)             & = c_n(0)\exp(-i\frac{E_nt}{\hbar})                                                             \\
  \ket{\phi(t)}      & = \sum_nc_n(0)\exp(-i\frac{E_nt}{\hbar})\ket{\phi_n}
\end{align}
となる。


\subsection{ラーモア歳差運動}
\begin{align}
  \ket{\sigma(t)}
\end{align}



\section{角運動量代数}
\begin{definition}
  角運動量演算子 $\hat{\bm{L}}$ を次のように定義する。
  \begin{align}
    \hat{\bm{L}} & = \hat{\rr}\cross\hat{\pp}
  \end{align}
  その無次元量 $\hat{\bm{j}}$ は $\hbar\hat{\bm{j}} = \hat{\bm{L}}$ と書ける。
\end{definition}

\begin{proposition}
  \begin{align}
    \hat{\bm{L}}^2              & = \hat{L}_x^2 + \hat{L}_y^2 + \hat{L}_z^2 \\
    [\hat{L}_i, \hat{L}_j]      & = i\hbar\epsilon_{ijk}\hat{L}_k           \\
    [\hat{\bm{L}}^2, \hat{L}_i] & = 0
  \end{align}
\end{proposition}

\begin{definition}
  $\hat{\bm{j}}$ を無次元の演算子として次の交換関係が成り立つとする。
  \begin{align}
    [\hat{j}_i, \hat{j}_j] = i\epsilon_{ijk}\hat{j}_k
  \end{align}
  $[\hat{\bm{j}}^2, \hat{j}_z] = 0$ より $\hat{\bm{j}}^2, \hat{j}_z$ は固有値 $\lambda, m$ とする同時固有状態 $\ket{\lambda, m}$ を持つ。
  上昇演算子 $\hat{j}_+$ と下降演算子 $\hat{j}_-$ を次のように定義する。
  \begin{align}
    \hat{j}_\pm & = \hat{j}_x \pm i\hat{j}_y
  \end{align}
\end{definition}

\begin{proposition}
  \begin{align}
    [\hat{\bm{j}}^2, \hat{j}_z] = 0, \qquad [\hat{j}_+, \hat{j}_-] = 2\hat{j}_z, \qquad [\hat{j}_z, \hat{j}_\pm] = \pm \hat{j}_\pm, \qquad [\hat{\bm{j}}^2, \hat{j}_\pm] = 0 \\
    \hat{\bm{j}}^2 = \frac{1}{2}(\hat{j}_+\hat{j}_- + \hat{j}_-\hat{j}_+) + \hat{j}_z^2 = \hat{j}_-\hat{j}_+ + \hat{j}_z + \hat{j}_z^2 = \hat{j}_+\hat{j}_- - \hat{j}_z + \hat{j}_z^2
  \end{align}
\end{proposition}
\begin{proof}
\end{proof}

\begin{proposition}
  上昇演算子 $\hat{j}_+$ を演算させると $\hat{j}_z$ の固有値は 1 つ上昇し、下降演算子 $\hat{j}_-$ を演算させると $\hat{j}_z$ の固有値が 1 つ下降する。
  \begin{align}
    \hat{j}_\pm\ket{\lambda, m} = \ket{\lambda, m\pm 1}
  \end{align}
\end{proposition}
\begin{proof}
  このとき上昇,下降演算子を作用させたとき
  \begin{align}
    \hat{\bm{j}}^2(\hat{j}_\pm\ket{\lambda, m}) & = \hat{j}_\pm\hat{\bm{j}}^2\ket{\lambda, m} = \lambda \hat{j}_\pm\ket{\lambda, m}               \\
    \hat{j}_z(\hat{j}_\pm\ket{\lambda, m})      & = (\hat{j}_\pm \hat{j}_z \pm \hat{j}_\pm)\ket{\lambda, m} = (m\pm 1)\hat{j}_\pm\ket{\lambda, m}
  \end{align}
  より $\hat{j}_\pm\ket{\lambda, m} = \ket{\lambda, m\pm 1}$ とかける。
\end{proof}

\begin{proposition}
  $\hat{j}_z$ の固有値 $m$ の上限と下限は存在する。
\end{proposition}
\begin{proof}
  \begin{align}
    \bra{\lambda, m}\hat{\bm{j}}^2\ket{\lambda, m} & = \bra{\lambda, m}(\hat{j}_x^2 + \hat{j}_y^2 + \hat{j}_z^2)\ket{\lambda, m}                       \\
                                                   & = \bra{\lambda, m}\hat{j}_x^2\ket{\lambda, m} + \bra{\lambda, m}\hat{j}_y^2\ket{\lambda, m} + m^2 \\
                                                   & = \lambda
  \end{align}
  $\hat{j}_x, \hat{j}_y$ はエルミート演算子より $\bra{\lambda, m}\hat{j}_x^2\ket{\lambda, m} \geq 0$, $\bra{\lambda, m}\hat{j}_y^2\ket{\lambda, m} \geq 0$ であり $0 \leq m^2 \leq \lambda$
\end{proof}

\begin{proposition}
  $\hat{j}_z$ の固有値 $m$ は非負の整数または半整数 $j$ を用いて $m = -j,-j+1,\ldots,j-1,j$ と書ける。
\end{proposition}
\begin{proof}
  次のような関係式が成り立つ。
  \begin{align}
    \bra{\lambda, m}\hat{\bm{j}}^2\ket{\lambda, m} & = \bra{\lambda, m}(\hat{j}_-\hat{j}_+ + \hat{j}_z^2 + \hat{j}_z)\ket{\lambda, m} = \bra{\lambda, m}\hat{j}_-\hat{j}_+\ket{\lambda, m} + m^2 + m \\
                                                   & = \bra{\lambda, m}(\hat{j}_+\hat{j}_- + \hat{j}_z^2 - \hat{j}_z)\ket{\lambda, m} = \bra{\lambda, m}\hat{j}_+\hat{j}_-\ket{\lambda, m} + m^2 - m \\
                                                   & = \lambda
  \end{align}
  これより $m$ の上限値 $j$ と置くと $\lambda = j(j + 1)$ となり、下限値 $j - n$ と置くと $\lambda = (j - n)(j - n - 1)$ となる。
  \begin{align}
    \begin{dcases}
      \lambda = j(j + 1) \\
      \lambda = (j - n)(j - n - 1)
    \end{dcases}
    \iff
    \begin{dcases}
      \lambda = j(j + 1) \\
      j = \frac{n}{2}
    \end{dcases}
  \end{align}
  より $j$ は非負の整数または半整数であることがわかる。これより $m = -j, -j + 1,\ldots, j-1, j$ である。
\end{proof}

\begin{definition}
  $\ket{\lambda, m}$ を $\ket{j, m}$ と表現する。
\end{definition}

\begin{proposition}
  \begin{align}
    \hat{\bm{j}}^2\ket{j, m} & = j(j + 1)\ket{j, m}                          \\
    \hat{j}_z\ket{j, m}      & = m\ket{j, m}                                 \\
    \hat{j}_\pm\ket{j, m}    & = \sqrt{(j \mp m)(j \pm m + 1)}\ket{j, m\pm1}
  \end{align}
\end{proposition}


この角運動量が複数あるときについて考える。角運動量の合成とは合成系の角運動量固有状態を部分系の角運動量固有状態で表すことである。角運動量演算子 $\hat{\bm{j}}_1, \hat{\bm{j}}_2$ について
\begin{align}
  \hat{\bm{j}} & = \hat{\bm{j}}_1 + \hat{\bm{j}}_2
\end{align}
とおく。このとき
\begin{align}
  [\hat{j}_{a,i}, \hat{j}_{b,j}]   & = i\delta_{ab}\epsilon_{ijk}\hat{j}_{ck} \\
  [\hat{j}_a, \hat{j}_b]           & = i\epsilon_{abc}\hat{j}_c               \\
  [\hat{\bm{j}}^2, \hat{j}_a]      & = 0                                      \\
  [\hat{\bm{j}}^2, \hat{\bm{j}}_s] & = 0
\end{align}
となる。また状態についても
\begin{align}
  \ket{j, m}\rangle & = \sum_{m_1, m_2}C_{j_1m_1j_2m_2}^{jm}\ket{j_1, m_1}\ket{j_2, m_2}
\end{align}
とおき, 次のようになるとする。
\begin{align}
  \hat{\bm{j}}^2\ket{j, m}\rangle & = j(j+1)\ket{j, m}\rangle                           \\
  \hat{j}_z\ket{j, m}\rangle      & = m\ket{j, m}\rangle                                \\
  \hat{j}_\pm\ket{j, m}\rangle    & = \sqrt{(j\mp m)(j\pm m + 1)}\ket{j, m\pm 1}\rangle \\
  \hat{\bm{j}}_s\ket{j, m}\rangle & = j_s(j_s + 1)\ket{j, m}\rangle
\end{align}
この上で
\begin{align}
  \hat{j}_z\ket{j, m}\rangle      & = (j_{1z} + j_{2z})\sum_{m_1, m_2}C_{j_1m_1j_2m_2}^{jm}\ket{j_1, m_1}\ket{j_2, m_2} \\
                                  & = \sum_{m_1, m_2}C_{j_1m_1j_2m_2}^{jm}(m_1 + m_2)\ket{j_1, m_1}\ket{j_2, m_2}       \\
                                  & = m\sum_{m_1, m_2}C_{j_1m_1j_2m_2}^{jm}\ket{j_1, m_1}\ket{j_2, m_2}                 \\
  \hat{\bm{j}}^2\ket{j, m}\rangle & = (j_-j_+ + j_z^2 + j_z)\ket{j, m}\rangle                                           \\
                                  & = (j_-j_+ + m^2 + m)\ket{j, m}\rangle                                               \\
                                  & = j(j + 1)\ket{j, m}\rangle
\end{align}
より状態の係数比較して $m \neq m_1 + m_2$ のとき $C_{j_1m_1j_2m_2}^{jm} = 0$ となる。$m$ の最大値 $j_{\max} = j_1 + j_2$ である。
\begin{align}
  \sum_{j=j_{\min}}^{j_{\max}} (2j + 1) = (2j_1 + 1)(2j_2 + 1)
\end{align}
より $j_{\min} = |j_1 - j_2|$ となる。
\begin{align}
  \ket{1, 1}\rangle  & = \ket{\uparrow\uparrow}                                                                                               \\
  \ket{1, 0}\rangle  & = \frac{1}{\sqrt{2}}j_-\ket{1, 1}\rangle = \frac{1}{\sqrt{2}}\qty(\ket{\uparrow\downarrow} + \ket{\downarrow\uparrow}) \\
  \ket{1, -1}\rangle & = \ket{\downarrow\downarrow}                                                                                           \\
  \ket{0, 0}\rangle  & = \frac{1}{\sqrt{2}}\qty(\ket{\uparrow\downarrow} - \ket{\downarrow\uparrow})
\end{align}

\begin{definition}[スピン角運動量]
  量子力学的粒子にはスピンという内部自由度がある。この演算子をスピン角運動量演算子 $\hat{\bm{S}}$ といい、位置演算子 $\hat{\rr}$、運動量演算子 $\hat{\pp}$、角運動量演算子 $\hat{\bm{L}}$ と交換する。
  \begin{align}
    [\hat{r}_i, \hat{S}_j] = 0, \qquad [\hat{p}_i, \hat{S}_j] = 0, \qquad [\hat{L}_i, \hat{S}_j] = 0
  \end{align}
  全角運動量 $\hat{\bm{J}}$ は軌道角運動量 $\hat{\bm{L}}$ とスピン角運動量 $\hat{\bm{S}}$ の和で与えられる。
  \begin{align}
    \hat{\bm{J}} & = \hat{\bm{L}} + \hat{\bm{S}}
  \end{align}
  無次元化されたスピン角運動量演算子 $\hat{\bm{s}}$ は次の交換関係を満たす。
  \begin{align}
    [\hat{s}_i, \hat{s}_j] = i\varepsilon_{ijk}\hat{s}_k
  \end{align}
  スピン演算子の 2 乗 $\hat{\bm{s}}^2$ や昇降演算子 $\hat{s}_\pm$ を次のように定義する。
  \begin{align}
    \hat{\bm{s}}^2 = \hat{s}_x + \hat{s}_y + \hat{s}_z, \qquad \hat{s}_\pm = \hat{s}_x \pm i\hat{s}_y
  \end{align}
\end{definition}
角運動量演算子の固有値は整数だけであったが、スピン角運動量演算子は半整数と成り得る。

\begin{proposition}
  \begin{align}
    [\hat{\bm{s}}^2, \hat{s}_z] = 0, \qquad [\hat{s}_+, \hat{s}_-] = 2\hat{s}_z, \qquad [\hat{s}_z, \hat{s}_\pm] = \pm \hat{s}_\pm, \qquad [\hat{\bm{s}}^2, \hat{s}_\pm] = 0 \\
    \hat{\bm{s}}^2 = \frac{1}{2}(\hat{s}_+\hat{s}_- + \hat{s}_-\hat{s}_+) + \hat{s}_z^2 = \hat{s}_-\hat{s}_+ + \hat{s}_z + \hat{s}_z^2 = \hat{s}_+\hat{s}_- - \hat{s}_z + \hat{s}_z^2
  \end{align}
\end{proposition}

\begin{proposition}
  スピン $s = 1/2$ では $\hat{s}_z$ の固有状態が 2 つあり、それぞれ固有値 $m_s = \pm1/2$ を持つ $\ket{\uparrow}$, $\ket{\downarrow}$ とおく。
  \begin{align}
    \hat{s}_z\ket{\uparrow} = \frac{1}{2}\ket{\uparrow}, \qquad \hat{s}_z\ket{\downarrow} = -\frac{1}{2}\ket{\downarrow}, \qquad \hat{\bm{s}}^2\ket{\uparrow} = \frac{3}{4}\ket{\uparrow}, \qquad \hat{\bm{s}}^2\ket{\downarrow} = \frac{3}{4}\ket{\downarrow}
  \end{align}
  スピン昇降演算子を用いると固有状態は互いに入れ替わる。
  \begin{align}
    \hat{s}_+\ket{\uparrow} = 0, \qquad \hat{s}_+\ket{\downarrow} = \ket{\uparrow}, \qquad \hat{s}_-\ket{\uparrow} = \ket{\downarrow}, \qquad \hat{s}_-\ket{\downarrow} = 0
  \end{align}
\end{proposition}

\begin{definition}[パウリ行列]
  \begin{align}
    \sigma_1 = \begin{pmatrix}
                 0 & 1 \\
                 1 & 0
               \end{pmatrix}, \qquad
    \sigma_2 = \begin{pmatrix}
                 0 & -i \\
                 i & 0
               \end{pmatrix}, \qquad
    \sigma_3 = \begin{pmatrix}
                 1 & 0  \\
                 0 & -1
               \end{pmatrix}
  \end{align}
\end{definition}

\begin{proposition}[パウリ行列の性質]
  エルミート性を満たす。
  \begin{align}
    \sigma_i^\dagger         & = \sigma_i                                  \\
    \sigma_i\sigma_j         & = \delta_{ij}I + i\varepsilon_{ijk}\sigma_k \\
    [\sigma_i, \sigma_j]     & = 2i\varepsilon_{ijk}\sigma_k               \\
    \qty{\sigma_i, \sigma_j} & = 2\delta_{ij}I
  \end{align}
\end{proposition}
\begin{proof}
  1, 2 は調べることで成り立つ。
  \begin{align}
    [\sigma_i, \sigma_j]     & = \sigma_i\sigma_j - \sigma_j\sigma_i                                                       \\
                             & = (\delta_{ij}I + i\varepsilon_{ijk}\sigma_k) - (\delta_{ji}I + i\varepsilon_{jik}\sigma_k) \\
                             & = 2i\varepsilon_{ijk}\sigma_k                                                               \\
    \qty{\sigma_i, \sigma_j} & = \sigma_i\sigma_j + \sigma_j\sigma_i                                                       \\
                             & = (\delta_{ij}I + i\varepsilon_{ijk}\sigma_k) + (\delta_{ji}I + i\varepsilon_{jik}\sigma_k) \\
                             & = 2\delta_{ij}I
  \end{align}
\end{proof}

\begin{proposition}
  $\hat{s}_z$ の固有状態は次のように表現される。
  \begin{align}
    \ket{\uparrow} = \mqty(1   \\ 0), \qquad
    \ket{\downarrow} = \mqty(0 \\ 1)
  \end{align}
  スピン演算子は次のように表現される。
  \begin{align}
    \hat{s}_i = \frac{1}{2}\sigma_i, \qquad
    \hat{s}_+ = \mqty(0                 & 1 \\ 0 & 0), \qquad
    \hat{s}_- = \mqty(0                 & 0 \\ 1 & 0), \qquad
    \hat{\bm{s}}^2 = \frac{3}{4}\mqty(1 & 0 \\ 0 & 1)
  \end{align}
\end{proposition}

\begin{proposition}
  完全系を貼るので任意の状態 $\ket{\sigma}$ はその線形結合で書ける。
  \begin{align}
    \ket{\sigma} & = c_1\ket{\uparrow} + c_2\ket{\downarrow}
  \end{align}


\end{proposition}



\section{摂動論}
近似法の一種。有限和で止めるとユニタリティはなくなる。重ね合わせの原理を満たさない。
\begin{proposition}
  1 次, 2 次の固有値 $E_n^{(1)}, E_n^{(2)}$ と固有状態 $\ket{\phi_n^{(1)}}, \ket{\phi_n^{(2)}}$ は定数 $c_n^{(1)}, c_n^{(2)}$ を用いて次のようになる。
  \begin{align}
    E_n^{(1)}          & = \bra{\phi_n^{(0)}}V\ket{\phi_n^{(0)}}                                                                                                                                                                                                                                                                                                                                                             \\
    \ket{\phi_n^{(1)}} & = \sum_{m\neq n}\frac{\bra{\phi_m^{(0)}}V\ket{\phi_n^{(0)}}}{E_n^{(0)} - E_m^{(0)}}\ket{\phi_m^{(0)}} + c_n^{(1)}\ket{\phi_n^{(0)}}                                                                                                                                                                                                                                                                 \\
    E_n^{(2)}          & = \sum_{m\neq n}\frac{\qty|\bra{\phi_m^{(0)}}V\ket{\phi_n^{(0)}}|^2}{E_n^{(0)} - E_m^{(0)}}                                                                                                                                                                                                                                                                                                         \\
    \ket{\phi_n^{(2)}} & = c_n^{(1)}\sum_{m\neq n}\qty(\frac{\bra{\phi_m^{(0)}}V\ket{\phi_n^{(0)}}}{E_n^{(0)} - E_m^{(0)}} - E_n^{(1)}\frac{\bra{\phi_m^{(0)}}V\ket{\phi_n^{(0)}}}{\qty(E_n^{(0)} - E_m^{(0)})^2} + \sum_{k\neq n}\frac{\bra{\phi_m^{(0)}}V\ket{\phi_k^{(0)}}\bra{\phi_k^{(0)}}V\ket{\phi_n^{(0)}}}{\qty(E_n^{(0)} - E_m^{(0)})\qty(E_n^{(0)} - E_k^{(0)})})\ket{\phi_m^{(0)}} + c_n^{(2)}\ket{\phi_n^{(0)}}
  \end{align}
  \begin{align}
    \hat{H}_0\ket{\phi_n^{(0)}} = E_n^{(0)}\ket{\phi_n^{(0)}}
  \end{align}
\end{proposition}
\begin{proof}
  \begin{align}
    \hat{H}\ket{\phi_n} & = E_n\ket{\phi_n}                                  \\
    \hat{H}             & = \hat{H}_0 + \lambda V                            \\
    \ket{\phi_n}        & = \sum_{i = 0}^{\infty}\lambda^i\ket{\phi_n^{(i)}} \\
    E_n                 & = \sum_{i = 0}^{\infty}\lambda^iE_n^{(i)}
  \end{align}
  \begin{align}
    (H - E_n)\ket{\phi_n} & = \qty(\qty(H_0 + \lambda V) - \qty(\sum_{i = 0}^{\infty}\lambda^iE_n^{(i)}))\qty(\sum_{i = 0}^{\infty}\lambda^i\ket{\phi_n^{(i)}}) \\
                          & = \sum_{i = 0}^{\infty}\lambda^i\qty(\sum_{j+k=i}\qty(\delta_{0j}H_0 + \delta_{1j}V - E_n^{(j)})\ket{\phi_n^{(k)}}) = 0
  \end{align}
  これより各 $\lambda$ の次数について比較して次のようになる。
  \begin{align}
    \sum_{j+k=i}\qty(\delta_{0j}H_0 + \delta_{1j}V - E_n^{(j)})\ket{\phi_n^{(k)}} = 0
  \end{align}
  ここでは 0, 1, 2 次についてのみ考える。
  \begin{align}
     & \begin{dcases}
         \qty(E_n^{(0)} - H_0)\ket{\phi_n^{(0)}} = 0                                     \\
         \qty(E_n^{(0)} - H_0)\ket{\phi_n^{(1)}} = \qty(V - E_n^{(1)})\ket{\phi_n^{(0)}} \\
         \qty(E_n^{(0)} - H_0)\ket{\phi_n^{(2)}} = \qty(V - E_n^{(1)})\ket{\phi_n^{(1)}} - E_n^{(2)}\ket{\phi_n^{(0)}}
       \end{dcases}\label{perturbation}
  \end{align}
  まず 0 次については次のように書ける。
  \begin{align}
    H_0\ket{\phi_n^{(0)}} = E_n^{(0)}\ket{\phi_n^{(0)}}
  \end{align}
  式 \eqref{perturbation} に $\bra{\phi_m^{(0)}}$ を掛けると
  \begin{align}
         & \begin{dcases}
             \bra{\phi_m^{(0)}}\qty(E_n^{(0)} - H_0)\ket{\phi_n^{(1)}} = \bra{\phi_m^{(0)}}\qty(V - E_n^{(1)})\ket{\phi_n^{(0)}} \\
             \bra{\phi_m^{(0)}}\qty(E_n^{(0)} - H_0)\ket{\phi_n^{(2)}} = \bra{\phi_m^{(0)}}\qty(V - E_n^{(1)})\ket{\phi_n^{(1)}} - \bra{\phi_m^{(0)}}E_n^{(2)}\ket{\phi_n^{(0)}}
           \end{dcases}          \\
    \iff & \begin{dcases}
             \qty(E_n^{(0)} - E_m^{(0)})\braket{\phi_m^{(0)}}{\phi_n^{(1)}} = \bra{\phi_m^{(0)}}V\ket{\phi_n^{(0)}} - E_n^{(1)}\delta_{mn} \\
             \qty(E_n^{(0)} - E_m^{(0)})\braket{\phi_m^{(0)}}{\phi_n^{(2)}} = \bra{\phi_m^{(0)}}V\ket{\phi_n^{(1)}} - E_n^{(1)}\braket{\phi_m^{(0)}}{\phi_n^{(1)}} - E_n^{(2)}\delta_{mn}
           \end{dcases}
  \end{align}
  よって 1 次, 2 次の固有値 $E_n^{(1)}, E_n^{(2)}$ と固有状態 $\ket{\phi_n^{(1)}}, \ket{\phi_n^{(2)}}$ は定数 $c_n^{(1)}, c_n^{(2)}$ を用いて次のようになる。
  \begin{align}
    E_n^{(1)}          & = \bra{\phi_n^{(0)}}V\ket{\phi_n^{(0)}}                                                                                                                                                                                                                                                                                                                                                             \\
    \ket{\phi_n^{(1)}} & = \sum_{m\neq n}\frac{\bra{\phi_m^{(0)}}V\ket{\phi_n^{(0)}}}{E_n^{(0)} - E_m^{(0)}}\ket{\phi_m^{(0)}} + c_n^{(1)}\ket{\phi_n^{(0)}}                                                                                                                                                                                                                                                                 \\
    E_n^{(2)}          & = \sum_{m\neq n}\frac{\qty|\bra{\phi_m^{(0)}}V\ket{\phi_n^{(0)}}|^2}{E_n^{(0)} - E_m^{(0)}}                                                                                                                                                                                                                                                                                                         \\
    \ket{\phi_n^{(2)}} & = c_n^{(1)}\sum_{m\neq n}\qty(\frac{\bra{\phi_m^{(0)}}V\ket{\phi_n^{(0)}}}{E_n^{(0)} - E_m^{(0)}} - E_n^{(1)}\frac{\bra{\phi_m^{(0)}}V\ket{\phi_n^{(0)}}}{\qty(E_n^{(0)} - E_m^{(0)})^2} + \sum_{k\neq n}\frac{\bra{\phi_m^{(0)}}V\ket{\phi_k^{(0)}}\bra{\phi_k^{(0)}}V\ket{\phi_n^{(0)}}}{\qty(E_n^{(0)} - E_m^{(0)})\qty(E_n^{(0)} - E_k^{(0)})})\ket{\phi_m^{(0)}} + c_n^{(2)}\ket{\phi_n^{(0)}}
  \end{align}
\end{proof}

\end{document}