\RequirePackage{plautopatch}
\documentclass[uplatex,dvipdfmx,a4paper,11pt]{jlreq}
\usepackage{bxpapersize}
\usepackage[utf8]{inputenc}
\usepackage{fontenc}
\usepackage{lmodern}
\usepackage{otf}
\usepackage{amsmath}
\usepackage{amssymb}
\usepackage{amsthm}
\usepackage{ascmac}
% \usepackage[hyphens]{url}
\usepackage{physics}
\usepackage{braket}
\usepackage{verbatimbox}
\usepackage{bm}
\usepackage{url}
% \usepackage[dvipdfmx,hiresbb,final]{graphicx}
\usepackage{hyperref}
\usepackage{pxjahyper}
\usepackage{tikz}\usetikzlibrary{cd}
\usepackage{listings}
\usepackage{color}
\usepackage{mathtools}
\usepackage{xspace}
\usepackage{xy}
\usepackage{xypic}
%
\title{フーリエ解析}
\author{Anko}
\makeatletter
%
\DeclareMathOperator{\lcm}{lcm}
\DeclareMathOperator{\Kernel}{Ker}
\DeclareMathOperator{\Image}{Im}
\DeclareMathOperator{\ch}{ch}
\DeclareMathOperator{\Aut}{Aut}
\DeclareMathOperator{\Log}{Log}
\DeclareMathOperator{\Arg}{Arg}
\DeclareMathOperator{\sgn}{sgn}
%
\newcommand{\CC}{\mathbb{C}}
\newcommand{\RR}{\mathbb{R}}
\newcommand{\QQ}{\mathbb{Q}}
\newcommand{\ZZ}{\mathbb{Z}}
\newcommand{\NN}{\mathbb{N}}
\newcommand{\FF}{\mathbb{F}}
\newcommand{\PP}{\mathbb{P}}
\newcommand{\GG}{\mathbb{G}}
\newcommand{\TT}{\mathbb{T}}
\newcommand{\calB}{\mathcal{B}}
\newcommand{\calF}{\mathcal{F}}
\newcommand{\ignore}[1]{}
\newcommand{\floor}[1]{\left\lfloor #1 \right\rfloor}
% \newcommand{\abs}[1]{\left\lvert #1 \right\rvert}
\newcommand{\lt}{<}
\newcommand{\gt}{>}
\newcommand{\id}{\mathrm{id}}
\newcommand{\rot}{\curl}
\renewcommand{\angle}[1]{\left\langle #1 \right\rangle}

\let\oldcite=\cite
\renewcommand\cite[1]{\hyperlink{#1}{\oldcite{#1}}}

\let\oldbibitem=\bibitem
\renewcommand{\bibitem}[2][]{\label{#2}\oldbibitem[#1]{#2}}

% theorem環境の設定
% - 冒頭に改行
% - 末尾にdiamond (amsthm)
\theoremstyle{definition}
\newcommand*{\newscreentheoremx}[2]{
  \newenvironment{#1}[1][]{
    \begin{screen}
    \begin{#2}[##1]
      \leavevmode
      \newline
  }{
    \end{#2}
    \end{screen}
  }
}
\newcommand*{\newqedtheoremx}[2]{
  \newenvironment{#1}[1][]{
    \begin{#2}[##1]
      \leavevmode
      \newline
      \renewcommand{\qedsymbol}{\(\diamond\)}
      \pushQED{\qed}
  }{
      \qedhere
      \popQED
    \end{#2}
  }
}
\newtheorem{theorem*}{定理}

\newqedtheoremx{theorem}{theorem*}
\newcommand*\newqedtheorem@unstarred[2]{%
  \newtheorem{#1*}[theorem*]{#2}
  \newqedtheoremx{#1}{#1*}
}
\newcommand*\newqedtheorem@starred[2]{%
  \newtheorem*{#1*}{#2}
  \newqedtheoremx{#1}{#1*}
}
\newcommand*{\newqedtheorem}{\@ifstar{\newqedtheorem@starred}{\newqedtheorem@unstarred}}

\newtheorem{sctheorem*}{定理}
\newscreentheoremx{sctheorem}{sctheorem*}
\newcommand*\newscreentheorem@unstarred[2]{%
  \newtheorem{#1*}[theorem*]{#2}
  \newscreentheoremx{#1}{#1*}
}
\newcommand*\newscreentheorem@starred[2]{%
  \newtheorem*{#1*}{#2}
  \newscreentheoremx{#1}{#1*}
}
\newcommand*{\newscreentheorem}{\@ifstar{\newscreentheorem@starred}{\newscreentheorem@unstarred}}

%\newtheorem*{definition}{定義}
%\newtheorem{theorem}{定理}
%\newtheorem{proposition}[theorem]{命題}
%\newtheorem{lemma}[theorem]{補題}
%\newtheorem{corollary}[theorem]{系}

\newqedtheorem{lemma}{補題}
\newqedtheorem{corollary}{系}
\newqedtheorem{example}{例}
\newqedtheorem{proposition}{命題}
\newqedtheorem{remark}{注意}
\newqedtheorem{thesis}{主張}
\newqedtheorem{notation}{記法}
\newqedtheorem{problem}{問題}
\newqedtheorem{algorithm}{アルゴリズム}

\newscreentheorem*{definition}{定義}

\renewenvironment{proof}[1][\proofname]{\par
  \normalfont
  \topsep6\p@\@plus6\p@ \trivlist
  \item[\hskip\labelsep{\bfseries #1}\@addpunct{\bfseries}]\ignorespaces\quad\par
}{%
  \qed\endtrivlist\@endpefalse
}
\renewcommand\proofname{証明}

\makeatother

\begin{document}
\maketitle

\section{フーリエ解析}
\subsection{フーリエ級数}
\begin{definition}[内積]
  関数の正規直交関数系による展開
  区間 $[a, b]$ 上の
\end{definition}


\begin{definition}[複素フーリエ級数]
  $\TT = \RR/(2\pi\ZZ)$ 上 関数 $f: \TT\to\CC$ に対し
  区間 $[-\pi, \pi]$ において定義された実数値関数 $f(x)$ が連続かつ区分的に $C^1$ 級かつ周期的である ($f(-\pi) = f(\pi)$) ならば $f(x)$ は
  \begin{align}
    f(x) & = \sum_{n\in\ZZ}c_ne^{inx}                          \\
    c_n  & := \frac{1}{2\pi}\int_{-\pi}^\pi f(x)e^{-inx}\dd{x}
  \end{align}
\end{definition}
例
\begin{theorem}[Bessel の不等式]
  \begin{align}
    \sum_{n\in\ZZ}|\hat{f}(n)|^2 \leq \|f\|_2^2
  \end{align}
\end{theorem}


\begin{theorem}[平均値の定理]
  区間 $[a, b]$ で連続、 $(a, b)$ で微分可能な関数 $f(x)$ について $a < c < b$ となる $c$ が存在して次のようになる。
  \begin{align}
    \frac{f(b) - f(a)}{b - a} = f'(c)
  \end{align}
\end{theorem}

\begin{proposition}
  \begin{align}
     & \int_{-\pi}^\pi \sin(mx)\cos(nx)\dd{x} = 0               \\
     & \int_{-\pi}^\pi \cos(mx)\cos(nx)\dd{x} = \pi\delta_{m,n} \\
     & \int_{-\pi}^\pi \sin(mx)\sin(nx)\dd{x} = \pi\delta_{m,n} \\
     & \int_{-\pi}^\pi \cos(nx)\dd{x} = 2\pi\delta_{n,0}        \\
     & \int_{-\pi}^\pi \sin(nx)\dd{x} = 0
  \end{align}
\end{proposition}
\begin{proof}
  \begin{align}
    \int_{-\pi}^\pi \sin(mx)\cos(nx)\dd{x} & = \frac{1}{2}\int_{-\pi}^\pi \qty[\sin(m + n)x + \sin(m - n)x]\dd{x}                           \\
                                           & = \frac{1}{2}\qty[- \frac{\cos(m + n)x}{m + n} - \frac{\cos(m - n)x}{m - n}]_{-\pi}^\pi        \\
                                           & = 0                                                                                            \\
    \int_{-\pi}^\pi \cos(mx)\cos(nx)\dd{x} & = \frac{1}{2}\int_{-\pi}^\pi \qty[\cos(m - n)x + \cos(m + n)x]\dd{x}                           \\
                                           & = \begin{dcases}
                                                 \frac{1}{2}\qty[\frac{\sin(m - n)x}{m - n} + \frac{\sin(m + n)x}{m + n}]_{-\pi}^\pi & (m \neq n) \\
                                                 \frac{1}{2}\qty[x + \frac{\sin(m + n)x}{m + n}]_{-\pi}^\pi                          & (m = n)
                                               \end{dcases} \\
                                           & = \pi\delta_{m,n}                                                                              \\
    \int_{-\pi}^\pi \sin(mx)\sin(nx)\dd{x} & = \frac{1}{2}\int_{-\pi}^\pi \qty[\cos(m - n)x - \cos(m + n)x]\dd{x}                           \\
                                           & = \begin{dcases}
                                                 \frac{1}{2}\qty[\frac{\sin(m - n)x}{m - n} - \frac{\sin(m + n)x}{m + n}]_{-\pi}^\pi & (m \neq n) \\
                                                 \frac{1}{2}\qty[x - \frac{\sin(m + n)x}{m + n}]_{-\pi}^\pi                          & (m = n)
                                               \end{dcases} \\
                                           & = \pi\delta_{m,n}
  \end{align}
  \begin{align}
    \int_{-\pi}^\pi \cos(nx)\dd{x} & = \begin{dcases}
                                         \qty[\frac{\sin(nx)}{n}]_{-\pi}^\pi & (n \neq 0) \\
                                         \qty[x]_{-\pi}^\pi                  & (n = 0)
                                       \end{dcases}  \\
                                   & = 2\pi\delta_{n,0}                                      \\
    \int_{-\pi}^\pi \sin(nx)\dd{x} & = \begin{dcases}
                                         \qty[-\frac{\cos(nx)}{n}]_{-\pi}^\pi & (n \neq 0) \\
                                         \qty[0]_{-\pi}^\pi                   & (n = 0)
                                       \end{dcases} \\
                                   & = 0
  \end{align}
\end{proof}

\begin{definition}[$2\pi$ の周期をもつ関数のフーリエ級数]
  \begin{align}
    f(x) & \sim \frac{a_0}{2} + \sum_{n=1}^{\infty}(a_n\cos(nx) + b_n\sin(nx)) \\
    a_n  & := \frac{1}{\pi}\int_{-\pi}^\pi f(x)\cos(nx)\dd{x}                  \\
    b_n  & := \frac{1}{\pi}\int_{-\pi}^\pi f(x)\sin(nx)\dd{x}
  \end{align}
\end{definition}

\begin{definition}[$2\pi$ の周期をもつ関数の複素フーリエ級数]
  \begin{align}
    f(x) & = \sum_{n=-\infty}^{\infty}c_ne^{inx}               \\
    c_n  & := \frac{1}{2\pi}\int_{-\pi}^\pi f(x)e^{-inx}\dd{x}
  \end{align}
\end{definition}
\begin{proof}
  \begin{align}
    f(x) & \sim \frac{a_0}{2} + \sum_{n=1}^{\infty}(a_n\cos(nx) + b_n\sin(nx))                                               \\
         & = \frac{a_0}{2} + \sum_{n=1}^{\infty}\qty(\frac{a_n}{2}(e^{inx} + e^{-inx}) + \frac{b_n}{2i}(e^{inx} - e^{-inx})) \\
         & = \frac{a_0}{2} + \sum_{n=1}^{\infty}\qty(\frac{a_n - ib_n}{2}e^{inx} + \frac{a_n + ib_n}{2}e^{-inx})             \\
         & = \sum_{n=-\infty}^{\infty}c_ne^{inx}
  \end{align}
  ただし $c_n$ は次のように定める。
  \begin{align}
    c_n & := \begin{dcases}
               \frac{a_n - ib_n}{2} & (n > 0) \\
               \frac{a_0}{2}        & (n = 0) \\
               \frac{a_n - ib_n}{2} & (n < 0)
             \end{dcases}
  \end{align}
  \begin{align}
    \frac{1}{2\pi}\int_{-\pi}^\pi f(x)e^{-inx}\dd{x} & = \frac{1}{2\pi}\int_{-\pi}^\pi\sum_{m=-\infty}^{\infty}c_me^{imx}e^{-inx}\dd{x} \\
                                                     & = \frac{1}{2\pi}\sum_{m=-\infty}^{\infty}c_m\int_{-\pi}^\pi e^{i(m - n)x}\dd{x}  \\
                                                     & = \frac{1}{2\pi}\sum_{m=-\infty}^{\infty}c_m 2\pi\delta_{m,n}                    \\
                                                     & = c_n
  \end{align}
\end{proof}

\begin{definition}[一般の周期をもつ関数のフーリエ級数]
  \begin{align}
    f(x) & \sim \frac{a_0}{2} + \sum_{n=1}^{\infty}\qty(a_n\cos\frac{n\pi x}{l} + b_n\sin\frac{n\pi x}{l}) \\
    a_n  & := \frac{1}{l}\int_{-l}^\pi f(x)\cos\frac{n\pi x}{l}\dd{x}                                      \\
    b_n  & := \frac{1}{l}\int_{-l}^\pi f(x)\sin\frac{n\pi x}{l}\dd{x}
  \end{align}
\end{definition}
\begin{definition}[一般の周期をもつ関数の複素フーリエ級数]
  \begin{align}
    f(x) & = \sum_{n=-\infty}^{\infty}c_ne^{i\frac{n\pi}{l}x}         \\
    c_n  & := \frac{1}{2l}\int_{-l}^l f(x)e^{-i\frac{n\pi}{l}x}\dd{x}
  \end{align}
\end{definition}

\begin{theorem}
  \begin{align}
    \lim_{n\to\infty}a_n = 0
  \end{align}
\end{theorem}

\begin{lemma}[コーシーの不等式]
  実数の数列 $\lbrace p_n\rbrace_n, \lbrace q_n\rbrace_n$ について次の不等式が成立する。
  \begin{align}
    \qty(\sum_{n=1}^{N}p_n^2)\qty(\sum_{n=1}^{N}q_n^2) \geq \qty(\sum_{n=1}^{N}p_nq_n)^2
  \end{align}
\end{lemma}
\begin{proof}
  $x$ について次の 2 次関数の判別式を考えることで求まる。
  \begin{align}
    \sum_{n=1}^{N}(p_nx + q_n)^2 \geq 0
  \end{align}
\end{proof}

\begin{theorem}[ワイエルシュトラスの M テスト]
  区間 $[a, b]$ で定義された関数列の無限級数 $s(x)$ の各項の絶対値が上界 $M_n$ をもち、 $M_n$ の総和が収束するならばもとの級数は $[a, b]$ で一様収束する。
  \begin{align}
    s(x) = \sum_{n=1}^{\infty}f_n(x)
  \end{align}
\end{theorem}
\begin{proof}

\end{proof}
\end{document}