\RequirePackage{plautopatch}
\documentclass[uplatex,dvipdfmx,a4paper,11pt]{jlreq}
\input{preamble.tex}
%
\title{フーリエ解析}
\author{Anko}


\begin{document}
\maketitle
\tableofcontents
\clearpage

\section{フーリエ解析}
\subsection{フーリエ級数}
\begin{definition}[内積]
  関数の正規直交関数系による展開
  区間 $[a, b]$ 上の
\end{definition}


\begin{definition}[複素フーリエ級数]
  $\TT = \RR/(2\pi\ZZ)$ 上 関数 $f: \TT\to\CC$ に対し
  区間 $[-\pi, \pi]$ において定義された実数値関数 $f(x)$ が連続かつ区分的に $C^1$ 級かつ周期的である ($f(-\pi) = f(\pi)$) ならば $f(x)$ は
  \begin{align}
    f(x) & = \sum_{n\in\ZZ}c_ne^{inx}                          \\
    c_n  & := \frac{1}{2\pi}\int_{-\pi}^\pi f(x)e^{-inx}\dl{x}
  \end{align}
\end{definition}
例
\begin{theorem}[Bessel の不等式]
  \begin{align}
    \sum_{n\in\ZZ}|\hat{f}(n)|^2 \leq \|f\|_2^2
  \end{align}
\end{theorem}


\begin{theorem}[平均値の定理]
  区間 $[a, b]$ で連続、 $(a, b)$ で微分可能な関数 $f(x)$ について $a < c < b$ となる $c$ が存在して次のようになる。
  \begin{align}
    \frac{f(b) - f(a)}{b - a} = f'(c)
  \end{align}
\end{theorem}

\begin{proposition}
  \begin{align}
     & \int_{-\pi}^\pi \sin(mx)\cos(nx)\dl{x} = 0               \\
     & \int_{-\pi}^\pi \cos(mx)\cos(nx)\dl{x} = \pi\delta_{m,n} \\
     & \int_{-\pi}^\pi \sin(mx)\sin(nx)\dl{x} = \pi\delta_{m,n} \\
     & \int_{-\pi}^\pi \cos(nx)\dl{x} = 2\pi\delta_{n,0}        \\
     & \int_{-\pi}^\pi \sin(nx)\dl{x} = 0
  \end{align}
\end{proposition}
\begin{proof}
  \begin{align}
    \int_{-\pi}^\pi \sin(mx)\cos(nx)\dl{x} & = \frac{1}{2}\int_{-\pi}^\pi \ab[\sin(m + n)x + \sin(m - n)x]\dl{x}                           \\
                                           & = \frac{1}{2}\ab[- \frac{\cos(m + n)x}{m + n} - \frac{\cos(m - n)x}{m - n}]_{-\pi}^\pi        \\
                                           & = 0                                                                                           \\
    \int_{-\pi}^\pi \cos(mx)\cos(nx)\dl{x} & = \frac{1}{2}\int_{-\pi}^\pi \ab[\cos(m - n)x + \cos(m + n)x]\dl{x}                           \\
                                           & = \begin{dcases}
                                                 \frac{1}{2}\ab[\frac{\sin(m - n)x}{m - n} + \frac{\sin(m + n)x}{m + n}]_{-\pi}^\pi & (m \neq n) \\
                                                 \frac{1}{2}\ab[x + \frac{\sin(m + n)x}{m + n}]_{-\pi}^\pi                          & (m = n)
                                               \end{dcases} \\
                                           & = \pi\delta_{m,n}                                                                             \\
    \int_{-\pi}^\pi \sin(mx)\sin(nx)\dl{x} & = \frac{1}{2}\int_{-\pi}^\pi \ab[\cos(m - n)x - \cos(m + n)x]\dl{x}                           \\
                                           & = \begin{dcases}
                                                 \frac{1}{2}\ab[\frac{\sin(m - n)x}{m - n} - \frac{\sin(m + n)x}{m + n}]_{-\pi}^\pi & (m \neq n) \\
                                                 \frac{1}{2}\ab[x - \frac{\sin(m + n)x}{m + n}]_{-\pi}^\pi                          & (m = n)
                                               \end{dcases} \\
                                           & = \pi\delta_{m,n}
  \end{align}
  \begin{align}
    \int_{-\pi}^\pi \cos(nx)\dl{x} & = \begin{dcases}
                                         \ab[\frac{\sin(nx)}{n}]_{-\pi}^\pi & (n \neq 0) \\
                                         \ab[x]_{-\pi}^\pi                  & (n = 0)
                                       \end{dcases}  \\
                                   & = 2\pi\delta_{n,0}                                     \\
    \int_{-\pi}^\pi \sin(nx)\dl{x} & = \begin{dcases}
                                         \ab[-\frac{\cos(nx)}{n}]_{-\pi}^\pi & (n \neq 0) \\
                                         \ab[0]_{-\pi}^\pi                   & (n = 0)
                                       \end{dcases} \\
                                   & = 0
  \end{align}
\end{proof}

\begin{definition}[$2\pi$ の周期をもつ関数のフーリエ級数]
  \begin{align}
    f(x) & \sim \frac{a_0}{2} + \sum_{n=1}^{\infty}(a_n\cos(nx) + b_n\sin(nx)) \\
    a_n  & := \frac{1}{\pi}\int_{-\pi}^\pi f(x)\cos(nx)\dl{x}                  \\
    b_n  & := \frac{1}{\pi}\int_{-\pi}^\pi f(x)\sin(nx)\dl{x}
  \end{align}
\end{definition}

\begin{definition}[$2\pi$ の周期をもつ関数の複素フーリエ級数]
  \begin{align}
    f(x) & = \sum_{n=-\infty}^{\infty}c_ne^{inx}               \\
    c_n  & := \frac{1}{2\pi}\int_{-\pi}^\pi f(x)e^{-inx}\dl{x}
  \end{align}
\end{definition}
\begin{proof}
  \begin{align}
    f(x) & \sim \frac{a_0}{2} + \sum_{n=1}^{\infty}(a_n\cos(nx) + b_n\sin(nx))                                              \\
         & = \frac{a_0}{2} + \sum_{n=1}^{\infty}\ab(\frac{a_n}{2}(e^{inx} + e^{-inx}) + \frac{b_n}{2i}(e^{inx} - e^{-inx})) \\
         & = \frac{a_0}{2} + \sum_{n=1}^{\infty}\ab(\frac{a_n - ib_n}{2}e^{inx} + \frac{a_n + ib_n}{2}e^{-inx})             \\
         & = \sum_{n=-\infty}^{\infty}c_ne^{inx}
  \end{align}
  ただし $c_n$ は次のように定める。
  \begin{align}
    c_n & := \begin{dcases}
               \frac{a_n - ib_n}{2} & (n > 0) \\
               \frac{a_0}{2}        & (n = 0) \\
               \frac{a_n - ib_n}{2} & (n < 0)
             \end{dcases}
  \end{align}
  \begin{align}
    \frac{1}{2\pi}\int_{-\pi}^\pi f(x)e^{-inx}\dl{x} & = \frac{1}{2\pi}\int_{-\pi}^\pi\sum_{m=-\infty}^{\infty}c_me^{imx}e^{-inx}\dl{x} \\
                                                     & = \frac{1}{2\pi}\sum_{m=-\infty}^{\infty}c_m\int_{-\pi}^\pi e^{i(m - n)x}\dl{x}  \\
                                                     & = \frac{1}{2\pi}\sum_{m=-\infty}^{\infty}c_m 2\pi\delta_{m,n}                    \\
                                                     & = c_n
  \end{align}
\end{proof}

\begin{definition}[一般の周期をもつ関数のフーリエ級数]
  \begin{align}
    f(x) & \sim \frac{a_0}{2} + \sum_{n=1}^{\infty}\ab(a_n\cos\frac{n\pi x}{L} + b_n\sin\frac{n\pi x}{L}) \\
    a_n  & := \frac{1}{L}\int_{-L}^\pi f(x)\cos\frac{n\pi x}{L}\dl{x}                                     \\
    b_n  & := \frac{1}{L}\int_{-L}^\pi f(x)\sin\frac{n\pi x}{L}\dl{x}
  \end{align}
\end{definition}
\begin{definition}[一般の周期をもつ関数の複素フーリエ級数]
  \begin{align}
    f(x) & = \sum_{n=-\infty}^{\infty}c_ne^{i\frac{n\pi}{L}x}         \\
    c_n  & := \frac{1}{2L}\int_{-L}^L f(x)e^{-i\frac{n\pi}{L}x}\dl{x}
  \end{align}
\end{definition}

\begin{theorem}
  \begin{align}
    \lim_{n\to\infty}a_n = 0
  \end{align}
\end{theorem}

\begin{lemma}[コーシーの不等式]
  実数の数列 $\lbrace p_n\rbrace_n, \lbrace q_n\rbrace_n$ について次の不等式が成立する。
  \begin{align}
    \ab(\sum_{n=1}^{N}p_n^2)\ab(\sum_{n=1}^{N}q_n^2) \geq \ab(\sum_{n=1}^{N}p_nq_n)^2
  \end{align}
\end{lemma}
\begin{proof}
  $x$ について次の 2 次関数の判別式を考えることで求まる。
  \begin{align}
    \sum_{n=1}^{N}(p_nx + q_n)^2 \geq 0
  \end{align}
\end{proof}

\begin{theorem}[ワイエルシュトラスの M テスト]
  区間 $[a, b]$ で定義された関数列の無限級数 $s(x)$ の各項の絶対値が上界 $M_n$ をもち、 $M_n$ の総和が収束するならばもとの級数は $[a, b]$ で一様収束する。
  \begin{align}
    s(x) = \sum_{n=1}^{\infty}f_n(x)
  \end{align}
\end{theorem}
\begin{proof}

\end{proof}

\begin{theorem}
  $f(x) = x^n$ を $[-1, 1]$ で Fourier 変換を行うことで $\zeta$ 関数の値がわかる
\end{theorem}
\begin{proof}
  偶関数のとき $f(x) = x^{2n}$ となる。
  \begin{align}
    f(x) & = \frac{a_0}{2} + \sum_{n=1}^{\infty}\ab(a_n\cos(n\pi x))
  \end{align}
  \begin{align}
    a_0 & = \int_{-1}^{1}x^{2m}\dl{x} = \frac{2}{2m+1}                               \\
    a_n & = \int_{-1}^{1}x^{2m}\cos(n\pi x)\dl{x}                                    \\
        & = 2\int_{0}^{1}x^{2m}\cos(n\pi x)\dl{x}                                    \\
        & = 2\sum_{k=0}^{2m}(-1)^k\ab[(x^{2m})^{(k)}(\cos(n\pi x))^{(-k-1)}]_{0}^{1}
  \end{align}
  $k$ が偶数のとき $\sin(n\pi x)$ があるから $x = 0, 1$ 両方で $0$ となる。
  また $k \neq 2m$ のとき $x = 0$ で $0$ となる。
  これより $k$ が奇数 ($2s - 1$) かつ $x = 1$ のときのみを考えればよい。
  \begin{align}
    a_n & = -2\sum_{s=1}^{m}\ab[\frac{(2m)!}{(2m - k)!}x^{2m-k}\frac{(-1)^s}{(n\pi)^{k+1}}\cos(n\pi x)]_0^1 \\
        & = -2\sum_{s=1}^{m}\frac{(2m)!}{(2m - 2s + 1)!}\frac{(-1)^{s + n}}{(n\pi)^{2s}}
  \end{align}
  \begin{align}
    x^{2m} & = \frac{1}{2m+1} - 2\sum_{n=1}^{\infty}\sum_{s=1}^{m}\frac{(2m)!}{(2m - 2s + 1)!}\frac{(-1)^{s + n}}{(n\pi)^{2s}}\cos(n\pi x)
  \end{align}
  $m = 1$ のとき
  \begin{align}
    x^2 & = \frac{1}{3} + 4\sum_{n=1}^{\infty}(-1)^{n}\frac{\cos(n\pi x)}{(n\pi)^{2}}
  \end{align}
  \begin{align}
    0^2 & = \frac{1}{3} + \frac{4}{\pi^2}\sum_{n=1}^{\infty}\frac{(-1)^{n}}{n^2} \\
    1^2 & = \frac{1}{3} + \frac{4}{\pi^2}\sum_{n=1}^{\infty}\frac{1}{n^2}
  \end{align}
  \begin{align}
    \eta(2)  & = \frac{\pi^2}{12} \\
    \zeta(2) & = \frac{\pi^2}{6}
  \end{align}
  \begin{align}
    \frac{1}{L}\int_{L}^{-L}(f(x))^2\dl{x} & = \frac{a_0^2}{2} + \sum_{n=1}^{\infty}(a_n^2 + b_n^2)
  \end{align}
  \begin{align}
    \zeta(4) = \frac{\pi^4}{90}
  \end{align}
  奇関数 $f(x) = x^{2m - 1}$
  \begin{align}
    f(x) & = \sum_{n=1}^{\infty}\ab(b_n\sin(n\pi x))                                          \\
    b_n  & = \int_{-1}^{1}x^{2m - 1}\sin(n\pi x)\dl{x}                                        \\
         & = \sum_{k=0}^{2m - 1}(-1)^k\ab[(x^{2m - 1})^{(k)}(\sin(n\pi x))^{(-k-1)}]_{-1}^{1}
  \end{align}
  $k$ が奇数のとき $\sin$ でどちらも $0$
  $k = 2s$
  \begin{align}
    b_n & = \sum_{s=0}^{m - 1}\ab[\frac{(2m - 1)!}{(2m - 1 - k)!}x^{2m - 1 - k}\frac{(-1)^{s+1}}{(n\pi)^{k + 1}}\cos(n\pi x)]_{-1}^{1} \\
        & = \sum_{s=0}^{m - 1}\frac{(2m - 1)!}{(2m - 1 - 2s)!}\frac{(-1)^{s+1}}{(n\pi)^{2s + 1}}\ab((-1)^n - (-1)^{2m - 1 - 2s + n})   \\
        & = 2\sum_{s=0}^{m - 1}\frac{(2m - 1)!}{(2m - 1 - 2s)!}\frac{(-1)^{s+n+1}}{(n\pi)^{2s + 1}}
  \end{align}
  \begin{align}
    f(x) & = 2\sum_{n=1}^{\infty}\sum_{s=0}^{m - 1}\frac{(2m - 1)!}{(2m - 1 - 2s)!}\frac{(-1)^{s+n+1}}{(n\pi)^{2s + 1}}\sin(n\pi x)
  \end{align}

  $m = 2$ のとき
  \begin{align}
    x^3 & = \sum_{n=1}^{\infty}(-1)^{n}\ab(-\frac{2}{n\pi} + \frac{12}{(n\pi)^{3}})\sin(n\pi x)
  \end{align}
  \begin{align}
    \ab(\frac{1}{2})^3 & = \sum_{n=1}^{\infty}(-1)^{n}\ab(-\frac{2}{n\pi} + \frac{12}{(n\pi)^{3}})\sin\ab(\frac{n\pi}{2}) \\
                       & = -\sum_{n=1}^{\infty}\ab(-\frac{2}{(2n - 1)\pi} + \frac{12}{((2n - 1)\pi)^{3}})(-1)^n           \\
                       & = \frac{2}{\pi}\varepsilon(1) - \frac{12}{\pi^{3}}\varepsilon(3)                                 \\
  \end{align}
  パーバセル
  \begin{align}
    \zeta(6)
  \end{align}

\end{proof}



\end{document}