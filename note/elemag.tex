\documentclass[uplatex,a4paper,dvipdfmx]{jsarticle}

\usepackage{okumacro}
%ルビ用%
\usepackage{indentfirst}
%字下げを保存するための設定 \parでインデント+改行%
\usepackage[dvipdfmx]{graphicx}
%画像挿入パッケージ。graphix=Windows,graphics=Mac%
\usepackage{wrapfig}
%文章を図に回り込ませるパッケージ%
\usepackage{amsfonts}
\usepackage{amssymb}
%数式色々%
\usepackage{bm}
%ベクトル%
\usepackage{url}
%url中の_や\にエラーをはかせないためのパッケージ%
\usepackage{comment}
%複数行コメントのためのパッケージ%
\usepackage{listings}
%コードのためのパッケージ(英語のみ)%
\usepackage{physics}
%物理関係のパッケージ%
\usepackage{amsmath}
\usepackage{mathcomp}
\usepackage{mathtools}
%定理証明関係のパッケージ%
\usepackage{amsthm}
%amsthm%
\theoremstyle{definition}
\newtheorem{dfn}{Definition}[section]
\newtheorem{prop}[dfn]{Proposition}
\newtheorem{lem}[dfn]{Lemma}
\newtheorem{thm}[dfn]{Theorem}
\newtheorem{cor}[dfn]{Corollary}
\newtheorem{rem}[dfn]{Remark}
\newtheorem{fact}[dfn]{Fact}
\renewcommand{\qedsymbol}{$\blacksquare$}
%数学関係のパッケージ%
\usepackage{docmute}
%ファイル分割%
\lstset{
  language=C++,
  breaklines=true,
  keywordstyle = {\color[rgb]{0,0,1}},
  stringstyle = {\color[rgb]{1,0,0}},
  commentstyle = { \color[rgb]{0,1,0}},
  numbers=left,
  frame=lines
}
%各種設定%
\usepackage{color}
%色付け 使うときは\documentclass[dvipdfmx]を追加すること!%
\usepackage{ascmac}
\usepackage{otf}
%ギリシャ数字%
\usepackage{siunitx}
%SI単位系%
\usepackage{tikz}
%tikz%
\usetikzlibrary{intersections, calc, arrows, positioning, arrows.meta,automata}
%tikzlibrary%
\newcommand{\LR}{\Leftrightarrow}
\begin{document}
\title{}
\author{
  学籍番号:21B00349\\
  氏名:宇佐見 大希\\
}
\maketitle

\newcommand{\ZZ}{\mathbb{Z}}
\newcommand{\RR}{\mathbb{R}}
\newcommand{\CC}{\mathbb{C}}
\newcommand{\EE}{\vb*{E}}
\newcommand{\Et}{\tilde{E}}
\newcommand{\ET}{\tilde{\vb*{E}}}
\newcommand{\Ec}{\mathcal{E}}
\newcommand{\EC}{\vb*{\mathcal{E}}}
\newcommand{\BB}{\vb*{B}}
\newcommand{\LL}{\vb*{L}}
\renewcommand{\AA}{\vb*{A}}
\newcommand{\ee}{\vb*{\epsilon}}
\newcommand{\kk}{\vb*{k}}
\newcommand{\rr}{\vb*{r}}
\renewcommand{\SS}{\vb*{S}}
\newcommand{\JJ}{\vb*{J}}
\newcommand{\ii}{\mathrm{i}}

虚数単位は立体 $\ii$ で書くこととする.

\section{直線偏光した基底による電磁波の展開}

位置 $\rr$, 時刻 $t$ において, 真空中に伝搬する電磁波の複素電場 $\tilde{\EE}(\rr, t)$, 複素磁場 $\tilde{\BB}(\rr, t)$ の一般解は次のように表される.
\begin{align}
  \tilde{\EE}(\rr, t) & = \iiint_{\RR^3} dV(\kk)(E_1(\kk)\ee_1(\kk) + E_2(\kk)\ee_2(\kk))e^{\ii(\kk\cdot\rr - \omega(\kk)t)} \label{tilde E}  \\
  \tilde{\BB}(\rr, t) & = \iiint_{\RR^3} dV(\kk)(B_1(\kk)\ee_1(\kk) + B_2(\kk)\ee_2(\kk))e^{\ii(\kk\cdot\rr - \omega(\kk)t)}. \label{tilde B}
\end{align}
電磁波の分散関係 $\omega(\kk)$ は光速度 $c$ を用いて次のように与えられる.
\begin{align}
  \omega(\kk) = c\lvert\kk\rvert.
\end{align}
また, 各 $\kk\in\RR^3$ に対して, $\ee_1(\kk), \ee_2(\kk), \kk/\lvert\kk\rvert\in\RR^3$ は右手系の正規直交系を成す.

\begin{align}
  |\ee_1(\kk)| = |\ee_2(\kk)| & = \left|\frac{\kk}{|\kk|}\right| = 1                                      \\
  \ee_1(\kk)\cdot\ee_2(\kk)   & = \ee_1(\kk)\cdot\frac{\kk}{|\kk|} = \ee_2(\kk)\cdot\frac{\kk}{|\kk|} = 0 \\
  \ee_1(\kk)\times\ee_2(\kk)  & = \frac{\kk}{|\kk|}.
\end{align}

また, 各 $\kk\in\RR^3$ に対して, $E_1(\kk), E_2(\kk)\in\CC$ と $B_1(\kk), B_2(\kk)\in\CC$ は, 電場 $\tilde{\EE}(\rr, t)$ と磁場 $\tilde{\BB}(\rr, t)$ の波数 $\kk$ のそれぞれ $\ee_1(\kk)$, $\ee_2(\kk)$ 方向の複素振幅である. これらは CGS Gauss 単位系では次の関係式を満たす.
\begin{align}
  B_1(\kk) = -E_2(\kk), B_2(\kk) = E_1(\kk).
\end{align}

また, $\kk/|\kk| = 0$ のとき電磁波が横波であることを表す.
このとき, $\tilde{\EE}(\rr, t), \tilde{\BB}(\rr, t)$ の展開式 \eqref{tilde E} \eqref{tilde B} は「電磁波に対する直線偏光をもつ基底による展開」と呼ばれる. 横波のときの基底は
\begin{align}
  \ee_1(\kk)e^{\ii(\kk\cdot\rr-\omega(\kk)t)}, \ee_2(\kk)e^{\ii(\kk\cdot\rr-\omega(\kk)t)}
\end{align}
と書け, これらを「直線偏光を持つ基底」と呼ばれる.

なお, 物理的な電場 $\EE(\rr, t)$, 磁場 $\BB(\rr, t)$ はそれらの複素表現の実部を取ることで求められる.
\begin{align}
  \EE(\rr, t) = \Re\tilde{\EE}(\rr, t), \BB(\rr, t) = \Re\tilde{\BB}(\rr, t)
\end{align}

\textbf{Q 21B-1.} よって次のように電場の複素振幅の絶対値 $a_i(\kk)$ と偏角 $\epsilon_i(\kk)$ をおくと
\begin{align}
  E_i(\kk) = a_i(\kk) e^{\ii\epsilon_i(\kk)}, a_i(\kk) \geq 0, \epsilon_i(\kk)\in\RR \quad (\kk\in\RR^3, i = 1, 2)
\end{align}

次のように展開される.
\begin{align}
  \EE(\rr, t) & = \Re\tilde{\EE}(\rr, t)                                                                                                                                                                                          \\
              & = \Re\iiint_{\RR^3} dV(\kk)(E_1(\kk)\ee_1(\kk) + E_2(\kk)\ee_2(\kk))e^{\ii(\kk\vdot\rr - \omega(\kk)t)}                                                                                                           \\
              & = \iiint_{\RR^3} dV(\kk)\left(\Re(E_1(\kk)e^{\ii(\kk\cdot\rr - \omega(\kk)t)})\ee_1(\kk) + \Re(E_2(\kk)e^{\ii(\kk\cdot\rr - \omega(\kk)t)})\ee_2(\kk)\right)                                                      \\
              & = \iiint_{\RR^3} dV(\kk)\left(\Re(a_1(\kk)e^{\ii(\kk\cdot\rr - \omega(\kk)t + \epsilon_1(\kk))})\ee_1(\kk) + \Re(a_2(\kk)e^{\ii(\kk\cdot\rr - \omega(\kk)t + \epsilon_2(\kk))})\ee_2(\kk)\right)                  \\
              & = \iiint_{\RR^3} dV(\kk)\lbrace a_1(\kk)\cos(\kk\cdot\rr - \omega(\kk)t + \epsilon_1(\kk))\ee_1(\kk) + a_2(\kk)\cos(\kk\cdot\rr - \omega(\kk)t + \epsilon_2(\kk))\ee_2(\kk)\rbrace \label{E}                      \\
  \BB(\rr, t) & = \Re\tilde{\BB}(\rr, t)                                                                                                                                                                                          \\
              & = \iiint_{\RR^3} dV(\kk)\left(\Re(-E_2(\kk)e^{\ii(\kk\cdot\rr - \omega(\kk)t)})\ee_1(\kk) + \Re(E_1(\kk)e^{i(\kk\cdot\rr - \omega(\kk)t)})\ee_2(\kk)\right)                                                       \\
              & = \iiint_{\RR^3} dV(\kk)\lbrace -a_2(\kk)\cos(\kk\cdot\rr - \omega(\kk)t + \epsilon_2(\kk))\ee_1(\kk) + a_1(\kk)\cos(\kk\cdot\rr - \omega(\kk)t + \epsilon_1(\kk))\ee_2(\kk)\rbrace              & \Box \label{B}
\end{align}

これより真空中を伝搬する電場 $\EE(\rr, t)$ の表式 \eqref{E} が分かれば, 磁場 $\BB(\rr, t)$ の表式 \eqref{B} が分かる. \\

\textbf{Q 21B-2.} \\
ここでは直線偏光の単色波について考える. 単色波 (monochromatic wave) とは1つの振動数しか持たない波のことである. \\
波数 $\kk\in\RR^3$ で定まった電場の振動方向 $\ee_i(\kk)$ を持つ単色波のとき, 複素振幅を $\tilde{E} = ae^{\ii\epsilon}$ とおくと, 電場の複素表示 $\tilde{\EE}(\rr, t)$ は次の通りとなる.
\begin{align}
  \tilde{\EE}(\rr, t) = \tilde{E}\ee_i(\kk)e^{\ii(\kk\vdot\rr-\omega(\kk)t)}
\end{align}
これより物理的な電場は次のようになる.
\begin{align}
  \EE(\rr, t) & = \Re\tilde{\EE}(\rr, t)                                               \\
              & = \Re\left(\tilde{E}\ee_i(\kk)e^{\ii(\kk\vdot\rr-\omega(\kk)t)}\right) \\
              & = a\ee_i(\kk)\Re\cos(\kk\vdot\rr-\omega(\kk)t + \epsilon)
\end{align}
$\EE(\rr, t)$ の $\ee_1, \ee_2$ 方向の成分を $E_1(\rr, t), E_2(\rr, t)$ と書くとすると, $\ee_1, \ee_2$ 方向に進む場合について
\begin{align}
  \mqty[E_1 \\ E_2] &= \mqty[a\cos(\kk\vdot\rr - \omega(\kk)t + \epsilon) \\ 0] & (i = 1) \\
  \mqty[E_1 \\ E_2] &= \mqty[0 \\ a\cos(\kk\vdot\rr - \omega(\kk)t + \epsilon)] & (i = 2)
\end{align}
となる. 位置 $\rr$ を固定し, 時間 $t$ を動かしたときに $(E_1, E_2)$ が作る軌跡の図形は Lissajous 図形と呼ばれている. 電場の振動方向が $\ee_1, \ee_2$ 方向に進むとき, 上式より
\begin{align}
  \qty{\qty(E_1(\rr, t), E_2(\rr, t)):t\in\RR} & = \qty{\qty(E_1,E_2)\in\RR^2:-a\leq E_1\leq a\land E_2 = 0} & (i = 1) \\
  \qty{\qty(E_1(\rr, t), E_2(\rr, t)):t\in\RR} & = \qty{\qty(E_1,E_2)\in\RR^2:-a\leq E_2\leq a\land E_1 = 0} & (i = 2)
\end{align}
が成り立つ. \\
このような単色波を直線偏光を持つという. また, 電場ベクトルの振動方向と電磁場の進行方向で定まる平面を電場の振動面という.


\section{円偏光した基底による電磁波の展開}

まず円偏光した基底を定義する.
\begin{align}
  \ee_+(\kk) & e^{\ii(\kk\vdot\rr - \omega(\kk)t)},              & \ee_-(\kk) & e^{\ii(\kk\vdot\rr - \omega(\kk)t)}              & (\kk\in\RR^3) \\
  \ee_+(\kk) & = \frac{1}{\sqrt{2}}(\ee_1(\kk) + \ii\ee_2(\kk)), & \ee_-(\kk) & = \frac{1}{\sqrt{2}}(\ee_1(\kk) - \ii\ee_2(\kk)) & (\kk\in\RR^3)
\end{align}
このとき
\begin{align}
  \ee_\pm(\kk)^* = \ee_\mp(\kk)
\end{align}
が成り立つ.

\textbf{Q 21B-3.} 各 $\kk\in\RR^3$ に対して, $\ee_+(\kk), \ee_-(\kk), \kk/|\kk|\in\CC^3$ は正規直交基底を成す.
\begin{proof}
  \begin{align}
    \ee_+(\kk)^*\vdot\ee_+(\kk)                            & = \frac{1}{2}(\ee_1(\kk) - \ii\ee_2(\kk))\vdot(\ee_1(\kk) + \ii\ee_2(\kk)) \\
                                                           & = \frac{1}{2}(\ee_1(\kk)^2 + \ee_2(\kk)^2)                                 \\
                                                           & = 1                                                                        \\
    \ee_-(\kk)^*\vdot\ee_-(\kk)                            & = \frac{1}{2}(\ee_1(\kk) + \ii\ee_2(\kk))\vdot(\ee_1(\kk) - \ii\ee_2(\kk)) \\
                                                           & = \frac{1}{2}(\ee_1(\kk)^2 + \ee_2(\kk)^2)                                 \\
                                                           & = 1                                                                        \\
    \left(\frac{\kk}{|\kk|}\right)^*\vdot\frac{\kk}{|\kk|} & = 1                                                                        \\
    \ee_+(\kk)^*\vdot\ee_-(\kk)                            & = \frac{1}{2}(\ee_1(\kk) - \ii\ee_2(\kk))\vdot(\ee_1(\kk) - \ii\ee_2(\kk)) \\
                                                           & = \frac{1}{2}(\ee_1(\kk)^2 - \ee_2(\kk)^2 - 2\ii\ee_1(\kk)\vdot\ee_2(\kk)) \\
                                                           & = 0                                                                        \\
    \ee_+(\kk)^*\vdot\frac{\kk}{|\kk|}                     & = \frac{1}{\sqrt{2}}(\ee_1(\kk) - \ii\ee_2(\kk))\vdot\frac{\kk}{|\kk|}     \\
                                                           & = 0                                                                        \\
    \ee_-(\kk)^*\vdot\frac{\kk}{|\kk|}                     & = \frac{1}{\sqrt{2}}(\ee_1(\kk) + \ii\ee_2(\kk))\vdot\frac{\kk}{|\kk|}     \\
                                                           & = 0
  \end{align}
  よりこれらは正規直交系となる.
\end{proof}

\textbf{Q 21B-4.} 円偏光を持つ基底による電磁場の展開は次のように表される.
\begin{align}
  \tilde{\EE}(\rr, t) & = \iiint_{\RR^3} dV(\kk)\qty{E_+(\kk)\ee_+(\kk) + E_-(\kk)\ee_-(\kk)}e^{\ii(\kk\vdot\rr - \omega(\kk)t)} \\
  \tilde{\BB}(\rr, t) & = \iiint_{\RR^3} dV(\kk)\qty{B_+(\kk)\ee_+(\kk) + B_-(\kk)\ee_-(\kk)}e^{\ii(\kk\vdot\rr - \omega(\kk)t)}
\end{align}
ただし, 円偏光の複素振幅は次のように定義される.
\begin{align}
  E_+(\kk) & = \frac{1}{\sqrt{2}}\qty(E_1(\kk) - \ii E_2(\kk)), & E_-(\kk) & = \frac{1}{\sqrt{2}}\qty(E_1(\kk) + \ii E_2(\kk)) \\
  B_+(\kk) & = \frac{1}{\sqrt{2}}\qty(B_1(\kk) - \ii B_2(\kk)), & B_-(\kk) & = \frac{1}{\sqrt{2}}\qty(B_1(\kk) + \ii B_2(\kk))
\end{align}

\begin{proof}
  $\tilde{\EE}(\rr, t)$ と $\tilde{\BB}(\rr, t)$ の表示式 \eqref{tilde E}, \eqref{tilde B} より次のように式変形できる.
  \begin{align}
    \tilde{\EE}(\rr, t) & = \iiint_{\RR^3} dV(\kk)\qty(E_1(\kk)\ee_1(\kk) + E_2(\kk)\ee_2(\kk))e^{\ii(\kk\cdot\rr - \omega(\kk)t)}                                                  \\
                        & = \iiint_{\RR^3} dV(\kk)\left(\frac{1}{\sqrt{2}}\qty(E_1(\kk) - \ii E_2(\kk))\cdot\frac{1}{\sqrt{2}}\qty(\ee_1(\kk) + \ii\ee_2(\kk)) \right.              \\
                        & \left. + \frac{1}{\sqrt{2}}\qty(E_1(\kk) + \ii E_2(\kk))\cdot\frac{1}{\sqrt{2}}\qty(\ee_1(\kk) - \ii\ee_2(\kk))\right)e^{\ii(\kk\cdot\rr - \omega(\kk)t)} \\
                        & = \iiint_{\RR^3} dV(\kk)\qty{E_+(\kk)\ee_+(\kk) + E_-(\kk)\ee_-(\kk)}e^{\ii(\kk\vdot\rr - \omega(\kk)t)}                                                  \\
    \tilde{\BB}(\rr, t) & = \iiint_{\RR^3} dV(\kk)(B_1(\kk)\ee_1(\kk) + B_2(\kk)\ee_2(\kk))e^{\ii(\kk\cdot\rr - \omega(\kk)t)}                                                      \\
                        & = \iiint_{\RR^3} dV(\kk)\left(\frac{1}{\sqrt{2}}\qty(B_1(\kk) - \ii B_2(\kk))\cdot\frac{1}{\sqrt{2}}\qty(\ee_1(\kk) + \ii\ee_2(\kk)) \right.              \\
                        & \left. + \frac{1}{\sqrt{2}}\qty(B_1(\kk) + \ii B_2(\kk))\cdot\frac{1}{\sqrt{2}}\qty(\ee_1(\kk) - \ii\ee_2(\kk))\right)e^{\ii(\kk\cdot\rr - \omega(\kk)t)} \\
                        & = \iiint_{\RR^3} dV(\kk)\qty{B_+(\kk)\ee_+(\kk) + B_-(\kk)\ee_-(\kk)}e^{\ii(\kk\vdot\rr - \omega(\kk)t)}
  \end{align}
  これより示せた.
\end{proof}

\textbf{Q 21B-5.} 波数 $\kk\in\RR^3$ を持ち, 複素電場ベクトル $\tilde{\EE}(\rr, t)$ がベクトル $\ee_i(\kk)$ に比例している単色波を考える.
\begin{align}
  \tilde{\EE}(\rr, t) = \tilde{E}\ee_i(\kk)e^{\ii(\kk\vdot\rr-\omega(\kk)t)} \quad (i = +, -)
\end{align}
このとき物理的な電場は複素振幅を $\tilde{E} = ae^{\ii\varepsilon}$ とすると
\begin{align}
  \EE(\rr, t) & = \Re\tilde{\EE}(\rr, t)                                                                                                                        \\
              & = \Re\qty(\tilde{E}\ee_i(\kk)e^{\ii(\kk\vdot\rr - \omega(\kk)t)})                                                                               \\
              & = a\Re\qty(\ee_i(\kk)e^{\ii(\kk\vdot\rr - \omega(\kk)t + \varepsilon)})                                                                         \\
              & = \frac{a}{\sqrt{2}}\qty{\ee_1(\kk)\cos(\kk\vdot\rr - \omega(\kk)t + \varepsilon) \mp \ee_2(\kk)\sin(\kk\vdot\rr - \omega(\kk)t + \varepsilon)}
\end{align}
これより, $\EE(\rr, t)$ の $\ee_1, \ee_2$ 方向の成分を $E_1(\rr, t), E_2(\rr, t)$ と書くとすると, $\ee_\pm$ 方向に進む場合について
\begin{equation}
  \mqty[E_1 \\ E_2] = \frac{a}{\sqrt{2}}\mqty[\cos(\kk\vdot\rr - \omega(\kk)t + \varepsilon) \\ \mp\sin(\kk\vdot\rr - \omega(\kk)t + \varepsilon)] \quad (i = \pm). \label{circle E vector}
\end{equation}
となり, この軌跡である Lissajous 図形は次のように原点を中心とする半径 $a/\sqrt{2}$ の円となる.
\begin{align}
  \qty{\qty(E_1(\rr, t), E_2(\rr, t)):t\in\RR} = \qty{\qty(E_1, E_2)\in\RR^2:E_1^2 + E_2^2 = \frac{a^2}{2}} \quad (i=\pm)
\end{align}
これよりこのときの電場は円偏光を持つという. また $i = +$ と $i = -$ については \eqref{circle E vector} より $t$ が増加する方向を考えると次のように解釈できる.
\begin{align}
  i & = + \iff \text{「Lissajous 図形の円上を点 $(E_1, E_2)$ は左回りに周回する。」} \\
  i & = - \iff \text{「Lissajous 図形の円上を点 $(E_1, E_2)$ は右回りに周回する。」}
\end{align}
また $\ee_1(\kk)\times\ee_2(\kk) = \kk/|\kk|$ より次のようにも解釈できる.
\begin{align}
  i & = + \iff \text{「やってくる光に向かって観測すると, 電場ベクトルは左回りに回転する。」} \\
  i & = - \iff \text{「やってくる光に向かって観測すると, 電場ベクトルは右回りに回転する。」}
\end{align}
光の helicity という言葉を用いると
\begin{align}
  i & = + \iff \text{「円偏光の helicity は $+1$ である。」} \\
  i & = - \iff \text{「円偏光の helicity は $-1$ である。」}
\end{align}
と説明する.

\section{一般の偏光状態と偏光楕円}
一般の偏光状態は複素係数 $\tilde{E}_1, \tilde{E}_2\in\CC$ を用いて
\begin{align}
  \tilde{\EE}(\rr, t) & = \tilde{E}_1\ee_1(\kk)e^{\ii(\kk\vdot\rr - \omega(\kk)t)} + \tilde{E}_2\ee_2(\kk)e^{\ii(\kk\vdot\rr - \omega(\kk)t)} \label{tilde E def}
\end{align}
で与えられる.

\textbf{Q 21B-6.} 物理的な電場 $\EE(\rr, t)$ は複素振幅を $\tilde{E}_i = a_ie^{\ii\varepsilon_i}, a_i \geq 0, \varepsilon_i\in\RR \quad (i = 1, 2)$ とすると
\begin{align}
  \EE(\rr, t) & = \Re\tilde{\EE}(\rr, t)                                                                                                                      \\
              & = \Re\qty(\tilde{E}_1\ee_1(\kk)e^{\ii(\kk\vdot\rr - \omega(\kk)t)} + \tilde{E}_2\ee_2(\kk)e^{\ii(\kk\vdot\rr - \omega(\kk)t)})                \\
              & = a_1\ee_1(\kk)\cos(\kk\vdot\rr - \omega(\kk)t + \varepsilon_1) + a_2\ee_2(\kk)\cos(\kk\vdot\rr - \omega(\kk)t + \varepsilon_2) \label{E def}
\end{align}
となる. これより, $\EE(\rr, t)$ の $\ee_1, \ee_2$ 方向の成分を $E_1(\rr, t), E_2(\rr, t)$ と書くとすると, $\ee_\pm$ 方向に進む場合について
\begin{align}
  \mqty[E_1 \\ E_2] = \mqty[a_1\cos(\kk\vdot\rr - \omega(\kk)t + \varepsilon_1) \\ a_2\cos(\kk\vdot\rr - \omega(\kk)t + \varepsilon_2)] \label{genral E vector}
\end{align}
となる.

\textbf{Q 21B-7.} $\varepsilon = \varepsilon_2 - \varepsilon_1$ とおくと式 \eqref{genral E vector} から Lissajous 図形は次のようになり, それが楕円の方程式であることを示せ.
\begin{align}
  \qty{(E_1(\rr, t), E_2(\rr, t)):t\in\RR} = \qty{(E_1, E_2)\in\RR^2:\qty(\frac{E_1}{a_1})^2 + \qty(\frac{E_2}{a_2})^2 - 2\cos\varepsilon\frac{E_1}{a_1}\frac{E_2}{a_2} = \sin^2{\varepsilon}}
\end{align}
\begin{proof}
  式の簡略化の為に $A_1 = \kk\vdot\rr - \omega(\kk)t + \varepsilon_1, A_2 = \kk\vdot\rr - \omega(\kk)t + \varepsilon_2$ とおくと
  \begin{align}
    \sin^2\varepsilon & = \sin^2(A_2 - A_1)                                                                                   \\
                      & = \qty(\sin{A_2}\cos{A_1} - \cos{A_2}\sin{A_1})^2                                                     \\
                      & = \sin^2{A_2}\cos^2{A_1} + \sin^2{A_1}\cos^2{A_2} - 2\cos{A_1}\cos{A_2}\sin{A_1}\sin{A_2}             \\
                      & = (1 - \cos^2{A_2})\cos^2{A_1} + (1 - \cos^2{A_1})\cos^2{A_2} - 2\cos{A_1}\cos{A_2}\sin{A_1}\sin{A_2} \\
                      & = \cos^2{A_1} + \cos^2{A_2} - 2\qty(\cos{A_1}\cos{A_2} - \sin{A_1}\sin{A_2})\cos{A_1}\cos{A_2}        \\
                      & = \cos^2{A_1} + \cos^2{A_2} - 2\cos\varepsilon\cos{A_1}\cos{A_2}
  \end{align}
  ここで式 \eqref{genral E vector} より
  \begin{align}
    \sin^2\varepsilon & = \cos^2{A_1} + \cos^2{A_2} - 2\cos\varepsilon\cos{A_1}\cos{A_2}                                     \\
                      & = \qty(\frac{E_1}{a_1})^2 + \qty(\frac{E_2}{a_2})^2 - 2\cos\varepsilon\frac{E_1}{a_1}\frac{E_2}{a_2}
  \end{align}
  を得られる. よって時間 $t$ を動かした軌跡の任意の点 $(E_1, E_2)$ を考えることで
  \begin{align}
    \qty{(E_1(\rr, t), E_2(\rr, t)):t\in\RR} = \qty{(E_1, E_2)\in\RR^2:\qty(\frac{E_1}{a_1})^2 + \qty(\frac{E_2}{a_2})^2 - 2\cos\varepsilon\frac{E_1}{a_1}\frac{E_2}{a_2} = \sin^2{\varepsilon}}
  \end{align}
  となる. そしてこれは式の形から二次曲線の中で回転移動させた楕円の方程式である.
\end{proof}

\textbf{Q 21B-8.} 式 \eqref{genral E vector} より
\begin{align}
  \mqty[E_1 \\ E_2] = \mqty[a_1\cos(\kk\vdot\rr - \omega(\kk)t + \varepsilon_1) \\ a_2\cos(\kk\vdot\rr - \omega(\kk)t + \varepsilon_1 + \varepsilon)] \label{E12 phase}
\end{align}
となるから「$E_2$ は $E_1$ より $\varepsilon$ だけ位相が遅れている。」または「$E_1$ は $E_2$ より $\varepsilon$ だけ位相が進んでいる。」といえる. \\

\textbf{Q 21B-9.}
\begin{align}
  \qty(\frac{E_1}{a_1})^2 + \qty(\frac{E_2}{a_2})^2 - 2\cos\varepsilon\frac{E_1}{a_1}\frac{E_2}{a_2} & = \sin^2{\varepsilon}
\end{align}
を二次形式を用いて表すと
\begin{align}
  \mqty[E_1 & E_2]\mqty[\frac{1}{a_1^2} & -\frac{\cos\varepsilon}{a_1a_2} \\ -\frac{\cos\varepsilon}{a_1a_2} & \frac{1}{a_2^2}]\mqty[E_1 \\ E_2] & = \sin^2\varepsilon
\end{align}
となる. ここで対称行列において異なる固有ベクトルは直交するから直交行列で対角化できるから, 角度 $\psi$ を上手く選ぶと, 座標系 $(E_1, E_2)$ から角度 $\psi$ だけ回転させた座標系 $(\xi, \eta)$ では二次形式を標準化できる.

座標系 $(\xi, \eta)$ は回転行列を用いて次のように表される.
\begin{align}
  \mqty[E_1 \\ E_2] = \mqty[\cos\psi & -\sin\psi \\ \sin\psi & \cos\psi]\mqty[\xi \\ \eta] \label{E12 xi eta}
\end{align}
これより座標系 $(\xi, \eta)$ での二次形式は次のようになる.
\begin{align}
  \sin^2\varepsilon & = \mqty[E_1 & E_2]\mqty[\frac{1}{a_1^2}                                                                                                                                                 & -\frac{\cos\varepsilon}{a_1a_2}                                                                            \\ -\frac{\cos\varepsilon}{a_1a_2} & \frac{1}{a_2^2}]\mqty[E_1 \\ E_2] \\
                    & = \mqty[\xi & \eta]\mqty[\cos\psi                                                                                                                                                       & \sin\psi                                                                                                   \\ -\sin\psi & \cos\psi]\mqty[\frac{1}{a_1^2} & -\frac{\cos\varepsilon}{a_1a_2} \\ -\frac{\cos\varepsilon}{a_1a_2} & \frac{1}{a_2^2}]\mqty[\cos\psi & -\sin\psi \\ \sin\psi & \cos\psi]\mqty[\xi \\ \eta] \\
                    & = \mqty[\xi & \eta]\mqty[\frac{1}{2}\qty(\frac{1}{a_1^2} + \frac{1}{a_2^2}) + \frac{1}{2}\qty(\frac{1}{a_1^2} - \frac{1}{a_2^2})\cos{2\psi} - \frac{\cos\varepsilon}{a_1a_2}\sin{2\psi} & -\frac{1}{2}\qty(\frac{1}{a_1^2} - \frac{1}{a_2^2})\sin{2\psi} - \frac{\cos\varepsilon}{a_1a_2}\cos{2\psi} \\ -\frac{1}{2}\qty(\frac{1}{a_1^2} - \frac{1}{a_2^2})\sin{2\psi} - \frac{\cos\varepsilon}{a_1a_2}\cos{2\psi} & \frac{1}{2}\qty(\frac{1}{a_1^2} + \frac{1}{a_2^2}) - \frac{1}{2}\qty(\frac{1}{a_1^2} - \frac{1}{a_2^2})\cos{2\psi} - \frac{\cos\varepsilon}{a_1a_2}\sin{2\psi}]\mqty[\xi \\ \eta] \label{xi eta}
\end{align}
これより二次形式を標準化するような角度 $\psi$ の条件は
\begin{align}
       & -\frac{1}{2}\qty(\frac{1}{a_1^2} - \frac{1}{a_2^2})\sin{2\psi} - \frac{\cos\varepsilon}{a_1a_2}\cos{2\psi} = 0 \\
  \iff & \tan{2\psi} = \frac{2a_1a_2}{a_1^2 - a_2^2}\cos\varepsilon \label{psi condition}
\end{align}
である. 式変形すると
\begin{align}
  \tan^2{2\psi} = \frac{1}{\cos^2{2\psi}} - 1 & = \frac{4a_1^2a_2^2           }{(a_1^2 - a_2^2)^2}\cos^2\varepsilon \\
  (a_1^2 - a_2^2)^2\sec{2\psi}                & = \qty{(a_1^2 - a_2^2)^2 + 4a_1^2a_2^2\cos^2\varepsilon}\cos{2\psi}
\end{align}
となる. このとき, $\xi$ と $\eta$ に対する方程式は
\begin{align}
  \qty(\frac{\xi}{a_\xi})^2 + \qty(\frac{\eta}{a_\eta})^2 = 1 \label{eclipse}
\end{align}
と書ける. ここで用いられる係数 $a_\xi, a_\eta$ については二次形式 \eqref{xi eta} から次のような条件を満たす.
\begin{align}
  \frac{1}{a_\xi^2}  & = \frac{1}{\sin^2\varepsilon}\qty(\frac{1}{2}\qty(\frac{1}{a_1^2} + \frac{1}{a_2^2}) + \frac{1}{2}\qty(\frac{1}{a_1^2} - \frac{1}{a_2^2})\cos{2\psi} - \frac{\cos\varepsilon}{a_1a_2}\sin{2\psi})                                        \\
                     & = \frac{1}{\sin^2\varepsilon}\qty(\frac{1}{2}\qty(\frac{1}{a_1^2} + \frac{1}{a_2^2}) + \qty(\frac{1}{2}\qty(\frac{1}{a_1^2} - \frac{1}{a_2^2}) - \frac{\cos\varepsilon}{a_1a_2}\tan{2\psi})\cos{2\psi})                                  \\
                     & = \frac{1}{\sin^2\varepsilon}\qty(\frac{1}{2}\qty(\frac{1}{a_1^2} + \frac{1}{a_2^2}) + \qty(\frac{1}{2}\qty(\frac{1}{a_1^2} - \frac{1}{a_2^2}) - \frac{\cos\varepsilon}{a_1a_2}\frac{2a_1a_2}{a_1^2 - a_2^2}\cos\varepsilon)\cos{2\psi}) \\
                     & = \frac{(a_1^2 + a_2^2)(a_1^2 - a_2^2) - \qty{(a_1^2 - a_2^2)^2 + 4a_1^2a_2^2\cos^2\varepsilon}\cos{2\psi}}{2a_1^2a_2^2(a_1^2 - a_2^2)\sin^2\varepsilon}                                                                                 \\
                     & = \frac{a_1^2 + a_2^2 - (a_1^2 - a_2^2)\sec{2\psi}}{2a_1^2a_2^2\sin^2\varepsilon}                                                                                                                                                        \\
  \frac{1}{a_\eta^2} & = \frac{1}{\sin^2\varepsilon}\qty(\frac{1}{2}\qty(\frac{1}{a_1^2} + \frac{1}{a_2^2}) - \frac{1}{2}\qty(\frac{1}{a_1^2} - \frac{1}{a_2^2})\cos{2\psi} - \frac{\cos\varepsilon}{a_1a_2}\sin{2\psi})                                        \\
                     & = \frac{a_1^2 + a_2^2 + (a_1^2 - a_2^2)\sec{2\psi}}{2a_1^2a_2^2\sin^2\varepsilon}
\end{align}
これらを整理して $a_\xi, a_\eta$ の具体的でシンプルな表式を求める.
\begin{align}
  \frac{1}{a_\xi^2} + \frac{1}{a_\eta^2}     & = \frac{a_\xi^2 + a_\eta^2}{a_\xi^2a_\eta^2} = \frac{a_1^2 + a_2^2}{a_1^2a_2^2\sin^2\varepsilon}                                        \\
  \frac{1}{a_\xi^2} - \frac{1}{a_\eta^2}     & = -\frac{a_\xi^2 - a_\eta^2}{a_\xi^2a_\eta^2} = -\frac{(a_1^2 - a_2^2)\sec{2\psi}}{a_1^2a_2^2\sin^2\varepsilon}                         \\
  \frac{1}{a_\xi^2} \cdot \frac{1}{a_\eta^2} & = \frac{1}{a_\xi^2a_\eta^2} = \frac{(a_1^2 + a_2^2)^2 - (a_1^2 - a_2^2)^2\sec^2{2\psi}}{4a_1^4a_2^4\sin^4\varepsilon}                   \\
                                             & = \frac{(a_1^2 + a_2^2)^2 - \qty{(a_1^2 - a_2^2)^2 + 4a_1^2a_2^2\cos^2\varepsilon}\cos{2\psi}\sec{2\psi}}{4a_1^4a_2^4\sin^4\varepsilon} \\
                                             & = \frac{1}{a_1^2a_2^2\sin^2\varepsilon}                                                                                                 \\
  a_\xi^2a_\eta^2                            & = a_1^2a_2^2\sin^2\varepsilon \label{axiaeta a1a2}                                                                                      \\
  a_\xi^2 + a_\eta^2                         & = \qty(\frac{1}{a_\xi^2} + \frac{1}{a_\eta^2})a_\xi^2a_\eta^2 = a_1^2 + a_2^2 \label{a^2}                                               \\
  a_\xi^2 - a_\eta^2                         & = - \qty(\frac{1}{a_\xi^2} - \frac{1}{a_\eta^2})a_\xi^2a_\eta^2 = (a_1^2 - a_2^2)\sec{2\psi}                                            \\
                                             & = (a_1^2 - a_2^2)\frac{1}{\cos{2\psi}} \label{axi - aeta}                                                                               \\
                                             & = (a_1^2 - a_2^2)\frac{\cos^2{2\psi} + \sin^2{2\psi}}{\cos{2\psi}}                                                                      \\
                                             & = (a_1^2 - a_2^2)(\cos{2\psi} + \sin{2\psi}\tan{2\psi})                                                                                 \\
                                             & = (a_1^2 - a_2^2)\cos{2\psi} + 2a_1a_2\cos\varepsilon\sin{2\psi}                                                                        \\
  a_\xi^2                                    & = \frac{1}{2}\qty((a_\xi^2 + a_\eta^2) + (a_\xi^2 - a_\eta^2)) = \frac{1}{2}(1 + \sec{2\psi})a_1^2 + \frac{1}{2}(1 - \sec{2\psi})a_2^2  \\
                                             & = a_1^2\cos^2\psi + a_2^2\sin^2\psi + 2a_1a_2\cos\varepsilon\cos\psi\sin\psi                                                            \\
  a_\eta^2                                   & = \frac{1}{2}\qty((a_\xi^2 + a_\eta^2) - (a_\xi^2 - a_\eta^2)) = \frac{1}{2}(1 - \sec{2\psi})a_1^2 + \frac{1}{2}(1 + \sec{2\psi})a_2^2  \\
                                             & = a_1^2\cos^2\psi + a_2^2\sin^2\psi - 2a_1a_2\cos\varepsilon\cos\psi\sin\psi
\end{align}
このようにして
\begin{align}
  a_\xi^2  & = a_1^2\cos^2\psi + a_2^2\sin^2\psi + 2a_1a_2\cos\varepsilon\cos\psi\sin\psi \\
  a_\eta^2 & = a_1^2\cos^2\psi + a_2^2\sin^2\psi - 2a_1a_2\cos\varepsilon\cos\psi\sin\psi
\end{align}
が求まる. \\

\textbf{Q 21B-10.}
条件式 \eqref{psi condition} において $\tan{2\psi} = \tan{2\qty(\psi + \frac{\pi}{2})}$ より位相を $\pi/2$ ずらしたものも解となる. これより $\psi$ が解ならば $\psi, \psi + \pi/2, \psi + \pi, \psi + 3\pi/2$ は解となる. これは楕円を $\psi$ だけ回転させたときに $\psi, \psi + \pi$ と $\psi + \pi/2, \psi + 3\pi/2$ は主軸の方向は同じであり, $\psi + \pi/2, \psi + 3\pi/2$ は $\psi$ の軸方向の状態と比べ, 長軸と短軸が反対となった状態となる. これより $\psi\in[0,\pi)$ かつ長軸, 短軸は $\xi, \eta$ 軸があてがわれると制限すると $\psi$ は一意に決まる. \\

\textbf{Q 21B-11.}
楕円偏光の回転の向きは $E_1, E_2$ の位相差 $\varepsilon$ の正弦 $\sin\varepsilon$ の正負が関係していること, つまり次のような関係となることを示せ.
\begin{align}
  \sin\varepsilon > 0 & \iff \textrm{「楕円偏光は左偏光である。」} \\
  \sin\varepsilon < 0 & \iff \textrm{「楕円偏光は右偏光である。」}
\end{align}
\begin{proof}
  まず, $\xi,\eta$ は式 \eqref{E12 xi eta} より $E_1, E_2$ を用いて次のように表される.
  \begin{align}
    \mqty[\xi \\ \eta] = \mqty[\cos\psi & \sin\psi \\ -\sin\psi & \cos\psi]\mqty[E_1 \\ E_2] \label{xi eta E12}
  \end{align}
  また, 式 \eqref{E12 phase} より $E_1, E_2$ は $p = \kk\vdot\rr - \omega(\kk)t + \varepsilon_1$ を用いて次のように表される.
  \begin{align}
    E_1 & = a_1\cos{p}            \label{E1 p}    \\
    E_2 & = a_2\cos(p + \varepsilon) \label{E2 p}
  \end{align}
  これより $\xi, \eta$ の表式 \eqref{xi eta E12} は三角関数の合成公式より $E_1, E_2$ の表式 \eqref{E1 p} \eqref{E2 p} を用いて次のように表される.
  \begin{align}
    \xi  & = E_1\cos\psi + E_2\sin\psi                                                                  \\
         & = a_1\cos\psi\cos{p} + a_2\sin\psi\cos(p + \varepsilon)                                      \\
         & = a_1\cos\psi\cos{p} + a_2\sin\psi(\cos{p}\cos\varepsilon - \sin{p}\sin\varepsilon)          \\
         & = (\ \  a_1\cos\psi + a_2\cos\varepsilon\sin\psi)\cos{p} - a_2\sin\varepsilon\sin\psi\sin{p} \\
         & = A_\xi\cos(p + \theta_\xi) \label{xi}                                                       \\
    \eta & = -E_1\sin\psi + E_2\cos\psi                                                                 \\
         & = (-a_1\sin\psi + a_2\cos\varepsilon\cos\psi)\cos{p} - a_2\sin\varepsilon\cos\psi\sin{p}     \\
         & = A_\eta\cos(p + \theta_\eta) \label{eta}
  \end{align}
  ただし, 振幅因子 $A_\xi, A_\eta$ は次のように与える.
  \begin{align}
    A_\xi  & = \sqrt{\qty(\ \ a_1\cos\psi + a_2\cos\varepsilon\sin\psi)^2 + (a_2\sin\varepsilon\sin\psi)^2} \\
    A_\eta & = \sqrt{\qty(-a_1\sin\psi + a_2\cos\varepsilon\cos\psi)^2 + (a_2\sin\varepsilon\cos\psi)^2}
  \end{align}
  また, 位相のずれ $\theta_\xi, \theta_\eta$ は次のように定められる.
  \begin{align}
    \cos\theta_\xi  & = \frac{\ \ a_1\cos\psi + a_2\cos\varepsilon\sin\psi}{A_\xi}, \quad \sin\theta_\xi = \frac{a_2\sin\varepsilon\sin\psi}{A_\xi}  \\
    \cos\theta_\eta & = \frac{-a_1\sin\psi + a_2\cos\varepsilon\cos\psi}{A_\xi},    \quad \sin\theta_\eta = \frac{a_2\sin\varepsilon\cos\psi}{A_\xi}
  \end{align}
  ここで 振幅因子 $A_\xi, A_\eta$ は式変形を行うことで次のように $a_\xi, a_\eta$ と一致する.
  \begin{align}
    A_\xi^2  & = \qty(\ \ a_1\cos\psi + a_2\cos\varepsilon\sin\psi)^2 + (a_2\sin\varepsilon\sin\psi)^2 \\
             & = a_1^2\cos^2\psi + a_2^2\sin^2\psi + 2a_1a_2\cos\varepsilon\cos\psi\sin\psi            \\
             & = a_\xi^2                                                                               \\
    A_\eta^2 & = \qty(-a_1\sin\psi + a_2\cos\varepsilon\cos\psi)^2 + (a_2\sin\varepsilon\cos\psi)^2    \\
             & = a_\eta^2
  \end{align}
  これは, 楕円の方程式 \eqref{eclipse} から $\xi, \eta$ の振幅は $a_\xi, a_\eta$ であることが分かるので自明である.

  次に位相差 $\theta_\eta - \theta_\xi$の表式を求める.
  \begin{align}
    \frac{A_\eta}{A_\xi}e^{\ii(\theta_\eta - \theta_\xi)}
     & = \frac{A_\eta e^{\ii\theta_\eta}}{A_\xi e^{\ii\theta_\xi}}                                                                                                                                                                                                 \\
     & = \frac{(-a_1\sin\psi + a_2\cos\varepsilon\cos\psi) + \ii a_2\sin\varepsilon\cos\psi}{(\quad a_1\cos\psi + a_2\cos\varepsilon\sin\psi) + \ii a_2\sin\varepsilon\sin\psi}                                                                                    \\
     & = \frac{\qty((-a_1\sin\psi + a_2\cos\varepsilon\cos\psi) + \ii a_2\sin\varepsilon\cos\psi)\qty((a_1\cos\psi + a_2\cos\varepsilon\sin\psi) - \ii a_2\sin\varepsilon\sin\psi)}{(a_1\cos\psi + a_2\cos\varepsilon\sin\psi)^2 + (a_2\sin\varepsilon\sin\psi)^2} \\
     & = \frac{\ii a_1a_2\sin\varepsilon}{(a_1\cos\psi + a_2\cos\varepsilon\sin\psi)^2 + (a_2\sin\varepsilon\sin\psi)^2}
  \end{align}
  これより, 絶対値と位相を考えると, 次のようになる.
  \begin{align}
    \frac{A_\eta}{A_\xi}              & = \frac{a_1a_2\lvert\sin\varepsilon\rvert}{(a_1\cos\psi + a_2\cos\varepsilon\sin\psi)^2 + (a_2\sin\varepsilon\sin\psi)^2} \\
    e^{\ii(\theta_\eta - \theta_\xi)} & = \mathrm{sgn}(\sin\varepsilon)\ii
  \end{align}
  すると, 位相差 $\theta_\eta - \theta_\xi$ について
  \begin{align}
    \theta_\eta - \theta_\xi & =
    \begin{dcases}
      \frac{\pi}{2}  & (\sin\varepsilon > 0) \\
      \frac{3\pi}{2} & (\sin\varepsilon < 0)
    \end{dcases}
  \end{align}
  となるので $\xi, \eta$ の表式 \eqref{xi} \eqref{eta} は次のようになる.
  \begin{align}
    \xi  & = A_\xi\cos(p + \theta_\xi)                                          \\
    \eta & = A_\eta\cos(p + \theta_\eta)                                        \\
         & = \mp A_\eta\sin(p + \theta_\xi) \qquad (\sin\varepsilon \gtrless 0)
  \end{align}
  よって, 次の式が確立される.
  \begin{align}
    \sin\varepsilon > 0 & \iff \textrm{「楕円偏光は左偏光である。」} \\
    \sin\varepsilon < 0 & \iff \textrm{「楕円偏光は右偏光である。」}
  \end{align}
\end{proof}

\textbf{Q 21B-12.}
光学では伝統的に楕円の形と偏光の回転の向きを記述するパラメータ $\chi$ が用いられている. この問題では $\chi$ について調べる.

まずパラメータ $\chi\in\RR$ は短軸 $a_\eta$, 長軸 $a_\xi$ を用いて次のように定義される.
\begin{align}
  \tan{|\chi|} := \frac{a_\eta}{a_\xi}
\end{align}
楕円の軸の長さを一意に決定させるには $|\chi|$ の範囲は $0 \leq |\chi| \leq \frac{\pi}{4}$ である必要がある. このとき, 式 \eqref{axiaeta a1a2}, \eqref{a^2} と $a_\xi, a_\eta, a_1, a_2 > 0$ より
\begin{align}
  \sin{2|\chi|} & = 2\sin|\chi|\cos|\chi| = 2\tan|\chi|\cos^2|\chi|           \\
                & = \frac{2\tan|\chi|}{1 + \tan^2|\chi|}                      \\
                & = \frac{2a_\xi a_\eta}{a_\xi^2 + a_\eta^2}                  \\
                & = \frac{2a_1 a_2}{a_1^2 + a_2^2}\lvert\sin\varepsilon\rvert
\end{align}
となる. この式を用いて $\chi$ を次のように再定義する.
\begin{align}
  \sin{2\chi} & := -\frac{2a_1 a_2}{a_1^2 + a_2^2}\sin\varepsilon \label{chi def}
\end{align}
これは $\chi$ の範囲が $-\frac{\pi}{4} \leq \chi \leq \frac{\pi}{4}$ を満たすとき, 元の定義の拡張となっている.
\begin{align}
  \sin2\chi = \frac{2\tan\chi}{1 + \tan^2\chi}
  =        & -\frac{2a_1 a_2}{a_1^2 + a_2^2}\sin\varepsilon                                                                                                          \\
  =        & \pm\frac{2a_1 a_2}{a_1^2 + a_2^2}\lvert\sin\varepsilon\rvert                                                             & (\sin\varepsilon \lessgtr 0) \\
  =        & \pm\frac{2a_\xi a_\eta}{a_\xi^2 + a_\eta^2} = \frac{2\qty(\pm\frac{a_\eta}{a_\xi})}{1 + \qty(\pm\frac{a_\eta}{a_\xi})^2}                                \\
  \iff     & \tan\chi = \pm\frac{a_\eta}{a_\xi}                                                                                       & (\sin\varepsilon \lessgtr 0) \\
  \implies & \tan|\chi| = \frac{a_\eta}{a_\xi}
\end{align}
また, 次の対応が分かる.
\begin{align}
  \chi \gtrless 0 \iff \tan\chi \gtrless 0 \iff \sin\varepsilon \lessgtr 0 \iff
  \begin{cases}
    \textrm{「楕円偏光は右偏光である。」} \\
    \textrm{「楕円偏光は左偏光である。」}
  \end{cases}
\end{align}
これより $-\frac{\pi}{4} \leq \chi \leq \frac{\pi}{4}$ という範囲のとき, 楕円偏光の形と偏光の回転の向きを一意に決定できる. \\

\textbf{Q 21B-13.}
この問題では楕円偏光の中でも円偏光はどのような条件の下で生じるのかを考える.
楕円偏光の方程式 \eqref{eclipse} は次のようであった.
\begin{align}
  \qty(\frac{\xi}{a_\xi})^2 + \qty(\frac{\eta}{a_\eta})^2 = 1
\end{align}
これより円偏光になる条件は $a_\xi = a_\eta$ のときである. $a_1, a_2$ の楕円偏光の方程式より
\begin{align}
  \textrm{「楕円偏光は円偏光である。」}
   & \iff a_\xi = a_\eta \iff \tan\chi = \pm 1 \iff \chi = \pm \frac{\pi}{4}     \\
   & \iff a_1 = a_2 \land -\frac{2a_1 a_2}{a_1^2 + a_2^2}\sin\varepsilon = \pm 1 \\
   & \iff a_1 = a_2 \land \varepsilon\in\pi\qty(\ZZ + \frac{1}{2})
\end{align}\\

\textbf{Q 21B-14.}
同様に直線偏光はどうのような条件の下で生じるのかを考える. 楕円偏光の方程式 \eqref{eclipse} と $\chi$ の定義より直線偏光となる条件は $a_\eta = 0$ となる.
\begin{align}
  \textrm{「楕円偏光は直線偏光である。」}
   & \iff a_\eta = 0 \iff \tan\chi = 0 \iff \chi = 0 \\
   & \iff a_1a_2 = 0 \lor \varepsilon\in\pi\ZZ
\end{align}
このとき振動面の角度 $\psi$ は $\varepsilon = n\pi \quad (n\in\ZZ)$ とおくと式 \eqref{psi condition} より
\begin{align}
  \tan2\psi & = \frac{2\tan\psi}{1 + \tan^2\psi} = \frac{2a_1a_2}{a_1^2 - a_2^2}\cos\varepsilon = \frac{2\qty(\pm\frac{a_2}{a_1})}{1 + \qty(\pm\frac{a_2}{a_1})^2} & \qty(nの偶奇) \\
  \tan\psi  & = \pm\frac{a_2}{a_1}                                                                                                                                 & \qty(nの偶奇) \\
  \psi      & =
  \begin{cases}
    \alpha       & (n: 偶数) \\
    \pi - \alpha & (n: 奇数)
  \end{cases}
\end{align}
となる. \\

\textbf{Q 21B-15.}
式 \eqref{axi - aeta} は次のようであった.
\begin{align}
  a_\xi^2 - a_\eta^2 & = \frac{a_1^2 - a_2^2}{\cos2\psi}
\end{align}
$a_\xi, a_\eta$ はそれぞれ長軸, 短軸の長さであるから左辺は正である. よって次のようになる.
\begin{align}
  \mathrm{sgn}(a_1 - a_2) = \mathrm{sgn}(\cos2\psi) \label{sgn psi}
\end{align} \\

\textbf{Q 21B-16.}
式 \eqref{psi condition}, \eqref{chi def}, \eqref{sgn psi}, $-\frac{\pi}{4} \leq \chi \leq \frac{\pi}{4}$ より次の事がわかる.
\begin{align}
  \cos^22\chi & = 1 - \sin^22\chi = 1 - \qty(\frac{2a_1a_2}{a_1^2 + a_2^2}\sin\varepsilon)^2                                  \\
              & = \qty(\frac{a_1^2 - a_2^2}{a_1^2 + a_2^2})^2\qty(1 - \qty(\frac{2a_1a_2}{a_1^2 - a_2^2}\cos^2\varepsilon)^2) \\
              & = \qty(\frac{a_1^2 - a_2^2}{a_1^2 + a_2^2})^2\qty(1 - \tan^22\psi)                                            \\
              & = \qty(\frac{a_1^2 - a_2^2}{a_1^2 + a_2^2}\frac{1}{\cos2\psi})^2                                              \\
  \cos2\chi   & = \frac{a_1^2 - a_2^2}{a_1^2 + a_2^2}\frac{1}{\cos2\psi} \geq 0 \label{cos 2chi}
\end{align} \\

\section{位相のずれの規約: $\varepsilon$ と $\delta$}
$E_1, E_2$ を次の表式で表すことがある.
\begin{align}
  \mqty[E_1                                                 \\ E_2] &= \mqty[a_1\cos(\kk\vdot\rr - \omega(\kk)t + \varepsilon_1) \\ a_2\cos(\kk\vdot\rr - \omega(\kk)t + \varepsilon_2)] \\
   & = \mqty[a_1\cos(\omega(\kk)t - \kk\vdot\rr + \delta_1) \\ a_2\cos(\omega(\kk)t - \kk\vdot\rr + \delta_2)] \\
   & = \mqty[a_1\cos q                                      \\ a_2\cos(q + \delta)] \label{E12 delta}
\end{align}
ただし $\delta_i = - \varepsilon_i (i = 1, 2), q = \omega(\kk)t - \kk\vdot\rr + \delta_1, \delta = \delta_2 - \delta_1$ とおく. すると $\delta = -\varepsilon$ という関係が成り立つ. \\

\textbf{Q 21B-17.}
式 \eqref{E12 delta} より場所 $\rr$ に留まり, 時間 $t$ の経過とともに, 電場ベクトルの波動を観測する立場から見て, 位相差 $\delta$ について「$E_2$ は $E_1$ より $\delta$ だけ位相が進んでいる。」つまり「$E_1$ は $E_2$ より $\delta$ だけ位相が遅れている。」と言える.

同様に楕円偏光の向きについても次のようになる.
\begin{align}
  \sin\delta < 0 \iff \sin\varepsilon > 0 \iff \textrm{「楕円偏光は左偏光である。」} \\
  \sin\delta > 0 \iff \sin\varepsilon < 0 \iff \textrm{「楕円偏光は右偏光である。」}
\end{align}

\section{Stokes パラメータ}
\textbf{Q 21B-18.}
波数 $\kk\in\RR^3$ を持つ一般の単色波の電場は \eqref{E def} 式より4つのパラメータ $a_1, a_2 \geq 0; \varepsilon = -\delta_1, \varepsilon = -\delta\in\RR$ によって記述される. これらは関係式 $\tilde{E}_i = a_ie^{\ii \varepsilon_i}$ によって結びついている. 一般の単色波の電場の状態を記述する 4 個のパラメータは, 上手く用意すれば, 次のように異なる役割を持つ 3 個のグループに分けられる.

\begin{enumerate}
  \item 時間の原点を指定する実パラメータ 1 個.場所 $\rr\in\RR^3$ に留まって観測するとします. 特定の時刻 $t\in\RR$ において, 電場の 1 成分と 2 成分で指定される点 $(E_1, E_2)$ が Lissajous 図形である偏光楕円上のどこにあるのかを, この実パラメータが指定します.このパラメータは $\varepsilon_1$ あるいは $\delta_1$ に取ることができます.
  \item 電場のスケールを指定する実パラメータ 1 個.このパラメータを大きくすることは, $(E_1, E_2)$ 面上の Lissajous 図形である偏光楕円の傾きと形を保って, 楕円を相似に大きくすることに対応します.このパラメータは電場の強度 $|E|^2 = a_1^2 + a_2^2$ に取ることができます.これから見るよう後者の方が便利です.
  \item 電場の偏光状態を指定する実パラメータ 2 個.電場の偏光状態は $(E_1, E_2)$ 面上の Lissajous 図形である偏光楕円の傾きと形, 加えて, 周回の向きにより記述されます.これを記述するパラメータは, 偏光楕円の傾きを指定する角度 $\psi$ と, 偏光楕円の形と周回の向きを絶対値と符号で指定する角度 $\chi$ によって用意できます.
\end{enumerate}

これらは互いに独立であることが明白であるから十分性は成り立つ. 必要性に関しては電場の定義となる実パラメータが4つと等しい数であることから成り立つ. よってこれら4つのパラメータで電場を表現できる.

\textbf{Q 21B-19.}
偏光状態のパラメータ $(\psi, \chi)$ について次の事柄がわかっている.
\begin{enumerate}
  \item パラメータ $\psi$ は $0\leq\psi<\pi$ の範囲の値を取る. $\psi$ は $(E_1, E_2)$ 面において $E_1$ 軸から計った偏光楕円の長軸の角度である. 偏光楕円が円に縮退していない場合は, パラメータ $\psi$ の値は一意に定まる。偏光楕円が円に縮退している場合は, パラメータ $\psi$ の値は定まらない(どの値でもかまわない).
  \item  パラメータ $\chi$ は $-\frac{\pi}{4}\leq\chi\leq\frac{\pi}{4}$の範囲の値を取る. 偏光楕円の短半径 $a_\eta$ と長半径 $a_\xi$ を用いて $\tan|\chi| = a_\eta/a_\xi$ と書ける. $\chi$ が正ならば右偏光, 負ならば左偏光である. 特に $\chi = \pm \pi/4$ が円偏光, $\chi = 0$ が直線偏光である.
\end{enumerate}
$\psi = \pi$ のとき $\psi = 0$ と比べて, 軸の正の向きは逆であるが主軸の方向は同じなので偏光楕円の軌跡は等しく, 同一視できる. また, $\chi = \pm\frac{\pi}{4}$ のとき, 偏光楕円が円に縮退している為, 楕円偏光の長軸の角度を変えても楕円偏光の軌跡は等しく, 同一視できる. これより偏光楕円において同一視できる関係を $\sim$ とおくと, 次のように書ける.
\begin{align}
  (\psi = 0, \chi)              & \sim (\psi = \pi, \chi)             & (-\frac{\pi}{4}\leq\chi\leq\frac{\pi}{4}) \\
  (\psi, \chi = \frac{\pi}{4})  & \sim (\psi', \chi = \frac{\pi}{4})  & (0\leq\psi,\psi'\leq\pi)                  \\
  (\psi, \chi = -\frac{\pi}{4}) & \sim (\psi', \chi = -\frac{\pi}{4}) & (0\leq\psi,\psi'\leq\pi)
\end{align}
これより $\psi$ を球面の経度, $\chi$ を球面の緯度と捉えると $\sim$ による同値類は2次元球面 $S^2$ と同相になる. 例えば $\chi = \frac{\pi}{4}$ は北極, $\chi = 0$ は赤道, $\chi = -\frac{\pi}{4}$ は南極と対応する. また, $\psi = 0,\pi$ が Greenwich 子午線として同一視される.
\begin{align}
  \qty{(\psi, \chi):0\leq\psi\leq\pi \land -\frac{\pi}{4}\leq\chi\leq\frac{\pi}{4}}/\sim\ \cong S^2
\end{align}
さらに電場の強度 $|E|^2$ のパラメータ空間は非負実数空間 $\RR_{\geq0}$ であり, 電場の強度と偏光状態は独立である為, 電場の強度を半径と見なすことができる. よって強度と偏光のパラメータ空間は 3 次元実 Euclid 空間 $\RR^3$ と同相である.
\begin{align}
  \RR_{\geq 0}\times S^2 \cong \RR^3
\end{align}
このことから強度と偏光状態のパラメータを 3 次元空間 $\RR^3$ 上の点と対応させて考える. その点を極座標 $(s_0, \theta, \phi)$ で表すこととする. 上での対応させ方から次のように定義できる.
\begin{align}
  s_0    & = a_1^2 + a_2^2         \\
  \theta & = \frac{\pi}{2} - 2\chi \\
  \phi   & = 2\psi
\end{align}
これより右手系の直交座標 $(s_1, s_2, s_3)$ で表すと
\begin{align}
  s_1 & = s_0\cos2\psi\cos2\chi \\
  s_2 & = s_0\sin2\psi\cos2\chi \\
  s_3 & = s_0\sin2\chi
\end{align}
である. このようにして用意された電場の強度と偏光を記述する 4 つの実パラメータの組 $(s_0, s_1, s_2, s_3)$ は「Stokesパラメータ」と呼ばれる. Stokesパラメータはパラメータ $a_1, a_2, \psi, \chi$ を用いて次のように表される.
\begin{align}
  s_0 & = a_1^2 + a_2^2                        \\
  s_1 & = s_0\cos2\psi\cos2\chi \label{s1 def} \\
  s_2 & = s_0\sin2\psi\cos2\chi \label{s2 def} \\
  s_3 & = s_0\sin2\chi \label{s3 def}
\end{align}
このような状況を「完全偏光」と呼び, より一般的な「部分偏光」をこれから考える. また完全偏光において関係式 $s_0^2 = s_1^2 + s_2^2 + s_3^2$ を満たす. このように点 $(s_1, s_2, s_3)$ は原点を中心とする半径 $s_0$ の球面上にある. この球面を「Poincaré 球面」と呼ぶ. \\

\textbf{Q 21B-22.}
上の定義を用いて表される次のベクトルを「Stokesベクトル」と呼ぶ.
\begin{align}
  \vb{S} = \mqty[s_0 \\ s_1 \\ s_2 \\ s_3] = \mqty[s_0 & s_1 & s_2 & s_3]^\top
\end{align}
このとき具体的な例として以下のようなものがある.
\begin{table}[hbtp]
  \label{table:Stokes Jones}
  \centering
  \begin{tabular}{|c|c|c|c|}
    \hline
    偏光状態                   & 呼び名                                        & Stokesベクトル $\vb{S}$ & Jonesベクトル $\vb{J}$                                                             \\
    \hline \hline
    直線偏光(水平)               & 水平 $\mathcal{P}$ 状態: LHP                   & $\mqty[1            & 1                  & 0  & 0]^\top$  & $\mqty[1                   & 0]^\top$    \\
    直線偏光(垂直)               & 鉛直 $\mathcal{P}$ 状態: LNP                   & $\mqty[1            & -1                 & 0  & 0]^\top$  & $\mqty[0                   & 1]^\top$    \\
    直線偏光($+45$\textdegree) & +45\textdegree の $\mathcal{P}$ 状態: L+45P   & $\mqty[1            & 0                  & 1  & 0]^\top$  & $\frac{1}{\sqrt{2}}\mqty[1 & 1]^\top$    \\
    直線偏光($-45$\textdegree) & $-45$\textdegree の $\mathcal{P}$ 状態: L-45P & $\mqty[1            & 0                  & -1 & 0]^\top$  & $\frac{1}{\sqrt{2}}\mqty[1 & -1]^\top$   \\
    円偏光(右回転)               & $\mathcal{R}$ 状態: RCP                      & $\mqty[1            & 0                  & 0  & 1]^\top$  & $\frac{1}{\sqrt{2}}\mqty[1 & \ii]^\top$  \\
    円偏光(左回転)               & $\mathcal{L}$ 状態: LCP                      & $\mqty[1            & 0                  & 0  & -1]^\top$ & $\frac{1}{\sqrt{2}}\mqty[1 & -\ii]^\top$ \\
    \hline
  \end{tabular}
  \caption{重要な偏光状態の Stokes ベクトル}
\end{table} \\

\section{Stokesパラメータの応用}
スピンの量子状態 $\ket{\Psi}$ は2つの基底状態 $\ket{\uparrow}, \ket{\downarrow}$ の重ね合わせで与えられる.
\begin{align}
  \ket{\Psi} = c_1\ket{\uparrow} + c_2\ket{\downarrow} \quad (c_1, c_2 \in \CC)
\end{align}
ここで次の $S^1$ gauge 変換に対し観測可能量の期待値は変化しない.
\begin{align}
  \ket{\Psi} \mapsto \ket{\Psi'} = e^{\ii\delta}\ket{\Psi} \quad (\delta\in\RR)
\end{align}
規格化条件と $S^1$ gauge 変換による同一視を課すと 2 つの実数パラメータにより指定でき, これは $S^2$ と同相であるという事実が知られている. よってスピンの量子状態と光の偏光状態は $S^2$ と同相な物理量であることがわかる. 近年, これらは量子計算の基礎的な資源と見なされている.

\section{Stokes パラメータの測定}
Stokes パラメータ $s_0, s_1, s_2, s_3$ の最も重要な性質はこれらが直接的に観測可能な物理量であることである. どのように測定可能なのかを学習する.
\subsection{Stokes パラメータの $a_1, a_2$ と $\varepsilon$ (あるいは $\delta$) による表現}
直線偏光と円偏光について以下のように定義していた.
\begin{align}
  \tilde{\EE}(\rr, t) & = \tilde{E}_1\ee_1(\kk)e^{\ii(\kk\vdot\rr - \omega(\kk)t)} + \tilde{E}_2\ee_2(\kk)e^{\ii(\kk\vdot\rr - \omega(\kk)t)} \label{ET line def}   \\
                      & = \tilde{E}_+\ee_+(\kk)e^{\ii(\kk\vdot\rr - \omega(\kk)t)} + \tilde{E}_-\ee_-(\kk)e^{\ii(\kk\vdot\rr - \omega(\kk)t)} \label{ET circle def} \\
  \tilde{E}_i         & = a_ie^{\ii\varepsilon_i}, a_i \geq 0, \varepsilon_i\in\RR \quad (i = 1, 2, +, -)
\end{align}
このとき次の等式が成り立つ.
\begin{align}
  \tilde{E}_+ & = \frac{1}{\sqrt{2}}\qty(\tilde{E}_1 - \ii\tilde{E}_2) \\
  \tilde{E}_- & = \frac{1}{\sqrt{2}}\qty(\tilde{E}_1 + \ii\tilde{E}_2)
\end{align}
\\

\textbf{Q 21B-23.}
このとき円偏光のパラメータを直線偏光のパラメータを用いて表すことを考える.
\begin{align}
  a_+^2 + a_-^2
   & = |\tilde{E}_+|^2 + |\tilde{E}_+|^2                                                                                                      \\
   & = \frac{1}{2}\qty(|a_1e^{\ii\varepsilon_1} - \ii a_2e^{\ii\varepsilon_2}|^2 + |a_1e^{\ii\varepsilon_1} + \ii a_2e^{\ii\varepsilon_2}|^2) \\
   & = a_1^2 + a_2^2                                                                                                                          \\
  a_+^2 - a_-^2
   & = |\tilde{E}_+|^2 - |\tilde{E}_+|^2                                                                                                      \\
   & = \frac{1}{2}\qty(|a_1e^{\ii\varepsilon_1} - \ii a_2e^{\ii\varepsilon_2}|^2 - |a_1e^{\ii\varepsilon_1} + \ii a_2e^{\ii\varepsilon_2}|^2) \\
   & = 2a_1a_2\sin(\varepsilon_2 - \varepsilon_1)                                                                                             \\
  2a_+a_-\cos(\varepsilon_- - \varepsilon_+)
   & = \Re(2\tilde{E}_-\tilde{E}_+^*)                                                                                                         \\
   & = \Re((a_1^2 - a_2^2) + \ii(2a_1a_2\cos(\varepsilon_2 - \varepsilon_1)))                                                                 \\
   & = a_1^2 - a_2^2                                                                                                                          \\
  2a_+a_-\sin(\varepsilon_- - \varepsilon_+)
   & = \Im(2\tilde{E}_-\tilde{E}_+^*)                                                                                                         \\
   & = \Im((a_1^2 - a_2^2) + \ii(2a_1a_2\cos(\varepsilon_2 - \varepsilon_1)))                                                                 \\
   & = 2a_1a_2\cos(\varepsilon_2 - \varepsilon_1)
\end{align}

\textbf{Q 21B-24.}
Stokesパラメータを直線偏光のパラメータで表現することを考える. 式 \eqref{cos 2chi}, \eqref{chi def} より
\begin{align}
  s_0 & = a_1^2 + a_2^2                               \\
  s_1 & = s_0\cos2\psi\cos2\chi                       \\
      & = a_1^2 - a_2^2                               \\
  s_2 & = s_0\sin2\psi\cos2\chi                       \\
      & = s_0\cos2\psi\cos2\chi\cdot\tan2\psi         \\
      & = 2a_1a_2\cos\delta = 2a_1a_2\cos\varepsilon  \\
  s_3 & = s_0\sin2\chi                                \\
      & = 2a_1a_2\sin\delta = -2a_1a_2\sin\varepsilon
\end{align}

\subsection{電場の通常の規約による複素表示 $\tilde{E}$ を用いて}
\textbf{Q 21B-25, Q 21B-26.}
電場の複素表示 $\ET$ と一般の楕円偏光の基底ベクトル $\ee_i, \ee_j\in\CC^3\quad ((i, j) = (1, 2), (+, -))$ の直交性より
\begin{align}
  \ee_i(\kk)^*\vdot\ET
   & = \Et_ie^{\ii(\kk\vdot\rr - \omega(\kk)t)} = a_ie^{\ii(\kk\vdot\rr - \omega(\kk)t + \varepsilon_i)} \\
  \qty(\ee_i(\kk)^*\vdot\ET)^*\qty(\ee_j(\kk)^*\vdot\ET)
   & = \Et_i^*\Et_j = a_ia_je^{\ii(\varepsilon_j - \varepsilon_i)} \label{Ei Ej}
\end{align}
となる. 特に直線偏光の基底ベクトル $\ee_1, \ee_2\in\RR^3$ より, $\ee_i = \ee_i^*\quad(i=1,2)$ となる. \\

\textbf{Q 21B-27.}
式 \eqref{Ei Ej} よりStokesパラメータは直線偏光のパラメータを用いて
\begin{align}
  s_0 & = a_1^2 + a_2^2 = \qty|\ee_1(\kk)\vdot\ET|^2 + \qty|\ee_2(\kk)\vdot\ET|^2                                      \\
  s_1 & = a_1^2 - a_2^2 = \qty|\ee_1(\kk)\vdot\ET|^2 - \qty|\ee_2(\kk)\vdot\ET|^2                                      \\
  s_2 & = 2a_1a_2\cos(\varepsilon_2 - \varepsilon_1) = 2\Re\qty{\qty(\ee_1(\kk)\vdot\ET)^*\qty(\ee_2(\kk)\vdot\ET)}    \\
  s_3 & = - 2a_1a_2\sin(\varepsilon_2 - \varepsilon_1) = -2\Im\qty{\qty(\ee_1(\kk)\vdot\ET)^*\qty(\ee_2(\kk)\vdot\ET)}
\end{align}
と表される. \\

\textbf{Q 21B-28.}
さらに Q 21B-23 より円偏光のパラメータを用いて
\begin{align}
  s_0 & = a_+^2 + a_-^2 = \qty|\ee_+(\kk)^*\vdot\ET|^2 + \qty|\ee_-(\kk)^*\vdot\ET|^2                                   \\
  s_1 & = 2a_+a_-\cos(\varepsilon_- - \varepsilon_+) = 2\Re\qty{\qty(\ee_+(\kk)^*\vdot\ET)^*\qty(\ee_-(\kk)^*\vdot\ET)} \\
  s_2 & = 2a_+a_-\sin(\varepsilon_- - \varepsilon_+) = 2\Im\qty{\qty(\ee_+(\kk)^*\vdot\ET)^*\qty(\ee_-(\kk)^*\vdot\ET)} \\
  s_3 & = -\qty(a_+^2 - a_-^2) = -\qty{\qty|\ee_+(\kk)^*\vdot\ET|^2 - \qty|\ee_-(\kk)^*\vdot\ET|^2}
\end{align}
と表される. これらの $s_3$ にマイナスが含まれているところが気に入らないらしく, 光学で使われる電場の複素表示を用いることで対称性のよい形にできることを次の節で示す. \\

\subsection{電場の光学の規約による複素表示 $\EC$ を用いて}
電場の複素表示の複素共役 $\EC(\rr, t)$ を定義する.
\begin{align}
  \EC(\rr, t) = \ET(\rr, t)^*
\end{align}
これを「光学の流儀の複素表示」あるいは「新しい複素表示」と呼ぶこととする. \\

\textbf{Q 21B-29, Q 21B-30.}
式 \eqref{ET line def}, \eqref{ET circle def} に複素共役を取って新しい複素表示は次のように表される.
\begin{align}
  \EC(\rr, t) & = \Ec_1\ee_1(\kk)e^{\ii(\omega(\kk)t - \kk\vdot\rr)} + \Ec_2\ee_2(\kk)e^{\ii(\omega(\kk)t - \kk\vdot\rr)}      \\
  \Ec_i       & = \Et_i^* = a_ie^{\ii\delta_i} \qquad (i = 1, 2)                                                               \\
  \EC(\rr, t) & = \Ec_+\ee_+(\kk)e^{\ii(\omega(\kk)t - \kk\vdot\rr)} + \Ec_-\ee_-(\kk)e^{\ii(\omega(\kk)t - \kk\vdot\rr)}      \\
  \Ec_\pm     & = \Et_\mp^* = \mathcal{A}_\pm e^{\ii\delta_\pm} \quad (\mathcal{A}_\pm = a_\mp, \delta_\pm = -\varepsilon_\mp)
\end{align}
円偏光において新しい複素振幅 $\Ec_\pm$ と古い複素振幅 $\Et_\pm$ は成分の添字の $+$ と $-$ が反転して結びついていることに注意すべきである. また, これより次の式を導ける.
\begin{align}
  \ee_i(\kk)\vdot\EC
   & = \Ec_ie^{\ii(\omega(\kk)t - \kk\vdot\rr)} = \qty(\Et_i e^{\ii(\kk\vdot\rr - \omega(\kk)t)})^* = \qty(\ee_i(\kk)\vdot\ET)^* \label{EC ET converter 12}          \\
  \ee_\pm(\kk)^*\vdot\EC
   & = \Ec_\pm e^{\ii(\omega(\kk)t - \kk\vdot\rr)} = \qty(\Et_\mp e^{\ii(\kk\vdot\rr - \omega(\kk)t)})^* = \qty(\ee_\mp(\kk)^*\vdot\ET)^* \label{EC ET converter +-}
\end{align}
\\

\textbf{Q 21B-31.}
また上式の考察より次のように偏光の向きも逆転する.
\begin{align}
  \EC(\rr, t) = \Ec_+\ee_+(\kk)e^{\ii(\omega(\kk)t - \kk\vdot\rr)}
   & \iff \textrm{「円偏光は右偏光である。」}          \\
   & \iff \textrm{「helicity が $-1$ である。」} \\
  \EC(\rr, t) = \Ec_-\ee_-(\kk)e^{\ii(\omega(\kk)t - \kk\vdot\rr)}
   & \iff \textrm{「円偏光は左偏光である。」}          \\
   & \iff \textrm{「helicity が $+1$ である。」}
\end{align}
\\

\textbf{Q 21B-32.}
式 \eqref{EC ET converter 12} より Stokes パラメータは $\EC$ を用いて次のように表される.
\begin{align}
  s_0 & = \qty|\ee_1(\kk)\vdot\ET|^2 + \qty|\ee_2(\kk)\vdot\ET|^2 = |\ee_1(\kk)\vdot\EC|^2 + |\ee_2(\kk)\vdot\EC|^2                    \\
  s_1 & = \qty|\ee_1(\kk)\vdot\ET|^2 - \qty|\ee_2(\kk)\vdot\ET|^2 = |\ee_1(\kk)\vdot\EC|^2 - |\ee_2(\kk)\vdot\EC|^2                    \\
  s_2 & = 2\Re\qty{\qty(\ee_1(\kk)\vdot\ET)^*\qty(\ee_2(\kk)\vdot\ET)} = 2\Re\qty{\qty(\ee_1(\kk)\vdot\EC)^*\qty(\ee_2(\kk)\vdot\EC)}  \\
  s_3 & = -2\Im\qty{\qty(\ee_1(\kk)\vdot\ET)^*\qty(\ee_2(\kk)\vdot\ET)} = 2\Im\qty{\qty(\ee_1(\kk)\vdot\EC)^*\qty(\ee_2(\kk)\vdot\EC)}
\end{align}
\\

\textbf{Q 21B-33.}
式 \eqref{EC ET converter +-} より Stokes パラメータは $\EC$ を用いて次のように表される.
\begin{align}
  s_0 & = \qty|\ee_+(\kk)^*\vdot\ET|^2 + \qty|\ee_-(\kk)^*\vdot\ET|^2 = \qty|\ee_+(\kk)^*\vdot\EC|^2 + \qty|\ee_-(\kk)^*\vdot\EC|^2                          \\
  s_1 & = 2\Re\qty{\qty(\ee_+(\kk)^*\vdot\ET)^*\qty(\ee_-(\kk)^*\vdot\ET)} = 2\Re\qty{\qty(\ee_+(\kk)^*\vdot\EC)^*\qty(\ee_-(\kk)^*\vdot\EC)}                \\
  s_2 & = 2\Im\qty{\qty(\ee_+(\kk)^*\vdot\ET)^*\qty(\ee_-(\kk)^*\vdot\ET)} = 2\Im\qty{\qty(\ee_+(\kk)^*\vdot\EC)^*\qty(\ee_-(\kk)^*\vdot\EC)}                \\
  s_3 & = -\qty{\qty|\ee_+(\kk)^*\vdot\ET|^2 - \qty|\ee_-(\kk)^*\vdot\ET|^2} = \qty|\ee_+(\kk)^*\vdot\EC|^2 - \qty|\ee_-(\kk)^*\vdot\EC|^2 \label{s3 circle}
\end{align}

\subsection{直線偏光子を用いると何が測定できるか?}
まず一般的な楕円偏光の電場において, 次のように Jones ベクトル $\JJ$ を定義する.
\begin{align}
  \JJ         & = \mqty[\Ec_1         \\ \Ec_2]\in\CC^2 \\
  \EC(\rr, t) & = \mqty[\Ec_1(\rr, t) \\ \Ec_2(\rr, t)] = \mqty[\Ec_1 \\ \Ec_2]e^{\ii(\omega(\kk)t - \kk\vdot\rr)}
\end{align}

\textbf{Q 21B-34.}
このとき Stokes ベクトル $\SS$ を Jones ベクトル $\JJ$ を用いて次のように表される.
\begin{align}
  s_0 & = |\Ec_1|^2 + |\Ec_2|^2 \\
  s_1 & = |\Ec_1|^2 - |\Ec_2|^2 \\
  s_2 & = 2\Re(\Ec_1^*\Ec_2)    \\
  s_3 & = 2\Im(\Ec_1^*\Ec_2)
\end{align}

\textbf{Q 21B-35.}
また光の強度 $I$ について Jones ベクトル $\JJ$ の絶対値の二乗に比例することが分かる.
\begin{align}
  I & = \left\langle\EE(\rr, t)\right\rangle                   \\
    & = \left\langle\qty{\Re\EC(\rr,t)}^2\right\rangle         \\
    & = \frac{1}{2}\left\langle\qty|\EC(\rr,t)|^2\right\rangle \\
    & = \frac{1}{2}|\JJ|^2 \label{Jones strength}
\end{align}

\textbf{Q 21B-36.}
以上の定義を用いて \ref{table:Stokes Jones} での Jones ベクトルの正当性が分かる. \\

任意の光学素子を取り上げると, Jones ベクトルの変換は表現論より行列 $M(2; \CC)$ で書ける. これを Jones 行列と呼ぶ. また素子の1つとして直線偏光子があり, 最も単純な直線偏光子は針金格子偏光子 (wire grid polarizer) でしょう. 電磁場の波よりもずっと細い直径を持つ直線状の伝導性の良い金属の針金を多数用意し, それらを波長 $\lambda$ よりずっと狭い間隔だけ離して等間隔に並行に板状に並べる.このような板状の物体が「針金格子偏光子」である. \\

\textbf{Q 21B-37.}
すきまと並行な成分において定常波を作ることができない為, 電磁波を通さない. よって「針金格子偏光子の透過軸は, 針金が並べられた面内で, すきまと垂直な方向である」. \\

\textbf{Q 21B-38.}
1軸 $\ee_1(\kk)$ から測った角度が $\theta$ であるときの直線偏光子の Jones 行列を $T^{\textrm{直線偏光子}}(\theta)$ とおく. このとき, 直線偏光子の定義より
\begin{align}
  T^{\textrm{直線偏光子}}(0)                 & = \mqty[1 & 0 \\ 0 & 0] \\
  T^{\textrm{直線偏光子}}\qty(\frac{\pi}{2}) & = \mqty[0 & 0 \\ 0 & 1]
\end{align}
となる. \\

\textbf{Q 21B-39.}
Jones 行列が $T$ の光学素子 $d$ について角度 $\theta$ だけ回転させた光学素子を $d(\theta)$ として, その Jones 行列 $T(\theta)$ について回転させた座標系で $T$ を適用していると考えられる為, 次のような関係式が成り立つ.
\begin{align}
  T(\theta) = R(\theta)TR(-\theta)
\end{align}

\textbf{Q 21B-40.}
直線偏光子の Jones 行列について上での考察より次のようになる.
\begin{align}
  T^{\textrm{直線偏光子}}(\theta) & = R(\theta)T^{\textrm{直線偏光子}}(0)R(-\theta)                        \\
                             & = \mqty[\cos^2\theta                       & \cos\theta\sin\theta \\ \sin\theta\cos\theta & \sin^2\theta]
\end{align}
また次のように行列 $T^{\textrm{直線偏光子}}(\theta)$ は射影演算子を表す行列であることが分かる.
\begin{align}
  \qty{T^{\textrm{直線偏光子}}(\theta)}^2
   & = \mqty[\cos^2\theta                       & \cos\theta\sin\theta \\ \sin\theta\cos\theta & \sin^2\theta] \\
   & = \qty{T^{\textrm{直線偏光子}}(\theta)}^\dagger                        \\
   & = T^{\textrm{直線偏光子}}(\theta)
\end{align}

\textbf{Q 21B-41.}
水平状態 $\JJ = [\Ec, 0]^t\in\CC^2$ の光を直線偏光子 $T^{\textrm{直線偏光子}}(\theta)$ に通したときの出力の光の強度 $I(\theta)$ を考える. 式 \eqref{Jones strength} より Jones ベクトルを用いて光の強度を表せられる.
\begin{align}
  I(\theta) & = \frac{1}{2}\qty|T^{\textrm{直線偏光子}}(\theta)\mqty[\Ec \\ 0]|^2 \\
            & = \frac{1}{2}\qty|\Ec\mqty[\cos^2\theta               \\ \cos\theta\sin\theta]|^2 \\
            & = \frac{1}{2}|\Ec|^2\cos^2\theta                      \\
            & = I(0)\cos^2\theta
\end{align}
これを Malus の法則と呼ぶ. \\

\textbf{Q 21B-42.}
一般の偏光状態 $\JJ = [\Ec_1, \Ec_2]^t\in\CC^2$ の光を直線偏光子 $T^{\textrm{直線偏光子}}(\theta)$ に通したときの出力の光の強度 $I(\theta)$ を考える.
\begin{align}
  I(\theta) & = \frac{1}{2}\qty|T^{\textrm{直線偏光子}}(\theta)\JJ|^2                                                                                     \\
            & = \frac{1}{2}\JJ^\dagger\qty{T^{\textrm{直線偏光子}}(\theta)}^\dagger T^{\textrm{直線偏光子}}(\theta)\JJ                                         \\
            & = \frac{1}{2}\JJ^\dagger T^{\textrm{直線偏光子}}(\theta)\JJ                                                                                 \\
            & = \frac{1}{2}\qty(|\Ec_1|^2\cos^2\theta + |\Ec_2|^2\sin^2\theta + (\Ec_1^*\Ec_2 + \Ec_1\Ec_2^*)\cos\theta\sin\theta)                   \\
            & = \frac{1}{4}\qty(|\Ec_1|^2 + |\Ec_2|^2) + \frac{1}{4}\qty(|\Ec_1|^2 - |\Ec_2|^2)\cos2\theta + \frac{1}{2}\Re(\Ec_1^*\Ec_2)\sin2\theta \\
            & = \frac{1}{4}\qty{s_0 + s_1\cos2\theta + s_2\sin2\theta} \label{I s012}                                                                \\
            & = \frac{1}{4}\qty{s_0 + \sqrt{s_1^2 + s_2^2}\cos(2\theta - \varphi)}
\end{align}
ただし,
\begin{align}
  \cos\varphi = \frac{s_1}{\sqrt{s_1^2 + s_2^2}}, \sin\varphi = \frac{s_2}{\sqrt{s_1^2 + s_2^2}}
\end{align}
である. これらは次の極めて重要な事実を教えてくれている.
「与えられた単色光の Stokes パラメータのうちの 3 個 $s_0, s_1, s_2$ は, その光をいろいろな角度 $\theta$ に傾けた直線偏光子に透過して強度を測定することによって決定できる.」 \\

\textbf{Q 21B-43.}
上の事実について具体式を考える. 式 \eqref{I s012} より
\begin{align}
  I(0)                  & = s_0 + s_1 \\
  I\qty(\frac{\pi}{4})  & = s_0 + s_2 \\
  I\qty(\frac{\pi}{2})  & = s_0 - s_1 \\
  I\qty(\frac{3\pi}{4}) & = s_0 - s_2
\end{align}
よって Stokes パラメータ $s_0, s_1, s_2$ は次のように表される.
\begin{align}
  s_0 & = 2\qty{I(0) + I\qty(\frac{\pi}{2})}                  \\
      & = 2\qty{I\qty(\frac{\pi}{4}) + I\qty(\frac{\pi}{2})}  \\
  s_1 & = 2\qty{I(0) - I\qty(\frac{\pi}{2})}                  \\
  s_2 & = 2\qty{I\qty(\frac{\pi}{4}) - I\qty(\frac{3\pi}{4})}
\end{align}
これらは次の意味を表す.
\begin{align}
  s_0 & \propto (\textrm{全強度})                                                        \\
  s_1 & \propto (\textrm{水平偏光成分の強度}) - (\textrm{鉛直偏光成分の強度})                           \\
  s_2 & \propto (+45\textrm{\textdegree 偏光成分の強度}) - (-45\textrm{\textdegree 偏光成分の強度})
\end{align}

\subsection{$1/4$ 波長板も用いると何が測定できるか?}
式 \eqref{s3 circle} より $s_3$ は次のような意味を持つ.
\begin{align}
  s_3 \propto (\textrm{右円偏光成分の強度}) - (\textrm{左円偏光成分の強度})
\end{align}
これを測定するにはどうすればよいのか? これは円偏光成分を直線偏光成分に変換できれば測定できる. この変換が $1/4$ 波長板を用いて実行できることをここで学ぶ. \\

遅相子 (wave retarder) は直線偏光成分のうちの片方をもう片方に対して一定の位相だけ遅らせる変換を行い出力する光学素子である.
\begin{align}
  T^{\textrm{遅相子}}(\phi) = \mqty[e^{\ii\frac{\phi}{2}} & 0 \\ 0 & e^{-\ii\frac{\phi}{2}}]
\end{align}
この行列は位相を回転する操作を行うと次と同等となる.
\begin{align}
  \mqty[1 & 0 \\ 0 & e^{-\ii\phi}]
\end{align}

\textbf{Q 21B-44.}
遅相子に純粋に直線偏光した光を入れることを考える.
\begin{align}
  T^{\textrm{遅相子}}(\phi)\mqty[1 \\ 0] & = \mqty[e^{\ii\frac{\phi}{2}} \\ 0] \\
  T^{\textrm{遅相子}}(\phi)\mqty[0 \\ 1] & = \mqty[0 \\ e^{-\ii\frac{\phi}{2}}]
\end{align}
これらは回転する操作を行えば何も変化しないことが分かる. \\

\textbf{Q 21B-45.}
$1/2$ 波長板の Jones 行列を次のように定義する.
\begin{align}
  T^{1/2\textrm{波長板}} & = T^{\textrm{遅相子}}(\pi) = \mqty[\ii & 0 \\ 0 & -\ii]
\end{align}
これに $+45$\textdegree 方向に直線偏光した光と $-45$\textdegree 方向に直線偏光した光を通すと
\begin{align}
  T^{1/2\textrm{波長板}}\frac{1}{\sqrt{2}}\mqty[1 \\ 1] & = \frac{\ii}{\sqrt{2}}\mqty[1 \\ -1] \\
  T^{1/2\textrm{波長板}}\frac{1}{\sqrt{2}}\mqty[1 \\ -1] & = \frac{\ii}{\sqrt{2}}\mqty[1 \\ 1]
\end{align}
となり, これは $1/2$ 波長板によって L+45P と L-45P は相互変換する. \\

また 右円偏光した光と左円偏光した光を通すと
\begin{align}
  T^{1/2\textrm{波長板}}\frac{1}{\sqrt{2}}\mqty[1 \\ \ii] & = \frac{\ii}{\sqrt{2}}\mqty[1 \\ -\ii] \\
  T^{1/2\textrm{波長板}}\frac{1}{\sqrt{2}}\mqty[1 \\ -\ii] & = \frac{\ii}{\sqrt{2}}\mqty[1 \\ \ii]
\end{align}
となり, これは $1/2$ 波長板によって RCP と LCP は相互変換する. \\

\textbf{Q 21B-46.}
$1/4$ 波長板の Jones 行列を次のように定義する.
\begin{align}
  T^{1/4\textrm{波長板}} & = T^{\textrm{遅相子}}\qty(\frac{\pi}{2}) = \mqty[e^{\ii\frac{\pi}{4}} & 0 \\ 0 & e^{-\ii\frac{\pi}{4}}] = \mqty[\frac{1 + \ii}{\sqrt{2}} & 0 \\ 0 & \frac{1 - \ii}{\sqrt{2}}]
\end{align}
これに L+45P, L-45P を通すとそれぞれ LCP, RCP へ変換されることが分かる.
\begin{align}
  T^{1/4\textrm{波長板}}\frac{1}{\sqrt{2}}\mqty[1 \\ 1] & = \frac{1 + \ii}{2}\mqty[1 \\ -\ii] \\
  T^{1/4\textrm{波長板}}\frac{1}{\sqrt{2}}\mqty[1 \\ -1] & = \frac{1 + \ii}{2}\mqty[1 \\ \ii]
\end{align}
同様に RCP, LCP を通すとそれぞれ L+45P, L-45P へ変換されることが分かる.
\begin{align}
  T^{1/4\textrm{波長板}}\frac{1}{\sqrt{2}}\mqty[1 \\ \ii] & = \frac{1 + \ii}{2}\mqty[1 \\ 1] \\
  T^{1/4\textrm{波長板}}\frac{1}{\sqrt{2}}\mqty[1 \\ -\ii] & = \frac{1 + \ii}{2}\mqty[1 \\ -1]
\end{align}
光学領域での $1/4$ 波長板はサランラップを半ダースほど向きを揃えて重ねることにより自作できるらしい. \\

\textbf{Q 21B-47.}
円偏光した光は $1/4$ 波長板により直線偏光に変換し, その光強度を求めることで右偏光, 左偏光の光強度が求まる. 式 \eqref{s3 circle} より $s3$ が求まる.

\section{準単色光と部分偏光}
今まで単色光のときを考えていたが, 波数や角振動数に広がりを持つ場合を考える. 波数, スペクトル線の幅, 角振動数の広がり $\Delta k, \Delta\nu, \Delta\omega$ とおく. \\

\textbf{Q 21B-48.}
波数 $\kk\in\RR^3$ を中心にして, 広がり $|\Delta\kk|\sim\Delta k = c^{-1}\Delta\nu$ を持つ準単色光を考える. このとき電場の複素表示 $\EC(\rr, t)$ は次のように Fourier 変換される.
\begin{align}
  \EC(\rr, t)
   & = \iiint_{|\kk' - \kk|\leq\Delta k}dV(\kk')\qty{\Ec_1(\kk')\ee_1(\kk') + \Ec_2(\kk')\ee_2(\kk')}e^{\ii(\omega(\kk')t - \kk'\vdot\rr)}                                                                                                                                           \\
   & = \iiint_{|\delta\kk|\leq\Delta k}dV(\delta\kk)\qty{\Ec_1(\kk + \delta\kk)\ee_1(\kk + \delta\kk) + \Ec_2(\kk + \delta\kk)\ee_2(\kk + \delta\kk)}e^{\ii(\omega(\kk + \delta\kk)t - (\kk + \delta\kk)\vdot\rr)}                                                                   \\
   & = e^{\ii(\omega(\kk)t - \kk\vdot\rr)}\iiint_{|\delta\kk|\leq\Delta k}dV(\delta\kk)\qty{\Ec_1(\kk + \delta\kk)\ee_1(\kk + \delta\kk) + \Ec_2(\kk + \delta\kk)\ee_2(\kk + \delta\kk)}e^{\ii\qty{(\omega(\kk + \delta\kk) - \omega(\kk))t - \delta\kk\vdot\rr}} \label{EC fourier} \\
\end{align}
ここでコヒーレンス時間 $t_c = \Delta\nu^{-1}$ より十分短い時間間隔 $\Delta t \ll t_c$ のとき $(\omega(\kk + \delta\kk) - \omega(\kk))\Delta t \sim \Delta\nu t_c \ll 1$ となるので式 \eqref{EC fourier} は単色光と見なすことができる. \\

\textbf{Q 21B-49.}
コヒーレンス時間を超える時間スケールではコヒーレンス時間 $t_c$ 程度の時間間毎ごとに定まる Stokes パラメータの時間平均を取ることによって, 準単色波の Stokes パラメータ $s_0, s_1, s_2, s_3$ を定義することが出来る. これより1, 2軸の複素振幅を $a_1e^{\ii\delta_1}, a_2e^{\ii\delta_2}$ とおくと平均値の線形性より
\begin{align}
  s_0 & = \langle a_1^2\rangle + \langle a_2^2\rangle \\
  s_1 & = \langle a_1^2\rangle - \langle a_2^2\rangle \\
  s_2 & = 2\langle a_1a_2\cos\delta\rangle            \\
  s_3 & = 2\langle a_1a_2\sin\delta\rangle
\end{align}
と書ける. 完全偏光において関係式 $s_0^2 = s_1^2 + s_2^2 + s_3^2$ を満たしていたが, 準単色波のとき
\begin{align}
  s_0^2 & = \qty(\langle a_1^2\rangle + \langle a_2^2\rangle)^2                                    \\
        & = s_1^2 + 4\langle a_1^2\rangle\langle a_2^2\rangle                                      \\
        & \geq s_1^2 + 4\langle a_1a_2\rangle^2                                                    \\
        & = s_1^2 + 4\langle a_1a_2\rangle^2\langle\cos^2\delta + \sin^2\delta\rangle              \\
        & \geq s_1^2 + (2\langle a_1a_2\cos\delta\rangle)^2 + (2\langle a_1a_2\sin\delta\rangle)^2 \\
        & = s_1^2 + s_2^2 + s_3^2
\end{align}
このような不等式となるので新しいパラメータ $p\in[0,1]$ を用いて式 \eqref{s1 def} \eqref{s2 def} \eqref{s3 def} を修正する.
\begin{align}
  s_0 & = \langle a_1^2\rangle + \langle a_2^2\rangle \\
  s_1 & = ps_0\cos2\psi\cos2\chi                      \\
  s_2 & = ps_0\sin2\psi\cos2\chi                      \\
  s_3 & = ps_0\sin2\chi
\end{align}
パラメータ $p$ は準単色光の「偏光度」(degree of polarization) と呼ばれる.

準単色光の偏光度 $p$ に関するいくつかの用語と重要な性質をまとめる.
\begin{enumerate}
  \item $p = 1$ の光は「完全偏光」状態にあると言われます. また, $p = 0$ の光は「まったく偏光していない」(completely unpolarized) あるいは「自然光」(natural light) と呼ばれます. それに対して, 一般の $0 \leq p \leq 1$ の光は「部分偏光」状態にあると言われます.
  \item 本物の単色光は完全偏光状態 ($p = 1$) にあります.そして, 完全偏光状態 ($p = 1$) は必ず単色光です. つまり, 単色光と完全偏光状態はまったく同義です.
  \item 部分偏光状態にある光の状態点は半径 $s_0$ の Poincaré 球面の内部の点に対応します. Poincaré 球面の表面の各点が完全偏光状態に対応します. Poincaré 球面の中心の点がまったく偏光していない状態に対応します.
  \item 太陽の光や白熱電球の光はまったく偏光していない状態 ($p = 0$) にあります. それに比べて, レーザーの光は完全偏光状態にごく近いです ($p \approx 1$).
\end{enumerate}

\section{helicity とは何か?}
\subsection{電磁波の角運動量のスピン部分と軌道部分}
\textbf{Q 21B-50.}
電磁波の角運動量 $\LL$ は次のようであった.
\begin{align}
  \LL & = \frac{1}{4\pi c}\iiint_{\RR^3}dV(\rr)\rr\times(\EE\times\BB)
\end{align}
中身の部分を $\BB = \vnabla\times\AA$ を用いて展開すると
\begin{align}
  \rr\times(\EE\times\BB) & = \rr\times(\EE\times(\vnabla\times\AA))                 \\
                          & = \rr\times(\vnabla(\EE\vdot\AA) - (\EE\vdot\vnabla)\AA) \\
                          & ... \textrm{分からなかった}                                     \\
                          & = \EE\times\AA + \EE_j(\rr\times\vnabla)A_j
\end{align}
よって角運動量は次のように書ける.
\begin{align}
  \LL & = \LL_{spin} + \LL_{orbit}
\end{align}
ただし
\begin{align}
  \LL_{spin}  & = \frac{1}{4\pi c}\iiint_{\RR^3}dV(\rr)\EE\times\AA                             \\
  \LL_{orbit} & = \frac{1}{4\pi c}\iiint_{\RR^3}dV(\rr)\sum_{j=x,y,z}\EE_j(\rr\times\vnabla)A_j
\end{align}
である. \\

\textbf{Q 21B-51.}
円偏光の基底ベクトルの定義より,
\begin{align}
  \ee_\pm(\kk)\times\frac{\kk}{|\kk|}
   & = \frac{1}{\sqrt{2}}\qty(\ee_1(\kk) \pm \ii\ee_2(\kk))\times\frac{\kk}{|\kk|}                                    \\
   & = \frac{1}{\sqrt{2}}\qty(-\ee_2(\kk) \pm \ii\ee_1(\kk))                                                          \\
   & = \pm\ii\ee_\pm(\kk)                                                                                             \\
  \ee_\pm(\kk)\times\ee_\pm(\kk)
   & = 0                                                                                                              \\
  \ee_\pm(\kk)\times\ee_\mp(\kk)
   & = \frac{1}{\sqrt{2}}\qty(\ee_1(\kk) \pm \ii\ee_2(\kk))\times\frac{1}{\sqrt{2}}\qty(\ee_1(\kk) \mp \ii\ee_2(\kk)) \\
   & = \mp\ii\ee_1(\kk)\times\ee_2(\kk)                                                                               \\
   & = \mp\ii\frac{\kk}{|\kk|}
\end{align}
となる. \\

\textbf{Q 21B-52}
波数 $\kk$ について対称性が成り立つようにすることで矛盾なく次のように定義できる.
\begin{align}
  \ee_1(-\kk) = \ee_2(\kk), \ee_2(-\kk) = \ee_1(\kk)
\end{align}
このとき次の式が導かれる.
\begin{align}
  \ee_\pm(-\kk) & = \frac{1}{\sqrt{2}}\qty(\ee_1(-\kk)\pm\ii\ee_2(-\kk)) \\
                & = \frac{1}{\sqrt{2}}\qty(\ee_2(\kk)\pm\ii\ee_1(\kk))   \\
                & = \pm\ii\ee_\mp(\kk)
\end{align}

\textbf{Q 21B-53}
$\AA$ について Fourier 変換すると次のようになる.
\begin{align}
  \AA(\rr, t) & = \sum_{j=\pm}\iiint_{\RR^3}\frac{dV(\kk)}{(2\pi)^3}\qty{\ee_j(\kk)a_j(\kk)e^{\ii(\kk\vdot\rr - \omega(\kk)t)} + \ee_j^*(\kk)a_j^*(\kk)e^{-\ii(\kk\vdot\rr - \omega(\kk)t)}}
\end{align}
このとき $\vnabla\vdot\AA$ について基底ベクトルの直交性より
\begin{align}
  \vnabla\vdot\AA & = \sum_{j=\pm}\iiint_{\RR^3}\frac{dV(\kk)}{(2\pi)^3}\qty{\ii\kk\vdot\ee_j(\kk)a_j(\kk)e^{\ii(\kk\vdot\rr - \omega(\kk)t)} - \ii\kk\vdot\ee_j^*(\kk)a_j^*(\kk)e^{-\ii(\kk\vdot\rr - \omega(\kk)t)}} \\
                  & = 0
\end{align}
また電場 $\EE$ と磁場 $\BB$ について
\begin{align}
  \EE(\rr, t) = -\frac{1}{c}\pdv{\AA}{t} & = \frac{1}{c}\sum_{j=\pm}\iiint_{\RR^3}\frac{dV(\kk)}{(2\pi)^3}\ii\omega(\kk)\qty{\ee_j(\kk)a_j(\kk)e^{\ii(\kk\vdot\rr - \omega(\kk)t)} - \ee_j^*(\kk)a_j^*(\kk)e^{-\ii(\kk\vdot\rr - \omega(\kk)t)}} \\
                                         & = \sum_{j=\pm}\iiint_{\RR^3}\frac{dV(\kk)}{(2\pi)^3}\ii|\kk|\qty{\ee_j(\kk)a_j(\kk)e^{\ii(\kk\vdot\rr - \omega(\kk)t)} - \ee_j^*(\kk)a_j^*(\kk)e^{-\ii(\kk\vdot\rr - \omega(\kk)t)}}                  \\
  \BB(\rr, t) = \vnabla\times\AA         & = \sum_{j=\pm}\iiint_{\RR^3}\frac{dV(\kk)}{(2\pi)^3}|\kk|\qty{j\ee_j(\kk)a_j(\kk)e^{\ii(\kk\vdot\rr - \omega(\kk)t)} + j\ee_j^*(\kk)a_j^*(\kk)e^{-\ii(\kk\vdot\rr - \omega(\kk)t)}}
\end{align}
となる. \\

\textbf{Q 21B-54.}
Diracのデルタ関数の公式より
\begin{align}
         & \iiint_{\RR^3}dV(\rr)\EE(\rr, t)\times\AA(\rr, t)                                                                                                                                                                    \\
  =      & \iiint_{\RR^3}dV(\rr)\qty(\sum_{j=\pm}\iiint_{\RR^3}\frac{dV(\kk)}{(2\pi)^3}\qty{\ee_j(\kk)a_j(\kk)e^{\ii(\kk\vdot\rr - \omega(\kk)t)} + \ee_j^*(\kk)a_j^*(\kk)e^{-\ii(\kk\vdot\rr - \omega(\kk)t)}})                \\
  \times & \qty(\sum_{j=\pm}\iiint_{\RR^3}\frac{dV(\kk)}{(2\pi)^3}\ii|\kk|\qty{\ee_j(\kk)a_j(\kk)e^{\ii(\kk\vdot\rr - \omega(\kk)t)} - \ee_j^*(\kk)a_j^*(\kk)e^{-\ii(\kk\vdot\rr - \omega(\kk)t)}})                             \\
  =      & \iiint_{\RR^3}dV(\rr)\iiint_{\RR^3}\frac{dV(\kk)}{(2\pi)^3}\iiint_{\RR^3}\frac{dV(\kk')}{(2\pi)^3}\ii|\kk|\frac{\kk}{|\kk|}                                                                                          \\
         & \sum_{j=\pm}\qty{2j(a_j(\kk)a_j^*(\kk'))e^{j\ii((\kk - \kk')\vdot\rr)} + ja_j(\kk)a_j(-\kk')e^{\ii(j(\kk + \kk')\vdot\rr - 2\omega(\kk)t)} - ja_j^*(\kk)a_j^*(-\kk')e^{-\ii(j(\kk + \kk')\vdot\rr - 2\omega(\kk)t)}} \\
  =      & \sum_{j=\pm}\iiint_{\RR^3}\frac{dV(\kk)}{(2\pi)^3}\kk\qty{2j|a_j(\kk)|^2 + \ii a_j(\kk)a_j(-\kk)e^{-2\ii\omega(\kk)t} - \ii a_j^*(\kk)a_j^*(-\kk)e^{2\ii\omega(\kk)t}}
\end{align}
これより, この時間平均は次のようになる.
\begin{align}
  \left\langle\iiint_{\RR^3}dV(\rr)\EE(\rr, t)\times\AA(\rr, t)\right\rangle = 2\iiint_{\RR^3}\frac{dV(\kk)}{(2\pi)^3}\kk\qty{|a_+(\kk)|^2 - |a_-(\kk)|^2}
\end{align}
したがって角運動量のスピン成分 $\LL_{spin}$ の時間平均は次のようになる.
\begin{align}
  \langle\LL_{spin}\rangle & = \frac{1}{2\pi c}\iiint_{\RR^3}\frac{dV(\kk)}{(2\pi)^3}\kk\qty{|a_+(\kk)|^2 - |a_-(\kk)|^2}
\end{align}

これより次のようなことを教えてくれる.
\begin{enumerate}
  \item 電磁場の角運動量のスピン部分 $L_{spin}$ の時間平均 $\langle\LL_{spin}\rangle$ に対して、各 Fourier モード $(k, \pm)$ は波数ベクトル $\kk$ に重みづけをした形で寄与する。つまり、各 Fourier モードは縦波として $\langle\LL_{spin}\rangle$ へ寄与する。
  \item 波数 $\kk$ のモードの寄与する重みは、左円偏光の強度 $|a_+(\kk)|^2$ から右円偏光の強度 $|a_-(\kk)|^2$ を引いた差 $|a_+(\kk)|^2 - |a_-(\kk)|^2$ に比例する。
  \item つまり、波数 $\kk$ の左円偏光のモードは方向 $\kk/|\kk|$ のスピン角運動量にプラスの寄与をする。一方、波数 $\kk$ の右円偏光のモードは方向 $\kk/|\kk|$ のスピン角運動量にマイナスの寄与をする。
  \item (直線偏光ではなく)円偏光による分解が、電磁波の角運動量に直結している。
\end{enumerate}

\subsection{有限の広がりを持つ円偏光の近似的平面波の角運動量}
積分値が発散しないように波長 $\lambda=2\pi/k$ よりずっと大きな $L$ 程度の有限の領域 $D\subseteq\RR^2$ だけで振幅がゼロでなく, ほぼ一定であるような近似的平面波を考える. \\

\textbf{Q21B-55.}
このとき円偏光の近似的平面波の電場の複素表示 $\ET(x,y,z,t)$ を次のように与える.
\begin{align}
  \ET(x,y,z,t) & = \qty{f(x,y)(\vb{e}_x\pm\ii\vb{e}_y) + g(x,y)\vb{e}_z}e^{\ii(\kk z-\omega t)} \label{def ET D}
\end{align}
また, 復号 $\pm$ により, 2つの円偏光を同時に考察する.
\begin{align}
  \pm\iff \mathrm{helicity} = \pm 1 \iff \begin{cases}
                                           左円偏光 \\
                                           右円偏光
                                         \end{cases}
\end{align}
このとき $f(x,y), g(x,y)$ にはMaxwellの方程式より次のような関係がある.
\begin{align}
  \nabla\vdot\ET(x,y,z,t) & = \pdv{f(x,y)}{x}\pm\ii\pdv{f(x,y)}{y} - \ii kg(x,y) = 0  \\
  g(x,y)                  & = \frac{\ii}{k}\qty{\pdv{f(x,y)}{x}\pm\ii\pdv{f(x,y)}{y}}
\end{align}
これより両辺を領域 D で積分すると近似によって
\begin{align}
  g\sim\frac{1}{kL}f
\end{align}
となることがわかり, $L\to\infty$ で縦成分 $g$ は消える. \\

\textbf{Q 21B-56.}
Coulomb ゲージよりベクトルポテンシャル $\tilde{\AA}$ について次の式が成り立つ.
\begin{align}
  \ET         & = -\frac{1}{c}\pdv{\tilde{\AA}}{t}      \\
  \tilde{\AA} & = -c\int\ET dt = \frac{c}{\ii\omega}\ET
\end{align}

\textbf{Q 21B-57.}
同様に Coulomb ゲージより磁場 $\tilde{\BB}$ について次の式が成り立つ.
\begin{align}
  \tilde{\BB} & = \vnabla\times\tilde{\AA} = \frac{c}{\ii\omega}\vnabla\times\ET                                \\
              & = \qty(\pm kf + \pdv{g}{y}, ikf -\pdv{g}{x}, \pm\ii\pdv{f}{x} - \pdv{f}{y})e^{\ii(kz-\omega t)} \\
              & \sim \pm k\ET
\end{align}
ただしオーダー $O((\frac{1}{kL})^2)$ の項は無視する近似を用いた. \\
プリント間違っていそう. \\

\textbf{Q 21B-58.}
物理的な電場を $\EE = \Re\ET$ とおくと上で議論したことから
\begin{align}
  \BB & = \pm k\Im\ET \label{B tilde E} \\
  \AA & = \frac{c}{\omega}\Im\ET
\end{align}
となる. \\

\textbf{Q 21B-59.}
関数 $f(x,y)$ が円筒対称性を持つときを考える. つまり $f(x,y)$ は xy 平面の極座標 $(\rho, \varphi)$ として $\rho=\sqrt{x^2 + y^2}$ のみの関数となる. このとき関数 $g(x,y)$ は次のように表される.
\begin{align}
  g & = \frac{\ii}{k}\qty(\pdv{f}{\rho}\pdv{\rho}{x}\pm\ii\pdv{f}{\rho}\pdv{\rho}{y}) \\
    & = \frac{\ii}{k}e^{\pm\ii\varphi}\dv{f}{\rho} \label{def g}
\end{align}

\textbf{Q 21B-60.}
角運動量のスピン部分 $\LL_{spin}$ を求める.
\begin{align}
  \EE \times \AA & = \frac{c}{\omega}\Re\ET\times\Im\ET                                                                                                                           \\
                 & = \frac{c}{2\omega}\Im\qty(\ET^*\times\ET)                                                                                                                     \\
                 & = \frac{c}{2\omega}\Im\qty(\qty(\mp 2\ii\Re(f^*g), -2\ii\Im(f^*g), \pm 2\ii|f|^2))                                                                             \\
                 & = \frac{c}{2\omega}\qty(\mp 2\Re(f^*g), -2\Im(f^*g), \pm 2|f|^2)                                                                                               \\
  \Re(f^*g)      & = \Re\qty(f^*\qty(\frac{\ii}{k}e^{\pm\ii\varphi}\dv{f}{\rho})) = \frac{1}{k}\qty{\mp\sin\varphi\Re\qty(f^*\dv{f}{\rho}) - \cos\varphi\Im\qty(f^*\dv{f}{\rho})} \\
  \Im(f^*g)      & = \Im\qty(f^*\qty(\frac{\ii}{k}e^{\pm\ii\varphi}\dv{f}{\rho})) = \frac{1}{k}\qty{\cos\varphi\Re\qty(f^*\dv{f}{\rho}) \mp \sin\varphi\Im\qty(f^*\dv{f}{\rho})}
\end{align}
ここで $xy$ 面内では電磁波が実質的にゼロでない領域を内部に含み, $z$ 方向には十分に長い体積 $V$ を取る. このとき $xy$ 面内では円筒対称に近似的平面波となっているので相殺して十分小さくなる, よって次のようになる.
\begin{align}
  \LL_{spin} & = \frac{1}{4\pi c}\iiint_V dV(\rr)\EE\times\AA               \\
             & = \pm\frac{1}{4\pi\omega}\qty(\iiint_V dV(\rr)|f|^2)\vb{e}_z
\end{align}

\textbf{Q 21B-61.}
角運動量の軌道部分 $\LL_{orbit}$ を求める. Einsteinの縮約を用いて
\begin{align}
  E_j(\rr\times\vnabla)A_j
                    & = \frac{c}{\omega}\qty{\Re(\Et_j)(\rr\times\vnabla)\Im(\Et_j)}                                                                                                                                                                                                                                       \\
                    & = \frac{c}{2\omega}\qty{\Im\qty(\Et_j(\rr\times\vnabla)\Et_j) + \Im\qty(\Et_j^*(\rr\times\vnabla)\Et_j)}                                                                                                                                                                                             \\
  \Et_j(\rr\times\vnabla)\Et_j
                    & = f\mqty|\vb{e}_x                                                                                                                                                                                                                                                              & \vb{e}_y & \vb{e}_z \\ x & y & z \\ \pdv{f}{x} & \pdv{f}{y} & \ii kf | e^{2\ii(\kk z-\omega t)} + \ii f\mqty|\vb{e}_x                                                                                        & \vb{e}_y & \vb{e}_z \\ x & y & z \\ \ii\pdv{f}{y} & \ii\pdv{f}{y} & -kf | e^{2\ii(\kk z-\omega t)} + g\mqty|\vb{e}_x                                                                                        & \vb{e}_y & \vb{e}_z \\ x & y & z \\ \pdv{g}{x} & \pdv{g}{y} & \ii kg | e^{2\ii(\kk z-\omega t)} \\
                    & = g\mqty|\vb{e}_x                                                                                                                                                                                                                                                              & \vb{e}_y & \vb{e}_z \\ x & y & z \\ \pdv{g}{x} & \pdv{g}{y} & \ii kg | e^{2\ii(\kk z-\omega t)} \\
  \Et_j^*(\rr\times\vnabla)\Et_j
                    & = f^*\mqty|\vb{e}_x                                                                                                                                                                                                                                                            & \vb{e}_y & \vb{e}_z \\ x & y & z \\ \pdv{f}{x} & \pdv{f}{y} & \ii kf | - \ii f^*\mqty|\vb{e}_x                                                                                        & \vb{e}_y & \vb{e}_z \\ x & y & z \\ \ii\pdv{f}{y} & \ii\pdv{f}{y} & - kf | + g^*\mqty|\vb{e}_x                                                                                        & \vb{e}_y & \vb{e}_z \\ x & y & z \\ \pdv{g}{x} & \pdv{g}{y} & \ii kg | \\
                    & = 2f^*\mqty|\vb{e}_x                                                                                                                                                                                                                                                           & \vb{e}_y & \vb{e}_z \\ x & y & z \\ \pdv{f}{x} & \pdv{f}{y} & \ii kf | + g^*\mqty|\vb{e}_x                                                                                        & \vb{e}_y & \vb{e}_z \\ x & y & z \\ \pdv{g}{x} & \pdv{g}{y} & \ii kg | \\
  g\mqty|\vb{e}_x   & \vb{e}_y                                                                                                                                                                                                                                                                       & \vb{e}_z            \\ x & y & z \\ \pdv{g}{x} & \pdv{g}{y} & \ii kg |
                    & = \frac{\ii}{k}\dv{f}{\rho}e^{\pm\ii\varphi}\frac{\ii}{k}\qty[\ii ky\dv{f}{\rho}e^{\pm\ii\varphi} - z\qty{\pdv{}{y}\dv{f}{\rho}e^{\pm\ii\varphi}}, z\qty{\pdv{}{x}\dv{f}{\rho}e^{\pm\ii\varphi}} - \ii kx\dv{f}{\rho}e^{\pm\ii\varphi}, \pm\ii\dv{f}{\rho}e^{\pm\ii\varphi}]^t                       \\
                    & = -\frac{1}{k^2}\dv{f}{\rho}e^{\pm 2\ii\varphi}\left[\ii k\rho\sin\varphi\dv{f}{\rho} - z\qty{\sin\varphi\dv[2]{f}{\rho} \pm \frac{\ii\cos\varphi}{\rho}\dv{f}{\rho}},\right.                                                                                                                        \\
                    & \quad \left. z\qty{\cos\varphi\dv[2]{f}{\rho} \mp \frac{\ii\sin\varphi}{\rho} \dv{f}{\rho}} - \ii k\rho\cos\varphi\dv{f}{\rho}, \pm\ii\dv{f}{\rho}\right]^t                                                                                                                                          \\
  g^*\mqty|\vb{e}_x & \vb{e}_y                                                                                                                                                                                                                                                                       & \vb{e}_z            \\ x & y & z \\ \pdv{g}{x} & \pdv{g}{y} & \ii kg |
                    & = \frac{1}{k^2}\qty(\dv{f}{\rho})^*\left[\ii k\rho\sin\varphi\dv{f}{\rho} - z\qty{\sin\varphi\dv[2]{f}{\rho} \pm \frac{\ii\cos\varphi}{\rho}\dv{f}{\rho}},\right.                                                                                                                                    \\
                    & \quad \left. z\qty{\cos\varphi\dv[2]{f}{\rho} \mp \frac{\ii\sin\varphi}{\rho} \dv{f}{\rho}} - \ii k\rho\cos\varphi\dv{f}{\rho}, \pm\ii\dv{f}{\rho}\right]^t                                                                                                                                          \\
  f^*\mqty|\vb{e}_x & \vb{e}_y                                                                                                                                                                                                                                                                       & \vb{e}_z            \\ x & y & z \\ \pdv{f}{x} & \pdv{f}{y} & \ii kf |
                    & = f^* \qty[\ii kyf - z\pdv{f}{y}, z\pdv{f}{x} - \ii kxf, x\pdv{f}{y} - y\pdv{f}{x}]^t                                                                                                                                                                                                                \\
                    & = f^* \qty(\ii k\rho f - z\dv{f}{\rho})\qty[\sin\varphi, -\cos\varphi, 0]^t
\end{align}
となる. これらを角運動量の軌道部分の定義式に代入して $\sin\psi, \cos\psi, e^{\ii z}$ の依存性があるとき積分すると相殺されることから
\begin{align}
  \LL_{orbit} & = \frac{1}{4\pi c}\iiint_VdV(\rr)E_j(\rr\times\nabla)A_j                                                                   \\
              & = \frac{1}{8\pi\omega}\iiint_VdV(\rr)\qty{\Im\qty(\Et_j(\rr\times\vnabla)\Et_j) + \Im\qty(\Et_j^*(\rr\times\vnabla)\Et_j)} \\
              & = \frac{1}{8\pi\omega}\iiint_VdV(\rr)\qty{\Im\qty(\pm\frac{1}{k^2}\qty(\dv{f}{\rho})^*\ii\dv{f}{\rho})}                    \\
              & = \pm\frac{1}{8\pi\omega}\frac{1}{k^2}\qty(\iiint_VdV(\rr)\qty|\dv{f}{\rho}|^2)\vb{e}_z
\end{align}
となる. \\

\textbf{Q 21B-62.}
以下は式を見る事で分かる.
\begin{enumerate}
  \item 角運動量のスピン部分 $\LL_{spin}$ と軌道部分 $\LL_{orbit}$ の両者ともに、電磁波の伝播方向 $+z$ に平行な成分しか持たない。
  \item 軌道部分 $\LL_{orbit}$ はスピン部分 $\LL_{spin}$ に比較して大きさが小さい。両者の大きさの比は小さいパラメータ $1/(kL)$の 2 乗のスケールである:
        \begin{align}
          \frac{|\LL_{orbit}|}{|\LL_{spin}|} \sim \qty(\frac{1}{kL})^2
        \end{align}
  \item よって、xy 平面内での電磁波の広がりを大きくする極限 $1/(kL) \to 0$ において、角運動量の軌道部分 $\LL_{orbit}$ はスピン部分 $\LL_{spin}$ に比べて無視できるようになって、電磁波の全角運動量 $\LL = \LL_{spin} + \LL_{orbit}$ はスピン部分 $\LL_{spin}$ だけからなるようになる:
        \begin{align}
          \LL \to \LL_{spin} = \pm \frac{1}{4\pi\omega}\qty(\iiint_VdV(\rr)|f|^2)\vb{e}_z \label{L limit}
        \end{align}
  \item この結果を見る限り、「平面波の角運動量ベクトルの方向は、偏光状態が左円偏光ならば進行方向に平行であり、偏光状態が右円偏光ならば進行方向に反平行である。」と言える。
\end{enumerate}

\textbf{Q 21B-63.}
CGSガウス単位系でのエネルギー密度の総和を考えると
\begin{align}
  U = \frac{1}{8\pi}\iiint_VdV(\rr)\qty(\qty|\EE|^2 + \qty|\BB|^2)
\end{align}
式 \eqref{def ET D}, \eqref{B tilde E}, \eqref{def g} より
\begin{align}
  \qty|\ET|^2 & = |\EE|^2 + |\BB|^2                          \\
              & = 2|f|^2 + |g|^2                             \\
              & = 2|f|^2 + \frac{1}{k^2}\qty|\dv{f}{\rho}|^2
\end{align}
となるので
\begin{align}
  U         & = U_{spin} + U_{orbit}                                 \\
  U_{spin}  & = \frac{1}{4\pi}\iiint_VdV(\rr)|f|^2                   \\
  U_{orbit} & = \frac{1}{8k^2\pi}\iiint_VdV(\rr)\qty|\dv{f}{\rho}|^2 \\
\end{align}
次の式より電磁場の広がりを十分大きくすると $U_{spin}$ が主要項となる.
\begin{align}
  \frac{U_{orbit}}{U_{spin}} & \sim \qty(\frac{1}{kL})^2                                                \\
  U                          & = U_{spin}                \qquad \qty(\frac{1}{kL}\to 0) \label{U limit}
\end{align}

\textbf{Q 21B-64.}
式 \eqref{L limit}, \eqref{U limit} より次の式が導かれる.
\begin{align}
  L_z = \pm\frac{1}{\omega}U
\end{align}

\textbf{Q 21B-65.}
電磁波を担う実体が光子 (photon) であることを認めると, 1個の光子のエネルギーは $\hbar\omega\ (\omega = c|\kk|)$, 運動量は $\hbar\kk$ である事実が知られている. Q21B-64 の結果と対応原理を組み合わせて光子の角運動量の進行方向の成分が次のようにわかる.
\begin{align}
  \textrm{1個の光子の角運動量 $\LL$ の $\kk/|\kk|$ 方向の成分} =
  \begin{cases}
    +\hbar & 左円偏光状態 \\
    -\hbar & 右円偏光状態
  \end{cases}
\end{align}
このように helicity とは光子の角運動量の進行方向の成分を $\hbar$ 単位で測った量である.




\end{document}