% \usepackage[top=30truemm,bottom=30truemm,left=20truemm,right=20truemm]{geometry}
% \usepackage{float}
% \usepackage{longtable}
% \usepackage{pgfplots}
% \usepackage{diagbox}
% \usepackage{here}
% \usepackage{url}
% \pgfplotsset{compat=newest}
% \newlength\figHeight
% \newlength\figWidth

\documentclass[a4paper,dvipdfmx]{jsarticle}

%ルビ用%
\usepackage{okumacro}
%字下げを保存するための設定 \parでインデント+改行%
\usepackage{indentfirst}
%画像挿入パッケージ。graphix=Windows,graphics=Mac%
\usepackage[dvipdfmx]{graphicx}
%refがハイパーリンクとなる
\usepackage[dvipdfmx]{hyperref}
\hypersetup{
  setpagesize=false,
  bookmarksnumbered=true,
  bookmarksopen=true,
  colorlinks=false,
  linkcolor=blue,
  citecolor=red,
}
%文章を図に回り込ませるパッケージ%
\usepackage{wrapfig}
%ベクトル%
\usepackage{bm}
%url中の_や\にエラーをはかせないためのパッケージ%
\usepackage{url}
%複数行コメントのためのパッケージ%
\usepackage{comment}
%コードのためのパッケージ(英語のみ)%
\usepackage{listings}
%物理関係のパッケージ%
\usepackage{physics}
%数学関係のパッケージ%
\usepackage{amsmath,amsfonts,amssymb}

\DeclareMathOperator{\Ker}{Ker}
\DeclareMathOperator{\ch}{ch}
\usepackage{mathtools}
%定理環境のパッケージ%
\usepackage{amsthm}
\theoremstyle{plain}

\newtheorem{theorem}{Theorem}
\newtheorem*{theorem*}{Theorem}
\theoremstyle{definition}
\newtheorem{definition}{Definition}
\theoremstyle{plain}
\newtheorem{note}{Note}
\newtheorem*{note*}{Note}
\numberwithin{equation}{section}
\numberwithin{theorem}{section}
\numberwithin{definition}{section}
\numberwithin{note}{section}
\theoremstyle{definition}
\newtheorem{dfn}{Definition}[section]
\newtheorem{prop}[dfn]{Proposition}
\newtheorem{lem}[dfn]{Lemma}
\newtheorem{thm}[dfn]{Theorem}
\newtheorem{cor}[dfn]{Corollary}
\newtheorem{rem}[dfn]{Remark}
\newtheorem{fact}[dfn]{Fact}
\renewcommand{\qedsymbol}{$\blacksquare$}
%ファイル分割%
\usepackage{docmute}
%各種設定%
\lstset{
  language=C++,
  breaklines=true,
  keywordstyle = {\color[rgb]{0,0,1}},
  stringstyle = {\color[rgb]{1,0,0}},
  commentstyle = {\color[rgb]{0,1,0}},
  numbers=left,
  frame=lines
}
%色付け 使うときは\documentclass[dvipdfmx]を追加すること!%
\usepackage{color}
%ギリシャ数字%
\usepackage{ascmac}
\usepackage{otf}
%SI単位系%
\usepackage{siunitx}
%tikz%
\usepackage{tikz}
\usetikzlibrary{intersections, calc, arrows, positioning, arrows.meta,automata}
%上部に書く
\usepackage{fancyhdr}
\pagestyle{fancyplain}
    \fancyhead[RO,RE]{\rightmark}
    \fancyhead[LE,LO]{\leftmark}
    \rhead{\thepage}
\renewcommand\plainheadrulewidth{.4pt}% headrule on plain pages
\makeatletter
\renewcommand*\l@section{\@dottedtocline{1}{1.5em}{2.3em}}

\begin{document}
\title{抽象代数}
\author{
  anko
}
\maketitle
\tableofcontents
\clearpage

\section{群論}
\begin{dfn}[群]
  空集合でない集合 $G$ 上で単位元, 逆元のある結合的な演算が定義されているとき $G$ を群という.
\end{dfn}

\begin{dfn}[群の位数]
  群 $G$ の濃度 $|G|$ を群 $G$ の位数という. \\
  また $G$ について $|G| < \infty$ のとき有限群, $|G| \geq \infty$ のとき無限群という.
\end{dfn}

\begin{dfn}[部分群]
  群 $G$ の部分集合 $H$ が群となるとき $H$ は $G$ の部分群であるという.
\end{dfn}

\section{環論}
\begin{dfn}[環]
  集合 $A$ に2つの演算 $+$, $\times$ が定義されていて加法, 乗法に関してそれぞれ可換群, モノイドになるかつ分配法則を満たすとき $A$ を環という.
\end{dfn}

\begin{prop}
  $\forall a\in A\ 0a = a0 = 0$
\end{prop}
\begin{proof}
  $0a = (0 + 0)a = 0a + 0a$ より $0a = 0$. 逆も同様.
\end{proof}

\begin{prop}
  $1=0$ となる環 $\iff$ $0$ 以外の元のない自明な環.
\end{prop}
\begin{proof}
  $a = 1a = 0a = 0$ より任意の元は $0$ となり自明な環となる. 逆は自明.
\end{proof}

具体例
\begin{enumerate}
  \item $\mathbb{Z}\subset\mathbb{Q}\subset\mathbb{R}\subset\mathbb{C}$ の和積は可換環となる.
  \item 行列 $M_n(\mathbb{R})$ は非可換環となる.
  \item 関数 $C^\infty(\mathbb{R})$ の和積は可換環となる.
  \item 群環 (有限群から可換環への写像の像の総和) の和積は環となる.
  \item 2次の環 $\mathbb{Z}[\sqrt{d}]$ は環になる.
\end{enumerate}

\begin{dfn}[準同型・同型]
  $A$, $B$ を環, $\phi:A\to B$ を写像とする.
  \begin{enumerate}
    \item 任意の $x,y\in A$ に対し $\phi(x+y) = \phi(x) + \phi(y)$, $\phi(xy) = \phi(x)\phi(y)$ が成り立ち, $\phi(1_A) = 1_B$ であるとき, $\phi$ を準同型という.
    \item $\phi$ が準同型で逆写像が存在し, 逆写像も準同型であるとき, $\phi$ は同型であるという. また, このとき, $A$, $B$ は同型であるといい, $A\cong B$ と書く.
    \item $A = B$ なら準同型・同型を自己準同型・自己同型という. 環 $A$ の自己同型全体の集合を $\mathrm{Aut}^{\mathrm{al}}A$ と書く.
  \end{enumerate}
\end{dfn}

\begin{prop}
  $\phi:A\to B$ が環の準同型なら $\phi(0_A) = 0_B$ である.
\end{prop}
\begin{proof}
  $\phi(0_A) = \phi(0_A + 0_A) = \phi(0_A) + \phi(0_A)$ より $\phi(0_A) = 0_B$ となる.
\end{proof}

\begin{prop}
  $A$, $B$, $C$ を環, $\phi:A\to B$, $\psi:B\to C$ を準同型とするとき, その合成 $\phi\circ\psi:A\to C$ も準同型である. 同様に $\phi$, $\psi$ が同型なら, $\phi\circ\psi$ も同型である.
\end{prop}
\begin{proof}
  $\psi\circ\phi(x + y) = \psi(\phi(x + y)) = \psi(\phi(x) + \phi(y)) = \psi(\phi(x)) + \psi(\phi(y)) = \psi\circ\phi(x) + \psi\circ\phi(y)$ となり, $\psi\circ\phi(xy) = \psi\circ\phi(x)\psi\circ\phi(y)$ や $\psi\circ\phi(1_A) = 1_C$ も同様に示せるから $\psi\circ\phi$ は準同型である. 同型も同様.
\end{proof}

\begin{prop}
  $\phi:A\to B$ が環の準同型ならば, 単射 $\iff$ $\Ker{\phi} = \qty{0}$
\end{prop}
\begin{proof}
  ($\implies$) $\phi$ が環の準同型であるから $\phi(0_A) = 0_B$ より $0_A\in\Ker\phi$. また元 $\forall x, y \in \Ker\phi$ について $\phi$ の単射性より $\phi(x) = \phi(y) \implies x = y$ となり, $\Ker\phi$ には $0$ 以外の元は存在しない. \\
  ($\impliedby$) $\phi(x) = \phi(y)$ となる $x, y$ について
  \begin{align}
    1 & = \phi(x)\phi(y)^{-1} = \phi(x)\phi(y^{-1}) = \phi(xy^{-1}) \\
    1 & = xy^{-1}
  \end{align}
  より $x = y$ となるから $\phi$ は単射である.
\end{proof}

\begin{dfn}[$n$ 変数多項式]
  $A$ 係数あるいは $A$ 上の $n$ 変数 $x = (x_1,\cdots,x_n)$ の多項式とは, $\mathbb{N}^n$ から $A$ への写像で有限個の $(i_1,\cdots,i_n)\in\mathbb{N}^n$ を除いて値が $0$ になるものと, 変数 $x = (x_1,\cdots,x_n)$ の組のことである. この写像の $(i_1,\cdots,i_n)\in\mathbb{N}$ での値が $a_{i_1,\cdots,i_n}$ なら, この多項式を
  $$
    f(x) = f(x_1,\cdots,x_n) = \sum_{i_1,\cdots,i_n\geq 0}a_{i_1,\cdots,i_n}x_1^{i_1}\cdots x_n^{i_n}
  $$
  などと書く. すべての $a_{i_1,\cdots,i_n}$ が $0$ である多項式を $0$ と書く. 各 $a_{i_1,\cdots,i_n}x_1^{i_1}\cdots x_n^{i_n}$ を $f(x)$ の項, $a_{i_1,\cdots,i_n}$ を係数という. 特に $a_{0,\cdots,0}$ を $f(x)$ の定数項という.
\end{dfn}
\begin{dfn}[$n$ 変数多項式の代入]
  $c = (c_1,\cdots,c_n) \in A^n$ とするとき
  $$
    f(c) = f(c_1,\cdots,c_n) = \sum_{i_1,\cdots,i_n}a_{i_1,\cdots,i_n}c_1^{i_1}\cdots c_n^{i_n}
  $$
  とする. この値を考えることを代入という.
\end{dfn}
\begin{dfn}[$n$ 変数多項式の次数]
  $f(x)$ の次数 $\deg f(x)$ を
  $$
    \deg f(x) = \begin{cases}
      \max\lbrace i_1 + \cdots + i_n \mid a_{i_1,\cdots,i_n} \neq 0 \rbrace & (f(x) \neq 0) \\
      -\infty                                                               & (f(x) = 0)
    \end{cases}
  $$
  と定義する.
\end{dfn}
\begin{dfn}[$A$ 係数あるいは $A$ 上の $n$ 変数多項式環]
  2つの $n$ 変数多項式
  $$
    f(x) = \sum_{i_1,\cdots,i_n}a_{i_1,\cdots,i_n}x_1^{i_1}\cdots x_n^{i_n}, \quad g(x) = \sum_{i_1,\cdots,i_n}b_{i_1,\cdots,i_n}x_1^{i_1}\cdots x_n^{i_n}
  $$
  は, $a_{i_1,\cdots,i_n} = b_{i_1,\cdots,i_n}$ がすべての $i_1,\cdots,i_n$ に対して成り立つとき多項式の同値関係 $f(x) = g(x)$ であると定義する. また次のように多項式の和差積を定義する.
  \begin{align}
    (f\pm g)(x) & = \sum_{i_1,\cdots,i_n}(a_{i_1,\cdots,i_n}\pm b_{i_1,\cdots,i_n})x_1^{i_1}\cdots x_n^{i_n}   \\
    f(x)g(x)    & = \sum_{i_1,\cdots,j_n}a_{i_1,\cdots,i_n}b_{j_1,\cdots,j_n}x_1^{i_1+j_1}\cdots x_n^{i_n+j_n}
  \end{align}
  すると多項式全体の集合 $A[x]$ は環となり, $A$ 係数あるいは $A$ 上の $n$ 変数多項式環という.
\end{dfn}

\begin{dfn}[無限変数多項式環]
  無限変数多項式環 $A[x_i]_{i\in I}$ とは $n > 0$ を整数とするとき, $X_n$ を $\mathbb{N}^n$ から $A$ への写像 $a$ で有限個の $(i_1,\cdots,i_n)\in\mathbb{N}^n$ を除いて値が $0$ であるものと $\lbrace1,\cdots,n\rbrace$ から $I$ への単射写像 $\phi$ の組全体の集合とする. $X_n$ には $\mathfrak{S}_n$ が作用し, その軌道の集合を $Y_n$ とする. $(a,\phi)\in X_n$ で代表される $Y_n$ の元に対し,
  $$\sum_{i_1,\cdots,i_n\in\mathbb{N}}a(i_1,\cdots,i_n)x_{\phi(1)}^{i_1}\cdots x_{\phi(n)}^{i_n}$$
  と書く. これは代表元のとりかたによらず定まる. $\lbrace Y_n\rbrace_n$ は集合族となり, $n\leq m$ なら $Y_n \subseteq Y_m$ とみなせる. $A[x_i]_{i\in I} = \bigcup_n Y_n$ と定義すればよい. $A[x_i]_{i\in I}$ が集合として存在するときそれを無限変数多項式環という.
\end{dfn}

\begin{lem}
  $$
    A[x_1,\cdots,x_n]\cong A[x_1,\cdots,x_{n-1}][x_n]
  $$
\end{lem}
\begin{proof}
  $$
    f(x_1,\cdots,x_n) = \sum_{i_n}\left(\sum_{i_1,\cdots,i_{n-1}}a_{i_1,\cdots,i_n}x_1^{i_1}\cdots x_{n-1}^{i_{n-1}}\right)x_n^{i_n}
  $$
\end{proof}

\begin{dfn}[環の鎖] a
  \begin{enumerate}
    \item 環: 加法が可換群, 乗法がモノイドであり, また分配法則を満たす.
    \item 可換環: 環について乗法が可換である.
    \item 整域: 可換環 $A$ について任意の $a,b\in A\backslash\lbrace0\rbrace$ に対し, $ab\neq 0$ となる.
    \item 正規環: 整域 $A$ について商体 $K$ の元が $A$ 上整なら $A$ の元となる.
    \item 一意分解環 (UFD): 整域について任意の元は素元分解できる.
    \item 単項イデアル整域 (PID): 整域について任意のイデアルが単項イデアルである.
    \item ユークリッド環: 整域 $A$ について写像 $d:A\backslash\lbrace0\rbrace\to\mathbb{N}$ があり, $a,b\in A$ で $b\neq 0$ なら, $q,r\in A$ があり, $a = qb + r$ で $r = 0$ または $d(r)<d(b)$ となる.
    \item 体: 可換環 $A$ の乗法が $A\backslash\lbrace0\rbrace$ において可換群となる.
  \end{enumerate}
\end{dfn}

\begin{thm}[環の鎖]
  環$\impliedby$可換環$\impliedby$整域$\impliedby$正規環$\impliedby$ UFD $\impliedby$ PID $\impliedby$ユークリッド環$\impliedby$体
\end{thm}
\begin{proof}
  (可換環$\implies$環) 自明. \\
  (整域$\implies$可換環) 自明. \\
  (正規環$\implies$整域) 自明. \\
  (UFD$\implies$正規環) $\alpha\in K$ を解に持つモニック多項式 $f(x) = x^n+\cdots+a_1x + a_0$ について $a_0 \neq 0$ とすると, $\alpha \neq 0$ である. ここで $\alpha$ を既約分数として $\alpha = \beta/\gamma$ と表すと, $\gamma^nf(\alpha) = \beta^n + a_{n-1}\gamma\beta^{n-1} + \cdots + a_0\gamma^n = 0$ より $\beta^n = -\gamma(a_1\beta^{n-1} + \cdots + a_n\gamma^{n-1})$ なので, $\gamma\in A^\times$ となる. よって $\alpha\in A$ である. \\
  (PID$\implies$UFD) \\
  (ユークリッド環$\implies$PID) あるイデアル $I\subseteq A$ に対し, $x = \min\lbrace d(y)\mid I\ni y \neq 0\rbrace$ とおくと $I = (x)$ となることを示す. イデアル $I$ の元 $\forall z=qx+r\in I$ について $r=0\lor d(r)<d(x)$ であり, $0 = d(r) < d(x)$ より $r = 0$ となる. よって $z = qx \in (x)$ となるので $I = (x)$ である. \\
  (体$\implies$ユークリッド環)

\end{proof}


\begin{prop}
  部分環に性質が引き継がれる.
  \begin{enumerate}
    \item 環の部分環は環である.
    \item 可換環の部分環は可換環である.
    \item 整域の部分環は整域である.
    \item 正規環の部分環は正規環である.
    \item UFDの部分環はUFDである.
    \item PIDの部分環はPIDではない?
    \item ユークリッド環の部分環はユークリッド環ではない?
    \item 体の部分環は体ではない.
  \end{enumerate}
\end{prop}
\begin{proof}
  それぞれ証明する. 反例を挙げる.
  \begin{enumerate}
    \item 定義から自明.
    \item 任意の元 $a,b\in A$ について可換ならばその部分集合も成り立つ.
    \item
    \item
    \item
    \item
    \item
    \item 有理数体 $\mathbb{Q}$ の部分環 $\mathbb{Z}$ は体ではない.
  \end{enumerate}
\end{proof}

\begin{prop}[素イデアルと極大イデアルの関係]
  素イデアル
  \begin{enumerate}
    \item $A$ が環なら, $A$ の任意の極大イデアルは素イデアルである.
    \item $A$ が単項イデアル整域なら, $(0)$ でない任意の素イデアルは極大イデアルである. したがって, $p$ が素元なら, $A/(p)$ は体である.
  \end{enumerate}
\end{prop}
\begin{prop}[素元と既約元の関係]
  素元
  \begin{enumerate}
    \item $A$ が整域なら, $A$ の素元は既約元である.
    \item $A$ が一意分解環なら, $A$ の既約元は素元である.
  \end{enumerate}
\end{prop}
\begin{proof}

\end{proof}

\begin{prop}
  体の多項式環はユークリッド環である.
\end{prop}
\begin{proof}
  $d = \deg$ とすると成り立つ.
\end{proof}

\begin{prop}[正規環]
  $f(x) = a_nx^n + \cdots + a_0\in A[x]$ で $a_0, a_n \neq 0$, $\alpha\in K$
\end{prop}

\begin{dfn}
  \begin{enumerate}
    \item ネーター環
    \item アルティン環
  \end{enumerate}
\end{dfn}

\section{加群}
\begin{dfn}
  環 $R$ 上の行列の集合について定義する.
  \begin{enumerate}
    \item $m\times n$ 行列の集合を $M_{m,n}(R)$.
    \item $n$ 次正方行の集合を $M_n(R)$.
    \item $M_n(R)$ の乗法群(正則行列の集合)を一般線形群 $GL_n(R)$.
    \item $GL_n(R)$ の $\det$ の核 (行列式の値が単位元) を特殊線形群 $SL_n(R)$.
  \end{enumerate}
\end{dfn}

\section{体論}

\begin{dfn}
  体 $K$ について自然な環準同型 $\phi:\mathbb{Z}\ni n\mapsto n\cdot 1\in K$ の核 $\mathrm{Ker}(\phi)\subset\mathbb{Z}$ は準同型定理 $\mathbb{Z}/\mathrm{Ker}(\phi)\cong\Im(\phi)$ と $\Im(\phi)$ は整域であることから素イデアルである. $\mathbb{Z}$ の素イデアルは $(0)$ または素数 $p$ があり $(p)$ であるから $\mathrm{Ker}(\phi)$ も $(0)$ または $(p)$ である. このとき標数 $\mathrm{ch}\ K$ は それぞれ $0$, $p$ であるという.
\end{dfn}
\begin{thm}
  標数に対応する体の性質
  \begin{enumerate}
    \item 標数 $0$ の体なら, $\mathbb{Q}$ を含む.
    \item 標数 $p > 0$ の体なら, $\mathbb{F}_p$ を含む.
  \end{enumerate}
\end{thm}
\begin{proof}

\end{proof}

\begin{dfn}
  $K$ が標数 $p>0$ の体, $n$ が正の整数で $q = p^n$ とおくと $\mathrm{Frob}_q:K\ni x\mapsto x^q\in K$ をフロベニウス準同型という.
\end{dfn}
\begin{thm}
  フロベニウス準同型は体の準同型である.
\end{thm}
\begin{proof}
  \begin{align}
    \mathrm{Frob}_q(x + y) & = (x + y)^q = x^q + y^q = \mathrm{Frob}_q(x) + \mathrm{Frob}_q(y) \\
    \mathrm{Frob}_q(xy)    & = (xy)^q = x^qy^q = \mathrm{Frob}_q(x)\mathrm{Frob}_q(y)          \\
    \mathrm{Frob}_q(0)     & = 0                                                               \\
    \mathrm{Frob}_q(1)     & = 1
  \end{align}
\end{proof}


\begin{dfn}
  体の拡大の定義
  \begin{enumerate}
    \item $L/K$ が体の拡大とする. $L$ の $K$ 上のベクトル空間としての次元を $[L:K]$ と書く. 拡大次数 $[L:K]$ が有限なら $L/K$ を有限次拡大, そうでなければ無限次拡大という.
    \item $n$ 変数有理式. $\alpha\in L$ により $K(\alpha)$ となるとき, $L$ を $K$ の単拡大という. 有限個の元 $\alpha_1,\cdots,\alpha_n\in L$ により $L = K(\alpha_1,\cdots,\alpha_n)$ となるとき, $L$ は $K$ 上有限生成であるという.
    \item $L/K$ が体の拡大とする. $L$ のすべての元が $K$ 上代数的なら $L/K$ は代数拡大という. そうでなければ, 超越拡大という.
    \item $K$ を任意の定数でない1変数多項式 $f(x)\in K[x]$ に対し $\alpha\in K$ があり, $f(\alpha) = 0$ となるとき, $K$ を代数閉体という. $L/K$ が代数拡大であり $L$ が代数閉体であるとき, $L$ を $K$ の代数閉包という.
    \item $L/K$ が代数拡大とする. $L$ のすべての元が $K$ 上分離的なら $L/K$ を分離拡大という.
    \item $L/K$ が代数拡大とする. $\alpha\in L$ なら $\alpha$ の $K$ 上の最小多項式が $L$ 上では1次式の積になるとき, $L/K$ を正規拡大という.
    \item $L/K$ が分離拡大かつ正規拡大ならガロア拡大という.
  \end{enumerate}
\end{dfn}

\begin{prop}
  $L/M$, $M/K$ が有限次拡大なら $L/K$ も有限次拡大で $[L:K] = [L:M][M:K]$ となる.
\end{prop}
\begin{proof}
  $l = [L:M]$, $m = [M:K]$, $\lbrace x_1,\cdots,x_l \rbrace$ を $L$ の $M$ 上の基底, $\lbrace y_1,\cdots,y_m \rbrace$ を $M$ の $K$ 上の基底とする. \\
  このとき $z\in L$ なら $a_i\in M$ があり, $z = \sum_ia_ix_i$ となる. また $b_{ij}\in K$ があり $z = \sum_{i,j}b_{ij}x_iy_j$ となる. よって, $B=\lbrace x_iy_j\mid i=1,\cdots,l, j = 1,\cdots,m\rbrace$ は $K$ 加群として $L$ を生成する.
\end{proof}

\begin{prop}
  既約な最小多項式
  \begin{enumerate}
    \item $L = K[x]/(f(x))$ は体で $[L:K] = \deg f(x)$ である.
  \end{enumerate}
\end{prop}
\begin{proof}
\end{proof}

\begin{thm}
  以下は互いに同値である.
  \begin{enumerate}
    \item $L/K$ が有限生成かつ代数拡大である.
    \item $L/K$ が有限個の代数的な元から生成している.
    \item $L/K$ が有限次拡大である.
  \end{enumerate}
\end{thm}
\begin{proof}
  (1$\implies$2) \\
  代数的でない元から生成すれば超越拡大となることから代数的な元から生成している. \\
  (2$\implies$3) \\
  代数的な元から生成するとその最小多項式の次数が拡大次数となることから $n$ に関する帰納法を用いて $L/K$ は有限次拡大である. \\
  (3$\implies$1) \\
  $n = [L : K]$ として $x\in L$ なら $1, x, \cdots, x^n$ は一次従属である. よって, $a_0+\cdots+a_nx^n = 0$ となる $a_0,\cdots,a_n$ が存在する. つまり, $x$ は代数的な元である. これより $L/K$ は代数拡大である. \\
  また, 無限生成ならば無限次拡大であるから有限生成である. \\
\end{proof}

\begin{prop}
  $L/K$ が代数拡大, $\alpha\in L$ で, $L\supset M\supset K$ を中間体とする. $\alpha$ が $K$ 上分離的なら $M$ 上でも分離的である.
\end{prop}
\begin{proof}
  $\alpha\in\overline{K}$ の $K$ 上の最小多項式 $f(x)$ が分離多項式であり, $\overline{K} = \overline{M}$ かつ $M$ 上の最小多項式は $f(x)$ を割り切るから分離多項式となり, $M$ 上でも分離的である.
\end{proof}

\begin{prop}
  $f(x)\in K[x]$ を $K$ 上既約な多項式とする. このとき以下は互いに同値である.
  \begin{enumerate}
    \item $f(x)$ は $\overline{K}$ 上で重根を持つ
    \item $f'(x) = 0$
    \item $\ch K = p > 0$ であり, $K$ 上既約な分離多項式 $g(x)$ と $n > 0$ があり, $f(x) = g(x^{p^n})$ となる.
  \end{enumerate}
\end{prop}
\begin{proof}
  ($1\iff 2$) \\
  $f'(x) = 0$ ならば \\
  ($2\iff 3$)
\end{proof}

\begin{thm}
  Steinitzの定理
  \begin{enumerate}
    \item $K$ の代数拡大 $L$ で, 代数閉体であるものが存在する.
    \item $L_1\supseteq M_1\supseteq K$, $L_2\supseteq M_2\supseteq K$
  \end{enumerate}
\end{thm}
\begin{proof}
\end{proof}

\begin{dfn}
  数論に用いられる用語の定義
  \begin{enumerate}
    \item $\mathbb{Q}$ の有限次拡大体を代数体という.
    \item $x\in\mathbb{C}$ が $\mathbb{Z}$ 上整なら代数的整数であるという.
    \item $\mathbb{C}$ の代数的整数の集合を $\Omega$ と書き, 代数的整数環という.
    \item $L$ を代数体, $\Omega$ を代数的整数環とするとき, $L\cap\Omega$ を $L$ の整数環という.
  \end{enumerate}
\end{dfn}
\end{document}