\RequirePackage{plautopatch}
\documentclass[uplatex,dvipdfmx,a4paper,11pt]{jlreq}
\usepackage{bxpapersize}
\usepackage[utf8]{inputenc}
\usepackage{fontenc}
\usepackage{lmodern}
\usepackage{otf}
\usepackage{amsmath}
\usepackage{amssymb}
\usepackage{amsthm}
\usepackage{ascmac}
% \usepackage[hyphens]{url}
\usepackage{physics2}
\usephysicsmodule{ab, ab.braket, doubleprod, diagmat, xmat}
\usepackage[DIF = {
op-symbol = d,
op-order-nudge = 1 mu,
outer-Ldelim = \left . ,
outer-Rdelim = \right |,
sub-nudge = 0 mu
}]{diffcoeff}
% \usepackage{braket}
\usepackage{verbatimbox}
\usepackage{bm}
\usepackage{url}
% \usepackage[dvipdfmx,hiresbb,final]{graphicx}
\usepackage{hyperref}
\usepackage{pxjahyper}
\usepackage{tikz}\usetikzlibrary{cd}
\usepackage{listings}
\usepackage{color}
\usepackage{mathtools}
\usepackage{xspace}
\usepackage{xy}
\usepackage{xypic}
%
\title{解析力学}
\author{anko9801}
\makeatletter
%
\DeclareMathOperator{\lcm}{lcm}
\DeclareMathOperator{\Kernel}{Ker}
\DeclareMathOperator{\Image}{Im}
\DeclareMathOperator{\ch}{ch}
\DeclareMathOperator{\Aut}{Aut}
\DeclareMathOperator{\Log}{Log}
\DeclareMathOperator{\Arg}{Arg}
\DeclareMathOperator{\sgn}{sgn}
%
\newcommand{\CC}{\mathbb{C}}
\newcommand{\RR}{\mathbb{R}}
\newcommand{\QQ}{\mathbb{Q}}
\newcommand{\ZZ}{\mathbb{Z}}
\newcommand{\NN}{\mathbb{N}}
\newcommand{\FF}{\mathbb{F}}
\newcommand{\PP}{\mathbb{P}}
\newcommand{\GG}{\mathbb{G}}
\newcommand{\TT}{\mathbb{T}}
\newcommand{\EE}{\bm{E}}
\newcommand{\BB}{\bm{B}}
\renewcommand{\AA}{\bm{A}}
\newcommand{\rr}{\bm{r}}
\newcommand{\kk}{\bm{k}}
\newcommand{\pp}{\bm{p}}
\newcommand{\calB}{\mathcal{B}}
\newcommand{\calF}{\mathcal{F}}
\newcommand{\ignore}[1]{}
\newcommand{\floor}[1]{\left\lfloor #1 \right\rfloor}
% \newcommand{\abs}[1]{\left\lvert #1 \right\rvert}
\newcommand{\lt}{<}
\newcommand{\gt}{>}
\newcommand{\id}{\mathrm{id}}
\newcommand{\rot}{\curl}
\renewcommand{\angle}[1]{\left\langle #1 \right\rangle}
\newcommand\mqty[1]{\begin{pmatrix}#1\end{pmatrix}}
\newcommand\vmqty[1]{\begin{vmatrix}#1\end{vmatrix}}
\numberwithin{equation}{section}

\let\oldcite=\cite
\renewcommand\cite[1]{\hyperlink{#1}{\oldcite{#1}}}

\let\oldbibitem=\bibitem
\renewcommand{\bibitem}[2][]{\label{#2}\oldbibitem[#1]{#2}}

% theorem環境の設定
% - 冒頭に改行
% - 末尾にdiamond (amsthm)
\theoremstyle{definition}
\newcommand*{\newscreentheoremx}[2]{
  \newenvironment{#1}[1][]{
    \begin{screen}
    \begin{#2}[##1]
      \leavevmode
      \newline
  }{
    \end{#2}
    \end{screen}
  }
}
\newcommand*{\newqedtheoremx}[2]{
  \newenvironment{#1}[1][]{
    \begin{#2}[##1]
      \leavevmode
      \newline
      \renewcommand{\qedsymbol}{\(\diamond\)}
      \pushQED{\qed}
  }{
      \qedhere
      \popQED
    \end{#2}
  }
}
\newtheorem{theorem*}{定理}[section]

\newqedtheoremx{theorem}{theorem*}
\newcommand*\newqedtheorem@unstarred[2]{%
  \newtheorem{#1*}[theorem*]{#2}
  \newqedtheoremx{#1}{#1*}
}
\newcommand*\newqedtheorem@starred[2]{%
  \newtheorem*{#1*}{#2}
  \newqedtheoremx{#1}{#1*}
}
\newcommand*{\newqedtheorem}{\@ifstar{\newqedtheorem@starred}{\newqedtheorem@unstarred}}

\newtheorem{sctheorem*}{定理}[section]
\newscreentheoremx{sctheorem}{sctheorem*}
\newcommand*\newscreentheorem@unstarred[2]{%
  \newtheorem{#1*}[theorem*]{#2}
  \newscreentheoremx{#1}{#1*}
}
\newcommand*\newscreentheorem@starred[2]{%
  \newtheorem*{#1*}{#2}
  \newscreentheoremx{#1}{#1*}
}
\newcommand*{\newscreentheorem}{\@ifstar{\newscreentheorem@starred}{\newscreentheorem@unstarred}}

%\newtheorem*{definition}{定義}
%\newtheorem{theorem}{定理}
%\newtheorem{proposition}[theorem]{命題}
%\newtheorem{lemma}[theorem]{補題}
%\newtheorem{corollary}[theorem]{系}

\newqedtheorem{lemma}{補題}
\newqedtheorem{corollary}{系}
\newqedtheorem{example}{例}
\newqedtheorem{proposition}{命題}
\newqedtheorem{remark}{注意}
\newqedtheorem{thesis}{主張}
\newqedtheorem{notation}{記法}
\newqedtheorem{problem}{問題}
\newqedtheorem{algorithm}{アルゴリズム}

\newscreentheorem*{axiom}{公理}
\newscreentheorem*{definition}{定義}

\renewenvironment{proof}[1][\proofname]{\par
  \normalfont
  \topsep6\p@\@plus6\p@ \trivlist
  \item[\hskip\labelsep{\bfseries #1}\@addpunct{\bfseries}]\ignorespaces\quad\par
}{%
  \qed\endtrivlist\@endpefalse
}
\renewcommand\proofname{証明}

\makeatother

\begin{document}
\maketitle
\tableofcontents
\clearpage

\section{ニュートン力学の復習}
\begin{definition}
  ニュートンの運動の 3 法則
  \begin{itemize}
    \item 第 1 法則: 物体が力を受けないとき、物体の運動状態は変化しない。(慣性の法則)
    \item 第 2 法則: 物体に力が働くとき、力に比例する加速度を持つ。(運動の法則)
    \item 第 3 法則: 物体 1, 2 の間に力が働くとき、互いに大きさが等しく逆向きとなる。(作用・反作用の法則)
  \end{itemize}
\end{definition}
これらを定式化する。
\begin{align}
  m\diff[2]{\rr}{t} = \bm{F}
\end{align}

例えば

- 電磁気力
- ばね
- 垂直抗力
- 慣性力
- 外力
- 重力

\begin{definition}[仕事]
  \begin{align}
    \dl{W} & = \bm{F}\cdot\dl{\rr}        \\
    W      & = \int_C \bm{F}\cdot\dl{\rr}
  \end{align}
  一般に仕事は経路 $C$ に依存する。
  仕事が経路に依らない力を保存力という。
\end{definition}
保存力のとき始点 $\rr_0$ と終点 $\rr$ を用いて
\begin{align}
  W = \int_{\rr_0}^{\rr}\bm{F}\cdot\dl{\rr}
\end{align}
ポテンシャルエネルギー $U(\rr)$ を基準点 $\rr_0$ から $\rr$ までに進むのに必要な仕事と定義する。
\begin{align}
  U & = -\int_{\rr_0}^{\rr}\bm{F}\cdot\dl{\rr}
\end{align}
これは保存力でしか定義できない。

\begin{theorem}[エネルギー保存則]
  \begin{align}
    T + U(\rr)
  \end{align}
\end{theorem}


\section{解析力学}
ニュートンの運動方程式の書き換えを行う。目的は解きやすくするためと定式化の適用範囲を拡張するためである。
\begin{align}
  U(\rr) = -\int_{\rr_0}^{\rr}\bm{F}\dl{\rr}
\end{align}
保存力に関するニュートンの運動方程式
\begin{align}
  m\diff[2]{\rr}{t} = -\nabla U
\end{align}
運動エネルギー
\begin{align}
  T & = \frac{1}{2}m\dot\rr^2
\end{align}
ポテンシャルエネルギー
\begin{table}[hbtp]
  \centering
  \begin{tabular}{|c|c|c|}
    \hline \hline
    保存力         & ポテンシャルエネルギー                                  & \\
    \hline \hline
    重力 (local)  & $mgh$                                        & \\
    重力 (global) & $-G\dfrac{Mm}{r}$                            & \\
    バネの力        & $\dfrac{1}{2}kx^2$                           & \\
    電磁気力        & $- q\phi(\rr, t) + q\dot\rr\cdot\AA(\rr, t)$ & \\
    \hline
  \end{tabular}
  \caption{ポテンシャルエネルギーの表式}
  \label{table:potential}
\end{table}

\subsection{ラグランジュ形式}
\begin{definition}[ラグランジアン]
  ラグランジュ形式において、ラグランジアン (Lagrangian) とよばれる物理量が基本的な量となる。
  \begin{align}
    L & = \frac{1}{2}m\dot{\rr}^2 - U(\rr)
  \end{align}
\end{definition}
\begin{align}
  \diffp{L}{\dot\rr} = m\dot\rr, \quad \diffp{L}{\rr} = -\nabla U
\end{align}
運動方程式は Euler-Lagrange 方程式とよばれる。
\begin{align}
  \frac{\partial L}{\partial \rr} - \frac{d}{dt}\left(\frac{\partial L}{\partial \dot{\rr}}\right) = 0
\end{align}
一般座標 (generalized coordinates) $q_i = q_i(t)$ ($i = 1,2,\ldots,n$) を用いて系がラグランジアン $L = L(q_i, \dot{q}_i)$ で記述される場合、運動方程式は
\begin{align}
  \frac{\partial L}{\partial q_i} - \frac{d}{dt}\left(\frac{\partial L}{\partial \dot{q}_i}\right) = 0
\end{align}
ラグランジアンは座標系による?
すべての物体は運動方程式を解くことでどう動くかわかる
保存力を考えているのでポテンシャルが存在するので次のように表現できる。
\begin{align}
  \frac{\partial L}{\partial q_i} - \frac{d}{dt}\left(\frac{\partial L}{\partial \dot{q}_i}\right) & = -\frac{\partial U}{\partial q_i} - \frac{dp_i}{dt} = -(\nabla U)_i - ma_i
\end{align}
ある座標変換に対して作用積分が対称ならばその対称性に対応する保存量が存在する


\subsection{ハミルトン形式}
\begin{definition}[ハミルトニアン]
  ハミルトニアン (Hamiltonian)
  \begin{align}
    H           & = \frac{\pp^2}{2m} + U(\rr, \dot\rr) \\
    H(q_i, p_i) & = \sum_{i}p_i\dot{q}_i - L
  \end{align}
\end{definition}
\begin{align}
  \diffp{H}{\pp} = \frac{\pp}{m}, \quad \diffp{H}{\rr} = \nabla U
\end{align}
ニュートンの運動方程式は $\pp = m\dot\rr$ より
\begin{align}
  \diff{\rr}{t} = \diffp{H}{\pp}, \quad \diff{\pp}{t} = -\diffp{H}{\rr}
\end{align}
電磁場において $\pp = m\dot\rr$ とはならない。


\subsection{変分原理}
ニュートンの運動の 3 法則の代わりに変分原理 (variational principle) あるいはハミルトンの原理 (Hamilton's principle) と呼ばれる次の原理が存在する。
\begin{axiom}[変分原理]
  物体の運動は作用が極値を取るような経路をたどる。
\end{axiom}
作用とは
\begin{align}
  S[q]                           & = \int_{t_1}^{t_2}L(q(t), \dot{q}(t), t)\dl{t}                                                 \\
  \frac{\delta S[q]}{\delta q_i} & = \frac{\partial L}{\partial q_i} - \frac{d}{dt}\ab(\frac{\partial L}{\partial \dot{q}_i}) = 0
\end{align}
ボールを投げたときに放物線を描くのはどんな運動もさまざまな軌道の中で等速に近くポテンシャルが高くなるような"ちょうどいい"軌道を選ぶ。

短いスケールではなく全体の動き
変分原理にトキメキを感じるくらい基礎のように見えたならあなたはセンスがあります。

ただトキメキを感じなければ運動方程式と数学的に同値な表現と思った方がいいです。一般の座標系にしただけで意味がない

\subsection{Noether の定理}

\begin{theorem}[Noether's theorem]
  無限小変換に対して
  \begin{align}
    t   & \to t' = t + \delta t       \\
    q_i & \to q_i' = q_i + \delta q_i
  \end{align}
  作用積分が不変に保たれるならば保存量 $\Phi$ が存在する。
  \begin{align}
    \Phi = p_i(\delta q_i - \dot{q}_i\delta t) + L\delta t
  \end{align}
\end{theorem}
\begin{proof}
  \begin{align}
    \delta S = S' - S & = \int_{t_1'}^{t_2'}L\ab(q_i', \diff{q_i'}{t'})\dl{t'} - \int_{t_1}^{t_2}L(q_i, \dot{q}_i)\dl{t}                                               \\
                      & = \int_{t_1}^{t_2}L\ab(q_i', \diff{q_i'}{t'})\diff{t'}{t}\dl{t} - \int_{t_1}^{t_2}L(q_i, \dot{q}_i)\dl{t}                                      \\
                      & = \int_{t_1}^{t_2}\ab(L\ab(q_i', \diff{q_i'}{t'}) - L(q_i, \dot{q}_i))\dl{t} + \int_{t_1}^{t_2}L\ab(q_i', \diff{q_i'}{t'})\dot{\delta t}\dl{t} \\
                      & = \int_{t_1}^{t_2}\ab(\delta L + L\ab(q_i, \dot{q}_i)\dot{\delta t})\dl{t}
  \end{align}
  任意の時間で $\delta S = 0$ となるから $\delta L + L\ab(q_i, \dot{q}_i)\dot{\delta t} = 0$ となる。
  \begin{align}
    \delta\dot{q}_i & = \diff{q_i'}{t'} - \dot{q}_i = \diff{t}{t'}\diff{q_i'}{t} - \dot{q}_i = (1 - \dot{\delta t})(\dot{q}_i + \dot{\delta q_i}) - \dot{q}_i = - \dot{q}_i\dot{\delta t} + \dot{\delta q_i}
  \end{align}
  \begin{align}
    \delta L                                      & = L\ab(q_i', \diff{q_i'}{t'}) - L(q_i, \dot{q}_i)                                                                                                                                                                                                      \\
                                                  & = \diffp{L}{q_i}\delta q_i + \diffp{L}{\dot{q}_i}\ab(- \dot{q}_i\dot{\delta t} + \dot{\delta q_i})                                                                                                                                                     \\
                                                  & = \ab(\diffp{L}{q_i} - \diff{}{t}\ab(\diffp{L}{\dot{q}_i}))\delta q_i + \diff{}{t}\ab(\diffp{L}{\dot{q}_i}(\delta q_i - \dot{q}_i\delta t)) + \diff{}{t}\ab(\diffp{L}{\dot{q}_i}\dot{q}_i)\delta t                                                     \\
    \delta L + L\ab(q_i, \dot{q}_i)\dot{\delta t} & = \diff{}{t}\ab(\diffp{L}{\dot{q}_i}(\delta q_i - \dot{q}_i\delta t) + L\delta t) + \diff{}{t}\ab(\diffp{L}{\dot{q}_i}\dot{q}_i)\delta t - \diff{L}{t}\delta t                                                                                         \\
                                                  & = \diff{}{t}\ab(\diffp{L}{\dot{q}_i}(\delta q_i - \dot{q}_i\delta t) + L\delta t) + \ab(\diff{}{t}\ab(\diffp{L}{\dot{q}_i})\dot{q}_i + \diffp{L}{\dot{q}_i}\ddot{q}_i)\delta t - \ab(\diffp{L}{q_i}\dot{q}_i + \diffp{L}{\dot{q}_i}\ddot{q}_i)\delta t \\
                                                  & = \diff{}{t}\ab(\diffp{L}{\dot{q}_i}(\delta q_i - \dot{q}_i\delta t) + L\delta t) - \ab(\diffp{L}{q_i} - \diff{}{t}\ab(\diffp{L}{\dot{q}_i}))\dot{q}_i\delta t                                                                                         \\
                                                  & = \diff{}{t}\ab(\diffp{L}{\dot{q}_i}(\delta q_i - \dot{q}_i\delta t) + L\delta t) = 0
  \end{align}
\end{proof}

\begin{example}
  微少量 $\varepsilon$ を用いて表現する。
  $p_i(\delta q_i - \dot{q}_i\delta t) + L\delta t$
  \begin{itemize}
    \item 並進対称性 $\delta q_i = \varepsilon$, $\delta t = 0$ より $p_i$ は保存量となる。
    \item 回転対称性 $\delta x_i = -\varepsilon y_i$, $\delta y_i = \varepsilon x_i$, $\delta t = 0$ より $x_ip_{y_i} - y_ip_{x_i} = (\rr\times\pp)_z = L_z$ は保存量となる。
    \item 時間推進対称性 $\delta q_i = 0$, $\delta t = -\varepsilon$ より $p_i\dot{q}_i - L = H$ は保存量となる。
  \end{itemize}
\end{example}


\subsection{正準変換}

\subsection{解析力学の復習:点正準変換}
ある $N$ 自由度の系の一般化座標を $q_1, \ldots, q_N$ として Lagrange 形式では一般化座標 $q_i$ と一般化速度 $\dot{q}_i$ を用いて表現される. このとき一般化運動量 $p_i$ は次のように定められる.
\begin{align}
  L   & = L(q_1,\ldots,q_N,\dot{q}_1,\ldots,\dot{q}_N),                                                                                     \\
  p_i & = \ab(\diffp{L}{\dot{q}_i})_{q_1,\ldots,q_N,\dot{q}_1,\ldots,\dot{q}_{i-1},\dot{q}_{i+1},\ldots,\dot{q}_N} \qquad (i = 1,\ldots,N).
\end{align}
一方 Hamilton 形式では一般化座標 $q_i$ と一般化運動量 $p_i$ を用いて表現される.
\begin{align}
  H             & = H(q_1,\ldots,q_N,p_1,\ldots,p_N) = \sum_{i=1}^{N}p_i\dot{q}_i - L,              \\
  \diff{q_i}{t} & = \diffp{H}{p_i}, \qquad \diff{p_i}{t} = -\diffp{H}{q_i} \qquad (i = 1,\ldots,N).
\end{align}

\begin{itembox}[l]{Q 17-1.}
  Lagrange 形式での一般座標変換 $(q_1,\ldots,q_N)\to(Q_1,\ldots,Q_N)$ に対応する Hamilton 形式で正準変換を点正準変換といい, $(q_1,\ldots,q_N,p_1,\ldots,p_N)\to(Q_1,\ldots,Q_N,P_1,\ldots,P_N)$ を求める.
  \begin{align}
    q_i = f_i(Q_1,\ldots,Q_N).
  \end{align}
\end{itembox}

(i) 新しい運動量 $P_j$ は Lagrange 形式を用いて次のように求められる.
\begin{align}
  P_j & = \ab(\diffp{L}{\dot{Q}_j})_{Q_1,\ldots,Q_N,\dot{Q}_1,\ldots,\dot{Q}_{j-1},\dot{Q}_{j+1},\ldots,\dot{Q}_N} \qquad (j=1,2,\ldots,N) \\
      & = \sum_{i=1}^{N}\diffp{L}{\dot{q}_i}\diffp{\dot{q}_i}{\dot{Q}_j}                                                                   \\
      & = \sum_{i=1}^{N}p_i\diffp{q_i}{Q_j}                                                                                                \\
      & = \sum_{i=1}^{N}\diffp{f_i(Q_1,\ldots,Q_N)}{Q_j}p_i.
\end{align}

(ii) また新しい Hamilton 関数は定義式から古い Hamilton 関数と一致する.
\begin{align}
  H' = H'(Q_1,\ldots,Q_N,P_1,\ldots,P_N) = \sum_{j=1}^{N}P_j\dot{Q}_j - L = \sum_{j=1}^{N}\sum_{i=1}^{N}\diffp{f_i(Q_1,\ldots,Q_N)}{Q_j}p_i\dot{Q}_j - L = \sum_{i=1}^{N}p_i\dot{q}_i - L = H
\end{align}

\subsection{1 次元結晶における平衡位置の回りの調和振動を記述する Hamilton 関数}
直線上に等間隔の平衡位置を持って並んだ $N$ 個の原子からなる 1 次元結晶を物理系として記述して古典力学により考察する. $i$ 番目の原子の位置座標の平衡位置からのずれを $q_i$ として, その運動量を $p_i$ とする.
\begin{itembox}[l]{Q 17-2.}
  1 次元結晶の Hamilton 関数は次のように表される.
  \begin{align}
    H^{1次元結晶}(q_1,\ldots,q_N, p_1,\ldots,p_N) & := \frac{1}{2m}\sum_{i=1}^{N}p_i^2 + \frac{1}{2}\kappa\sum_{i=0}^{N}(q_i - q_{i+1})^2
  \end{align}
  ただし $\kappa$ は隣り合った原子の間の原子間力のバネ定数とし, 両端の原子は固定されている $q_0 = q_{N+1} = 0$ と仮定する.
\end{itembox}

$i$ 番目の原子の運動エネルギーは運動量 $p_i$ を用いて次のように表される.
\begin{align}
  \frac{p_i^2}{2m}.
\end{align}
また隣り合う $i, i+1$ 番目の原子の原子間力のポテンシャルエネルギーはバネ定数 $\kappa$ を用いて次のように表される.
\begin{align}
  \frac{1}{2}\kappa(q_i - q_{i+1})^2.
\end{align}
これより Hamilton 関数は次のように表される.
\begin{align}
  H^{1次元結晶}(q_1,\ldots,q_N, p_1,\ldots,p_N) & := \frac{1}{2m}\sum_{i=1}^{N}p_i^2 + \frac{1}{2}\kappa\sum_{i=0}^{N}(q_i - q_{i+1})^2.
\end{align}

\subsection{1 次元結晶における平衡位置の回りの調和振動の基準モードの計算}

\begin{itembox}[l]{Q 17-3.}
  固定端境界条件の 1 次元結晶の系を考えているので Fourier 展開した基底が基準振動となる.
  \begin{align}
    H^{1次元結晶}(Q_1,\ldots,Q_N, P_1,\ldots,P_N) & = \sum_{j=1}^{N}\ab(\frac{1}{2m}P_j^2 + \frac{1}{2}m\omega_j^2Q_j^2).
  \end{align}
  ただし, $\omega_j$ を次のように定める.
  \begin{align}
    \omega_j = 2\sqrt{\frac{\kappa}{m}}\sin\ab(\frac{\pi}{2(N+1)}j).
  \end{align}
\end{itembox}

固定端境界条件の 1 次元結晶の系を考えているので Fourier Sine 展開の基底が基準振動になっているとする.
\begin{align}
  q_i^{(j)} & = \sqrt{\frac{2}{N+1}}\sin\ab(\frac{\pi}{N+1}ji).
\end{align}
まず計算に必要な関数を定義する. \\

(i) $\alpha \neq 0 \pmod{2\pi}$ に対して $F(\alpha), G(\alpha)$ を次のように定義する.
\begin{align}
  F(\alpha) & := \sum_{i=1}^{N}\cos(\alpha i), \\
  G(\alpha) & := \sum_{i=1}^{N}\sin(\alpha i).
\end{align}
このとき $F(\alpha), G(\alpha)\in\RR$ より $F(\alpha) + \sqrt{-1}G(\alpha)\in\CC$ の実部と虚部はそれぞれ $F(\alpha), G(\alpha)$ と対応した値となる. Euler の公式を用いて次のように計算できる.
\begin{align}
  F(\alpha) + \sqrt{-1}G(\alpha) & = \sum_{i=1}^{N}e^{\sqrt{-1}\alpha i}                                                                                                                                                   \\
                                 & = \frac{e^{\sqrt{-1}\alpha} - e^{\sqrt{-1}\alpha (N+1)}}{1 - e^{\sqrt{-1}\alpha}}                                                                                                       \\
                                 & = \frac{2e^{\sqrt{-1}\alpha}e^{\sqrt{-1}\alpha \frac{N}{2}}\sin{\alpha \frac{N}{2}}}{2e^{\sqrt{-1}\alpha\frac{1}{2}}\sin{\alpha\frac{1}{2}}}                                            \\
                                 & = \frac{e^{\sqrt{-1}\frac{\alpha}{2}(N+1)}\sin{\frac{\alpha}{2}N}}{\sin{\frac{\alpha}{2}}}                                                                                              \\
                                 & = \frac{\cos\ab(\frac{\alpha}{2}(N+1))\sin{\frac{\alpha}{2}N}}{\sin{\frac{\alpha}{2}}} + \sqrt{-1}\frac{\sin\ab(\frac{\alpha}{2}(N+1))\sin{\frac{\alpha}{2}N}}{\sin{\frac{\alpha}{2}}}.
\end{align}
これより実部虚部の対応から $F(\alpha), G(\alpha)$ が求まる.
\begin{align}
  F(\alpha) & := \sum_{i=1}^{N}\cos(\alpha i) = \frac{\cos\ab(\frac{\alpha}{2}(N+1))\sin\ab(\frac{\alpha}{2}N)}{\sin{\frac{\alpha}{2}}}, \\
  G(\alpha) & := \sum_{i=1}^{N}\sin(\alpha i) = \frac{\sin\ab(\frac{\alpha}{2}(N+1))\sin\ab(\frac{\alpha}{2}N)}{\sin{\frac{\alpha}{2}}}.
\end{align}

(ii) $j, j' = 1,\ldots,N$ とすると $j - j' = -(N - 1),\ldots,N - 1$ かつ $j + j' = 2,\ldots,2N$ である. これより $j - j' = 0$ である場合に限り $j - j' = 0 \pmod{2(N+1)}$ が成り立ち, $j + j' = 0 \pmod{2(N+1)}$ が成り立つ場合は存在せず, 逆に主結合子の前件が恒偽ならばその論理式は真である. よって次の同値関係が成り立つ.
\begin{align}
   & \frac{\pi}{N+1}(j - j') = 0 \pmod{2\pi} \iff j - j' = 0 \pmod{2(N+1)} \iff j = j', \label{Q17-3. ii-1} \\
   & \frac{\pi}{N+1}(j + j') = 0 \pmod{2\pi} \iff j + j' = 0 \pmod{2(N+1)} \iff false. \label{Q17-3. ii-2}
\end{align}

(iii) $j, j' = 1,\ldots,N$ に対して次のように内積を定義する. このときこの内積の正規直交関係を示す.
\begin{align}
  (q^{(j)}, q^{(j')}) & := \sum_{i = 1}^{N}q_i^{(j)}q_i^{(j')}.
\end{align}
まず (i), (ii) を用いることで次のように式変形できる.
\begin{align}
  (q^{(j)}, q^{(j')}) & := \sum_{i = 1}^{N}q_i^{(j)}q_i^{(j')}                                                                                                                                                                                                                                             \\
                      & = \frac{2}{N+1}\sum_{i = 1}^{N}\sin\ab(\frac{\pi}{N+1}ji)\sin\ab(\frac{\pi}{N+1}j'i)                                                                                                                                                                                               \\
                      & = \frac{1}{N+1}\sum_{i = 1}^{N}\ab(\cos\ab(\frac{\pi}{N+1}(j - j')i) - \cos\ab(\frac{\pi}{N+1}(j + j')i))                                                                                                                                                                          \\
                      & = \begin{dcases}
                            \frac{1}{N+1}\ab(\frac{\cos\ab(\frac{\pi}{2}(j - j'))\sin\ab(\frac{N\pi}{2(N+1)}(j - j'))}{\sin\ab(\frac{\pi}{2(N+1)}(j - j'))} - \frac{\cos\ab(\frac{\pi}{2}(j + j'))\sin\ab(\frac{N\pi}{2(N+1)}(j + j'))}{\sin\ab(\frac{\pi}{2(N+1)}(j + j'))}) & (j \neq j') \\
                            \frac{1}{N+1}\ab(N - \frac{\cos\ab(j\pi)\sin\ab(\frac{jN}{N+1}\pi)}{\sin\ab(\frac{j}{N+1}\pi)})                                                                                                                                                   & (j = j')
                          \end{dcases}.
\end{align}
先に $j \neq j'$ の場合を考える. 括弧内を通分した分子の第一項と第二項についてそれぞれ計算する. 第一項について
\begin{align}
    & \cos\ab(\frac{\pi}{2}(j - j'))\sin\ab(\frac{N\pi}{2(N+1)}(j - j'))\sin\ab(\frac{\pi}{2(N+1)}(j + j'))                                                      \\
  = & \cos\ab(\frac{j - j'}{2}\pi)\ab(\cos\ab(\frac{(N-1)j - (N+1)j'}{2(N+1)}\pi) - \cos\ab(\frac{(N+1)j - (N-1)j'}{2(N+1)}\pi))                                 \\
  = & \cos\ab(\frac{j - j'}{2}\pi)\cos\ab(\frac{(N-1)j - (N+1)j'}{2(N+1)}\pi) - \cos\ab(\frac{j - j'}{2}\pi)\cos\ab(\frac{(N+1)j - (N-1)j'}{2(N+1)}\pi)          \\
  = & \cos\ab(\frac{j}{N+1}\pi) + \cos\ab(\frac{Nj - (N+1)j'}{N+1}\pi) - \cos\ab(\frac{j'}{N+1}\pi) - \cos\ab(\frac{(N+1)j - Nj'}{N+1}\pi). \label{Q17-3. iii 1}
\end{align}
第二項について
\begin{align}
    & \cos\ab(\frac{\pi}{2}(j + j'))\sin\ab(\frac{N\pi}{2(N+1)}(j + j'))\sin\ab(\frac{\pi}{2(N+1)}(j - j'))                                                      \\
  = & \cos\ab(\frac{j + j'}{2}\pi)\ab(\cos\ab(\frac{(N-1)j + (N+1)j'}{2(N+1)}\pi) - \cos\ab(\frac{(N+1)j + (N-1)j'}{2(N+1)}\pi))                                 \\
  = & \cos\ab(\frac{j + j'}{2}\pi)\cos\ab(\frac{(N-1)j + (N+1)j'}{2(N+1)}\pi) - \cos\ab(\frac{j + j'}{2}\pi)\cos\ab(\frac{(N+1)j + (N-1)j'}{2(N+1)}\pi)          \\
  = & \cos\ab(\frac{Nj + (N+1)j'}{N+1}\pi) + \cos\ab(\frac{j}{N+1}\pi) - \cos\ab(\frac{(N+1)j + Nj'}{N+1}\pi) - \cos\ab(\frac{j'}{N+1}\pi). \label{Q17-3. iii 2}
\end{align}
これより分子は次のようになる.
\begin{align}
  \eqref{Q17-3. iii 1} - \eqref{Q17-3. iii 2} & = \ab(\cos\frac{j}{N+1}\pi + \cos\ab(\frac{Nj}{N+1} - j')\pi - \cos\frac{j'}{N+1}\pi - \cos\ab(j - \frac{Nj'}{N+1})\pi)                 \\
                                              & - \ab(\cos\ab(\frac{Nj}{N+1} + j')\pi + \cos\frac{j}{N+1}\pi - \cos\ab(j + \frac{Nj'}{N+1})\pi - \cos\frac{j'}{N+1}\pi)                 \\
                                              & = \cos\ab(\frac{Nj}{N+1} - j')\pi - \cos\ab(\frac{Nj}{N+1} + j')\pi + \cos\ab(j + \frac{Nj'}{N+1})\pi - \cos\ab(j - \frac{Nj'}{N+1})\pi \\
                                              & = 2\sin\ab(j'\pi)\sin\ab(\frac{Nj}{N+1}\pi) - 2\sin\ab(j\pi)\sin\ab(\frac{Nj'}{N+1}\pi)                                                 \\
                                              & = 0 \qquad (\because j, j'\in\ZZ).
\end{align}
よって $j \neq j'$ のときは $(q^{(j)}, q^{(j')}) = 0$ となる.

次に $j = j'$ の場合を考える. これは $j$ が奇数か偶数かで場合分けして考える.
\begin{align}
  \frac{\cos\ab(j\pi)\sin\ab(\frac{jN}{N+1}\pi)}{\sin\ab(\frac{j}{N+1}\pi)} & =
  \begin{dcases}
    \frac{\cos\ab(2k\pi)\sin\ab(\frac{2kN}{N+1}\pi)}{\sin\ab(\frac{2k}{N+1}\pi)}           & (j = 2k, k\in\ZZ)   \\
    \frac{\cos\ab((2k-1)\pi)\sin\ab(\frac{(2k-1)N}{N+1}\pi)}{\sin\ab(\frac{2k-1}{N+1}\pi)} & (j = 2k-1, k\in\ZZ)
  \end{dcases} \\ & =
  \begin{dcases}
    \frac{1\cdot\sin\ab(2k\pi\frac{N}{N+1} - 2k\pi)}{\sin\ab(2k\pi\frac{1}{N+1})} \\
    \frac{-1 \cdot -\sin\ab((2k-1)\pi\frac{N}{N+1} - (2k-1)\pi)}{\sin\ab((2k-1)\pi\frac{1}{N+1})}
  \end{dcases}           \\
                                                                            & = -1.
\end{align}
よって $j = j'$ のときは $(q^{(j)}, q^{(j')}) = 1$ となる. これより, まとめると次の式が成り立つ.
\begin{align}
  (q^{(j)}, q^{(j')}) = \delta_{j,j'}.
\end{align}


(iv) ここで行列 $A_{ij} := q_i^{(j)}$ を定義する. このとき次の計算から $A_{ij}$ は直交行列であるとわかる.
\begin{align}
  (A^{\top}A)_{ij} & = \sum_{k=1}^{N}A_{ik}^\top A_{kj} = \sum_{k=1}^{N}A_{ki}A_{kj} = \sum_{k=1}^{N}q_k^{(i)}q_k^{(j)} = (q^{(i)}, q^{(j)}) = \delta_{i,j}.
\end{align}

(v) また $A_{ij}$ が直交行列であるから次のような正規直交関係もある.
\begin{align}
  (AA^{\top})_{ij} & = \sum_{k=1}^{N}A_{ik}A_{kj}^{\top} = \sum_{k=1}^{N}A_{ik}A_{jk} = \sum_{k=1}^{N}q_i^{(k)}q_j^{(k)} = \delta_{i,j}.
\end{align}

(vi) ここで原子の変位を表す古い座標系 $q_1, \ldots, q_N$ を $q^{(1)}, \ldots, q^{(N)}$ で離散 Fourier Sine 展開した振幅を新しい座標系 $Q_1, \ldots, Q_N$ と定義する.
\begin{align}
  q_i = \sum_{j=1}^{N}Q_jq_i^{(j)}.
\end{align}
これは点正準変換を用いて新しい運動量を古い運動量を表せられる.
\begin{align}
  P_j = \sum_{i=1}^{N}\diffp{q_i}{Q_j}p_i = \sum_{i=1}^{N}q_i^{(j)}p_i.
\end{align}

(vii) Hamilton 関数の運動エネルギーの表式の核の部分について次のように表される.
\begin{align}
  \sum_{j=1}^{N}P_j^2 = \sum_{j=1}^{N}\ab(\sum_{i=1}^{N}q_i^{(j)}p_i)^2 = \sum_{j=1}^{N}\sum_{i=1}^{N}\sum_{i'=1}^{N}(q_i^{(j)}p_i)(q_{i'}^{(j)}p_{i'}) = \sum_{i=1}^{N}p_i^2.
\end{align}

(viii) Hamilton 関数のポテンシャルエネルギーの核の部分について次のような表される.
\begin{align}
  \sum_{i=0}^{N}(q_i - q_{i+1})^2 & = \sum_{i=0}^{N}\ab(\sum_{j=1}^{N}\ab(Q_jq_i^{(j)} - Q_jq_{i+1}^{(j)}))^2                                                     \\
                                  & = \sum_{i=0}^{N}\sum_{j=1}^{N}\sum_{j'=1}^{N}\ab(Q_jq_i^{(j)} - Q_jq_{i+1}^{(j)})\ab(Q_{j'}q_i^{(j')} - Q_{j'}q_{i+1}^{(j')}) \\
                                  & = \sum_{j=1}^{N}\sum_{j'=1}^{N}\sum_{i=0}^{N}(q_i^{(j)} - q_{i+1}^{(j)})(q_i^{(j')} - q_{i+1}^{(j')})Q_jQ_{j'}                \\
                                  & = \sum_{j=1}^{N}\sum_{j'=1}^{N}B_{j,j'}Q_jQ_{j'}.
\end{align}
ただし, $B_{j,j'}$ を次のように定める.
\begin{align}
  B_{j,j'} := \sum_{i=0}^{N}(q_i^{(j)} - q_{i+1}^{(j)})(q_i^{(j')} - q_{i+1}^{(j')}).
\end{align}

(ix) 次に $B_{j,j'}$ を求める. まず $q_i^{(j)} - q_{i+1}^{(j)}$ は次のように求められる.
\begin{align}
  q_i^{(j)} - q_{i+1}^{(j)} & = \sqrt{\frac{2}{N+1}}\sin\ab(\frac{\pi}{N+1}ji) - \sqrt{\frac{2}{N+1}}\sin\ab(\frac{\pi}{N+1}j(i+1)) \\
                            & = \sqrt{\frac{2}{N+1}}\ab(\sin\ab(\frac{\pi}{N+1}ji) - \sin\ab(\frac{\pi}{N+1}j(i+1)))                \\
                            & = -2\sqrt{\frac{2}{N+1}}\cos\ab(\frac{\pi}{2}\frac{(2i+1)j}{N+1})\sin\ab(\frac{\pi}{2}\frac{j}{N+1}).
\end{align}

(x) これより $B_{j,j'}$ は次のように計算できる.
\begin{align}
  B_{j,j'} & = \sum_{i=0}^{N}(q_i^{(j)} - q_{i+1}^{(j)})(q_i^{(j')} - q_{i+1}^{(j')})                                                                                                                                                         \\
           & = \sum_{i=0}^{N}\ab(-2\sqrt{\frac{2}{N+1}}\cos\ab(\frac{\pi}{2}\frac{(2i+1)j}{N+1})\sin\ab(\frac{\pi}{2}\frac{j}{N+1}))\ab(-2\sqrt{\frac{2}{N+1}}\cos\ab(\frac{\pi}{2}\frac{(2i+1)j'}{N+1})\sin\ab(\frac{\pi}{2}\frac{j'}{N+1})) \\
           & = 4\sin\ab(\frac{\pi}{2}\frac{j}{N+1})\sin\ab(\frac{\pi}{2}\frac{j'}{N+1})\frac{2}{N+1}\sum_{i=0}^{N}\cos\ab(\frac{\pi}{N+1}j\ab(i + \frac{1}{2}))\cos\ab(\frac{\pi}{N+1}j'\ab(i + \frac{1}{2}))                                 \\
           & = 4\sin\ab(\frac{\pi}{2}\frac{j}{N+1})\sin\ab(\frac{\pi}{2}\frac{j'}{N+1})\frac{1}{N+1}\sum_{i=0}^{N}\ab(\cos\ab(\frac{\pi}{N+1}(j + j')\ab(i + \frac{1}{2})) + \cos\ab(\frac{\pi}{N+1}(j - j')\ab(i + \frac{1}{2})))            \\
           & = 4\sin\ab(\frac{\pi}{2}\frac{j}{N+1})\sin\ab(\frac{\pi}{2}\frac{j'}{N+1})\tilde{B}_{j,j'}.
\end{align}
ただし, $\tilde{B}_{j,j'}$ を次のように定める.
\begin{align}
  \tilde{B}_{j,j'} & := \frac{1}{N+1}\sum_{i=0}^{N}\ab(\cos\ab(\frac{\pi}{N+1}(j + j')\ab(i + \frac{1}{2})) + \cos\ab(\frac{\pi}{N+1}(j - j')\ab(i + \frac{1}{2}))).
\end{align}

(xi) さらに $\tilde{B}_{j,j'}$ は次のように計算できる.
\begin{align}
  \tilde{B}_{j,j'} & = \frac{1}{N+1}\sum_{i=0}^{N}\ab(\cos\ab(\pi\frac{j + j'}{N+1}\ab(i + \frac{1}{2})) + \cos\ab(\pi\frac{j - j'}{N+1}\ab(i + \frac{1}{2}))) \\
                   & \ \begin{aligned}
                         = \frac{1}{N+1}\sum_{i=0}^{N}\bigg[ & \quad\cos\ab(\frac{\pi}{2}\frac{j + j'}{N+1})\cos\ab(\pi\frac{j + j'}{N+1}i)    \\
                                                             & - \sin\ab(\frac{\pi}{2}\frac{j + j'}{N+1})\sin\ab(\pi\frac{j + j'}{N+1}i)       \\
                                                             & + \cos\ab(\frac{\pi}{2}\frac{j - j'}{N+1})\cos\ab(\pi\frac{j - j'}{N+1}i)       \\
                                                             & - \sin\ab(\frac{\pi}{2}\frac{j - j'}{N+1})\sin\ab(\pi\frac{j - j'}{N+1}i)\bigg]
                       \end{aligned}                                    \\
                   & \ \begin{aligned}
                         = \frac{1}{N+1}\bigg[ & \quad\cos\ab(\frac{\pi}{2}\frac{j + j'}{N+1})\ab(1 + F\ab(\pi\frac{j + j'}{N+1})) \\
                                               & - \sin\ab(\frac{\pi}{2}\frac{j + j'}{N+1})G\ab(\pi\frac{j + j'}{N+1})             \\
                                               & + \cos\ab(\frac{\pi}{2}\frac{j - j'}{N+1})\ab(1 + F\ab(\pi\frac{j - j'}{N+1}))    \\
                                               & - \sin\ab(\frac{\pi}{2}\frac{j - j'}{N+1})G\ab(\pi\frac{j - j'}{N+1})\bigg]
                       \end{aligned}.
\end{align}

(xii) まず $\tilde{B}_{j,j'}$ について $j = j'$ の場合を考える.
\begin{align}
  \tilde{B}_{j,j'} & = \tilde{B}_{j,j}                                                                                                                                                                                                                            \\
                   & = \frac{1}{N+1}\ab[\cos\ab(\frac{1}{N+1}j\pi)\ab(1 + F\ab(\frac{2}{N+1}j\pi)) - \sin\ab(\frac{1}{N+1}j\pi)G\ab(\frac{2}{N+1}j\pi) + \ab(1 + N) - 0]                                                                                          \\
                   & = 1 + \frac{1}{N+1}\ab(\cos\ab(\frac{1}{N+1}j\pi)\ab(1 + \frac{\cos\ab(j\pi)\sin\ab(\frac{N}{N+1}j\pi)}{\sin\ab(\frac{1}{N+1}j\pi)}) - \sin\ab(\frac{1}{N+1}j\pi)\frac{\sin\ab(j\pi)\sin\ab(\frac{N}{N+1}j\pi)}{\sin\ab(\frac{1}{N+1}j\pi)}) \\
                   & = 1 + \frac{1}{N+1}\ab(\cos\ab(\frac{1}{N+1}j\pi) + \ab(\cos\ab(\frac{1}{N+1}j\pi)\cos\ab(j\pi) - \sin\ab(\frac{1}{N+1}j\pi)\sin\ab(j\pi))\frac{\sin\ab(\frac{N}{N+1}j\pi)}{\sin\ab(\frac{1}{N+1}j\pi)})                                     \\
                   & = 1 + \frac{1}{N+1}\ab(\cos\ab(\frac{1}{N+1}j\pi) + \cos\ab(\frac{N+2}{N+1}j\pi)\frac{\sin\ab(\frac{N}{N+1}j\pi)}{\sin\ab(\frac{1}{N+1}j\pi)})                                                                                               \\
                   & = 1 + \frac{1}{N+1}\ab(\cos\ab(\frac{1}{N+1}j\pi)\sin\ab(\frac{1}{N+1}j\pi) + \cos\ab(\frac{N+2}{N+1}j\pi)\sin\ab(\frac{N}{N+1}j\pi))\bigg/\sin\ab(\frac{1}{N+1}j\pi)                                                                        \\
                   & = 1 + \frac{1}{N+1}\ab(\frac{1}{2}\sin\ab(\frac{2}{N+1}j\pi) + \frac{1}{2}\sin\ab(-\frac{2}{N+1}j\pi))\bigg/\sin\ab(\frac{1}{N+1}j\pi)                                                                                                       \\
                   & = 1.
\end{align}

(xiii) 次に $\tilde{B}_{j,j'}$ について $j \neq j'$ の場合を考える.
\begin{align}
  \tilde{B}_{j,j'} & = \tilde{B}_{j,j'}                                                                                                                                                                      \\
                   & \ \begin{aligned}
                         = \frac{1}{N+1}\bigg[ & \quad\cos\ab(\frac{\pi}{2}\frac{j + j'}{N+1})\ab(1 + F\ab(\pi\frac{j + j'}{N+1})) \\
                                               & - \sin\ab(\frac{\pi}{2}\frac{j + j'}{N+1})G\ab(\pi\frac{j + j'}{N+1})             \\
                                               & + \cos\ab(\frac{\pi}{2}\frac{j - j'}{N+1})\ab(1 + F\ab(\pi\frac{j - j'}{N+1}))    \\
                                               & - \sin\ab(\frac{\pi}{2}\frac{j - j'}{N+1})G\ab(\pi\frac{j - j'}{N+1})\bigg]
                       \end{aligned}                                                                                             \\
                   & \ \begin{aligned}
                         = \frac{1}{N+1}\Bigg[ & \quad\cos\ab(\frac{\pi}{2}\frac{j + j'}{N+1})\ab(1 + \frac{\cos\ab(\frac{1}{2}(j+j')\pi)\sin\ab(\frac{N}{2(N+1)}(j+j')\pi)}{\sin\ab(\frac{1}{2(N+1)}(j+j')\pi)}) \\
                                               & - \sin\ab(\frac{\pi}{2}\frac{j + j'}{N+1})\frac{\sin\ab(\frac{1}{2}(j+j')\pi)\sin\ab(\frac{N}{2(N+1)}(j+j')\pi)}{\sin\ab(\frac{1}{2(N+1)}(j+j')\pi)}             \\
                                               & + \cos\ab(\frac{\pi}{2}\frac{j - j'}{N+1})\ab(1 + \frac{\cos\ab(\frac{1}{2}(j-j')\pi)\sin\ab(\frac{N}{2(N+1)}(j-j')\pi)}{\sin\ab(\frac{1}{2(N+1)}(j-j')\pi)})    \\
                                               & - \sin\ab(\frac{\pi}{2}\frac{j - j'}{N+1})\frac{\sin\ab(\frac{1}{2}(j-j')\pi)\sin\ab(\frac{N}{2(N+1)}(j-j')\pi)}{\sin\ab(\frac{1}{2(N+1)}(j-j')\pi)}\Bigg]
                       \end{aligned}                                                                                              \\
                   & \ \begin{aligned}
                         = \frac{1}{N+1}\Bigg[ & \quad\cos\ab(\frac{\pi}{2}\frac{j + j'}{N+1}) + \ab(\cos\ab(\frac{\pi}{2}\frac{j + j'}{N+1})\cos\ab(\frac{j+j'}{2}\pi) - \sin\ab(\frac{\pi}{2}\frac{j + j'}{N+1})\sin\ab(\frac{j+j'}{2}\pi))\frac{\sin\ab(\frac{N(j+j')}{2(N+1)}\pi)}{\sin\ab(\frac{j+j'}{2(N+1)}\pi)}    \\
                                               & + \cos\ab(\frac{\pi}{2}\frac{j - j'}{N+1}) + \ab(\cos\ab(\frac{\pi}{2}\frac{j - j'}{N+1})\cos\ab(\frac{j-j'}{2}\pi) - \sin\ab(\frac{\pi}{2}\frac{j - j'}{N+1})\sin\ab(\frac{j-j'}{2}\pi))\frac{\sin\ab(\frac{N(j-j')}{2(N+1)}\pi)}{\sin\ab(\frac{j-j'}{2(N+1)}\pi)}\Bigg]
                       \end{aligned}                                                                          \\
                   & \ \begin{aligned}
                         = \frac{1}{N+1}\Bigg[ & \quad\cos\ab(\frac{\pi}{2}\frac{j + j'}{N+1}) + \cos\ab(\frac{N+2}{2(N+1)}(j + j')\pi)\frac{\sin\ab(\frac{N}{2(N+1)}(j+j')\pi)}{\sin\ab(\frac{1}{2(N+1)}(j+j')\pi)}    \\
                                               & + \cos\ab(\frac{\pi}{2}\frac{j - j'}{N+1}) + \cos\ab(\frac{N+2}{2(N+1)}(j - j')\pi)\frac{\sin\ab(\frac{N}{2(N+1)}(j-j')\pi)}{\sin\ab(\frac{1}{2(N+1)}(j-j')\pi)}\Bigg]
                       \end{aligned}                                                                      \\
                   & \ \begin{aligned}
                         = \frac{1}{N+1}\Bigg[ & \quad\frac{1}{2}\ab(\sin\ab(\frac{j + j'}{N+1}\pi) + \sin\ab((j + j')\pi) + \sin\ab(-\frac{j+j'}{N+1}\pi))\bigg/\sin\ab(\frac{1}{2(N+1)}(j+j')\pi)    \\
                                               & + \frac{1}{2}\ab(\sin\ab(\frac{j - j'}{N+1}\pi) + \sin\ab((j - j')\pi) + \sin\ab(-\frac{j-j'}{N+1}\pi))\bigg/\sin\ab(\frac{1}{2(N+1)}(j-j')\pi)\bigg]
                       \end{aligned} \\
                   & = 0.
\end{align}
よって (xii), (xiii) の考察から次の式が成り立つ.
\begin{align}
  \tilde{B}_{j,j'} = \delta_{j,j'}.
\end{align}

(xiv) これより $B_{j,j'}$ は (x) の考察から次のようになる.
\begin{align}
  B_{j,j'} & = 4\sin\ab(\frac{\pi}{2}\frac{j}{N+1})\sin\ab(\frac{\pi}{2}\frac{j'}{N+1})\tilde{B}_{j,j'} \\
           & = \delta_{j,j'}4\sin^2\ab(\frac{\pi}{2(N+1)}j).
\end{align}

(xv) ポテンシャルエネルギーの表式 (vii) に代入して次のようになる.
\begin{align}
  \sum_{i=0}^{N}(q_i - q_{i+1})^2 & = \sum_{j=1}^{N}\sum_{j'=1}^{N}B_{j,j'}Q_jQ_{j'}                                     \\
                                  & = \sum_{j=1}^{N}\sum_{j'=1}^{N}\delta_{j,j'}4\sin^2\ab(\frac{\pi}{2(N+1)}j)Q_jQ_{j'} \\
                                  & = 4\sum_{j=1}^{N}\sin^2\ab(\frac{\pi}{2(N+1)}j)Q_j^2.
\end{align}

(xvi) よって Hamilton 関数は (vii) (xv) から次のように表される.
\begin{align}
  H^{1次元結晶}(q_1,\ldots,q_N, p_1,\ldots,p_N) & = \frac{1}{2m}\sum_{i=1}^{N}p_i^2 + \frac{1}{2}\kappa\sum_{i=0}^{N}(q_i - q_{i+1})^2         \\
                                            & = \frac{1}{2m}\sum_{j=1}^{N}P_j^2 + 2\kappa\sum_{j=1}^{N}\sin^2\ab(\frac{\pi}{2(N+1)}j)Q_j^2 \\
  H^{1次元結晶}(Q_1,\ldots,Q_N, P_1,\ldots,P_N) & = \sum_{j=1}^{N}\ab(\frac{1}{2m}P_j^2 + \frac{1}{2}m\omega_j^2Q_j^2).
\end{align}
ただし, $\omega_j$ を次のように定めた.
\begin{align}
  \omega_j = 2\sqrt{\frac{\kappa}{m}}\sin\ab(\frac{\pi}{2(N+1)}j) \qquad (j = 1,\ldots,N).
\end{align}

\begin{itembox}[l]{Q 17-4.}
  1 次元結晶中の波数 $k$ に対する分散関係 $\omega(k)$ は次のようになる.
  \begin{align}
    \omega(k) & = 2\sqrt{\frac{\kappa}{m}}\sin\ab(\frac{1}{2}ka) \approx \sqrt{\frac{\kappa}{m}}ka + \mathcal{O}((ka)^3) \qquad (ka\ll 1).
  \end{align}
\end{itembox}

(i) $j = 1,\ldots,N$ に対して $j$ 番目の基準振動 $q_i^{(j)}$ は次のように計算される.
\begin{align}
  q_i^{(j)} & = \sqrt{\frac{2}{N+1}}\sin\ab(\frac{\pi}{N+1}ji)             \\
            & = \sqrt{\frac{2}{N+1}}\sin\ab(\frac{\pi}{a}\frac{j}{N+1}x_i) \\
            & = \sqrt{\frac{2}{N+1}}\sin\ab(k_jx_i).
\end{align}
ただし, $i$ 番目の原子の平衡位置の座標を $x_i = ai$ とし, $j$ 番目の基準振動の波数 $k_j$ を次のように定める.
\begin{align}
  k_j := \frac{\pi}{a}\frac{j}{N+1} \qquad (j = 1,\ldots,N).
\end{align}

(ii) 基準振動 $q_i^{(j)}$ の角振動数 $\omega_j$ を波数 $k_j$ の関数として次のように表される.
\begin{align}
  \omega(k_j) & = 2\sqrt{\frac{\kappa}{m}}\sin\ab(\frac{\pi}{2(N+1)}j) \\
              & = 2\sqrt{\frac{\kappa}{m}}\sin\ab(\frac{1}{2}k_ja).
\end{align}
よって分散関係 $\omega = \omega(k)$ は次のように与えられる.
\begin{align}
  \omega(k) & = 2\sqrt{\frac{\kappa}{m}}\sin\ab(\frac{1}{2}ka).
\end{align}

(iii) この 1 次元結晶を伝わる線形波動 (弾性波, 音波) が波数ごとに異なる速さを持って伝播するということから, 1次元結晶中にこれらを重ね合わせて波束が作られたとすると次第に波束の形が変化していき最終的に崩壊する.

(iv) 十分に長波長 $ka\ll 1$ のとき次のように近似することで分散関係 $\omega(k)$ は線形関係となる.
\begin{align}
  \omega(k) & = 2\sqrt{\frac{\kappa}{m}}\sin\ab(\frac{1}{2}ka)                         \\
            & \approx 2\sqrt{\frac{\kappa}{m}}\ab(\frac{1}{2}ka + \mathcal{O}((ka)^3)) \\
            & = \sqrt{\frac{\kappa}{m}}ka + \mathcal{O}((ka)^3) \qquad (ka\ll 1).
\end{align}

(v) 長波長の極限での弾性波の速さを音速という. 固体の音速 $v$ は次のようになる.
\begin{align}
  v & = \lim_{ka\to 0}\frac{\omega(k)}{k} = \sqrt{\frac{\kappa}{m}}a.
\end{align}

(vi) (iv), (v) の考察より十分に長波長のとき分散関係が線形関係となるので 1 次元結晶中では線形波動は音速 $v$ と等しい速さを持って伝搬する.

\begin{itembox}[l]{Q 17-5.}
  1 次元結晶における基準振動の角振動数 $\omega_j$ の分布を明らかにする.
\end{itembox}

(i)(ii) $\omega_j$ は次のように表されることから $j=1,\ldots,N$ に対して単調増加となる.
\begin{align}
  \omega_j & = 2\sqrt{\frac{\kappa}{m}}\sin\ab(\frac{\pi}{2(N+1)}j).
\end{align}
これより $\omega_j$ の最大値と最小値は次のようになる.
\begin{align}
  \omega_{\max} & := \max_{1\leq j\leq N}\omega_j = \omega_N = 2\sqrt{\frac{\kappa}{m}}\sin\ab(\frac{\pi N}{2(N+1)}) \approx 2\sqrt{\frac{\kappa}{m}},                                                          \\
  \omega_{\min} & := \min_{1\leq j\leq N}\omega_j = \omega_1 = 2\sqrt{\frac{\kappa}{m}}\sin\ab(\frac{\pi}{2(N+1)}) \approx 2\sqrt{\frac{\kappa}{m}}\frac{\pi}{2(N+1)} = \sqrt{\frac{\kappa}{m}}\frac{\pi}{N+1}.
\end{align}
\subsection{3 次元結晶における平衡位置の回りの調和振動を記述する Hamilton 関数}
立方格子の各点に平衡位置を持つ $N^3$ 個の原子が全体として立方体に並んだ 3 次元結晶を物理系として記述して、古典力学により考察する。任意の $i_x,i_y,i_z = 1,\ldots,N$ に対してラベル $(i_x,i_y,i_z)$ を持つ原子の平衡位置は格子定数 $a$ を用いて $(ai_x,ai_y,ai_z)$ であるとする.
\begin{itembox}[l]{Q 17-6.}
  このとき 3 次元結晶の Hamilton 関数は次のように与えられる.
  \begin{align}
       & H^{3次元結晶}((q_{i_x, i_y, i_z, \alpha}, p_{i_x, i_y, i_z, \alpha})_{1\leq i_x,i_y,i_z\leq N,\alpha=x,y,z})                                                                                                                                                        \\
    := & \frac{1}{2m}\sum_{i_x=1}^{N}\sum_{i_y=1}^{N}\sum_{i_z=1}^{N}\sum_{\alpha=x,y,z}p_{i_x,i_y,i_z,\alpha}^2                                                                                                                                                         \\
    +  & \frac{1}{2}\kappa\sum_{i_x=0}^{N}\sum_{i_y=0}^{N}\sum_{i_z=0}^{N}\sum_{\alpha=x,y,z}\ab((q_{i_x,i_y,i_z,\alpha} - q_{i_x+1,i_y,i_z,\alpha})^2 + (q_{i_x,i_y,i_z,\alpha} - q_{i_x,i_y+1,i_z,\alpha})^2 + (q_{i_x,i_y,i_z,\alpha} - q_{i_x,i_y,i_z+1,\alpha})^2).
  \end{align}
  ただし $m$ は 1 個の原子の質量であり, $\kappa$ は隣り合った原子間の原子間力のバネ定数とする. また立方体の表面は固定されているとする.
  \begin{align}
    i_x = 0, N+1 \lor i_y = 0, N+1 \lor i_z = 0, N+1 \implies q_{i_x,i_y,i_z,\alpha} = 0.
  \end{align}
\end{itembox}
Q17-3 の考察から 1 次元結晶の系の Hamilton 関数は次のように与えられる.
\begin{align}
  H^{1次元結晶}(q_1,\ldots,q_N, p_1,\ldots,p_N) & := \frac{1}{2m}\sum_{i=1}^{N}p_i^2 + \frac{1}{2}\kappa\sum_{i=0}^{N}(q_i - q_{i+1})^2.
\end{align}
3 次元結晶の系は $N^3$ 個の原子と $3$ 個の自由度があり, それらの原子間力は独立にそれぞれの自由度と原子に働くと考えられる. これより 3 次元結晶の系の Hamilton 関数 $H^{3次元結晶}((q_{i_x, i_y, i_z, \alpha}, p_{i_x, i_y, i_z, \alpha})_{1\leq i_x,i_y,i_z\leq N,\alpha=x,y,z})$ は次のように書ける.
\begin{align}
     & H^{3次元結晶}((q_{i_x, i_y, i_z, \alpha}, p_{i_x, i_y, i_z, \alpha})_{1\leq i_x,i_y,i_z\leq N,\alpha=x,y,z})                                                                                                                                                        \\
  := & \frac{1}{2m}\sum_{i_x=1}^{N}\sum_{i_y=1}^{N}\sum_{i_z=1}^{N}\sum_{\alpha=x,y,z}p_{i_x,i_y,i_z,\alpha}^2                                                                                                                                                         \\
  +  & \frac{1}{2}\kappa\sum_{i_x=0}^{N}\sum_{i_y=0}^{N}\sum_{i_z=0}^{N}\sum_{\alpha=x,y,z}\ab((q_{i_x,i_y,i_z,\alpha} - q_{i_x+1,i_y,i_z,\alpha})^2 + (q_{i_x,i_y,i_z,\alpha} - q_{i_x,i_y+1,i_z,\alpha})^2 + (q_{i_x,i_y,i_z,\alpha} - q_{i_x,i_y,i_z+1,\alpha})^2).
\end{align}
ただし $m$ は 1 個の原子の質量であり, $\kappa$ は隣り合った原子間の原子間力のバネ定数とする. また立方体の表面は固定されているとする.
\begin{align}
  i_x = 0, N+1 \lor i_y = 0, N+1 \lor i_z = 0, N+1 \implies q_{i_x,i_y,i_z,\alpha} = 0.
\end{align}

\subsection{3 次元結晶における平衡位置の回りの調和振動の基準モードの計算}
固定端境界条件の 3 次元結晶の系を考えているので 1 次元の Fourier Sine 展開の基底 3 つの直積が基準振動になっていると予想できる. これより古い座標 $q_{i_x,i_y,i_z,\alpha}$ を基準振動 $q_{i_x}^{(j_x)}q_{i_y}^{(j_y)}q_{i_z}^{(j_z)}$ で展開したときの振幅を新しい座標 $Q_{j_x,j_y,j_z,\alpha}$ とする.
\begin{align}
  q_{i_x,i_y,i_z,\alpha} & = \sum_{j_x=1}^{N}\sum_{j_y=1}^{N}\sum_{j_z=1}^{N}Q_{j_x,j_y,j_z,\alpha}q_{i_x}^{(j_x)}q_{i_y}^{(j_y)}q_{i_z}^{(j_z)}.
\end{align}
この新しい座標 $Q_{j_x,j_y,j_z,\alpha}$ に対応する新しい運動量を $P_{j_x, j_y, j_z, \alpha}$ とおくと Hamilton 関数について次のように表される.

\begin{itembox}[l]{Q 17-7.}
  新しい座標と運動量 $Q_{j_x, j_y, j_z, \alpha}, P_{j_x, j_y, j_z, \alpha}$ において Hamilton 関数は次のように表される.
  \begin{align}
    H^{3次元結晶}((Q_{j_x, j_y, j_z, \alpha}, P_{j_x, j_y, j_z, \alpha})_{1\leq j_x,j_y,j_z\leq N,\alpha=x,y,z}) & = \sum_{j_x=1}^{N}\sum_{j_y=1}^{N}\sum_{j_z=1}^{N}\sum_{\alpha=x,y,z}\ab(\frac{1}{2m}P_{j_x,j_y,j_z,\alpha}^2 + \frac{1}{2}m\omega_{j_x,j_y,j_z}^2Q_{j_x,j_y,j_z,\alpha}^2).
  \end{align}
  ただし, $\omega_{j_x,j_y,j_z}$ は次のように定めた.
  \begin{align}
    \omega_{j_x,j_y,j_z} & = 2\sqrt{\frac{\kappa}{m}}\sqrt{\sin^2\ab(\frac{\pi}{2(N+1)}j_x) + \sin^2\ab(\frac{\pi}{2(N+1)}j_y) + \sin^2\ab(\frac{\pi}{2(N+1)}j_z)}.
  \end{align}
\end{itembox}

(i) Q17-1 の考察より新しい運動量を古い運動量と座標, 新しい座標から求めることができる.
\begin{align}
  P_{j_x,j_y,j_z,\alpha} & = \sum_{i_x=1}^{N}\sum_{i_y=1}^{N}\sum_{i_z=1}^{N}\diffp{q_{i_x,i_y,i_z,\alpha}}{Q_{j_x,j_y,j_z,\alpha}}p_{i_x,i_y,i_z,\alpha} \\
                         & = \sum_{i_x=1}^{N}\sum_{i_y=1}^{N}\sum_{i_z=1}^{N}q_{i_x}^{(j_x)}q_{i_y}^{(j_y)}q_{i_z}^{(j_z)}p_{i_x,i_y,i_z,\alpha}.
\end{align}

(ii) この点正準変換に対し, 運動エネルギーは新しい運動量を用いて表せられる.
\begin{align}
    & \sum_{j_x=1}^{N}\sum_{j_y=1}^{N}\sum_{j_z=1}^{N}P_{j_x,j_y,j_z,\alpha}^2                                                                                                                                                                                                                             \\
  = & \sum_{j_x=1}^{N}\sum_{j_y=1}^{N}\sum_{j_z=1}^{N}\ab(\sum_{i_x=1}^{N}\sum_{i_y=1}^{N}\sum_{i_z=1}^{N}q_{i_x}^{(j_x)}q_{i_y}^{(j_y)}q_{i_z}^{(j_z)}p_{i_x,i_y,i_z,\alpha})^2                                                                                                                           \\
  = & \sum_{j_x=1}^{N}\sum_{j_y=1}^{N}\sum_{j_z=1}^{N}\ab(\sum_{i_x=1}^{N}\sum_{i_y=1}^{N}\sum_{i_z=1}^{N}\sum_{i_x'=1}^{N}\sum_{i_y'=1}^{N}\sum_{i_z'=1}^{N}q_{i_x}^{(j_x)}q_{i_y}^{(j_y)}q_{i_z}^{(j_z)}p_{i_x,i_y,i_z,\alpha}q_{i_x'}^{(j_x)}q_{i_y'}^{(j_y)}q_{i_z'}^{(j_z)}p_{i_x',i_y',i_z',\alpha}) \\
  = & \sum_{i_x=1}^{N}\sum_{i_y=1}^{N}\sum_{i_z=1}^{N}\sum_{i_x'=1}^{N}\sum_{i_y'=1}^{N}\sum_{i_z'=1}^{N}\delta_{i_x,i_x'}\delta_{i_y,i_y'}\delta_{i_z,i_z'}p_{i_x,i_y,i_z,\alpha}p_{i_x',i_y',i_z',\alpha}                                                                                                \\
  = & \sum_{i_x=1}^{N}\sum_{i_y=1}^{N}\sum_{i_z=1}^{N}p_{i_x,i_y,i_z,\alpha}^2.
\end{align}

(iii) またポテンシャルエネルギーについても新しい座標で表すことができる.
\begin{align}
    & \sum_{i_x=0}^{N}\sum_{i_y=0}^{N}\sum_{i_z=0}^{N}(q_{i_x,i_y,i_z,\alpha} - q_{i_x+1,i_y,i_z,\alpha})^2                                                                                                                                                                                                                                              \\
  = & \sum_{i_x=0}^{N}\sum_{i_y=0}^{N}\sum_{i_z=0}^{N}\ab(\sum_{j_x=1}^{N}\sum_{j_y=1}^{N}\sum_{j_z=1}^{N}\ab(Q_{j_x,j_y,j_z,\alpha}q_{i_x}^{(j_x)}q_{i_y}^{(j_y)}q_{i_z}^{(j_z)} - Q_{j_x,j_y,j_z,\alpha}q_{i_x+1}^{(j_x)}q_{i_y}^{(j_y)}q_{i_z}^{(j_z)}))^2                                                                                            \\
  = & \sum_{i_x=0}^{N}\sum_{i_y=0}^{N}\sum_{i_z=0}^{N}\sum_{j_x=1}^{N}\sum_{j_y=1}^{N}\sum_{j_z=1}^{N}\sum_{j_x'=1}^{N}\sum_{j_y'=1}^{N}\sum_{j_z'=1}^{N}                                                                                                                                                                                                \\
    & \ab(Q_{j_x,j_y,j_z,\alpha}q_{i_x}^{(j_x)}q_{i_y}^{(j_y)}q_{i_z}^{(j_z)} - Q_{j_x,j_y,j_z,\alpha}q_{i_x+1}^{(j_x)}q_{i_y}^{(j_y)}q_{i_z}^{(j_z)})\ab(Q_{j_x',j_y',j_z',\alpha}q_{i_x}^{(j_x')}q_{i_y}^{(j_y')}q_{i_z}^{(j_z')} - Q_{j_x',j_y',j_z',\alpha}q_{i_x+1}^{(j_x')}q_{i_y}^{(j_y')}q_{i_z}^{(j_z')})                                       \\
  = & \sum_{i_x=0}^{N}\sum_{i_y=0}^{N}\sum_{i_z=0}^{N}\sum_{j_x=1}^{N}\sum_{j_y=1}^{N}\sum_{j_z=1}^{N}\sum_{j_x'=1}^{N}\sum_{j_y'=1}^{N}\sum_{j_z'=1}^{N}Q_{j_x,j_y,j_z,\alpha}\ab(q_{i_x}^{(j_x)} - q_{i_x+1}^{(j_x)})q_{i_y}^{(j_y)}q_{i_z}^{(j_z)}Q_{j_x',j_y',j_z',\alpha}\ab(q_{i_x}^{(j_x')} - q_{i_x+1}^{(j_x')})q_{i_y}^{(j_y')}q_{i_z}^{(j_z')} \\
  = & \sum_{j_x=1}^{N}\sum_{j_y=1}^{N}\sum_{j_z=1}^{N}\sum_{j_x'=1}^{N}\sum_{j_y'=1}^{N}\sum_{j_z'=1}^{N}B_{j_x,j_x'}\delta_{j_y,j_y'}\delta_{j_z,j_z'}Q_{j_x,j_y,j_z,\alpha}Q_{j_x',j_y',j_z',\alpha}                                                                                                                                                   \\
  = & \sum_{j_x=1}^{N}\sum_{j_y=1}^{N}\sum_{j_z=1}^{N}\sum_{j_x'=1}^{N}\sum_{j_y'=1}^{N}\sum_{j_z'=1}^{N}4\sin^2\ab(\frac{\pi}{2(N+1)}j_x)\delta_{j_x,j_x'}\delta_{j_y,j_y'}\delta_{j_z,j_z'}Q_{j_x,j_y,j_z,\alpha}Q_{j_x',j_y',j_z',\alpha}                                                                                                             \\
  = & 4\sum_{j_x=1}^{N}\sum_{j_y=1}^{N}\sum_{j_z=1}^{N}\sin^2\ab(\frac{\pi}{2(N+1)}j_x)Q_{j_x,j_y,j_z,\alpha}^2.
\end{align}

(iv) これより Hamilton 関数は新しい座標と運動量を用いて表すことができる.
\begin{align}
     & H^{3次元結晶}((q_{i_x, i_y, i_z, \alpha}, p_{i_x, i_y, i_z, \alpha})_{1\leq i_x,i_y,i_z\leq N,\alpha=x,y,z})                                                                                                                                                       \\
  := & \frac{1}{2m}\sum_{i_x=1}^{N}\sum_{i_y=1}^{N}\sum_{i_z=1}^{N}\sum_{\alpha=x,y,z}p_{i_x,i_y,i_z,\alpha}^2                                                                                                                                                        \\
  +  & \frac{1}{2}\kappa\sum_{i_x=0}^{N}\sum_{i_y=0}^{N}\sum_{i_z=0}^{N}\sum_{\alpha=x,y,z}\ab((q_{i_x,i_y,i_z,\alpha} - q_{i_x+1,i_y,i_z,\alpha})^2 + (q_{i_x,i_y,i_z,\alpha} - q_{i_x,i_y+1,i_z,\alpha})^2 + (q_{i_x,i_y,i_z,\alpha} - q_{i_x,i_y,i_z+1,\alpha})^2) \\
  =  & \frac{1}{2m}\sum_{j_x=1}^{N}\sum_{j_y=1}^{N}\sum_{j_z=1}^{N}\sum_{\alpha=x,y,z}P_{j_x,j_y,j_z,\alpha}^2                                                                                                                                                        \\
  +  & 2\kappa\sum_{i_x=0}^{N}\sum_{i_y=0}^{N}\sum_{i_z=0}^{N}\sum_{\alpha=x,y,z}\ab(\sin^2\ab(\frac{\pi}{2(N+1)}j_x)Q_{j_x,j_y,j_z,\alpha}^2 + \sin^2\ab(\frac{\pi}{2(N+1)}j_y)Q_{j_x,j_y,j_z,\alpha}^2 + \sin^2\ab(\frac{\pi}{2(N+1)}j_z)Q_{j_x,j_y,j_z,\alpha}^2)  \\
  =  & \sum_{j_x=1}^{N}\sum_{j_y=1}^{N}\sum_{j_z=1}^{N}\sum_{\alpha=x,y,z}\ab(\frac{1}{2m}P_{j_x,j_y,j_z,\alpha}^2 + 2\kappa\ab(\sin^2\ab(\frac{\pi}{2(N+1)}j_x) + \sin^2\ab(\frac{\pi}{2(N+1)}j_y) + \sin^2\ab(\frac{\pi}{2(N+1)}j_z))Q_{j_x,j_y,j_z,\alpha}^2)      \\
  =  & \sum_{j_x=1}^{N}\sum_{j_y=1}^{N}\sum_{j_z=1}^{N}\sum_{\alpha=x,y,z}\ab(\frac{1}{2m}P_{j_x,j_y,j_z,\alpha}^2 + \frac{1}{2}m\omega_{j_x,j_y,j_z}^2Q_{j_x,j_y,j_z,\alpha}^2).
\end{align}
ただし, $\omega_{j_x,j_y,j_z}$ は次のように定めた.
\begin{align}
  \omega_{j_x,j_y,j_z} & = 2\sqrt{\frac{\kappa}{m}}\sqrt{\sin^2\ab(\frac{\pi}{2(N+1)}j_x) + \sin^2\ab(\frac{\pi}{2(N+1)}j_y) + \sin^2\ab(\frac{\pi}{2(N+1)}j_z)}.
\end{align}

これより 3 次元結晶の模型の基準振動は位置や運動量に独立な角振動数 $\omega_{j_x,j_y,j_z}$ の調和振動子となることがわかった.

\begin{itembox}[l]{Q 17-9.}
  3 次元結晶の模型における調和振動子の角振動数の個数分布関数 $g(\omega)$ は次のように表される.
  \begin{align}
    g(\omega) & = 3\sum_{j_x=1}^{N}\sum_{j_y=1}^{N}\sum_{j_z=1}^{N}\delta(\omega - \omega(\bm{k}_{j_x,j_y,j_z})).
  \end{align}
\end{itembox}

(i) 調和振動子の角振動数 $\omega(\bm{k}_{j_x, j_y, j_z})$ の個数分布関数 $g(\omega)$ について $\omega(\bm{k}_{j_x, j_y, j_z})$ は離散的な値を持ち, 各基準モード $(j_x, j_y, j_z, \alpha)$ によってパラメータ化されるのでデルタ関数を用いて次のように表される.
\begin{align}
  g(\omega) & = \sum_{j_x=1}^{N}\sum_{j_y=1}^{N}\sum_{j_z=1}^{N}\sum_{\alpha=x,y,z}\delta(\omega - \omega(\bm{k}_{j_x,j_y,j_z})) \\
            & = 3\sum_{j_x=1}^{N}\sum_{j_y=1}^{N}\sum_{j_z=1}^{N}\delta(\omega - \omega(\bm{k}_{j_x,j_y,j_z})).
\end{align}
また $\omega(\bm{k}_{j_x,j_y,j_z})$ は $\omega(\bm{k}_{j_x,j_y,j_z})\geq 0$ に限られるから $\omega\geq 0$ となる.

(ii) これより調和振動子の総数は次のようになる.
\begin{align}
  \int_0^\infty\dl{\omega}g(\omega) & = 3\int_0^\infty\dl{\omega}\sum_{j_x=1}^{N}\sum_{j_y=1}^{N}\sum_{j_z=1}^{N}\delta(\omega - \omega(\bm{k}_{j_x,j_y,j_z})) \\
                                    & = 3N^3.
\end{align}

ただこのような調和振動子の角振動数の個数分布関数 $g(\omega)$ をさらに簡単にすることは分散関係 $\omega(\bm{k})$ の複雑さのためにできない為, これに統計力学を適用しても計算がすぐに行き詰まる.

\subsection{量子論での基準モード}
今まで古典力学により行ってきた考察を量子力学に翻訳する. まず Debye 模型の Hamilton 関数は次のように与えられる.
\begin{align}
  \hat{H} & = \frac{1}{2m}\sum_{i_x=1}^{N}\sum_{i_y=1}^{N}\sum_{i_z=1}^{N}\sum_{\alpha=x,y,z}\hat{p}_{i_x,i_y,i_z,\alpha}^2                                                                                                                                                                                       \\
          & + \frac{1}{2}\kappa\sum_{i_x=0}^{N}\sum_{i_y=0}^{N}\sum_{i_z=0}^{N}\sum_{\alpha=x,y,z}\ab((\hat{q}_{i_x,i_y,i_z,\alpha} - \hat{q}_{i_x+1,i_y,i_z,\alpha})^2 + (\hat{q}_{i_x,i_y,i_z,\alpha} - \hat{q}_{i_x,i_y+1,i_z,\alpha})^2 + (\hat{q}_{i_x,i_y,i_z,\alpha} - \hat{q}_{i_x,i_y,i_z+1,\alpha})^2).
\end{align}
ただし $m$ は 1 個の原子の質量であり, $\kappa$ は隣り合った原子間の原子間力のバネ定数とする. また立方体の表面は固定されているとする.
\begin{align}
  i_x = 0, N+1\lor i_y = 0, N+1\lor i_z = 0, N+1 \implies \hat{q}_{i_x,i_y,i_z,\alpha} = 0.
\end{align}
また位置演算子 $\hat{q}_{i_x,i_y,i_z,\alpha}$ と運動量演算子 $\hat{p}_{i_x',i_y',i_z',\alpha'}$ は正準交換関係を満たす.
\begin{align}
  \ab[\hat{q}_{i_x,i_y,i_z,\alpha}, \hat{p}_{i_x',i_y',i_z',\alpha'}] & = \sqrt{-1}\hbar\delta_{i_x,i_x'}\delta_{i_y,i_y'}\delta_{i_z,i_z'}\delta_{\alpha,\alpha'}, \\
  \ab[\hat{q}_{i_x,i_y,i_z,\alpha}, \hat{q}_{i_x',i_y',i_z',\alpha'}] & = \ab[\hat{p}_{i_x,i_y,i_z,\alpha}, \hat{p}_{i_x',i_y',i_z',\alpha'}] = 0                   \\
  (1\leq i_x,i_y,i_z,i_x',i_y',i_z'                                   & \leq N, \alpha,\alpha' = x,y,z).
\end{align}
古典論での点正準変換を量子論でも行う. $(\hat{q}_{i_x,i_y,i_z,\alpha}, \hat{p}_{i_x,i_y,i_z,\alpha})_{1\leq i_x,i_y,i_z\leq N,\alpha=x,y,z}\to(\hat{Q}_{j_x,j_y,j_z,\alpha}, \hat{P}_{j_x,j_y,j_z,\alpha})_{1\leq j_x,j_y,j_z\leq N,\alpha=x,y,z}$ を次のように定める.
\begin{align}
  \hat{q}_{i_x,i_y,i_z,\alpha} & = \sum_{j_x=1}^{N}\sum_{j_y=1}^{N}\sum_{j_z=1}^{N}\hat{Q}_{j_x,j_y,j_z,\alpha}q_{i_x}^{(j_x)}q_{i_y}^{(j_y)}q_{i_z}^{(j_z)} \qquad (1\leq i_x,i_y,i_z \leq N, \alpha = x,y,z), \\
  \hat{P}_{j_x,j_y,j_z,\alpha} & = \sum_{i_x=1}^{N}\sum_{i_y=1}^{N}\sum_{i_z=1}^{N}\hat{p}_{i_x,i_y,i_z,\alpha}q_{i_x}^{(j_x)}q_{i_y}^{(j_y)}q_{i_z}^{(j_z)} \qquad (1\leq j_x,j_y,j_z \leq N, \alpha = x,y,z).
\end{align}

\begin{itembox}[l]{Q 17-13.}
  新しい位置演算子 $\hat{Q}_{j_x,j_y,j_z,\alpha}$ と運動量演算子 $\hat{P}_{j_x',j_y',j_z',\alpha'}$ について正準交換関係を満たす.
  \begin{align}
    \ab[\hat{Q}_{j_x,j_y,j_z,\alpha}, \hat{P}_{j_x',j_y',j_z',\alpha'}] & = \sqrt{-1}\hbar\delta_{j_x,j_x'}\delta_{j_y,j_y'}\delta_{j_z,j_z'}\delta_{\alpha,\alpha'}, \\
    \ab[\hat{Q}_{j_x,j_y,j_z,\alpha}, \hat{Q}_{j_x',j_y',j_z',\alpha'}] & = \ab[\hat{P}_{j_x,j_y,j_z,\alpha}, \hat{P}_{j_x',j_y',j_z',\alpha'}] = 0                   \\
    (1\leq j_x,j_y,j_z,j_x',j_y',j_z'                                   & \leq N, \alpha,\alpha' = x,y,z).
  \end{align}
\end{itembox}
まず $\hat{q}_{i_x,i_y,i_z,\alpha}$, $\hat{P}_{j_x',j_y',j_z',\alpha'}$ の交換関係について左を展開するものと右を展開するもので分けて計算すると次のようになる.
\begin{align}
  \ab[\hat{q}_{i_x,i_y,i_z,\alpha}, \hat{P}_{j_x',j_y',j_z',\alpha'}] & = \ab[\hat{q}_{i_x,i_y,i_z,\alpha}, \sum_{i_x'=1}^{N}\sum_{i_y'=1}^{N}\sum_{i_z'=1}^{N}\hat{p}_{i_x',i_y',i_z',\alpha'}q_{i_x'}^{(j_x')}q_{i_y'}^{(j_y')}q_{i_z'}^{(j_z')}]                      \\
                                                                      & = \sum_{i_x'=1}^{N}\sum_{i_y'=1}^{N}\sum_{i_z'=1}^{N}\ab[\hat{q}_{i_x,i_y,i_z,\alpha}, \hat{p}_{i_x',i_y',i_z',\alpha'}]q_{i_x'}^{(j_x')}q_{i_y'}^{(j_y')}q_{i_z'}^{(j_z')}                      \\
                                                                      & = \sum_{i_x'=1}^{N}\sum_{i_y'=1}^{N}\sum_{i_z'=1}^{N}\sqrt{-1}\hbar\delta_{i_x,i_x'}\delta_{i_y,i_y'}\delta_{i_z,i_z'}\delta_{\alpha,\alpha'}q_{i_x'}^{(j_x')}q_{i_y'}^{(j_y')}q_{i_z'}^{(j_z')} \\
                                                                      & = \sqrt{-1}\hbar\delta_{\alpha,\alpha'}q_{i_x}^{(j_x')}q_{i_y}^{(j_y')}q_{i_z}^{(j_z')},                                                                                                         \\
  \ab[\hat{q}_{i_x,i_y,i_z,\alpha}, \hat{P}_{j_x',j_y',j_z',\alpha'}] & = \ab[\sum_{j_x=1}^{N}\sum_{j_y=1}^{N}\sum_{j_z=1}^{N}\hat{Q}_{j_x,j_y,j_z,\alpha}q_{i_x}^{(j_x)}q_{i_y}^{(j_y)}q_{i_z}^{(j_z)}, \hat{P}_{j_x',j_y',j_z',\alpha'}]                               \\
                                                                      & = \sum_{j_x=1}^{N}\sum_{j_y=1}^{N}\sum_{j_z=1}^{N}\ab[\hat{Q}_{j_x,j_y,j_z,\alpha}, \hat{P}_{j_x',j_y',j_z',\alpha'}]q_{i_x}^{(j_x)}q_{i_y}^{(j_y)}q_{i_z}^{(j_z)}.
\end{align}
これより $q_{i_x}^{(j_x)}q_{i_y}^{(j_y)}q_{i_z}^{(j_z)}$ の直交性から次のことがわかる.
\begin{align}
  \ab[\hat{Q}_{j_x,j_y,j_z,\alpha}, \hat{P}_{j_x',j_y',j_z',\alpha'}] & = \sqrt{-1}\hbar\delta_{j_x,j_x'}\delta_{j_y,j_y'}\delta_{j_z,j_z'}\delta_{\alpha,\alpha'}.
\end{align}
同様に $\hat{q}_{i_x,i_y,i_z,\alpha}$ 同士, $\hat{P}_{j_x,j_y,j_z,\alpha}$ 同士の交換関係について計算すると次のようになる.
\begin{align}
  \ab[\hat{q}_{i_x,i_y,i_z,\alpha}, \hat{q}_{i_x',i_y',i_z',\alpha'}] & = \ab[\sum_{j_x=1}^{N}\sum_{j_y=1}^{N}\sum_{j_z=1}^{N}\hat{Q}_{j_x,j_y,j_z,\alpha}q_{i_x}^{(j_x)}q_{i_y}^{(j_y)}q_{i_z}^{(j_z)}, \sum_{j_x'=1}^{N}\sum_{j_y'=1}^{N}\sum_{j_z'=1}^{N}\hat{Q}_{j_x',j_y',j_z',\alpha'}q_{i_x'}^{(j_x')}q_{i_y'}^{(j_y')}q_{i_z'}^{(j_z')}] \\
                                                                      & = \sum_{j_x=1}^{N}\sum_{j_y=1}^{N}\sum_{j_z=1}^{N}\sum_{j_x'=1}^{N}\sum_{j_y'=1}^{N}\sum_{j_z'=1}^{N}\ab[\hat{Q}_{j_x,j_y,j_z,\alpha}, \hat{Q}_{j_x',j_y',j_z',\alpha'}]q_{i_x}^{(j_x)}q_{i_y}^{(j_y)}q_{i_z}^{(j_z)}q_{i_x'}^{(j_x')}q_{i_y'}^{(j_y')}q_{i_z'}^{(j_z')} \\
                                                                      & = 0,                                                                                                                                                                                                                                                                     \\
  \ab[\hat{P}_{j_x,j_y,j_z,\alpha}, \hat{P}_{j_x',j_y',j_z',\alpha'}] & = \ab[\sum_{i_x=1}^{N}\sum_{i_y=1}^{N}\sum_{i_z=1}^{N}\hat{p}_{i_x,i_y,i_z,\alpha}q_{i_x}^{(j_x)}q_{i_y}^{(j_y)}q_{i_z}^{(j_z)}, \sum_{i_x'=1}^{N}\sum_{i_y'=1}^{N}\sum_{i_z'=1}^{N}\hat{p}_{i_x',i_y',i_z',\alpha}q_{i_x'}^{(j_x')}q_{i_y'}^{(j_y')}q_{i_z'}^{(j_z')}]  \\
                                                                      & = \sum_{i_x=1}^{N}\sum_{i_y=1}^{N}\sum_{i_z=1}^{N}\sum_{i_x'=1}^{N}\sum_{i_y'=1}^{N}\sum_{i_z'=1}^{N}\ab[\hat{p}_{i_x,i_y,i_z,\alpha}, \hat{p}_{i_x',i_y',i_z',\alpha}]q_{i_x}^{(j_x)}q_{i_y}^{(j_y)}q_{i_z}^{(j_z)}q_{i_x'}^{(j_x')}q_{i_y'}^{(j_y')}q_{i_z'}^{(j_z')}  \\
                                                                      & = 0.
\end{align}
これより $q_{i_x}^{(j_x)}q_{i_y}^{(j_y)}q_{i_z}^{(j_z)}q_{i_x'}^{(j_x')}q_{i_y'}^{(j_y')}q_{i_z'}^{(j_z')}$ の直交性から次のことがわかる.
\begin{align}
  \ab[\hat{Q}_{j_x,j_y,j_z,\alpha}, \hat{Q}_{j_x',j_y',j_z',\alpha'}] = \ab[\hat{P}_{j_x,j_y,j_z,\alpha}, \hat{P}_{j_x',j_y',j_z',\alpha'}] = 0.
\end{align}
よって示された.
\begin{align}
  \ab[\hat{Q}_{j_x,j_y,j_z,\alpha}, \hat{P}_{j_x',j_y',j_z',\alpha'}] & = \sqrt{-1}\hbar\delta_{j_x,j_x'}\delta_{j_y,j_y'}\delta_{j_z,j_z'}\delta_{\alpha,\alpha'}, \\
  \ab[\hat{Q}_{j_x,j_y,j_z,\alpha}, \hat{Q}_{j_x',j_y',j_z',\alpha'}] & = \ab[\hat{P}_{j_x,j_y,j_z,\alpha}, \hat{P}_{j_x',j_y',j_z',\alpha'}] = 0                   \\
  (1\leq j_x,j_y,j_z,j_x',j_y',j_z'                                   & \leq N, \alpha,\alpha' = x,y,z).
\end{align}

\begin{itembox}[l]{Q 17-14.}
  Hamilton 演算子 $\hat{H}$ は独立な調和振動子の Hamilton 演算子の和となる.
  \begin{align}
    \hat{H} & = \sum_{j_x=1}^{N}\sum_{j_y=1}^{N}\sum_{j_z=1}^{N}\sum_{\alpha=x,y,z}\ab(\frac{1}{2m}\hat{P}_{j_x,j_y,j_z,\alpha}^2 + \frac{1}{2}m\omega_{j_x,j_y,j_z}^2\hat{Q}_{j_x,j_y,j_z,\alpha}^2).
  \end{align}
  ただし $\omega_{j_x,j_y,j_z}$ は次のように与えられる.
  \begin{align}
    \omega_{j_x,j_y,j_z} & = 2\sqrt{\frac{\kappa}{m}}\sqrt{\sin^2\ab(\frac{\pi}{2(N+1)}j_x) + \sin^2\ab(\frac{\pi}{2(N+1)}j_y) + \sin^2\ab(\frac{\pi}{2(N+1)}j_z)}.
  \end{align}
\end{itembox}
Q 17-7 で位置, 運動量が演算子だとしても同様に計算できるよう書いたので同じ結果が得られる. よって Hamilton 演算子は次のように書ける.
\begin{align}
  \hat{H} & = \sum_{j_x=1}^{N}\sum_{j_y=1}^{N}\sum_{j_z=1}^{N}\sum_{\alpha=x,y,z}\ab(\frac{1}{2m}\hat{P}_{j_x,j_y,j_z,\alpha}^2 + \frac{1}{2}m\omega_{j_x,j_y,j_z}^2\hat{Q}_{j_x,j_y,j_z,\alpha}^2).
\end{align}
ただし $\omega_{j_x,j_y,j_z}$ は次のように与えられる.
\begin{align}
  \omega_{j_x,j_y,j_z} & = 2\sqrt{\frac{\kappa}{m}}\sqrt{\sin^2\ab(\frac{\pi}{2(N+1)}j_x) + \sin^2\ab(\frac{\pi}{2(N+1)}j_y) + \sin^2\ab(\frac{\pi}{2(N+1)}j_z)}.
\end{align}















\begin{itembox}[l]{Q 17-12.}
  Debye 模型における調和振動子の角振動数の個数分布関数 $g(\omega)$ は次のように表される.
  \begin{align}
    g(\omega) & = \begin{dcases}
                    \frac{9N^3}{\omega_D}\ab(\frac{\omega}{\omega_D})^2 & (\omega\leq\omega_D) \\
                    0                                                   & (\omega > \omega_D)
                  \end{dcases} \\
    \omega_D  & = (6\pi^2)^{1/3}\sqrt{\frac{\kappa}{m}}.
  \end{align}
\end{itembox}

(i) Debye 模型における調和振動子の角振動数の個数分布関数 $g(\omega)$ は $\omega(\bm{k}_{j_x, j_y, j_z})$ が固有モード $(j_x, j_y, j_z, \alpha)$ によってパラメータ化されるのでデルタ関数を用いて次のように表される.
\begin{align}
  g(\omega) & = \sum_{j_x=1}^{N}\sum_{j_y=1}^{N}\sum_{j_z=1}^{N}\sum_{\alpha=x,y,z}\delta(\omega - \omega(\bm{k}_{j_x,j_y,j_z}))      \\
            & = 3\sum_{j_x=1}^{N}\sum_{j_y=1}^{N}\sum_{j_z=1}^{N}\delta(\omega - \omega(\bm{k}_{j_x,j_y,j_z})) \qquad (\omega\geq 0).
\end{align}

(ii) また調和振動子の総数は 3 次元結晶の模型と同様に $3N^3$ となる.
\begin{align}
  \int_0^\infty\dl{\omega}g(\omega) & = 3\int_0^\infty\dl{\omega}\sum_{j_x=1}^{N}\sum_{j_y=1}^{N}\sum_{j_z=1}^{N}\sum_{\alpha=x,y,z}\delta(\omega - \omega(\bm{k}_{j_x,j_y,j_z})) \\
                                    & = 3N^3.
\end{align}

(iii) ここでDebye 模型における調和振動子の角振動数の個数分布関数 $g(\omega)$ を具体的に計算すると次のようになる.
\begin{align}
  g(\omega) & = 3\sum_{j_x=1}^{N}\sum_{j_y=1}^{N}\sum_{j_z=1}^{N}\delta(\omega - \omega(\bm{k}_{j_x,j_y,j_z}))                                                                                 \\
            & = 3\sum_{j_x=1}^{N}\sum_{j_y=1}^{N}\sum_{j_z=1}^{N}\delta\ab(\omega - \sqrt{\frac{\kappa}{m}}a\ab|\frac{\pi}{a(N+1)}(j_x,j_y,j_z)|)                                              \\
            & = 3\sum_{j_x=1}^{N}\sum_{j_y=1}^{N}\sum_{j_z=1}^{N}\delta\ab(\omega - \sqrt{\frac{\kappa}{m}}\frac{\pi}{N+1}\sqrt{j_x^2 + j_y^2 + j_z^2})                                        \\
            & = 3\sqrt{\frac{m}{\kappa}}\frac{N+1}{\pi}\sum_{j_x=1}^{N}\sum_{j_y=1}^{N}\sum_{j_z=1}^{N}\delta\ab(\sqrt{\frac{m}{\kappa}}\frac{N+1}{\pi}\omega - \sqrt{j_x^2 + j_y^2 + j_z^2}).
\end{align}

(iv) またデルタ関数を少し広がった有限の Gauss 分布とすることで $g(\omega)$ を滑らかな分布として近似できる. これより総和は次のように積分で置き換えられることが言える.
\begin{align}
  g(\omega) & = 3\sqrt{\frac{m}{\kappa}}\frac{N+1}{\pi}\sum_{j_x=1}^{N}\sum_{j_y=1}^{N}\sum_{j_z=1}^{N}\delta\ab(\sqrt{\frac{m}{\kappa}}\frac{N+1}{\pi}\omega - \sqrt{j_x^2 + j_y^2 + j_z^2})                    \\
            & \approx 3\sqrt{\frac{m}{\kappa}}\frac{N+1}{\pi}\int_{1}^{N}\dl{j_x}\int_{1}^{N}\dl{j_y}\int_{1}^{N}\dl{j_z}\delta\ab(\sqrt{\frac{m}{\kappa}}\frac{N+1}{\pi}\omega - \sqrt{j_x^2 + j_y^2 + j_z^2}).
\end{align}

(v) ここで $\omega$ に関する次の条件が成り立つとする.
\begin{align}
  \sqrt{\frac{m}{\kappa}}\frac{N+1}{\pi}\omega \leq N. \label{omega_condition}
\end{align}
特に $g(\omega)$ の被積分関数の積分値は次のような幾何学的解釈で近似できる.
\begin{align}
          & \int_{1}^{N}\dl{j_x}\int_{1}^{N}\dl{j_y}\int_{1}^{N}\dl{j_z}\delta\ab(\sqrt{\frac{m}{\kappa}}\frac{N+1}{\pi}\omega - \sqrt{j_x^2 + j_y^2 + j_z^2})                            \\
  =       & \int_V\dl{\bm{r}}\delta\ab(|\bm{r}| - \sqrt{\frac{m}{\kappa}}\frac{N+1}{\pi}\omega) \qquad \ab(V := \lbrace (x, y, z)\mid 1\leq x\leq N, 1\leq y\leq N, 1\leq z\leq N\rbrace) \\
  \approx & \ab(半径 \sqrt{\frac{m}{\kappa}}\frac{N+1}{\pi}\omega の 2 次元球面 S_2 を第 1 象限で切り取った曲面の表面積).
\end{align}
これより $g(\omega)$ は次のように書ける.
\begin{align}
  g(\omega) & \approx 3\sqrt{\frac{m}{\kappa}}\frac{N+1}{\pi}\times\ab(半径 \sqrt{\frac{m}{\kappa}}\frac{N+1}{\pi}\omega の 2 次元球面 S_2 を第 1 象限で切り取った曲面の表面積).
\end{align}

(vi) それを具体的に計算すると次のようになる.
\begin{align}
  g(\omega) & \approx 3\sqrt{\frac{m}{\kappa}}\frac{N+1}{\pi}\times\ab(半径 \sqrt{\frac{m}{\kappa}}\frac{N+1}{\pi}\omega の 2 次元球面 S_2 を第 1 象限で切り取った曲面の表面積) \\
            & = 3\sqrt{\frac{m}{\kappa}}\frac{N+1}{\pi}\times\frac{4\pi}{8}\ab(\sqrt{\frac{m}{\kappa}}\frac{N+1}{\pi}\omega)^2                           \\
            & = \frac{3\pi}{2}\ab(\sqrt{\frac{m}{\kappa}}\frac{N+1}{\pi})^3\omega^2.
\end{align}

(vii) $\omega$ に関する条件 \eqref{omega_condition} が成り立たない場合は立方体の積分範囲と球面の表面の共通部分の面積となるので複雑な式となってしまう. ただ Debye 模型は低温における比熱の振る舞いからの要請により $\omega(\bm{k})$ が大きいときは気にしなくて良い模型でした.
これより $g(\omega)$ の $(j_x, j_y, j_z)$ に関する積分範囲を立方体から球へ修正することが許され, 次のように $g(\omega)$ は表される.
\begin{align}
  g(\omega) & = \begin{dcases}
                  \frac{3\pi}{2}\ab(\sqrt{\frac{m}{\kappa}}\frac{N}{\pi})^3\omega^2 & (\omega\leq\omega_D) \\
                  0                                                                 & (\omega > \omega_D)
                \end{dcases}.
\end{align}
ただし $N\gg 1$ であることから $N+1$ を $N$ と近似し, また打ち切る角振動数 $\omega_D$ を次のように定める.
\begin{align}
  \int_0^\infty\dl{\omega}g(\omega) & = \int_0^{\omega_D}\dl{\omega}g(\omega) = 3N^3.
\end{align}
この $\omega_D$ を Debye の角振動数という.

(viii) これより Debye の角振動数 $\omega_D$ は次のように計算される.
\begin{align}
  \int_0^{\omega_D}\dl{\omega}g(\omega) & = \int_0^{\omega_D}\dl{\omega}\frac{3\pi}{2}\ab(\sqrt{\frac{m}{\kappa}}\frac{N}{\pi})^3\omega^2 = \frac{\pi}{2}\ab(\sqrt{\frac{m}{\kappa}}\frac{N}{\pi})^3\omega_D^3 = 3N^3, \\
  \omega_D                              & = \ab(3N^3\frac{2}{\pi})^{1/3}\sqrt{\frac{\kappa}{m}}\frac{\pi}{N} = (6\pi^2)^{1/3}\sqrt{\frac{\kappa}{m}}.
\end{align}

(ix) また Debye の角振動数 $\omega_D$ を用いて $g(\omega)$ は次のように表される.
\begin{align}
  g(\omega) & = \begin{dcases}
                  \frac{3\pi}{2}\ab(\sqrt{\frac{m}{\kappa}}\frac{N}{\pi})^3\omega^2 & (\omega\leq\omega_D) \\
                  0                                                                 & (\omega > \omega_D)
                \end{dcases} \\
            & = \begin{dcases}
                  \frac{9N^3}{\omega_D}\ab(\frac{\omega}{\omega_D})^2 & (\omega\leq\omega_D) \\
                  0                                                   & (\omega > \omega_D)
                \end{dcases}.
\end{align}

現実の物質に Debye 模型を当てはめるときには, それぞれの物質は固有の Debye 角振動数 $\omega_D$ を持つことになる.



\end{document}