\documentclass[a4paper,11pt]{jlreq}

% \usepackage{indentfirst} % 最初の段落にインデント
% \usepackage{wrapfig} % 表や画像の周りに文字を回り込ませる
% \usepackage{comment} % コメント環境
% \usepackage{docmute} % ファイル分割
\usepackage{listings} % ソースコードの挿入
\lstset{
  language=C++,
  breaklines=true,
  keywordstyle = {\color[rgb]{0,0,1}},
  stringstyle = {\color[rgb]{1,0,0}},
  commentstyle = { \color[rgb]{0,1,0}},
  numbers=left,
  frame=lines
}
\usepackage{bxpapersize} % A4判サイズを指定する
\usepackage[utf8]{inputenc}
\usepackage{fontenc} % フォントエンコーディング指定
\usepackage{lmodern} % Latin Modern フォント
\usepackage{otf}
\usepackage{amsmath}
\usepackage{amssymb}
\usepackage{amsthm}
\usepackage{ascmac}
% \usepackage[hyphens]{url}
\usepackage{mhchem}
\usepackage{siunitx}
\usepackage{physics2}
\usephysicsmodule{ab, ab.braket, doubleprod, diagmat, xmat}
\usepackage[DIF = {
      op-symbol = d,
      op-order-nudge = 1 mu,
      outer-Ldelim = \left . ,
      outer-Rdelim = \right |,
      sub-nudge = 0 mu
    }]{diffcoeff}
% \usepackage{braket}
\usepackage{verbatimbox}
\usepackage{bm} % 太字斜体
\usepackage{url}
% \usepackage[dvipdfmx,hiresbb,final]{graphicx}
\usepackage{hyperref} % リンク埋め込み
\usepackage{pxjahyper}
\usepackage{tikz} % グラフや図形を描く
\usetikzlibrary{cd}
% \usetikzlibrary{cd, intersections, calc, arrows, positioning, arrows.meta, automata}
\usepackage{tikz-feynhand}
\usepackage{listings}
\usepackage{color}
\usepackage{mathtools}
\usepackage{xspace}
\usepackage{xy}
\usepackage{xypic}

\makeatletter
%
\DeclareMathOperator{\lcm}{lcm}
\DeclareMathOperator{\Kernel}{Ker}
\DeclareMathOperator{\Image}{Im}
\DeclareMathOperator{\ch}{ch}
\DeclareMathOperator{\Aut}{Aut}
\DeclareMathOperator{\Log}{Log}
\DeclareMathOperator{\Arg}{Arg}
\DeclareMathOperator{\sgn}{sgn}
\DeclareMathOperator{\Res}{Res}
%
\newcommand{\CC}{\mathbb{C}}
\newcommand{\RR}{\mathbb{R}}
\newcommand{\QQ}{\mathbb{Q}}
\newcommand{\ZZ}{\mathbb{Z}}
\newcommand{\NN}{\mathbb{N}}
\newcommand{\FF}{\mathbb{F}}
\newcommand{\PP}{\mathbb{P}}
\newcommand{\GG}{\mathbb{G}}
\newcommand{\TT}{\mathbb{T}}
\newcommand{\EE}{\bm{E}}
\newcommand{\BB}{\bm{B}}
\renewcommand{\AA}{\bm{A}}
\newcommand{\rr}{\bm{r}}
\newcommand{\kk}{\bm{k}}
\newcommand{\pp}{\bm{p}}
\newcommand{\calB}{\mathcal{B}}
\newcommand{\calF}{\mathcal{F}}
\newcommand{\ignore}[1]{}
\newcommand{\floor}[1]{\left\lfloor #1 \right\rfloor}
% \newcommand{\abs}[1]{\left\lvert #1 \right\rvert}
\newcommand{\lt}{<}
\newcommand{\gt}{>}
\newcommand{\id}{\mathrm{id}}
\newcommand{\rot}{\curl}
\renewcommand{\angle}[1]{\left\langle #1 \right\rangle}
\newcommand\mqty[1]{\begin{pmatrix}#1\end{pmatrix}}
\newcommand\vmqty[1]{\begin{vmatrix}#1\end{vmatrix}}
\numberwithin{equation}{section}

\let\oldcite=\cite
\renewcommand\cite[1]{\hyperlink{#1}{\oldcite{#1}}}

\let\oldbibitem=\bibitem
\renewcommand{\bibitem}[2][]{\label{#2}\oldbibitem[#1]{#2}}

% theorem環境の設定
% - 冒頭に改行
% - 末尾にdiamond (amsthm)
\theoremstyle{definition}
\newcommand*{\newscreentheoremx}[2]{
  \newenvironment{#1}[1][]{
    \begin{screen}
    \begin{#2}[##1]
    \leavevmode
    \newline
  }{
    \end{#2}
    \end{screen}
  }
}
\newcommand*{\newqedtheoremx}[2]{
  \newenvironment{#1}[1][]{
    \begin{#2}[##1]
    \leavevmode
    \newline
    \renewcommand{\qedsymbol}{\(\diamond\)}
    \pushQED{\qed}
  }{
    \qedhere
    \popQED
    \end{#2}
  }
}
\newtheorem{theorem*}{定理}[section]

\newqedtheoremx{theorem}{theorem*}
\newcommand*\newqedtheorem@unstarred[2]{%
  \newtheorem{#1*}[theorem*]{#2}
  \newqedtheoremx{#1}{#1*}
}
\newcommand*\newqedtheorem@starred[2]{%
  \newtheorem*{#1*}{#2}
  \newqedtheoremx{#1}{#1*}
}
\newcommand*{\newqedtheorem}{\@ifstar{\newqedtheorem@starred}{\newqedtheorem@unstarred}}

\newtheorem{sctheorem*}{定理}[section]
\newscreentheoremx{sctheorem}{sctheorem*}
\newcommand*\newscreentheorem@unstarred[2]{%
  \newtheorem{#1*}[theorem*]{#2}
  \newscreentheoremx{#1}{#1*}
}
\newcommand*\newscreentheorem@starred[2]{%
  \newtheorem*{#1*}{#2}
  \newscreentheoremx{#1}{#1*}
}
\newcommand*{\newscreentheorem}{\@ifstar{\newscreentheorem@starred}{\newscreentheorem@unstarred}}

%\newtheorem*{definition}{定義}
%\newtheorem{theorem}{定理}
%\newtheorem{proposition}[theorem]{命題}
%\newtheorem{lemma}[theorem]{補題}
%\newtheorem{corollary}[theorem]{系}

\newqedtheorem{lemma}{補題}
\newqedtheorem{corollary}{系}
\newqedtheorem{example}{例}
\newqedtheorem{proposition}{命題}
\newqedtheorem{remark}{注意}
\newqedtheorem{thesis}{主張}
\newqedtheorem{notation}{記法}
\newqedtheorem{problem}{問題}
\newqedtheorem{algorithm}{アルゴリズム}

\newscreentheorem*{axiom}{公理}
\newscreentheorem*{definition}{定義}

\renewenvironment{proof}[1][\proofname]{\par
  \normalfont
  \topsep6\p@\@plus6\p@ \trivlist
  \item[\hskip\labelsep{\bfseries #1}\@addpunct{\bfseries}]\ignorespaces\quad\par
}{%
  \qed\endtrivlist\@endpefalse
}
\renewcommand\proofname{証明}

\makeatother


\title{プラズマ物理学}
\author{21B00349 宇佐見大希}

\begin{document}
\maketitle
\tableofcontents
\clearpage

\section{プラズマ}
プラズマとは高エネルギー状態で励起・電離し、陽イオンと電子が熱運動している状態のこと。



\begin{table}[h]
  \centering
  \begin{tabular}{|c|ccc|}
    \hline
    種類   & 質量                 & 電荷           & 力 \\
    \hline
    陽イオン & $M \approx 1800m$  & $Z_\sigma e$ &   \\
    電子   & $m = 511$ \si{keV} & $-e$         &   \\
    \hline
  \end{tabular}
  \caption{}
  \label{table:particles}
\end{table}
\begin{align}
  \alpha_\sigma = \frac{n_\sigma}{N_\sigma}
\end{align}

\subsection{散乱断面積}

\begin{align}
  D = r_1 + r_2
\end{align}
\begin{align}
  \sigma(\chi) = \frac{D^2}{4}
\end{align}
\begin{align}
  \sigma(\chi) = \ab|\frac{b}{\sin b}\diff{b}{\chi}|
\end{align}
\begin{align}
  Q = \int W(\chi)\sigma(\chi)\dl{\Omega} = \int_0^\pi W(\chi)\sigma(\chi)2\pi\sin\chi\dl{\chi}
\end{align}

\begin{align}
  \begin{dcases}
    F_r = m(\ddot{r} - r\dot{\theta}^2) = \frac{q_0q}{r^2} \\
    F_\theta = \frac{1}{r}\diff{}{t}(mr^2\dot{\theta}) = 0
  \end{dcases}
\end{align}
これらをエネルギー積分することで保存量を見出すことができる。
\begin{align}
   & \diff{}{t}(mr^2\dot\theta) = \diff{}{t}L = 0                                                                                                                                                                          \\
   & m\dot{r}(\ddot{r} - r\dot{\theta}^2) - \frac{q_0q}{r^2}\dot{r} = m\dot{r}\ddot{r} - \frac{L^2}{mr^3}\dot{r} - \frac{q_0q}{r^2}\dot{r} = \diff{}{t}\ab(\frac{1}{2}m\dot{r}^2 + \frac{L^2}{2mr^2} + \frac{q_0q}{r}) = 0
\end{align}
これより初期状態において速度 $v_0$, $r\to\infty$ に対して最近接距離 $r_{\min}$ のとき $\dot{r} = 0$ となるから次のように求まる。
\begin{align}
  \frac{1}{2}mv_0^2 & = \frac{L_0^2}{2mr_{\min}^2} + \frac{q_0q}{r_{\min}}, \qquad L_0 = mv_0b \\
  r_{\min}^2        & - \frac{2q_0q}{mv_0^2}r_{\min} - b^2 = 0                                 \\
  r_{\min}          & = \frac{q_0q}{mv_0^2} + \sqrt{\ab(\frac{q_0q}{mv_0^2})^2 + b^2}
\end{align}

次の関係式が成り立つ。
\begin{align}
  \tan\frac{\chi}{2} & = \frac{q_0q}{bmv_0^2} \\
  \frac{3}{2}k_BT    & = \frac{1}{2}mv_0^2
\end{align}
これを課題 1 の式に代入することで次のような関係式が成り立つ。
\begin{align}
  r_{\min} & = \frac{q_0q}{mv_0^2}(1 + \sqrt{1 + \tan^2\frac{\chi}{2}})
\end{align}
散乱角を $\chi\ll 1$ とすると $n \sim 10^{20}$ \si{m^{-3}} 温度 $T\sim 10$ \si{keV} より
\begin{align}
  r_{\min} = \frac{q_0q}{2mv_0^2} = \frac{q_0q}{2k_BT} = 5.0\times 10^{-5}\si{m}
\end{align}



\subsection{Maxwell-Boltzmann 分布}
統計力学より速度分布関数は熱平衡状態を特徴付ける系の熱速度 $v_t = \sqrt{k_BT/m}$ を用いて次のように書ける。
\begin{align}
  f(\bm{v}) & = n\ab(\frac{m}{2\pi k_BT})^{3/2}\exp\ab(-\frac{mv^2/2}{k_BT}) = \frac{n}{(2v_t^2\pi)^{3/2}}\exp\ab(-\frac{v^2}{2v_t^2})
\end{align}
これを積分すると
\begin{align}
  \int f(\bm{v}; t)\dl{\bm{v}} & = \frac{n}{(2v_t^2\pi)^{3/2}}\int\exp\ab(-\frac{v^2}{2v_t^2})\dl{\bm{v}} = n
\end{align}
より任意の物理量 $Q(\bm{v})$ の全速度空間の平均値は次のようになる。
\begin{align}
  \langle Q(\bm{v}) \rangle = \int Q(\bm{v})f(\bm{v}; t)\dl{\bm{v}}\bigg/ \int f(\bm{v}; t)\dl{\bm{v}} = \frac{1}{n}\int Q(\bm{v})f(\bm{v}; t)\dl{\bm{v}}
\end{align}
$\dl{\bm{v}} = 4\pi v^2\dl{v}$
\begin{align}
  \langle 1\rangle                        & = \frac{1}{(2v_t^2\pi)^{3/2}}\int\exp\ab(-\frac{v^2}{2v_t^2})\dl{\bm{v}} = 1                                                  \\
  \langle q\bm{v}\rangle                  & = \frac{1}{(2v_t^2\pi)^{3/2}}\int q\bm{v}\exp\ab(-\frac{v^2}{2v_t^2})\dl{\bm{v}} = \bm{0}                                     \\
  \ab\langle \frac{1}{2}mv^2\rangle       & = \frac{1}{(2v_t^2\pi)^{3/2}}\int\frac{1}{2}mv^2\exp\ab(-\frac{v^2}{2v_t^2})\dl{\bm{v}} = \frac{3}{2}mv_t^2 = \frac{3}{2}k_BT \\
  \ab\langle \frac{1}{2}mv^2\bm{v}\rangle & = \frac{1}{(2v_t^2\pi)^{3/2}}\int\frac{1}{2}mv^2\bm{v}\exp\ab(-\frac{v^2}{2v_t^2})\dl{\bm{v}} = \bm{0}                        \\
  \langle v^3\rangle                      & = \frac{1}{(2v_t^2\pi)^{3/2}}\int_0^\infty 4\pi v^5\exp\ab(-\frac{v^2}{2v_t^2})\dl{v} = 4\pi\ab(\frac{2v_t^2}{\pi})^{3/2}     \\
  \langle v^4\rangle                      & = \frac{1}{(2v_t^2\pi)^{3/2}}\int_0^\infty 4\pi v^6\exp\ab(-\frac{v^2}{2v_t^2})\dl{v} = 15v_t^4 = 15\frac{T^2}{m^2}
\end{align}
さらに電場が掛かっている状態のとき Boltzmann 分布
\begin{align}
  f(\bm{v}, \rr)                 & = n_0\ab(\frac{m}{2\pi k_BT})^{3/2}\exp\ab(-\frac{mv^2/2 + q\varphi(\rr)}{k_BT}) = \frac{n(\rr)}{(2v_t^2\pi)^{3/2}}\exp\ab(-\frac{v^2}{2v_t^2}) \\
  \int f(\bm{v}, \rr)\dl{\bm{v}} & = n(\rr) = n_0\exp\ab(-\frac{q\varphi(\rr)}{k_BT})
\end{align}
速度 $\bm{u}_0 = (0, 0, u_0)$ で移流している温度 $T$ のプラズマは次のように与えられる。
Debye 長の 2 乗程度大きく個別運動をしつつ、より大きなスケールでは集団振動していることが分かる。


\subsection{Debye 遮蔽}
電⼦は質量が軽い $m \ll 1$ として慣性項を無視、圧力 $p = nT$ を用いて電⼦温度は空間的に⼀様であるとすると
\begin{align}
  mn\diff{\bm{v}_e}{t} & = nq\ab(\EE + \frac{1}{c}\bm{v}\times\BB) - \nabla p \\
  0                    & = -nq\nabla\varphi(\rr) - T\nabla n                  \\
  n                    & = n_0\exp\ab(-\frac{q\varphi(\rr)}{T})               \\
  \varphi(\rr)         & = -\frac{T}{q}\ln\ab(\frac{n}{n_0})
\end{align}
$q = -e, Ze$ $m = m_e, M$, $Zn_i = n_e$
\begin{align}
  \nabla\varphi(\rr) = -\frac{T\nabla n}{nq} = \frac{T\nabla n_e}{n_ee} = -\frac{T\nabla n_i}{n_iZe} \\
  \delta n_i = - \frac{Ze\varphi}{T_i}n_{i0}
\end{align}
ポテンシャルとそれを構築する電荷分布は次のようになる。
\begin{align}
  \varphi(\rr) & = \frac{q_0}{r}\exp(-k_dr)                                                                                                    \\
  \rho(\rr)    & = -\frac{1}{4\pi}\nabla^2 \varphi(\rr) = -\frac{1}{r^2}\diffp{}{r}\ab(r^2\diffp{\varphi}{r}) = -\frac{k_d^2q_0}{r}\exp(-k_dr)
\end{align}
これより総電荷は次のようになる。
\begin{align}
  \int\rho(\rr)\dl{\rr} & = -k_d^2q_0\int\frac{\exp(-k_dr)}{r}\dl{\rr} = -q_0
\end{align}

\subsection{プラズマ振動}


\section{集団運動と個別運動}
\begin{align}
  \rho(\rr, t)  & = \sum_{i}\delta(\rr - \rr_i)                                             \\
  \rho_{\kk}(t) & = \int\rho(\rr, t)e^{-i\kk\cdot\rr}\dl{\rr} = \sum_{i}e^{-i\kk\cdot\rr_i}
\end{align}
\begin{align}
  \nabla^2\varphi(\rr_i) & = 4\pi e\rho(\rr_i) = 4\pi e\sum_{j \neq i}\delta(\rr_i - \rr_j) = 4\pi e\sum_{j \neq i}\frac{1}{(2\pi)^3}\int e^{i\kk\cdot(\rr_i - \rr_j)}\dl{\kk}        \\
  \nabla^2\varphi(\rr_i) & = \frac{1}{(2\pi)^3}\int\varphi(\kk)\nabla^2e^{i\kk\cdot\rr_i}\dl{\kk} = \frac{1}{(2\pi)^3}\int(-k^2)\varphi(\kk)e^{i\kk\cdot\rr_i}\dl{\kk}                \\
  \varphi(\kk)           & = -\frac{4\pi e}{k^2}\sum_{j \neq i}e^{-i\kk\cdot\rr_j}                                                                                                    \\
  \varphi(\rr)           & = \frac{1}{(2\pi)^3}\int\varphi(\kk)e^{i\kk\cdot\rr}\dl{\kk} = -\frac{4\pi e}{(2\pi)^3}\int\sum_{j \neq i} \frac{1}{k^2}e^{i\kk\cdot(\rr - \rr_j)}\dl{\kk} \\
                         & = -4\pi e\sum_{\kk}{}'\sum_{j \neq i} \frac{1}{k^2}e^{i\kk\cdot(\rr - \rr_j)}                                                                              \\
  \nabla\varphi(\rr)     & = -4\pi ei\sum_{\kk}{}'\sum_{j \neq i} \frac{\kk}{k^2}e^{i\kk\cdot(\rr - \rr_j)}
\end{align}
\begin{align}
  \dot{\rho}_{\kk}  & = -i\sum_{i}(\kk\cdot\bm{v}_i)e^{-i\kk\cdot\rr_i}                                                                                                                                            \\
  \ddot{\rho}_{\kk} & = -\sum_{i}[(\kk\cdot\bm{v}_i)^2 + i\kk\cdot\dot{\bm{v}}_i]e^{-i\kk\cdot\rr_i}                                                                                                               \\
                    & = -\sum_{i}(\kk\cdot\bm{v}_i)^2e^{-i\kk\cdot\rr_i} - \sum_{i}i\kk\cdot\ab(\frac{e\nabla\varphi(\rr_i)}{m})e^{-i\kk\cdot\rr_i}                                                                \\
                    & = -\sum_{i}(\kk\cdot\bm{v}_i)^2e^{-i\kk\cdot\rr_i} - \frac{4\pi e^2}{m}\sum_{i}\kk\cdot\ab(\sum_{\bm{q}}{}'\sum_{j\neq i}\frac{\bm{q}}{q^2}e^{i\bm{q}\cdot(\rr - \rr_j)})e^{-i\kk\cdot\rr_i} \\
                    & = -\sum_{i}(\kk\cdot\bm{v}_i)^2e^{-i\kk\cdot\rr_i} - \frac{4\pi e^2}{m}\sum_{i}\sum_{\bm{q}}{}'\rho_{\bm{q}}\ab(\frac{\kk\cdot\bm{q}}{q^2})e^{-i\kk\cdot\rr_i}e^{i\bm{q}\cdot\rr_i}          \\
                    & = -\frac{T}{m}k^2\rho_{\kk} - \frac{4\pi e^2}{m}\sum_{\bm{q}}{}'\ab(\frac{\kk\cdot\bm{q}}{q^2})\rho_{\bm{q}}\rho_{\kk - \bm{q}}                                                              \\
                    & = -\frac{T}{m}k^2\rho_{\kk} - \omega_p^2\rho_{\kk} - \frac{4\pi e^2}{m}\sum_{\bm{q}\neq\kk}{}'\ab(\frac{\kk\cdot\bm{q}}{q^2})\rho_{\bm{q}}\rho_{\kk - \bm{q}}
\end{align}
$\kk = \bm{q}$ のとき $\rho_0 = n_0$ であるから
\begin{align}
  \ddot\rho_{\kk} + \ab[\omega_p^2 + \frac{3k_BT}{m}k^2]\rho_{\kk} = - \frac{4\pi e^2}{m}\sum_{\bm{q} \neq \kk}^{}{}'\ab(\frac{\bm{k}\cdot\bm{q}}{q^2})\rho_{\bm{q}}\rho_{\bm{k} - \bm{q}}
\end{align}
つまり運動方程式において波数 $\bm{q}$ と $\bm{k} - \bm{q}$ の波から波数 $\bm{k}$ の波に相互作用する。
3 次元ではなく 1 次元であると仮定する以外で係数 3 は消すことが出来なかった。



\subsection{ガウス積分}
\begin{align}
  \int_{-\infty}^{\infty}\exp\ab(-\frac{x^2}{\alpha})\dl{x} = (\alpha\pi)^{1/2}
\end{align}


\end{document}