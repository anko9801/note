\RequirePackage{plautopatch}
\documentclass[uplatex,dvipdfmx,a4paper,11pt]{jlreq}
\usepackage{bxpapersize}
\usepackage[utf8]{inputenc}
\usepackage{fontenc}
\usepackage{lmodern}
\usepackage{otf}
\usepackage{amsmath}
\usepackage{amssymb}
\usepackage{amsthm}
\usepackage{ascmac}
% \usepackage[hyphens]{url}
\usepackage{physics}
\usepackage{braket}
\usepackage{verbatimbox}
\usepackage{bm}
\usepackage{url}
% \usepackage[dvipdfmx,hiresbb,final]{graphicx}
\usepackage{hyperref}
\usepackage{pxjahyper}
\usepackage{tikz}\usetikzlibrary{cd}
\usepackage{listings}
\usepackage{color}
\usepackage{mathtools}
\usepackage{xspace}
\usepackage{xy}
\usepackage{xypic}
%
\title{可換環論}
\author{Anko}
\makeatletter
%
\DeclareMathOperator{\lcm}{lcm}
\DeclareMathOperator{\Kernel}{Ker}
\DeclareMathOperator{\Image}{Im}
\DeclareMathOperator{\ch}{ch}
\DeclareMathOperator{\Aut}{Aut}
\DeclareMathOperator{\Log}{Log}
\DeclareMathOperator{\Arg}{Arg}
\DeclareMathOperator{\sgn}{sgn}
%
\newcommand{\CC}{\mathbb{C}}
\newcommand{\RR}{\mathbb{R}}
\newcommand{\QQ}{\mathbb{Q}}
\newcommand{\ZZ}{\mathbb{Z}}
\newcommand{\NN}{\mathbb{N}}
\newcommand{\FF}{\mathbb{F}}
\newcommand{\PP}{\mathbb{P}}
\newcommand{\GG}{\mathbb{G}}
\newcommand{\TT}{\mathbb{T}}
\newcommand{\calB}{\mathcal{B}}
\newcommand{\calF}{\mathcal{F}}
\newcommand{\ignore}[1]{}
\newcommand{\floor}[1]{\left\lfloor #1 \right\rfloor}
% \newcommand{\abs}[1]{\left\lvert #1 \right\rvert}
\newcommand{\lt}{<}
\newcommand{\gt}{>}
\newcommand{\id}{\mathrm{id}}
\newcommand{\rot}{\curl}
\renewcommand{\angle}[1]{\left\langle #1 \right\rangle}

\let\oldcite=\cite
\renewcommand\cite[1]{\hyperlink{#1}{\oldcite{#1}}}

\let\oldbibitem=\bibitem
\renewcommand{\bibitem}[2][]{\label{#2}\oldbibitem[#1]{#2}}

% theorem環境の設定
% - 冒頭に改行
% - 末尾にdiamond (amsthm)
\theoremstyle{definition}
\newcommand*{\newscreentheoremx}[2]{
  \newenvironment{#1}[1][]{
    \begin{screen}
    \begin{#2}[##1]
      \leavevmode
      \newline
  }{
    \end{#2}
    \end{screen}
  }
}
\newcommand*{\newqedtheoremx}[2]{
  \newenvironment{#1}[1][]{
    \begin{#2}[##1]
      \leavevmode
      \newline
      \renewcommand{\qedsymbol}{\(\diamond\)}
      \pushQED{\qed}
  }{
      \qedhere
      \popQED
    \end{#2}
  }
}
\newtheorem{theorem*}{定理}

\newqedtheoremx{theorem}{theorem*}
\newcommand*\newqedtheorem@unstarred[2]{%
  \newtheorem{#1*}[theorem*]{#2}
  \newqedtheoremx{#1}{#1*}
}
\newcommand*\newqedtheorem@starred[2]{%
  \newtheorem*{#1*}{#2}
  \newqedtheoremx{#1}{#1*}
}
\newcommand*{\newqedtheorem}{\@ifstar{\newqedtheorem@starred}{\newqedtheorem@unstarred}}

\newtheorem{sctheorem*}{定理}
\newscreentheoremx{sctheorem}{sctheorem*}
\newcommand*\newscreentheorem@unstarred[2]{%
  \newtheorem{#1*}[theorem*]{#2}
  \newscreentheoremx{#1}{#1*}
}
\newcommand*\newscreentheorem@starred[2]{%
  \newtheorem*{#1*}{#2}
  \newscreentheoremx{#1}{#1*}
}
\newcommand*{\newscreentheorem}{\@ifstar{\newscreentheorem@starred}{\newscreentheorem@unstarred}}

%\newtheorem*{definition}{定義}
%\newtheorem{theorem}{定理}
%\newtheorem{proposition}[theorem]{命題}
%\newtheorem{lemma}[theorem]{補題}
%\newtheorem{corollary}[theorem]{系}

\newqedtheorem{lemma}{補題}
\newqedtheorem{corollary}{系}
\newqedtheorem{example}{例}
\newqedtheorem{proposition}{命題}
\newqedtheorem{remark}{注意}
\newqedtheorem{thesis}{主張}
\newqedtheorem{notation}{記法}
\newqedtheorem{problem}{問題}
\newqedtheorem{algorithm}{アルゴリズム}

\newscreentheorem*{definition}{定義}

\renewenvironment{proof}[1][\proofname]{\par
  \normalfont
  \topsep6\p@\@plus6\p@ \trivlist
  \item[\hskip\labelsep{\bfseries #1}\@addpunct{\bfseries}]\ignorespaces\quad\par
}{%
  \qed\endtrivlist\@endpefalse
}
\renewcommand\proofname{証明}

\makeatother

\begin{document}
\maketitle
\tableofcontents
\clearpage

\section{環論}
\begin{definition}[環]
  集合 $A$ に2つの演算 $+$, $\times$ が定義されていて加法, 乗法に関してそれぞれ可換群, モノイドになるかつ分配法則を満たすとき $A$ を環という.
\end{definition}

\begin{proposition}
  $\forall a\in A\ 0a = a0 = 0$
\end{proposition}
\begin{proof}
  $0a = (0 + 0)a = 0a + 0a$ より $0a = 0$. 逆も同様.
\end{proof}

\begin{proposition}
  $1=0$ となる環 $\iff$ $0$ 以外の元のない自明な環.
\end{proposition}
\begin{proof}
  $a = 1a = 0a = 0$ より任意の元は $0$ となり自明な環となる. 逆は自明.
\end{proof}

具体例
\begin{enumerate}
  \item $\mathbb{Z}\subset\mathbb{Q}\subset\mathbb{R}\subset\mathbb{C}$ の和積は可換環となる.
  \item 行列 $M_n(\mathbb{R})$ は非可換環となる.
  \item 関数 $C^\infty(\mathbb{R})$ の和積は可換環となる.
  \item 群環 (有限群から可換環への写像の像の総和) の和積は環となる.
  \item 2次の環 $\mathbb{Z}[\sqrt{d}]$ は環になる.
\end{enumerate}

\begin{definition}[準同型・同型]
  $A$, $B$ を環, $\phi:A\to B$ を写像とする.
  \begin{enumerate}
    \item 任意の $x,y\in A$ に対し $\phi(x+y) = \phi(x) + \phi(y)$, $\phi(xy) = \phi(x)\phi(y)$ が成り立ち, $\phi(1_A) = 1_B$ であるとき, $\phi$ を準同型という.
    \item $\phi$ が準同型で逆写像が存在し, 逆写像も準同型であるとき, $\phi$ は同型であるという. また, このとき, $A$, $B$ は同型であるといい, $A\cong B$ と書く.
    \item $A = B$ なら準同型・同型を自己準同型・自己同型という. 環 $A$ の自己同型全体の集合を $\mathrm{Aut}^{\mathrm{al}}A$ と書く.
  \end{enumerate}
\end{definition}


\begin{proposition}
  $\phi:A\to B$ が環の準同型なら $\phi(0_A) = 0_B$ である.
\end{proposition}
\begin{proof}
  $\phi(0_A) = \phi(0_A + 0_A) = \phi(0_A) + \phi(0_A)$ より $\phi(0_A) = 0_B$ となる.
\end{proof}

\begin{proposition}
  $A$, $B$, $C$ を環, $\phi:A\to B$, $\psi:B\to C$ を準同型とするとき, その合成 $\phi\circ\psi:A\to C$ も準同型である. 同様に $\phi$, $\psi$ が同型なら, $\phi\circ\psi$ も同型である.
\end{proposition}
\begin{proof}
  $\psi\circ\phi(x + y) = \psi(\phi(x + y)) = \psi(\phi(x) + \phi(y)) = \psi(\phi(x)) + \psi(\phi(y)) = \psi\circ\phi(x) + \psi\circ\phi(y)$ となり, $\psi\circ\phi(xy) = \psi\circ\phi(x)\psi\circ\phi(y)$ や $\psi\circ\phi(1_A) = 1_C$ も同様に示せるから $\psi\circ\phi$ は準同型である. 同型も同様.
\end{proof}

\begin{proposition}
  $\phi:A\to B$ が環の準同型ならば, 単射 $\iff$ $\Kernel{\phi} = \qty{0}$
\end{proposition}
\begin{proof}
  ($\implies$) $\phi$ が環の準同型であるから $\phi(0_A) = 0_B$ より $0_A\in\Kernel\phi$. また元 $\forall x, y \in \Kernel\phi$ について $\phi$ の単射性より $\phi(x) = \phi(y) \implies x = y$ となり, $\Kernel\phi$ には $0$ 以外の元は存在しない. \\
  ($\impliedby$) $\phi(x) = \phi(y)$ となる $x, y$ について
  \begin{align}
    1 & = \phi(x)\phi(y)^{-1} = \phi(x)\phi(y^{-1}) = \phi(xy^{-1}) \\
    1 & = xy^{-1}
  \end{align}
  より $x = y$ となるから $\phi$ は単射である.
\end{proof}

\begin{definition}[$n$ 変数多項式]
  $A$ 係数あるいは $A$ 上の $n$ 変数 $x = (x_1,\cdots,x_n)$ の多項式とは, $\mathbb{N}^n$ から $A$ への写像で有限個の $(i_1,\cdots,i_n)\in\mathbb{N}^n$ を除いて値が $0$ になるものと, 変数 $x = (x_1,\cdots,x_n)$ の組のことである. この写像の $(i_1,\cdots,i_n)\in\mathbb{N}$ での値が $a_{i_1,\cdots,i_n}$ なら, この多項式を
  $$
    f(x) = f(x_1,\cdots,x_n) = \sum_{i_1,\cdots,i_n\geq 0}a_{i_1,\cdots,i_n}x_1^{i_1}\cdots x_n^{i_n}
  $$
  などと書く. すべての $a_{i_1,\cdots,i_n}$ が $0$ である多項式を $0$ と書く. 各 $a_{i_1,\cdots,i_n}x_1^{i_1}\cdots x_n^{i_n}$ を $f(x)$ の項, $a_{i_1,\cdots,i_n}$ を係数という. 特に $a_{0,\cdots,0}$ を $f(x)$ の定数項という.
\end{definition}
\begin{definition}[$n$ 変数多項式の代入]
  $c = (c_1,\cdots,c_n) \in A^n$ とするとき
  $$
    f(c) = f(c_1,\cdots,c_n) = \sum_{i_1,\cdots,i_n}a_{i_1,\cdots,i_n}c_1^{i_1}\cdots c_n^{i_n}
  $$
  とする. この値を考えることを代入という.
\end{definition}
\begin{definition}[$n$ 変数多項式の次数]
  $f(x)$ の次数 $\deg f(x)$ を
  $$
    \deg f(x) = \begin{cases}
      \max\lbrace i_1 + \cdots + i_n \mid a_{i_1,\cdots,i_n} \neq 0 \rbrace & (f(x) \neq 0) \\
      -\infty                                                               & (f(x) = 0)
    \end{cases}
  $$
  と定義する.
\end{definition}
\begin{definition}[$A$ 係数あるいは $A$ 上の $n$ 変数多項式環]
  2つの $n$ 変数多項式
  $$
    f(x) = \sum_{i_1,\cdots,i_n}a_{i_1,\cdots,i_n}x_1^{i_1}\cdots x_n^{i_n}, \quad g(x) = \sum_{i_1,\cdots,i_n}b_{i_1,\cdots,i_n}x_1^{i_1}\cdots x_n^{i_n}
  $$
  は, $a_{i_1,\cdots,i_n} = b_{i_1,\cdots,i_n}$ がすべての $i_1,\cdots,i_n$ に対して成り立つとき多項式の同値関係 $f(x) = g(x)$ であると定義する. また次のように多項式の和差積を定義する.
  \begin{align}
    (f\pm g)(x) & = \sum_{i_1,\cdots,i_n}(a_{i_1,\cdots,i_n}\pm b_{i_1,\cdots,i_n})x_1^{i_1}\cdots x_n^{i_n}   \\
    f(x)g(x)    & = \sum_{i_1,\cdots,j_n}a_{i_1,\cdots,i_n}b_{j_1,\cdots,j_n}x_1^{i_1+j_1}\cdots x_n^{i_n+j_n}
  \end{align}
  すると多項式全体の集合 $A[x]$ は環となり, $A$ 係数あるいは $A$ 上の $n$ 変数多項式環という.
\end{definition}

\begin{definition}[無限変数多項式環]
  無限変数多項式環 $A[x_i]_{i\in I}$ とは $n > 0$ を整数とするとき, $X_n$ を $\mathbb{N}^n$ から $A$ への写像 $a$ で有限個の $(i_1,\cdots,i_n)\in\mathbb{N}^n$ を除いて値が $0$ であるものと $\lbrace1,\cdots,n\rbrace$ から $I$ への単射写像 $\phi$ の組全体の集合とする. $X_n$ には $\mathfrak{S}_n$ が作用し, その軌道の集合を $Y_n$ とする. $(a,\phi)\in X_n$ で代表される $Y_n$ の元に対し,
  $$\sum_{i_1,\cdots,i_n\in\mathbb{N}}a(i_1,\cdots,i_n)x_{\phi(1)}^{i_1}\cdots x_{\phi(n)}^{i_n}$$
  と書く. これは代表元のとりかたによらず定まる. $\lbrace Y_n\rbrace_n$ は集合族となり, $n\leq m$ なら $Y_n \subseteq Y_m$ とみなせる. $A[x_i]_{i\in I} = \bigcup_n Y_n$ と定義すればよい. $A[x_i]_{i\in I}$ が集合として存在するときそれを無限変数多項式環という.
\end{definition}

\begin{theorem}
  $$
    A[x_1,\cdots,x_n]\cong A[x_1,\cdots,x_{n-1}][x_n]
  $$
\end{theorem}
\begin{proof}
  $$
    f(x_1,\cdots,x_n) = \sum_{i_n}\left(\sum_{i_1,\cdots,i_{n-1}}a_{i_1,\cdots,i_n}x_1^{i_1}\cdots x_{n-1}^{i_{n-1}}\right)x_n^{i_n}
  $$
\end{proof}

\begin{definition}[環の鎖] a
  \begin{enumerate}
    \item 環: 加法が可換群, 乗法がモノイドであり, また分配法則を満たす.
    \item 可換環: 環について乗法が可換である.
    \item 整域: 可換環 $A$ について任意の $a,b\in A\backslash\lbrace0\rbrace$ に対し, $ab\neq 0$ となる.
    \item 正規環: 整域 $A$ について商体 $K$ の元が $A$ 上整なら $A$ の元となる.
    \item 一意分解環 (UFD): 整域について任意の元は素元分解できる.
    \item 単項イデアル整域 (PID): 整域について任意のイデアルが単項イデアルである.
    \item ユークリッド環: 整域 $A$ について写像 $d:A\backslash\lbrace0\rbrace\to\mathbb{N}$ があり, $a,b\in A$ で $b\neq 0$ なら, $q,r\in A$ があり, $a = qb + r$ で $r = 0$ または $d(r)<d(b)$ となる.
    \item 体: 可換環 $A$ の乗法が $A\backslash\lbrace0\rbrace$ において可換群となる.
  \end{enumerate}
\end{definition}

\begin{theorem}[環の鎖]
  環$\impliedby$可換環$\impliedby$整域$\impliedby$正規環$\impliedby$ UFD $\impliedby$ PID $\impliedby$ユークリッド環$\impliedby$体
\end{theorem}
\begin{proof}
  (可換環$\implies$環) 自明. \\
  (整域$\implies$可換環) 自明. \\
  (正規環$\implies$整域) 自明. \\
  (UFD$\implies$正規環) $\alpha\in K$ を解に持つモニック多項式 $f(x) = x^n+\cdots+a_1x + a_0$ について $a_0 \neq 0$ とすると, $\alpha \neq 0$ である. ここで $\alpha$ を既約分数として $\alpha = \beta/\gamma$ と表すと, $\gamma^nf(\alpha) = \beta^n + a_{n-1}\gamma\beta^{n-1} + \cdots + a_0\gamma^n = 0$ より $\beta^n = -\gamma(a_1\beta^{n-1} + \cdots + a_n\gamma^{n-1})$ なので, $\gamma\in A^\times$ となる. よって $\alpha\in A$ である. \\
  (PID$\implies$UFD) \\
  (ユークリッド環$\implies$PID) あるイデアル $I\subseteq A$ に対し, $x = \min\lbrace d(y)\mid I\ni y \neq 0\rbrace$ とおくと $I = (x)$ となることを示す. イデアル $I$ の元 $\forall z=qx+r\in I$ について $r=0\lor d(r)<d(x)$ であり, $0 = d(r) < d(x)$ より $r = 0$ となる. よって $z = qx \in (x)$ となるので $I = (x)$ である. \\
  (体$\implies$ユークリッド環)
\end{proof}


\begin{proposition}
  部分環に性質が引き継がれる.
  \begin{enumerate}
    \item 環の部分環は環である.
    \item 可換環の部分環は可換環である.
    \item 整域の部分環は整域である.
    \item 正規環の部分環は正規環である.
    \item UFDの部分環はUFDである.
    \item PIDの部分環はPIDではない?
    \item ユークリッド環の部分環はユークリッド環ではない?
    \item 体の部分環は体ではない.
  \end{enumerate}
\end{proposition}
\begin{proof}
  それぞれ証明する. 反例を挙げる.
  \begin{enumerate}
    \item 定義から自明.
    \item 任意の元 $a,b\in A$ について可換ならばその部分集合も成り立つ.
    \item
    \item
    \item
    \item
    \item
    \item 有理数体 $\mathbb{Q}$ の部分環 $\mathbb{Z}$ は体ではない.
  \end{enumerate}
\end{proof}

\begin{proposition}[素イデアルと極大イデアルの関係]
  素イデアル
  \begin{enumerate}
    \item $A$ が環なら, $A$ の任意の極大イデアルは素イデアルである.
    \item $A$ が単項イデアル整域なら, $(0)$ でない任意の素イデアルは極大イデアルである. したがって, $p$ が素元なら, $A/(p)$ は体である.
  \end{enumerate}
\end{proposition}
\begin{proposition}[素元と既約元の関係]
  素元
  \begin{enumerate}
    \item $A$ が整域なら, $A$ の素元は既約元である.
    \item $A$ が一意分解環なら, $A$ の既約元は素元である.
  \end{enumerate}
\end{proposition}
\begin{proof}

\end{proof}

\begin{proposition}
  体の多項式環はユークリッド環である.
\end{proposition}
\begin{proof}
  $d = \deg$ とすると成り立つ.
\end{proof}

\begin{proposition}[正規環]
  $f(x) = a_nx^n + \cdots + a_0\in A[x]$ で $a_0, a_n \neq 0$, $\alpha\in K$
\end{proposition}

\begin{definition}
  \begin{enumerate}
    \item ネーター環
    \item アルティン環
  \end{enumerate}
\end{definition}

\section{加群}
\begin{definition}
  環 $R$ 上の行列の集合について定義する.
  \begin{enumerate}
    \item $m\times n$ 行列の集合を $M_{m,n}(R)$.
    \item $n$ 次正方行の集合を $M_n(R)$.
    \item $M_n(R)$ の乗法群(正則行列の集合)を一般線形群 $GL_n(R)$.
    \item $GL_n(R)$ の $\det$ の核 (行列式の値が単位元) を特殊線形群 $SL_n(R)$.
  \end{enumerate}
\end{definition}


\end{document}
