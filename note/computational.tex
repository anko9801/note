\RequirePackage{plautopatch} % upLaTeX 用パッチを当てる
\documentclass[uplatex,dvipdfmx,a4paper,11pt]{jlreq}
\usepackage{bxpapersize} % A4判サイズを指定する
\usepackage[utf8]{inputenc} %
\usepackage[T1]{fontenc} % フォントエンコーディング指定
\usepackage{lmodern} % Latin Modern フォント
\usepackage{otf}
\usepackage{amsmath} % 数式環境
\usepackage{amssymb} % 数学記号やフォント
\usepackage{amsthm} % 定理環境
\usepackage{ascmac}
\usepackage{mathtools} % amsmath の拡張
\usepackage{siunitx} % 単位
\usepackage{physics}
% \usepackage{physics2}
\usepackage{braket} % braket 記法
\usepackage{verbatimbox}
\usepackage{bm} % ベクトルで使われる太字斜体
\usepackage[hyphens]{url}
% \usepackage[dvipdfmx,hiresbb,final]{graphicx} % 画像の挿入
\usepackage[dvipdfmx]{graphicx}
\usepackage{indentfirst} % 最初の段落にインデント
\usepackage{wrapfig} % 表や画像の周りに文字を回り込ませる
\usepackage{comment} % コメント環境
\usepackage{docmute} % ファイル分割
\usepackage{hyperref} % ハイパーリンクを埋め込む
\usepackage{pxjahyper}
\usepackage{tikz} % グラフや図形を描く
\usetikzlibrary{cd}
\usetikzlibrary{intersections, calc, arrows, positioning, arrows.meta,automata}
\usepackage{listings} % ソースコードの挿入
\lstset{
  language=C++,
  breaklines=true,
  keywordstyle = {\color[rgb]{0,0,1}},
  stringstyle = {\color[rgb]{1,0,0}},
  commentstyle = { \color[rgb]{0,1,0}},
  numbers=left,
  frame=lines
}
\usepackage{color}
\usepackage{xspace}
\usepackage{xy}
\usepackage{xypic}
%
\title{計算物理}
\author{Anko}
\makeatletter
%
\DeclareMathOperator{\lcm}{lcm}
\DeclareMathOperator{\Kernel}{Ker}
\DeclareMathOperator{\Image}{Im}
\DeclareMathOperator{\ch}{ch}
\DeclareMathOperator{\Aut}{Aut}
\DeclareMathOperator{\Log}{Log}
\DeclareMathOperator{\Arg}{Arg}
\DeclareMathOperator{\sgn}{sgn}
%
\newcommand{\CC}{\mathbb{C}}
\newcommand{\RR}{\mathbb{R}}
\newcommand{\QQ}{\mathbb{Q}}
\newcommand{\ZZ}{\mathbb{Z}}
\newcommand{\NN}{\mathbb{N}}
\newcommand{\FF}{\mathbb{F}}
\newcommand{\PP}{\mathbb{P}}
\newcommand{\GG}{\mathbb{G}}
\newcommand{\TT}{\mathbb{T}}
\newcommand{\calB}{\mathcal{B}}
\newcommand{\calF}{\mathcal{F}}
\newcommand{\ignore}[1]{}
\newcommand{\floor}[1]{\left\lfloor #1 \right\rfloor}
% \newcommand{\abs}[1]{\left\lvert #1 \right\rvert}
\newcommand{\lt}{<}
\newcommand{\gt}{>}
\newcommand{\id}{\mathrm{id}}
\newcommand{\rot}{\curl}
\renewcommand{\angle}[1]{\left\langle #1 \right\rangle}
\newcommand{\EE}{\bm{E}}
\newcommand{\BB}{\bm{B}}
\renewcommand{\AA}{\bm{A}}
\newcommand{\rr}{\bm{r}}
\newcommand{\kk}{\bm{k}}
\newcommand{\pp}{\bm{p}}

\let\oldcite=\cite
\renewcommand\cite[1]{\hyperlink{#1}{\oldcite{#1}}}

\let\oldbibitem=\bibitem
\renewcommand{\bibitem}[2][]{\label{#2}\oldbibitem[#1]{#2}}

% theorem環境の設定
% - 冒頭に改行
% - 末尾にdiamond (amsthm)
\theoremstyle{definition}
\newcommand*{\newscreentheoremx}[2]{
  \newenvironment{#1}[1][]{
    \begin{screen}
    \begin{#2}[##1]
      \leavevmode
      \newline
  }{
    \end{#2}
    \end{screen}
  }
}
\newcommand*{\newqedtheoremx}[2]{
  \newenvironment{#1}[1][]{
    \begin{#2}[##1]
      \leavevmode
      \newline
      \renewcommand{\qedsymbol}{\(\diamond\)}
      \pushQED{\qed}
  }{
      \qedhere
      \popQED
    \end{#2}
  }
}
\newtheorem{theorem*}{定理}

\newqedtheoremx{theorem}{theorem*}
\newcommand*\newqedtheorem@unstarred[2]{%
  \newtheorem{#1*}[theorem*]{#2}
  \newqedtheoremx{#1}{#1*}
}
\newcommand*\newqedtheorem@starred[2]{%
  \newtheorem*{#1*}{#2}
  \newqedtheoremx{#1}{#1*}
}
\newcommand*{\newqedtheorem}{\@ifstar{\newqedtheorem@starred}{\newqedtheorem@unstarred}}

\newtheorem{sctheorem*}{定理}
\newscreentheoremx{sctheorem}{sctheorem*}
\newcommand*\newscreentheorem@unstarred[2]{%
  \newtheorem{#1*}[theorem*]{#2}
  \newscreentheoremx{#1}{#1*}
}
\newcommand*\newscreentheorem@starred[2]{%
  \newtheorem*{#1*}{#2}
  \newscreentheoremx{#1}{#1*}
}
\newcommand*{\newscreentheorem}{\@ifstar{\newscreentheorem@starred}{\newscreentheorem@unstarred}}

%\newtheorem*{definition}{定義}
%\newtheorem{theorem}{定理}
%\newtheorem{proposition}[theorem]{命題}
%\newtheorem{lemma}[theorem]{補題}
%\newtheorem{corollary}[theorem]{系}

\newqedtheorem{lemma}{補題}
\newqedtheorem{corollary}{系}
\newqedtheorem{example}{例}
\newqedtheorem{proposition}{命題}
\newqedtheorem{remark}{注意}
\newqedtheorem{thesis}{主張}
\newqedtheorem{notation}{記法}
\newqedtheorem{problem}{問題}
\newqedtheorem{algorithm}{アルゴリズム}

\newscreentheorem*{definition}{定義}

\renewenvironment{proof}[1][\proofname]{\par
  \normalfont
  \topsep6\p@\@plus6\p@ \trivlist
  \item[\hskip\labelsep{\bfseries #1}\@addpunct{\bfseries}]\ignorespaces\quad\par
}{%
  \qed\endtrivlist\@endpefalse
}
\renewcommand\proofname{証明}

\makeatother

\begin{document}
\maketitle
\tableofcontents
\clearpage

\section{差分法}
\begin{definition}
  連続な時間 $t$ に対し、離散化された時間のパラメータは微小時間 $\Delta t$ を用いて $t_n = n\Delta t$ と書く。同様に空間についても $x_i = i\Delta x$ とする。
  時間・空間について連続な関数 $f(x, t)$ に対する離散値を $f_i^{(n)} := f(x_i, t_n)$ と書く。
  空間について $n$ 次精度とは $\mathcal{O}(\Delta x^n)$ となること
\end{definition}



\begin{theorem}[1階微分の差分公式]
  1次前進差分、1次後退差分、2次中心差分は次のように求まる。
  \begin{align}
    f_i' & \approx \frac{f_{i+1} - f_i}{\Delta x} + \mathcal{O}(\Delta x)        \\
    f_i' & \approx \frac{f_i - f_{i-1}}{\Delta x} + \mathcal{O}(\Delta x)        \\
    f_i' & \approx \frac{f_{i+1} - f_{i-1}}{2\Delta x} + \mathcal{O}(\Delta x^2)
  \end{align}
\end{theorem}
\begin{proof}
  $f(x \pm \Delta x)$ を 2 次まで Taylor 展開すると
  \begin{align}
    f(x \pm \Delta x) & = f_{i \pm 1} = f_i \pm f_i'\Delta x + \frac{1}{2}f_i''\Delta x^2 + \mathcal{O}(\Delta x^3)
  \end{align}
  より次の式が成り立つ。
  \begin{align}
    f_{i + 1} - f_i       & = f_i'\Delta x + \mathcal{O}(\Delta x^2)  \\
    f_i - f_{i - 1}       & = f_i'\Delta x + \mathcal{O}(\Delta x^2)  \\
    f_{i + 1} - f_{i - 1} & = 2f_i'\Delta x + \mathcal{O}(\Delta x^3)
  \end{align}
  よって微分は
  \begin{align}
    f_i' & \approx \frac{f_{i+1} - f_i}{\Delta x} + \mathcal{O}(\Delta x)        \\
    f_i' & \approx \frac{f_i - f_{i-1}}{\Delta x} + \mathcal{O}(\Delta x)        \\
    f_i' & \approx \frac{f_{i+1} - f_{i-1}}{2\Delta x} + \mathcal{O}(\Delta x^2)
  \end{align}
\end{proof}

\begin{theorem}[2次精度2階微分の差分公式]
  2次中心差分は次のように書ける。
  \begin{align}
    f''(x_n) & \approx \frac{f(x_{n+1}) - 2f(x_n) + f(x_{n-1})}{\Delta x^2} + \mathcal{O}(\Delta x^2)
  \end{align}
\end{theorem}
\begin{proof}
  $f(x \pm \Delta x)$ を 3 次まで Taylor 展開すると
  \begin{align}
    f(x \pm \Delta x) & = f_{i \pm 1} = f_i \pm f_i'\Delta x + \frac{1}{2}f_i''\Delta x^2 \pm \frac{1}{3!}f_i'''\Delta x^3 + \mathcal{O}(\Delta x^4)
  \end{align}
  となりこの和について計算することで求まる。
  \begin{align}
    f_{i + 1} + f_{i - 1} & = 2f_i + f_i''\Delta x^2 + \mathcal{O}(\Delta x^4)                          \\
    f_i''                 & = \frac{f_{i + 1} - 2f_i + f_{i - 1}}{\Delta x^2} + \mathcal{O}(\Delta x^2)
  \end{align}
\end{proof}

\begin{theorem}[2階微分の差分公式]
  $f(x\pm k\Delta x)$ の Taylor 展開で 2 次以外の項を相殺することで次の式が得られる.
  \begin{align}
    f''(x_n) & \approx \frac{-f(x_{n+2}) + 16f(x_{n+1}) - 30f(x_n) + 16f(x_{n-1}) - f(x_{n-2})}{12\Delta x^2} + \mathcal{O}(\Delta x^4)
  \end{align}
\end{theorem}

\subsection{安全性解析}
\begin{definition}[フォン・ノイマン (von Neumann) の安定性解析]
  時間が経つと共に振幅が増大しないことは安定性の条件となる。
  \begin{align}
    f(x, t)                             & = \sum_kA_k(t)e^{ikx} \\
    \qty|\frac{A_k^{(n+1)}}{A_k^{(n)}}| & \leq 1
  \end{align}
\end{definition}

\begin{definition}[CFL (Courant-Friedrichs-Lewy) 条件]
  計算上の情報の伝播する速さより物理的な情報の伝播する速さの方が小さい。
  \begin{align}
    \frac{\Delta x}{\Delta t} \geq c \\
    \nu := \frac{c\Delta t}{\Delta x} \leq 1
  \end{align}
\end{definition}

\begin{definition}[拡散方程式]
  \begin{align}
    \pdv{f(x, t)}{t} & = \kappa\pdv[2]{f(x, t)}{x}
  \end{align}
\end{definition}

\begin{theorem}
  \begin{align}
    f_i^{(n+1)} & = f_i^{(n)} + \frac{\kappa\Delta t}{\Delta x^2}\qty(f_{i+1}^{(n)} - 2f_i^{(n)} + f_{i-1}^{(n)})
  \end{align}
\end{theorem}
\begin{proof}
  拡散方程式は時間について(1次)前進差分、空間について(2次)中心差分を取る。これを FTCS (Forward Time Centered Space) スキームという。
  \begin{align}
    \frac{f_i^{(n+1)} - f_i^{(n)}}{\Delta t} & = \kappa\frac{f_{i+1}^{(n)} - 2f_i^{(n)} + f_{i-1}^{(n)}}{\Delta x^2}                           \\
    f_i^{(n+1)}                              & = f_i^{(n)} + \frac{\kappa\Delta t}{\Delta x^2}\qty(f_{i+1}^{(n)} - 2f_i^{(n)} + f_{i-1}^{(n)})
  \end{align}
  このように時間発展を求められる。これは安全性を満たす。
  \begin{align}
    \frac{A^{(n+1)}e^{ikx_i} - A^{(n)}e^{ikx_i}}{\Delta t} & = \kappa\frac{A^{(n)}e^{ikx_{i+1}} - 2A^{(n)}e^{ikx_{i}} + A^{(n)}e^{ikx_{i-1}}}{\Delta x^2} \\
    \qty|\frac{A_k^{(n+1)}}{A_k^{(n)}}|                    & = \qty|1 + \frac{\kappa\Delta t}{\Delta x^2}(e^{ik\Delta x} - 2 + e^{-ik\Delta x})|          \\
                                                           & = \qty|1 - \frac{2\kappa\Delta t}{\Delta x^2}(1 - \cos(k\Delta x))| \leq 1
  \end{align}
\end{proof}

\begin{definition}[移流方程式]
  \begin{align}
    \pdv{f(x, t)}{t} & = -c\pdv{f(x, t)}{x}
  \end{align}
\end{definition}

\begin{theorem}
  移流方程式は時間について(1次)前進差分、空間について(2次)中心差分を取ると安全性を満たさない。 \\
  時間について(1次)前進差分、空間について(1次)後退差分を取ると安全性を満たす。
  \begin{align}
    \frac{f_i^{(n+1)} - f_i^{(n)}}{\Delta t} & = - c\frac{f_i^{(n)} - f_{i-1}^{(n)}}{\Delta x}
  \end{align}
\end{theorem}
\begin{proof}
  時間について(1次)前進差分、空間について(2次)中心差分を取る。
  \begin{align}
    \frac{f_i^{(n+1)} - f_i^{(n)}}{\Delta t} & = -c\frac{f_{i+1}^{(n)} - f_{i-1}^{(n)}}{2\Delta x}
  \end{align}
  安全性解析すると
  \begin{align}
    \qty(A(t_{n+1}) - A(t_n))\frac{e^{ikx_i}}{\Delta t} & = -c\qty(e^{ik\Delta x} - e^{-ik\Delta x})\frac{A(t_n)e^{ikx_i}}{2\Delta x}      \\
    \qty|\frac{A(t_{n+1})}{A(t_n)}|                     & = \sqrt{\qty|1 - i\nu\sin(k\Delta x)|^2} = \sqrt{1 + \nu^2\sin^2(k\Delta x)} > 1
  \end{align}
  CFL 条件
  \begin{align}
    \frac{\Delta x}{\Delta t} & = -c\frac{f_{i+1}^{(n)} - f_{i-1}^{(n)}}{2(f_i^{(n+1)} - f_i^{(n)})} \geq c
  \end{align}
  1次後退差分で離散化したものについて
  \begin{align}
    \frac{f_i^{(n+1)} - f_i^{(n)}}{\Delta t} & = -c\frac{f_i^{(n)} - f_{i-1}^{(n)}}{\Delta x}
  \end{align}
  安定性解析すると
  \begin{align}
    \qty(A(t_{n+1}) - A(t_n))\frac{e^{ikx_i}}{\Delta t} & = -c\qty(1 - e^{-ik\Delta x})\frac{A(t_n)e^{ikx_i}}{\Delta x} \\
    \qty|\frac{A(t_{n+1})}{A(t_n)}|                     & = \sqrt{\qty|1 - \nu(1 - e^{-ik\Delta x})|^2}                 \\
                                                        & = \sqrt{1 - 2\nu(1 - \nu)(1 - \cos(k\Delta x))} \leq 1        \\
    \iff \nu                                            & \leq 1
  \end{align}
\end{proof}
これは差分により拡散項が増えてしまったからである。



\begin{definition}[Navier-Stokes 方程式]
  \begin{align}
    \pdv{\bm{v}}{t} & = -(\bm{v}\vdot\vnabla)\bm{v} - \frac{1}{\rho}\vnabla p + \nu\vnabla^2\bm{v} + \bm{F}
  \end{align}
  まず無次元量とする。
  \begin{align}
    \bm{v} & = V\tilde{\bm{v}} & \bm{r} & = L\tilde{\bm{r}} & t & = \frac{L}{V}\tilde{t} & p & = \rho V^2\tilde{p} & \vnabla & = \frac{1}{L}\tilde{\vnabla} & \pdv{t} & = \frac{V}{L}\pdv{\tilde{t}} & \mathrm{Re} & = \frac{UV}{\nu}
  \end{align}
  レイノルズ数 $\mathrm{Re}$ は慣性力と粘性力の比に対応する。
  \begin{align}
    \pdv{\tilde{\bm{v}}}{\tilde{t}} & = -(\tilde{\bm{v}}\vdot\tilde{\vnabla})\tilde{\bm{v}} - \tilde{\vnabla}\tilde{p} + \frac{1}{\mathrm{Re}}\tilde{\vnabla}^2\tilde{\bm{v}}
  \end{align}
  これ以降、無次元量を表すチルダは略す。
\end{definition}
\begin{theorem}
  Navier-Stokes 方程式
\end{theorem}
\begin{proof}
  $\omega = \vnabla\cross\bm{v}$ とすると
  \begin{align}
    \pdv{\omega_i}{t} & = \pdv{t}(\vnabla\cross\bm{v})_i                                                                                                                        \\
                      & = -\vnabla\cross((\bm{v}\vdot\vnabla)\bm{v}) - \vnabla\cross\vnabla p + \frac{1}{\mathrm{Re}}\vnabla\cross\laplacian\bm{v}                              \\
                      & = -\varepsilon_{ijk}\partial_j((v_l\partial_l)v_k) + \frac{1}{\mathrm{Re}}\varepsilon_{ijk}\partial_j\partial_l\partial_lv_k                            \\
                      & = -\varepsilon_{ijk}(\partial_jv_l\partial_lv_k + v_l\partial_j\partial_lv_k) + \frac{1}{\mathrm{Re}}\partial_l\partial_l\varepsilon_{ijk}\partial_jv_k \\
                      & = -\varepsilon_{ijk}\partial_jv_l\partial_lv_k - (\bm{v}\vdot\vnabla)(\vnabla\cross\bm{v}) + \frac{1}{\mathrm{Re}}\laplacian(\vnabla\cross\bm{v})       \\
                      & = -\varepsilon_{ijk}\partial_jv_l\partial_lv_k - (\bm{v}\vdot\vnabla)\omega + \frac{1}{\mathrm{Re}}\laplacian\omega
  \end{align}
  $\bm{v}(x, y, t) = (u(x, y, t), v(x, y, t), 0)$ とするとき、次の渦度方程式となる。
  \begin{align}
    \pdv{\omega_3}{t} & = - (\bm{v}\vdot\vnabla)\omega_3 + \frac{1}{\mathrm{Re}}\laplacian\omega_3
  \end{align}
  ここでは非圧縮性液体 $\vnabla\vdot\bm{v} = 0$ のとき $\bm{v} = \vnabla\cross\Phi$ と書ける。ここで流れ関数 $\Phi$ を導入する。
  \begin{align}
    u      & = - \pdv{\Phi}{y} \quad v = \pdv{\Phi}{x}              \\
    \omega & = \pdv[2]{\Phi}{x} + \pdv[2]{\Phi}{y} = \laplacian\Phi \\
    \omega & = \vnabla\cross(\vnabla\cross\Phi)                     \\
           & = \vnabla\vdot(\vnabla\Phi) - \laplacian\Phi
  \end{align}
  流れ関数を用いると解くべき方程式は渦度方程式とポアソン方程式に分けることができる。
  \begin{align}
    \pdv{\omega}{t} & = \pdv{\Phi}{x}\pdv{\omega}{y} - \pdv{\Phi}{y}\pdv{\omega}{x} + \frac{1}{\mathrm{Re}}\laplacian\omega \\
    \laplacian\Phi  & = - \omega
  \end{align}
\end{proof}

\begin{theorem}
  ポアソン方程式については差分法で解ける。
  \begin{align}
     & \laplacian\Phi = - \omega                                                                                                                  \\
     & \frac{\Phi_{i+1,j} - 2\Phi_{i,j} + \Phi_{i-1,j}}{\Delta x^2} + \frac{\Phi_{i,j+1} - 2\Phi_{i,j} + \Phi_{i,j-1}}{\Delta y^2} = \omega_{i,j} \\
     & \Phi_{i,j} = \frac{1}{4}\qty(\Phi_{i+1,j} + \Phi_{i-1,j} + \Phi_{i,j+1} + \Phi_{i,j-1} - \Delta x^2\omega_{i,j})
  \end{align}
  初期値を適当にセットして、この漸化式を収束するまで繰り返し更新する。この漸化式を改良させて収束を早めることができる。次の漸化式を順にヤコビ法, ガウス・ザイデル法, SOR 法という。
  \begin{align}
    \Phi_{i,j}^{\mathrm{new}} & = \frac{1}{4}\qty(\Phi_{i+1,j} + \Phi_{i-1,j} + \Phi_{i,j+1} + \Phi_{i,j-1} - \Delta x^2\omega_{i,j})                                                                                    \\
    \Phi_{i,j}^{\mathrm{new}} & = \frac{1}{4}\qty(\Phi_{i+1,j} + \Phi_{i-1,j}^{\mathrm{new}} + \Phi_{i,j+1} + \Phi_{i,j-1}^{\mathrm{new}} - \Delta x^2\omega_{i,j})                                                      \\
    \Phi_{i,j}^{\mathrm{new}} & = C_{SOR}\frac{1}{4}\qty(\Phi_{i+1,j} + \Phi_{i-1,j}^{\mathrm{new}} + \Phi_{i,j+1} + \Phi_{i,j-1}^{\mathrm{new}} - \Delta x^2\omega_{i,j}) + (1 - C_{SOR})\Phi_{i,j} & (0 < C_{SOR} < 2)
  \end{align}
\end{theorem}

\section{ヌメロフ法}
\section{LAPACK}
\section{Taylor 展開法}
\begin{definition}
  現在の情報のみを使って時間発展を記述する方法を陽解法という。未来の情報も使って時間発展を記述する方法を陰解法という。
\end{definition}
\begin{definition}
  Taylor展開法
  Crank-Nicolson 法は時間と空間について2次精度の陰解法
\end{definition}

\begin{definition}[TDSE]
  \begin{align}
    i\hbar\pdv{\psi}{t} & = \qty(-\frac{\hbar^2}{2m}\laplacian + V(\rr))\psi(\rr, t)
  \end{align}
\end{definition}
\begin{align}
  - \frac{\hbar^2}{2m}
  \mqty(-2     & 1      &        &         &    &    \\
  1            & -2     & 1      &         &    &    \\
               & 1      & -2     & 1       &    &    \\
               &        &        & \ddots  &    &    \\
               &        &        & 1       & -2 & 1)
  \mqty(\psi_\alpha(x_1)                             \\ \psi_\alpha(x_2) \\ \psi_\alpha(x_3) \\ \vdots \\ \psi_\alpha(x_N)) +
  \mqty(V(x_1) &        &        &                   \\
               & V(x_2) &        &                   \\
               &        & \ddots &                   \\
               &        &        & V(x_N))
  \mqty(\psi_\alpha(x_1)                             \\ \psi_\alpha(x_2) \\ \vdots \\ \psi_\alpha(x_N))
  =
  E\mqty(\psi_\alpha(x_1)                            \\ \psi_\alpha(x_2) \\ \vdots \\ \psi_\alpha(x_N))
\end{align}

\begin{definition}[時間依存Gross-Pitaevskii方程式: TDGPE (時間依存非線形Schrödinger方程式: TDNSE)]
  非線形項
  \begin{align}
    i\hbar\pdv{\Psi}{t} & = \qty(-\frac{\hbar^2}{2m}\laplacian + V_{ext}(\rr, t) + g|\Psi(\rr, t)|^2)\Psi(\rr, t) & \qty(g = \frac{4\pi\hbar^2 a_s}{m})
  \end{align}

\end{definition}

\end{document}