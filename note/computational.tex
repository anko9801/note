\RequirePackage{plautopatch}
\documentclass[uplatex,diffipdfmx,a4paper,11pt]{jlreq}
\input{preamble.tex}
\newcommand{\x}{\bm{x}}
\newcommand{\y}{\bm{y}}
\renewcommand{\u}{\bm{u}}
\title{計算物理}
\author{anko9801}
\begin{document}
\maketitle
\tableofcontents
\clearpage

\section{差分法}
\begin{definition}
  連続な時間 $t$ に対し、離散化された時間のパラメータは微小時間 $\Delta t$ を用いて $t_n = n\Delta t$ と書く。同様に空間についても $x_i = i\Delta x$ とする。
  時間・空間について連続な関数 $f(x, t)$ に対する離散値を $f_i^{(n)} := f(x_i, t_n)$ と書く。
  空間について $n$ 次精度とは $\mathcal{O}(\Delta x^n)$ となること
\end{definition}



\begin{theorem}[1階微分の差分公式]
  1次前進差分、1次後退差分、2次中心差分は次のように求まる。
  \begin{align}
    f_i' & \approx \frac{f_{i+1} - f_i}{\Delta x} + \mathcal{O}(\Delta x)        \\
    f_i' & \approx \frac{f_i - f_{i-1}}{\Delta x} + \mathcal{O}(\Delta x)        \\
    f_i' & \approx \frac{f_{i+1} - f_{i-1}}{2\Delta x} + \mathcal{O}(\Delta x^2)
  \end{align}
\end{theorem}
\begin{proof}
  $f(x \pm \Delta x)$ を 2 次まで Taylor 展開すると
  \begin{align}
    f(x \pm \Delta x) & = f_{i \pm 1} = f_i \pm f_i'\Delta x + \frac{1}{2}f_i''\Delta x^2 + \mathcal{O}(\Delta x^3)
  \end{align}
  より次の式が成り立つ。
  \begin{align}
    f_{i + 1} - f_i       & = f_i'\Delta x + \mathcal{O}(\Delta x^2)  \\
    f_i - f_{i - 1}       & = f_i'\Delta x + \mathcal{O}(\Delta x^2)  \\
    f_{i + 1} - f_{i - 1} & = 2f_i'\Delta x + \mathcal{O}(\Delta x^3)
  \end{align}
  よって微分は
  \begin{align}
    f_i' & \approx \frac{f_{i+1} - f_i}{\Delta x} + \mathcal{O}(\Delta x)        \\
    f_i' & \approx \frac{f_i - f_{i-1}}{\Delta x} + \mathcal{O}(\Delta x)        \\
    f_i' & \approx \frac{f_{i+1} - f_{i-1}}{2\Delta x} + \mathcal{O}(\Delta x^2)
  \end{align}
\end{proof}

\begin{theorem}[2次精度2階微分の差分公式]
  2次中心差分は次のように書ける。
  \begin{align}
    f''(x_n) & \approx \frac{f(x_{n+1}) - 2f(x_n) + f(x_{n-1})}{\Delta x^2} + \mathcal{O}(\Delta x^2)
  \end{align}
\end{theorem}
\begin{proof}
  $f(x \pm \Delta x)$ を 3 次まで Taylor 展開すると
  \begin{align}
    f(x \pm \Delta x) & = f_{i \pm 1} = f_i \pm f_i'\Delta x + \frac{1}{2}f_i''\Delta x^2 \pm \frac{1}{3!}f_i'''\Delta x^3 + \mathcal{O}(\Delta x^4)
  \end{align}
  となりこの和について計算することで求まる。
  \begin{align}
    f_{i + 1} + f_{i - 1} & = 2f_i + f_i''\Delta x^2 + \mathcal{O}(\Delta x^4)                          \\
    f_i''                 & = \frac{f_{i + 1} - 2f_i + f_{i - 1}}{\Delta x^2} + \mathcal{O}(\Delta x^2)
  \end{align}
\end{proof}

\begin{theorem}[2階微分の差分公式]
  $f(x\pm k\Delta x)$ の Taylor 展開で 2 次以外の項を相殺することで次の式が得られる.
  \begin{align}
    f''(x_n) & \approx \frac{-f(x_{n+2}) + 16f(x_{n+1}) - 30f(x_n) + 16f(x_{n-1}) - f(x_{n-2})}{12\Delta x^2} + \mathcal{O}(\Delta x^4)
  \end{align}
\end{theorem}

\subsection{安全性解析}
\begin{definition}[フォン・ノイマン (von Neumann) の安定性解析]
  時間が経つと共に振幅が増大しないことは安定性の条件となる。
  \begin{align}
    f(x, t)                            & = \sum_kA_k(t)e^{ikx} \\
    \ab|\frac{A_k^{(n+1)}}{A_k^{(n)}}| & \leq 1
  \end{align}
\end{definition}

\begin{definition}[CFL (Courant-Friedrichs-Lewy) 条件]
  計算上の情報の伝播する速さより物理的な情報の伝播する速さの方が小さい。
  \begin{align}
    \frac{\Delta x}{\Delta t} \geq c \\
    \nu := \frac{c\Delta t}{\Delta x} \leq 1
  \end{align}
\end{definition}

\begin{definition}[拡散方程式]
  \begin{align}
    \diffp{f(x, t)}{t} & = \kappa\diffp[2]{f(x, t)}{x}
  \end{align}
\end{definition}

\begin{theorem}
  \begin{align}
    f_i^{(n+1)} & = f_i^{(n)} + \frac{\kappa\Delta t}{\Delta x^2}\ab(f_{i+1}^{(n)} - 2f_i^{(n)} + f_{i-1}^{(n)})
  \end{align}
\end{theorem}
\begin{proof}
  拡散方程式は時間について(1次)前進差分、空間について(2次)中心差分を取る。これを FTCS (Forward Time Centered Space) スキームという。
  \begin{align}
    \frac{f_i^{(n+1)} - f_i^{(n)}}{\Delta t} & = \kappa\frac{f_{i+1}^{(n)} - 2f_i^{(n)} + f_{i-1}^{(n)}}{\Delta x^2}                          \\
    f_i^{(n+1)}                              & = f_i^{(n)} + \frac{\kappa\Delta t}{\Delta x^2}\ab(f_{i+1}^{(n)} - 2f_i^{(n)} + f_{i-1}^{(n)})
  \end{align}
  このように時間発展を求められる。これは安全性を満たす。
  \begin{align}
    \frac{A^{(n+1)}e^{ikx_i} - A^{(n)}e^{ikx_i}}{\Delta t} & = \kappa\frac{A^{(n)}e^{ikx_{i+1}} - 2A^{(n)}e^{ikx_{i}} + A^{(n)}e^{ikx_{i-1}}}{\Delta x^2} \\
    \ab|\frac{A_k^{(n+1)}}{A_k^{(n)}}|                     & = \ab|1 + \frac{\kappa\Delta t}{\Delta x^2}(e^{ik\Delta x} - 2 + e^{-ik\Delta x})|           \\
                                                           & = \ab|1 - \frac{2\kappa\Delta t}{\Delta x^2}(1 - \cos(k\Delta x))| \leq 1
  \end{align}
\end{proof}

\begin{definition}[移流方程式]
  \begin{align}
    \diffp{f(x, t)}{t} & = -c\diffp{f(x, t)}{x}
  \end{align}
\end{definition}

\begin{theorem}
  移流方程式は時間について(1次)前進差分、空間について(2次)中心差分を取ると安全性を満たさない。 \\
  時間について(1次)前進差分、空間について(1次)後退差分を取ると安全性を満たす。
  \begin{align}
    \frac{f_i^{(n+1)} - f_i^{(n)}}{\Delta t} & = - c\frac{f_i^{(n)} - f_{i-1}^{(n)}}{\Delta x}
  \end{align}
\end{theorem}
\begin{proof}
  時間について(1次)前進差分、空間について(2次)中心差分を取る。
  \begin{align}
    \frac{f_i^{(n+1)} - f_i^{(n)}}{\Delta t} & = -c\frac{f_{i+1}^{(n)} - f_{i-1}^{(n)}}{2\Delta x}
  \end{align}
  安全性解析すると
  \begin{align}
    \ab(A(t_{n+1}) - A(t_n))\frac{e^{ikx_i}}{\Delta t} & = -c\ab(e^{ik\Delta x} - e^{-ik\Delta x})\frac{A(t_n)e^{ikx_i}}{2\Delta x}      \\
    \ab|\frac{A(t_{n+1})}{A(t_n)}|                     & = \sqrt{\ab|1 - i\nu\sin(k\Delta x)|^2} = \sqrt{1 + \nu^2\sin^2(k\Delta x)} > 1
  \end{align}
  CFL 条件
  \begin{align}
    \frac{\Delta x}{\Delta t} & = -c\frac{f_{i+1}^{(n)} - f_{i-1}^{(n)}}{2(f_i^{(n+1)} - f_i^{(n)})} \geq c
  \end{align}
  1次後退差分で離散化したものについて
  \begin{align}
    \frac{f_i^{(n+1)} - f_i^{(n)}}{\Delta t} & = -c\frac{f_i^{(n)} - f_{i-1}^{(n)}}{\Delta x}
  \end{align}
  安定性解析すると
  \begin{align}
    \ab(A(t_{n+1}) - A(t_n))\frac{e^{ikx_i}}{\Delta t} & = -c\ab(1 - e^{-ik\Delta x})\frac{A(t_n)e^{ikx_i}}{\Delta x} \\
    \ab|\frac{A(t_{n+1})}{A(t_n)}|                     & = \sqrt{\ab|1 - \nu(1 - e^{-ik\Delta x})|^2}                 \\
                                                       & = \sqrt{1 - 2\nu(1 - \nu)(1 - \cos(k\Delta x))} \leq 1       \\
    \iff \nu                                           & \leq 1
  \end{align}
\end{proof}
これは差分により拡散項が増えてしまったからである。



\begin{definition}[Navier-Stokes 方程式]
  \begin{align}
    \diffp{\bm{v}}{t} & = -(\bm{v}\cdot\nabla)\bm{v} - \frac{1}{\rho}\nabla p + \nu\nabla^2\bm{v} + \bm{F}
  \end{align}
  まず無次元量とする。
  \begin{align}
    \bm{v} & = V\tilde{\bm{v}} & \bm{r} & = L\tilde{\bm{r}} & t & = \frac{L}{V}\tilde{t} & p & = \rho V^2\tilde{p} & \nabla & = \frac{1}{L}\tilde{\nabla} & \diffp{}{t} & = \frac{V}{L}\diffp{}{\tilde{t}} & \mathrm{Re} & = \frac{UV}{\nu}
  \end{align}
  レイノルズ数 $\mathrm{Re}$ は慣性力と粘性力の比に対応する。
  \begin{align}
    \diffp{\tilde{\bm{v}}}{\tilde{t}} & = -(\tilde{\bm{v}}\cdot\tilde{\nabla})\tilde{\bm{v}} - \tilde{\nabla}\tilde{p} + \frac{1}{\mathrm{Re}}\tilde{\nabla}^2\tilde{\bm{v}}
  \end{align}
  これ以降、無次元量を表すチルダは略す。
\end{definition}
\begin{theorem}
  Navier-Stokes 方程式
\end{theorem}
\begin{proof}
  $\omega = \nabla\times\bm{v}$ とすると
  \begin{align}
    \diffp{\omega_i}{t} & = \diffp{t}(\nabla\times\bm{v})_i                                                                                                                       \\
                        & = -\nabla\times((\bm{v}\cdot\nabla)\bm{v}) - \nabla\times\nabla p + \frac{1}{\mathrm{Re}}\nabla\times\nabla^2\bm{v}                                     \\
                        & = -\varepsilon_{ijk}\partial_j((v_l\partial_l)v_k) + \frac{1}{\mathrm{Re}}\varepsilon_{ijk}\partial_j\partial_l\partial_lv_k                            \\
                        & = -\varepsilon_{ijk}(\partial_jv_l\partial_lv_k + v_l\partial_j\partial_lv_k) + \frac{1}{\mathrm{Re}}\partial_l\partial_l\varepsilon_{ijk}\partial_jv_k \\
                        & = -\varepsilon_{ijk}\partial_jv_l\partial_lv_k - (\bm{v}\cdot\nabla)(\nabla\times\bm{v}) + \frac{1}{\mathrm{Re}}\nabla^2(\nabla\times\bm{v})            \\
                        & = -\varepsilon_{ijk}\partial_jv_l\partial_lv_k - (\bm{v}\cdot\nabla)\omega + \frac{1}{\mathrm{Re}}\nabla^2\omega
  \end{align}
  $\bm{v}(x, y, t) = (u(x, y, t), v(x, y, t), 0)$ とするとき、次の渦度方程式となる。
  \begin{align}
    \diffp{\omega_3}{t} & = - (\bm{v}\cdot\nabla)\omega_3 + \frac{1}{\mathrm{Re}}\nabla^2\omega_3
  \end{align}
  ここでは非圧縮性液体 $\nabla\cdot\bm{v} = 0$ のとき $\bm{v} = \nabla\times\Phi$ と書ける。ここで流れ関数 $\Phi$ を導入する。
  \begin{align}
    u      & = - \diffp{\Phi}{y} \quad v = \diffp{\Phi}{x}            \\
    \omega & = \diffp[2]{\Phi}{x} + \diffp[2]{\Phi}{y} = \nabla^2\Phi \\
    \omega & = \nabla\times(\nabla\times\Phi)                         \\
           & = \nabla\cdot(\nabla\Phi) - \nabla^2\Phi
  \end{align}
  流れ関数を用いると解くべき方程式は渦度方程式とポアソン方程式に分けることができる。
  \begin{align}
    \diffp{\omega}{t} & = \diffp{\Phi}{x}\diffp{\omega}{y} - \diffp{\Phi}{y}\diffp{\omega}{x} + \frac{1}{\mathrm{Re}}\nabla^2\omega \\
    \nabla^2\Phi      & = - \omega
  \end{align}
\end{proof}

\begin{theorem}
  ポアソン方程式については差分法で解ける。
  \begin{align}
     & \nabla^2\Phi = - \omega                                                                                                                    \\
     & \frac{\Phi_{i+1,j} - 2\Phi_{i,j} + \Phi_{i-1,j}}{\Delta x^2} + \frac{\Phi_{i,j+1} - 2\Phi_{i,j} + \Phi_{i,j-1}}{\Delta y^2} = \omega_{i,j} \\
     & \Phi_{i,j} = \frac{1}{4}\ab(\Phi_{i+1,j} + \Phi_{i-1,j} + \Phi_{i,j+1} + \Phi_{i,j-1} - \Delta x^2\omega_{i,j})
  \end{align}
  初期値を適当にセットして、この漸化式を収束するまで繰り返し更新する。この漸化式を改良させて収束を早めることができる。次の漸化式を順にヤコビ法, ガウス・ザイデル法, SOR 法という。
  \begin{align}
    \Phi_{i,j}^{\mathrm{new}} & = \frac{1}{4}\ab(\Phi_{i+1,j} + \Phi_{i-1,j} + \Phi_{i,j+1} + \Phi_{i,j-1} - \Delta x^2\omega_{i,j})                                                                                    \\
    \Phi_{i,j}^{\mathrm{new}} & = \frac{1}{4}\ab(\Phi_{i+1,j} + \Phi_{i-1,j}^{\mathrm{new}} + \Phi_{i,j+1} + \Phi_{i,j-1}^{\mathrm{new}} - \Delta x^2\omega_{i,j})                                                      \\
    \Phi_{i,j}^{\mathrm{new}} & = C_{SOR}\frac{1}{4}\ab(\Phi_{i+1,j} + \Phi_{i-1,j}^{\mathrm{new}} + \Phi_{i,j+1} + \Phi_{i,j-1}^{\mathrm{new}} - \Delta x^2\omega_{i,j}) + (1 - C_{SOR})\Phi_{i,j} & (0 < C_{SOR} < 2)
  \end{align}
\end{theorem}

\section{ヌメロフ法}
\section{LAPACK}
\section{Taylor 展開法}
\begin{definition}
  現在の情報のみを使って時間発展を記述する方法を陽解法という。未来の情報も使って時間発展を記述する方法を陰解法という。
\end{definition}
\begin{definition}
  Taylor展開法
  Crank-Nicolson 法は時間と空間について2次精度の陰解法
\end{definition}

\begin{definition}[TDSE]
  \begin{align}
    i\hbar\diffp{\psi}{t} & = \ab(-\frac{\hbar^2}{2m}\nabla^2 + V(\rr))\psi(\rr, t)
  \end{align}
\end{definition}
\begin{align}
  - \frac{\hbar^2}{2m}
  \mqty(-2     & 1      &        &         &    &    \\
  1            & -2     & 1      &         &    &    \\
               & 1      & -2     & 1       &    &    \\
               &        &        & \ddots  &    &    \\
               &        &        & 1       & -2 & 1)
  \mqty(\psi_\alpha(x_1)                             \\ \psi_\alpha(x_2) \\ \psi_\alpha(x_3) \\ \cdots \\ \psi_\alpha(x_N)) +
  \mqty(V(x_1) &        &        &                   \\
               & V(x_2) &        &                   \\
               &        & \ddots &                   \\
               &        &        & V(x_N))
  \mqty(\psi_\alpha(x_1)                             \\ \psi_\alpha(x_2) \\ \cdots \\ \psi_\alpha(x_N))
  =
  E\mqty(\psi_\alpha(x_1)                            \\ \psi_\alpha(x_2) \\ \cdots \\ \psi_\alpha(x_N))
\end{align}

\begin{definition}[時間依存Gross-Pitaevskii方程式: TDGPE (時間依存非線形Schrödinger方程式: TDNSE)]
  非線形項
  \begin{align}
    i\hbar\diffp{\Psi}{t} & = \ab(-\frac{\hbar^2}{2m}\nabla^2 + V_{ext}(\rr, t) + g|\Psi(\rr, t)|^2)\Psi(\rr, t) & \ab(g = \frac{4\pi\hbar^2 a_s}{m})
  \end{align}

\end{definition}

\section{動的システム}
\begin{definition}[システム]
  システムとは入力に対して出力を返す"機能"を持つ対象の総称である。
  これは次のようにモデル化できる。

  静的システム (static system) とは入力 $\u$ に対して一意の出力 $\y$ を返すシステムである。
  \begin{align}
    \y(t) = \bm{f}(\u(t))
  \end{align}
  動的システム (dynamic system) とは内部状態 $\x$ を持ち、入力に対して状態が変化し、状態から出力が返されるシステムである。
  \begin{align}
    \diff{\x}{t} & = \bm{f}(\x, \u, t) \\
    \y           & = \bm{g}(\x, t)
  \end{align}
  第 1 式を状態方程式、第 2 式を出力方程式と呼ぶ。
\end{definition}
まず平衡点 $\x_0, \u_0, \y_0$ において展開する。
\begin{align}
  \dot\x & \approx \diffp{\bm{f}}{\x}[(\x_0, \u_0)]\x + \diffp{\bm{f}}{\u}[(\x_0, \u_0)]\u \\
  \y     & \approx \diffp{\bm{g}}{\x}[\x_0]\x
\end{align}
これより
\begin{align}
  \dot\x & = A\x + B\u \\
  \y     & = C\x
\end{align}
と定式化できる。
$\x = T\bm{z}$ とおくと
\begin{align}
  \dot{\bm{z}} & = T^{-1}AT\bm{z} + T^{-1}B\u \\
  \y           & = CT\bm{z}
\end{align}
により $A$ を対角化できる。
\begin{theorem}
  \begin{align}
    \y & = C\exp(A(t - t_0))\x_0 + C\int_{t_0}^{t}\exp(A(t - \tau))B\u(\tau)\dl{\tau}
  \end{align}
\end{theorem}
\begin{proof}
  まず $\u = 0$ のときの状態方程式は次のように解ける。
  \begin{align}
    \dot\x & = A\x                  \\
    \x     & = \exp(A(t - t_0))\x_0
  \end{align}
  これより一般において $\x = \exp(At)\kk$ とおくと
  \begin{align}
    \dot\x  & = A\x + \exp(At)\dot\kk = A\x + B\u \\
    \dot\kk & = \exp(-At)B\u
  \end{align}
  この微分方程式を解くと
  \begin{align}
    \kk(t) - \kk_0 & = \int_{t_0}^{t}\exp(-A\tau)B\u(\tau)\dl{\tau}             \\
    \kk            & = \exp(-At_0)\x_0 + \int_{t_0}^{t}\exp(-A\tau)B\u\dl{\tau}
  \end{align}
  となる。よって
  \begin{align}
    \x & = \exp(A(t - t_0))\x_0 + \int_{t_0}^{t}\exp(A(t - \tau))B\u\dl{\tau}         \\
    \y & = C\exp(A(t - t_0))\x_0 + C\int_{t_0}^{t}\exp(A(t - \tau))B\u(\tau)\dl{\tau}
  \end{align}
\end{proof}





\begin{example}[]

\end{example}
\begin{example}[バネ・マス・ダンパ系]
  \begin{align}
    m\ddot y + b\dot y + ky = f
  \end{align}
  $x_1 = y := y$, $x_2 := \dot y$, $u = f$
  \begin{align}
    \begin{pmatrix}
      \dot x_1 \\
      \dot x_2
    \end{pmatrix}
                    & = \begin{pmatrix}
                          0            & 1            \\
                          -\frac{k}{m} & -\frac{b}{m} \\
                        \end{pmatrix}
    \begin{pmatrix}
      x_1 \\
      x_2
    \end{pmatrix} +
    \begin{pmatrix}
      0 \\
      \frac{1}{m}
    \end{pmatrix}u &
    y               & = \begin{pmatrix}
                          1 & 0
                        \end{pmatrix}
    \begin{pmatrix}
      x_1 \\
      x_2
    \end{pmatrix}
  \end{align}
\end{example}
\begin{example}[RLC]
  \begin{align}
    L\ddot q + R\dot q + \frac{1}{C}q = e_i
  \end{align}
\end{example}


\end{document}