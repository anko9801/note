\RequirePackage{plautopatch}
\documentclass[uplatex,dvipdfmx,a4paper,11pt]{jlreq}
\usepackage{bxpapersize}
\usepackage[utf8]{inputenc}
\usepackage{fontenc}
\usepackage{lmodern}
\usepackage{otf}
\usepackage{amsmath}
\usepackage{amssymb}
\usepackage{amsthm}
\usepackage{ascmac}
% \usepackage[hyphens]{url}
\usepackage{physics}
\usepackage{braket}
\usepackage{verbatimbox}
\usepackage{bm}
\usepackage{url}
% \usepackage[dvipdfmx,hiresbb,final]{graphicx}
\usepackage{hyperref}
\usepackage{pxjahyper}
\usepackage{tikz}\usetikzlibrary{cd}
\usepackage{listings}
\usepackage{color}
\usepackage{mathtools}
\usepackage{xspace}
\usepackage{xy}
\usepackage{xypic}
%
\title{物理数学}
\author{Anko}
\makeatletter
%
\DeclareMathOperator{\lcm}{lcm}
\DeclareMathOperator{\Kernel}{Ker}
\DeclareMathOperator{\Image}{Im}
\DeclareMathOperator{\ch}{ch}
\DeclareMathOperator{\Aut}{Aut}
\DeclareMathOperator{\Log}{Log}
\DeclareMathOperator{\Arg}{Arg}
\DeclareMathOperator{\sgn}{sgn}
%
\newcommand{\CC}{\mathbb{C}}
\newcommand{\RR}{\mathbb{R}}
\newcommand{\QQ}{\mathbb{Q}}
\newcommand{\ZZ}{\mathbb{Z}}
\newcommand{\NN}{\mathbb{N}}
\newcommand{\FF}{\mathbb{F}}
\newcommand{\PP}{\mathbb{P}}
\newcommand{\GG}{\mathbb{G}}
\newcommand{\TT}{\mathbb{T}}
\newcommand{\calB}{\mathcal{B}}
\newcommand{\calF}{\mathcal{F}}
\newcommand{\ignore}[1]{}
\newcommand{\floor}[1]{\left\lfloor #1 \right\rfloor}
% \newcommand{\abs}[1]{\left\lvert #1 \right\rvert}
\newcommand{\lt}{<}
\newcommand{\gt}{>}
\newcommand{\id}{\mathrm{id}}
\newcommand{\rot}{\curl}
\renewcommand{\angle}[1]{\left\langle #1 \right\rangle}
\newcommand{\EE}{\bm{E}}
\newcommand{\BB}{\bm{B}}
\renewcommand{\AA}{\bm{A}}
\newcommand{\rr}{\bm{r}}
\newcommand{\kk}{\bm{k}}
\newcommand{\pp}{\bm{p}}

\let\oldcite=\cite
\renewcommand\cite[1]{\hyperlink{#1}{\oldcite{#1}}}

\let\oldbibitem=\bibitem
\renewcommand{\bibitem}[2][]{\label{#2}\oldbibitem[#1]{#2}}

% theorem環境の設定
% - 冒頭に改行
% - 末尾にdiamond (amsthm)
\theoremstyle{definition}
\newcommand*{\newscreentheoremx}[2]{
  \newenvironment{#1}[1][]{
    \begin{screen}
    \begin{#2}[##1]
      \leavevmode
      \newline
  }{
    \end{#2}
    \end{screen}
  }
}
\newcommand*{\newqedtheoremx}[2]{
  \newenvironment{#1}[1][]{
    \begin{#2}[##1]
      \leavevmode
      \newline
      \renewcommand{\qedsymbol}{\(\diamond\)}
      \pushQED{\qed}
  }{
      \qedhere
      \popQED
    \end{#2}
  }
}
\newtheorem{theorem*}{定理}

\newqedtheoremx{theorem}{theorem*}
\newcommand*\newqedtheorem@unstarred[2]{%
  \newtheorem{#1*}[theorem*]{#2}
  \newqedtheoremx{#1}{#1*}
}
\newcommand*\newqedtheorem@starred[2]{%
  \newtheorem*{#1*}{#2}
  \newqedtheoremx{#1}{#1*}
}
\newcommand*{\newqedtheorem}{\@ifstar{\newqedtheorem@starred}{\newqedtheorem@unstarred}}

\newtheorem{sctheorem*}{定理}
\newscreentheoremx{sctheorem}{sctheorem*}
\newcommand*\newscreentheorem@unstarred[2]{%
  \newtheorem{#1*}[theorem*]{#2}
  \newscreentheoremx{#1}{#1*}
}
\newcommand*\newscreentheorem@starred[2]{%
  \newtheorem*{#1*}{#2}
  \newscreentheoremx{#1}{#1*}
}
\newcommand*{\newscreentheorem}{\@ifstar{\newscreentheorem@starred}{\newscreentheorem@unstarred}}

%\newtheorem*{definition}{定義}
%\newtheorem{theorem}{定理}
%\newtheorem{proposition}[theorem]{命題}
%\newtheorem{lemma}[theorem]{補題}
%\newtheorem{corollary}[theorem]{系}

\newqedtheorem{lemma}{補題}
\newqedtheorem{corollary}{系}
\newqedtheorem{example}{例}
\newqedtheorem{proposition}{命題}
\newqedtheorem{remark}{注意}
\newqedtheorem{thesis}{主張}
\newqedtheorem{notation}{記法}
\newqedtheorem{problem}{問題}
\newqedtheorem{algorithm}{アルゴリズム}

\newscreentheorem*{definition}{定義}

\renewenvironment{proof}[1][\proofname]{\par
  \normalfont
  \topsep6\p@\@plus6\p@ \trivlist
  \item[\hskip\labelsep{\bfseries #1}\@addpunct{\bfseries}]\ignorespaces\quad\par
}{%
  \qed\endtrivlist\@endpefalse
}
\renewcommand\proofname{証明}

\makeatother

\begin{document}
\maketitle


\section{微分積分学}
\subsection{測度}
測度とはある範囲の集合に非負の実数あるいは $\infty$ を対応させる集合関数である。
\begin{definition}[加法族]
  集合 $X$ の部分集合の族 $\bm{B}$ が次の条件を満たすとき、$X$ 上の加法族と呼ぶ。
  \begin{enumerate}
    \item $\emptyset\in\bm{B}$
    \item $A\in\bm{B}$ ならば $A^c\in\bm{B}$
    \item $A_n\in\bm{B}$ ならば $\bigcup_{n=1}^\infty A_n\in\bm{B}$
  \end{enumerate}
\end{definition}
\begin{definition}[測度]
  $\bm{B}$ を $X$ 上の加法族とするとき $\mu$ が $(X, \bm{B})$ 上の測度であるとは
  \begin{enumerate}
    \item $A\in\bm{B}$ に対し $0\leq \mu(A)\leq\infty$ $\mu(\emptyset) = 0$
    \item $A_n$
  \end{enumerate}
\end{definition}

\section{微分幾何学}
\subsection{多様体}
接ベクトル
接ベクトル束
\subsection{微分形式}
\begin{definition}[微分形式]
  $\land:$
  \begin{align}
    u\land v := u\otimes v - v\otimes u
  \end{align}
  $0$-形式 \\
  $k$-形式$\omega_{\mu_1\cdots\mu_k}\in C^\infty(U)$
  \begin{align}
    \omega = \frac{1}{k!}\omega_{\mu_1\cdots\mu_k}dx^{\mu_1}\land\cdots\land dx^{\mu_k}
  \end{align}
\end{definition}
\begin{definition}[外微分]
  外微分 (exterior derivative) $d: \Omega^k(M)\to\Omega^{k+1}(M)$ を次のように定義する。
  \begin{align}
    d\omega & := d\qty(\frac{1}{k!}\omega_{\mu_1\cdots\mu_k})\land dx^{\mu_1}\land\cdots\land dx^{\mu_k}                \\
            & = \frac{1}{k!}\pdv{\omega_{\mu_1\cdots\mu_k}}{x^{\nu}}dx^{\nu}\land dx^{\mu_1}\land\cdots\land dx^{\mu_k}
  \end{align}
\end{definition}
\begin{align}
  dx\land dy & = d(r\cos\theta)\land d(r\sin\theta)                                                                                                                  \\
             & = (dr\cos\theta - r\sin\theta d\theta)\land (dr\sin\theta + r\cos\theta d\theta)                                                                      \\
             & = (\cos\theta\sin\theta)dr\land dr + (r\cos^2\theta)dr\land d\theta - (r\sin^2\theta) d\theta\land dr - (r^2\sin\theta\cos\theta)d\theta\land d\theta \\
             & = r dr\land d\theta
\end{align}

\begin{definition}[1 の分割 (partition of unity)]
\end{definition}

\begin{theorem}[ストークスの定理]
\end{theorem}






\end{document}
