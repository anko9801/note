\RequirePackage{plautopatch}
\documentclass[uplatex,dvipdfmx,a4paper,11pt]{jlreq}
\usepackage{bxpapersize}
\usepackage[utf8]{inputenc}
\usepackage{fontenc}
\usepackage{lmodern}
\usepackage{otf}
\usepackage{amsmath}
\usepackage{amssymb}
\usepackage{amsthm}
\usepackage{ascmac}
% \usepackage[hyphens]{url}
\usepackage{physics}
\usepackage{braket}
\usepackage{verbatimbox}
\usepackage{bm}
\usepackage{url}
% \usepackage[dvipdfmx,hiresbb,final]{graphicx}
\usepackage{hyperref}
\usepackage{pxjahyper}
\usepackage{tikz}\usetikzlibrary{cd}
\usepackage{listings}
\usepackage{color}
\usepackage{mathtools}
\usepackage{xspace}
\usepackage{xy}
\usepackage{xypic}
%
\title{集合論}
\author{Anko}
\makeatletter
%
\DeclareMathOperator{\lcm}{lcm}
\DeclareMathOperator{\Kernel}{Ker}
\DeclareMathOperator{\Image}{Im}
\DeclareMathOperator{\ch}{ch}
\DeclareMathOperator{\Aut}{Aut}
\DeclareMathOperator{\Log}{Log}
\DeclareMathOperator{\Arg}{Arg}
\DeclareMathOperator{\sgn}{sgn}
%
\newcommand{\CC}{\mathbb{C}}
\newcommand{\RR}{\mathbb{R}}
\newcommand{\QQ}{\mathbb{Q}}
\newcommand{\ZZ}{\mathbb{Z}}
\newcommand{\NN}{\mathbb{N}}
\newcommand{\FF}{\mathbb{F}}
\newcommand{\PP}{\mathbb{P}}
\newcommand{\GG}{\mathbb{G}}
\newcommand{\TT}{\mathbb{T}}
\newcommand{\calB}{\mathcal{B}}
\newcommand{\calF}{\mathcal{F}}
\newcommand{\ignore}[1]{}
\newcommand{\floor}[1]{\left\lfloor #1 \right\rfloor}
% \newcommand{\abs}[1]{\left\lvert #1 \right\rvert}
\newcommand{\lt}{<}
\newcommand{\gt}{>}
\newcommand{\id}{\mathrm{id}}
\newcommand{\rot}{\curl}
\renewcommand{\angle}[1]{\left\langle #1 \right\rangle}

\let\oldcite=\cite
\renewcommand\cite[1]{\hyperlink{#1}{\oldcite{#1}}}

\let\oldbibitem=\bibitem
\renewcommand{\bibitem}[2][]{\label{#2}\oldbibitem[#1]{#2}}

% theorem環境の設定
% - 冒頭に改行
% - 末尾にdiamond (amsthm)
\theoremstyle{definition}
\newcommand*{\newscreentheoremx}[2]{
  \newenvironment{#1}[1][]{
    \begin{screen}
    \begin{#2}[##1]
      \leavevmode
      \newline
  }{
    \end{#2}
    \end{screen}
  }
}
\newcommand*{\newqedtheoremx}[2]{
  \newenvironment{#1}[1][]{
    \begin{#2}[##1]
      \leavevmode
      \newline
      \renewcommand{\qedsymbol}{\(\diamond\)}
      \pushQED{\qed}
  }{
      \qedhere
      \popQED
    \end{#2}
  }
}
\newtheorem{theorem*}{定理}

\newqedtheoremx{theorem}{theorem*}
\newcommand*\newqedtheorem@unstarred[2]{%
  \newtheorem{#1*}[theorem*]{#2}
  \newqedtheoremx{#1}{#1*}
}
\newcommand*\newqedtheorem@starred[2]{%
  \newtheorem*{#1*}{#2}
  \newqedtheoremx{#1}{#1*}
}
\newcommand*{\newqedtheorem}{\@ifstar{\newqedtheorem@starred}{\newqedtheorem@unstarred}}

\newtheorem{sctheorem*}{定理}
\newscreentheoremx{sctheorem}{sctheorem*}
\newcommand*\newscreentheorem@unstarred[2]{%
  \newtheorem{#1*}[theorem*]{#2}
  \newscreentheoremx{#1}{#1*}
}
\newcommand*\newscreentheorem@starred[2]{%
  \newtheorem*{#1*}{#2}
  \newscreentheoremx{#1}{#1*}
}
\newcommand*{\newscreentheorem}{\@ifstar{\newscreentheorem@starred}{\newscreentheorem@unstarred}}

%\newtheorem*{definition}{定義}
%\newtheorem{theorem}{定理}
%\newtheorem{proposition}[theorem]{命題}
%\newtheorem{lemma}[theorem]{補題}
%\newtheorem{corollary}[theorem]{系}

\newqedtheorem{lemma}{補題}
\newqedtheorem{corollary}{系}
\newqedtheorem{example}{例}
\newqedtheorem{proposition}{命題}
\newqedtheorem{remark}{注意}
\newqedtheorem{thesis}{主張}
\newqedtheorem{notation}{記法}
\newqedtheorem{problem}{問題}
\newqedtheorem{algorithm}{アルゴリズム}

\newscreentheorem*{axiom}{公理}
\newscreentheorem*{definition}{定義}

\renewenvironment{proof}[1][\proofname]{\par
  \normalfont
  \topsep6\p@\@plus6\p@ \trivlist
  \item[\hskip\labelsep{\bfseries #1}\@addpunct{\bfseries}]\ignorespaces\quad\par
}{%
  \qed\endtrivlist\@endpefalse
}
\renewcommand\proofname{証明}

\makeatother

\begin{document}
\maketitle
\tableofcontents
\clearpage

\section{公理的集合論の基礎}
\subsection{論理}
\begin{axiom}[外延性]
  \begin{align}
    \forall x\forall y(\forall z(z\in x\iff z\in y)\implies x=y)
  \end{align}
\end{axiom}
\begin{axiom}[内包性図式]
  変数 $y$ を自由変数として用いない任意の論理式 $\phi$ を用いて次のように表せられる。
  \begin{align}
    \exists y\forall x(x\in y\iff x\in z\land \phi)
  \end{align}
\end{axiom}

\begin{definition}[内包性図式]
  $\qty{x\in z:\phi}\coloneq y \thickspace\text!{s.t.}\thickspace \forall x(x\in y\longleftrightarrow x\in z\land \phi)$。これは外延性より唯一つに定まる.
\end{definition}

\begin{theorem}
  \begin{align}
    \exists y\forall x(x\notin y)
  \end{align}
\end{theorem}
\begin{proof}
  内包性より $\exists y\forall x(x\in y\iff x\in z\land x\neq x)$ となる。
  $z$ には真のクラスは含まれないから1つも元のない集合となる $\exists y\forall x(x\in y\iff x\notin z)$。
  これより、$\forall x(x\notin y)$ となる $y$ の存在が正当化される。
\end{proof}

\begin{definition}
  上の集合 $y$ を空集合 $\emptyset$ と呼ぶ。
\end{definition}

\begin{theorem}
  \begin{align}
    \lnot\exists z\forall x(x\in z)
  \end{align}
\end{theorem}
\begin{proof}
  $\forall x(x\in z)$ となる $z$ が存在すると仮定すると, 内包性より $\qty{x\in z: x\notin x} = \qty{x: x\notin x}$ が存在, つまり $\exists y\forall x(x\in y\longleftrightarrow x\notin x)$ となる。しかし $x$ に $y$ を代入することで $y\in y\longleftrightarrow y\notin y$ より矛盾。よってそのような $z$ は存在しない.
\end{proof}

\begin{axiom}[対]
  \begin{align}
    \forall x\forall y\exists z(x\in z\land y\in z)
  \end{align}
\end{axiom}

\begin{definition}
  対
  \begin{align}
    \qty{x, y}\coloneq\qty{v\in z:v=x\lor v=y}
  \end{align}
  単集合
  \begin{align}
    \qty{x}\coloneq\qty{x, x}
  \end{align}
  順序対
  \begin{align}
    \ev{x, y}\coloneq\qty{\qty{x}, \qty{x,y}}
  \end{align}
\end{definition}
\begin{axiom}[和集合]
  \begin{align}
    \forall\mathcal{F}\exists A\forall Y\forall x(x\in Y\land Y\in\mathcal{F}\to x\in A)
  \end{align}
\end{axiom}
\begin{definition}
  和集合
  \begin{align}
    \bigcup\mathcal{F}\coloneq\qty{x:\exists Y\in\mathcal{F}(x\in Y)}
  \end{align}
  共通部分
  \begin{align}
    \bigcap\mathcal{F}\coloneq\qty{x:\forall Y\in\mathcal{F}(x\in Y)}
  \end{align}
  和集合
  \begin{align}
    A\cup B\coloneq\bigcup\qty{A, B}
  \end{align}
  積集合
  \begin{align}
    A\cap B\coloneq\bigcap\qty{A, B}
  \end{align}
  差集合
  \begin{align}
    A\setminus B\coloneq\qty{x\in A:x\notin B}
  \end{align}
\end{definition}
\begin{axiom}[置換図式]
  \begin{align}
    \forall x\in A\exists!y\phi(x,y)\to\exists Y\forall x\in A\exists y\in Y\phi(x,y)
  \end{align}
\end{axiom}

\begin{theorem}[置換図式]
  \begin{align}
    \forall x\in A\exists!y\phi(x,y)\to\exists\qty{y:\exists x\in A\phi(x,y)}
  \end{align}
\end{theorem}
\begin{proof}
  $\forall x\in A\exists!y\phi(x,y)$ を仮定すると, 置換公理より $\forall x\in A\exists y\in Y\phi(x,y)$ を満たす集合 $Y$ が存在し, 内包性公理より $\qty{y\in Y:\exists x\in A\phi(x,y)}$ が存在する。これは $Y'=\qty{y:\exists x\in A\phi(x,y)}$ とおくと, $Y'$ の濃度は $A$ の濃度以下であるから集合として存在し, $Y = Y'$ となる。
\end{proof}

\begin{definition}[直積集合]
  $A\times B\coloneq\qty{\ev{x,y}:x\in A\land y\in B}$
\end{definition}
\begin{proof}
  置換公理と内包性公理より, 各 $y\in B$ に対し,
  \begin{align}
    \forall x\in A\exists!z\qty{z=\ev{x,y}}
    \mathrm{prod}(A,y) \coloneq \qty{z:\exists x\in A(z=\ev{x, y})}
  \end{align}
  また, 次のように定義できる。
  \begin{align}
    \forall y\in B\exists!z\qty{z=\mathrm{prod}\qty{A,y}}
    \mathrm{prod}'(A,B) \coloneq \qty{\mathrm{prod}\qty{A,y}: y\in B}
  \end{align}
  $A\times B\coloneq\bigcup\mathrm{prod}'(A, B)$ と置くことで定義の正当性が分かる。
\end{proof}

\begin{definition}
  関係

  任意の要素が順序対となる集合.

  定義域, 値域

  関係 $R$ に対し, 定義域 $\mathrm{dom}(R)$ と値域 $\mathrm{ran}(R)$ は次のように定義する。
  \begin{align}
    \mathrm{dom}(R) & = \qty{x:\exists y\qty{\ev{x,y}\in R}} \\
    \mathrm{ran}(R) & = \qty{y:\exists x\qty{\ev{x,y}\in R}}
  \end{align}
  関係 $R$ は通常 $R\subset \mathrm{dom}(R)\times\mathrm{ran}(R)$ となる場合だけに使われる。

  $R^{-1}\coloneq\qty{\ev{x, y}:\ev{y, x}\in R}$
\end{definition}

\begin{theorem}
  関係 $R$ に対し $\qty{R^{-1}}^{-1} = R$ となる。
\end{theorem}
\begin{proof}
  $R$ は関係であるから任意の $R$ の元は順序対であり, それぞれに対し反転を2回行えば元に戻る。
\end{proof}

\begin{definition}[関数]
  関係 $f$ が $\forall x\in\mathrm{dom}(f),\exists\!y\in\mathrm{ran}(f)\qty{\ev{x,y}\in f}$ を満たすとき $f$ を関数と呼ぶ。また, 関数 $f$ について $A=\mathrm{dom}(f), B\supset\mathrm{ran}(f)$ を満たすとき, $f:A\to B$ と書く。

  関数の制限
\end{definition}
\begin{definition}[狭義全順序]
  集合 $A$ 関係 $R$ に対し, 次を満たす組 $\ev{A, R}$ を狭義全順序と呼ぶ。
  \begin{align}
    \text{推移律}\quad  & \forall x,y,z\in A(xRy\land yRz\to xRz) \\
    \text{三分律}\quad  & \forall x, y\in A(x=y\lor xRy\lor yRx)  \\
    \text{非反射律}\quad & \forall x\in A\qty{\lnot\qty{xRx}}
  \end{align}
\end{definition}

\begin{theorem}
  $\ev{A, R}$ が狭義全順序ならば, 任意の $B\subset A$ について $\ev{B, R}$ は狭義全順序となる。
\end{theorem}
\begin{proof}
  $R\subset \mathrm{dom}(R)\times\mathrm{ran}(R)$ より集合に対して関係の集合は依存していない。また推移律, 三分律, 非反射律は存在を示している訳ではないので $B$ に対しても成立する。よって $\ev{B, R}$ は狭義全順序となる。
\end{proof}

\begin{definition}
  同型写像
  集合と関係の対 $\ev{A, R}, \ev{B, S}$ について
  全単射 $f:A\to B$ が存在し $\forall x,y\in A(xRy\iff f(x)Sf(y))$ となるとき $\ev{A,R}\cong\ev{B,S}$ と書き, $f$ を同型写像と呼ぶ。

  整列順序
  全順序 $\ev{A,R}$ について $A$ の空でない任意の部分集合に必ず $R$-最小の要素があるとき, $\ev{A,R}$ が整列順序であるという。

  切片
  $\mathrm{pred}(A,x,R)\coloneq\qty{y\in A:yRx}$
\end{definition}

\begin{theorem}
  $\ev{A,R}$ を整列順序とするとき, 任意の $x\in A$ に対して $\ev{A,R}\ncong\ev{\mathrm{pred}(A,x,R), R}$ である。
\end{theorem}
\begin{proof}
  $f:A\to\mathrm{pred}(A,x,R)$ が同型写像であると仮定すると, 集合 $\qty{y\in A:f(y)\neq y}$ の $R$-最小要素 $y$ が。
\end{proof}

\begin{theorem}
  $\ev{A,R}, \ev{B,S}$ を互いに同型な整列順序とするとき, この間の同型写像は唯一つ存在する。
\end{theorem}
\begin{proof}
  仮に2つの同型写像 $f, g$ が存在したとき $f(y)\neq g(y)$ であるような $y\in A$ のうち $R$-最小の $y$ を考えると矛盾。
\end{proof}

\begin{theorem}
  $\ev{A,R}, \ev{B,S}$ を整列順序とするとき, 次の3つの命題は互いに背反である。
  \begin{align}
    \text{(a)}\quad & \ev{A,R}\cong\ev{B,S}                                        \\
    \text{(b)}\quad & \exists y\in B\qty{\ev{A,R}\cong\ev{\mathrm{pred}(B,y,S),S}} \\
    \text{(c)}\quad & \exists x\in A\qty{\ev{\mathrm{pred}(A,x,R),R}\cong\ev{B,S}}
  \end{align}
\end{theorem}
\begin{proof}
  次のように $f$ を定める。
  \begin{align}
    f & = \qty{\ev{v, w}: v\in A\land w\in B\land\ev{\mathrm{pred}(A,v,R), R}\cong\ev{\mathrm{pred}(B,w,S),S}}
  \end{align}
  このとき, $f$ は $A$ のある切片から $B$ のある切片への同型写像となるが, これら二つの切片の両方が真の切片となることはありえない。
\end{proof}

\begin{axiom}[選択公理]
  $\forall A\exists R(\text{RはAを整列順序づけする})$
\end{axiom}


\subsection{順序数}
\begin{definition}[推移的]
  集合 $x$ の任意の要素が同時に $x$ の部分集合でもあるとき $x$ が推移的であると呼ぶ。
\end{definition}
\begin{definition}[順序数]
  推移的な集合 $x$ が $\in$ によって整列順序づけされるとき, $x$ を順序数と呼ぶ。
\end{definition}

\begin{theorem}
  \begin{enumerate}
    \item $x$ が順序数で $y\in x$ なら, $y$ も順序数で $y=\mathrm{pred}(x,y)$。
    \item $x$ と $y$ が順序数で $x\cong y$ なら, $x=y$。
    \item $x$ と $y$ が順序数なら, $x\in y, y\in x, y=x$ のどれか1つだけが成立する。
    \item $x$ と $y$ と $z$ が順序数で $x\in y, y\in z$ であれば, $x\in z$ である。
    \item $C$ が順序数の空でない集合であれば, $\exists x\in C\forall y\in C(x\in y\lor x=y)$。
  \end{enumerate}
\end{theorem}
\begin{proof}
  \begin{enumerate}
    \item 推移的であるから $y$ も順序数であり $\mathrm{pred}(x,y) = \qty{z\in x:z \in y} = y$ となる。
    \item 具体的な集合は空集合しか定義されていないから $x, y$ が同じように推移的であるならば $x = y$ であることが分かる。
    \item (1), (2) と定理(ref)より成立する。$x=\qty{x}$ などの自分自身の元となる推移的な集合は順序数でないから2つ同時に成立することはない。
    \item 推移的である為, 成り立つ。
    \item $C$ の $\in$-最小 $x$ について $x\cap C = 0$ となる。よって $0$ に含まれる元は存在しないことと(3)より $x$ は条件を満たす。
  \end{enumerate}
  よって全て示された。
\end{proof}

\begin{theorem}
  $\lnot\exists z\forall x(x\text!{は順序数}\to x\in z)$
\end{theorem}
\begin{proof}
  仮に任意の順序数を含む集合 $z$ があるとすると, 集合
  \begin{align}
    ON & = \qty{x:x\text{は順序数}}
  \end{align}
  が存在し, これは順序数となるが, $ON \in ON$ となり, 整列順序付けできない為, 順序数ではない。よって矛盾し, そのような集合 $z$ は存在しない。
\end{proof}

\begin{lemma}
  順序数の集合 $A$ が $\forall x\in A\forall y\in x(y\in A)$ ならば $A$ は順序数である。
\end{lemma}

\begin{theorem}
  $\ev{A, R}$ が整列順序であれば, あるただ一つに定まる順序数 $C$ について $\ev{A, R}\cong C$ となる。
\end{theorem}
\begin{proof}
  定理(ref)(3)より唯一性はわかる。$B = \qty{a\in A:\exists x(x\text!{は順序数} \land x\cong\ev{\mathrm{pred}(A, a, R)})}$ とおくと, 置換公理より
  \begin{align}
    \forall a\in B\exists!x(x\cong\ev{\mathrm{pred}(A, a, R)}) \\
    C\coloneq\qty{x:\exists a\in B\qty{x\cong\ev{\mathrm{pred}(A, a, R)}}}
  \end{align}
  となる $C$ が存在し, 関数 $f$ を $f:a\mapsto x$ とおくと $f\subset B\times C$ となる。
\end{proof}

\end{document}
