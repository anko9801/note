\RequirePackage{plautopatch}
\documentclass[uplatex,dvipdfmx,a4paper,11pt]{jlreq}
\usepackage{bxpapersize}
\usepackage[utf8]{inputenc}
\usepackage{fontenc}
\usepackage{lmodern}
\usepackage{otf}
\usepackage{amsmath}
\usepackage{amssymb}
\usepackage{amsthm}
\usepackage{ascmac}
% \usepackage[hyphens]{url}
\usepackage{physics}
\usepackage{braket}
\usepackage{verbatimbox}
\usepackage{bm}
\usepackage{url}
% \usepackage[dvipdfmx,hiresbb,final]{graphicx}
\usepackage{hyperref}
\usepackage{pxjahyper}
\usepackage{tikz}\usetikzlibrary{cd}
\usepackage{listings}
\usepackage{color}
\usepackage{mathtools}
\usepackage{xspace}
\usepackage{xy}
\usepackage{xypic}
%
\title{物理数学}
\author{Anko}
\makeatletter
%
\DeclareMathOperator{\lcm}{lcm}
\DeclareMathOperator{\Kernel}{Ker}
\DeclareMathOperator{\Image}{Im}
\DeclareMathOperator{\ch}{ch}
\DeclareMathOperator{\Aut}{Aut}
\DeclareMathOperator{\Log}{Log}
\DeclareMathOperator{\Arg}{Arg}
\DeclareMathOperator{\sgn}{sgn}
%
\newcommand{\CC}{\mathbb{C}}
\newcommand{\RR}{\mathbb{R}}
\newcommand{\QQ}{\mathbb{Q}}
\newcommand{\ZZ}{\mathbb{Z}}
\newcommand{\NN}{\mathbb{N}}
\newcommand{\FF}{\mathbb{F}}
\newcommand{\PP}{\mathbb{P}}
\newcommand{\GG}{\mathbb{G}}
\newcommand{\TT}{\mathbb{T}}
\newcommand{\calB}{\mathcal{B}}
\newcommand{\calF}{\mathcal{F}}
\newcommand{\ignore}[1]{}
\newcommand{\floor}[1]{\left\lfloor #1 \right\rfloor}
% \newcommand{\abs}[1]{\left\lvert #1 \right\rvert}
\newcommand{\lt}{<}
\newcommand{\gt}{>}
\newcommand{\id}{\mathrm{id}}
\newcommand{\rot}{\curl}
\renewcommand{\angle}[1]{\left\langle #1 \right\rangle}
\newcommand{\EE}{\bm{E}}
\newcommand{\BB}{\bm{B}}
\renewcommand{\AA}{\bm{A}}
\newcommand{\rr}{\bm{r}}
\newcommand{\kk}{\bm{k}}
\newcommand{\pp}{\bm{p}}

\let\oldcite=\cite
\renewcommand\cite[1]{\hyperlink{#1}{\oldcite{#1}}}

\let\oldbibitem=\bibitem
\renewcommand{\bibitem}[2][]{\label{#2}\oldbibitem[#1]{#2}}

% theorem環境の設定
% - 冒頭に改行
% - 末尾にdiamond (amsthm)
\theoremstyle{definition}
\newcommand*{\newscreentheoremx}[2]{
  \newenvironment{#1}[1][]{
    \begin{screen}
    \begin{#2}[##1]
      \leavevmode
      \newline
  }{
    \end{#2}
    \end{screen}
  }
}
\newcommand*{\newqedtheoremx}[2]{
  \newenvironment{#1}[1][]{
    \begin{#2}[##1]
      \leavevmode
      \newline
      \renewcommand{\qedsymbol}{\(\diamond\)}
      \pushQED{\qed}
  }{
      \qedhere
      \popQED
    \end{#2}
  }
}
\newtheorem{theorem*}{定理}

\newqedtheoremx{theorem}{theorem*}
\newcommand*\newqedtheorem@unstarred[2]{%
  \newtheorem{#1*}[theorem*]{#2}
  \newqedtheoremx{#1}{#1*}
}
\newcommand*\newqedtheorem@starred[2]{%
  \newtheorem*{#1*}{#2}
  \newqedtheoremx{#1}{#1*}
}
\newcommand*{\newqedtheorem}{\@ifstar{\newqedtheorem@starred}{\newqedtheorem@unstarred}}

\newtheorem{sctheorem*}{定理}
\newscreentheoremx{sctheorem}{sctheorem*}
\newcommand*\newscreentheorem@unstarred[2]{%
  \newtheorem{#1*}[theorem*]{#2}
  \newscreentheoremx{#1}{#1*}
}
\newcommand*\newscreentheorem@starred[2]{%
  \newtheorem*{#1*}{#2}
  \newscreentheoremx{#1}{#1*}
}
\newcommand*{\newscreentheorem}{\@ifstar{\newscreentheorem@starred}{\newscreentheorem@unstarred}}

%\newtheorem*{definition}{定義}
%\newtheorem{theorem}{定理}
%\newtheorem{proposition}[theorem]{命題}
%\newtheorem{lemma}[theorem]{補題}
%\newtheorem{corollary}[theorem]{系}

\newqedtheorem{lemma}{補題}
\newqedtheorem{corollary}{系}
\newqedtheorem{example}{例}
\newqedtheorem{proposition}{命題}
\newqedtheorem{remark}{注意}
\newqedtheorem{thesis}{主張}
\newqedtheorem{notation}{記法}
\newqedtheorem{problem}{問題}
\newqedtheorem{algorithm}{アルゴリズム}

\newscreentheorem*{definition}{定義}

\renewenvironment{proof}[1][\proofname]{\par
  \normalfont
  \topsep6\p@\@plus6\p@ \trivlist
  \item[\hskip\labelsep{\bfseries #1}\@addpunct{\bfseries}]\ignorespaces\quad\par
}{%
  \qed\endtrivlist\@endpefalse
}
\renewcommand\proofname{証明}

\makeatother

\begin{document}
\maketitle

\section{複素関数}
この章では
\subsection{複素数}
高校では $i^2 = -1$ を満たす $i$ を虚数単位といってこれを実数に付け加えたものを複素数と言っていました。これを代数の知識を用いて定義し直します。

\begin{definition}
  $\RR[x]/(x^2 + 1)$ を複素数体といい、 $\CC$ と書く。この元を複素数といい、 $x$ を虚数単位 (imaginary unit) といい、$i$ と書くことにする。このとき任意の元 $z\in\CC$ は 2 つの実数 $x, y\in\RR$ を用いて $z = x + iy$ と表される。このとき関数 $\Re, \Im$ を次のように定義する。
  \begin{align}
    \begin{cases}
      x = \Re z \\
      y = \Im z
    \end{cases}
  \end{align}
\end{definition}

\begin{proposition}
  複素数 $z_1 = x_1 + iy_1$, $z_2 = x_2 + iy_2$, $z = x + iy$ とすると次が成り立つ。
  \begin{align}
     & z_1 = z_2 \iff x_1 = x_2 \land y_1 = y_2            \\
     & z_1 + z_2 = (x_1 + x_2) + i(y_1 + y_2)              \\
     & z_1z_2 = (x_1x_2 - y_1y_2) + i(x_1y_2 + x_2y_1)     \\
     & z^{-1} = \frac{x}{x^2 + y^2} - i\frac{y}{x^2 + y^2}
  \end{align}
\end{proposition}
\begin{proof}
  複素数体 $\CC$ において $\lbrace 1, i\rbrace$ は基底となるから線形独立であり $x + iy = 0$ ならば $x = 0$ かつ $y = 0$ である。
  これより次のようになる。
  \begin{align}
    z_1 = z_2 & \iff z_1 - z_2 = 0 \iff (x_1 - x_2) + i(y_1 - y_2) = 0                \\
              & \iff x_1 - x_2 = 0 \land y_1 - y_2 = 0 \iff x_1 = x_2 \land y_1 = y_2
  \end{align}
  他の等式は計算すれば満たすことが分かる。
\end{proof}

また交換法則、結合法則、分配法則を満たすことが分かるので $\CC$ は可換体です。
\begin{definition}
  複素数 $z = x + iy$ の絶対値 (absolute value)、共役複素数 (complex conjugate) をそれぞれ次のように定義する。
  \begin{align}
    |z|          & = \sqrt{x^2 + y^2} \\
    \overline{z} & = x - iy
  \end{align}
\end{definition}
\begin{proposition}
  複素数 $z, z_1, z_2\in\CC$ に対して次の式が成り立つ。
  \begin{align}
     & |z|^2 = z\overline{z}                                   \\
     & \qty||z_1| - |z_2|| \leq |z_1 + z_2| \leq |z_1| + |z_2|
  \end{align}
\end{proposition}
\begin{proof}
  最初の式については $z = x + iy$ と置くと次のように表される。
  \begin{align}
    |z|^2 = x^2 + y^2 = (x + iy)(x - iy) = z\overline{z}
  \end{align}
  次に第二式の各項についてそれぞれを二乗した値を比較する。
  \begin{align}
    \qty(|z_1| - |z_2|)^2 & = \qty(\sqrt{x_1^2 + y_1^2} - \sqrt{x_2^2 + y_2^2})^2 = x_1^2 + y_1^2 + x_2^2 + y_2^2 - 2\sqrt{(x_1^2 + y_1^2)(x_2^2 + y_2^2)} \\
    |z_1 + z_2|^2         & = (x_1 + x_2)^2 + (y_1 + y_2)^2 = x_1^2 + x_2^2 + y_1^2 + y_2^2 + 2(x_1x_2 + y_1y_2)                                           \\
    (|z_1| + |z_2|)^2     & = |z_1|^2 + |z_2|^2 + 2|z_1||z_2| = x_1^2 + y_1^2 + x_2^2 + y_2^2 + 2\sqrt{(x_1^2 + y_1^2)(x_2^2 + y_2^2)}
  \end{align}
  ここで $x_1^2y_2^2 + x_2^2y_1^2 \geq 2x_1x_2y_1y_2$ であるから $x^2 \leq y^2\implies |x|\leq |y|$ より第二式が成り立つ。
\end{proof}

無限遠点

リーマン面
リーマン球面

\subsection{複素変数の関数}

\begin{theorem}[コーシー・リーマンの方程式]
  複素変数の関数 $f(z)\in\CC(z) = f(x + iy) = u(x, y) + iv(x, y)$ が $z = z_0$ で微分可能であるとき導関数 $f'(z_0)$ は次のように書くことができる。
  \begin{align}
    f'(z_0) = \lim_{\Delta z\to 0}\frac{f(z_0 + \Delta z) - f(z_0)}{\Delta z}
  \end{align}
  このとき実関数 $u, v$ は次の式を満たす。
  \begin{align}
    \pdv{u}{x} = \pdv{v}{y}, \pdv{u}{y} = -\pdv{v}{x}
  \end{align}
\end{theorem}
\begin{proof}
  $\Delta z = \Delta x + 0i$
  \begin{align}
    f'(z_0) & = \lim_{\Delta z\to 0}\frac{f(z_0 + \Delta z) - f(z_0)}{\Delta z}                                                     \\
            & = \lim_{\Delta x\to 0}\frac{u(x_0 + \Delta x, y_0) - u(x_0, y_0) + i(v(x_0 + \Delta x, y_0) - v(x_0, y_0))}{\Delta x} \\
            & = \pdv{u(x_0, y_0)}{x} + i\pdv{v(x_0, y_0)}{x}
  \end{align}
  $\Delta z = 0 + i\Delta y$
  \begin{align}
    f'(z_0) & = \lim_{\Delta z\to 0}\frac{f(z_0 + \Delta z) - f(z_0)}{\Delta z}                                                      \\
            & = \lim_{\Delta y\to 0}\frac{u(x_0, y_0 + \Delta y) - u(x_0, y_0) + i(v(x_0, y_0 + \Delta y) - v(x_0, y_0))}{i\Delta y} \\
            & = \pdv{v(x_0, y_0)}{y} - i\pdv{u(x_0, y_0)}{y}
  \end{align}
  これより領域において次が成り立つ
  \begin{align}
    \pdv{u}{x} = \pdv{v}{y}, \pdv{u}{y} = -\pdv{v}{x}
  \end{align}
\end{proof}


\subsection{正則関数}
\begin{theorem}
  Riemman Roch の定理
\end{theorem}

\begin{definition}[初等関数]
  指数関数 $2\pi i$ 周期で同じ
  \begin{align}
    f(z) = e^z = e^{x + iy} = e^x(\cos y + i\sin y)
  \end{align}
  三角関数
  \begin{align}
    \sin z & := \frac{e^{iz} - e^{-iz}}{2i}                          \\
    \cos z & := \frac{e^{iz} + e^{-iz}}{2}                           \\
    \tan z & := \frac{1}{i}\frac{e^{iz} - e^{-iz}}{e^{iz} + e^{-iz}}
  \end{align}

  双曲線関数
  \begin{align}
    \sinh z & := \frac{e^{z} - e^{-z}}{2}              \\
    \cosh z & := \frac{e^{z} + e^{-z}}{2}              \\
    \tanh z & := \frac{e^{z} - e^{-z}}{e^{z} + e^{-z}}
  \end{align}

  対数関数
  \begin{align}
    \log z & := \log r + i\theta \qquad (z\neq 0) \\
    \theta & = \arg z = \Arg z + 2n\pi            \\
    \Log z & := \log r + i\Arg z
  \end{align}
\end{definition}
\begin{proposition}
  \begin{align}
    \sin^{-1} z = -i\log(iz + (1 - z^2)^{1/2})
  \end{align}
\end{proposition}

\begin{theorem}[一致の定理]
  領域 $D$ で正則な関数 $f(z), g(z)$ があり、$D$ の小領域もしくは曲線上で一致しているとき、領域 $D$ 全体で $f(z) = g(z)$ が成り立つ。
\end{theorem}
\begin{proof}
  \begin{align}
    f(z) & = \sum_{n=0}^{\infty}\frac{f^{(n)}(z_0)}{n!}(z - z_0)^n \\
    g(z) & = \sum_{n=0}^{\infty}\frac{g^{(n)}(z_0)}{n!}(z - z_0)^n
  \end{align}
\end{proof}
\begin{theorem}[解析接続]
\end{theorem}

\section{微分積分学}
\subsection{測度}
測度とはある範囲の集合に非負の実数あるいは $\infty$ を対応させる集合関数である。
\begin{definition}[加法族]
  集合 $X$ の部分集合の族 $\bm{B}$ が次の条件を満たすとき、$X$ 上の加法族と呼ぶ。
  \begin{enumerate}
    \item $\emptyset\in\bm{B}$
    \item $A\in\bm{B}$ ならば $A^c\in\bm{B}$
    \item $A_n\in\bm{B}$ ならば $\bigcup_{n=1}^\infty A_n\in\bm{B}$
  \end{enumerate}
\end{definition}
\begin{definition}[測度]
  $\bm{B}$ を $X$ 上の加法族とするとき $\mu$ が $(X, \bm{B})$ 上の測度であるとは
  \begin{enumerate}
    \item $A\in\bm{B}$ に対し $0\leq \mu(A)\leq\infty$ $\mu(\emptyset) = 0$
    \item $A_n$
  \end{enumerate}
\end{definition}

\section{ベクトル解析}

\begin{definition}[Kronecker のデルタ]
  \begin{align}
    \delta_{ij} = \begin{cases}
                    1 & (i = j)    \\
                    0 & (i \neq j)
                  \end{cases}
  \end{align}
\end{definition}
\begin{definition}[レビ・チビタの完全反対称テンソル (Levi-Civita antisymmetric tensor)]
  $\epsilon_{ijk}$ は $1,-1,0$ の値を取る. ${\epsilon_{xyz} = 1}$ であり, 任意の2つの添字の交換に対して符号を変え, また任意の2つの添字の値が等しければ $0$ となる.
  \begin{align}
    \epsilon_{\mu_1\cdots\mu_k} & :=
    \begin{dcases}
      \sgn\begin{pmatrix}
            1     & \cdots & k     \\
            \mu_1 & \cdots & \mu_k
          \end{pmatrix} & (\mu_1\cdots\mu_k が順列のとき) \\
      0                         & (else)
    \end{dcases} \\
                                & =
    \begin{dcases}
      1  & (\mu_1\cdots\mu_k が偶置換のとき) \\
      -1 & (\mu_1\cdots\mu_k が奇置換のとき) \\
      0  & (else)
    \end{dcases}
  \end{align}
\end{definition}

\begin{definition}[Einstein の縮約記法]
  同じ項で添字が重なる場合はその添字について和を取る.
  \begin{align}
    A_iB_i = \sum_{i=x,y,z}A_iB_i
  \end{align}
\end{definition}

\begin{definition}
  ベクトルについて内積と外積を定義する。
  \begin{align}
    \bm{A}\vdot\bm{B}  & = A_xB_x + A_yB_y + A_zB_z = A_iB_i                                                  \\
    \bm{A}\cross\bm{B} & = (A_yB_z - A_zB_y, A_zB_x - A_xB_z, A_xB_y - A_yB_x) = \bm{e}_i\epsilon_{ijk}A_jB_k
  \end{align}
\end{definition}
\begin{definition}
  \begin{align}
    \grad{\bm{A}} & = \qty(\pdv{A_x}{x}, \pdv{A_y}{y}, \pdv{A_z}{z})                                                                                    \\
    \div{\bm{A}}  & = \pdv{A_x}{x} + \pdv{A_y}{y} + \pdv{A_z}{z} = \partial_iA_i                                                                        \\
    \rot{\bm{A}}  & = \qty(\pdv{A_z}{y} - \pdv{A_y}{z}, \pdv{A_x}{z} - \pdv{A_z}{x}, \pdv{A_y}{x} - \pdv{A_x}{y}) = \bm{e}_i\epsilon_{ijk}\partial_jA_k
  \end{align}
\end{definition}

\begin{theorem}
  \begin{align}
    \grad{(f+g)}               & = \grad{f} + \grad{g}                                                                                               \\
    \grad{(fg)}                & = f\grad{g} + g\grad{f}                                                                                             \\
    \grad{(\bm{A}\vdot\bm{B})} & = \bm{A}\cross(\rot{\bm{B}}) + \bm{B}\cross(\rot{\bm{A}}) + (\bm{A}\vdot\vnabla)\bm{B} + (\bm{B}\vdot\vnabla)\bm{A} \\
    \div{(\bm{A}+\bm{B})}      & = \div{\bm{A}} + \div{\bm{B}}                                                                                       \\
    \div{(f\bm{A})}            & = f(\div{\bm{A}}) + \bm{A}\vdot(\grad{f})                                                                           \\
    \div{(\bm{A}\cross\bm{B})} & = \bm{B}\vdot(\rot{\bm{A}}) - \bm{A}\vdot(\rot{\bm{B}})                                                             \\
    \rot{(\bm{A}+\bm{B})}      & = \rot{\bm{A}} + \rot{\bm{B}}                                                                                       \\
    \rot{(f\bm{A})}            & = f(\rot{\bm{A}}) - \bm{A}\cross(\grad{f})                                                                          \\
    \rot{(\bm{A}\cross\bm{B})} & = \bm{A}(\div{\bm{B}}) - \bm{B}(\div{\bm{A}}) + (\bm{B}\vdot\vnabla)\bm{A} - (\bm{A}\vdot\vnabla)\bm{B}             \\
    \div{(\rot{\bm{A}})}       & = 0                                                                                                                 \\
    \rot{(\grad{f})}           & = \bm{0}                                                                                                            \\
    \rot{(\rot{\bm{A}})}       & = \div{(\grad{\bm{A}})} - \laplacian{\bm{A}}
  \end{align}
\end{theorem}
\begin{proof}
  ベクトルのときは各要素について考える。
  \begin{align}
    (\grad{(f+g)})_i               & = \partial_i(f+g) = \partial_i f + \partial_i g = (\grad{f} + \grad{g})_i                                                           \\
    (\grad{(fg)})_i                & = \partial_i(fg) = f\partial_i g + g\partial_i f = (f\grad{g} + g\grad{f})_i                                                        \\
    (\grad{(\bm{A}\vdot\bm{B})})_i & = \partial_i(A_jB_j)                                                                                                                \\
                                   & = (A_j\partial_iB_j + B_j\partial_iA_j) - (A_j\partial_jB_i + B_j\partial_jA_i) + (A_j\partial_jB_i + B_j\partial_jA_i)             \\
                                   & = (\delta_{il}\delta_{jm} - \delta_{im}\delta_{jl})(A_j\partial_lB_m + B_j\partial_lA_m) + (A_j\partial_jB_i + B_j\partial_jA_i)    \\
                                   & = \epsilon_{kij}\epsilon_{klm}(A_j\partial_lB_m + B_j\partial_lA_m) + (A_j\partial_jB_i + B_j\partial_jA_i)                         \\
                                   & = \epsilon_{ijk}A_j\epsilon_{klm}\partial_lB_m + \epsilon_{ijk}B_j\epsilon_{klm}\partial_lA_m + A_j\partial_jB_i + B_j\partial_jA_i \\
                                   & = (\bm{A}\cross(\rot{\bm{B}}) + \bm{B}\cross(\rot{\bm{A}}) + (\bm{A}\vdot\vnabla)\bm{B} + (\bm{B}\vdot\vnabla)\bm{A})_i
  \end{align}
  \begin{align}
    \div{(\bm{A}+\bm{B})}      & = \partial_i(A_i + B_i) = \partial_i A_i + \partial_i B_i         \\
                               & = \div{\bm{A}}+\div{\bm{B}}                                       \\
    \div{(f\bm{A})}            & = \partial_i(fA_i) = f\partial_iA_i + A_i\partial_if              \\
                               & = f(\div{\bm{A}}) + \bm{A}\vdot(\grad{f})                         \\
    \div{(\bm{A}\cross\bm{B})} & = \partial_i(\epsilon_{ijk}A_jB_k)                                \\
                               & = \epsilon_{ijk}(B_k\partial_iA_j + A_j\partial_iB_k)             \\
                               & = B_k\epsilon_{kij}\partial_iA_j - A_j\epsilon_{jik}\partial_iB_k \\
                               & = \bm{B}\vdot(\rot{\bm{A}}) - \bm{A}\vdot(\rot{\bm{B}})
  \end{align}
  \begin{align}
    (\rot{(\bm{A} + \bm{B})})_i    & = \epsilon_{ijk}\partial_j(A_k + B_k)                                                                   \\
                                   & = \epsilon_{ijk}\partial_jA_k + \epsilon_{ijk}\partial_jB_k                                             \\
                                   & = \rot{\bm{A}} + \rot{\bm{B}}                                                                           \\
    (\rot{(f\bm{A})})_i            & = \epsilon_{ijk}\partial_j(fA_k)                                                                        \\
                                   & = f\epsilon_{ijk}\partial_jA_k + \epsilon_{ijk}A_k\partial_jf                                           \\
                                   & = f\epsilon_{ijk}\partial_jA_k - \epsilon_{ikj}A_k\partial_jf                                           \\
                                   & = (f(\rot{\bm{A}}) - \bm{A}\cross(\grad{f}))_i                                                          \\
    (\rot{(\bm{A}\cross\bm{B})})_i & = \epsilon_{ijk}\partial_j\epsilon_{klm}A_lB_m                                                          \\
                                   & = \epsilon_{kij}\epsilon_{klm}(B_m\partial_jA_l + A_l\partial_jB_m)                                     \\
                                   & = (\delta_{il}\delta_{jm} - \delta_{im}\delta_{jl})(B_m\partial_jA_l + A_l\partial_jB_m)                \\
                                   & = (B_j\partial_jA_i + A_i\partial_jB_j) - (B_i\partial_jA_j + A_j\partial_jB_i)                         \\
                                   & = A_i\partial_jB_j - B_i\partial_jA_j + B_j\partial_jA_i - A_j\partial_jB_i                             \\
                                   & = \bm{A}(\div{\bm{B}}) + \bm{B}(\div{\bm{A}}) + (\bm{B}\vdot\vnabla)\bm{A} - (\bm{A}\vdot\vnabla)\bm{B}
  \end{align}
  \begin{align}
    \div{(\rot{\bm{A}})}     & = \partial_i(\epsilon_{ijk}\partial_jA_k) = \epsilon_{ijk}\partial_i\partial_jA_k = 0 \\
    (\rot{(\grad{f})})_i     & = \epsilon_{ijk}\partial_j\partial_kf = 0                                             \\
    (\rot{(\rot{\bm{A}})})_i & = \epsilon_{ijk}\partial_j(\epsilon_{klm}\partial_lA_m)                               \\
                             & = \epsilon_{kij}\epsilon_{klm}\partial_j\partial_lA_m                                 \\
                             & = (\delta_{il}\delta_{jm} - \delta_{im}\delta_{jl})\partial_j\partial_lA_m            \\
                             & = \partial_j\partial_iA_j - \partial_j^2A_i                                           \\
                             & = (\div{(\grad{\bm{A}})} - \laplacian{\bm{A}})_i
  \end{align}
\end{proof}

\section{微分幾何学}
\subsection{多様体}
接ベクトル
接ベクトル束
\subsection{微分形式}
\begin{definition}[微分形式]
  $\land:$
  \begin{align}
    u\land v := u\otimes v - v\otimes u
  \end{align}
  $0$-形式 \\
  $k$-形式$\omega_{\mu_1\cdots\mu_k}\in C^\infty(U)$
  \begin{align}
    \omega = \frac{1}{k!}\omega_{\mu_1\cdots\mu_k}dx^{\mu_1}\land\cdots\land dx^{\mu_k}
  \end{align}
\end{definition}
\begin{definition}[外微分]
  外微分 (exterior derivative) $d: \Omega^k(M)\to\Omega^{k+1}(M)$ を次のように定義する。
  \begin{align}
    d\omega & := d\qty(\frac{1}{k!}\omega_{\mu_1\cdots\mu_k})\land dx^{\mu_1}\land\cdots\land dx^{\mu_k}                \\
            & = \frac{1}{k!}\pdv{\omega_{\mu_1\cdots\mu_k}}{x^{\nu}}dx^{\nu}\land dx^{\mu_1}\land\cdots\land dx^{\mu_k}
  \end{align}
\end{definition}
\begin{align}
  dx\land dy & = d(r\cos\theta)\land d(r\sin\theta)                                                                                                                  \\
             & = (dr\cos\theta - r\sin\theta d\theta)\land (dr\sin\theta + r\cos\theta d\theta)                                                                      \\
             & = (\cos\theta\sin\theta)dr\land dr + (r\cos^2\theta)dr\land d\theta - (r\sin^2\theta) d\theta\land dr - (r^2\sin\theta\cos\theta)d\theta\land d\theta \\
             & = r dr\land d\theta
\end{align}

\begin{definition}[1 の分割 (partition of unity)]
\end{definition}

\begin{theorem}[ストークスの定理]
\end{theorem}

\section{フーリエ解析}
\subsection{フーリエ級数}
\begin{definition}[内積]
  関数の正規直交関数系による展開
  区間 $[a, b]$ 上の
\end{definition}


\begin{definition}[複素フーリエ級数]
  $\TT = \RR/(2\pi\ZZ)$ 上 関数 $f: \TT\to\CC$ に対し
  区間 $[-\pi, \pi]$ において定義された実数値関数 $f(x)$ が連続かつ区分的に $C^1$ 級かつ周期的である ($f(-\pi) = f(\pi)$) ならば $f(x)$ は
  \begin{align}
    f(x) & = \sum_{n\in\ZZ}c_ne^{inx}                          \\
    c_n  & := \frac{1}{2\pi}\int_{-\pi}^\pi f(x)e^{-inx}\dd{x}
  \end{align}
\end{definition}
例
\begin{theorem}[Bessel の不等式]
  \begin{align}
    \sum_{n\in\ZZ}|\hat{f}(n)|^2 \leq \|f\|_2^2
  \end{align}
\end{theorem}


\begin{theorem}[平均値の定理]
  区間 $[a, b]$ で連続、 $(a, b)$ で微分可能な関数 $f(x)$ について $a < c < b$ となる $c$ が存在して次のようになる。
  \begin{align}
    \frac{f(b) - f(a)}{b - a} = f'(c)
  \end{align}
\end{theorem}

\begin{proposition}
  \begin{align}
     & \int_{-\pi}^\pi \sin(mx)\cos(nx)\dd{x} = 0               \\
     & \int_{-\pi}^\pi \cos(mx)\cos(nx)\dd{x} = \pi\delta_{m,n} \\
     & \int_{-\pi}^\pi \sin(mx)\sin(nx)\dd{x} = \pi\delta_{m,n} \\
     & \int_{-\pi}^\pi \cos(nx)\dd{x} = 2\pi\delta_{n,0}        \\
     & \int_{-\pi}^\pi \sin(nx)\dd{x} = 0
  \end{align}
\end{proposition}
\begin{proof}
  \begin{align}
    \int_{-\pi}^\pi \sin(mx)\cos(nx)\dd{x} & = \frac{1}{2}\int_{-\pi}^\pi \qty[\sin(m + n)x + \sin(m - n)x]\dd{x}                            \\
                                           & = \frac{1}{2}\qty[- \frac{\cos(m + n)x}{m + n} - \frac{\cos(m - n)x}{m - n}]_{-\pi}^\pi         \\
                                           & = 0                                                                                             \\
    \int_{-\pi}^\pi \cos(mx)\cos(nx)\dd{x} & = \frac{1}{2}\int_{-\pi}^\pi \qty[\cos(m - n)x + \cos(m + n)x]\dd{x}                            \\
                                           & = \begin{dcases}
                                                 \frac{1}{2}\qty[\frac{\sin(m - n)x}{m - n} + \frac{\sin(m + n)x}{m + n}]_{-\pi}^\pi & (m \neq n) \\
                                                 \frac{1}{2}\qty[x + \frac{\sin(m + n)x}{m + n}]_{-\pi}^\pi                          & (m = n)
                                               \end{dcases} \\
                                           & = \pi\delta_{m,n}                                                                               \\
    \int_{-\pi}^\pi \sin(mx)\sin(nx)\dd{x} & = \frac{1}{2}\int_{-\pi}^\pi \qty[\cos(m - n)x - \cos(m + n)x]\dd{x}                            \\
                                           & = \begin{dcases}
                                                 \frac{1}{2}\qty[\frac{\sin(m - n)x}{m - n} - \frac{\sin(m + n)x}{m + n}]_{-\pi}^\pi & (m \neq n) \\
                                                 \frac{1}{2}\qty[x - \frac{\sin(m + n)x}{m + n}]_{-\pi}^\pi                          & (m = n)
                                               \end{dcases} \\
                                           & = \pi\delta_{m,n}
  \end{align}
  \begin{align}
    \int_{-\pi}^\pi \cos(nx)\dd{x} & = \begin{dcases}
                                         \qty[\frac{\sin(nx)}{n}]_{-\pi}^\pi & (n \neq 0) \\
                                         \qty[x]_{-\pi}^\pi                  & (n = 0)
                                       \end{dcases}  \\
                                   & = 2\pi\delta_{n,0}                                       \\
    \int_{-\pi}^\pi \sin(nx)\dd{x} & = \begin{dcases}
                                         \qty[-\frac{\cos(nx)}{n}]_{-\pi}^\pi & (n \neq 0) \\
                                         \qty[0]_{-\pi}^\pi                   & (n = 0)
                                       \end{dcases} \\
                                   & = 0
  \end{align}
\end{proof}

\begin{definition}[$2\pi$ の周期をもつ関数のフーリエ級数]
  \begin{align}
    f(x) & \sim \frac{a_0}{2} + \sum_{n=1}^{\infty}(a_n\cos(nx) + b_n\sin(nx)) \\
    a_n  & := \frac{1}{\pi}\int_{-\pi}^\pi f(x)\cos(nx)\dd{x}                  \\
    b_n  & := \frac{1}{\pi}\int_{-\pi}^\pi f(x)\sin(nx)\dd{x}
  \end{align}
\end{definition}

\begin{definition}[$2\pi$ の周期をもつ関数の複素フーリエ級数]
  \begin{align}
    f(x) & = \sum_{n=-\infty}^{\infty}c_ne^{inx}               \\
    c_n  & := \frac{1}{2\pi}\int_{-\pi}^\pi f(x)e^{-inx}\dd{x}
  \end{align}
\end{definition}
\begin{proof}
  \begin{align}
    f(x) & \sim \frac{a_0}{2} + \sum_{n=1}^{\infty}(a_n\cos(nx) + b_n\sin(nx))                                               \\
         & = \frac{a_0}{2} + \sum_{n=1}^{\infty}\qty(\frac{a_n}{2}(e^{inx} + e^{-inx}) + \frac{b_n}{2i}(e^{inx} - e^{-inx})) \\
         & = \frac{a_0}{2} + \sum_{n=1}^{\infty}\qty(\frac{a_n - ib_n}{2}e^{inx} + \frac{a_n + ib_n}{2}e^{-inx})             \\
         & = \sum_{n=-\infty}^{\infty}c_ne^{inx}
  \end{align}
  ただし $c_n$ は次のように定める。
  \begin{align}
    c_n & := \begin{dcases}
               \frac{a_n - ib_n}{2} & (n > 0) \\
               \frac{a_0}{2}        & (n = 0) \\
               \frac{a_n - ib_n}{2} & (n < 0)
             \end{dcases}
  \end{align}
  \begin{align}
    \frac{1}{2\pi}\int_{-\pi}^\pi f(x)e^{-inx}\dd{x} & = \frac{1}{2\pi}\int_{-\pi}^\pi\sum_{m=-\infty}^{\infty}c_me^{imx}e^{-inx}\dd{x} \\
                                                     & = \frac{1}{2\pi}\sum_{m=-\infty}^{\infty}c_m\int_{-\pi}^\pi e^{i(m - n)x}\dd{x}  \\
                                                     & = \frac{1}{2\pi}\sum_{m=-\infty}^{\infty}c_m 2\pi\delta_{m,n}                    \\
                                                     & = c_n
  \end{align}
\end{proof}

\begin{definition}[一般の周期をもつ関数のフーリエ級数]
  \begin{align}
    f(x) & \sim \frac{a_0}{2} + \sum_{n=1}^{\infty}\qty(a_n\cos\frac{n\pi x}{l} + b_n\sin\frac{n\pi x}{l}) \\
    a_n  & := \frac{1}{l}\int_{-l}^\pi f(x)\cos\frac{n\pi x}{l}\dd{x}                                      \\
    b_n  & := \frac{1}{l}\int_{-l}^\pi f(x)\sin\frac{n\pi x}{l}\dd{x}
  \end{align}
\end{definition}
\begin{definition}[一般の周期をもつ関数の複素フーリエ級数]
  \begin{align}
    f(x) & = \sum_{n=-\infty}^{\infty}c_ne^{i\frac{n\pi}{l}x}         \\
    c_n  & := \frac{1}{2l}\int_{-l}^l f(x)e^{-i\frac{n\pi}{l}x}\dd{x}
  \end{align}
\end{definition}

\begin{theorem}
  \begin{align}
    \lim_{n\to\infty}a_n = 0
  \end{align}
\end{theorem}

\begin{lemma}[コーシーの不等式]
  実数の数列 $\lbrace p_n\rbrace_n, \lbrace q_n\rbrace_n$ について次の不等式が成立する。
  \begin{align}
    \qty(\sum_{n=1}^{N}p_n^2)\qty(\sum_{n=1}^{N}q_n^2) \geq \qty(\sum_{n=1}^{N}p_nq_n)^2
  \end{align}
\end{lemma}
\begin{proof}
  $x$ について次の 2 次関数の判別式を考えることで求まる。
  \begin{align}
    \sum_{n=1}^{N}(p_nx + q_n)^2 \geq 0
  \end{align}
\end{proof}

\begin{theorem}[ワイエルシュトラスの M テスト]
  区間 $[a, b]$ で定義された関数列の無限級数 $s(x)$ の各項の絶対値が上界 $M_n$ をもち、 $M_n$ の総和が収束するならばもとの級数は $[a, b]$ で一様収束する。
  \begin{align}
    s(x) = \sum_{n=1}^{\infty}f_n(x)
  \end{align}
\end{theorem}
\begin{proof}

\end{proof}

\section{特殊関数}
\begin{definition}
  複素平面上で $\Re z > 1$ を満たす領域内にある閉曲線 $C$ 上の点 $z$ に対して次の関数は一様収束し正則な関数となる.
  \begin{align}
    \Gamma(z) := \int_0^\infty e^{-t}t^{z-1}\dd{t}
  \end{align}
\end{definition}

\begin{proposition}
  \begin{align}
    \Gamma(z + 1)           & = z\Gamma(z) \\
    \Gamma(1)               & = 1          \\
    \Gamma(n + 1)           & = n!         \\
    \Gamma\qty(\frac{1}{2}) & = \sqrt{\pi}
  \end{align}
\end{proposition}

\begin{proposition}[スターリングの公式 (Stirling's formula)]
  \begin{align}
    \Gamma(x + 1) & = \sqrt{2\pi x}e^{-x}x^x \qquad (x \gg 1)
  \end{align}
\end{proposition}

\begin{proposition}
  ガウスの公式 (Gauss's formula)
  \begin{align}
    \Gamma(z) & = \lim_{n\to\infty}\frac{n!n^z}{z(z+1)\cdots(z+n)} \\
  \end{align}
\end{proposition}
\begin{proposition}
  ワイエルシュトラスの公式 (Weierstrass' formula) $\gamma$ はオイラーの定数 (Euler's constant) とする.
  \begin{align}
    \frac{1}{\Gamma(z)} & = ze^{\gamma z}\prod_{n=1}^{\infty}\qty(1 + \frac{z}{n})e^{-z/n}              \\
    \gamma              & := \lim_{n\to\infty}\qty(\sum_{m=1}^{n}\frac{1}{m} - \log n) = 0.577216\cdots
  \end{align}
\end{proposition}

\begin{definition}
  ベータ関数 (Beta function)
  \begin{align}
    B(z, \zeta) := \int_{0}^{1}t^{z-1}(1-t)^{\zeta-1}\dd{t}
  \end{align}
\end{definition}

\begin{proposition}
  \begin{align}
    B(z, \zeta)            & = B(\zeta, z)                                                          \\
    B(z, \zeta)            & = 2\int_0^{\pi/2}\sin^{2z - 1}\theta\cos^{2\zeta - 1}\theta\dd{\theta} \\
    B(z, \zeta)            & = \int_0^{\infty}\frac{u^{z-1}}{(1 + u)^{z + \zeta}}\dd{\theta}        \\
    B(z, \zeta)            & = \frac{\Gamma(z)\Gamma(\zeta)}{\Gamma(z + \zeta)}                     \\
    \Gamma(z)\Gamma(1 - z) & = \frac{\pi}{\sin\pi z}
  \end{align}
\end{proposition}

\begin{definition}[ルジャンドル微分方程式]
  \begin{align}
    (1 - x^2)y'' - 2xy' + \lambda y = 0
  \end{align}
\end{definition}

\begin{align}
  y = \sum_{j=0}^{\infty}a_jx^j
\end{align}

\begin{definition}[ルジャンドルの陪微分方程式]
  ルジャンドルの陪微分方程式
  \begin{align}
    (1 - x^2)y'' - 2xy' + \qty(n(n + 1) - \frac{m^2}{1 - x^2})y = 0
  \end{align}
  これを満たす独立な 2 つの解 $P_n^m(x)$ と $Q_n^m(x)$ を第一種および第二種ルジャンドル陪関数はルジャンドル関数で表される。
\end{definition}

\begin{definition}
  ベッセルの微分方程式 (Bessel's equation)
  \begin{align}
    x^2y'' + xy' + (x^2 - \nu^2)y = 0
  \end{align}
\end{definition}
\begin{definition}
  ベッセルの微分方程式 (Bessel's equation)
  \begin{align}
    x^2y'' + xy' + (x^2 - \nu^2)y = 0
  \end{align}
\end{definition}

\begin{definition}
  ラゲール多項式
  \begin{align}
    \frac{e^{-xz/(1-z)}}{1-z} & = \sum_{n=0}^{\infty}L_n(x)\frac{z^n}{n!}
  \end{align}
\end{definition}
\begin{proposition}
  \begin{align}
    L_n(x)     & = e^x\dv[n]{x}(x^ne^{-x})                            \\
    L_n(x)     & = \sum_{l=0}^{n}\frac{(-1)^l(n!)^2}{(l!)^2(n-l)!}x^l \\
    L_{n+1}(x) & = (2n + 1 - x)L_n(x) - n^2L_{n-1}(x)                 \\
    xL_n'(x)   & = nL_n(x) - n^2L_{n-1}(x)                            \\
    L_n(0)     & = n!
  \end{align}
\end{proposition}

\begin{definition}
  次の級数展開の右辺に現れる $H_n(x)$ をエルミート多項式 (Hermite polynomials) という.
  \begin{align}
    e^{-t^2 + 2tx} = \sum_{n=0}^{\infty}\frac{1}{n!}H_n(x)t^n
  \end{align}
  また, 左辺の関数はエルミート多項式の母関数 (generating function) という.
\end{definition}
\begin{proposition}
  \begin{align}
    H_n(x) = (-1)^ne^{x^2}\dv[n]{x}
  \end{align}
\end{proposition}

\begin{definition}
  超幾何関数
  \begin{align}
    x(1 - x)y'' + [c - (a + b + 1)x]y' - aby = 0
  \end{align}
\end{definition}
\begin{proposition}
  \begin{align}
    e^x         & = \lim_{b\to\infty}{}_2F_1\qty(1,b,1;\frac{x}{b}) \\
    \log(1 + x) & = x\cdot{}_2F_1(1,1,2;-x)
  \end{align}
\end{proposition}

\subsection{境界値問題}
\begin{definition}
  ラプラス方程式 (Laplace equation)
  \begin{align}
    \pdv[2]{u}{x} + \pdv[2]{u}{y} = 0
  \end{align}
  ポアソン方程式 (Poisson equation)
  \begin{align}
    \pdv[2]{u}{x} + \pdv[2]{u}{y} = -\rho(x, y)
  \end{align}
  波動方程式 (wave equation)
  \begin{align}
    \frac{1}{c^2}\pdv[2]{u}{t} = \pdv[2]{u}{x}
  \end{align}
  熱伝導方程式 (heat conduction equation) \\
  $\kappa$ を熱伝導率 (thermal conductivity)
  \begin{align}
    \pdv{u}{t} & = \kappa\pdv[2]{u}{x} + q(x)
  \end{align}
\end{definition}

\begin{proposition}
  ラプラス方程式を満たし
  \begin{align}
    \pdv[2]{u}{x} + \pdv[2]{u}{y} = 0
  \end{align}
  次の境界条件を満たす関数 $u(x, y)$ を求める。
  \begin{align}
    u(0, y) = 0, u(a, y) = 0, u(x, 0) = f(x), u(x, b) = 0
  \end{align}
\end{proposition}
\begin{proof}
  これは変数分離法が使えないと思う。
  \begin{align}
    u(x, y) = X(x)Y(y)
  \end{align}
  ラプラス方程式
  \begin{align}
    X''(x)Y(y)          & + X(x)Y''(y) = 0                                                      \\
    \frac{X''(x)}{X(x)} & = - \frac{Y''(y)}{Y(y)}                                               \\
    X''(x)              & = - \lambda^2X(x)                                                     \\
    Y''(y)              & = \lambda^2Y(y)                                                       \\
    X(x)                & = \sin(\frac{n\pi x}{a})                                              \\
    \lambda             & = \frac{n\pi}{a}                                                      \\
    f(x)                & = \sum_{n=1}^{\infty}A_n\sin(\frac{n\pi x}{a})\sinh(\frac{n\pi b}{a})
  \end{align}
\end{proof}
\begin{theorem}[ガウス積分]
  \begin{align}
     & \int_0^{\infty} e^{-\alpha x^2}\dd{x} = \frac{1}{2}\sqrt{\frac{\pi}{a}}                     \\
     & \int_0^{\infty} x^{2n}e^{-x^2/a^2}\dd{x} = \sqrt{\pi}(2n - 1)!!\frac{a^{2n + 1}}{2^{n + 1}} \\
     & \int_0^{\infty} x^{2n + 1}e^{-x^2/a^2}\dd{x} = \frac{n!}{2}a^{2n + 2}                       \\
     & \int_{-\infty}^{\infty} e^{-k^2/4}e^{ikx}\dd{k} = 2\sqrt{\pi}e^{-x^2}
  \end{align}
\end{theorem}
\begin{proof}
  まず積分値を $I$ とおく。
  \begin{align}
    I & := \int_{-\infty}^{\infty} e^{-\alpha x^2}\dd{x}
  \end{align}
  ここで $I^2$ を変数変換して計算する。
  \begin{align}
    I^2 & = \qty(\int_{-\infty}^{\infty} e^{-\alpha x^2}\dd{x})\qty(\int_0^{\infty} e^{-\alpha x^2}\dd{x}) \\
        & = \int_{-\infty}^{\infty}\int_{-\infty}^{\infty} e^{-\alpha (x^2 + y^2)}\dd{x}\dd{y}             \\
        & = \int_{0}^{\infty}\int_{0}^{2\pi} e^{-\alpha r^2}r\dd{\theta}\dd{r}                             \\
        & = 2\pi\qty[-\frac{e^{-\alpha r^2}}{2\alpha}]_{0}^{\infty} = \frac{\pi}{a}
  \end{align}
  よって
  \begin{align}
    I & = \sqrt{\frac{\pi}{a}}
  \end{align}
  また
  \begin{align}
    \int_0^{\infty} x^{2n}e^{-\alpha x^2}\dd{x} & = (-1)^n\int_0^{\infty} \pdv[n]{\alpha}e^{-\alpha x^2}\dd{x}      \\
                                                & = (-1)^n\pdv[n]{\alpha}\int_0^{\infty} e^{-\alpha x^2}\dd{x}      \\
                                                & = (-1)^n\pdv[n]{\alpha}\qty(\frac{1}{2}\sqrt{\frac{\pi}{\alpha}}) \\
                                                & = \sqrt{\pi}\frac{(2n - 1)!!}{2^{n+1}}\alpha^{-(2n + 1)/2}
  \end{align}
  \begin{align}
    \int_0^{\infty} x^{2n + 1}e^{-\alpha x^2}\dd{x} & = (-1)^n\int_0^{\infty}\pdv[n]{\alpha}xe^{-\alpha x^2}\dd{x} \\
                                                    & = (-1)^n\pdv[n]{\alpha}\int_0^{\infty}xe^{-\alpha x^2}\dd{x} \\
                                                    & = (-1)^n\pdv[n]{\alpha}\frac{1}{2\alpha}                     \\
                                                    & = \frac{n!}{2}\alpha^{-(n + 1)}
  \end{align}
\end{proof}





\end{document}